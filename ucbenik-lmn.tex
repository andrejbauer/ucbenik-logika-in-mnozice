\documentclass[11pt,a4paper,twoside]{book}

%%%%%%%%%%%%%%%%%%%%%%%%%%%%%%%%%%%%%%%%%%%%%%%%%%%%%%%%%
%%%  Imported Packages
%%%%%%%%%%%%%%%%%%%%%%%%%%%%%%%%%%%%%%%%%%%%%%%%%%%%%%%%%
\usepackage[slovene]{babel}
\usepackage[utf8]{inputenc}
\usepackage[T1]{fontenc}

\usepackage{ifthen}
\usepackage{amssymb}
\usepackage{amsmath}
\usepackage{mathabx}
\usepackage{theorem}
\usepackage{textgreek}
\usepackage{wasysym}
\usepackage{phonetic}
\usepackage{bm}
\usepackage{tablefootnote}
\usepackage{xcolor}
\usepackage{tikz}
\usepackage{tkz-graph}
\usepackage{xparse}
\usepackage{mathrsfs}
\usepackage{ulem}
\usepackage{charter}

%%%%%%%%%%%%%%%%%%%%%%%%%%%%%%%%%%%%%%%%%%%%%%%%%%%%%%%%%%%%%
%%  Page Style & Margins (A4 page = 210mm x 297mm)

% PAGE GEOMETRY
\usepackage[papersize={210mm,297mm}, % A4
            twoside,
            includehead,
            top=1in, % margina na vrhu strani
            bottom=1in, % margina na dnu strani
            inner=1in, % margina na notranji strani strani
            outer=1in, % margina na zunanji strani strani
            bindingoffset=0pt % dodatna margina na notranji strani
           ]{geometry}

%%%%%%%%%%%%%%%%%%%%%%%%%%%%%%%%%%%%%%%%%%%%%%%%%%%%%%%%%%%%%
% HYPERLINKING AND PDF METADATA
\usepackage[backref=page,
            colorlinks,
            citecolor=linkcolor,
            linkcolor=linkcolor,
            urlcolor=linkcolor,
            unicode,
            pdfauthor={Andrej Bauer, Davorin Lešnik, Marko Petkovšek, Matija Pretnar},
            pdftitle={Logika in množice},
            pdfsubject={matematika},
            pdfkeywords={logika,množice,osnove matematike}]{hyperref}
\renewcommand{\backref}[1]{}
\renewcommand{\backrefalt}[4]{%
   \ifcase #1 %
   (Ni citirano.)
   \or
   (Citirano na strani\ #2.)
   \else
   (Citirano na straneh\ #2.)
   \fi}

\definecolor{linkcolor}{rgb}{0,0,0} % Barva hiperpovezav

%%%%%%%%%%%%%%%%%%%%%%%%%%%%%%%%%%%%%%%%%%%%%%%%%%%%%%%%%%%%%
%%%  Theorems etc.
%%%%%%%%%%%%%%%%%%%%%%%%%%%%%%%%%%%%%%%%%%%%%%%%%%%%%%%%%%%%%
        {
                \theorembodyfont{\itshape}

                \newtheorem{izrek}{Izrek}[section]
                \newtheorem{lema}[izrek]{Lema}
                \newtheorem{trditev}[izrek]{Trditev}
                \newtheorem{posledica}[izrek]{Posledica}
        }

        {
                \theorembodyfont{\rmfamily}
                \newtheorem{definicija}[izrek]{Definicija}
                \newtheorem{opomba}[izrek]{Opomba}
                \newtheorem{primer}[izrek]{Primer}
                \newtheorem{zgled}[izrek]{Zgled}
                \newtheorem{vaja}[izrek]{Vaja}
        }

%%%%%%  Proofs
%%%%%%%%%%%%%%%%%%%%%%%%%%%%%%%%%%%%%%%%%%%%%%%%%%%%%%%%%%%%%
        \newenvironment{dokaz}{
                \goodbreak\par
                \textit{Dokaz.}%
        }{%
                \nopagebreak
                \hfill{\vrule width 1ex height 1ex depth 0ex}
                \medskip
                \goodbreak
        }
%%%%%%%%%%%%%%%%%%%%%%%%%%%%%%%%%%%%%%%%%%%%%%%%%%%%%%%%%%%%%
%%%%%% Macros
%%%%%%%%%%%%%%%%%%%%%%%%%%%%%%%%%%%%%%%%%%%%%%%%%%%%%%%%%%%%%%%%%%%%%%%%%%%%%%%%%%%%%%%%%%%%%%%%%%%%%%%%%%%%%%%%%%%%%%
%%%  Commands
%%%%%%%%%%%%%%%%%%%%%%%%%%%%%%%%%%%%%%%%%%%%%%%%%%%%%%%%%%%%%%%%%%%%%%%%%%%%%%%%%%%%%%%%%%%%%%%%%%%%%%%%%%%%%%%%%%%%%%


%%%%%%  Auxiliary
%%%%%%%%%%%%%%%%%%%%%%%%%%%%%%%%%%%%%%%%%%%%%%%%%%%%%%%%%%%%%
\newcommand{\sizedescriptor}[2]
{
\ifthenelse{\equal{#1}{0}}{}{
\ifthenelse{\equal{#1}{1}}{\big}{
\ifthenelse{\equal{#1}{2}}{\Big}{
\ifthenelse{\equal{#1}{3}}{\bigg}{
\ifthenelse{\equal{#1}{4}}{\Bigg}{
#2}}}}}
}

\newcommand{\someref}{{\small\textcolor{blue}{[\textbf{ref.}]}}}
\newcommand{\intermission}{\bigskip\medskip}
\newcommand{\qt}[1]{{\quotedblbase}{#1}{‘‘}}  % text in quotation marks
\newcommand{\nls}[1]{\qt{\textit{#1}}}  % sentence in a natural language

\definecolor{andrejcolor}{rgb}{0.7,0,0.7}
\definecolor{davorincolor}{rgb}{0,0.45,1}
\definecolor{markocolor}{rgb}{0.7,0.4,0}
\definecolor{matijacolor}{rgb}{0,0.6,0.4}

\newcommand{\andrej}[1]{{\small\textcolor{andrejcolor}{(#1 \ \mbox{--Andrej})}}}
\newcommand{\davorin}[1]{{\small\textcolor{davorincolor}{(#1 \ \mbox{--Davorin})}}}
\newcommand{\marko}[1]{{\small\textcolor{markocolor}{(#1 \ \mbox{--Marko})}}}
\newcommand{\matija}[1]{{\small\textcolor{matijacolor}{(#1 \ \mbox{--Matija})}}}

\definecolor{notecolor}{rgb}{0.6,0.5,0.7}
\newcommand{\note}[1]{{\small\textcolor{notecolor}{(#1)}}}
\newcommand{\alert}[1]{{\small\textcolor{red}{\textbf{#1}}}}


%%%%%%  Logical Quantifiers, λ- and ι-Expressions
%%%%%%%%%%%%%%%%%%%%%%%%%%%%%%%%%%%%%%%%%%%%%%%%%%%%%%%%%%%%%

%  no parenthesis (add x in front of the name of the command)
\NewDocumentCommand{\xall}
{m O{\empty} m}
{\forall\, {#1} \ifthenelse{\equal{#2}{}}{}{\in{#2}} \,.\, {#3}}
\NewDocumentCommand{\xsome}
{m O{\empty} m}
{\exists\, {#1} \ifthenelse{\equal{#2}{}}{}{\in{#2}} \,.\, {#3}}
\NewDocumentCommand{\xexactlyone}
{m O{\empty} m}
{\exists\;\!!\, {#1} \ifthenelse{\equal{#2}{}}{}{\in{#2}} \,.\, {#3}}
\NewDocumentCommand{\xlam}
{m O{\empty} m O{\empty}}
{\lambda{#1} \ifthenelse{\equal{#2}{}}{}{\in{#2}} \,.\, {#3} \ifthenelse{\equal{#4}{}}{}{\in{#4}}}
\NewDocumentCommand{\xthat}
{m O{\empty} m}
{\iota{#1} \ifthenelse{\equal{#2}{}}{}{\in{#2}} \,.\, {#3}}

%  with parenthesis -- the first optional argument is the size (values 0-4)
\NewDocumentCommand{\all}
{O{auto} m O{\empty} m}
{\xall{#2}[#3]{\sizedescriptor{#1}{\left}( {#4} \sizedescriptor{#1}{\right})}}
\NewDocumentCommand{\some}
{O{auto} m O{\empty} m}
{\xsome{#2}[#3]{\sizedescriptor{#1}{\left}( {#4} \sizedescriptor{#1}{\right})}}
\NewDocumentCommand{\exactlyone}
{O{auto} m O{\empty} m}
{\xexactlyone{#2}[#3]{\sizedescriptor{#1}{\left}( {#4} \sizedescriptor{#1}{\right})}}
\NewDocumentCommand{\lam}
{O{auto} m O{\empty} m O{\empty}}
{\xlam{#2}[#3]{\sizedescriptor{#1}{\left}( {#4} \sizedescriptor{#1}{\right})}[#5]}
\NewDocumentCommand{\that}
{O{auto} m O{\empty} m}
{\xthat{#2}[#3]{\sizedescriptor{#1}{\left}( {#4} \sizedescriptor{#1}{\right})}}


%%%%%%  Logic
%%%%%%%%%%%%%%%%%%%%%%%%%%%%%%%%%%%%%%%%%%%%%%%%%%%%%%%%%%%%%
\newcommand{\tvs}{\Omega}  % set of truth values
\newcommand{\true}{\top}  % truth
\newcommand{\false}{\bot}  % falsehood
\newcommand{\etrue}{\bm{\top}}  % emphasized truth
\newcommand{\efalse}{\bm{\bot}}  % emphasized falsehood
\newcommand{\impl}{\Rightarrow}  % implication sign
\newcommand{\revimpl}{\Leftarrow}  % reverse implication sign
\newcommand{\lequ}{\Leftrightarrow}  % equivalence sign
\newcommand{\xor}{\mathbin{\veebar}}  % exclusive disjunction sign
\newcommand{\shf}{\mathbin{\uparrow}}  % Sheffer connective
\newcommand{\luk}{\mathbin{\downarrow}}  % Łukasiewicz connective


%%%%%%  Sets
%%%%%%%%%%%%%%%%%%%%%%%%%%%%%%%%%%%%%%%%%%%%%%%%%%%%%%%%%%%%%
%  \set{1, 2, 3}         ->  {1, 2, 3}
%  \set{a \in X}{a < 1}  ->  {a ∈ X | a < 1}
\NewDocumentCommand{\set}
{O{auto} m G{\empty}}
{\sizedescriptor{#1}{\left}\{ {#2} \ifthenelse{\equal{#3}{}}{}{ \; \sizedescriptor{#1}{\middle}| \; {#3}} \sizedescriptor{#1}{\right}\}}
%\newcommand{\vsubset}{\Mapstochar\cap}
%\newcommand{\finseq}[1]{{#1}^*}
\newcommand{\pst}{\mathcal{P}}
\renewcommand{\complement}[1]{{#1}^C}


%%%%%%  Number Sets, Intervals
%%%%%%%%%%%%%%%%%%%%%%%%%%%%%%%%%%%%%%%%%%%%%%%%%%%%%%%%%%%%%
\newcommand{\NN}{\mathbb{N}}
\newcommand{\ZZ}{\mathbb{Z}}
\newcommand{\QQ}{\mathbb{Q}}
\newcommand{\RR}{\mathbb{R}}
\newcommand{\CC}{\mathbb{C}}
\newcommand{\intoo}[3][\RR]{{#1}_{(#2, #3)}}
\newcommand{\intcc}[3][\RR]{{#1}_{[#2, #3]}}
\newcommand{\intoc}[3][\RR]{{#1}_{(#2, #3]}}
\newcommand{\intco}[3][\RR]{{#1}_{[#2, #3)}}


%%%%%%  Maps and Relations
%%%%%%%%%%%%%%%%%%%%%%%%%%%%%%%%%%%%%%%%%%%%%%%%%%%%%%%%%%%%%
\newcommand{\id}[1][]{\textrm{Id}_{#1}}  % identity map
\newcommand{\argbox}{{\;\!\fbox{\phantom{M}}\;\!}}  % box for a function argument
\newcommand{\rstr}[1]{\left.{#1}\right|}  % map restriction
\newcommand{\im}{\mathrm{im}}  % map image
\newcommand{\parto}{\mathrel{\rightharpoonup}}  % partial mapping sign
\NewDocumentCommand{\rel}
{O{\empty} O{\empty}}
{\ifthenelse{\equal{#1}{}}{\mathscr{R}}{{#1} \mathrel{\mathscr{R}} {#2}}}  % a relation
\NewDocumentCommand{\srel}
{O{\empty} O{\empty}}
{\ifthenelse{\equal{#1}{}}{\mathscr{S}}{{#1} \mathrel{\mathscr{S}} {#2}}}  % a second relation
\newcommand{\dom}{\mathrm{dom}}  % domain
\newcommand{\cod}{\mathrm{cod}}  % codomain
\newcommand{\dd}[1]{D_{#1}}  % domain of definition
\newcommand{\rn}[1]{Z_{#1}}  % range
\newcommand{\graph}[1]{\Gamma_{#1}}  % graph of a (partial) function
\NewDocumentCommand{\img}  % image
{O{\empty} m G{\empty}}
{{#2}_*\ifthenelse{\equal{#3}{}}{}{\!\sizedescriptor{#1}{\left}( {#3} \sizedescriptor{#1}{\right})}}
\NewDocumentCommand{\pim}  % preimage
{O{\empty} m G{\empty}}
{{#2}^*\ifthenelse{\equal{#3}{}}{}{\!\sizedescriptor{#1}{\left}( {#3} \sizedescriptor{#1}{\right})}}
\newcommand{\ec}[2][]{[\:\!{#2}\:\!]_{#1}}  % equivalence class
\newcommand{\transposed}[1]{\widehat{#1}}


%%%%%%  Misc.
%%%%%%%%%%%%%%%%%%%%%%%%%%%%%%%%%%%%%%%%%%%%%%%%%%%%%%%%%%%%%
\newcommand{\df}[1]{\emph{\textbf{#1}}}  % defined notion
\newcommand{\oper}{\mathop{\circledast}}  % symbol for a general operation
\newcommand{\ism}{\cong}  % isomorphic
\newcommand{\equ}{\sim}  % equivalent
\newcommand{\dfeq}{\mathrel{\mathop:}=}  % definitional equality
\newcommand{\revdfeq}{=\mathrel{\mathop:}}  % reverse definitional equality
\newcommand{\isdefined}[1]{{#1}\!\downarrow}  % given value is defined
\newcommand{\kleq}{\simeq}  % Kleene equality
\newcommand{\claim}[3]{{#1} \;\colon\; \frac{#2}{#3}}  % claim, divided on context, assumptions, conclusions
\newcommand{\unit}{\mathord{\bm{*}}}  % element in a generic singleton
\NewDocumentEnvironment{implproof}  % proof of an implication
{O{\empty} G{\empty} O{=>} G{\empty}}
{
\begin{description}
\item[\quad$\sizedescriptor{#1}{\left}({#2}
\ifthenelse{\equal{#3}{=>}}{\impl}{
\ifthenelse{\equal{#3}{<=}}{\revimpl}{
\ifthenelse{\equal{#3}{->}}{\rightarrow}{
\ifthenelse{\equal{#3}{<-}}{\leftarrow}{
#3
}}}} {#4}\sizedescriptor{#1}{\right})$]\ \vspace{0.3em}\\
}
{
\end{description}
}


%%%%%%%%%%%%%%%%%%%%%%%%%%%%%%%%%%%%%%%%%%%%%%%%%%%%%%%%%%%%%%%%%%%%%%%%%%%%%%%%%%%%%%%%%%%%%%%%%%%%%%%%%%%%%%%%%%%%%%


\begin{document}

   %--------------------------------------------------------------------
   %--------------------------------------------------------------------
   % TITLE PAGE


        \title{Logika in množice} % če se spremeni naslov, je treba spremeniti tudi zgoraj v paketu hyperref
        \author{Andrej Bauer \and Davorin Lešnik \and Marko Petkovšek \and Matija Pretnar}

        \maketitle


   %--------------------------------------------------------------------
   %--------------------------------------------------------------------
   % Foreword

   \chapter*{Predgovor}%\addcontentsline{toc}{chapter}{\numberline{}Predgovor}


        %--------------------------------------------------------------------
        %--------------------------------------------------------------------
        % TOC


        \tableofcontents
        \listoftables


        %--------------------------------------------------------------------
        %--------------------------------------------------------------------
        % BODY


        \chapter{Matematično izražanje}

	\section{Simbolni zapis}
	\section{Pravila dokazovanja}
	\section{Definicije}
        \chapter{Logika in pravila sklepanja (dodatno poglavje)}
\label{chap:logika}


\textbf{Opomba:} To poglavje je del učbenika v nastajanju in ni povsem v skladu s predavanji. Kljub temu ga vključujem v te zapiske, ker vsebuje precej koristnih nasvetov in misli.

%%%%%%%%%%%%%%%%%%%%%%%%%%%%%%%%%%%%%%%%%%%%%%%%%%%%%%%%%%%%%%%%%%%%%%
\section{Kaj je matematični dokaz?}
\label{sec:kaj-je-dokaz}

V srednji šoli se dijaki pri matematiki učijo, \emph{kako} se kaj
izračuna. Na univerzi imajo študentje matematike pred seboj
zahtevnejšo nalogo: poleg \emph{kako} morajo vedeti tudi \emph{zakaj}.
Od njih se pričakuje, da bodo računske postopke znali tudi utemeljiti,
ne pa samo slediti pravilom, ki jih je predpisal učitelj. Razumeti
morajo dokaze znamenitih izrekov in sami poiskati dokaze preprostih
izjav. Da bi se lažje spopadli s temi novimi nalogami, bomo prvi del
predmeta Logika in množice posvetili matematični infrastrukturi:
izjavam, pra\-vi\-lom sklepanja in dokazom. Učili se bomo, kako pišemo
dokaze, kako jih analiziramo in kako jih sami poiščemo.

Osrednji pojem matematične aktivnosti je \emph{dokaz}. Namen dokaza je
s pomočjo točno določenih in vnaprej dogovorjenih \emph{pravil
  sklepanja} utemeljiti neko matematično \emph{izjavo}. Načeloma mora
dokaz vsebovati vse podrobnosti in natanko opisati posamezne korake
sklepanja, ki privedejo do želene matematične izjave. Ker bi bili taki
dokazi zelo dolgi in bi vsebovali nezanimive podrobnosti, matematiki
običajno predstavijo samo oris ali glavno zamisel dokaza. Izkušenemu
matematiku to zadošča, saj zna oris sam dopolniti do pravega dokaza.
Matematik začetnik seveda potrebuje več podrobnosti. Poglejmo si
primer.

\begin{izrek}
  \label{izr:n3-n-deljivo-3}
  Za vsako naravno število $n$ je $n^3 - n$ deljivo s~$3$.
\end{izrek}

\noindent
Po kratkem premisleku bi izkušeni matematik zapisal:

\begin{quote}
  \begin{proof}
    Očitno.
  \end{proof}
\end{quote}

\noindent
To ni dokaz, izkušeni matematik nam le dopoveduje, da je (zanj) izrek
zelo lahek in da nima smisla izgubljati časa s pisanjem dokaza.
Začetnik, ki težko razume že sam izrek, bo ob takem ">dokazu"< seveda
zgrožen. Verjetno bo najprej preizkusil izrek na nekaj primerih:
%
\begin{align*}
  1^3 - 1 &= 0,\\
  2^3 - 2 &= 8 - 2 = 6,\\
  3^3 - 3 &= 27 - 4 = 24,\\
  4^3 - 4 &= 64 - 4 = 60.
\end{align*}
%
Res dobivamo večkratnike števila~$3$. Ali smo izrek s tem dokazali? Seveda ne!
Preizkusili smo le štiri primere, ostane pa jih še neskončno mnogo. Kdor
misli, da lahko iz nekaj primerov sklepa na splošno veljavnost, naj v poduk
vzame naslednjo nalogo.

\begin{vaja}
  Ali je $n^2 - n + 41$ praštevilo za vsako naravno število~$n$?
\end{vaja}

\noindent
Ko izkušenega matematika prosimo, da naj nam vsaj pojasni idejo dokaza,
zapiše:

\begin{quote}
  \begin{proof}
    Število $n^3 - n$ je zmnožek treh zaporednih naravnih števil.
  \end{proof}
\end{quote}

\noindent
To še vedno ni dokaz, ampak samo namig. Starejši študenti matematike pa
bi iz namiga morali znati sestaviti naslednji dokaz:

\begin{quote}
  \begin{proof}
    Ker je $n^3 - n = (n-1) \cdot n \cdot (n+1)$, je $n^3 - n$ zmnožek
    treh zaporednih naravnih števil, od katerih je eno deljivo s~$3$,
    torej je tudi $n^3 - n$ deljivo s~$3$.
  \end{proof}
\end{quote}

\noindent
(Mimogrede, s škatlico $\Box$ označimo konec dokaza.) Čeprav bi bila
večina matematikov s tem dokazom zadovoljna, bi morali za popoln dokaz
preveriti še nekaj podrobnosti:
%
\begin{enumerate}
\item Ali res velja $n^3 - n = (n-1) \cdot n \cdot (n+1)$?
\item Ali je res, da je izmed treh zaporednih naravnih števil eno
  vedno deljivo s~$3$?
\item Ali je res, da je zmnožek treh števil deljiv s~$3$, če je eno od
  števil deljivo s~$3$?
\end{enumerate}
%
S srednješolskim znanjem algebre ugotovimo, da je odgovor na prvo
vprašanje pritrdilen. Tudi odgovora na drugo in tretje vprašanje sta
očitno pritrdilna, mar ne? To pa ne pomeni, da ju ni treba dokazati.
Nasprotno, zgodovina matematike nas uči, da moramo prav ">očitne"<
izjave še posebej skrbno preveriti.

\begin{vaja}
  Kakšno je tvoje mnenje o resničnosti naslednjih izjav? Vprašaj
  starejše kolege, asistente in učitelje, kaj menijo oni. Ali znajo
  svoje mnenje utemeljiti z dokazi?
  %
  \begin{enumerate}
  \item Sodih števil je manj kot naravnih števil.
  \item Kroglo je mogoče razdeliti na pet delov tako, da lahko iz njih
    sestavimo dve krogli, ki sta enako veliki kot prvotna krogla.
  \item Sklenjena krivulja v ravnini, ki ne seka same sebe, razdeli
    ravnino na dve območji, eno omejeno in eno neomejeno.
  \item S krivuljo ne moremo prekriti notranjosti kvadrata.
  \item Če ravnino razdelimo na tri območja, potem zagotovo obstaja
    točka, ki je dvomeja in ni tromeja med območji.
  \end{enumerate}
\end{vaja}

\noindent
%
Vrnimo s k izreku~\ref{izr:n3-n-deljivo-3}. Če dokaz zapišemo preveč
podrobno, postane dolgočasen in ne\-ra\-zumljiv:

\begin{quote}
  \begin{proof}
    Naj bo $n$ poljubno naravno število. Tedaj velja
    %
    \begin{align*}
      n^3 - n
      &= n \cdot n^2 - n \cdot 1 \\
      &= n \cdot (n^2 - 1) \\
      &= n \cdot ((n + 1) \cdot (n - 1)) \\
      &= n \cdot ((n - 1) \cdot (n + 1)) \\
      &= (n \cdot (n - 1)) \cdot (n + 1) \\
      &= (n - 1) \cdot n \cdot (n + 1).
    \end{align*}
    %
    Vidimo, da je $n^3 - n$ zmnožek treh zaporednih naravnih števil.
    Dokažimo, da je eno od njih deljivo s~$3$. Število $n$ lahko
    enolično zapišemo kot $n = 3 k + r$, kjer je $k$ naravno število
    in $r = 0$, $r = 1$ ali $r = 2$. Obravavajmo tri primere:
    %
    \begin{itemize}
    \item če je $r = 0$, je $n = 3 k$, zato je $n$ deljiv s~$3$,
    \item če je $r = 1$, je $n - 1 = (3 k + 1) - 1 = 3 k + (1 - 1) = 3
      k + 0 = 3 k$, zato je $n-1$ deljiv s~$3$,
    \item če je $r = 2$, je $n + 1 = (3 k + 2) + 1 = 3 k + (2 + 1) = 3
      k + 3 = 3 k + 3 \cdot 1 = 3 (k +1)$, zato je $n+1$ deljiv s~$3$.
    \end{itemize}
    %
    Vemo torej, da je $n - 1$, $n$ ali $n + 1$ deljiv s~$3$.
    Obravnavamo tri primere:
    %
    \begin{itemize}
    \item Če je $n - 1$ deljiv s~$3$, tedaj  obstaja naravno število
      $k$, da je $n - 1 = 3 k$. V tem primeru je $(n - 1) n (n + 1) =
      (3 k) n (n + 1) = 3 (k n (n + 1))$, zato je $(n - 1) n (n + 1)$
      deljivo s~$3$.
    \item Če je $n$ deljiv s~$3$, tedaj obstaja naravno število $k$,
      da je $n = 3 k$. V tem primeru je $(n - 1) n (n + 1) = (n - 1)
      (3 k) n (n + 1) = (3 k) (n - 1) (n + 1) = 3 (k (n - 1) (n +
      1))$, zato je $(n - 1) n (n + 1)$ deljivo s~$3$.
    \item Če je $n + 1$ deljiv s~$3$, tedaj obstaja naravno število
      $k$, da je $n + 1 = 3 k$. V tem primeru je $(n - 1) n (n + 1) =
      (n - 1) n (3 k) = (n - 1) (3 k) n = (3 k) (n - 1) n = 3 (k (n -
      1) n)$, zato je $(n - 1) n (n + 1)$ deljivo s~$3$.
    \end{itemize}
    %
    V vsakem primeru je $(n - 1) n (n + 1)$ deljivo s~$3$. Ker smo
    dokazali, da je $n^3 = n = (n - 1) n (n + 1)$, je tudi $n^3 - n$
    deljivo s~$3$.
  \end{proof}
\end{quote}

\begin{vaja}
  S kolegi se igraj naslednjo igro.\footnote{%
    Igranje odsvetujemo zunaj prostorov Fakultete za matematiko in fiziko.}
  Prvi igralec v zgornjem dokazu poišče korak, ki ga je treba še dodatno
  utemeljiti. Drugi igralec ga utemelji. Nato prvi igralec poišče nov korak,
  ki ga je treba še dodatno utemeljiti in igra se ponovi. Zgubi tisti, ki se prvi naveliča igrati. Ali lahko igra traja neskončno dolgo?
\end{vaja}

Matematični dokaz ima dvojno vlogo. Po eni strani je utemeljitev matematične
izjave, zato mora biti čim bolj podroben. V idealnem primeru bi bil dokaz
zapisan tako, da bi lahko njegovo pravilnost preverili mehansko, z
računalnikom. Po drugi strani je dokaz sredstvo za komunikacijo idej med
matematiki, zato mora vsebovati ravno pravo mero podrobnosti. Mera pa je
odvisna od tega, komu je dokaz namenjen. Te socialne komponente se študenti
učijo skozi prakso v toku študija. Dokazu kot povsem matematičnemu pojmu pa se
bomo posvetili prav pri predmetu Logika in množice. Pojasnili bomo, kaj je
dokaz kot matematični konstrukt in kako ga zapišemo tako podrobno, da je res
mehansko preverljiv. Naučili se bomo tudi nekaj preprostih tehnik iskanja
dokazov, ki pa še zdaleč ne bodo zadostovale za reševanje zares zanimivih
matematičnih problemov, ki zahtevajo poglobljeno znanje, vztrajnost, kanček
talenta in nekaj sreče.


%%%%%%%%%%%%%%%%%%%%%%%%%%%%%%%%%%%%%%%%%%%%%%%%%%%%%%%%%%%%%%%%%%%%%%

\section{Simbolni zapis matematičnih izjav}
\label{sec:simbolni-zapis-izjav}

Matematična \textbf{izjava} je smiselno besedilo, ki izraža kako lastnost ali
razmerje med matematičnimi objekti (števili, liki, funkcijami, množicami
itn.). Primeri matematičnih izjav:
%
\begin{itemize}
\item $2 + 2 = 5$.
\item Točke $P$, $Q$ in $R$ so kolinearne.
\item Enačba $x^2 + 1 = 0$ nima realnih rešitev.
\item $a > 5$.
\item $\phi \lor \psi \lthen (\lnot \phi \lthen \psi)$.
\end{itemize}
%
Vidimo, da je lahko izjava resnična, neresnična, ali pa je resničnost
izjave \emph{odvisna} od vrednosti spremenljivk, ki nastopajo v njej.
Primeri besedila, ki niso matematične izjave:
%
\begin{itemize}
\item Ali je $2 + 2 = 5$?
\item Za vsak $x > 5$.
\item Študenti bi morali znati reševati diferencialne enačbe.
\item Od nekdaj lepe so Ljubljanke slovele, al lepše od Urške bilo ni nobene.
\item $\phi \lor ) \psi \lthen \psi$.
\end{itemize}
%
Matematične izjave običajno pišemo kombinirano v naravnem jeziku in z
matematični simboli, saj so tako najlažje razumljive ljudem. Za
potrebe matematične logike pa izjave pišemo \emph{samo} z
matematičnimi simboli. Tako zapisani izjavi pravimo \textbf{logična
  formula}. V ta namen moramo nadomestiti osnovne gradnike izjav, kot
so ">in"<, ">ali"< in ">za vsak"<, z \textbf{logičnimi operacijami}.
Le-te delimo na tri sklope. V prvi sklop sodita \textbf{logični
  konstanti}:
%
\begin{itemize}
\item resnica $\top$,
\item neresnica $\bot$.
\end{itemize}
%
V računalništvu resnico $\top$ pogosto označimo z $1$ ali \texttt{True} in
neresnico $\bot$ z $0$ ali \texttt{False}. Naslednji sklop so \textbf{logični
vezniki}, s katerimi sestavljamo nove izjave iz že danih:
%
\begin{itemize}
\item konjunkcija $\phi \land \psi$, beremo ">$\phi$ in $\psi$"<,
\item disjunkcija $\phi \lor \psi$, beremo ">$\phi$ ali $\psi$"<,
\item implikacija $\phi \lthen \psi$, beremo ">če $\phi$ potem $\psi$"<,
\item ekvivalenca $\phi \liff \psi$, beremo ">$\phi$ če, in samo če, $\psi$"< ali pa ">$\phi$ natanko tedaj, kadar~$\psi$"<,
\item negacija $\lnot \phi$, beremo ">ne $\phi$"<,
\end{itemize}
%
V tretji sklop sodita \textbf{logična kvantifikatorja}:
%
\begin{itemize}
\item univerzalni kvantifikator $\all{x \in S} \phi$, beremo ">za vse $x$
  iz $S$ velja $\phi$"<,
\item eksistenčni kvantifikator $\some{x \in S} \phi$, beremo ">obstaja
  $x$ v $S$, da velja $\phi$"<,
\end{itemize}
%
Pri tem je $S$ množica, razred\footnote{V poglavju~\ref{chap:mnozice}
  bomo spoznali razliko med množicami in razredi, zaenkrat si $S$
  predstavljamo kot množico.} ali tip spremenljivke~$x$. V praksi se
uporablja več inačic zapisa za kvantifikatorje, kot so ">$\forall x : S
.\, \phi$"<, ">$\forall x \in S : \phi$"< in ">$(\forall x \in S)
\phi$"<. Srečamo tudi zapis ">$\phi, \forall x \in S$"<, ki pa ga
odsvetujemo.

\textbf{Neomejena kvantifikatorja} $\all{x} \phi$ in
$\some{x} \phi$ se uporabljata, kadar je vnaprej znana množica $S$,
po kateri teče spremenljivka~$x$. V matematičnem besedilu je običajno
razvidna iz spremnega besedila, včasih pa je treba upoštevati
ustaljene navade: $n$ je naravno ali celo število, $x$ realno, $f$ je
funkcija ipd.

V uporabi so nekatere ustaljene okrajšave:
%
\begin{xalignat*}{3}
  &\some{x,y \in S} \phi,&
  &\text{pomeni}&
  &\some{x}{S} (\some{y}{S} \phi),\\
  %
  &\all{x \in S,y \in T} \phi,&
  &\text{pomeni}&
  &\all{x \in S}(\all{y \in T} \phi),\\
  %
  &\phi \liff \psi \liff \rho \liff \sigma&
  &\text{pomeni}&
  &(\phi \liff \psi) \land (\psi \liff \rho) \land (\rho \liff \sigma),\\
  %
  &f(x) = g(x) = h(x) = i(x)&
  &\text{pomeni}&
  &f(x) = g(x) \land g(x) = h(x) \land h(x) = i(x),\\
  &a \leq b < c \leq d&
  &\text{pomeni}&
  &a \leq b \land b < c \land c \leq d.
\end{xalignat*}
%
Nekatere okrajšave odsvetujemo. V nizu neenakosti naj gredo vse
primerjave v isto smer. Torej ne pišemo $a \leq b < c \geq d$, ker se
zlahka zmotimo in mislimo, da velja $a \geq d$. To bi morali zapisati
ločeno kot $a \leq b < c$ in $c \geq d$. Prav tako ne nizamo neenakosti,
saj premnogi iz $f(x) \neq g(x) \neq h(x)$ ">sklepajo"< $f(x) \neq
h(x)$, čeprav neenakost \emph{ni} tranzitivna relacija. Zapis $f(x) =
g(x) \neq h(x) = i(x)$ je v redu, saj ena sama neenakost ne povzroči
težav.

\begin{vaja}
  Zapiši $f(x) = g(x) \neq h(x) = i(x)$ brez okrajšav.
\end{vaja}

Povejmo še nekaj o pisanju oklepajev. Oklepaji povedo, katera
operacija ima prednost. Kadar manjkajo, moramo poznati dogovorjeno
\textbf{prioriteto} operacij. Na primer, ker ima množenje višjo
prioriteto kot seštevanje, je $5 \cdot 3 + 8$ enako $(5 \cdot 3) + 8$
in ne $5 \cdot (3 + 8)$. Tudi logične operacije imajo svoje
prioritete, ki pa niso tako splošno znane kot prioritete aritmetičnih
operacij. Zato bodite pazljivi, ko berete tuje besedilo.

Mi bomo privzeli naslednje prioritete logičnih operacij:
%
\begin{itemize}
\item negacija $\lnot$ ima prednost pred
\item konjunkcijo $\land$, ki ima prednost pred
\item disjunkcijo $\lor$, ki ima prednost pred
\item implikacijo $\lthen$, ki ima prednost pred
\item kvantifikatorjema $\forall$ in $\exists$.
\end{itemize}
%
Na primer:
%
\begin{itemize}
\item $\lnot \phi \lor \psi$ je isto kot $(\lnot \phi) \lor \psi$,
\item $\lnot \lnot \phi \lthen \phi$ je isto kot $(\lnot(\lnot\phi))
  \lthen \phi$,
\item $\phi \lor \psi \land \rho$ je isto kot $\phi \lor (\psi \land \rho)$,
\item $\phi \land \psi \lthen \phi \lor \psi$ je isto kot $(\phi
  \land \psi) \lthen (\phi \lor \psi)$,
\item $\all{x \in S} \phi \lthen \psi$ je isto kot $\all{x \in S} (\phi
    \lthen \psi)$,
\item $\some{x \in S} \phi \land \psi$ je isto kot $\some{x \in S} (\phi
    \land \psi)$.
\end{itemize}

V aritmetiki poznamo poleg prioritete operacij tudi \textbf{levo} in
\textbf{desno asociranost}. Denimo, seštevanje je levo asocirano,
ker beremo $5 + 3 + 7$ kot $(5 + 3) + 7$, saj najprej izračunamo $5 +
3$ in nato $8 + 7$. Pri seštevanju to sicer ni pomembno in bi lahko
seštevali tudi $3 + 7$ in nato $5 + 10$. Drugače je z odštevanjem,
kjer $5 - 3 - 7$ pomeni $(5 - 3) - 7$ in ne $5 - (3 - 7)$. Tudi za
logične operacije velja dogovor o njihovi asociranosti. Konjunkcija in
disjunkcija sta levo asocirani:
% 
\begin{align*}
  \phi \land \psi \land \rho
  &\qquad\text{pomeni}\qquad
  (\phi \land \psi) \land \rho,\\
  \phi \lor \psi \lor \rho
  &\qquad\text{pomeni}\qquad
  (\phi \lor \psi) \lor \rho.
\end{align*}
%
Za disjunkcijo in konjunkcijo sicer ni pomembno, kako postavimo
oklepaje, ker sta obe možnosti med seboj ekvivalentni, vendar je prav,
da natančno določimo, katera od njiju je mišljena. V logiki je
implikacija desno asocirana:
%
\begin{equation*}
  \phi \lthen \psi \lthen \rho
  \qquad\text{pomeni}\qquad
  \phi \lthen (\psi \lthen \rho).
\end{equation*}
%
Tu \emph{ni} vseeno, kako postavimo oklepaje, saj $\phi \lthen (\psi
\lthen \rho)$ in $(\phi \lthen \psi) \lthen \rho$ v splošnem nista
ekvivalentna. Vendar pozor! Ko matematiki, ki niso logiki, v
matematičnem besedilu zapišejo
%
\begin{equation*}
  \phi \lthen \psi \lthen \rho,
\end{equation*}
%
s tem skoraj vedno mislijo
%
\begin{equation*}
  (\phi \lthen \psi) \land (\psi \lthen \rho).
\end{equation*}
%
Zakaj? Zato ker je to priročen zapis, ki nakazuje zaporedje sklepov
">iz $\phi$ sledi $\psi$ in nato iz $\psi$ sledi $\rho$"<, še posebej,
če je zapisan v več vrsticah. Recimo, za nenegativni števili $x$ in
$y$ bi takole zapisali utemeljitev neenakosti med aritmetično in
geometrijsko sredino:
%
\begin{align*}
  & (x - y)^2 \geq 0 \lthen \\
  & x^2 - 2 x y + y^2 \geq 0 \lthen
  \tag{razstavimo}\\
  & x^2 + 2 x y + y^2 \geq 4 x y \lthen
  \tag{prištejemo $4 x y$}\\
  & (x + y)^2 \geq 4 x y \lthen
  \tag{faktoriziramo}\\
  & \frac{(x + y)^2}{4} \geq x y \lthen
  \tag{delimo s $4$}\\
  & \frac{x+y}{2} \geq \sqrt{x y}.
  \tag{korenimo}
\end{align*}
%ANDREJ: meni je Gordon rekel, da utemeljitve sledijo sklepu, torej so
% eno vrstico niže.
%
Matematiki radi celo spustijo znak $\lthen$ in preprosto vsak
naslednji sklep napišejo v svojo vrstico. Ker torej velja tak ustaljen
način pisanja zaporedja sklepov, je varneje pisati $\phi \lthen (\psi
\lthen \rho)$ brez oklepajev, da ne povzročamo zmede.
%ANDREJ: zadnjega stavka ne razumem.

%%%%%%%%%%%%%%%%%%%%%%%%%%%%%%%%%%%%%%%%%%%%%%%%%%%%%%%%%%%%%%%%%%%%%%

\section{Kako beremo in pišemo simbolni zapis}
\label{sec:simbolni-zapis}

Izjave, zapisane v simbolni obliki, ni težko prebrati. Na primer,
%
\begin{equation*}
  \all{x, y \in \RR}
    x^2 = 4 \land y^2 = 4 \lthen x = y,
\end{equation*}
%
preberemo:
%
\begin{quote}
  ">Za vse realne $x$ in $y$, če je $x^2$ enako $4$ in $y^2$ enako
  $4$, potem je $x$ enako $y$."<
\end{quote}
%
Več izkušenj pa je potrebnih, da \emph{razumemo} matematični pomen
take izjave, v tem primeru:
%
\begin{quote}
  ">Enačba $x^2 = 4$ ima največ eno realno rešitev."<
\end{quote}
%
Začetnik potrebuje nekaj vaje, da se navadi brati simbolni zapis. Tudi
prevod v obratno smer, iz besedila v simbolno obliko, ni enostaven,
zato povejmo, kako se prevede nekatere standardne fraze.

\subsubsection{">$\phi$ je zadosten pogoj za $\psi$."<}

To pomeni, da zadošča dokazati $\phi$ zato, da dokažemo $\psi$, ali v
simbolni obliki
%
\begin{equation*}
  \phi \lthen \psi.
\end{equation*}

\subsubsection{">$\phi$ je potreben pogoj za $\psi$."<}

To pomeni, da $\psi$ ne more veljati, ne da bi veljal~$\phi$. Z drugimi
besedami, če velja $\psi$, potem velja tudi $\phi$, kar se v simbolni obliki
zapiše
%
\begin{equation*}
  \psi \lthen \phi.
\end{equation*}

\subsubsection{">$\phi$ je zadosten in potreben pogoj za $\psi$."<}

To je kombinacija prejšnjih dveh primerov, ki trdi, da iz $\phi$ sledi
$\psi$ in iz $\psi$ sledi $\phi$, kar pa je ekvivalenca:
%
\begin{equation*}
  \phi \liff \psi.
\end{equation*}

\begin{vaja}
  Je ">$n$ je sod in $n > 2$"< \textbf{potreben} ali \textbf{zadosten}
  pogoj za ">$n$ ni praštevilo"<?
\end{vaja}


\subsubsection{">Naslednje izjave so ekvivalentne: $\phi$, $\psi$, $\rho$ in $\sigma$."<}

To pomeni, da sta vsaki dve izmed danih izjav ekvivalentni, se pravi
%
\begin{equation*}
  (\phi \liff \psi) \land (\phi \liff \rho) \land (\phi \liff \sigma) \land (\psi \liff \rho)
  \land (\psi \liff \sigma) \land (\rho \liff \sigma).
\end{equation*}
%
Ker je ekvivalenca tranzitivna relacija, ni treba obravnavati vseh
kombinacij, zadostujejo že tri, ki dane izjave ">povežejo"< med seboj:
%
\begin{equation*}
  (\phi \liff \psi) \land (\psi \liff \rho) \land (\rho \liff \sigma).
\end{equation*}
%
To pišemo krajše kar kot
%
\begin{equation*}
  \phi \liff \psi \liff \rho \liff \sigma,
\end{equation*}
%
čeprav je formalno gledano tako zapis nepravilen. V
razdelku~\ref{sec:ekvivalenca} bomo spoznali, kako se tako zaporedje
ekvivalenc dokaže s ciklom implikacij $\phi \lthen \psi \lthen \rho
\lthen \sigma \lthen \phi$.

\begin{vaja}
  Podaj konkretne primere izjav $\phi$, $\psi$ in $\rho$, iz katerih
  je razvidno, da izjava $(\phi \liff \psi) \land (\psi \liff \rho)$
  \emph{ni} ekvivalentna niti $(\phi \liff \psi) \liff \rho$ niti
  $\phi \liff (\psi \liff \rho)$.
\end{vaja}



\subsubsection{">Za vsak $x$ iz $S$, za katerega velja $\phi$, velja tudi
  $\psi$."<}

To lahko preberemo tudi kot ">Za vsak $x$ iz $S$, če zanj velja $\phi$,
potem velja $\psi$,"< kar je v simbolni obliki
%
\begin{equation*}
  \all{x \in S} \phi \lthen \psi.
\end{equation*}
%
Tudi izjave oblike ">vsi $\phi$-ji so $\psi$-ji"< so te oblike, denimo ">vsa
od dva večja praštevila so liha"< zapišemo
%
\begin{equation*}
  \all{n \in \NN} n > 2 \land \text{$n$ je praštevilo} \lthen \text{$n$ je lih}.
\end{equation*}

\begin{vaja}
  V simbolni obliki zapiši ">$n$ je lih"< in ">$n$ je praštevilo"<.
  Namig: $n$ je lih, kadar obstaja naravno število $k$, za katerega
  velja $n = 2 k + 1$, in $n$ je praštevilo, kadar \emph{ni} zmnožek
  dveh naravnih števil, ki sta obe večji od~$1$.
\end{vaja}


\subsubsection{">Enačba $f(x) = g(x)$ nima realne rešitve."<}

To lahko povemo takole: ni res, da obstaja $x \in \RR$, za katerega bi
veljalo $f(x) = g(x)$. S simboli zapišemo
%
\begin{equation*}
  \lnot \some{x \in \RR} f(x) = g(x).
\end{equation*}
%
Opozoriti velja, da iz same enačbe ne moremo vedno sklepati, kaj je
neznanka. V enačbi $a x^2 + b x + c = 0$ bi za neznanko lahko načeloma
imeli katerokoli od štirih spremenljivk $a$, $b$, $c$ in $x$, ali pa
kar vse. Večina matematikov bi sicer uganila, da je najverjetneje
neznanka $x$, vendar se v splošnem ne moremo zanašati na običaje in
uganjevanje, ampak moramo točno povedati, kateri simboli so
\textbf{neznanke} in kateri \textbf{parametri}.

\begin{vaja}
  Zapiši v simbolni obliki: ">Sistem enačb
  %
  \begin{align*}
    a_1 x + b_1 y &= c_1,\\
    a_2 x + b_2 y &= c_2
  \end{align*}
  %
  nima pozitivnih realnih rešitev $x, y$."<
\end{vaja}

\begin{vaja}
  Zapiši v simbolni obliki:
  \begin{enumerate}
  \item ">Enačba $f(x) = g(x)$ ima največ eno realno rešitev."<
  \item ">Enačba $f(x) = g(x)$ ima več kot eno realno rešitev."<
  \item ">Enačba $f(x) = g(x)$ ima natanko dve realni rešitvi."<
  \end{enumerate}
\end{vaja}


\subsubsection{">Brez izgube za splošnost."<}

V matematičnih besedilih najdemo frazo ">brez izgube za splošnost"<
kot v naslednjem primeru.

\begin{izrek}
  \label{izrek:abc-vsota-razlik-soda}
  Za vsa cela števila $a$, $b$ in $c$ je $|a-b|+|b-c|+|c-a|$ sodo
  število.
\end{izrek}

\begin{proof}
  Brez izgube za splošnost smemo predpostaviti $a \geq b \geq c$.
  Tedaj velja
  %
  \begin{equation*}
    |a-b| + |b-c| + |c-a| = (a - b) + (b - c) - (c - a) = 2 (a - c),
  \end{equation*}
  %
  kar je sodo število.
\end{proof}

Fraza ">brez izgube za splošnost"< nakazuje, da dokaz obravnava le eno
od večih možnosti. Načeloma bi morali obravnavati tudi ostale
možnosti, ki pa jih je pisec dokaza opustil, ker so bodisi zelo lahke
bodisi zelo podobne tisti, ki jo dokaz obravnava. Za začetnika je
najtežje dognati, katere so preostale možnosti in zakaj se je pisec
dokaza pravzaprav odločil zanje. Avtor zgornjega dokaza je verjetno
opazil, da števila $a$, $b$ in $c$ v izrazu $|a-b|+|b-c|+|c-a|$
nastopajo \emph{simetrično}: če jih premešamo, se izraz ne spremeni.
Denimo, ko zamenjamo $a$ in $b$, dobimo $|b-a|+|a-c|+|c-b|$, kar je
enako prvotnemu izrazu $|a-b|+|b-c|+|c-a|$. Torej lahko izmed šestih
možnosti
%
\begin{xalignat*}{3}
  & a \geq b \geq c,&
  & a \geq c \geq b,&
  & b \geq a \geq c,\\
  & b \geq c \geq a,&
  & c \geq a \geq b,&
  & c \geq b \geq a
\end{xalignat*}
%
obravnavamo le eno. Seveda pisanje dokazov, pri katerih večji del
dokaza opustimo, zahteva pazljivost in nekaj izkušenj.

\begin{vaja}
  Dokaži izrek~\ref{izrek:abc-vsota-razlik-soda} tako, da obravnavaš
  samo možnost $b \geq c \geq a$ in zraven dopišeš ">brez izgube za
  splošnost"<.
\end{vaja}


%%%%%%%%%%%%%%%%%%%%%%%%%%%%%%%%%%%%%%%%%%%%%%%%%%%%%%%%%%%%%%%%%%%%%%
\section{Definicije}
\label{sec:definicije}


Poznamo tri vrste definicij. Prva in najpreprostejša je definicija, ki
služi kot \textbf{okrajšava} za daljši izraz. To smemo vedno nadomestiti
s prvotnim izrazom. Na primer, funkcija ">hiperbolični tangens"<
$\tanh(x)$ je definirana kot
%
\begin{equation*}
  \tanh(x) = \frac{e^{2 x} - 1}{e^{2 x} + 1}.
\end{equation*}
%
Lahko bi shajali tudi brez zapisa $\tanh(x)$, vendar bi morali potem
povsod pisati daljši izraz $\frac{e^{2 x} - 1}{e^{2 x} + 1}$, kar bi
bilo nepregledno.

Druga vrsta definicije je vpeljava novega matematičnega pojma.
Študenti prvega letnika matematike spoznajo celo vrsto novih pojmov
(grupa, vektorski prostor, limita, stekališče, metrika itn.), s
katerimi si razširijo sposobnost matematičnega razmišljanja.
Matematiki cenijo dobre definicije in vpeljavo novih matematičnih
pojmov vsaj toliko, kot dokaze težkih izrekov.

Tretja vrsta definicije je \textbf{konstrukcija} matematičnega objekta s
pomočjo dokaza o enoličnem obstoju. O tem bomo povedali več v
razdelku~\ref{sec:enolicni-obstoj}.

\section{Pravila sklepanja in dokazi}
\label{sec:pravila-sklepanja-in-dokazi}


Povedali smo že, da je dokaz utemeljitev neke matematične izjave. V
razdelku~\ref{sec:kaj-je-dokaz} smo govorili o tem, da so dokazi
mešanica besedila in simbolov, ki jih matematiki uporabljajo tako za
utemeljitev matematičnih izjav, kakor tudi za razlago in podajanje
matematičnih idej. V tem razdelku se posvetimo \textbf{formalnim
  dokazom}, ki so logične konstrukcije namenjene mehanskemu
preverjanju pravilnosti izjav.

Za vsako logično operacijo bomo podali \textbf{formalna pravila
  sklepanja}, ki jih smemo uporabljati v formalnem dokazu. Pravilo
sklepanja shematsko zapišemo
%
\begin{equation*}
  \inferrule{\phi \\ \psi \\ \rho}{\sigma}
\end{equation*}
%
in ga preberemo: ">Če smo dokazali $\phi$, $\psi$ in $\rho$, smemo
sklepati $\sigma$."< Izjavam nad črto pravimo \textbf{hipoteze}, izjavi
pod črto pa \textbf{sklep}. Hipotez je lahko nič ali več, sklep mora
biti natanko en. Pravilo sklepanja brez hipotez se imenuje
\textbf{aksiom}.

Da bomo lahko pojasnili, kaj je dokaz, podajmo pravila sklepanja za
$\top$ in $\land$, ki jih bomo v naslednjem razdelku še enkrat bolj
pozorno obravnavali:
%
\begin{mathpar}
  \inferrule{\quad}{\top}
  %
  \and
  %
  \inferrule
  {\phi \\ \psi}
  {\phi \land \psi}
  %
  \and
  %
  \inferrule
  {\phi \land \psi}
  {\phi}
  %
  \and
  %
  \inferrule
  {\phi \land \psi}
  {\psi}  
\end{mathpar}
%
Po vrsti beremo:
%
\begin{itemize}
\item Velja $\top$.
\item Če velja $\phi$ in $\psi$, smemo sklepati $\phi \land \psi$.
\item Če velja $\phi \land \psi$, smemo sklepati $\phi$.
\item Če velja $\phi \land \psi$, smemo sklepati $\psi$.
\end{itemize}
%
Formalni dokaz ima drevesno obliko in prikazuje, kako iz danih
\textbf{hipotez} dokažemo neko \textbf{sodbo}. Pri dnu je zapisana izjava,
ki jo dokazujemo, nad njo pa dokaz. Vsako vejišče je eno od pravil
sklepanja. Vsaka veja se mora zaključiti z aksiomom ali s hipotezo.
Oglejmo si dokaz izjave $(\alpha \land \alpha) \land (\top
\land \beta)$ iz hipoteze $\beta \land \alpha$:
%
\begin{equation*}
  \inferrule{
    \inferrule{
      \inferrule{\beta \land \alpha}{\alpha}
      \\
      \inferrule{\beta \land \alpha}{\alpha}}
      {\alpha \land \alpha}
    \\
    \inferrule{
      \inferrule{ }{\top}
      \\
      \inferrule{\beta \land \alpha}{\beta}
    }{\top \land \beta}
  }{(\alpha \land \alpha) \land (\top \land \beta)}
\end{equation*}
%
Dokaz se razveji na dve veji, vsaka od njiju pa še na dve veji. Tako
pri vrhu dobimo štiri liste, od katerih se trije izjava $\beta \land
\alpha$ in en aksiom za $\top$.

\begin{vaja}
  Preveri, da je vsako vejišče v zgornjem dokazu res uporaba enega od
  zgoraj podanih pravil sklepanja.
\end{vaja}

V praksi matematično besedilo bolj ali manj odraža strukturo
formalnega dokaza, le da se besedilo ne veji, ampak so sestavni kosi
dokaza zloženi v zaporedje. Formalni dokazi so uporabni, kadar želimo
preveriti veljavnost najbolj osnovnih logičnih dejstev. Ni mišljeno,
da bi matematiki pisali ali preverjali velike formalne dokaze
pomembnih matematičnih izrekov, to je delo za račualnike. Formalna
pravila sklepanja in formalni dokazi so za matematike pomembni, ker
nam omogočajo, da natančno in v celoti povemo, kakšna so ">pravila
igre"< v matematiki.


\section{Izjavni račun}
\label{sec:izjavni-racun}

Izjavni račun je tisti del logike, ki govori o logičnih konstantah
$\bot$, $\top$ in o logičnih operacijah $\land$, $\lor$, $\lthen$,
$\liff$, $\lnot$. Za vsako od njih podamo formalna pravila sklepanja,
ki so dveh vrst. Pravila \textbf{vpeljave} povedo, kako se izjave
dokaže, pravila \textbf{uporabe} pa povedo, kako se že dokazane izjave uporabi.

\subsection{Konjunkcija}
\label{sec:konjunkcija}

Konjunkcija ima eno pravilo vpeljave in dve pravili uporabe:
%
\begin{mathpar}
  \inferrule
  {\phi \\ \psi}
  {\phi \land \psi}
  \and
  \inferrule
  {\phi \land \psi}
  {\phi}  
  %
  \and
  %
  \inferrule
  {\phi \land \psi}
  {\psi}
\end{mathpar}
%
Pravilo vpeljave pove, da konjunkcijo $\phi \land \psi$ dokažemo
tako, da dokažemo posebej $\phi$ in posebej $\psi$. Pravili uporabe pa
povesta, da lahko $\phi \land \psi$ ">razstavimo"< na izjavi~$\phi$
in~$\psi$.

V matematičnem besedilu je dokaz konjunkcije $\phi \land \psi$ zapisan
kot zaporedje dveh pod-dokazov:
%
\begin{quote}
  \it 
  %
  Dokazujemo $\phi \land \psi$:
  \begin{enumerate}
  \item (Dokaz $\phi$)
  \item (Dokaz $\psi$)
  \end{enumerate}
  Dokazali smo $\phi \land \psi$.
\end{quote}
%
Manj podroben dokaz ne vsebuje začetnega in končnega stavka, ampak
samo dokaza za $\phi$ in $\psi$. Bralec mora sam ugotoviti, da je s
tem dokazana izjava $\phi \land \psi$.

\subsection{Implikacija}
\label{sec:implikacija}

Preden zapišemo pravila sklepanja za implikacijo, si oglejmo primer
neformalnega dokaza.

\begin{izrek}
  Če je $x > 2$, potem je $x^3 + x + 7 > 16$.
\end{izrek}

\begin{proof}
  Predpostavimo, da velja $x > 2$. Tedaj je $x^3 > 2^3 = 8$, zato
  velja
  %
  \begin{equation*}
    x^3 + x + 7 > 8 + 2 + 7 = 17 > 16.
  \end{equation*}
  %
  Dokazali smo $x > 2 \lthen x^3 + x + 7 > 16$.
\end{proof}

\noindent
%
Prvi stavek dokaza z besedico ">predpostavimo"< uvede \textbf{začasno
  hipotezo} $x > 2$, iz katere nato izpeljemo posledico $x^3 + x + 7 >
16$. Implikacijo $\phi \lthen \psi$ torej dokažemo tako, da začasno
predpostavimo $\phi$ in dokažemo $\psi$. Tako pravilo vpeljave
zapišemo
%
\begin{equation*}
  \inferrule{\infer*{\psi}{[\phi]}}{\phi \lthen \psi}  
\end{equation*}
%
Zapis $[\phi]$ z oglatimi oklepaji pomeni, da $\phi$ ni prava, ampak
samo začasna hipoteza. Zapis
%
\begin{equation*}
  \infer*{\psi}{[\phi]}
\end{equation*}
%
pomeni ">dokaz izjave $\phi$ s pomočjo začasne hipoteze $\phi$."<

Pravilo uporabe za implikacijo se imenuje \textbf{modus ponens} in se
glasi
%
\begin{mathpar}
  \inferrule{\phi \lthen \psi \\ \phi}{\psi}
\end{mathpar}
%
V matematičnem besedilu se modus ponens pojavi kot uporaba že prej
dokazanega izreka izreka oblike $\phi \lthen \psi$.

\subsection{Disjunkcija}
\label{sec:disjunkcija}

Disjunkcija ima dve pravili vpeljave in eno pravilo uporabe:
%
\begin{mathpar}
  \inferrule
  {\phi}
  {\phi \lor \psi}
  \and
  \inferrule
  {\psi}
  {\phi \lor \psi}
  \and
  \inferrule
  {\phi \lor \psi \\ \infer*{\rho}{[\phi]} \\ \infer*{\rho}{[\psi]}}
  {\rho}
\end{mathpar}
%
Pravili sklepanja povesta, da lahko dokažemo disjunkcijo $\phi \lor
\psi$ tako, da dokažemo enega od disjunktov.

Pojasnimo še pravilo uporabe. Denimo, da bi radi dokazali $\rho$, pri
čemer že vemo, da velja $\phi \lor \psi$. Pravilo uporabe pravi, da je
treba obravnavati dva primera: iz začasne hipoteze $\phi$ je treba
dokazati $\rho$ in iz začasne hipoteze $\psi$ je treba dokazati
$\rho$.

Ponazorimo pravilo uporabe v dokazu izjave $(\alpha \lor \gamma) \land
(\beta \lor \gamma)$ iz hipoteze $(\alpha \land \beta) \lor \gamma$.
Dokazno drevo je precej veliko, v njem pa se dvakrat pojavi uporaba
disjunkcije:
%
\begin{equation*}
  \inferrule
  {\inferrule*
    {\inferrule*{}{(\alpha \land \beta) \lor \gamma}
      \\
      \inferrule*{
        \inferrule
        {[\alpha \land \beta]}
        {\alpha}
      }{\alpha \lor \gamma}
      \\
      \inferrule*{[\gamma]}{\alpha \lor \gamma}
    }
    {\alpha \lor \gamma}
    \\
    \inferrule*
    {\inferrule*{}{(\alpha \land \beta) \lor \gamma}
      \\
      \inferrule*{
        \inferrule
        {[\alpha \land \beta]}
        {\beta}
      }{\beta \lor \gamma}
      \\
      \inferrule*{[\gamma]}{\beta \lor \gamma}
    }
    {\beta \lor \gamma}
  }
  {(\alpha \lor \gamma) \land (\beta \lor \gamma)}
\end{equation*}
%
Poglejmo na primer levo vejo tega dokaza, desna je podobna:
%
\begin{equation*}
  \inferrule*
    {\inferrule*{}{(\alpha \land \beta) \lor \gamma}
      \\
      \inferrule*{
        \inferrule
        {[\alpha \land \beta]}
        {\alpha}
      }{\alpha \lor \gamma}
      \\
      \inferrule*{[\gamma]}{\alpha \lor \gamma}
    }
    {\alpha \lor \gamma}
\end{equation*}
%
Res je to uporaba disjunkcije $\phi \lor \psi$, kjer smo vzeli $\phi =
\alpha \land \beta$ in $\psi = \gamma$, dokazali pa smo izjavo $\rho =
\alpha \lor \gamma$.

\begin{vaja}
  Iz hipoteze $(\alpha \lor \gamma) \land (\beta \lor \gamma)$ dokaži
  $(\alpha \land \beta) \lor \gamma$.
\end{vaja}

V besedilu dokažemo disjunkcijo s pravilom za vpeljavo takole:
%
\begin{quote}
  \it
  %
  Dokazujemo $\phi \lor \psi$. Zadostuje dokazati $\phi$:
  \begin{enumerate}
  \item[] (Dokaz $\phi$.)
  \end{enumerate}
  %
  Dokazali smo $\phi \lor \psi$.
\end{quote}
%
Pravilo uporabe disjunkcije se v besedilu zapiše kot obravnava
primerov:
%
\begin{quote}
  \it
  %
  Dokazujemo $\rho$. To bomo dokazali z obravnavo primerov $\phi$ in
  $\psi$:
  \begin{enumerate}
  \item (Dokaz $\phi \lor \rho$)
  \item Predpostavimo, da velja $\phi$. (Dokaz $\rho$)
  \item Predpostavimo, da velja $\psi$. (Dokaz $\rho$)
  \end{enumerate}
  %
  Dokazali smo $\rho$.
\end{quote}
%
Še primer konkretnega dokaza, ki je tako napisan.

\begin{izrek}
  \label{izrek:x-3-5}
  Naj bo $x$ realno število. Če je $|x - 3| > 5$, potem je $x^4 > 15$.
\end{izrek}

\begin{proof}
  Dokazujemo $|x - 3| > 5 \lthen x^4 > 15$. Predostavimo $|x - 3| > 5$
  in dokažimo $x^4 > 15$. To bomo dokazali z obravavo primerov $x \leq
  3$ in $x \geq 3$:
  %
  \begin{enumerate}
  \item $x \leq 3 \lor x \geq 3$ velja, ker so realna števila linearno
    urejena z relacijo $\leq$.
  \item Predpostavimo $x \leq 3$. Tedaj je $x - 3 \leq 0$ in zato $|x
    - 3| = 3 - x$, od koder sledi $3 - x = |x - 3| > 5$, oziroma $x <
    -2$. Tako dobimo
    %
    \begin{equation*}
      x^4 > (-2)^4 = 16 > 15.
    \end{equation*}
  \item Predpostavimo $x \geq 3$. Tedaj je $x - 3 \geq 0$ in zato$|x -
    3| = x - 3$, od koder sledi $x - 3 = |x - 3| > 5$, oziroma $x >
    8$. Tako dobimo
    %
    \begin{equation*}
      x^4 > 8^4 = 4096 > 15.
    \end{equation*}
  \end{enumerate}
  %
  Iz predpostavke $|x - 3| > 5$ smo izpeljali $x^4 > 15$. S tem smo
  dokazali $|x - 3| > 5 \lthen x^4 > 15$.
\end{proof}

Težji del tega dokaza se skriva v izbiri disjunkcije. Kako je pisec
uganil, da je treba obravnavati primera $x \leq 3$ in $x \geq 3$?
Zakaj ni raje obravnaval $x < 3$ in $x \geq 3$, ali morda $x \leq 17$
in $x \geq 17$? Odgovor se skriva v definiciji absolutne vrednosti:
%
\begin{equation*}
  |a| =
  \begin{cases}
    a & \text{če je $a \geq 0$,}\\
    -a & \text{če je $a \leq 0$.}
  \end{cases}
\end{equation*}
%
Ker v izreku nastopa izraz $|x - 3|$, bo obravnava primerov $x - 3
\geq 0$ in $x - 3 \leq 0$ omogočila, da $|x - 3|$ poenostavimo enkrat
v $x - 3$ in drugič v $3 - x$. Seveda pa je $x - 3 \geq 0$
ekvivalentno $x \geq 3$ in $x - 3 \leq 0$ ekvivalentno $x \leq 3$.

\begin{vaja}
  Ali bi lahko izrek~\ref{izrek:x-3-5} dokazali tudi z obravnavo
  primerov $x < 3$ in $x \geq 3$?
\end{vaja}

\subsection{Resnica in neresnica}
\label{sec:resnica-neresnica}

Logična konstanta $\top$ označuje resnico. Kar je res, je res, in tega
ni treba posebej dokazovati. To dejstvo izraža aksiom
%
\begin{equation*}
  \inferrule{\qquad}{\top}
\end{equation*}
%
Logična konstanta $\top$ nima pravila uporabe, ker iz $\top$ ne moremo
sklepati nič koristnega.

Logična konstanta $\bot$ označuje neresnico. Ker se tega, kar ni res,
ne more dokazati, $\bot$ nima pravila vpeljave. Pravilo uporabe je
%
\begin{equation*}
  \inferrule{\quad\bot\quad}{\phi}
\end{equation*}
%
se imenuje \textbf{ex falso (sequitur) quodlibet}, kar pomeni ">iz
neresnice sledi karkoli"<.

V matematičnem besedilu se $\top$ in $\bot$ ne pojavljata pogosto, ker
matematiki izraze, v katerih se $\top$ in $\bot$ pojavita, vedno
poenostavijo s pomočjo ekvivalenc:
%
\begin{mathpar}
  \top \land \phi \liff \phi
  \and
  \top \lor \phi \liff \phi
  \and
  \bot \land \phi \liff \bot
  \and
  \bot \lor \phi \liff \phi
  \\
  (\top \lthen \phi) \liff \phi
  \and
  (\bot \lthen \phi) \liff \top
  \and
  (\phi \lthen \top) \liff \top
\end{mathpar}
%

\subsection{Ekvivalenca}
\label{sec:ekvivalenca}

Logična ekvivalenca $\phi \liff \psi$ je okrajšava za
%
\begin{equation*}
  (\phi \lthen \psi) \land (\psi \lthen \phi).
\end{equation*}
%
Ker je to konjunkcija (dveh implikacij), so pravila za vpeljavo in
uporabo ekvivalence samo poseben primer pravil sklepanja za
konjunkcijo:
%
\begin{mathpar}
  \inferrule
  {\phi \lthen \psi \\ \psi \lthen \phi}
  {\phi \liff \psi}
  \and
  \inferrule{\phi \liff \psi}{\phi \lthen \psi}
  \and
  \inferrule{\phi \liff \psi}{\psi \lthen \phi}
\end{mathpar}
%
V matematičnem besedilu ekvivalenco dokažemo takole:
%
\begin{quote}
  \it
  %
  Dokazujemo $\phi \liff \psi$:
  %
  \begin{enumerate}
  \item (Dokaz $\phi \lthen \psi$)
  \item (Dokaz $\psi \lthen \phi$)
  \end{enumerate}
  Dokazali smo $\phi \liff \psi$.
\end{quote}

Če sta izjavi $\phi$ in $\psi$ logično ekvivalentni, lahko eno
zamenjamo z drugo. To matematiki s pridom uporabljajo pri dokazovanju
izjav, čeprav pogosto sploh ne omenijo, katero ekvivalenco so
uporabili.

Kadar dokazujemo medsebojno ekvivalenco večih izjav $\phi_1$,
$\phi_2$, \ldots, $\phi_n$, zadostuje dokazati cikel implikacij
%
\begin{equation*}
  \phi_1 \lthen \phi_2 \lthen \cdots \lthen \phi_{n-1} \lthen \phi_n \lthen \phi_1.
\end{equation*}
%
(Ne spreglejte zadnje implikacije $\phi_n \lthen \phi_1$, ki zaključi
cikel). V besedilu to dokažemo:

\begin{quote}
  \it
  %
  Dokazujemo, da so izjave $\phi_1, \phi_2, \ldots, \phi_n$
  ekvivalentne:
  %
  \begin{enumerate}
  \item (Dokaz $\phi_1 \lthen \phi_2$)
  \item (Dokaz $\phi_2 \lthen \phi_3$)
  \item \dots
  \item (Dokaz $\phi_{n-1} \lthen \phi_n$)
  \item (Dokaz $\phi_n \lthen \phi_1$)
  \end{enumerate}
\end{quote}

\noindent
%
Seveda smemo pred samim dokazovanjem izjave $\phi_1, \ldots, \phi_n$
preurediti tako, da je zahtevane implikacije kar najlažje dokazati.
Dokaz lahko tudi razdelimo na dva ločena cikla implikacij
%
\begin{equation*}
  \phi_1 \lthen \cdots \lthen \phi_k \lthen \phi_1
\end{equation*}
%
in
%
\begin{equation*}
  \phi_{k+1} \lthen \cdots \lthen \phi_n \lthen \phi_{k+1}
\end{equation*}
%
in nato dokažemo še eno ekvivalenco $\phi_i \liff \phi_j$, pri čemer
je $\phi_i$ iz prvega in $\phi_j$ iz drugega cikla.

\subsection{Negacija}
\label{sec:negacija}


Negacija $\lnot\phi$ je definirana kot okrajšava za $\phi
\lthen \bot$. Iz pravil sklepanja za $\lthen$ in $\bot$ tako izpeljemo
pravili sklepanja za negacijo:
%
\begin{mathpar}
  \inferrule
  {\infer*{\bot}{[\phi]}}
  {\lnot \phi}
  %
  \and
  %
  \inferrule
  {\lnot\phi \\ \phi}
  {\psi}
\end{mathpar}
%
V besedilu dokazujemo $\lnot\phi$ takole:
%
\begin{quote}
  \it
  %
  Dokazujemo $\lnot\phi$.
  \begin{itemize}
  \item[] Predpostavimo $\phi$.
  \item[] (Dokaz $\bot$.)
  \end{itemize}
  Dokazali smo $\lnot\phi$.
\end{quote}
%
Tu ">Dokaz $\bot$"< pomeni, da iz danih predpostavk izpeljemo
protislovje. Mnogi matematiki menijo, da se takemu dokazu reče ">dokaz
s protislovjem"<, vendar to ni res. To je samo navaden dokaz negacije.
Dokazovanje s protislovjem bomo obravnavali v razdelku~\ref{sec:lem}.

Pravilo uporabe za $\lnot\phi$ v besedilu ni eksplicitno vidno, ampak
ga matematiki uporabijo, ko sredi dokaza, da velja $\psi$, izpeljejo
protislovje:
%
\begin{quote}
  \it
  %
  Dokazujemo $\psi$.
  %
  \begin{itemize}
  \item[] (Dokaz $\phi$.)
  \item[] (Dokaz $\lnot\phi$.)
  \end{itemize}
  %
  To je nesmisel, in ker iz nesmisla sledi karkoli, sledi $\psi$.
\end{quote}

\subsection{Aksiom o izključenem tretjem}
\label{sec:lem}

Aksiom o izključenem tretjem se glasi
%
\begin{equation*}
  \inferrule{ }{\phi \lor \lnot \phi}
\end{equation*}
%
Povedano z besedami, vsaka izjava je bodisi resnična bodisi
neresnična. Torej ni ">tretje možnosti"< za resničnostno vrednost
izjave $\phi$, od koder izhaja tudi ime aksioma.

Aksiom o izključenem tretjem omogoča \emph{posredne} dokaze izjav, od
katerih je najbolj znano \textbf{dokazovanje s protislovjem}: pri tem ne
utemeljimo izjave $\phi$, ampak utemeljimo, zakaj $\lnot\phi$
\emph{ne} velja. Natančneje povedano, izjavo $\phi$ zamenjamo z njej
ekvivalentno izjavo $\lnot\lnot\phi$ in dokažemo $\lnot\lnot\phi$.
Dokaz ekvivalence $\phi \liff \lnot\lnot\phi$ sestoji iz dokazov dveh
implikacij:
%
\begin{mathpar}
  \inferrule
  {\inferrule{
      \inferrule{[\lnot\phi] \\ [\phi]}{\bot}
    }
    {\lnot\lnot\phi}
  }
  {\phi \lthen \lnot\lnot\phi}
  %
  \and
  %
  \inferrule*
  {\inferrule*
    {\inferrule*{ }{\phi \lor \lnot\phi} \\
     [\phi] \\
     \inferrule*{
       \inferrule*{
         [\lnot\lnot\phi] \\ [\lnot\phi]
       }
       {\bot}
     }
     {\phi}
    }
    {\phi}
  }
  {\lnot\lnot\phi \lthen \phi}
\end{mathpar}
%
V dokazu $\lnot\lnot\phi \lthen \phi$ smo uporabili aksiom o
izključenem tretjem. V matematičnem besedilu se dokaz s protislovjem
glasi:
%
\begin{quote}
  \it
  %
  Dokažimo $\phi$ s protislovjem.
  %
  \begin{itemize}
  \item[] Predpostavimo, da bi veljalo $\lnot\phi$.
  \item[] (Dokaz neresnice $\bot$.)
  \end{itemize}
  %
  Ker torej $\lnot\phi$ pripelje do protislovja, velja $\phi$.
\end{quote}
%
Praviloma izvemo o vsebini matematične izjave~$\phi$ več, če jo
dokažemo neposredno. Dokazovanja s protislovjem zato ni smiselno
uporabljati vsepovprek, ampak le takrat, ko je zares potreben ali ko
nam zelo olajša dokazovanje.

Ostali načini za sestavljanje posrednih dokazov slonijo na
ekvivalencah
%
\begin{mathpar}
  (\phi \lor \psi) \liff \lnot (\lnot\phi \land \lnot\psi),\and
  (\phi \lor \psi) \liff (\lnot\phi \lthen \psi),\and
  (\phi \lthen \psi) \liff (\lnot\psi \lthen \lnot\phi),\and
  (\all{x \in S} \phi) \liff \lnot \some{x \in S} \lnot \phi,\and
  (\some{x \in S} \phi) \liff \lnot \all{x \in S} \lnot \phi.
\end{mathpar}
%
V vseh petih primerih implikacija $\lthen$ iz leve na desno velja brez
uporabe aksioma o izključenem tretjem. Za dokaz implikacij
$\Leftarrow$ iz desne na levo pa potrebujemo aksiom o izključenem
tretjem.

\begin{vaja}
  Sestavi formalne dokaze za zgornjih pet ekvivalenc. Pri dokazovanju
  ekvivalenc za $\forall$ in $\exists$ si pomagaj s pravili sklepanja
  iz razdelkov~\ref{sec:univerzalni-kvantifikator}
  in~\ref{sec:eksistencni-kvantifikator}.
\end{vaja}

Povejmo, kako zgornje ekvivalence uporabimo v besedilu za posredni
dokaz izjave:
%
\begin{itemize}
\item $(\phi \lor \psi) \liff \lnot (\lnot\phi \land \lnot\psi)$
  uporabimo takole:
  %
  \begin{quote}
    \it
    %
    Dokazujemo $\phi \lor \psi$.
    %
    \begin{itemize}
    \item[] Predpostavimo, da velja $\lnot\phi$ in $\lnot\psi$.
    \item[] (Dokaz neresnice $\bot$.)
    \end{itemize}
    %
    Ker torej nista $\phi$ in $\psi$ oba neresnična, je eden od njiju
    resničen. Dokazali smo $\phi \lor \psi$.
  \end{quote}
\item $(\phi \lor \psi) \liff (\lnot\phi \lthen \psi)$ uporabimo
  takole:
  %
  \begin{quote}
    \it
    %
    Dokazujemo $\phi \lor \psi$.
    %
    \begin{itemize}
    \item[] Predpostavimo $\lnot\phi$.
    \item[] (Dokaz $\psi$.)
    \end{itemize}
    %
    Če torej ne velja $\lnot\phi$, velja $\psi$. Torej velja vsaj
    eden, zato smo dokazali $\phi \lor \psi$.
  \end{quote}
\item $(\phi \lthen \psi) \liff (\lnot\psi \lthen \lnot\phi)$
  uporabimo takole:
  %
  \begin{quote}
    \it
    %
    Dokazujemo $\phi \lthen \psi$.
    %
    \begin{enumerate}
    \item Predpostavimo $\lnot\psi$.
    \item (Dokaz $\lnot\psi$.)
    \end{enumerate}
    %
    Dokazali smo, da iz $\phi$ sledi $\psi$.
  \end{quote}
\item $(\all{x \in S} \phi) \liff \lnot \some{x \in S} \lnot \phi$
  uporabimo takole:
  %
  \begin{quote}
    \it
    %
    Dokazujemo, da za vsak $x \in S$ velja $\phi$.
    %
    \begin{enumerate}
    \item Predpostavimo, da obstaja $x \in S$, za katerega $\phi$
      \emph{ne} velja.
    \item (Dokaz neresnice $\bot$.)
    \end{enumerate}
    %
    Predpostavka, da obstaja $x \in S$, za katerega $\phi$ ne velja,
    pripelje do protislovja. Torej za vsak $x \in S$ velja $\phi$.
  \end{quote}
\item $(\some{x \in S} \phi) \liff \lnot \all{x \in S} \lnot \phi$
  uporabimo takole:
  %
  \begin{quote}
    \it
    %
    Dokazujemo, da obstaja tak $x \in S$, za katerega velja $\phi$.
    %
    \begin{enumerate}
    \item Predpostavimo, da bi veljalo $\lnot\phi$ za vse $x \in S$.
    \item (Dokaz neresnice $\bot$.)
    \end{enumerate}
    %
    Predpostavka, da velja $\lnot\phi$ za vse $x \in S$, pripelje do
    protislovja. Torej obstaja $x \in S$, za katerega velja $\phi$.
  \end{quote}
\end{itemize}

Negacijo poljubne izjave $\phi$ tvorimo preprosto tako, da pred njo
postavimo $\lnot$. Vendar nam to ne pove dosti o matematični vsebini
negirane izjave. V večini primerov je negacijo lažje razumeti, če
simbol~$\lnot$ ">porinemo"< navznoter do osnovnih izjav z uporabo
naslednjih ekvivalenc:
%
\begin{align*}
  \lnot (\phi \land \psi) &\iff \lnot\phi \lor \lnot\psi \\
  \lnot (\phi \lor \psi) &\iff \lnot\phi \land \lnot\psi \\
  \lnot (\phi \lthen \psi) &\iff \phi \land \lnot\psi \\
  \lnot (\lnot \phi) &\iff \phi \\
  \lnot (\all{x \in S} \phi) &\iff \some{x \in S} \lnot\phi \\
  \lnot (\some{x \in S} \phi) &\iff \all{x \in S} \lnot\phi
\end{align*}

\begin{zgled}
  Denimo, da bi radi ovrgli izjavo
  % 
  \begin{quote}
    ">Vsako zaporedje pozitivnih realnih števil ima limito~$0$."<
  \end{quote}
  % 
  Da izjavo ovržemo, moramo dokazati njeno negacijo. Načeloma lahko
  negacijo tvorimo tako, da pred izjavo napišemo ">ni res, da velja
  \dots"<, a nam to ne pove, kako bi negacijo dokazali. Zapišimo
  prvotno izjavo v delni simbolni obliki:
  % 
  \begin{equation}
    \label{eq:pozitivno-limita-0}
    \all{a \in \RR^\NN}{\text{$(a_n)_n$ pozitivno zaporedje}
      \lthen \text{$0$ je limita zaporedja $(a_n)_n$}}.
  \end{equation}
  % 
  Zgornja pravila za računanje negacije nam povedo, da se
  $\lnot\forall$ spremeni v $\exists\lnot$ in da se nato implikacija
  oblike $\phi \lthen \psi$ spremeni v $\phi \land \lnot\psi$. Tako
  izrazimo negacijo izjave~\eqref{eq:pozitivno-limita-0}:
  % 
  \begin{equation*}
    \some{a \in \RR^\NN}{\text{$(a_n)_n$ pozitivno zaporedje}
      \land \lnot (\text{$0$ je limita zaporedja $(a_n)_n$})}.
  \end{equation*}
  % 
  To preberemo z besedami:
  % 
  \begin{quote}
    ">Obstaja tako zaporedje $(a_n)_n$, da je $(a_n)_n$ zaporedje
    pozitivnih števil in da $0$ ni limita zaporedja $(a_n)_n$."<
  \end{quote}
  %
  Če se še malo potrudimo, preberemo bolj razumljivo:
  % 
  \begin{quote}
    ">Obstaja tako zaporedje pozitivnih realnih števil, da $0$ ni
    njegova limita."<
  \end{quote}
  %
  S tem še nismo končali, saj je tudi ">Število $0$ ni limita
  zaporedja $(a_n)_n$"< negacija. Izjavo ">$0$ je limita zaporedja
  $(a_n)_n$"< najprej zapišemo simbolno:
  % 
  \begin{equation}
    \label{eq:limita-0}
    \all{\epsilon > 0}
      \some{m}{\NN}
        \all{n \geq m}
          |a_n - 0| < \epsilon.
  \end{equation}
  % 
  Z zgornjimi pravili za negiranje izračunamo negacijo
  izjave~\eqref{eq:limita-0}. Operacijo $\lnot$ postopoma ">porivamo"<
  navznoter:
  % 
  % 
  \begin{align*}
    \lnot \all{\epsilon > 0} \some{m \in \NN} \all{n \geq m} |a_n
          - 0| < \epsilon & \iff
    \\
    \some{\epsilon > 0} \lnot \some{m}{\NN} \all{n \geq m}
          |a_n - 0| < \epsilon &\iff
    \\
    \some{\epsilon > 0} \all{m}{\NN} \lnot \all{n \geq m}
          |a_n - 0| < \epsilon &\iff
    \\
    \some{\epsilon > 0} \all{m}{\NN} \some{n \geq m}
          \lnot (|a_n - 0| < \epsilon) &\iff
    \\
    \some{\epsilon > 0} \all{m}{\NN} \some{n \geq m}
          |a_n - 0| \geq \epsilon &\iff
    \\
    \some{\epsilon > 0} \all{m}{\NN} \some{n \geq m}
          a_n \geq \epsilon.
  \end{align*}
  % 
  V zadnjem koraku smo upoštevali, da za pozitivno število $a_n$ velja
  $|a_n - 0| = |a_n| = a_n$. Tako smo dobili podrobno zapisano
  negacijo prvotne izjave
  % 
  \begin{quote}
    ">Obstaja tako zaporedje pozitivnih števil $(a_n)_n$ in obstaja
    tak $\epsilon > 0$, da za vsak $m \in \NN$ obstaja $n \geq m$, za
    katerega velja $a_n > \epsilon$."<
  \end{quote}
  % 
  To izjavo pa znamo dokazati tako, da podamo konkreten primer
  zaporedja $(a_n)_n$ in konkretno vrednost $\epsilon$, ki zadoščata
  pogoju, denimo $a_n = 2 + n$ in $\epsilon = 1$. Res, če je $m \in
  \NN$ poljuben, lahko vzamemo kar $n = m$, saj potem velja $a_n = a_m
  = 2 + m > 1 = \epsilon$.

  Pričujoči primer smo zapisali zelo podrobno. Izkušeni matematik tega
  seveda ne bo pisal, saj bo izračunal negacijo prvotne izjave kar v
  glavi in takoj podal primer zaporedja, ki dokazuje, da prvotna
  izjava ne velja.
\end{zgled}

%%%%%%%%%%%%%%%%%%%%%%%%%%%%%%%%%%%%%%%%%%%%%%%%%%%%%%%%%%%%%%%%%%%%%%
\section{Predikatni račun}
\label{sec:predikatni-racun}

Predikatni račun je tisti del logike, ki obravnava predikate ter
kvantifikatorja~$\forall$ in~$\exists$.

Predikate tvorimo z logičnimi operacijami in kvantifikatorji iz
\textbf{osnovnih predikatov}. Katere osnovne predikate imamo na voljo,
je odvisno od snovi, ki jo obravnavamo.\footnote{Na primer, če
  obravnavamo ravninsko geometrijo, potem so osnovni predikati ">točka
  $x$ leži na premici $y$"<, ">premici $p$ in $q$ se sekata"< itn.}
Vedno imamo na voljo tudi \textbf{enakost} $x = y$, ki jo bomo
obravnavali v razdelku~\ref{sec:enakost}.

V osnovnih predikatih nastopajo \textbf{izrazi} ali \textbf{termi}. Katere
izraze lahko tvorimo je spet odvisno od tega, katere konstante in
operacije imamo na voljo. Na primer, če obravnavamo aritmetiko celih
števil, so na voljo operacije $+$, $-$, $\times$, če pa obravnavamo
realna števila, so na voljo operacije $+$, $-$, $\times$, $/$. V
izrazih vedno lahko nastopajo \textbf{spremenljivke}. Kadar uporabimo
spremenljivko, moramo povedati njen \textbf{tip} oziroma \textbf{množico}
vrednosti, ki jih lahko zavzame spremenljivka. Pogosto je tip
spremenljivke razviden iz spremnega besedila ali iz ustaljene uporabe:
$n$ se uporablja za naravno število, $x$ za realno, $f$ za funkcijo
ipd.

Ponazorimo pravkar definirane pojme s primerom. Predikat
%
\begin{equation*}
  0 < f(x) \land f(x) < \pi/4 \lthen \sin(2 f(x)) = 1/3
\end{equation*}
%
je sestavljen s pomočjo logičnih operacij $\land$ in $\lthen$ iz treh
osnovnih predikatov, zgrajenih iz osnovnih relacij $<$ in $=$,
%
\begin{mathpar}
  0 < f(x)
  \and
  f(x) < \pi/4
  \and
  \sin(2 f(x)) = 1/3,
\end{mathpar}
%
v katerih nastopa pet izrazov:
%
\begin{mathpar}
  0
  \and
  f(x)
  \and
  \pi/4
  \and
  \sin(2 f(x))
  \and
  1/3
\end{mathpar}
%
V teh izrazih nastopa spremenljivka $x$, katere tip je množica
realnih števil (to moramo uganiti) in spremenljivka $f$, ki označuje
funkcijo iz realnih v realna števila (tudi to moramo uganiti).
Nadalje, v izrazih nastopajo konstante $0$, $\pi$, $4$, $2$,
$1$ in $3$, operacija $\sin$ in operacija množenja.


%%%%%%%%%%%%%%%%%%%%%%%%%%%%%%%%%%%%%%%%%%%%%%%%%%
\subsection{Proste in vezane spremenljivke}
\label{sec:spremenljivke}

V predikatih in izrazih se pojavljajo spremenljivke. Pri tem moramo
ločiti med \textbf{prostimi} in \textbf{vezanimi} spremenljivkami. Oglejmo
si naslednja izraza in predikat:
%
\begin{equation*}
  \sum_{i=0}^{n} a_i,
  \qquad
  \int_0^1 f(t) \, d t,
  \qquad
  \forall x \in A .\, \phi(x) \;.
\end{equation*}
%
V vsoti je vezana spremenljivka $i$, spremenljivki $n$ in $a$ sta
prosti. To pomeni, da je $i$ neke vrste ">lokalna
spremenljivka"<,\footnote{Podobnost z lokalnimi spremenljivkami v
  programskih jezikih ni zgolj naključje. Lokalna spremenljivka in
  števec v zanki sta tudi primera vezanih spremenljivk v teoriji
  programskih jezikov.} katere veljavnost je samo znotraj vsote,
medtem ko sta spremenljiki $n$ in $a$ veljavni tudi zunaj samega
izraza. Podobno je v integralu $t$ vezana spremenljivka in $f$ prosta,
v izjavi na desni pa je vezana spremenljivka $x$, spremenljivki $A$ in
$\phi$ sta prosti.

Vezane spremenljivke so ">nevidne"< zunaj izraza in jih lahko vedno
preimenujemo, ne da bi spremenili pomen izraza (seveda se novo ime ne
sme mešati z ostalimi spremenljivkami, ki nastopajo v izrazu): izraza
$\int_0^1 f(t)\, d t$ in $\int_0^1 f(x)\, d x$ štejemo za
\emph{enaka}, ker se razlikujeta le v imenu vezane spremenljivke.
Spremenljivki, ki ni vezana, pravimo \textbf{prosta}. Izrazu, v katerem
ni prostih spremenljivk, pravimo \textbf{zaprt izraz}. Zaprta
logična izjava se imenuje \textbf{stavek}.

Pomembno se je zavedati, da vezana spremenljivka ">zunaj"< svojega
območja ne obstaja. Matematiki so glede tega precej površni in na
primer pišejo
%
\begin{equation*}
  \int x^2 \, d x = x^3/3 + C,
\end{equation*}
%
kar je strogo gledano nesmisel. Na levi strani v integralu stoji
vezana spremenljivka~$x$, ki je na desni ">pobegnila"< iz integrala.
Še več, če je $x \in \RR$ in $C \in \RR$, potem je izraz $x^3/3 + C$
\emph{število} (odvisno od vrednosti $x$ in $C$), saj je vsota dveh
realnih števil. Na desni strani bi morala stati oznaka za
\emph{funkcijo}, recimo
%
\begin{equation*}
  \int x^2 \, d x = (x \mapsto x^3/3 + C),
\end{equation*}
%
vendar tega v praksi nihče ne piše. Seveda pri vsem tem ostane še
vprašanje, kakšno vlogo ima v zgornjem izrazu~$C$. Pri analizi se
učimo, da je~$C$ ">poljubna konstanta"<. Poskusimo to razumeti
natančno s stališča logike. Besedico ">poljubno"< ponavadi razumemo
kot ">za vsak"<, vendar to ne gre, saj je
%
\begin{equation*}
  \all{C \in \RR} \int x^2 \, d x = (x \mapsto x^3/3 + C)
\end{equation*}
%
nesemisel. Če bi to bilo res, bi veljalo za $C = 1$ in za $C = 2$, od
koder bi dobili
%
\begin{equation*}
  (x \mapsto x^3/3 + 1) =
  \int x^2 \, d x =
  (x \mapsto x^3/3 + 2).
\end{equation*}
%
Potemtakem bi morali biti funkciji $(x \mapsto x^3/3 + 1)$ in $(x
\mapsto x^3/3 + 1)$ enaki, od koder sledi nesmisel $1 = 2$. Težave
nastopajo iz dejstva, da poskušamo nedoločeni integral razumeti kot
operacijo, ki slika funkcije v funkcije, kar ni. Nedoločeni integral
preslika funkcijo~$f$ v \emph{množico} vseh funkcij $F$, za katere
velja $F' = f$. Če bi to želeli zapisati zares pravilno, bi dobili
%
\begin{equation*}
  \int x^2 \, d x =
  \set{(x \mapsto x^3/3 + C) \such C \in \RR}.
\end{equation*}
%
Ali naj torej sklepamo, da so matematiki pravzaprav zelo površni pri
pisanju integralov? Da, s stališča formalne logike prav gotovo. Vendar
to ni nujno slabo: matematični zapis v praksi služi ljudem za
sporazumevanje in prav je, da si izberejo tak zapis, s katerim najbolj
učinkovito komunicirajo drug z drugim. Kljub temu pa se je treba
zavedati, kdaj gredo matematiki ">po bližnjici"< in ne zapišejo ali
povedo vsega dovolj natančno, da bi to bilo sprejemljivo za standarde,
ki jih postavlja formalna logika.


%%%%%%%%%%%%%%%%%%%%%%%%%%%%%%%%%%%%%%%%%%%%%%%%%%
\subsection{Substitucija}
\label{sec:substitucija}

\textbf{Substitucija} je osnovna sintaktična operacija, v kateri
\emph{proste} spremenljivke zamenjamo z izrazi. Zapis
%
\begin{equation*}
  \xsubst{e}{x_1 \subto e_1, \ldots, x_n \subto e_n}
\end{equation*}
%
pomeni: ">v izrazu $e$ \emph{hkrati} zamenjaj proste spremenljivke
$x_1$ z $e_1$, $x_2$ z $e_2$, \dots in $x_n$ z $e_n$."<  Na primer,
%
\begin{equation*}
  \subst{x^2 + y}{x \subto 3, y \subto 5, z \subto 12}
\end{equation*}
%
je enako $3^2 + 5$. Nič hudega ni, če se v substituciji omenja
spremenljivko $z$, ki se v izrazu $x^2 + y$ ne pojavi.

Ko naredimo substitucijo, moramo paziti, da se proste spremenljivke ne
">ujamejo"<. Denimo, da želimo v integralu
%
\begin{equation*}
  \int_0^1 \frac{x}{a - x^2} \; dx
\end{equation*}
%
parameter $a$ zamenjati z $y^2$. To naredimo s substitucijo
%
\begin{equation*}
  \xsubst{\left(\int_0^1 \frac{x}{a - x^2} \; dx\right)}{a \subto y^2} =
  \int_0^1 \frac{x}{y^2 - x^2} \; dx.
\end{equation*}
%
Vse lepo in prav. Kaj pa, če želimo $a$ zamenjati z $1 + x$? Ker je
spremenljivka $x$ vezana v integralu, \emph{ne smemo} delati takole:
%
\begin{equation*}
  \xsubst{\left(\int_0^1 \frac{x}{a - x^2} \; dx\right)}{a \subto x^2} =
  \int_0^1 \frac{x}{x^2 - x^2} \; dx ?!
\end{equation*}
%
Ker vstavljamo v integral spremenljivko $x$, moramo vezano
spremenljivko $x$ najprej preimenovati v kaj drugega, na primer $t$,
šele nato vstavimo:
%
\begin{equation*}
  \xsubst{\left(\int_0^1 \frac{x}{a - x^2}\; dt \right)}{a \subto x^2} =
  \xsubst{\left(\int_0^1 \frac{t}{a - t^2} \; dt\right)}{a \subto x^2} =
  \int_0^1 \frac{t}{x^2 - t^2} \; dt.
\end{equation*}
%
Podajmo še nekaj primerov substitucij:
%
\begin{align*}
  \subst{x + y + 1}{x \subto 2} &= 2 + y + 1 \;,
  \\
  \subst{x + y^2 + 1}{x \subto y, y \subto x} &= y + x^2 + 1 \;
  \\
  \subst{\subst{x + y^2 + 1}{x \subto y}}{y \subto x} &=
  x + x^2 + 1 \;,
  \\
  \textstyle
  \subst{x + \int_0^1 x \cdot y \;, d x}{x \subto 2}
  &= \textstyle  2 + \int_0^1 x \cdot y \;, d x \;,
  \\
  \textstyle
  \subst{\int_0^1 x \cdot y \; d x}{y \subto x^2}
  &= \textstyle \int_0^1 t \cdot x^2 \; d t \;.
\end{align*}
%
Ločiti je treba med hkratno in zaporedno substitucijo:
%
\begin{align*}
  \subst{x + y^2}{x \subto y, y \subto x} &= y + x^2
  \\
  \subst{\subst{x + y^2}{x \subto y}}{y \subto x} &=
  \subst{y + y^2}{y \subto x} = x + x^2
  \\
  \subst{\subst{x + y^2}{y \subto x}}{x \subto y} &=
  \subst{x + x^2}{x \subto y} = y + y^2.
\end{align*}
%

V nadaljevanju bomo obravnavali pravila sklepanja za univerzalne in
eksistenčne kvantifkatorje, v katerih se pojavi substitucija. Ker je
sam zapis za substitucijo nekoliko nepregleden, bomo uporabili
nekoliko manj pravilen, a bolj praktičen zapis. Denimo, da imamo
logično formulo $\phi$, v kateri se morda pojavi spremenljivka $x$, ni
pa to nujno. Tedaj pišemo $\phi(x)$. Če želimo zamenjati $x$ z izrazom
$e$, zapišemo $\phi(e)$. To je pravzaprav običajni zapis, kot ga
uporabljajo matematiki za zapis funkcij, mi pa smo ga uporabili za
zapis logičnih formul. Če bi uporabili zapis s substitucijo, bi
formulo označili samo s $\phi$ namesto s $\phi(x)$ in zamenjavo s
$\xsubst{\phi}{x \subto e}$ namesto s $\phi(e)$. Zakaj je ta bolj
priročen zapis hkrati manj pravilen? V formalni logiki strogo ločimo
med \emph{simbolnim zapisom} matematičnega pojma, ki je zaporedje
znakov na papirju, in njegovim \emph{pomenom}, ki je matematična
abstrakcija. Substitucija $\xsubst{\phi}{x \subto e}$ nam pove, kako
niz znakov $\phi$ predelamo v novi niz znakov, torej deluje na novoju
simbolnega zapisa. Ko pišemo $\phi(x)$ pa si že predstavljamo, da je
$\phi$ matematična funkcija, ki deluje na argumentu $x$. S tem nastopi
zmešnjava med simbolnim zapisom in pomenom. Dokler se zmešnjave
zavedamo, je vse v redu.

\subsection{Univerzalni kvantifikator}
\label{sec:univerzalni-kvantifikator}

Univerzalna kvantifikacija $\all{x \in S} \phi$ se prebere ">Za vse $x$
iz $S$ velja $\phi$."< Pravili sklepanja sta
%
\begin{mathpar}
  \inferrule
  {\infer*{\phi(x)}{[x \in S]}}{\all{x \in S} \phi(x)} \ \text{($x$ svež)}
  \and
  \inferrule{\all{x \in S} \phi(x) \\ e \in S}{\phi(e)}
\end{mathpar}
%
pri čemer je $x$ spremenljivka, $\phi(x)$ logična formula in $e$ poljuben izraz.

V besedilu dokažemo se pravilo vpeljave zapiše:
%
\begin{quote}
  \it
  %
  Dokazujemo $\all{x \in S} \phi(x)$:
  %
  \begin{itemize}
  \item[] Naj bo $x \in S$ poljuben.
  \item[] (Dokaz, da velja $\phi(x)$).
  \end{itemize}
  %
  Dokazali smo $\all{x \in S} \phi(x)$.
\end{quote}
%
Pravilo uporabe v besedilu ponavadi ni eksplicitno navedeno, če pa bi
ga že zapisali, bi šlo takole:
%
\begin{quote}
  \it
  %
  Dokazujemo, da velja $\phi(e)$:
  \begin{itemize}
  \item[] (Dokaz, da velja $\all{x \in S} \phi(x)$.)
  \item[] (Dokaz, da velja $e \in S$.)
  \end{itemize}
  %
  Torej velja $\phi(e)$.
\end{quote}


Ob pravilu vpeljave stoji stranski pogoj, da mora biti spremenljivka
$x$ ">sveža"<. To pomeni, da se $x$ ne sme pojavljati drugje v dokazu,
saj bi sicer lahko prišlo do zmešnjave med vezanimi in prostimi
spremenljivkami. V besedilu se dejstvo, da je $x$ svež izraža z
besedico ">poljuben"< ali ">katerikoli"<. Primer, kako gredo stvari
narobe, če ne pazimo in pomešamo spremenljivke:

\begin{izrek}[z napako v dokazu]
  Če je $x$ večji od~$42$, so vsa realna števila večja od~$23$.
\end{izrek}

\begin{proof}
  Denimo, da bi nekoliko nerodno zapisali izrek simbolno takole:
  %
  \begin{equation*}
    x > 42 \lthen \all{x \in \RR} x > 23.
  \end{equation*}
  %
  To je sicer dovoljeno, saj se prosti $x$, ki stoji zunaj $\forall$
  ni ujel, ni pa preveč smotrno, ker smo na dobri poti, da bomo
  zunanji prosti $x$ in vezanega znotraj $\forall$ pomešali. Res, če
  ne upoštevamo pravila, da mora biti $x$ svež, dobimo tale nepravi
  ">dokaz"<:
  %
  \begin{equation*}
    \inferrule*
    {
      \inferrule*
      {\inferrule*
        {[x > 42] \\ 42 > 23}
        {x > 23}
      }
      {\all{x \in \RR} x > 23}
    }
    {x > 42 \lthen \all{x \in \RR} x > 23}
  \end{equation*}
  %
  Pri pravilu za vpeljavo $\forall$ smo uporabili spremenljivko $x$,
  ki pa je že nastopala v začasni hipotezi $x > 42$. Z besedilom bi se
  isti dokaz glasil takole:
  %
  \begin{quote}
    ">Dokazujemo $x > 42 \lthen \all{x \in \RR} x > 23$. Predpostavimo,
    da velja $x > 42$ in dokažimo $\all{x \in \RR} x > 23$. Naj bo $x
    \in \RR$. Po predpostavki je $x > 42$ in ker je $42 > 23$, od tod
    sledi $x > 3$."<
  \end{quote}
  %
  Če bi izrek zapisali bolje kot $x > 42 \lthen \all{y \in \RR} y >
    23$, težav ne bi bilo, saj bi se prejšnji dokaz ">zataknil"<:
  %
  \begin{quote}
    ">Dokazujemo $x > 42 \lthen \all{y \in \RR} y > 23$. Predpostavimo,
    da velja $x > 42$ in dokažimo $\all{y \in \RR} y > 23$. Naj bo $y
    \in \RR$. (Kaj zdaj? Lahko sicer dokažemo $x > 23$, a zares bi
    morali dokazati $y > 23$, kar ne gre.)"<
  \end{quote}
\end{proof}

Pogoj, da mora biti spremenljivka $x$ v pravilu za vpeljavo ">sveža"<,
se v praksi kaže v tem, da pri uvajanju nove spremenljivke izberemo
zanjo novo ime, ki se še ni pojavilo v dokazu.


\subsection{Eksistenčni kvantifikator}
\label{sec:eksistencni-kvantifikator}

Eksistenčna kvantifikacija $\some{x \in S} \phi$ se prebere ">obstaja
$x$ iz $S$, za katerega velja $\phi$"< ali ">za neki $x$ iz $S$ velja
$\phi$."< Pravili sklepanja za eksistenčni kvantifikator se glasita
%
\begin{mathpar}
  \inferrule
  {\phi(e) \\ e \in S}
  {\some{x \in S} \phi(x)}
  \and
  \inferrule
  {\some{x \in S} \phi(x)
    \\
    \infer*{\psi}{[x \in S \land \phi(x)]}}
  {\psi}\ \text{($x$ svež)}
\end{mathpar}
%
kjer je $e$ poljuben izraz in $x$ spremenljivka. Pri tem mora biti $x$
v pravilu uporabe svež. V besedilu pravilo vpeljave uporabimo takole:
%
\begin{quote}
  \it
  %
  Dokazujemo $\some{x \in S} \phi(x)$:
  %
  \begin{enumerate}
  \item (Skonstruiramo element $e \in S$.)
  \item (Dokažemo, da velja $\phi(e)$.)
  \end{enumerate}
  %
  Dokazali smo $\some{x \in S} \phi(x)$.
\end{quote}
%
Pravilo uporabe pa se v besedilu izraža takole:
%
\begin{quote}
  \it
  %
  Dokazujemo $\psi$:
  %
  \begin{enumerate}
  \item (Dokaz izjave $\some{x \in S} \phi(x)$.)
  \item Predpostavimo, da za $x \in S$ velja $\phi(x)$:
    %
    \begin{itemize}
    \item[] (Dokaz izjave $\psi$.)
    \end{itemize}
  \end{enumerate}
  %
  Dokazali smo $\psi$.
\end{quote}

\subsubsection{Enolični obstoj}
\label{sec:enolicni-obstoj}

Poleg običajnega eksistenčnega kvantifikatorja $\exists$ poznamo tudi
\emph{enolični} eksistenčni kvantifikator $\exists!$. Izjavo
$\exactlyone{x}{S}{\phi}$ preberemo ">obstaja natanko en $x \in S$, za
katerega velja $\phi(x)$"<.

Enolični eksistenčni kvantifikator ni osnovni logični operator, ampak
je $\exactlyone{x}{S}{\phi}$ le okrajšava za
%
\begin{equation}
  \label{eq:uniqe-exists}
  \some{x \in S} \phi(x) \land (\all{y \in S} \phi(y) \lthen x = y).
\end{equation}
%
Z besedami preberemo to izjavo takole: ">obstaja $x$ iz $S$, za
katerega velja $\phi(x)$ in za vsak $y \in S$ za katerega velja
$\phi(y)$ sledi $x = y$"<. To je samo zapleten način, kako povedati,
da obstaja natanko en element množice~$S$, ki zadošča pogoju $\phi$.

Pravilo sklepanja za vpeljavo enoličnega obstoja izpeljemo
iz~\eqref{eq:uniqe-exists}:
%
\begin{equation*}
  \inferrule{
    e \in S
    \\
    \phi(e)
    \\
    \infer*{y = e}{y \in S \land \phi(y)}
  }
  {\exactlyone{x}{S}{\phi}}
\end{equation*}
%
V besedilu dokažemo enolični obstoj takole:
%
\begin{quote}
  \it
  %
  Dokazujemo, da obstaja natanko en $x \in S$, za katerega velja
  $\phi(x)$:
  %
  \begin{enumerate}
  \item Obstoj: (Konstrukcija elementa $e \in S$ in dokaz, da velja $\phi(x)$.)
  \item Enoličnost: denimo da za $y \in S$ velja $\phi(y)$:
    %
    \begin{itemize}
    \item[] (Dokaz, da je $e = y$).
    \end{itemize}
  \end{enumerate}
  %
  Dokazali smo $\exactlyone{x \in S} \phi(x)$.
\end{quote}

Če dokažemo enolični obstoj $\exactlyone{x \in S} \phi(x)$, lahko
vpeljemo novo konstanto $c$, ki označuje tisti element iz $S$, ki
zadošča pogoju~$\phi$, pri čemer moramo seveda paziti, da znaka $c$
nismo že prej uporabili za kak drug pomen. Nova konstanta~$c$ je
opredeljena s praviloma
%
\begin{mathpar}
  \inferrule{ }{\phi(c)}
  \and
  \inferrule{
    y \in S
    \\
    \phi(y)
  }
  {y = c}
\end{mathpar}
%
Če v formuli $\phi$ poleg spremenljivke $x$ nastopajo še druge proste
spremenljivke, denimo $y_1, \ldots, y_n$, potem je nova konstanta~$c$
v resnici \emph{funkcija} parametrov $y_1, \ldots, y_n$.

\subsection{Enakost in reševanje enačb}
\label{sec:enakost}

Enakost $=$ je osnovna relacija, ki zadošča naslednjim aksiomom in
pravilom sklepanja:
%
\begin{mathpar}
  \inferrule{ }{a = a}
  \and
  \inferrule{a = b}{b = a}
  \and
  \inferrule{a = b \\ b = c}{a = c}
  \and
  \inferrule{\phi(a) \\ a = b}{\phi(b)}
\end{mathpar}
%
Po vrsti so so pravilo \emph{refleksivnosti}, \emph{simetrije},
\emph{tranzitivnosti} in \emph{zamenjave}. Zaenkrat enakosti ne bomo
posvečali posebne pozornosti, saj jo v praksi študenti dobro
obvladajo.

V osnovni iz srednji šoli se učimo pravil za reševanje enačb: enačbi
smemo na obeh straneh prišteti ali odšteti poljuben izraz, pomnožiti
ali deliti smemo s poljubnim \emph{neničelnim} izrazom, ipd. Od kod
izhajajo ta pravila? Kaj sploh pomeni, da smo enačbo ">rešili"<? Ko
rešimo kvadratno enačbo
%
\begin{equation*}
  x^2 - 5 x + 6 = 0
\end{equation*}
%
običajno zapišemo rešitev takole:
%
\begin{equation*}
  x_1 = 2, \quad x_2 = 3.
\end{equation*}
%
Kako naj to razumemo iz stališča matematične logike? Treba je
pojasniti dvoje: kaj pomenita $x_1$ in $x_2$, saj v prvotni enačbi
nastopa spremenljivka $x$, ter kako naj razumemo vejico med izjavama
$x_1 = 2$ in $x_2 = 3$. Z indeksoma $1$ in $2$ štejemo rešitve enačbe
in sta v resnici nepotrebna,\footnote{Kako pa bi zapisali rešitve
  enačbe $x_1^2 - 5 x_1 + 6 x = 0$?} na kar kaže tudi dejstvo, da
pišemo $x = \ldots$ in ne $x_1 = \ldots$, kadar je rešitev ena sama.
Torej bi lahko rešitev zapisali kot
%
\begin{equation*}
  x = 2, \quad x = 3.
\end{equation*}
%
Sedaj pa je tudi jasno, da bi namesto vejice morala stati disjunkcija,
se pravi
%
\begin{equation*}
  x = 2 \lor x = 3.
\end{equation*}
%
Začetna enačba in tako zapisana rešitev sta logično ekvivalentni:
%
\begin{equation*}
  x^2 - 5 x + 6 = 0 \iff
  x = 2 \lor x = 3.
\end{equation*}
%
Povzemimo: reševanje enačbe je postopek, s katerim dano enačbo $f(x) =
g(x)$ prevedemo v njen \emph{logično ekvivalentno} obliko $x = a_1
\lor x = a_2 \lor \cdots \lor x = a_n$, iz katere so neposredno razvidne
rešitve enačbe.

Pravila za reševanje enačb torej niso nič drugega kot recepti, s
pomočjo katerih enačbo predelamo v njen \emph{ekvivalentno} obliko, ki
je korak bližje končni obliki, v kateri bi radi zapisali rešitev. To
pojasnjuje srednješolska pravila za reševanje enačb. Na primer, za
realna števila $a, b, c \in \RR$ vedno velja
%
\begin{equation*}
  a = b \lthen c \cdot a = c \cdot b,
\end{equation*}
%
medtem ko obratna implikacija
%
\begin{equation*}
  c \cdot a = c \cdot b \lthen a = b
\end{equation*}
%
za splošne $a$ in $b$ velja le v primeru, ko je $c \neq 0$. Ker pri
reševanju enačb potrebujemo implikacijo v obe smeri, srednješolce
učimo, da smejo enačbo množiti samo z od nič različnimi števili.

\begin{vaja}
  Kako bi srednješolcem pojasnil, od kod izvira pravilo za množenje
  enačbe z neničelnim številom?
\end{vaja}

\begin{vaja}
  Enačbo $f(x) = g(x)$ smo ">rešili"< z zaporedjem korakov
  %
  \begin{align*}
    f(x) = g(x) &\liff \\
    f_1(x) = g_1(x) &\liff \\
    \vdots & \\
    f_k(x) = g_k(x) &\lthen \\
    f_{k+1}(x) = g_{k+1}(x) &\liff \\
    \vdots & \\
    x = a_1 \lor \cdots \lor x = a_n
  \end{align*}
  %
  kjer smo v $k$-tem koraku namesto ekvivalence pomotoma naredili
  implikacijo. Smo s tem dobili preveč ali premalo rešitev prvotne
  enačbe?
\end{vaja}



%%% Local Variables: 
%%% mode: latex
%%% TeX-master: "lmn"
%%% End: 

        \chapter{Dokazovanje}\label{poglavje:dokazovanje}

        Matematične izsledke običajno podajamo preko jasno izraženih izjav. Med študijem matematike hitro opazite, da se takšne izjave podajajo pod imeni `izrek', `trditev', `lema', {posledica} in podobno. Kdaj uporabiti katerega teh imen ni natanko določeno, pač pa je prepuščeno presoji matematika. Približno vodilo je naslednje:
        \begin{itemize}
                \item
                        \df{izrek}: osrednji, bistven rezultat,
                \item
                        \df{trditev}: stranski rezultat,
                \item
                        \df{lema}: rezultat, ki sam po sebi nima toliko vsebine, se pa uporabi pri dokazovanju pomembnejšega rezultata,\footnote{Sicer ni nujno, da se resnična pomembnost izjav takoj pokaže. Mnogo je primerov, ko se kak matematični članek po določenem času začne ceniti ne toliko zaradi glavnega izreka, pač pa zaradi neke leme, ki se je za dokaz glavnega izreka uporabila.}
                \item
                        \df{posledica}: rezultat, ki je zanimiv sam po sebi, ki pa hitro sledi iz predhodne izjave.
        \end{itemize}

        Če skrbno analizirate izreke, trditve itd.~s predavanj (ali iz matematičnih člankov), opazite, da sestojijo iz treh delov: kontekst, predpostavke, sklepi.
        \begin{itemize}
                \item
                        \df{Kontekst} pove, katere objekte obravnavamo in kakšne vrste so.
                \item
                        \df{Predpostavke} so izjave, ki jih privzamemo.
                \item
                        \df{Sklepi} so izjave, ki jih (pri danih predpostavkah) dokazujemo.
        \end{itemize}

        Oglejmo si konkreten primer. Rolleov izrek je znan in uporaben izrek v analizi (če ga še niste spoznali, ga boste v kratkem).

        \begin{izrek}[Rolle]
                Naj bo $f$ realna funkcija, definirana na intervalu $\intcc{a}{b}$, kjer sta $a$ in $b$ realni števili in $a < b$. Če je $f$ zvezna na celem $\intcc{a}{b}$ in odvedljiva na odprtem intervalu $\intoo{a}{b}$ ter zavzame enaki vrednosti v krajiščih, tj.~$f(a) = f(b)$, tedaj ima $f$ stacionarno točko na $\intoo{a}{b}$.
        \end{izrek}

        Analizirajmo, kaj so kontekst, predpostavke in sklepi pri tem izreku.

        \begin{itemize}
                \item
                        Kontekst je sledeč:
                        \[a \in \RR, \qquad b \in \RR_{> a}, \qquad f \in \RR^{\intcc{a}{b}}.\]
                        To so objekti (in njihove vrste), o katerih govori izrek. Smiselno je, da jih zapišemo v tem vrstnem redu; na primer, $f$ zapišemo nazadnje, saj je njena domena odvisna od $a$ in $b$. Kadar imamo objekte, ki so neodvisni med sabo, jih lahko zapišemo v poljubnem vrstnem redu.
                \item
                        Predpostavke so tri. Vsako navedimo v običajnem jeziku in nato še s simbolnim matematičnim zapisom.
                        \begin{itemize}
                                \item
                                        $f$ je zvezna na $\intcc{a}{b}$.
                                        \[
                                                \hspace{-2em}
                                                \all{x \in \intcc{a}{b}}
                                                        \all{\epsilon \in \RR_{> 0}}
                                                                \some{\delta \in \RR_{> 0}}
                                                                        \all{y \in \intcc{a}{b}}
                                                                                (|x - y| < \delta \impl \big|f(x) - f(y)\big| < \epsilon
                                                                        )
                                        \]
                                \item
                                        $f$ je odvedljiva na $\intoo{a}{b}$.
                                        \begin{multline*}
                                                \all{x \in \intoo{a}{b}}
                                                        \some{v \in \RR}
                                                                \all{\epsilon \in \RR_{> 0}}
                                                                        \some{\delta \in \RR_{> 0}}
                                                \all{h \in \RR_{\neq 0}} \\
                                                        (|h| < \delta \implies \Big|\frac{f(x + h) - f(x)}{h} - v\Big| < \epsilon)
                                        \end{multline*}
                                \item
                                        $f$ na krajiščih intervala zavzame enaki vrednosti.
                                        \[f(a) = f(b)\]
                        \end{itemize}
                        Če se vam morda zdita formuli za zveznost in odvedljivost begajoči, imate dve tolažbi. Prva je ta, da se boste čez čas takšnih formul navadili. ;) Druga je, da so tudi drugi matematiki leni po naravi in zato uvedejo oznake za daljše izraze, ki se pogosto uporabljajo. Zgornja zveznost se na krajše zapiše $f \in \mathcal{C}(\intcc{a}{b})$ ($\mathcal{C}$ kot ``continuous'', tj.~zvezen), odvedljivost pa $f \in \mathcal{D}^1(\intoo{a}{b})$ ($\mathcal{D}$ kot ``differentiable'', tj.~odvedljiv, enka pa pomeni ``(vsaj) enkrat odvedljiv'').
                \item
                        Sklep je eden: $f$ ima stacionarno točko na $\intoo{a}{b}$, kar simbolno zapišemo takole.
                        \[\some{x \in \intoo{a}{b}} f'(x) = 0\]
        \end{itemize}

        V splošnem imamo določeno mero svobode, kako natančno razčleniti izrek. Na primer, za Rolleov izrek bi lahko kontekst zapisali tudi kot $a \in \RR, b \in \RR, f \in \RR^{\intcc{a}{b}}$ in pogoj $a < b$ dodali med predpostavke.

        Da ne bomo pisali dolgih seznamov, se dogovorimo za sledeče oznake. Izrek podamo tako, da najprej zapišemo kontekst, nato dvopičje, nato narišemo vodoravno črto, nad črto zapišemo predpostavke (ločene z vejicami), pod črto pa sklepe (ločene z vejicami). Rolleov izrek bi potemtakem povzeli takole.
        \[\claim{a \in \RR, b \in \RR_{> a}, f \in \RR^{\intcc{a}{b}}}{f \in \mathcal{C}(\intcc{a}{b}), f \in \mathcal{D}^1(\intoo{a}{b}), f(a) = f(b)}{\some{x \in \intoo{a}{b}} f'(x) = 0}\]

        V splošnem velja: vse proste spremenljivke, ki se pojavijo v predpostavkah ali sklepih, morajo biti navedene v kontekstu. Po domače povedano: če trdite, da za neko stvar nekaj velja, morate najprej povedati, o kateri stvari sploh govorite.

        Medtem ko je za težje matematične izreke potrebno obilo ustvarjalnosti, da se jih dokaže, pa lažje trditve pogosto lahko avtomatično dokažemo (dobesedno --- obstajajo avtomatični dokazovalniki \davorin{koliko povemo na to temo?}), pa tudi za težje je pomembno, da vemo, kako pristopiti k dokazu. Gre za to, da za vse logične veznike in kvantifikatorje obstajajo splošna pravila, kako ravnamo, če nastopajo kot predpostavke oziroma kot sklepi. To si bomo zdaj ogledali.

        \begin{itemize}
                \item\textbf{Konjunkcija}
                        \begin{itemize}
                                \item
                                        Če $p \land q$ nastopa kot \emph{predpostavka}:
                                        \begin{quote}
                                                predpostavko $p \land q$ nadomestimo s predpostavkama $p$, $q$ (to se pravi, pri dokazovanju lahko uporabimo tako predpostavko $p$ kot predpostavko $q$). S simboli, od trditve
                                                \[\claim{\Gamma}{\Pi', p \land q, \Pi''}{\Sigma}\]
                                                preidemo do trditve
                                                \[\claim{\Gamma}{\Pi', p, q, \Pi''}{\Sigma}\]
                                                (pri zapisih splošnih izrekov bomo kontekst označevali z $\Gamma$, predpostavke s $\Pi$ in sklepe s $\Sigma$).
                                        \end{quote}
                                \item
                                        Če $p \land q$ nastopa kot \emph{sklep}:
                                        \begin{quote}
                                                sklep $p \land q$ dokažemo tako, da dokažemo posebej $p$ in posebej $q$. S simboli:
                                                \[\claim{\Gamma}{\Pi}{\Sigma', p \land q, \Sigma''}\]
                                                preoblikujemo v
                                                \[\claim{\Gamma}{\Pi}{\Sigma', p, q, \Sigma''}\]
                                                (in se zavedamo, da je za dokaz izreka potrebno dokazati \emph{vse} sklepe).
                                        \end{quote}
                        \end{itemize}
                \item\textbf{Disjunkcija}
                        \begin{itemize}
                                \item
                                        Če $p \lor q$ nastopa kot \emph{predpostavka}:
                                        \begin{quote}
                                                ločimo primere: sklepe dokažemo posebej pri predpostavki $p$ (skupaj z ostalimi predpostavkami) in posebej pri predpostavki $q$ (skupaj z ostalimi). Torej, dokazati
                                                \[\claim{\Gamma}{\Pi', p \lor q, \Pi''}{\Sigma}\]
                                                pomeni isto, kot dokazati tako
                                                \[\claim{\Gamma}{\Pi', p, \Pi''}{\Sigma} \qquad \text{kot} \qquad \claim{\Gamma}{\Pi', q, \Pi''}{\Sigma}.\]
                                        \end{quote}
                                \item
                                        Če $p \lor q$ nastopa kot \emph{sklep}:
                                        \begin{quote}
                                                izberemo si enega od $p$, $q$ in ga dokažemo. Se pravi, če imamo
                                                \[\claim{\Gamma}{\Pi}{\Sigma', p \lor q, \Sigma''},\]
                                                si izberemo eno od trditev
                                                \[\claim{\Gamma}{\Pi}{\Sigma', p, \Sigma''} \qquad \text{oziroma} \qquad \claim{\Gamma}{\Pi}{\Sigma', q, \Sigma''}\]
                                                in jo izpeljemo.
                                        \end{quote}
                        \end{itemize}
                \item\textbf{Implikacija}
                        \begin{itemize}
                                \item
                                        Če $p \impl q$ nastopa kot \emph{predpostavka}:
                                        \begin{quote}
                                                če nam kadarkoli uspe izpeljati $p$, lahko dodamo $q$ med predpostavke. Torej, če znamo dokazati
                                                \[\claim{\Gamma}{\Pi', p \impl q, \Pi''}{q},\]
                                                potem za dokaz
                                                \[\claim{\Gamma}{\Pi', p \impl q, \Pi''}{\Sigma}\]
                                                zadostuje dokazati
                                                \[\claim{\Gamma}{\Pi', p \impl q, q, \Pi''}{\Sigma}\]
                                                (kar je lažje, ker imamo eno predpostavko več). To je smiselno: če vemo, da velja $p \impl q$ in dodatno ugotovimo, da velja $p$, potem vemo, da velja tudi $q$.
                                        \end{quote}
                                \item
                                        Če $p \impl q$ nastopa kot \emph{sklep}:
                                        \begin{quote}
                                                sklep $p \impl q$ nadomestimo s $q$, medtem ko $p$ dodamo med predpostavke. Pojasnimo. Trditev $p \impl q$ trdi nekaj samo v primeru, kadar $p$ velja --- v nasprotnem primeru je avtomatično resnična in ni ničesar za dokazati. Torej se lahko omejimo na primer, ko $p$ velja, se pravi, lahko predpostavimo $p$. Kadar $p$ velja, pa trditev $p \impl q$ pravi, da mora veljati tudi $q$. To pomeni, da pri predpostavki $p$ dokazujemo $q$. Simbolno, da dokažemo
                                                \[\claim{\Gamma}{\Pi}{\Sigma', p \impl q, \Sigma''},\]
                                                zadostuje dokazati
                                                \[\claim{\Gamma}{\Pi}{\Sigma', \Sigma''} \qquad \text{in} \qquad \claim{\Gamma}{\Pi, p}{q}.\]
                                        \end{quote}
                        \end{itemize}
                \item\textbf{Univerzalni kvantifikator}
                        \begin{itemize}
                                \item
                                        Če $\all{x \in X} \phi(x, y)$ nastopa kot \emph{predpostavka}:
                                        \begin{quote}
                                                če vemo za (ali med dokazom najdemo) katerikoli konkreten element $a \in X$, tedaj lahko med predpostavke dodamo $\phi(a, y)$. Namreč, če vemo, da lastnost $\phi$ (z morebitnimi nadaljnjimi parametri) velja za vse elemente množice $X$, potem ta lastnost velja za poljuben konkreten element. Simbolno, od
                                                \[\claim{\Gamma', a \in X, \Gamma''}{\Pi', \all{x \in X} \phi(x, y), \Pi''}{\Sigma}\]
                                                preidemo do
                                                \[\claim{\Gamma', a \in X, \Gamma''}{\Pi', \all{x \in X} \phi(x, y), \phi(a, y), \Pi''}{\Sigma}.\]
                                        \end{quote}
                                \item
                                        Če $\all{x \in X} \phi(x, y)$ nastopa kot \emph{sklep}:
                                        \begin{quote}
                                                v kontekst dodamo $x \in X$, sklep $\all{x \in X} \phi(x, y)$ pa nadomestimo s sklepom $\phi(x, y)$. S simboli, od
                                                \[\claim{\Gamma}{\Pi}{\Sigma', \all{x \in X} \phi(x, y), \Sigma''}\]
                                                preidemo do
                                                \[\claim{\Gamma, x \in X}{\Pi}{\Sigma', \phi(x, y), \Sigma''}\]
                                                Zakaj tako postopamo in kaj smo s tem pravzaprav naredili? Premislimo: želimo dokazati, da neka lastnost velja za vse elemente dane množice $X$. Če ima $X$ slučajno samo končno mnogo elementov, bi lahko lastnost preverili za vsakega posebej, ampak povečini delamo z neskončnimi množicami, kjer to ne deluje. Morda ima množica $X$ kakšno posebno lastnost, zaradi katere lahko univerzalni kvantifikator dokažemo na svojevrsten način (na primer, univerzalno kvantificirane izjave nad $\NN$ lahko dokazujemo z matematično indukcijo --- glej \note{razdelek o naravnih številih}), ampak to se zgodi v izjemnih primerih.

                                                V splošnem nimamo druge možnosti, kot da si izberemo simbol (tipično kar spremenljivko v kvantifikatorju), ki nam predstavlja poljuben, katerikoli element množice in zanj dokažemo želeno lastnost. Ideja je, da spremenljivka spet nastopa v vlogi ``škatlice'', kamor lahko vstavimo poljuben element množice $X$. Če nam je dokaz lastnosti uspel, ne da bi za spremenljivko predpostavili karkoli več, kot da predstavlja element množice $X$, tedaj dobimo dokaz lastnosti za katerikoli dejanski element množice $X$ tako, da v dobljeni dokaz namesto spremenljivke vstavimo ta element. Na ta način smo potem dejansko dobili dokaz lastnosti za vse elemente množice $X$.

                                                Besedni dokazi univerzalno kvantificirane izjave se zato tipično začnejo takole: ``Vzemimo poljuben $x \in X$. Dokažimo, da zanj velja dana lastnost.''
                                        \end{quote}
                        \end{itemize}
                \item\textbf{Eksistenčni kvantifikator}
                        \begin{itemize}
                                \item
                                        Če $\some{x \in X} \phi(x, y)$ nastopa kot \emph{predpostavka}:
                                        \begin{quote}
                                                v kontekst dodamo $x \in X$, eksistenčno predpostavko pa nadomestimo s $\phi(x, y)$. S simboli,
                                                \[\claim{\Gamma}{\Pi', \some{x \in X} \phi(x, y), \Pi''}{\Sigma}\]
                                                popravimo v
                                                \[\claim{\Gamma, x \in X}{\Pi', \phi(x, y), \Pi''}{\Sigma}.\]
                                                Zakaj to deluje? Naša predpostavka je, da v množici $X$ obstaja element z lastnostjo $\phi$ (z morebitnimi nadaljnjimi parametri). Torej si lahko vzamemo neki konkreten element množice $X$ s to lastnostjo, ki ga lahko uporabljamo kasneje v dokazu (za to ga moramo nekako označiti; v praksi ga tipično označimo kar z isto spremenljivko, kot v kvantifikatorju).
                                        \end{quote}
                                \item
                                        Če $\some{x \in X} \phi(x, y)$ nastopa kot \emph{sklep}:
                                        \begin{quote}
                                                da dokažemo eksistenčno izjavo, moramo podati neki konkreten element $x \in X$ in zanj dokazati dano lastnost $\phi(x, y)$. \davorin{Hm, kako točno to zapišemo simbolno v zgornji obliki?}
                                        \end{quote}
                        \end{itemize}
        \end{itemize}

        V zgornjem seznamu nismo omenili vseh veznikov in kvantifikatorjev. To je zato, ker jih pri dokazovanju nadomestimo z zgornjimi. Konkretno:
        \begin{itemize}
                \item
                        Za negacijo velja $\lnot{p} \equiv p \impl \false$. Med drugim to pomeni, da $\lnot{p}$ dokažemo na sledeči način: predpostavimo $p$ in iz tega izpeljemo neresnico.
                \item
                        Za ekvivalenco velja $p \lequ q \equiv (p \impl q) \land (p \revimpl q)$. To pomeni, da ekvivalenco dokažemo tako, da dokažemo implikacijo med $p$ in $q$ v obe smeri --- se pravi, enkrat predpostavimo $p$ in izpeljemo $q$, drugič pa predpostavimo $q$ in izpeljemo $p$.
                \item
                        Za veznike $\xor$, $\shf$, $\luk$ si preprosto izberemo eno od izražav z negacijo, konjunkcijo in disjunkcijo in nato delamo z njo.
                \item
                        Kvantifikator $\exactlyone{x \in X} \phi(x, y)$ ločimo na dva dela: na obstoj in enoličnost, in vsakega posebej dokažemo. Se pravi, skličemo se na izražavo
                        \[\exactlyone{x \in X} \phi(x, y) \equiv \some{x \in X} \phi(x, y) \land \all{a, b \in X} (\phi(a, y) \land \phi(b, y) \implies a = b).\]
                        Včasih je lažje, če najprej dokažemo obstoj elementa in ta element pri dokazu enoličnosti že uporabimo, torej dokazujemo izražavo
                        \[\exactlyone{x \in X} \phi(x, y) \equiv \some{x \in X} (\phi(x, y) \land \all{a \in X} (\phi(a, y) \implies a = x)).\]
        \end{itemize}

        Seveda ne bo možno dokazati vsakega izreka s slepim sledenjem zgornjim pravilom; včasih moramo uporabiti še kakšno dodatno strategijo. Spodnji dve sta zelo pogosti.
        \begin{itemize}
                \item
                        Med predpostavke dodamo trditev, za katero že vemo, da je resnična. Morda gre za trditev, ki smo jo že dokazali, morda pa gre kar za istorečje. Pogost primer tega je, da uporabimo zakon o izključenem tretjem in za dodatno predpostavko vzamemo $p \lor \lnot{p}$ (kjer je $p$ katerakoli konkretna izjava). Po zgornjih pravilih to potem pomeni, da ločimo primere in trditev dokažemo posebej pri predpostavki $p$ ter posebej pri predpostavki $\lnot{p}$.
                \item
                        Nekatere predpostavke ali sklepe nadomestimo z enakovrednimi izjavami. Na primer, velja
                        \[p \lor q \equiv \lnot(\lnot{p} \land \lnot{q}) \equiv \lnot{p} \impl q \equiv \lnot{q} \impl p.\]
                        To pomeni, da lahko disjunkcijo (poleg zgoraj omenjenega načina) dokažemo tudi tako, da predpostavimo, da nobena od možnosti ne velja, in od tod izpeljemo neresnico, ali pa predpostavimo, da ena od možnosti ne velja, in od tod izpeljemo drugo.

                        Zelo pogosta uporaba te ideje je \df{dokaz s protislovjem}, ki temelji na zakonu o dvojni negaciji $p \equiv \lnot\lnot{p}$. Izjavo torej lahko dokažemo tako, da predpostavimo njeno negacijo, in od tod izpeljemo neresnico. Tipičen besedni dokaz s protislovjem izgleda takole: ``Dokazujemo $p$. Pa recimo, da $p$ ne velja. Potem /neki sklepi/. To je v nasprotju s tem, kar smo dokazali prej, torej smo izpeljali protislovje. Se pravi, ni možno, da $p$ ne bi veljal, torej mora veljati.''
        \end{itemize}

        \note{mnogo zgovornih primerov dokazov, ki ponazorijo zgornje postopke}


\section{Vaje}


%%% Local Variables:
%%% mode: latex
%%% TeX-master: "ucbenik-lmn"
%%% End:


        \chapter{Konstrukcije množic}
                \section{Preprosti primeri}
                        \note{prazna množica, enojci}
                \section{Podmnožice}
                        \davorin{Če \qt{embedding} prevajamo kot \qt{vložitev}, kako potem prevedemo \qt{inclusion}? Imamo sicer tujko \qt{inkluzija}, ampak fino bi bilo imeti še slovenski izraz. Vključitev?}
                \section{Potenčna množica}
                        \davorin{Verjetno je smiselno, da ta razdelek sledi razdelku o podmnožicah. Morda kar združimo ta dva razdelka?}
                \section{Družine množic}
                \section{Produkt množic}
                \section{Vsota množic}
                \section{Unija in presek}
                \section{Eksponentna množica}
                        \davorin{Vrstni red teh razdelkov bomo najbrž še premešali.}

        \chapter{Preslikave}


\section{Prestavljeno iz matematičnega izražanje sem, začasno}
\label{sec:prest-iz-matem}



\section{Preslikave}

Množice ne obstajajo povsem ločene ena od druge, pač pa so med sabo povezane s \df{preslikavami} oziroma s tujko \df{funkcijami}\footnote{Nekateri uporabljajo izraz \qt{funkcija} samo za tiste preslikave, ki slikajo v realna ali kompleksna števila, ampak ta uporaba je že nekoliko zastarela. Dandanes večina matematiko besedo \qt{funkcija} obravnava kot sopomenko besede \qt{preslikava}. Tako jo bomo uporabljali tudi v tej knjigi.}. Posamična preslikava slika elemente ene množice po določenem predpisu v elemente druge množice.

Če je $f$ preslikava, ki slika iz množice $X$ v množico $Y$, to zapišemo $f\colon X \to Y$. Rečemo, da je množica $X$ \df{začetna množica} ali \df{domena} preslikave $f$, množica $Y$ pa je \df{ciljna množica} ali \df{kodomena} preslikave $f$.

Začetni množici ste v srednji šoli rekli tudi \qt{definicijsko območje}, ampak v tej knjigi bomo morali biti bolj previdni. Predpis za preslikavo po definiciji velja za vse elemente domene, ampak kasneje (v razdelku~\ref{razdelek:izpeljava-preslikav-iz-relacij}) obravnavamo delne preslikave, ki niso definirane na celi domeni; zanje je torej definicijsko območje manjše kot domena. Držimo se raje zgoraj danega poimenovanja.

Običaj je, da predpis preslikave podamo s pomočjo spremenljivke, tipično z oznako $x$. Na primer, če je $f$ preslikava kvadriranja, njen predpis zapišemo kot
\[f(x) = x^2.\]
Na tem mestu je potrebno poudariti več reči.
\begin{itemize}
\item
Velikokrat površno rečemo, da zgornji predpis podaja preslikavo. To ni povsem res --- to je zgolj predpis preslikave. Za to, da preslikavo v celoti podamo, je potrebno navesti tri stvari: poleg predpisa še domeno in kodomeno. Vse to je del informacije o preslikavi.

To se jasno pokaže, če začnemo razmišljati o lastnostih preslikav. Se še spomnite iz srednje šole, kaj pomeni, da je preslikava surjektivna? (Bomo ponovili v razdelku~\ref{razdelek:injektivnost-in-surjektivnost}.) Če vzamemo, da preslikava $f$ zadošča zgornjemu predpisu in jo obravnavamo kot preslikavo $f\colon \RR \to \RR$, ni surjektivna, če jo obravnavamo recimo kot preslikavo $f\colon \RR_{\geq 0} \to \RR_{\geq 0}$, pa je.
\item
Za spremenljivko $x$ velja isto, kot smo razpravljali že v prejšnjem razdelku pri lastnostih elementov množic: spremenljivka $x$ nima vnaprej določene vrednosti, pač pa predstavlja mesto, kamor lahko vstavimo poljubno vrednost. Seveda je potem vseeno, če vzamemo kakšno drugo črko ali čisto drug simbol: $f(y) = y^2$ določa isti predpis kot $f(x) = x^2$; prav tako $f(\heartsuit) = \heartsuit^2$. Se pravi, tudi v tem primeru gre za \note{dummy variable}. Če si torej izberemo neko vrednost, jo lahko vstavimo na mesto spremenljivke in poračunamo, npr.~$f(3) = 3^2 = 9$ oziroma $f(2\pi) = (2\pi)^2 = 4\pi^2$. Predstavljajte si, da je spremenljivka pravzaprav škatlica, kamor lahko vstavite vrednost, torej
\[f(\argbox) = \argbox^2.\]
\item
Alternativen način zapisa $f(x) = x^2$ je
\[f\colon x \mapsto x^2.\]
Pazimo: navadna puščica $\to$ podaja domeno in kodomeno, kot razloženo zgoraj. Repata puščica $\mapsto$ pa za posamičen element domene pove, v kateri element kodomene se preslika.

Zapis z repato puščico je še posebej uporaben, kadar želimo podati preslikavo, ne da bi nam bilo potrebno izbrati ime zanjo. Na primer, realno funkcijo kvadriranja lahko v celoti podamo takole:
\begin{align*}
\RR &\to \RR \\
x &\mapsto x^2
\end{align*}
(prva vrstica pove domeno in kodomeno, druga pa predpis). Tako podanim preslikavam potem rečemo \df{brezimne preslikave} (s tujko \df{anonimne funkcije}). Kasneje (v razdelku~\ref{razdelek:brezimne-preslikave}) bomo spoznali bolj strnjen zapis takih preslikav, ki je še posebej primeren za izvajanje operacij med preslikavami; takrat bomo takšno funkcijo zapisali kot $\xlam{x}[\RR]{x^2}[\RR]$.
\end{itemize}

\note{Sklop (kompozicija, kompozitum) preslikav. Identiteta kot enota za sklop. Razčlenitev (dekompozicija, faktorizacija) preslikav.}

\davorin{Definirati moramo tudi oznako $\set{f(x)}{x \in X}$, kar je druge vrste oznaka kot prej definirana $\set{x \in X}{\phi(x)}$. Se gremo primerjavo s Pythonom (razlika med \texttt{\{f(x) for x in X\}} in \texttt{\{x if phi(x)\}})? Smo matematični hipsterji in uvedemo oznako $\{f(x) \,|\, x \in X \,|\, \phi(x)\}$, ki ustreza \texttt{\{f(x) for x in X if phi(x)\}}, kar bi tudi prišlo prav?}

Zaenkrat smo imeli primere, ko je bil prepis preslikave dan z eno samo spremenljivko, npr.~$f(x) = x^2$. Zelo pogoste so pa tudi \df{preslikave več spremenljivk}, npr.~$f(x, y) = x^2 + y^2$. Že osnovne računske operacije so take --- na primer, pri seštevanju vzamemo \emph{dva} podatka in vrnemo rezultat (vsoto).

V takem primeru je smiselno reči: domena preslikave sestoji iz \df{dvojic} ali \df{parov} števil. Pri seštevanju je to, katero število je prvo, katero pa drugo, sicer nepomembno, pri kakšni drugi operaciji (npr.~že odštevanju), pa je, zato posebej zahtevajmo: gre za \df{urejene dvojice} (\df{pare}). Urejeno dvojico elementov $a$ in $b$ (v tem vrstem redu) po dogovoru zapišemo kot $(a, b)$. Vrednosti $a$ in $b$ imenujemo \df{komponenti} tega para; natančneje, $a$ je \df{prva komponenta}, $b$ pa \df{druga komponenta}.

Če imamo dve množici $A$ in $B$, tedaj množico vseh urejenih dvojic, katerih prva komponenta je element iz $A$, druga komponenta pa element iz $B$, označimo $A \times B$ in imenujemo \df{zmnožek} ali \df{produkt} množic $A$ in $B$. Glede na to, da obstaja mnogo operacij, ki se imenujejo \qt{produkt} (poznate že vsaj produkt števil, produkt števila z vektorjem, skalarni produkt vektorjev in vektorski produkt vektorjev, obstaja pa jih še precej več), je koristno produkt množic posebej poimenovati, da ga ločimo od drugih: zanj se je uveljavil izraz \df{kartezični produkt} (izhaja iz imena Cartesius, tj.~latinske različice priimka Renéja Descarta\footnote{René Descartes (1596 -- 1650) je bil francoski filozof, matematik in znanstvenik.}).

Seštevanje potemtakem lahko razumemo kot preslikavo $+\colon \RR \times \RR \to \RR$. V tem smislu še vedno gre za preslikavo, ki dan vhodni podatek preslika v neki rezultat, le da je vhodni podatek dvojica števil, ne pa zgolj eno število. Kadar imamo produkt več enakih faktorjev, ga lahko (kot običajno) zapišemo v obliki potence; pisali bi lahko tudi $+\colon \RR^2 \to \RR$.

Seveda nismo omejeni na preslikave samo ene ali dveh spremenljivk. Nič nam ne preprečuje definirati recimo $f(x, y, z) = 2x + y - 3z$. Smiselna domena te preslikave setoji iz \df{urejenih trojic} števil. V splošnem, če jemljemo elemente iz množic $A$, $B$, $C$, tedaj se množica vseh takih trojic označi z $A \times B \times C$. Prejšnji predpis določa potem preslikavo $f\colon \RR \times \RR \times \RR \to \RR$ (oziroma krajše $f\colon \RR^3 \to \RR$).

Spremenljivk je lahko še več; poleg dvojic in trojic tako dobimo še četverice, peterice, šesterice\ldots V splošnem takšna končna zaporedja elementov imenujemo \df{urejene večterice}. Tudi število spremenljivk je lahko označeno s črko; na primer, preslikava, ki računa povprečje $n$ števil (kjer $n \in \NN_{\geq 1}$), je dana kot
\begin{align*}
\RR^n &\to \RR \\
(x_1, x_2, \ldots, x_n) &\mapsto \frac{x_1 + x_2 + \ldots + x_n}{n}
\end{align*}
(če hočemo poudariti, da imajo naše večterice natanko $n$ komponent, jih imenujemo $n$-terice). Nadlega pri tem je sicer spet dvoumnost tropičja. Deloma jo je možno odpraviti tako, da celotno večterico označimo z eno spremenljivko. Pogosta izbira zapisa je $f(\bm{x})$ ali $f(\vec{x})$ (razlog za to je, da lahko večterico vidimo kot vektor).

Marsikdaj želimo delati ne samo z eno preslikavo, pač pa s celo množico preslikav naenkrat. Zato uvedemo: množica vseh preslikav, ki slikajo iz $X$ v $Y$, se označi kot $Y^X$; temu se reče \df{eksponent} množic $X$ in $Y$ (\note{na primernem mestu kasneje} bomo razložili, od kod ta oznaka).

\begin{zgled}
Množico vseh preslikav, ki realna števila slikajo nazaj v realna števila, označimo z $\RR^\RR$. Če nas zanimajo realne preslikave, ki so definirana samo na intervalu $\intoo{-1}{1}$, opazujemo množico $\RR^{\intoo{-1}{1}}$. Definiramo lahko preslikavo
\begin{align*}
\RR^{\intoo{-1}{1}} &\to \RR \\
f &\mapsto f(0),
\end{align*}
ki preslikavam priredi njihovo vrednost v točki $0$. Ta preslikava torej ima za argumente (tj.~vnose) celotne preslikave in ne števila! Sama po sebi je element množice $\RR^{\RR^{\intoo{-1}{1}}}$.
\end{zgled}

\begin{zgled}
Za poljubne množice $A$, $B$, $C$ lahko definiramo sledečo preslikavo, katere argumenti so pari preslikav.
\begin{align*}
B^A \times C^B &\to C^A \\
(f, g) &\mapsto g \circ f
\end{align*}
\end{zgled}


\davorin{Glede na to, da gre za slovenski učbenik, dajem izrazu `preslikava' prednost pred izrazom `funkcija'. Seveda pa sem pojasnil tudi slednji izraz (v prvem poglavju).}

\note{Uvod. Definicijsko območje in zaloga vrednosti \davorin{morda dodamo kot možno ime za zalogo vrednosti še prevod angleške besede `range', se pravi `razpon'?}. Zožitve (tako domene kot kodomene); oznake za to so $\rstr{f}_A$, $\rstr{f}^B$, $\rstr{f}_A^B$. Izvrednotenje (evalvacija) preslikave (če ne bomo tega pojasnili že pri eksponentih množic).}


\section{Brezimne preslikave}\label{razdelek:brezimne-preslikave}

\note{Tj.~anonimne oz.~čiste funkcije. Na tem mestu pride tudi $\lambda$-notacija in določena mera $\lambda$-računa.}

\davorin{Pripravil sem ukaza \texttt{$\backslash${lam}} in \texttt{$\backslash${xlam}}.}


\section{Slike in praslike}

Preslikava kot taka nam pove za posamične elemente, kam se slikajo. Marsikdaj pa nas zanima več: kam se slikajo celotne množice elementov. Na primer, zanima nas lahko, v kaj se projicira neko prostorsko telo na ravnino.

\note{luštna slika projekcije nekega prostorskega objekta na neko ravnino}

Da dobimo sliko celotne množice, moramo zbrati skupaj slike vseh posamičnih elementov množice. Smiselna je torej naslednja definicija.

\begin{definicija}\label{definicija:slika}
Naj bo $f\colon X \to Y$ preslikava. \df{Slika} množice $A \subseteq X$ je označena in definirana kot
\[\img{f}{A} \dfeq \set[1]{f(x)}{x \in A} = \set[1]{y \in Y}{\xsome{x}[A]{y = f(x)}}.\]
Ta predpis definira preslikavo $\img{f}\colon \pst(X) \to \pst(Y)$.
\end{definicija}

\begin{opomba}
Kot običajno, obstajajo različne oznake v uporabi. Sliko $\img{f}{A}$ se označuje tudi kot $f[A]$ ali celo kar kot $f(A)$. V slednjem primeru se predpostavlja zadostna matematična zrelost bralca, da zna razbrati, kdaj $f$ označuje preslikavo $f\colon X \to Y$, kdaj pa preslikavo $f\colon \pst(X) \to \pst(Y)$.

V tej knjigi se bomo načrtno izogibali takšnim dvoumnostim in za sliko dosledno uporabljali oznako iz definicije~\ref{definicija:slika}.
\end{opomba}

\begin{vaja}
Prepričaj se, da za poljubno preslikavo $f\colon X \to Y$ velja sledeče:
\begin{itemize}
\item
$\img{f}{X} = \rn{f}$,
\item
$\img{f}{\emptyset} = \emptyset$,
\item
$\img[1]{f}{\set{x}} = \set[1]{f(x)}$ za vsak $x \in X$.
\end{itemize}
\end{vaja}

\note{primeri in lastnosti slik že tu ali kasneje skupaj s primeri/lastnostmi praslik?}

Včasih pa imamo obratno nalogo: iz dane slike ugotoviti, kaj vse se je z neko preslikavo vanjo preslikalo. Zato vpeljemo še sledečo definicijo.

\begin{definicija}\label{definicija:praslika}
Naj bo $f\colon X \to Y$ preslikava. \df{Praslika} množice $B \subseteq Y$ je označena in definirana kot
\[\pim{f}{B} \dfeq \set[1]{x \in X}{f(x) \in B}.\]
Ta predpis definira preslikavo $\pim{f}\colon \pst(Y) \to \pst(X)$.
\end{definicija}

\begin{opomba}
Tudi za prasliko obstajajo različne oznake. Praslika $\pim{f}{B}$ se označi tudi kot $f^{-1}[B]$ ali kar kot $f^{-1}(B)$. V slednjem primeru se spet zanašamo na izkušenost bralca, da praslike $f^{-1}\colon \pst(Y) \to \pst(X)$ ne zamenja z obratom $f^{-1}\colon Y \to X$. Slednji morda sploh ne obstaja! (Praslika seveda obstaja za vse funkcije.)

Če inverz funkcije obstaja, tedaj velja $\pim[1]{f}{\set{y}} = \set[1]{f^{-1}(y)}$ za vsak $y \in Y$ (premisli!), kar nekoliko pojasni oznako $f^{-1}$ tudi za prasliko. Kljub vsemu, z namenom izogibanja dvoumnostim bomo se v tej knjigi skrbno držali oznake iz definicije~\ref{definicija:praslika} za prasliko.

Ko smo že pri alternativnih, potencialno zavajajočih oznakah: pri prasliki enojca se tipično spuščajo zaviti oklepaji, torej se namesto $\pim[1]{f}{\set{y}}$ piše $\pim{f}{y}$ (ali celo $f^{-1}(y)$).
\end{opomba}

\note{primeri, vaje}

\note{lastnosti: ohranjanje unij, presekov, komplementov}


\section{Injektivnost in surjektivnost}\label{razdelek:injektivnost-in-surjektivnost}

\note{Vključno z ekvivalenco z mono- in epimorfizmi.}


\section{Bijektivnost in obratne preslikave}\label{razdelek:bijektivnost-in-obratne-preslikave}

Kot dobro veste že iz srednje šole, injektivnost in surjektivnost nam omogočata definicijo bijektivnosti.

\begin{definicija}
Preslikava je \df{bijektivna}, kadar je injektivna in surjektivna.
\end{definicija}

To pomeni: če imamo bijektivno preslikavo (na kratko kar: \df{bijekcijo}) $f\colon X \to Y$, smo povezali elemente množice $X$ z elementi množice $Y$, in sicer tako, da vsakemu elementu v katerikoli od množic $X$ oz.~$Y$ pripišemo natanko en element druge množice.

\note{slika dveh množic s poparjenimi pikami}

Rečemo, da so elementi množice $X$ v \df{bijektivni korespondenci} (ali po slovensko \df{povratno enolični zvezi}) z elementi množice $Y$. Bijektivnost se na grafih kaže takole: preslikava je bijektivna, kadar vsaka vodoravnica seka njen graf natanko enkrat.

Bijektivne preslikave igrajo pomembno vlogo v matematiki. Oglejmo si tri primere.
\begin{itemize}
\item
Če imamo povratno enolično zvezo med elementi dveh množic, je jasno, da imata isto število elementov. To nam omogoča definicijo \df{kardinalnosti} množic --- glej poglavje~\note{o kardinalnosti}.
\item
Predstavljajmo si, da so elementi neke množice $X$ imena za določene objekte. Na bijektivno preslikavo $f\colon X \to Y$ lahko potem gledamo kot na preimenovanje teh objektov. Seveda preimenovanje ne spremeni narave (ali če hočete natančnejši izraz, matematične strukture) objektov --- z drugimi besedami, $X$ in $Y$ se razlikujeta zgolj po imenih svojih elementov. To nas privede do pojma \df{izomorfizma}. Za več podrobnosti glej poglavje~\note{o strukturiranih množicah}.
\item
Če imamo povratno enolično zvezo med elementi množic $X$ in $Y$, potem ta zveza ne podaja zgolj preslikave v smeri $X \to Y$, pač pa tudi v smeri $Y \to X$, ker za vsak element iz $Y$ obstaja enolično določen element iz $X$, ki se vanj preslika. Z drugimi besedami, bijektivne preslikave imajo \df{obrate}.
\end{itemize}

Povejmo več o obratih preslikav. Začnimo s formalno definicijo.

\begin{definicija}
Naj bo $f\colon X \to Y$ poljubna preslikava. Za preslikavo $g\colon Y \to X$ rečemo, da je \df{obrat} ali \df{inverz} preslikave $f$, kadar velja
\[g \circ f = \id[X] \qquad\qquad \text{in} \qquad\qquad f \circ g = \id[Y].\]
Z drugimi besedami, $g$ je obrat $f$, kadar slika v nasprotni smeri in za vsak $x \in X$ velja $g\big(f(x)\big) = x$ ter za vsak $y \in Y$ velja $f\big(g(y)\big) = y$. Kadar obrat preslikave $f$ obstaja, rečemo, da je $f$ \df{obrnljiva} (ali \df{invertibilna}) preslikava.
\end{definicija}

\begin{zgled}\label{zgled:logaritmiranje-je-obratno-od-eksponenciranja}
Kot veš že iz srednje šole, logaritmiranje je obratno od eksponenciranja. Če smo natančnejši: preslikavi $\xlam{x}[\RR]{b^x}[\RR_{> 0}]$ in $\xlam{x}[\RR_{>0}]{\log_b x}[\RR]$ sta si obratni pri vsaki osnovi $b \in \RR_{> 0} \setminus \set{1}$.
\end{zgled}

\begin{vaja}\label{vaja:enolicnost-obrata-preslikave}
Dokaži: če sta $g$ in $h$ obrata iste preslikave $f$, tedaj $g = h$.
\end{vaja}

Vaja~\ref{vaja:enolicnost-obrata-preslikave} pove, da je obrat funkcije enolično določen, tj.~vsaka funkcija ima kvečjemu en obrat. Zato lahko uvedemo izrecno oznako: obrat preslikave $f$ (kadar obstaja) označimo z $f^{-1}$. Velja torej: kadar je preslikava $f\colon X \to Y$ obrnljiva, določa preslikavo $f^{-1}\colon Y \to X$.

Ta oznaka je nekoliko nerodna --- pomembno se je zavedati, da $f^{-1}(x)$ pomeni obrat preslikave $f$, izvrednoten na $x$, medtem kot $\big(f(x)\big)^{-1}$ pomeni obratna vrednost (v smislu deljenja) izvrednotenja preslikave $f$ na $x$. Za primerjavo, kot omenjeno v zgledu~\ref{zgled:logaritmiranje-je-obratno-od-eksponenciranja}, je obrat eksponenciranja logaritmiranje, medtem ko je obratna vrednost od $b^x$ enaka $(b^x)^{-1} = \frac{1}{b^x} = b^{-x}$.

\begin{vaja}
Premisli: če ima preslikava $f$ obrat $f^{-1}$, tedaj je tudi $f^{-1}$ obrnljiva preslikava in velja $(f^{-1})^{-1} = f$ (torej, obrat obrata je izvorna preslikava).
\end{vaja}

\begin{vaja}
Pogosto rečemo, da sta seštevanje in odštevanje obratni operaciji. Strogo vzeto, ti dve operaciji nista obratni kot preslikavi, saj obe slikata (recimo, da ju gledamo na realnih številih) $\RR \times \RR \to \RR$, tj.~ne slikata v nasprotnih smereh. Ugotovi, v kakšnem smislu točno sta seštevanje in odštevanje obratni, tj.~kateri dve preslikavi sta pravzaprav druga drugi obratni.
\end{vaja}

Zakaj se sploh ukvarjamo z obrati? Pogosto obravnavamo preslikavo, ki izhaja iz nekega konkretnega (na primer fizikalnega) problema, v smislu, da preslikava vzame začetne podatke in nam vrne, kaj se bo na koncu zgodilo. Marsikdaj pa hočemo rešiti obraten problem: želimo določene končne rezultate in se sprašujemo, kaj morajo biti začetni pogoji, da jih bomo dosegli. V takem primeru pride prav obratna preslikava.

Kot omenjeno, obrat preslikave je enoličen. Ne velja pa, da za poljubne preslikave sploh obstaja. Na primer, naj bo $f$ edina možna preslikava $\set{0, 1} \to \set{\unit}$, torej tista, ki tako $0$ kot $1$ preslika v $\unit$. Nobena preslikava $g\colon \set{\unit} \to \set{0, 1}$ ne more biti obrat preslikave $f$, saj je $g \circ f$ gotovo konstantna in potemtakem ne more biti identiteta na $\set{0, 1}$.

Kdaj torej obstaja obrat preslikave?

\begin{trditev}
Za poljubno preslikavo $f\colon X \to Y$ sta ekvivalentni sledeči trditvi.
\begin{enumerate}
\item
Preslikava $f$ je obrnljiva.
\item
Preslikava $f$ je bijektivna.
\end{enumerate}
\end{trditev}

\begin{dokaz}
\begin{implproof}{1}{2}
Predpostavljamo, da obstaja obrat $f^{-1}$.

Dokažimo, da je $f$ injektivna. Vzemimo poljubna $x, y \in X$, za katera velja $f(x) = f(y)$. Tedaj $x = f^{-1}\big(f(x)\big) = f^{-1}\big(f(y)\big) = y$.

Dokažimo, da je $f$ surjektivna. Vzemimo poljuben $y \in Y$. Tedaj $y = f\big(f^{-1}(y)\big)$.
\end{implproof}
\begin{implproof}{2}{1}
Če je $f$ bijekcija, za vsak $y \in Y$ velja, da je $\pim[1]{f}{\set{y}}$ enojec (glej \note{ustrezne predhodne trditve v razdelku o injektivnosti in surjektivnosti}). Definirajmo $g\colon Y \to X$ na naslednji način: za vsak $y \in Y$ naj bo $g(y)$ tisti element $x \in X$, za katerega velja $\pim[1]{f}{\set{y}} = \set{x}$. \note{Iz lastnosti praslike sledi, da je $g$ obrat $f$.}
\end{implproof}
\end{dokaz}

Iz dokaza te trditve vidimo, da bi bilo koristno imeti oznako za \qt{tisti element}, če želimo podajati tovrstne preslikave s simboli. Naj bo $\phi$ lastnost elementov množice $X$ (torej predikat $\phi\colon X \to \tvs$), ki je resnična za natanko en element. Dogovorimo se, da
\[\xthat{x}[X]{\phi(x)}\]
pomeni \qt{tisti (edini) element množice $X$, ki ima lastnost $\phi$} (simbolček na začetku je mala grška črka jota). Zdaj lahko izrecno zapišemo: če je $f\colon X \to Y$ bijekcija, tedaj je njen obrat $f^{-1}\colon Y \to X$ dan s predpisom
\[f^{-1}(y) = \that[1]{x}[X]{f(x) = y}.\]

\davorin{Andrej, omenjal si, da želiš imeti to oznako. Če sem kaj zgrešil, prosim popravi.}

Zaenkrat smo to joto uporabljali zgolj kot okrajšavo za stavek v običajnem jeziku, ampak če želimo $\iota$-izraze uporabljati v matematičnih dokazih, jim moramo dati natančen matematični pomen. Definirajmo torej joto formalno matematično.

Naj bo $X$ poljubna množica. Na njej imamo enakost; obravnavajmo jo na tem mestu kot lastnost dvojic elementov iz $X$, torej kot predikat $=_X\colon X \times X \to \tvs$ (za vsak par elementov vrnemo resničnostno vrednost, ali sta komponenti para enaki). Transponirajmo to preslikavo; dobimo $\transposed{=_X}\colon X \to \tvs^X$. Ta transponiranka je injektivna: če se za $a, b \in X$ preslikavi $\xlam{x}[X]{a = x}$ in $\xlam{x}[X]{b = x}$ ujemata, se ujemata tudi njuni vrednosti pri $b$. Ker drži $b = b$, potem drži tudi $a = b$.

Če zožimo kodomeno preslikave $\transposed{=_X}$ na njeno sliko, potemtakem dobimo bijekcijo. Naj bo jota njen obrat, torej $\iota \dfeq \big(\rstr{\transposed{=_X}}^{\rn{\transposed{=_X}}}\big)^{-1}$. V tem smislu je zgornja oznaka $\xthat{x}[X]{\phi(x)}$ okrajšava za $\iota\big(\xlam{x}[X]{\phi(x)}\big)$ (kar bi seveda lahko še skrajšali do $\iota(\phi)$, ampak v praksi je to običajno manj zgovorno).


%%% Local Variables:
%%% mode: latex
%%% TeX-master: "ucbenik-lmn"
%%% End:

        \chapter{Relacije}\label{POGLAVJE: Relacije}

	V matematiki pogosto želimo izraziti, da so določeni objekti v nekem odnosu, npr.~eno število je večje od drugega; temu s tujko rečemo \df{relacija}. Kako to formalno izraziti? Ideja je, da relacijo podamo z množico vseh skupin elementov, ki so v relaciji. Na primer, relacijo $\leq$ na naravnih številih podamo kot podmnožico
	\[\set[1]{(a, b) \in \NN \times \NN}{\xsome{n}[\NN]{a + n = b}}.\]
	Torej, število $a$ je v relaciji $\leq$ s številom $b$ takrat, ko par $(a, b)$ pripada tej množici.
	
	Splošne relacije so lahko med poljubno mnogo elementi iz poljubnih (ne nujno istih) množic. Na primer, relacija komplanarnosti štirih točk v prostoru je podmnožica produkta $\RR^3 \times \RR^3 \times \RR^3 \times \RR^3$, relacija pripadnosti $\in$ med elementi neke množice $X$ in podmnožicami množice $X$ pa je podmnožica produkta $X \times \pst(X)$.
	
	Splošna definicija relacije je potemtakem naslednja.
	\begin{definicija}
		\df{Relacija} na družini množic $\mathscr{D}$ je podmnožica produkta $\prod_{X \in \mathscr{D}} X$.
	\end{definicija}
	
	V praksi se povečini uporabljajo relacije med dvema elementoma.
	\begin{definicija}
		\df{Dvojiška relacija}\footnote{Oziroma s tujko \df{binarna relacija}.} med elementi množic $X$ in $Y$ je podmnožica produkta $X \times Y$. \df{Dvojiška relacija} na množici $X$ je podmnožica produkta $X \times X$.
	\end{definicija}
	
	Skoraj vse relacije, ki nas zanimajo v tej knjigi, so dvojiške. Zato se dogovorimo, da z izrazom \qt{relacija} vselej mislimo dvojiško relacijo, razen če je izrecno rečeno drugače.
	
	Če je $R \subseteq X \times Y$ relacija, potemtakem lahko zapišemo, da sta $x \in X$ in $y \in Y$ v relaciji $R$ takole: $(x, y) \in R$. Ampak to vodi do čudnih zapisov v primeru običajnih relacij, npr.~$(2, 3) \in <$. To seveda raje zapišemo kot $2 < 3$ in posledično se dogovorimo, da v primeru dvojiške relacije raje uporabljamo zapis $x \mathrel{R} y$.
	
	
	\section{Grafi relacij}
	
		\GraphInit[vstyle = Normal]
		\tikzset
		{
			EdgeStyle/.append style = {->, bend left}
		}
		
		Relacije na majhnih množicah lahko lepo ponazorimo z usmerjenimi grafi. Graf relacije $R \subseteq X \times X$ je definiran takole: vozlišča grafa so elementi množice $X$ in za vsaka dva elementa $a, b \in X$, za katera velja $a \mathrel{R} b$, narišemo puščico od $a$ do $b$.
		
		\begin{zgled}
			Naj bo $X = \set{A, B, C, D, E, F}$ in naj bo
			\[R \dfeq \set{...}\]
			relacija na $X$. Njen graf izgleda takole.
			\begin{center}
				\begin{tikzpicture}
					\SetGraphUnit{3}
					\Vertex[Math=true, x=0, y=0]{A}
					\Vertex[Math=true, x=3, y=2]{B}
					\Vertex[Math=true, x=2, y=-3]{C}
					\Vertex[Math=true, x=6, y=1]{D}
					\Vertex[Math=true, x=8, y=-1]{E}
					\Vertex[Math=true, x=10, y=2]{F}
					
					\Edge(A)(B)
					\Loop[dist = 5em, dir = EA](B)
				\end{tikzpicture}
			\end{center}
		\end{zgled}
		\note{izgled grafa je še treba popraviti}
	
	
	\section{Operacije z relacijami}\label{RAZDELEK: Operacije z relacijami}
	
		Običajno je, da iz že danih matematičnih objektov lahko skonstruiramo nove preko določenih operacij. Z relacijami ni nič drugače; v tem razdelku si bomo ogledali običajne operacije na relacijah.
		
		Ker so relacije podmnožice, imamo za začetek vse operacije na podmnožicah. Torej, za poljubno družino $(R_i)_{i \in I}$ podmnožic produkta $X \times Y$ sta tudi unija $\bigcup_{i \in I} R_i$ in presek $\bigcap_{i \in I} R_i$ relaciji. Če je $R \subseteq X \times Y$ relacija, je njena komplementarna relacija $\complement{R} = X \times Y \setminus R \ \subseteq \ X \times Y$.
		
		Posebej imamo \df{prazno relacijo} $\emptyset \subseteq X \times Y$ (nobena dva elementa nista v relaciji) in \df{polno relacijo} $X \times Y\subseteq X \times Y$ (vsaka dva elementa sta v relaciji), ki sta si medsebojno komplementarni.
		
		Poleg operacij, ki jih relacije podedujejo od podmnožic, imamo še operacije, ki upoštevajo produktno strukturo.
		
		Če so $X$, $Y$, $Z$ množice in $R \subseteq X \times Y$, $S \subseteq Y \times Z$ relaciji, tedaj je \df{sklop} (\df{kompozitum}) \df{relacij} definiran kot
		\[S \circ R \dfeq \set[1]{(x, z) \in X \times Z}{\some{y}[Y]{x \mathrel{R} y \land y \mathrel{S} z}}\]
		(po vzoru funkcij tudi kompozicijo relacij pišemo v obratnem vrstnem redu; glej razdelek~\ref{RAZDELEK: Funkcije kot funkcijske relacije}). Sklapljanje je asociativna operacija, torej pri sklopu večih relacij oklepaji niso pomembni.
		
		Večkraten sklop relacije $R \subseteq X \times X$ same s sabo označimo
		\[R^n \dfeq \underbrace{R \circ R \circ \ldots \circ R}_{\text{$n$ $R$-jev}}\]
		za $n \in \NN_{\geq 2}$. Seveda je smiselno definirati, da je $R^1$ enak $R$ in da je $R^0$ relacija enakosti na množici $X$, saj je to enota za sklapljanje relacij na $X$, tj.~$=_X \circ R = R = R \circ =_X$ (premisli, da je to res!).
		
		Za poljubno relacijo $R \subseteq X \times Y$ definiramo \df{obratno} (\df{inverzno}) \df{relacijo} kot
		\[R^{-1} \dfeq \set{(y, x) \in Y \times X}{x \mathrel{R} y}.\]
		Posledično lahko za poljubno relacijo $R \subseteq X \times X$ definiramo njeno potenco s poljubno celo stopnjo: $R^{-n} \dfeq (R^{-1})^n = (R^n)^{-1}$.
		
		\begin{zgled}
			Naj bo $L$ množica ljudi. Vpeljimo oznake za naslednje relacije na $L$:
			\begin{itemize}
				\item
					$\texttt{St}$ je relacija \qt{je starš od},
				\item
					$\texttt{Oč}$ je relacija \qt{je oče od},
				\item
					$\texttt{Ma}$ je relacija \qt{je mati od},
				\item
					$\texttt{Si}$ je relacija \qt{je sin od},
				\item
					$\texttt{Hč}$ je relacija \qt{je hči od},
				\item
					$\texttt{Br}$ je relacija \qt{je brat od},
				\item
					$\texttt{Se}$ je relacija \qt{je sestra od}
			\end{itemize}
			
			Na primer: Marko $\texttt{Br}$ Metka pomeni \qt{Marko je brat od Metke.} (oz.~v lepši slovenščini \qt{Marko je Metkin brat.}).
			
			Velja med drugim:
			
			\begin{tabular}{l}
				$\texttt{Oč} \cup \texttt{Ma} = \texttt{St}$, \\
				$\texttt{St} \circ \texttt{St} = \texttt{St}^2 = \text{\qt{je stari starš od}}$, \\
				$\texttt{St} \circ \texttt{Br} = \text{\qt{je stric od}}$, \\
				$\texttt{Br} \cup \texttt{Se} = \text{\qt{je sorojenec od}}$, \\
				$\texttt{St}^{-1} = \text{\qt{je otrok od}}$, \\
				$\bigcup_{n \in \NN_{\geq 1}} \texttt{St}^n = \text{\qt{je prednik od}}$, \\
				$\bigcup_{n \in \NN_{\geq 1}} \texttt{St}^{-n} = \text{\qt{je potomec od}}$, \\
				$\texttt{St} \circ (\texttt{Br} \cup \texttt{Se}) \circ \texttt{Hč} = \text{\qt{je sestrična od}}$.
			\end{tabular}
			
			Sklapljanje relacij ni komutativno; na primer $\texttt{Ma} \circ \texttt{Oč}$ je stari oče po materini strani, $\texttt{Oč} \circ \texttt{Ma}$ pa stara mama po očetovi strani.
			
			\note{V tem zgledu sicer predpostavljamo, da je vsaka oseba bodisi moškega bodisi ženskega spola, kar ni čisto res. Ima kdo kakšno idejo, kako se temu izogniti (in še vedno imeti lahko razumljiv zgled)?}
		\end{zgled}
	
	
	\section{Lastnosti relacij}
		\note{Med drugim lastnosti relacij, izražene z operacijami. Mogoče združimo s prejšnjim razdelkom?}
	\section{Funkcije kot funkcijske relacije}\label{RAZDELEK: Funkcije kot funkcijske relacije}
	\section{Relacije urejenosti}
		\note{Vključno z urejenostnimi strukturami. Vključno z morfizmi?}
	\section{Ekvivalenčne relacije}
	\section{Kvocientne množice}
		\note{Sem dodajmo kanonični razcep funkcije (na surjekcijo/kvocient, bijekcijo, injekcijo/vložitev).}

        \chapter{Strukturirane množice}
                \note{Struktura na množici. Morfizmi, ki to strukturo ohranjajo. Izomorfnost. Definicija strukturirane množice preko njene karakterizacije --- potrebna obstoj in enoličnost (do izomorfizma). Primeri. Posebej primeri struktur urejenosti (izhaja iz razdelka o strukturah urejenosti v poglavju o relacijah) in osnovnih algebrskih struktur (pride prav kasneje pri konstrukciji številskih množic). Urejenostna in algebrska struktura se združita v pojmu mreže. Definicija (polnih) Boolovih mrež/kolobarjev in povezava z logiko. Širša slika strukturiranih množic --- kategorije.}

        \chapter{Številske množice}
                \note{Karakterizacije številskih množic. Dokaz obstoja in enoličnosti (do izomorfizma).}
                \section{Naravna števila}
                \section{Cela števila}
                \section{Racionalna števila}
                \section{Realna števila}
                \section{Kompleksna števila}
                        \davorin{Se ustavimo že pri realnih številih? Gremo še dlje do kvaternionov?}

        \chapter{Indukcija}
                \note{Dobro osnovano urejene in dobro urejene množice. Indukcija na dobro osnovano urejenih množicah. Strukturna indukcija.}

        \chapter{Aksiomatska teorija množic}
                \section{Zermelo-Fraenklovi aksiomi}
                \section{Aksiom izbire}
                \section{Kumulativna hierarhija}

        \chapter{Kardinalna števila}
                \section{Končnost in neskončnost}
                \section{Števnost}
                \section{Kardinalnost množice}

        \chapter{Ordinalna števila}
                \davorin{Mogoče združimo kardinalna in ordinalna števila v eno poglavje?}


\end{document}