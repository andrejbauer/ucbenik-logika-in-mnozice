\documentclass[11pt,a4paper,twoside]{book}


%%%%%%%%%%%%%%%%%%%%%%%%%%%%%%%%%%%%%%%%%%%%%%%%%%%%%%%%%%%%%%%%%%%%%%%%%%%%%%%%%%%%%%%%%%%%%%%%%%%%%%%%%%%%%%%%%%%%%%
%%%  Imported Packages
%%%%%%%%%%%%%%%%%%%%%%%%%%%%%%%%%%%%%%%%%%%%%%%%%%%%%%%%%
	\usepackage[slovene]{babel}
	\usepackage[utf8]{inputenc}
	\usepackage[T1]{fontenc}
	
	\usepackage{url}
	\usepackage{ifthen}
	\usepackage{amssymb}
	\usepackage{amsmath}
	\usepackage{theorem}
	\usepackage{phonetic}
	\usepackage{tablefootnote}
	\usepackage{color}
	\usepackage{xparse}
	\usepackage{ulem}
	\usepackage{charter}


%%%%%%%%%%%%%%%%%%%%%%%%%%%%%%%%%%%%%%%%%%%%%%%%%%%%%%%%%%%%%%%%%%%%%%%%%%%%%%%%%%%%%%%%%%%%%%%%%%%%%%%%%%%%%%%%%%%%%%
%%%  Theorems etc.
%%%%%%%%%%%%%%%%%%%%%%%%%%%%%%%%%%%%%%%%%%%%%%%%%%%%%%%%%%%%%
	{
		\theorembodyfont{\itshape}

		\newtheorem{izrek}{Izrek}[section]
		\newtheorem{lema}[izrek]{Lema}
		\newtheorem{trditev}[izrek]{Trditev}
		\newtheorem{posledica}[izrek]{Posledica}
	}

	{
		\theorembodyfont{\rmfamily}
		\newtheorem{definicija}[izrek]{Definicija}
		\newtheorem{opomba}[izrek]{Opomba}
		\newtheorem{primer}[izrek]{Primer}
		\newtheorem{zgled}[izrek]{Zgled}
	}

%%%%%%  Proofs
%%%%%%%%%%%%%%%%%%%%%%%%%%%%%%%%%%%%%%%%%%%%%%%%%%%%%%%%%%%%%
	\newenvironment{dokaz}{
		\goodbreak\par
		\textit{Dokaz.}%
	}{%
		\nopagebreak
		\hfill{\vrule width 1ex height 1ex depth 0ex}
		\medskip
		\goodbreak
	}
%%%%%%%%%%%%%%%%%%%%%%%%%%%%%%%%%%%%%%%%%%%%%%%%%%%%%%%%%%%%%%%%%%%%%%%%%%%%%%%%%%%%%%%%%%%%%%%%%%%%%%%%%%%%%%%%%%%%%%






%%%%%%%%%%%%%%%%%%%%%%%%%%%%%%%%%%%%%%%%%%%%%%%%%%%%%%%%%%%%%%%%%%%%%%%%%%%%%%%%%%%%%%%%%%%%%%%%%%%%%%%%%%%%%%%%%%%%%%
%%%  Commands
%%%%%%%%%%%%%%%%%%%%%%%%%%%%%%%%%%%%%%%%%%%%%%%%%%%%%%%%%%%%%%%%%%%%%%%%%%%%%%%%%%%%%%%%%%%%%%%%%%%%%%%%%%%%%%%%%%%%%%


%%%%%%  Auxiliary
%%%%%%%%%%%%%%%%%%%%%%%%%%%%%%%%%%%%%%%%%%%%%%%%%%%%%%%%%%%%%
	\newcommand{\sizedescriptor}[2]
	{
		\ifthenelse{\equal{#1}{0}}{}{
		\ifthenelse{\equal{#1}{1}}{\big}{
		\ifthenelse{\equal{#1}{2}}{\Big}{
		\ifthenelse{\equal{#1}{3}}{\bigg}{
		\ifthenelse{\equal{#1}{4}}{\Bigg}{
		#2}}}}}
	}
	\newcommand{\proven}[1]{\underline{#1}\vspace{0.2em}\\}
	\newcommand{\note}[1]{{\small\textcolor{blue}{(#1)}}}
	\newcommand{\someref}{{\small\textcolor{blue}{[\textbf{ref.}]}}}


%%%%%%  Logical Quantifiers and λ-Terms (x = no parenthesis, u = untyped)
%%%%%%%%%%%%%%%%%%%%%%%%%%%%%%%%%%%%%%%%%%%%%%%%%%%%%%%%%%%%%
	\newcommand{\all}[4][auto]{\forall\, #2 \,{\in}\, #3\,.\sizedescriptor{#1}{\left}({#4}\sizedescriptor{#1}{\right})}
	\newcommand{\some}[4][auto]{\exists\, #2 \,{\in}\, #3\,.\sizedescriptor{#1}{\left}({#4}\sizedescriptor{#1}{\right})}
	\newcommand{\exactlyone}[4][auto]{\exists\;\!!\, #2 \,{\in}\, #3\,.\sizedescriptor{#1}{\left}({#4}\sizedescriptor{#1}{\right})}
	
	\newcommand{\uall}[3][auto]{\forall\, #2\,.\sizedescriptor{#1}{\left}({#3}\sizedescriptor{#1}{\right})}
	\newcommand{\usome}[3][auto]{\exists\, #2\,.\sizedescriptor{#1}{\left}({#3}\sizedescriptor{#1}{\right})}
	\newcommand{\uexactlyone}[3][auto]{\exists\;\!!\, #2\,.\sizedescriptor{#1}{\left}({#3}\sizedescriptor{#1}{\right})}
	
	\newcommand{\xall}[3]{\forall\, #1 \,{\in}\, #2\,.\,#3}
	\newcommand{\xsome}[3]{\exists\, #1 \,{\in}\, #2\,.\,#3}
	\newcommand{\xexactlyone}[3]{\exists\;\!!\, #1 \,{\in}\, #2\,.\,#3}
	
	\newcommand{\xuall}[2]{\forall\, #1\,.\,#2}
	\newcommand{\xusome}[2]{\exists\, #1\,.\,#2}
	\newcommand{\xuexactlyone}[2]{\exists\;\!!\, #1,.\,#2}

	\DeclareDocumentCommand{\lam}{O{auto} m O{\empty} m O{\empty}}
		{\lambda\, {#2} \ifthenelse{\equal{#3}{}}{}{\in{#3}} \,.\,  \sizedescriptor{#1}{\left}( {#4} \ifthenelse{\equal{#5}{}}{}{\in{#5}} \sizedescriptor{#1}{\right})}
	\DeclareDocumentCommand{\xlam}{m O{\empty} m O{\empty}}
		{\lambda\, {#1} \ifthenelse{\equal{#2}{}}{}{\in{#2}} \,.\, {#3} \ifthenelse{\equal{#4}{}}{}{\in{#4}}}


%%%%%%  Sets
%%%%%%%%%%%%%%%%%%%%%%%%%%%%%%%%%%%%%%%%%%%%%%%%%%%%%%%%%%%%%
	%  \set{1, 2, 3}         ->  {1, 2, 3}
	%  \set{a \in X}{a < 1}  ->  {a ∈ X | a < 1}
	\DeclareDocumentCommand{\set}{O{auto} m G{\empty}}{ \sizedescriptor{#1}{\left} \{{#2} \ifthenelse{\equal{#3}{}}{}{ \; \sizedescriptor{#1}{\middle}| \; {#3}} \sizedescriptor{#1}{\right}\} }
	\newcommand{\vsubset}{\Mapstochar\cap}
	\newcommand{\finseq}[1]{{#1}^*}
	\newcommand{\pst}{\mathcal{P}}


%%%%%%  Number Sets, Intervals
%%%%%%%%%%%%%%%%%%%%%%%%%%%%%%%%%%%%%%%%%%%%%%%%%%%%%%%%%%%%%
	\newcommand{\NN}{\mathbb{N}}
	\newcommand{\ZZ}{\mathbb{Z}}
	\newcommand{\QQ}{\mathbb{Q}}
	\newcommand{\RR}{\mathbb{R}}
	\newcommand{\CC}{\mathbb{C}}
	\newcommand{\intoo}[3][\RR]{{#1}_{(#2, #3)}}
	\newcommand{\intcc}[3][\RR]{{#1}_{[#2, #3]}}
	\newcommand{\intoc}[3][\RR]{{#1}_{(#2, #3]}}
	\newcommand{\intco}[3][\RR]{{#1}_{[#2, #3)}}


%%%%%%  Misc.
%%%%%%%%%%%%%%%%%%%%%%%%%%%%%%%%%%%%%%%%%%%%%%%%%%%%%%%%%%%%%
	\newcommand{\intermission}{\bigskip\medskip}
	\newcommand{\df}[1]{\emph{\textbf{#1}}}  % defined notion
	\newcommand{\ism}{\cong}  % isomorphic
	\newcommand{\equ}{\sim}  % equivalent
	\newcommand{\dfeq}{\mathrel{\mathop:}=}  % definitional equality
	\newcommand{\dfeqrev}{=\mathrel{\mathop:}}  % reversed definitional equality
	\newcommand{\id}[1][]{\textrm{Id}_{#1}}  % identity map
	\newcommand{\impl}{\Rightarrow}  % implication sign
	\newcommand{\revimpl}{\Leftarrow}  % reverse implication sign
	\newcommand{\lequ}{\Leftrightarrow}  % equivalence sign
	\newcommand{\xor}{\mathbin{\veebar}}  % exclusive disjunction sign
	\newcommand{\shf}{\mathbin{\uparrow}}  % Sheffer connective
	\newcommand{\luk}{\mathbin{\downarrow}}  % Łukasiewicz connective
	\newcommand{\rstr}[1]{\left.{#1}\right|}  % map restriction
	\newcommand{\im}{\mathrm{im}}  % map image
	\newcommand{\parto}{\mathrel{\rightharpoonup}}  % partial mapping sign
	\newcommand{\qt}[1]{{\quotedblbase}{#1}{‘‘}}  % text in quotation marks
	\newcommand{\nls}[1]{\qt{\textit{#1}}}  % sentence in a natural language
	

%%%%%%%%%%%%%%%%%%%%%%%%%%%%%%%%%%%%%%%%%%%%%%%%%%%%%%%%%%%%%%%%%%%%%%%%%%%%%%%%%%%%%%%%%%%%%%%%%%%%%%%%%%%%%%%%%%%%%%






%%%%%%%%%%%%%%%%%%%%%%%%%%%%%%%%%%%%%%%%%%%%%%%%%%%%%%%%%%%%%%%%%%%%%%%%%%%%%%%%%%%%%%%%%%%%%%%%%%%%%%%%%%%%%%%%%%%%%%
%%  Page Style & Margins (A4 page = 210mm x 297mm)

\setlength{\textwidth}{15cm}
\setlength{\textheight}{224mm}

\setlength{\topmargin}{0cm}
\setlength{\evensidemargin}{0cm}
\setlength{\oddsidemargin}{\paperwidth}
\addtolength{\oddsidemargin}{-\textwidth}
\addtolength{\oddsidemargin}{-2in}

\renewcommand{\baselinestretch}{1.25}
\setlength{\parskip}{1.5ex}


%%%%%%%%%%%%%%%%%%%%%%%%%%%%%%%%%%%%%%%%%%%%%%%%%%%%%%%%%%%%%%%%%%%%%%%%%%%%%%%%%%%%%%%%%%%%%%%%%%%%%%%%%%%%%%%%%%%%%%






\begin{document}

   %--------------------------------------------------------------------
   %--------------------------------------------------------------------
   % TITLE PAGE
   
	
	\title{\Huge \textbf{\textsc{Logika in množice}}}
	\author{A.~Bauer, D.~Lešnik, M.~Petkovšek, M.~Pretnar}
	
	\maketitle
   
   
   %--------------------------------------------------------------------
   %--------------------------------------------------------------------
   % Foreword

   \chapter*{Predgovor}%\addcontentsline{toc}{chapter}{\numberline{}Predgovor}


	%--------------------------------------------------------------------
	%--------------------------------------------------------------------
	% TOC
	
	
	\addcontentsline{toc}{chapter}{\note{kazalo se naj začne na sodi strani, tako da lahko bralec naenkrat vidi celotno kazalo}}
	\tableofcontents
	
	
	%--------------------------------------------------------------------
	%--------------------------------------------------------------------
	% BODY
	
	
	\chapter{Matematično izražanje}

	\section{Simbolni zapis}
	\section{Pravila dokazovanja}
	\section{Definicije}
	
	\chapter{Konstrukcije množic}
		\section{Preprosti primeri}
			\note{prazna množica, enojci}
		\section{Podmnožice}
		\section{Potenčna množica}
		\section{Družine množic}
		\section{Produkt množic}
		\section{Vsota množic}
		\section{Unija in presek}
		\section{Eksponentna množica}
	
	\chapter{Funkcije}
		\note{Moramo se dogovoriti, kateremu izrazu bomo dali prednost --- `funkcija' ali `preslikava'.}
		\section{Slike in praslike}
		\section{Injektivnost in surjektivnost}
			\note{Vključno z ekvivalenco z mono- in epimorfizmi.}
		\section{Bijektivnost}
			\note{Pomemben del tega razdelka bodo inverzne funkcije. Mogoče lahko to dodamo v naslov.}
	
	\chapter{Relacije}
		\section{Operacije z relacijami}
		\section{Lastnosti relacij}
			\note{Med drugim lastnosti relacij, izražene z operacijami. Mogoče združimo ta dva razdelka?}
		\section{Funkcije kot funkcijske relacije}
		\section{Relacije urejenosti}
			\note{Vključno z urejenostnimi strukturami. Vključno z morfizmi?}
		\section{Ekvivalenčne relacije}
		\section{Kvocientne množice}
			\note{Sem dodajmo kanonični razcep funkcije (na surjekcijo/kvocient, bijekcijo, injekcijo/vložitev).}
	
	\chapter{Aksiomatska teorija množic}
		\section{Zermelo-Fraenklovi aksiomi}
		\section{Aksiom izbire}
		\section{Kumulativna hierarhija}
	
	\chapter{Kardinalna števila}
		\section{Končnost in neskončnost}
		\section{Števnost}
		\section{Kardinalnost množice}
	
	\chapter{Ordinalna števila}
	
	\chapter{\note{možne dodatne teme}}
		\begin{itemize}
			\item
				Več o ZFC
			\item
				Strukturirane množice in njihovi morfizmi
			\item
				Kategorije
			\item
				Številske množice (med drugim aksiom o neskončnosti, Peanovi aksiomi in debata, kako definiramo strukturirano množico preko njene karakterizacije, če obenem dokažemo obstoj in enoličnost (do izomorfizma))
		\end{itemize}
	
	
\end{document}