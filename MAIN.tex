\documentclass[11pt,a4paper,twoside]{book}


%%%%%%%%%%%%%%%%%%%%%%%%%%%%%%%%%%%%%%%%%%%%%%%%%%%%%%%%%%%%%%%%%%%%%%%%%%%%%%%%%%%%%%%%%%%%%%%%%%%%%%%%%%%%%%%%%%%%%%
%%%  Imported Packages
%%%%%%%%%%%%%%%%%%%%%%%%%%%%%%%%%%%%%%%%%%%%%%%%%%%%%%%%%
	\usepackage[slovene]{babel}
	\usepackage[utf8]{inputenc}
	\usepackage[T1]{fontenc}
	
	\usepackage{url}
	\usepackage{ifthen}
	\usepackage{amssymb}
	\usepackage{amsmath}
	\usepackage{theorem}
	\usepackage{phonetic}
	\usepackage{tablefootnote}
	\usepackage{color}
	%\usepackage[most]{tcolorbox}
	\usepackage{tkz-graph}
	\usepackage{xparse}
	\usepackage{mathrsfs}
	\usepackage{ulem}
	\usepackage{charter}


%%%%%%%%%%%%%%%%%%%%%%%%%%%%%%%%%%%%%%%%%%%%%%%%%%%%%%%%%%%%%%%%%%%%%%%%%%%%%%%%%%%%%%%%%%%%%%%%%%%%%%%%%%%%%%%%%%%%%%
%%%  Theorems etc.
%%%%%%%%%%%%%%%%%%%%%%%%%%%%%%%%%%%%%%%%%%%%%%%%%%%%%%%%%%%%%
	{
		\theorembodyfont{\itshape}

		\newtheorem{izrek}{Izrek}[section]
		\newtheorem{lema}[izrek]{Lema}
		\newtheorem{trditev}[izrek]{Trditev}
		\newtheorem{posledica}[izrek]{Posledica}
	}

	{
		\theorembodyfont{\rmfamily}
		\newtheorem{definicija}[izrek]{Definicija}
		\newtheorem{opomba}[izrek]{Opomba}
		\newtheorem{primer}[izrek]{Primer}
		\newtheorem{zgled}[izrek]{Zgled}
		\newtheorem{vaja}[izrek]{Vaja}
	}

%%%%%%  Proofs
%%%%%%%%%%%%%%%%%%%%%%%%%%%%%%%%%%%%%%%%%%%%%%%%%%%%%%%%%%%%%
	\newenvironment{dokaz}{
		\goodbreak\par
		\textit{Dokaz.}%
	}{%
		\nopagebreak
		\hfill{\vrule width 1ex height 1ex depth 0ex}
		\medskip
		\goodbreak
	}
%%%%%%%%%%%%%%%%%%%%%%%%%%%%%%%%%%%%%%%%%%%%%%%%%%%%%%%%%%%%%%%%%%%%%%%%%%%%%%%%%%%%%%%%%%%%%%%%%%%%%%%%%%%%%%%%%%%%%%






%%%%%%%%%%%%%%%%%%%%%%%%%%%%%%%%%%%%%%%%%%%%%%%%%%%%%%%%%%%%%%%%%%%%%%%%%%%%%%%%%%%%%%%%%%%%%%%%%%%%%%%%%%%%%%%%%%%%%%
%%%  Commands
%%%%%%%%%%%%%%%%%%%%%%%%%%%%%%%%%%%%%%%%%%%%%%%%%%%%%%%%%%%%%%%%%%%%%%%%%%%%%%%%%%%%%%%%%%%%%%%%%%%%%%%%%%%%%%%%%%%%%%


%%%%%%  Auxiliary
%%%%%%%%%%%%%%%%%%%%%%%%%%%%%%%%%%%%%%%%%%%%%%%%%%%%%%%%%%%%%
	\newcommand{\sizedescriptor}[2]
	{
		\ifthenelse{\equal{#1}{0}}{}{
		\ifthenelse{\equal{#1}{1}}{\big}{
		\ifthenelse{\equal{#1}{2}}{\Big}{
		\ifthenelse{\equal{#1}{3}}{\bigg}{
		\ifthenelse{\equal{#1}{4}}{\Bigg}{
		#2}}}}}
	}
	\newcommand{\proven}[1]{\underline{#1}\vspace{0.2em}\\}
	\newcommand{\someref}{{\small\textcolor{blue}{[\textbf{ref.}]}}}
	
	\definecolor{andrejcolor}{rgb}{0.7,0,0.7}
	\definecolor{davorincolor}{rgb}{0,0.45,1}
	\definecolor{markocolor}{rgb}{0.7,0.4,0}
	\definecolor{matijacolor}{rgb}{0,0.6,0.4}
	
	\newcommand{\andrej}[1]{{\small\textcolor{andrejcolor}{(#1 \ \mbox{--Andrej})}}}
	\newcommand{\davorin}[1]{{\small\textcolor{davorincolor}{(#1 \ \mbox{--Davorin})}}}
	\newcommand{\marko}[1]{{\small\textcolor{markocolor}{(#1 \ \mbox{--Marko})}}}
	\newcommand{\matija}[1]{{\small\textcolor{matijacolor}{(#1 \ \mbox{--Matija})}}}
	
	\definecolor{notecolor}{rgb}{0.6,0.5,0.7}
	\newcommand{\note}[1]{{\small\textcolor{notecolor}{(#1)}}}
	\newcommand{\alert}[1]{{\small\textcolor{red}{\textbf{#1}}}}


%%%%%%  Logical Quantifiers and λ-Terms
%%%%%%%%%%%%%%%%%%%%%%%%%%%%%%%%%%%%%%%%%%%%%%%%%%%%%%%%%%%%%
	
	%  x = no parenthesis
	\NewDocumentCommand{\xall}
		{m O{\empty} m}
		{\forall\, {#1} \ifthenelse{\equal{#2}{}}{}{\in{#2}} \,.\, {#3}}
	\NewDocumentCommand{\xsome}
		{m O{\empty} m}
		{\exists\, {#1} \ifthenelse{\equal{#2}{}}{}{\in{#2}} \,.\, {#3}}
	\NewDocumentCommand{\xexactlyone}
		{m O{\empty} m}
		{\exists\;\!!\, {#1} \ifthenelse{\equal{#2}{}}{}{\in{#2}} \,.\, {#3}}
	\NewDocumentCommand{\xlam}
		{m O{\empty} m O{\empty}}
		{\lambda\, {#1} \ifthenelse{\equal{#2}{}}{}{\in{#2}} \,.\, {#3} \ifthenelse{\equal{#4}{}}{}{\in{#4}}}
	
	%  with parenthesis -- the first optional argument is the size (values 0-4)
	\NewDocumentCommand{\all}
		{O{auto} m O{\empty} m}
		{\xall{#2}[#3]{\sizedescriptor{#1}{\left}( {#4} \sizedescriptor{#1}{\right})}}
	\NewDocumentCommand{\some}
		{O{auto} m O{\empty} m}
		{\xsome{#2}[#3]{\sizedescriptor{#1}{\left}( {#4} \sizedescriptor{#1}{\right})}}
	\NewDocumentCommand{\exactlyone}
		{O{auto} m O{\empty} m}
		{\xexactlyone{#2}[#3]{\sizedescriptor{#1}{\left}( {#4} \sizedescriptor{#1}{\right})}}
	\NewDocumentCommand{\lam}
		{O{auto} m O{\empty} m O{\empty}}
		{\xlam{#2}[#3]{\sizedescriptor{#1}{\left}( {#4} \sizedescriptor{#1}{\right})}[#5]}


%%%%%%  Sets
%%%%%%%%%%%%%%%%%%%%%%%%%%%%%%%%%%%%%%%%%%%%%%%%%%%%%%%%%%%%%
	%  \set{1, 2, 3}         ->  {1, 2, 3}
	%  \set{a \in X}{a < 1}  ->  {a ∈ X | a < 1}
	\NewDocumentCommand{\set}
		{O{auto} m G{\empty}}
		{\sizedescriptor{#1}{\left}\{ {#2} \ifthenelse{\equal{#3}{}}{}{ \; \sizedescriptor{#1}{\middle}| \; {#3}} \sizedescriptor{#1}{\right}\}}
	%\newcommand{\vsubset}{\Mapstochar\cap}
	%\newcommand{\finseq}[1]{{#1}^*}
	\newcommand{\pst}{\mathcal{P}}
	\renewcommand{\complement}[1]{{#1}^C}


%%%%%%  Number Sets, Intervals
%%%%%%%%%%%%%%%%%%%%%%%%%%%%%%%%%%%%%%%%%%%%%%%%%%%%%%%%%%%%%
	\newcommand{\NN}{\mathbb{N}}
	\newcommand{\ZZ}{\mathbb{Z}}
	\newcommand{\QQ}{\mathbb{Q}}
	\newcommand{\RR}{\mathbb{R}}
	\newcommand{\CC}{\mathbb{C}}
	\newcommand{\intoo}[3][\RR]{{#1}_{(#2, #3)}}
	\newcommand{\intcc}[3][\RR]{{#1}_{[#2, #3]}}
	\newcommand{\intoc}[3][\RR]{{#1}_{(#2, #3]}}
	\newcommand{\intco}[3][\RR]{{#1}_{[#2, #3)}}


%%%%%%  Misc.
%%%%%%%%%%%%%%%%%%%%%%%%%%%%%%%%%%%%%%%%%%%%%%%%%%%%%%%%%%%%%
	\newcommand{\intermission}{\bigskip\medskip}
	\newcommand{\df}[1]{\textit{\textbf{#1}}}  % defined notion
	\newcommand{\ism}{\cong}  % isomorphic
	\newcommand{\equ}{\sim}  % equivalent
	\newcommand{\dfeq}{\mathrel{\mathop:}=}  % definitional equality
	\newcommand{\dfeqrev}{=\mathrel{\mathop:}}  % reversed definitional equality
	\newcommand{\id}[1][]{\textrm{Id}_{#1}}  % identity map
	\newcommand{\impl}{\Rightarrow}  % implication sign
	\newcommand{\revimpl}{\Leftarrow}  % reverse implication sign
	\newcommand{\lequ}{\Leftrightarrow}  % equivalence sign
	\newcommand{\xor}{\mathbin{\veebar}}  % exclusive disjunction sign
	\newcommand{\shf}{\mathbin{\uparrow}}  % Sheffer connective
	\newcommand{\luk}{\mathbin{\downarrow}}  % Łukasiewicz connective
	\newcommand{\rstr}[1]{\left.{#1}\right|}  % map restriction
	\newcommand{\im}{\mathrm{im}}  % map image
	\newcommand{\parto}{\mathrel{\rightharpoonup}}  % partial mapping sign
	\newcommand{\qt}[1]{{\quotedblbase}{#1}{‘‘}}  % text in quotation marks
	\newcommand{\nls}[1]{\qt{\textit{#1}}}  % sentence in a natural language
	\NewDocumentCommand{\rel}
		{O{\empty} O{\empty}}
		{\ifthenelse{\equal{#1}{}}{\mathscr{R}}{{#1} \mathrel{\mathscr{R}} {#2}}}  % a relation
	\NewDocumentCommand{\srel}
		{O{\empty} O{\empty}}
		{\ifthenelse{\equal{#1}{}}{\mathscr{S}}{{#1} \mathrel{\mathscr{S}} {#2}}}  % a second relation
	\newcommand{\dom}{\mathrm{dom}}  % domain
	\newcommand{\cod}{\mathrm{cod}}  % codomain
	\newcommand{\dd}[1]{D_{#1}}  % domain of definition
	\newcommand{\rn}[1]{Z_{#1}}  % range
	\newcommand{\graph}[1]{\Gamma_{#1}}  % graph of a (partial) function
	\newcommand{\isdefined}[1]{{#1}\!\downarrow}  % given value is defined
	\newcommand{\kleq}{\simeq}  % Kleene equality
	\newcommand{\ec}[2][]{[\:\!{#2}\:\!]_{#1}}  % equivalence class
	

%%%%%%%%%%%%%%%%%%%%%%%%%%%%%%%%%%%%%%%%%%%%%%%%%%%%%%%%%%%%%%%%%%%%%%%%%%%%%%%%%%%%%%%%%%%%%%%%%%%%%%%%%%%%%%%%%%%%%%






%%%%%%%%%%%%%%%%%%%%%%%%%%%%%%%%%%%%%%%%%%%%%%%%%%%%%%%%%%%%%%%%%%%%%%%%%%%%%%%%%%%%%%%%%%%%%%%%%%%%%%%%%%%%%%%%%%%%%%
%%  Page Style & Margins (A4 page = 210mm x 297mm)

\setlength{\textwidth}{15cm}
\setlength{\textheight}{224mm}

\setlength{\topmargin}{0cm}
\setlength{\evensidemargin}{0cm}
\setlength{\oddsidemargin}{\paperwidth}
\addtolength{\oddsidemargin}{-\textwidth}
\addtolength{\oddsidemargin}{-2in}

\renewcommand{\baselinestretch}{1.25}
\setlength{\parskip}{1.5ex}


%%%%%%%%%%%%%%%%%%%%%%%%%%%%%%%%%%%%%%%%%%%%%%%%%%%%%%%%%%%%%%%%%%%%%%%%%%%%%%%%%%%%%%%%%%%%%%%%%%%%%%%%%%%%%%%%%%%%%%






\begin{document}

   %--------------------------------------------------------------------
   %--------------------------------------------------------------------
   % TITLE PAGE
   
	
	\title{\Huge \textbf{\textsc{Logika in množice}}}
	\author{A.~Bauer, D.~Lešnik, M.~Petkovšek, M.~Pretnar}
	
	\maketitle
   
   
   %--------------------------------------------------------------------
   %--------------------------------------------------------------------
   % Foreword

   \chapter*{Predgovor}%\addcontentsline{toc}{chapter}{\numberline{}Predgovor}


	%--------------------------------------------------------------------
	%--------------------------------------------------------------------
	% TOC
	
	
	\addcontentsline{toc}{chapter}{\alert{Kazalo se naj začne na sodi strani, tako da lahko bralec naenkrat vidi celotno kazalo.}}
	\tableofcontents
	
	
	%--------------------------------------------------------------------
	%--------------------------------------------------------------------
	% BODY
	
	
	\chapter{Matematično izražanje}

	\section{Simbolni zapis}
	\section{Pravila dokazovanja}
	\section{Definicije}
	
	\chapter{Konstrukcije množic}
		\section{Preprosti primeri}
			\note{prazna množica, enojci}
		\section{Podmnožice}
		\section{Potenčna množica}
		\section{Družine množic}
		\section{Produkt množic}
		\section{Vsota množic}
		\section{Unija in presek}
		\section{Eksponentna množica}
	
	\chapter{Funkcije}
		\davorin{Moramo se dogovoriti, kateremu izrazu bomo dali prednost --- `funkcija' ali `preslikava'.}
		\section{Slike in praslike}
			\davorin{Kako jih označevati? Bomo sliko označevali isto kot funkcijo samo (kot je običaj) in se zanašamo, da bodo študenti ločili, kdaj gre za $f\colon X \to Y$ in kdaj za $f\colon \pst(X) \to \pst(Y)$? Podobno, bomo prasliko označevali isto kot inverz in upali, da bodo ločili med $f^{-1}\colon Y \to X$ in $f^{-1}\colon \pst(Y) \to \pst(X)$? Ali uvedemo posebne oznake za sliko in prasliko, recimo $f_*$ in $f^*$?}
		\section{Injektivnost in surjektivnost}
			\note{Vključno z ekvivalenco z mono- in epimorfizmi.}
		\section{Bijektivnost}
			\davorin{Pomemben del tega razdelka bodo inverzne funkcije. Mogoče lahko to dodamo v naslov.}
			
			\davorin{Pri inverzih omenimo sledeče. Pogosto obravnavamo funkcijo, ki izhaja iz nekega konkretnega (na primer fizikalnega) problema, v smislu, da funkcija vzame začetne podatke in nam vrne, kaj se bo na koncu zgodilo. Marsikdaj pa hočemo rešiti obraten problem: želimo določene končne rezultate in se sprašujemo, kaj morajo biti začetni pogoji, da jih bomo dosegli. V tem primeru pridejo prav inverzi.}
	
	\chapter{Relacije}\label{POGLAVJE: Relacije}

	V matematiki pogosto želimo izraziti, da so določeni objekti v nekem odnosu, npr.~eno število je večje od drugega; temu s tujko rečemo \df{relacija}. Kako to formalno izraziti? Ideja je, da relacijo podamo z množico vseh skupin elementov, ki so v relaciji. Na primer, relacijo $\leq$ na naravnih številih podamo kot podmnožico
	\[\set[1]{(a, b) \in \NN \times \NN}{\xsome{n}[\NN]{a + n = b}}.\]
	Torej, število $a$ je v relaciji $\leq$ s številom $b$ takrat, ko par $(a, b)$ pripada tej množici.
	
	Splošne relacije so lahko med poljubno mnogo elementi iz poljubnih (ne nujno istih) množic. Na primer, relacija komplanarnosti štirih točk v prostoru je podmnožica produkta $\RR^3 \times \RR^3 \times \RR^3 \times \RR^3$, relacija pripadnosti $\in$ med elementi neke množice $X$ in podmnožicami množice $X$ pa je podmnožica produkta $X \times \pst(X)$.
	
	Splošna definicija relacije je potemtakem naslednja.
	\begin{definicija}
		\df{Relacija} na družini množic $\mathscr{D}$ je podmnožica produkta $\prod_{X \in \mathscr{D}} X$.
	\end{definicija}
	
	V praksi se povečini uporabljajo relacije med dvema elementoma.
	\begin{definicija}
		\df{Dvojiška relacija}\footnote{Oziroma s tujko \df{binarna relacija}.} med elementi množic $X$ in $Y$ je podmnožica produkta $X \times Y$. \df{Dvojiška relacija} na množici $X$ je podmnožica produkta $X \times X$.
	\end{definicija}
	
	Skoraj vse relacije, ki nas zanimajo v tej knjigi, so dvojiške. Zato se dogovorimo, da z izrazom \qt{relacija} vselej mislimo dvojiško relacijo, razen če je izrecno rečeno drugače.
	
	Če je $R \subseteq X \times Y$ relacija, potemtakem lahko zapišemo, da sta $x \in X$ in $y \in Y$ v relaciji $R$ takole: $(x, y) \in R$. Ampak to vodi do čudnih zapisov v primeru običajnih relacij, npr.~$(2, 3) \in <$. To seveda raje zapišemo kot $2 < 3$ in posledično se dogovorimo, da v primeru dvojiške relacije raje uporabljamo zapis $x \mathrel{R} y$.
	
	
	\section{Grafi relacij}
	
		\GraphInit[vstyle = Normal]
		\tikzset
		{
			EdgeStyle/.append style = {->, bend left}
		}
		
		Relacije na majhnih množicah lahko lepo ponazorimo z usmerjenimi grafi. Graf relacije $R \subseteq X \times X$ je definiran takole: vozlišča grafa so elementi množice $X$ in za vsaka dva elementa $a, b \in X$, za katera velja $a \mathrel{R} b$, narišemo puščico od $a$ do $b$.
		
		\begin{zgled}
			Naj bo $X = \set{A, B, C, D, E, F}$ in naj bo
			\[R \dfeq \set{...}\]
			relacija na $X$. Njen graf izgleda takole.
			\begin{center}
				\begin{tikzpicture}
					\SetGraphUnit{3}
					\Vertex[Math=true, x=0, y=0]{A}
					\Vertex[Math=true, x=3, y=2]{B}
					\Vertex[Math=true, x=2, y=-3]{C}
					\Vertex[Math=true, x=6, y=1]{D}
					\Vertex[Math=true, x=8, y=-1]{E}
					\Vertex[Math=true, x=10, y=2]{F}
					
					\Edge(A)(B)
					\Loop[dist = 5em, dir = EA](B)
				\end{tikzpicture}
			\end{center}
		\end{zgled}
		\note{izgled grafa je še treba popraviti}
	
	
	\section{Operacije z relacijami}\label{RAZDELEK: Operacije z relacijami}
	
		Običajno je, da iz že danih matematičnih objektov lahko skonstruiramo nove preko določenih operacij. Z relacijami ni nič drugače; v tem razdelku si bomo ogledali običajne operacije na relacijah.
		
		Ker so relacije podmnožice, imamo za začetek vse operacije na podmnožicah. Torej, za poljubno družino $(R_i)_{i \in I}$ podmnožic produkta $X \times Y$ sta tudi unija $\bigcup_{i \in I} R_i$ in presek $\bigcap_{i \in I} R_i$ relaciji. Če je $R \subseteq X \times Y$ relacija, je njena komplementarna relacija $\complement{R} = X \times Y \setminus R \ \subseteq \ X \times Y$.
		
		Posebej imamo \df{prazno relacijo} $\emptyset \subseteq X \times Y$ (nobena dva elementa nista v relaciji) in \df{polno relacijo} $X \times Y\subseteq X \times Y$ (vsaka dva elementa sta v relaciji), ki sta si medsebojno komplementarni.
		
		Poleg operacij, ki jih relacije podedujejo od podmnožic, imamo še operacije, ki upoštevajo produktno strukturo.
		
		Če so $X$, $Y$, $Z$ množice in $R \subseteq X \times Y$, $S \subseteq Y \times Z$ relaciji, tedaj je \df{sklop} (\df{kompozitum}) \df{relacij} definiran kot
		\[S \circ R \dfeq \set[1]{(x, z) \in X \times Z}{\some{y}[Y]{x \mathrel{R} y \land y \mathrel{S} z}}\]
		(po vzoru funkcij tudi kompozicijo relacij pišemo v obratnem vrstnem redu; glej razdelek~\ref{RAZDELEK: Funkcije kot funkcijske relacije}). Sklapljanje je asociativna operacija, torej pri sklopu večih relacij oklepaji niso pomembni.
		
		Večkraten sklop relacije $R \subseteq X \times X$ same s sabo označimo
		\[R^n \dfeq \underbrace{R \circ R \circ \ldots \circ R}_{\text{$n$ $R$-jev}}\]
		za $n \in \NN_{\geq 2}$. Seveda je smiselno definirati, da je $R^1$ enak $R$ in da je $R^0$ relacija enakosti na množici $X$, saj je to enota za sklapljanje relacij na $X$, tj.~$=_X \circ R = R = R \circ =_X$ (premisli, da je to res!).
		
		Za poljubno relacijo $R \subseteq X \times Y$ definiramo \df{obratno} (\df{inverzno}) \df{relacijo} kot
		\[R^{-1} \dfeq \set{(y, x) \in Y \times X}{x \mathrel{R} y}.\]
		Posledično lahko za poljubno relacijo $R \subseteq X \times X$ definiramo njeno potenco s poljubno celo stopnjo: $R^{-n} \dfeq (R^{-1})^n = (R^n)^{-1}$.
		
		\begin{zgled}
			Naj bo $L$ množica ljudi. Vpeljimo oznake za naslednje relacije na $L$:
			\begin{itemize}
				\item
					$\texttt{St}$ je relacija \qt{je starš od},
				\item
					$\texttt{Oč}$ je relacija \qt{je oče od},
				\item
					$\texttt{Ma}$ je relacija \qt{je mati od},
				\item
					$\texttt{Si}$ je relacija \qt{je sin od},
				\item
					$\texttt{Hč}$ je relacija \qt{je hči od},
				\item
					$\texttt{Br}$ je relacija \qt{je brat od},
				\item
					$\texttt{Se}$ je relacija \qt{je sestra od}
			\end{itemize}
			
			Na primer: Marko $\texttt{Br}$ Metka pomeni \qt{Marko je brat od Metke.} (oz.~v lepši slovenščini \qt{Marko je Metkin brat.}).
			
			Velja med drugim:
			
			\begin{tabular}{l}
				$\texttt{Oč} \cup \texttt{Ma} = \texttt{St}$, \\
				$\texttt{St} \circ \texttt{St} = \texttt{St}^2 = \text{\qt{je stari starš od}}$, \\
				$\texttt{St} \circ \texttt{Br} = \text{\qt{je stric od}}$, \\
				$\texttt{Br} \cup \texttt{Se} = \text{\qt{je sorojenec od}}$, \\
				$\texttt{St}^{-1} = \text{\qt{je otrok od}}$, \\
				$\bigcup_{n \in \NN_{\geq 1}} \texttt{St}^n = \text{\qt{je prednik od}}$, \\
				$\bigcup_{n \in \NN_{\geq 1}} \texttt{St}^{-n} = \text{\qt{je potomec od}}$, \\
				$\texttt{St} \circ (\texttt{Br} \cup \texttt{Se}) \circ \texttt{Hč} = \text{\qt{je sestrična od}}$.
			\end{tabular}
			
			Sklapljanje relacij ni komutativno; na primer $\texttt{Ma} \circ \texttt{Oč}$ je stari oče po materini strani, $\texttt{Oč} \circ \texttt{Ma}$ pa stara mama po očetovi strani.
			
			\note{V tem zgledu sicer predpostavljamo, da je vsaka oseba bodisi moškega bodisi ženskega spola, kar ni čisto res. Ima kdo kakšno idejo, kako se temu izogniti (in še vedno imeti lahko razumljiv zgled)?}
		\end{zgled}
	
	
	\section{Lastnosti relacij}
		\note{Med drugim lastnosti relacij, izražene z operacijami. Mogoče združimo s prejšnjim razdelkom?}
	\section{Funkcije kot funkcijske relacije}\label{RAZDELEK: Funkcije kot funkcijske relacije}
	\section{Relacije urejenosti}
		\note{Vključno z urejenostnimi strukturami. Vključno z morfizmi?}
	\section{Ekvivalenčne relacije}
	\section{Kvocientne množice}
		\note{Sem dodajmo kanonični razcep funkcije (na surjekcijo/kvocient, bijekcijo, injekcijo/vložitev).}
	
	\chapter{Aksiomatska teorija množic}
		\section{Zermelo-Fraenklovi aksiomi}
		\section{Aksiom izbire}
		\section{Kumulativna hierarhija}
	
	\chapter{Kardinalna števila}
		\section{Končnost in neskončnost}
		\section{Števnost}
		\section{Kardinalnost množice}
	
	\chapter{Ordinalna števila}
	
	\chapter{\note{možne dodatne teme}}
		\begin{itemize}
			\item
				Več o ZFC
			\item
				Strukturirane množice in njihovi morfizmi
			\item
				Kategorije
			\item
				Številske množice (med drugim aksiom o neskončnosti, Peanovi aksiomi in debata, kako definiramo strukturirano množico preko njene karakterizacije, če obenem dokažemo obstoj in enoličnost (do izomorfizma))
		\end{itemize}
	
	
\end{document}