\chapter{Indukcija}

\davorin{Zaenkrat bom \qt{well-founded} prevajal kot \qt{dobro osnovan}, ker je ta izraz uporabljal Marko pri LMN. Ampak dobro bi bilo, da bi ta prevod še predebatirali.}

V razdelku~\note{o naravnih številih} smo videli, kako lahko z matematično indukcijo dokazujemo univerzalno kvantificirane izjave na naravnih številih. Pravzaprav smo si ogledali indukcijo v dveh oblikah. V prvem primeru smo morali pokazati temelj indukcije in indukcijski korak s števila na njegovega naslednika. V drugem primeru smo imeli samo indukcijski korak, pri katerem pa smo sklepali na veljavnost izjave za neki element z veljavnosti izjave za vse prejšnje elemente.

Ideja indukcije pa je mnogo globlja in splošnejša, kot omenjeni primer na naravnih številih.

\section{Dobro osnovano urejene množice}

Na kratko rečeno, dobro osnovano urejene množice so množice s toliko strukture, da lahko na njih izvajamo indukcijo. Natančna definicija je sledeča.

\begin{definicija}
	Naj bo $X$ množica in $\ll$ relacija na njej.
	\begin{itemize}
		\item
			Za predikat $\phi$ na $X$ rečemo, da je \df{$\ll$-induktiven}, kadar za vsak $a \in X$ velja: če $\phi$ velja za vse $x \in X_{\ll a}$, tedaj velja tudi za $a$. Simbolno zapisano:
			\[\all[2]{a}[X]{\xall{x}[X_{\ll a}]{\phi(x)} \implies \phi(a)}.\]
		\item
			Množica $X$, skupaj z relacijo $\ll$, je \df{dobro osnovano urejena}, kadar je posod resničen predikat edini $\ll$-induktiven predikat.
	\end{itemize}
\end{definicija}

Torej: če želimo dokazati, da neka lastnost $\phi$ velja za vse elemente dobro osnovano urejene množice, dokažemo indukcijski korak za $\phi$ (v smislu: če lastnost velja za vse elemente, \qt{manjše} od $a$, potem velja tudi za $a$).

Zgornja definicija dobro osnovane urejenosti je neposredno naravnana na indukcijo. To je njena prednost, je pa tudi njena slabost: kako vemo, da neka relacija $\ll$ dejansko dobro osnovano ureja dano množico? Neposredno preveriti definicijo je lahko težje, kot pa neposredno preveriti željeno univerzalno kvantificirano lastnost; v tem primeru nismo nič pridobili. Zato je dobro imeti alternativne karakterizacije dobro osnovane urejenosti.

\begin{izrek}
	Naslednje izjave so ekvivalentne za poljubno množico $X$ in relacijo $\ll$ na njej.
	\begin{itemize}
		\item
			$\ll$ dobro osnovano ureja $X$.
		\item
			Ne obstaja neskončna padajoča veriga v $X$. Natančneje: ne obstaja zaporedje $a\colon \NN \to X$, za katerega velja $a_{n+1} \ll a_n$ za vse $n \in \NN$.
		\item
			Vsaka nahajajoča podmnožica $S \subseteq X$ ima minimalni element v naslednjem smislu: obstaja $a \in S$, tako da za noben $x \in S$ ne velja $x \ll a$.
	\end{itemize}
\end{izrek}

\begin{dokaz}
\end{dokaz}

Splošne dobro osnovane urejenosti so lahko precej razvejane (kot bomo kasneje videli iz primerov). Včasih se zato želimo omejiti na tako imenovane dobre urejenosti

\begin{definicija}
	Množica $X$, opremljena z relacijo $\ll$, je \df{dobra urejena}, kadar je dobro osnovano urejena in relacija $\ll$ je stroga linearna urejenost.
\end{definicija}

Tudi za dobre urejenosti imamo karakterizacijo.

\begin{izrek}
	Naslednji izjavi sta ekvivalentni za poljubno množico $X$ in relacijo $\ll$ na njej.
	\begin{itemize}
		\item
			$\ll$ dobro ureja $X$.
		\item
			Vsaka nahajajoča podmnožica $S \subseteq X$ ima najmanjši element v naslednjem smislu: obstaja $a \in S$, tako da za vsak $x \in S \setminus \set{a}$ velja $a \ll x$.
	\end{itemize}
\end{izrek}

\begin{dokaz}
\end{dokaz}

\note{primeri dobro (osnovano) urejenih množic, primeri dokazov z indukcijo na njih}

\section{Strukturna indukcija}


%%% Local Variables:
%%% mode: latex
%%% TeX-master: "ucbenik-lmn"
%%% End:
