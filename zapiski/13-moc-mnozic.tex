\chapter{Moč množic}

V tej lekciji bomo govorili o velikosti množic, končnih množicah in neskončnih množicah.

\section{Aksiom odvisne izbire}

Kasneje bom potrebovali inačico aksioma izbire, ki se glasi:

\begin{aksiom}[Odvisna izbira]
  Naj bo $A$ neprazna množica in $R \subseteq A \times A$ celovita relacija, se pravi
  $\all{x \in A} \some{y \in A} x \, R \, y$.
  %
  Tedaj obstaja tako zaporedje $a : \NN \to A$, da za vse $n \in N$ velja $a_n \, R \,  a_{n+1}$.
\end{aksiom}

Aksiom odvisne izbire sledi iz aksioma izbire, česar tu ne bomo dokazali.

Aksiom odvisne izbire se v praksi uporabi, kadar želimo konstruirati zaporedje $a : \NN \to A$, pri čemer sta izpolnjena dva pogoja:
%
\begin{enumerate}
\item za vsak člen zaporedja $a_n$ imamo na voljo eno ali več izbir,
\item izbire za člen $a_{n+1}$ so odvisne od tega, kaj smo izbrali za $a_n$.
\end{enumerate}
%
Primer uporabe bomo videli v nadaljevanju.

\section{Končne množice}

Kako bi definirali pojem ">končna množica"<?

\begin{definicija}
  Za vsako naravno število $n \in \NN$, naj bo
  \textbf{standardna končna množica} $[n] = \set{k \in \NN \such k < n}$.
\end{definicija}

\noindent
Torej velja
%
\begin{align*}
  [0] &= \{\} \\
  [1] &= \{0\} \\
  [2] &= \{0, 1\} \\
  [3] &= \{0, 1, 2\} \\
      &\vdots
\end{align*}

\begin{definicija}
  Množica je \textbf{končna}, če je izomorfna kaki standardni končni množici.
\end{definicija}


Velja naslednje (ne bomo dokazali): če je $A \iso [m]$ in $A \iso [n]$, potem je $m = n$. Torej za končno
množico~$A$ obstaja natanko en $n \in \NN$, da velja $A \iso [n]$. Temu $n$ pravimo \textbf{moč} množice $A$,
saj nam pove, koliko elementov ima $A$. Moč končne množice $A$ označimo z $|A|$.

Za moči končnih množic velja
%
\begin{align*}
  |[n]| &= n, \\
  |A \times B| &= |A| \times |B|, \\
  |A + B| &= |A| + |B|, \\
  |B^A| &= |B|^{|A|}.
\end{align*}
%
Zgornje enačbe je treba razumeti pravilno: na levi nastopajo $\times$, $+$ in potenciranje kot operacije na množicah, na desni pa kot operacije na naravnih številih.

Za unijo velja \textbf{pravilo vključitve in izključitve}:
%
\begin{equation*}
 |A \cup B| = |A| + |B| - |A \cap B|.
\end{equation*}
%
Pravilo se tako imenuje, ker smo pri štetju elementov $A \cup B$ \emph{vključili} elemente $A$ in $B$, nato pa \emph{izključili} elemente preseka $A \cap B$, da jih ne bi šteli dvakrat.
%
Pravilo vključitve in izključitve za tri množice se glasi
%
\begin{equation*}
  |A \cup B \cup C| = |A| + |B| + |C| - |A \cap B| - |B \cap C| - |C \cap A| + |A \cap B \cap C|.
\end{equation*}

\begin{vaja}
  Zapišite pravilo vključitve in izključitve za unijo $A_1 \cup A_2 \cup \cdots \cup A_n$.
\end{vaja}


\section{Neskončne množice}

\begin{definicija}
  Množica je \textbf{neskončna}, če ni končna.
\end{definicija}

\begin{izrek}
  Množica $A$ je neskončna natanko tedaj, ko obstaja injektivna preslikava $\NN \to A$.
\end{izrek}

\begin{dokaz}

($\lthen$)
%
Denimo, da $A$ ni končna.
Injektivno preslikavo $e : \NN \to A$ definiramo s pomočjo aksioma odvisne izbire.
Ker $A$ ni izomorfna $[0]$, ni prazna, torej obstaja $e(0) \in A$.
Denimo, da smo že definirali $e$ kot injektivno preslikavo $[n] \to A$.
Tedaj jo lahko razširimo na injektivno preslikavo $e : [n+1] \to A$ takole: ker $e$ ni surjektivna (če bi bila, bi veljalo $A \iso [n]$ in $A$ bi bila končna), obstaja $x \in A$, ki ni v sliki $e$.
Sedaj \emph{izberemo} $e(n) \in A$, ki ni v sliki.
Tako dobimo $e : \NN \to A$, ki je injektivna.

($\Leftarrow$)
%
Denimo, da obstaja injektivna preslikava $e : \NN \to A$.
Če bi za neki $n$ veljalo $A \iso [n]$, bi imeli izomorfizem $f : A \to [n]$.
Tedaj bi bil kompozitum $f \circ e : \NN \to [n]$ injektivna preslikava, ta pa ne obstaja (dokaz opustimo).
\end{dokaz}

\subsection{Moč množic}

Tudi neskončnim množicam želimo prirediti moč, se pravi, neko mero velikosti. Preden pa nam bo to uspelo, se najprej naučimo primerjati velikost množic, ne da bi pri tem govorili o ">številu elementov"<.

\begin{definicija}
  Množici $A$ in $B$ imata enako moč, sta \textbf{ekvipolentni}, kadar sta izomorfni.
\end{definicija}

Ekvipolentnost in izomorfnost sta torej sinonima, ki pa se uporabljata v različnih situacijah. O ekvipolentnosti govorimo, ko imamo v mislih velikost množic ali število elementov. Izomorfnost je širši pojem, ki se uporablja tudi v algebri, topologiji in povsod, kjer imamo opravka z matematičnimi strukturami, in pomeni ">enakovredna struktura"<.

Spomnimo se, da je izomorfnost in torej tudi ekvipolentnost ekvivalenčna relacija.
Torej lahko tvorimo ekvivalenčne razrede glede na ekvipolentnost: vsaki množici $A$ priredimo razred vseh množic, ki so jih ekvipolentne:
%
\begin{itemize}
\item $[\emptyset]_{\iso} = \{ \emptyset \}$,
\item $[\set{ \unit }]_{\iso}$ je \emph{pravi razred} vseh enojcev,
\item $[\set{0, 1}]_{\iso}$ je \emph{pravi razred} vseh množic z dvema elementoma,
\item itd.
\end{itemize}
%
Dejstvo, da so razredi glede na izomorfnost pravi razredi in ne množice, je precej nerodna reč, saj z njimi ne moremo udobno delati (potrebovali bi ">super razrede"<, katerih elementi so razredi).
Izognemo se jim tako, da namesto z razredi delamo z izborom predstavnikov.

Pravzaprav smo ta trik že uporabili, ko smo govorili o moči končnih množic, ko smo za predstavnike ekvipolnenčnih razredov končnih množic izbrali standardne končne množice. Le-te nam lahko služijo kot ">števila"<, s katerimi opišemo moči končnih množic, saj med standardno končno množico $[n]$ in številom $n$ ni bistvene razlike. (Še več, kasneje bomo videli, da lahko naravna števila obravnavamo tako, da dejansko so standardne končne množice!)

Kako bi torej izbrali predstavnike razredov za ekvipolentnost za vse množice?
Če bi nam to uspelo, bi take predstavnike lahko uporabili kot števila, imenujejo se \textbf{kardinalna števila}, s katerimi bi merili moč množic.

\begin{definicija}
  \textbf{Kardinalno število} je tako ordinalno število $\kappa$, za katerega velja $|\alpha| < |\kappa|$ za vse $\alpha \in \kappa$.
\end{definicija}

\begin{zgled}
  Tu ne bomo dokazali, da je vsaka množica ekvipolentna natanko enemu kardinalnemu številu. Raje si poskušajmo predstavljati kardinalna števila:
  % 
  \begin{itemize}
  \item Končni ordinali, ki so seveda kar naravna števila, so kardinalna števila, saj je naravno število strogo večje od
    svojih predhodnikov.
  \item Ordinal $\omega = \NN = \set{0, 1, 2, \ldots}$, ki vsebuje vse končne ordinale, je kardinalno število. Označujemo ga tudi z $\aleph_0$.
  \item Ordinal $\omega + 1 = \set{0, 1, 2, \ldots, \omega}$ \emph{ni} kardinalno število, saj je ekvipolenten~$\omega$. Prav tako so ordinali
    % 
    \begin{equation*}
      \omega + 2, \omega + 3, \ldots, \omega + \omega, \ldots, \omega + \omega + \omega, \ldots, \omega^2, \omega^3
    \end{equation*}
    % 
    vsi ekvipolentni $\omega$, zato niso kardinali. Pravzaprav si je precej težko predstavljati ordinal, katerega moč je strogo večja od $\omega$.
  \end{itemize}
\end{zgled}

Vsaki množici $A$ torej priredimo nekega predstavnika razreda $[A]_{\iso}$, ki ga označimo $|A|$ in ga imenujemo \textbf{moč} množice~$A$. Za končne množice so to kar naravna števila, za splošne množice pa so to kardinalna števila.

Moči množic lahko primerjamo med seboj, čeprav ne vemo, kaj točno naravna števila so!

\begin{definicija}
  Naj bosta $A$ in $B$ poljubni množici. Pravimo:
  %
  \begin{enumerate}
  \item $A$ ima enako moč kot $B$, pišemo $|A| = |B|$, ko obstaja bijektivna preslikava $A \to B$.
  \item $A$ ima moč manjšo ali enako $B$, pišemo $|A| \leq |B|$, ko obstaja injektivna preslikava $A \to B$.
  \item $A$ ima moč manjšo kot $B$, pišemo $|A| < |B|$, če velja $|A| \leq |B|$ in $|A| \neq |B|$.
  \end{enumerate}
\end{definicija}

\begin{izrek}
  \label{izr:leq-iff-empty-or-onto}
  $|A| \leq |B|$ natanko tedaj, ko je $A = \emptyset$ ali obstaja surjekcija $B \to A$.
\end{izrek}

\begin{dokaz}
  Denimo, da je $f : A \to B$ injektivna in $A \neq \emptyset$. Torej obstaja neki $a \in A$.
  Definiramo preslikavo $g : B \to A$ takole:
  %
  \begin{equation*}
    g(y) = x  \defiff f(x) = y \lor (y \not\in f_{*}(A) \land x = a).
  \end{equation*}
  %
  Povedano malo drugače:
  %
  \begin{equation*}
    g(y) =
    \begin{cases}
      f^{-1}(y) & \text{če $y \in f_{*}(A)$,} \\
      a         & \text{če $y \not\in f_{*}(A)$.}
    \end{cases}
  \end{equation*}
  %
  Ker velja $g \circ f = \id[A]$, je $g$ retrakcija in zato surjektivna.

  Obratno, denimo, da je $A$ prazna ali obstaja surjekcija $f : B \to A$. Če je $A$
  prazna, je edina preslikava $\emptyset \to B$ injektivna. Če je $f : B \to A$ surjektivna,
  ima prerez (zakaj?), ki je injektivna preslikava.
\end{dokaz}


\subsection{Cantorjev izrek}

\begin{izrek}[Cantor]
  $|A| < |\pow{A}|$.
\end{izrek}

\begin{dokaz}
  Najprej dokažimo $|A| \leq |\pow{A}|$. Iščemo injektivno preslikava $f : A \to \pow{A}$. Vzemimo $f(x) = \{x\}$. Zlahka preverimo, da je $f$ res injektivna.

  Sedaj dokazujemo, da ne obstaja bijekcija $A \to \pow{A}$. Dokazali bomo, da ne obstaja surjekcija $A \to \pow{A}$, kar zadostuje. Denimo, da je $g : A \to \pow{A}$ poljubna preslikava. Trdimo, da $g$ ni surjekcija. Res, podmnožica
  %
  \begin{equation*}
    S = \set{x \in A \mid x \not\in g(x) }
  \end{equation*}
  %
  ni v sliki $g$. Če bi bila, bi za neki $y \in A$ veljalo $g(y) = S$, a to bi vodilo v protislovje:
  % 
  \begin{itemize}
  \item velja $y \not\in S$: če $y \in S$ potem $y \not\in g(y) = S$ po definiciji $S$,
  \item velja $\lnot (y \not\in S)$: če $y \not\in S$ potem $y \not\in g(y) = S$.
  \end{itemize}
\end{dokaz}


\subsection{Števne in neštevne množice}

Kot smo že povedali, moč množice $\NN$ označimo z $\aleph_0$.

\begin{definicija}
  Množica $A$ je \textbf{števna}, če velja velja $|A| \leq \aleph_0$.
\end{definicija}

\begin{definicija}
  Množica $A$ je \textbf{neštevna}, če ni števna.
\end{definicija}

\begin{izrek}
  Za vsako množico $A$ so ekvivalentne naslednje izjave:
  %
  \begin{enumerate}
  \item $A$ je števna.
  \item Obstaja injektivna preslikava $A \to \NN$.
  \item $A$ je prazna ali obstaja surjektivna preslikava $\NN \to A$.
  \item Obstaja surjektivna preslikava $\NN \to \one + A$.
  \item $A$ je končna ali izomorfna $\NN$.
  \end{enumerate}
\end{izrek}

\begin{dokaz}
%
\begin{itemize}
\item[$(1 \lthen 2)$]
%
Če je $A$ števna, velja $|A| \leq \aleph_0 = |\NN|$, torej obstaja injektivna $A \to
\NN$ po definiciji relacije $\leq$.

\item[$(2 \lthen 3)$]
%
To sledi neposredno iz Izreka~\ref{izr:leq-iff-empty-or-onto}.

\item[$(3 \lthen 4)$]
%
Denimo, da je $A$ prazna ali obstaja surjektivna preslikava $\NN \to A$:
%
\begin{enumerate}
\item
  Če je $A = \emptyset$, potem seveda obstaja surjektivna preslikava $\NN \to \one + A$, in sicer
  $n \mapsto \inl \unit$.
\item 
  Če obstaja surjektivna preslikava $f : \NN \to A$, potem lahko definiramo surjektivno
  preslikavo $g : \NN \to \one + A$ s predpisom
  %
  \begin{equation*}
    g(n) =
    \begin{cases}
      \inl \unit      &\text{če $n = 0$,}\\
      \inr (f(n-1))   &\text{če $n > 0$.}
    \end{cases}
  \end{equation*}
\end{enumerate}

\item[$(4 \lthen 5)$]
%
Denimo, da obstaja surjektivna preslikava $r : \NN \to \one + A$.
Dokazali bomo, da je $A$ izomorfna $\NN$, če ni končna.
Predpostavimo torej, da $A$ ni končna.
Preslikava $r$ ima prerez $s : \one + A \to \NN$, ki je seveda injektivna preslikava.
Preslikav $s \circ \inr : A \to \NN$ je kompozitum injektivnih preslikav, zato je injektivna.
Ker $A$ ni končna, obstaja tudi injektivna preslikava $\NN \to A$.
Po izreku Cantor-Schröder-Bernstein, ki ga bomo dokazali spodaj, je torej $A$ izomorfna $\NN$.

\item[$(5 \lthen 1)$]
%
Če je $A$ končna, je števna, ker očitno velja $A = |[n]| \leq |\NN| = \aleph_0$.
Če je $A$ izomorfna $\NN$, potem velja $|A| = |\NN| \leq |\NN| = \aleph_0$.
\end{itemize}
\end{dokaz}

\begin{izrek}
  $\NN \times \NN \iso \NN$.
\end{izrek}

\begin{dokaz}
  Za vajo, poiščite dokaz v zapiskih iz analize ali na internetu.
\end{dokaz}

\begin{definicija}
  \textbf{Števna družina} je družina $A : I \to \Set$, katere indeksna množica~$I$ je števna.
\end{definicija}

\begin{izrek}
  Unija števne družine števnih množic je števna.
\end{izrek}

\begin{dokaz}
  Izrek bomo dokazali le za primer, ko je indeksna množica~$\NN$.
  %
  Najprej obravnavajmo unijo družine $A : \NN \to \Set$, kjer je $A_n$ števna za vse $n \in \NN$.
  Za vsak $n \in \NN$ obstaja surjektivna preslikava $\NN \to A_n + \one$. Po aksiomu izbire obstaja funkcija izbire
  %
  \begin{equation*}
    e \in \prod_{n \in \NN} \set{f : \NN \to A_n + \one \such \text{$f$ surjekcija}}.
  \end{equation*}
  %
  Definiramo $e' : \NN \times \NN \to \one + \bigcup_{n \in \NN} A_n$ s predpisom
  %
  \begin{equation*}
    e'(n, k) = e(n)(k).
  \end{equation*}
  %
  Trdimo, da je $e'$ surjekcija iz $\NN \times \NN$ na $\one + \bigcup_{n \in \NN} A_n$.
\end{dokaz}


\subsection{Cantor-Schröder-Bernsteinov izrek in zakon trihotomije}

\begin{izrek}[Cantor-Schröder-Bernstein]
  Če obstajata injektivni preslikava $A \to B$ in $B \to A$, potem obstaja bijektivna preslikava $A \to B$.
\end{izrek}

\begin{dokaz}
  Definirajmo družino $C : \NN \to \mathsf{Set}$ takole:
  %
  \begin{align*}
    C_0 &= A \setminus g_{*}(B), \\
    C_{n+1} &= g_{*}(f_{*}(C_n).
  \end{align*}
  %
  Naj bo $D = \bigcup_{n \in \NN} C_n$. Očitno je $C_n \subseteq A$ za vse
  $n \in \NN$, zato velja tudi $D \subseteq A$.

  Ker je $g$ injektivna, je bijekcija kot preslikava $g : B \to g_{*}(B)$, zato
  obstaja inverz $g^{-1} : g_{*}(B) \to B$. Trdimo, da velja
  $A \setminus D \subseteq g_{*}(B)$. Res, če velja $x \in A \setminus D$, tedaj
  $x \not\in D$ in zato $x \not\in C_0 = A \setminus g_{*}(B)$, od koder sledi
  $x \in g_{*}(B)$. Od tod sledi, da lahko $g^{-1}$ uporabimo na
  $x \in A \setminus D$.

  Definirajmo $h : A \to B$ s predpisom
  % 
  \begin{equation*}
    h(x) =
    \begin{cases}
      f(x), & \text{če $x \in D$,} \\
      g^{-1}(x) &\text{če $x \in A \setminus D$.}
    \end{cases}
  \end{equation*}
  %
  Dokažimo, da je $h$ injektivna preslikava.
  Denimo, da za $x, y \in A$ velja $h(x) = h(y)$. Obravnavamo štiri primere:
  %
  \begin{enumerate}
  \item Če je $x \in D$ in $y \in D$, potem je $f(x) = h(x) = h(y) = f(y)$ in
    zato $x = y$, saj je~$f$ injektivna.
  \item Če je $x \in A \setminus D$ in $y \in A \setminus D$, potem je
    $g^{-1}(x) = h(x) = h(y) = g^{-1}(y)$ in zato $x = y$, saj je $g^{-1}$
    injektivna.
  \item Če je $x \in D$ in $y \in A \setminus D$, potem je
    $f(x) = h(x) = h(y) = g^{-1}(y)$, zato je $y = g(g^{-1}(y)) = g(f(x))$.
    Obstaja $n \in \NN$, da je $x \in C_n$, od tod pa sledi
    $y = g(f(x)) \in C_{n+1} \subseteq D$, kar je v protislovju z
    $y \in A \setminus D$. Torej se ta primer sploh ne more zgoditi.
  \item Če je $x \in A \setminus D$ in $y \in D$, je razmislek kot v prejšnjem
    primeru, le da zamenjamo vlogi~$x$ in~$y$.
  \end{enumerate}

  Preveriti moramo še, da je $h$ surjektivna preslikava. Naj bo $z \in B$.
  Poiskati moramo tak $x \in A$, da velja $h(x) = z$. Obravnavamo dva primera:
  %
  \begin{enumerate}
  \item Če $z \in f_{*}(D)$, potem obstaja $x \in D$, da je $f(x) = y$, s tem pa
    velja tudi $h(x) = f(x) = z$.
  \item Če velja $z \not\in f_{*}(D)$, potem vzamemo $x = g(z)$. Preverimo, da
    velja $h(x) = z$.

    Najprej dokažimo $x \not\in D$. Če bi namreč veljalo $x \in D$, potem bi
    obstajal $n \in \NN$, da je $x \in C_n$. Poleg tega
    $x = g(z) \not\in A \setminus g_{*}(B) = C_0$, zato velja $n > 0$. Se pravi,
    da obstaja $y \in C_{n-1}$, da je $g(z) = x = g(f(y))$. Ker je $g$
    injektivna, sledi $z = f(y)$, kar je v nasprotju z predpostavko
    $z \not\in f_{*}(D)$. Torej res velja $x \not\in D$.

    Ker $x \not\in D$, velja $h(x) = g^{-1}(x) = g^{-1}(g(z)) = z$, kar smo
    želeli dokazati.
  \end{enumerate}
\end{dokaz}

\begin{posledica}
  Če $|A| \leq |B|$ in $|B| \leq |A|$, potem $|A| = |B|$.
\end{posledica}

\begin{dokaz}
  To sledi neposredno iz izreka CSB in definicije $\leq$.
\end{dokaz}

Brez dokaza omenimo še, da velja \textbf{zakon trihotomije}: za vsaki množici $A$ in $B$
velja
%
\begin{equation*}
  |A| < |B| \lor |A| = |B| \lor |B| < |A|.
\end{equation*}
%
Relacija $\leq$ torej uredi moči množic linearno.



\subsection{Moč kontinuuma in Cantorjeva hipoteza}

Na vajah boste spoznali, da ima množica realnih števil $\RR$ enako moč kot potenčna množica $\pow{\NN}$. Moči $\RR$ in $\pow{\NN}$ pravimo \textbf{moč kontinuuma} (ker je ">kontinuum"< tudi staro ime za $\RR$). Že Georg Cantor, utemeljitelj teorije množic, je postavil naslednji domnevo:
%
\begin{quote}
  \emph{\textbf{Cantorjeva hipoteza:} Vsaka neštevna podmnožica realnih števil je izomorfna~$\RR$.}
\end{quote}
%
Povedano, z drugimi besedami, po moči ni nobene množice strogo med $\NN$ in $\RR$. Cantorjeva hipoteza je ostala odprta dlje časa. Dokončno je Cohen pred dobrega pol stoletja dokazal naslednje:

\begin{izrek}[Cohen]
  Iz Zermelo-Fraenkelovih aksiomov teorije množic Cantorjeve hipoteze ne moremo niti dokazati niti ovreči.
\end{izrek}

Pravimo, da je Cantorjeva hipoteza \emph{neodvisna} od aksiomov teorije množic. Poznamo še posplošeno Cantorjevo hipotezo, ki se glasi:
%
\begin{quote}
  \textbf{Posplošena Cantorjeva hipoteza:}
  %
  Če je $|A| \leq |B| \leq |\pow{A}|$, potem je $|B| = |A|$ ali $|B| = |\pow{A}|$.
\end{quote}
%
Tudi posplošena Cantorjeva hipoteza je neodvisna od aksiomov teorije množic.
Danes vemo zelo veliko o tej hipotezi in poznamo še mnoge druge izjave o množicah, ki so neodvisne od Zermelo-Fraenkelovih aksiomov teorije množic.
Ti veljajo za nekakšno uradno različico teorije množic in jih bomo obravnavali na naslednjih predavanjih.


%%% Local Variables:
%%% mode: latex
%%% TeX-master: "lmn"
%%% End:
