\chapter{Aksiomatska teorija množic}

\section{Kodiranje matematičnih objektov z množicami}

Z množicami smo izrazili številne matematične objekte, na primer:
%
\begin{itemize}
\item ordinalna števila smo predstavili kot množice svojih predhodnikov,
\item preslikavo $f : A \to B$ lahko izrazimo kot funkcijsko relacijo med $A$ in $B$, torej kot
  podmnožico $A \times B$,
\item kvocientna množica $A/R$ je množica ekvivalenčnih razredov, ekvivalenčni razredi so spet
  množice,
\end{itemize}
%
Ali je možno vse matematične objekte predstaviti z množicami? Da!

\subsection{Urejeni pari}

Par $(x, y)$ lahko predstavimo z množico $\set{\set{x}, \set{x,y}}$. Tako dobimo
%
\begin{equation*}
  A \times B \defeq \set{ \set{\set{x}, \set{x,y}} \mid x \in A \land y \in B }.
\end{equation*}


\subsection{Vsota}

Elemente vsote $A + B$ kodiramo takole:
%
\begin{align*}
  \inl(x) &\defeq (x, 0) = \set{\set{x}, \set{x, \emptyset}}, \\
  \inr(x) &\defeq (x, 1) = \set{\set{x}, \set{x, \set{\emptyset}}}.
\end{align*}


\subsection{Naravna števila}

Kot smo že videli, lahko ordinalna števila kodiramo kot množice svojih predhodnikov, poseben primer pa so naravna števila, ki so končni ordinali.

Kako pa kodiramo operacijo naslednik? Definirajmo preslikavo \textbf{naslednik} ${}^{+} : \Set \to \Set$,
%
\begin{equation*}
  \suc{x} \defeq x \cup \set{x}.
\end{equation*}
%
Če si predstavljamo, da je $x$ število, tedaj so elementi $\suc{x}$ predhodniki~$x$ in še $x$, kar je ravno naslednik~$x$.
Naravna števila res dobimo tako, da na $\emptyset$ uporabljamo naslednik $\suc{{}}$:
%
\begin{align*}
    0 &= \emptyset \\
    1 &= \suc{0} = \set{ 0 } = \set{\emptyset} \\
    2 &= \suc{1} = \set{0, 1} = \set{\emptyset, \set{\emptyset}} \\
    3 &= \suc{2} = \set{0, 1, 2} = \set{\emptyset, \set{\emptyset}, \set{\emptyset, \set{\emptyset}}} \\
    4 &= \suc{3} = \set{0, 1, 2, 3} =
       \set{\emptyset, \set{\emptyset}, \set{\emptyset, \set{\emptyset}},
            \set{\emptyset, \set{\emptyset}, \set{\emptyset, \set{\emptyset}}}} \\
      &\vdots
\end{align*}

\subsection{Cela števila}

Cela števila so kvocient $\NN \times \NN$:
%
\begin{equation*}
    \ZZ \defeq (\NN \times \NN)/{\sim},
\end{equation*}
%
kjer je
%
\begin{equation*}
  (a,b) \sim (c,d) \defiff a + d = c + b.
\end{equation*}
%
Urejeni par $(a, b)$ predstavlja razliko števil $a$ in $b$.


\subsection{Racionalna števila}

Racionalna števila so kvocient:
%
\begin{equation*}
  \QQ = (\ZZ \times \set{n \in \NN \such n > 0})/{\approx},
\end{equation*}
%
kjer je
%
\begin{equation*}
    (a,m) \approx (b,n) \defiff a n = b m.
\end{equation*}
%
Urejeni par $(a, n)$ predstavlja kvocient števil $a$ in $n$.

\subsection{Realna števila}

Realno število je Dedekindov rez, torej podmnožica $\QQ$. Reze ste obravnavali pri Analizi, tako da jih na tem mestu ne
bomo obnavljali.

In tako naprej. Ne pravimo, da je kodiranje vseh matematičnih objektov z množicami vedno
dobra ideja, vendar pa je dejstvo, da je to možno, pomembno spoznanje osnov matematike. Iz
njega na primer sledi tole: če je teorija množic neprotislovna, potem je neprotislovna
tudi vsa matematika, ki jo lahko kodiramo z množicami (torej več ali manj vsa običajna
matematika).



\section{Zermelo-Fraenkelovi aksiomi}

Aksiomi opredeljujejo množice brez urelementov (">\emph{Vse} je množica"<). Za aksiomatizacijo razredov bi morali zapisati drugačne aksiome, kot so na primer von Neumann-Bernays-Gödelovi aksiomi.

\begin{description}

\item[Ekstenzionalnost:] množici $A$ in $B$, ki imata iste elemente, sta enaki.

\item[Neurejeni par]: za vsak $x$ in $y$ je $\set{x, y}$ množica, ki vsebuje natanko $x$ in $y$:
  %
  \begin{equation*}
    \all{x y z} z \in \set{x, y} \liff z = x \lor z = y
  \end{equation*}
  %
  Okrajšava: $\set{x} = \set{x, x}$.

\item[Unija:] za vsako množico $A$ je $\bigcup A$ množica, ki vsebuje natanko vse
  elemente množic iz $A$:
  %
  \begin{equation*}
    \all{A x} x \in \bigcup A \liff \some{B \in A} x \in B.
  \end{equation*}

\item[Prazna množica:] množica $\emptyset$ nima elementa:
  %
  \begin{equation*}
  \all{x} x \not\in \emptyset.
  \end{equation*}

\item[Neskončna množica] obstaja množica, ki vsebuje $\emptyset$ in je zaprta za operacijo naslednik
  ($\suc{x} = x \cup \set{x}$):
  %
  \begin{equation*}
    \some{A} \emptyset \in A \land \all{x \in A} \suc{x} \in A.
  \end{equation*}

\item[Podmnožica:] za vsako množico $A$ in formulo $\phi$ je $\set{x \in A \mid| \phi(x)}$
  množica, ki vsebuje natanko vse element iz $A$, ki zadoščajo $\phi$:
  %
  \begin{equation*}
    \all{y} y \in \{x \in A | \phi(x)\} \liff \phi(y).
  \end{equation*}

\item[Potenčna množica:] za vsako množico $A$ je $\pow{A}$ množica, ki vsebuje
  natanko vse njene podmnožice:
  %
  \begin{equation*}
    \all{S} S \in \pow{A} \liff S \subseteq A.
  \end{equation*}

\item[Zamenjava] če je $A$ množica in $f : A \to \Set$ preslikava, je
  %
  $
    \img{f}(A) = \set{ y \mid \some{x \in A} y = f(x) }
  $
  %
  množica.

\item[Dobra osnovanost:] relacija ${\in} \subseteq \Set \times \Set$ je dobro osnovana.

\item[Aksiom izbire:] vsaka družina nepraznih množic ima funkcijo izbire.
\end{description}


\section{Kumulativna hierarhija}

Če lahko vse matematične objekte kodiramo z množicami, potem lahko na razred
vseh množic $\Set$ gledamo kot na celotni matematični svet. Razred $\Set$ ima
zanimivo strukturo, ki ji pravimo \textbf{kumulativna hierarhija}. Namreč, s pomočjo
Zermelo-Fraenkelovih aksiomov lahko tvorimo vse množice iz $\emptyset$ z
operacijama potenčna množica in unija. Postopek je \textbf{transfiniten} (neskončen), ima pa toliko korakov, kot je ordinalnih števil:
%
\begin{align*}
  V_0 &= \emptyset \\
  V_1 &= \pow{V_0} = \set{\emptyset} \\
  V_2 &= \pow{V_1} = \set{\emptyset, \set{\emptyset}} \\
  V_3 &= \pow{V_2} = \set{\emptyset, \set{\emptyset}, \set{\set{\emptyset}}, \set{\emptyset, \set{\emptyset}}} \\
      &\vdots \\
  V_\omega &= \textstyle\bigcup_{k < \omega} V_k \\
  V_{\omega+1} &= \pow{V_\omega} \\
  V_{\omega+2} &= \pow{V_{\omega+1}} \\
  &\vdots \\
  V_{\omega + \omega} &= \textstyle\bigcup_{\alpha < \omega + \omega} V_\alpha \\
  &\vdots
\end{align*}
%
Splošna formula se glasi $V_\alpha = \textstyle\bigcup_{\beta < \alpha} \pow{V_\beta}$.


\begin{vaja}
  Koliko elementov ima $V_5$?
\end{vaja}

Bistvo kumulativne hierarhije je, da zaobjame vse množice.

\begin{izrek}[Kumulativna hierarhija]
  $\Set = \bigcup_{\alpha \in \On} V_\alpha$.
\end{izrek}

\begin{dokaz}
  Dokaz opustimo, povejmo le, da je za izrek bistven aksiom o dobro osnovanosti. Le ta nam zagotavlja, da se vsaka padajoča $\in$-veriga konča z~$\emptyset$.
\end{dokaz}


% \subsection{Zakon trihotomije}

% V tem razadelku podamo še oris dokaza, da je aksiom izbire ekvivalenten zakonu trihotomije.

% \textbf{Definicija:} Naj bo $(P, <)$ dobra urejenost. Podmnožica $I ⊆ P$ je **začetni
% segment**, če je doljna množica: iz $x < y$ in $y ∈ I$ sledi $x ∈ I$.

% \textbf{Definicija:} Naj bosta $(P, <_P)$ in $(Q, <_Q)$ dobri urejenosti. Pravimo, da
% je preslikava $e : P \to Q$ **vložitev**, kadar velja:

% 1. $e$ je strogo monotona in
% 2. slika $e_{P}$ je začetni segment v $Q$.

% Vložitev je injektivna preslikava.

% **Lemma 1:** Naj bosta $(P, <_P)$ in $(Q, <_Q)$ dobri urejenosti. Če obstaja
% injektivna preslikava $P \to Q$, potem obstaja tudi vložitev $P \to Q$.

% Dokaz: opuščen.

% **Lemma 2:** Naj bosta $(P, <_P)$ in $(Q, <_Q)$ dobri urejenosti. Tedaj bodisi
% obstaja vložitev $P \to Q$ ali vložitev $Q \to P$.

% Dokaz: opuščen.

% \textbf{Izrek:} Aksiom izbire je ekvivalenten zakonu trihotomije: za vse množice $X$ in $Y$ velja
% $|X| ≤ |Y|$ ali $|Y| ≤ |X|$.

% Dokaz:

% Najprej predpostavimo, da velja aksiom izbire. Naj bosta $X$ in $Y$ množici. Ker
% velja aksiom izbire, lahko $X$ in $Y$ dobro uredimo, denimo z relacijama $<_X$
% in $<_Y$. Iz zgornje leme sledi, da obstaja vložitev $X \to Y$ ali $Y \to X$.
% Ker so vložitve injektivne, torej velja $|X| ≤ |Y|$ ali $|Y| ≤ |X|$.

% Predpostavimo zdaj, da za vse množice $X$ in $Y$ velja $|X| ≤ |Y|$ ali $|Y| ≤
% |X|$. Dokazali bomo, da lahko vsako množico dobro uredimo, iz česar sledi aksiom izbire.

\section{Aksiom izbire}

Za konec povejmo še nekaj več o aksiomu izbire in Zornovi lemi, ki mu je ekvivalentna. Le-ta se uporablja v algebri.

\begin{lema}[Zornova lema]
  Če ima v delni urejenosti $(P, {\leq})$ vsaka veriga zgornjo mejo,
  potem ima $P$ maksimalni element.
\end{lema}

\begin{dokaz}
  Dokaz se naslanja na aksiom izbire in Bourbaki-Wittov izrek o negibnih točkah (glej
  spodaj). Naj bo $C$ množica vseh verig v $P$. Uredimo jo z $\subseteq$. Na njej definiramo preslikavo
  $f : C \to C$, ki razširi verigo, če ni maksimalna, sicer je ne spremeni (tu uporabimo
  izbiro):
  %
  \begin{itemize}
  \item Če je $V \in C$ maksimalna veriga v $P$ (glede na $\subseteq$), definiramo $f(V) \defeq V$.
  \item Če $V \in C$ ni maksimalna veriga v $P$, potem obstaja tak $x \in P \setminus V$, da je $V
    \cup \set{x}$ spet veriga. V tem primeru \emph{izberemo} tak $x$ in definiramo $f(V) \defeq V
    \cup \set{x}$.
  \end{itemize}
  %
  Po izreku Bourbaki-Witt ima $f$ negibno vrednost $V \in C$. Ta $V$ je maksimalna
  veriga $V$, saj bi sicer veljalo, da je $V = f(V) = V \cup \set{x}$ za neki $x \not\in V$,
  kar ni možno. Naj bo $m$ zgornja meja za verigo $V$. Trdimo, da je $m$
  maksimalni element v $P$: denimo, da velja $m \leq y$ za $m \in P$. Ker je $V \cup \set{y}$
  veriga, ki vsebuje maksimalno verigo $V$, sledi $V = V \cup \set{y}$, od tod pa $y \in V$
  ter $y \leq m$. Torej je $m = y$ in $m$ je res maksimalni element.
\end{dokaz}

\begin{definicija}
  Naj bo $(P, \leq)$ delna ureditev. Preslikava $f : P \to P$ je \textbf{progresivna}, ko
  velja $x \leq f(x)$ za vsak $x \in P$.
\end{definicija}

\begin{opomba}
  Progresivna preslikav ni nujno monotona. (Poiščite proti-primer!)
\end{opomba}

\begin{izrek}[Bourbaki-Witt]
  Naj bo $(P, {\leq})$ neprazna delna ureditev, v kateri ima vsaka veriga zgornjo mejo in $f : P \to P$ progresivna
  preslikava. Tedaj ima $f$ negibno točko: to je tak $x \in P$, da velja $f(x) = x$.
\end{izrek}

\begin{dokaz}
  Dokaz opustimo.
\end{dokaz}

\begin{izrek}
  V teoriji množic \emph{brez} aksioma izbire so naslednje izjave ekvivalentne:
  %
  \begin{enumerate}
  \item Aksiom izbire
  \item Zornova lema
  \item Princip dobre urejenosti: vsaka množica ima dobro ureditev.
  \end{enumerate}
\end{izrek}

\begin{dokaz}
  ($1 \lthen 2$) Glej Zornovo lemo.

  ($2 \lthen 3$) Skica dokaza: naj bo $A$ poljubna množica, ki jo želimo dobro urediti.
  %
  Definirajmo \emph{delne} dobre ureditev množice $A$: to so pari $(B,R)$, kjer je $B \subseteq A$
  in $R \subseteq B \times B$ dobra ureditev na $B$. Za delni dobri ureditvi $(B,R)$ in
  $(C,Q)$ pravimo, da je $(C,Q)$ \emph{razširitev} $(B,R)$, kadar velja $B \subseteq C$, $R \subseteq Q$ in
  še, da je $B$ začetni segment v $C$, kar pomeni
  %
  \begin{equation*}
    \all{x y \in C} x \rel{Q} y \land y \in B \lthen x \in B.
  \end{equation*}
  %
  Kadar je $(C,Q)$ razširitev $(B,R)$, pišemo $(B,R) \preceq (C,Q)$. Naj bo $P$ množica vseh delnih
  dobrih ureditev množice $A$,
  %
  \begin{equation*}
    P \defeq \set{ (B, R) \mid \text{$B \subseteq A$ in $R \subseteq B \times B$ in $R$ je dobra ureditev $B$}},
  \end{equation*}
  %
  urejena z relacijo $\preceq$. Očitno je $\preceq$ delna ureditev. Trdimo, da imajo verige v
  $P$ zgornje meje glede na $\preceq$: če je $V \subseteq P$ veriga dobro urejenih podmnožic
  $A$, je njena zgornja meja $(D,S)$ kar unija po komponentah:
  %
  \begin{align*}
    D &\defeq \bigcup \set{B \mid (B, R) \in V} \\
    S &= \bigcup \set{R \mid (B, R) \in V}.
  \end{align*}
  %
  Preverimo, da velja $(D,S) \in P$. Očitno je $(D,S)$ stroga linearna ureditev
  (vaja). Denimo, da bi v $(D,S)$ imeli neskončno padajočo verigo
  %
  \begin{equation*}
    \cdots \rel{S} x_3 \rel{S} x_2 \rel{S} x_1 \rel{S} x_0.
  \end{equation*}
  %
  Obstaja $(B,R) \in V$, da je $x_0 \in B$. Potem bi bila $x_0, x_1, x_2, x_3, \ldots$
  padajoča veriga v $(B,R)$, kar ni možno, saj je $(B,R)$ dobro urejena. Res, ker
  je $x_i \in V$, obstaja $(C,Q)$, da je $x_i \in C$. Če velja $(B,R) \preceq (C,Q)$, potem
  $x_i \in B$ po definicijo $\preceq$. Če velja $(C,Q) \preceq (B,R)$, potem $x_i \in B$, ker velja
  $C \subseteq B$. Torej je $(D,S)$ res delna ureditev $P$.

  Preverimo še, da velja $(B,R) \preceq (D,S)$ za vsak $(B,R) \in V$. Denimo, da je $y \in D$,
  $x \in B$ in $y \rel{S} x$. Obstaja $(C,Q) \in V$, da je $y \in C$. Če velja $(C,Q) \preceq (B,R)$,
  potem $y \in C \subseteq B$. Če pa velja $(B,R) \preceq (C,Q)$, potem je $y \in B$ po definiciji $\preceq$.

  Po Zornovi lemi obstaja maksimalni element $(B,R)$ v $P$. Trdimo, da je $B = A$. Če bi namreč
  obstajal $x \in B \setminus A$, bi lahko razširili $(B,R)$ na večjo dobro ureditev tako, da bi dodali $x$
  na konec $B$:
  %
  \begin{align*}
    & (B \cup \set{x}, R') \\
    & y \rel{R'} z \defiff z = x \land y \rel{R} z.
  \end{align*}
  %
  To ni možno, ker je $(B,R)$ maksimalna delna ureditev. Torej je res $A = B$ in
  našli so dobro ureditev $A$.

  $(3 \lthen 1)$ Naj bo $A : I \to \Set$ družina nepraznih množic. Naj bo $\prec$ dobra ureditev
  na uniji $\bigcup A$. Ker so vse množice $A_i$ neprazne, ima vsaka od njih prvi element
  glede na $\prec$. Torej lahko definiramo funkcijo izbire $f$ s predpisom
  $f(i) \defeq \text{">prvi element $A_i$"<}$.
\end{dokaz}

\begin{izrek}
  Vsak vektorski prostor ima bazo.
\end{izrek}

\begin{dokaz}
  Naj bo $L$ vektorski prostor. Definiramo množico
  %
  \begin{equation*}
    P \defeq \set{ B \subseteq L \mid \text{$B$ je linearno neodvisna} }.
  \end{equation*}
  %
  Množico $P$ delno uredimo z relacijo $\subseteq$. Trdimo, da imajo verige v $P$ zgornje
  meje: zgornja meja verige $V \subseteq P$, je kar njena unija $\bigcup_{B \in V} B$. Seveda je
  treba preveriti, da je unija verige linearno neodvisnih množic spet linearno
  neodvisna (vaja). Po Zornovi lemi obstaja maksimalni element v $P$, torej
  maksimalna linearno neodvisna množica $B$ v $L$. To pa je seveda vektorska baza
  za $L$.
\end{dokaz}
