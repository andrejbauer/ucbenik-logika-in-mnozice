\chapter{Podmnožice in potenčne množice}

\section{Podmnožice}

Pojem podmnožice že poznamo, a ga kljub temu ponovimo.

\subsection{Definicija relacije $\subseteq$}

Pravimo, da je množica $S$ \textbf{podmnožica} množice $T$, pišemo $S \subseteq T$, ko velja $\all{x \in S} x \in T$. Pravimo tudi, da je $S$ \textbf{vsebovana} v $T$ in da je $T$ \textbf{nadmnožica}~$S$.

Vedno velja $\emptyset \subseteq S$ in $S \subseteq S$.

Princip ekstenzionalnosti za množice pravi:
%
\begin{equation*}
  S = T \iff (\all{x \in S} x \in T) \land (\all{y \in T} y \in S)
\end{equation*}
%
kar lahko zapišemo s podmnožicami:
%
\begin{equation*}
  S = T \iff S \subseteq T \land T \subseteq S.
\end{equation*}
%
Vsaka podmnožica $S \subseteq A$ opredeljuje neko lastnost elementov iz $A$: tisti
elementi, ki imajo opredeljeno lastnost, so v $S$, ostali pa ne.

\begin{primer}
  Naj bo $P$ množica vseh praštevil, torej je $P \subseteq \NN$. Podmnožica $P$
  opredeljuje lastnost ">je praštevilo"<.
\end{primer}


\subsection{Kako tvorimo podmnožice}

Če je $\phi(x)$ logična formula, v kateri nastopa spremenljivka $x \in A$, lahko tvorimo množico
%
\begin{equation*}
    \set{ x \in A \such \phi(x) }.
\end{equation*}
%
Pri tem je $x$ vezana spremenljivka. Za to množico velja:
%
\begin{equation*}
    a \in \set{ x \in A \such \phi(x) } \iff a \in A \land \phi(a).
\end{equation*}
%
Povedano z besedami: elementi množice $\set{ x \in A \such \phi(x) }$ so tisti elementi iz $A$, ki zadoščajo pogoju $\phi$.
%
Velja $\set{ x \in A \such \phi(x) } \subseteq A$, prav tako pa
\begin{equation*}
  \set{x \in A \mid \phi(x)} \subseteq \set{x \in A \mid \psi(x)} \iff
  \all{x \in A} \phi(x) \lthen \psi(x).
\end{equation*}


\subsection{Kanonična inkluzija}

Za podmnožico $S \subseteq T$ definiramo \textbf{kanonično inkluzijo} ali \textbf{kanonično vključitev} $i_S : S \to T$, s predpisom $i_S : x \mapsto x$. Pozor, to ni identiteta, razen v primeru $S = T$.
Oznaka $i_S$ ni standardna, pravzaprav standardne oznake ni.

Če je $f : T \to U$ in $S \subseteq T$, pravimo kompozitumu $f \circ i_S$ \textbf{zožitev} preslikave $f$ na $S$, pišemo $\restrict{f}{S}$.


\section{Potenčna množica}

\subsection{Definicija potenčne množice}

Za vsako množico $A$ tvorimo množico $\pow{A}$, ki ji pravimo \textbf{potenčna množica}.
Elementi potenčne množice $\pow{A}$ so natanko podmnožice množice $A$:
%
\begin{equation*}
    S \in \pow{A} \iff S \subseteq A
\end{equation*}
%
Na primer $\pow{\emptyset} = \set{\emptyset}$ in
%
\begin{equation*}
  \pow{\set{a,b,c}} = \set{\set{}, \set{a}, \set{b}, \set{c}, \set{a,b}, \set{a,c}, \set{b,c}, \set{a,b,c}}.
\end{equation*}


\subsection{Karakteristične funkcije}

\textbf{Karakteristična funkcija} na množici $A$ je funkcija z domeno $A$ in kodomeno $\two$. Tu je $\two = \set{\bot, \top}$ množica resničnostnih vrednosti.

Eksponentna množica $\two^A$ je torej množica vseh karakterističnih funkcij na $A$.

\begin{opomba}
  Karakteristične funkcije se uporabljajo tudi v analizi, kjer jih
  običajno razumemo kot preslikave $A \to \set{0,1}$ namesto $A \to \set{\bot, \top}$. Ker sta množici $\set{\bot,\top}$ in $\set{0,1}$ izomorfni, to ni bistvena razlika.
\end{opomba}

Karakteristično funkcijo si lahko predstavljamo kot preslikavo, ki opredeljuje
neko lastnost elementov~$A$: tisti elementi, ki imajo opredeljeno lastnost, se
slikajo v $\top$, ostali pa v $\bot$.

\begin{primer}
  Preslikava $p : \NN \to \two$, definirana s predpisom
  %
  \begin{equation*}
    p(n) = 
    \begin{cases}
      \top & \text{če je $n$ praštevilo}, \\
      \bot & \text{če $n$ ni praštevilo}.
    \end{cases}
  \end{equation*}
  %
  je karakteristična preslikava lastnosti ">je praštevilo"<. Lahko bi jo zapisali tudi takole:
  %
  \begin{equation*}
    p(n) = (n > 1 \land \all{k, m} n = k \cdot m \lthen k = 1 \lor m = 1).
  \end{equation*}
\end{primer}


\subsection{Izomorfizem $\pow{A} \cong \two^A$}

Videli smo, da lahko neko lastnost elementov množice $A$ predstavimo bodisi s
podmnožico bodisi s karakteristično preslikavo. To nam da idejo, da med
podmnožicami $A$ in karakterističnimi preslikavami na $A$ obstaja neka zveza.

\begin{izrek}
  $\pow{A} \cong \two^A$.
\end{izrek}

\begin{dokaz}
  Definirajmo preslikavi
  %
  \begin{align*}
    \chi &: \pow{A} \to \two^A &
    \xi &: \two^A \to \pow{A} \\
    \chi_S(x) &\defeq
      \begin{cases}
        \top & \text{če $x \in S$,} \\
        \bot & \text{če $x \not\in S$,}
      \end{cases}
    &
    \xi_f &\defeq \set{x \in A \such f(x) = \top}.
  \end{align*}
  %
  Ta predpisa bi lahko krajše zapisali tudi takole:
  %
  \begin{align*}
  \chi_S(x) &\defeq (x \in S), &
  \xi_f &\defeq \set{x \in A \such f(x) }.
  \end{align*}
  %
  Preslikavi $\chi_S$ pravimo \textbf{karakteristična funkcija podmnožice $S$}.
  %
  Trdimo, da sta $\chi$ in $\xi$ inverza:
  %
  \begin{enumerate}
  \item 
    Dokažimo $\chi \circ \xi = \id[\two^A]$. Uporabimo princip ekstenzionalnosti za preslikave.
    Naj bo $f \in \two^A$. Dokažimo, da je $\chi_{\xi_f} = f$.
    Uporabimo princip ekstenzionalnosti za preslikave. Naj bo $x \in A$:
    %
    \begin{equation*}
      \chi_{\xi_f}(x) = (x \in \xi_f) = f(x).
    \end{equation*}

  \item
    Dokažimo $\xi \circ \chi = id_{\pow{A}}$. Uporabimo princip ekstenzionalnosti za preslikave. Naj bo $S \in \pow{A}$. Dokažimo, da je $\xi_{\chi_S} = S$:
    %
    \begin{equation*}
      \xi_{\chi_S} = \set{x \in A \such \chi_S(x)} = \set{x \in A \such x \in S} = S.
    \end{equation*}
  \end{enumerate}
\end{dokaz}

\subsection{Boolova algebra podmnožic}

Podmnožice množice $A$ tvorijo Boolovo algebro za operacije presek $\cap$, unija $\cup$ in relativni komplement. Nevtralni element za unijo je $\emptyset$ in nevtralni element za presek je $A$.

Definirajmo tudi operacijo \textbf{simetrična razlika $\oplus$}, ki podmnožicama $S, T \in A$ priredi podmnožico
%
\begin{equation*}
  S \oplus T \defeq (S \setminus T) \cup (T \setminus S) = (S \cup T) \setminus (S \cap T).
\end{equation*}
%
Potenčna množica $\pow{A}$ je za operacijo $\oplus$ Abelova grupa.
