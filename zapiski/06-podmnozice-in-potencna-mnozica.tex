\chapter{Podmnožice in potenčne množice}

\subsection{Definicija relacije `⊆`}

Pravimo, da je množica `S` **podmnožica** množice `T`, pišemo `S ⊆ T`, ko velja
`∀ x ∈ S . x ∈ T`. Pravimo tudi, da je `S` **vsebovana** v `T` in da je `T`
**nadmnožica** `S`.

Vedno velja `∅ ⊆ S` in `S ⊆ S`.

Princip ekstenzionalnosti za množice pravi:

    S = T ⇔ (∀ x ∈ S . S ∈ T) ∧ (∀ y ∈ T . y ∈ S)

kar lahko zapišemo s podmnožicami:

    S = T ⇔ S ⊆ T ∧ T ⊆ S

Vsaka podmnožica `S ⊆ A` opredeljuje neko lastnost elementov iz `A`: tisti
elementi, ki imajo opredeljeno lasnost, so v `S`, ostali pa ne.

*Primer:* naj bo `P` množica vseh praštevil, torej je `P ⊆ N`. Podmnožica `P`
opredeljuje lasnost "je praštevilo".

\subsection{Kako tvorimo podmnožice}

Če je `φ(x)` logična formula, v kateri nastopa spremenljivka `x ∈ A`, lahko tvorimo množico

    { x ∈ A ∣ φ(x) }

Pri tem je `x` vezana spremenljivka. Za to množico velja:

    a ∈ { x ∈ A ∣ φ(x) } ⇔ a ∈ A ∧ φ(a)

Povedano z besedami: elementi množice `{ x ∈ A ∣ φ(x) }` so tisti elementi iz `A`, ki zadoščajo pogoju `φ`.

Velja `{ x ∈ A ∣ φ(x) } ⊆ A`.

Poleg tega velja

    {x ∈ A | φ(x)} ⊆ {x ∈ A | ψ(x)} ⇔ ∀ x ∈ A . φ(x) ⇒ ψ(x)

\subsection{Kanonična inkluzija}

Za podmnožico `S ⊆ T` definiriamo **kanonično inkluzijo** ali **kanonično vključitev**
`i_S : S → T`, s predpisom `i_S : x ↦ x` (to ni identiteta, razen v primeru `S = T`!).
Oznaka `i_S` ni standardna, pravzaprav standardne oznake ni.

Če je `f : T → U` in `S ⊆ T`, pravimo kompozitumu `f ∘ i_S` **zožitev* preslikave `f` na
`S`, pišemo `f|_S`.

\section{Potenčna množica}

\subsection{Definicija potenčne množice}

Za vsako množico `A` tvorimo množico `P(A)`, ki ji pravimo **potenčna množica**.
Elementi potenčne množice `P(A)` so natanko podmnožice množice `A`:

    S ∈ P(A) ⇔ S ⊆ A

*Primer:* `P(∅) = {∅}`

*Primer:* `P({a,b,c}) = {{}, {a}, {b}, {c}, {a,b}, {a,c}, {b,c}, {a,b,c}}`

\subsection{Karakteristične funkcije}

**Karakteristična funkcija** na množici `A` je fukcija z domeno `A` in kodomeno `2`.
Tu je `2 = {⊥, ⊤}` množica resničnostnih vrednosti.

Eksponentna množica `2^A` je torej množica vseh karakterističnih funkcij na `A`.

*Opomba:* karakteristične funkcije se uporabljajo tudi v analizi, kjer jih
običajno razumemo kot preslikave `A → {0,1}` namesto `A → {⊥, ⊤}`. Ker sta
množici `{⊥,⊤}` in `{0,1}` izomorfni, to ni bistvena razlika.

Karakteristično funkcijo si lahko predstavljamo kot preslikavo, ki opredeljuje
neko lastnost elementov `A`: tisti elementi, ki imajo opredeljeno lastnost, se
slikajo v `⊤`, ostali pa v ‵⊥`.

*Primer:* preslikava `p : N → 2`, definirana s predpisom

    p(n) = ⊤, če n je praštevilo
    p(n) = ⊥, če n ni praštevilo

je karakteristična preslikava lastnosti "je praštevilo".

\subsection{Izomorfizem `P(A) ≅ 2^A`}

Videli smo, da lahko neko lastnost elementov množice `A` predstavimo bodisi s
podmnožico bodisi s karakteristično preslikavo. To nam da idejo, da med
podmnožicami `A` in karakterističnimi preslikavami na `A` obstaja neka zveza.

**Izrek:** `P(A) ≅ 2^A`

*Dokaz.* Definirajmo preslikavi

    χ : P(A) → 2^A
    ξ : 2^A → P(A)

s predpisoma

    χ_S(x) := ⊥ če x ∉ S
    χ_S(x) := ⊤ če x ∈ S

in

    ξ_f := {x ∈ A | f(x) = ⊤}.

Ta predpisa bi lahko krajše zapisali tudi takole:

    χ_S(x) := (x ∈ S)
    ξ_f := {x ∈ A | f(x) }

Preslikavi `χ_S` pravimo **karakteristična funkcija podmnožice `S`**.

Trdimo, da sta `χ` in `ξ` inverza:

1. Dokažimo `χ ∘ ξ = id_{2^A}`. Uporabimo princip ekstenzionalnosti za preslikave.
   Naj bo `f ∈ 2^A`. Dokažimo, da je `χ_{ξ_f} = f`.
   Uporabimo princip ekstenzionalnosti za preslikave. Naj bo `x ∈ A`:

        χ_{ξ_f}(x) = (x ∈ ξ_f) = f(x).

2. Dokažimo `ξ ∘ χ = id_{P(A)}`. Uporabimo princip ekstenzionalnosti za preslikave.
   Naj bo `S ∈ P(A)`. Dokažimo, da je `ξ_{χ_S} = S`:

        ξ_{χ_S} = {x ∈ A | χ_S(x)} = {x ∈ A | x ∈ S} = S             □


\subsection{Boolova algebra podmnožic}

Podmnožice množice `A` tvorijo Boolovo algebro za operaciji presek `∩` in unija `∪`.

Boolova algebra množic (unija, presek, komplement).

Operacija simetrična razlika `⊕`. Potentčna množica tvori komutativno grupo za
to opreacijo.
