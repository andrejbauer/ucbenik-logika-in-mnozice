\chapter{Ekvivalenčne relacije in kvocientne množice}

\section{Ekvivalenčne relacije}

> **Definicija:** Relacija `R ⊆ A × A` je **ekvivalenčna relacija**, če je refleksivna,
> tranzitivna in simetrična. Kadar velja `x R y`, pravimo, da sta `x` in `y` **ekvivalentna** (glede na `R`).

*Opomba:* Kdor reče "ekvivalentna relacija", je noob. Kdor reče, da sta "`x` in `y`
ekvivalenčna", je rookie.

Ekvivalenčne relacije se običajno označuje s simboli, ki so podobni znaku za enakost:
`≡`, `∼`, `≃`, `≅`.

**Primeri:**

1. Relacija “vzporednost” med premicami v ravnini.
2. Relacija “skladnost" med trikotniki v ravnini.
3. Relacija “podobnost" med trikotniki v ravnini.
4. Relacija “isti ostanek pri deljenju s 7” na množici `ℕ`.
4. Prazna relacija `∅ ⊆ A × A` je ekvivalenčna le v primeru, da je `A = ∅`.
5. Polna relacija `A × A` je ekvivalenčna.
6. Diagonala (oz. enakost) je ekvivalenčna relacija.


\subsection{Ekvivalenčna relacija porojena s preslikavo}

Posebej pomemben je primer ekvivalenčne relacije **porojene (ali inducirane) s preslikavo**:
naj bo `f : A → B` preslikava in definirajmo relacijo `∼_f ⊆ A × A` s predpisom

    x ∼_f y ⇔ f(x) = f(y)

Tedaj je `∼_f` ekvivalenčna relacija:

* refleksivnost: `x ~_f x` velja, ker velja `f(x) = f(x)`,
* tranzitivnost: če je `x ~_f y` in `y ~_f z`, potem je `f(x) = f(y)` in `f(y) = f(z)`, torej `f(x) = f(z)` in `x ~_f z`,
* simetričnost: če je `x ~_f y`, potem je `f(x) = f(y)`, torej `f(y) = f(x)` in `y ~_f x`.

Ali je vsaka ekvivalenčna relacija porojena z neko preslikavo?

**Primer:** premici sta vzporedni natanko tedaj, ko imata enaka smerna vektorja. Če je
torej `P` množica vseh premic, `R²` množica vektorjev v ravnini, in `s : P → R²`
preslikava, ki premici `P` priredi njen enotski smerni vektor, ki leži v zgornji polravnini ali
na pozitivnem delu osi `x`, tedaj velja

    p ∥ q ⇔ s(p) = s(q)

Torej je vzporednost porojena s preslikavo `s`.


\section{Ekvivalenčni razredi in kvocientne množice}

**Definicija:** Naj bo `E ⊆ A × A` ekvivalenčna relacija. **Ekvivalenčni razred** elementa
`x ∈ A` je množica `[x]_A := { y ∈ A ∣ x E y }`. Z besedami: ekvivalenčni razred `x` je
množica vseh elementov, ki so mu ekvivalentni.

Opomba: Kdor reče “ekvivalentni razred”, je newbie.

**Definicija:** Naj bo `E ⊆ A × A` ekvivalenčna relacija. **Kvocientna množica** ali
**kvocient** `A/E` je množica vseh ekvivalenčnih razredov:

    A/E := { ξ ⊆ P(A) | ∃ x ∈ A . ξ = [x]_A }.

Z izpeljanimi množicami lahko to zapišemo bolj razumljivo:

    A/E = { [x]_E | x ∈ A }.


**Kanonična kvocientna preslikava** `q_E : A → A/E` je preslikava, ki vsakemu elementu
priredi njegov ekvivalenčni razred: `q_E(x) := [x]_A`.

Kvocientni množici včasih pravimo tudi *faktorska množica*.


**Izrek:** Vsaka ekvivalenčna relacija je porojena z neko preslikavo.

*Dokaz.* Naj bo `E` ekvivalenčna relacija na `A`. Najprej ugotovimo naslednje: za vse `x,
y ∈ A` velja

    x E y ⇔ [x]_E = [y]_E

Dokaz `⇒`: če je `x E y` potem je `[x]_E ⊆ [y]_E`, ker iz `z E x` in `x E y` sledi `z E y`.
Podobno dokažemo `[y]_E ⊆ [x]_E`.

Dokaz `⇐`: če je `[x]_E = [y]_E` potem je `y ∈ [y]_E = [x]_E`, torej po definiciji `[x]_E`
dobimo `x E y`.

Sedaj izrek sledi zlahka: `q_E(x) = q_E(y) ⇔ [x]_E = [y]_E ⇔ x E y` □


\subsection{Razdelitev množice}

**Definicija:** **Razdelitev** ali **particija** množice `A` je množica nepraznih, paroma
disjunktnih množic, ki tvorijo pokritje `A` (kar pomeni, da je `A` enaka njihovi uniji). Se
pravi, to je množica `S ⊆ P(A)`, za katero velja:

1. Elementi razdelitve so neprazni: `∀ B ∈ S . B ≠ ∅`
2. Vsaka dva elementa razdelitve sta bodisi enaka bodisi disjunktna: `∀ B, C ∈ S . B = C ∨ B ∩ C = ∅`
3. Elementi razdelitve tvorijo pokritje `A`: `A = ⋃ S`.

**Primer:**

* Navpične premice tvorijo razdelitev ravnine.
* Množici sodih in lihih števil tvorita razdelitev naravnih števil.
* Množica `{{1,2}, {3,5}, {4,6,7}}` tvori razdelitev `{1,2,3,4,5,6,7}`.
* Množica `{{1,2,3,4,5,6,7}}` tvori razdelitev `{1,2,3,4,5,6,7}`.

**Izrek:** Naj bo `E ⊆ A × A` ekvivalenčna relacija. Njeni ekvivalenčni razredi tvorijo
razdelitev množice `A`.

*Dokaz.*

1. Naj bo `ξ ∈ P(A)` ekvivalenčni razred za `E`. Tedaj obstaja `x ∈ A`, da je `ξ = [x]_A`,
   torej je `x ∈ ξ` in zato `ξ ≠ ∅`.

2. Naj bosta `ζ, ξ ∈ P(A)`. Dokazali bomo `ζ ∩ ξ ≠ ∅ ⇒ ζ = ξ`. Če je `x ∈ ζ ∩ ξ`, potem
   velja `ζ ⊆ ξ` ker: naj bo `y ∈ ζ`, tedaj je `y E x` in ker je `x ∈ ξ` velja `y ∈ ξ`.
   Simetrično dokažemo `ξ ⊆ ζ.

3. Očitno je unija vseh ekvivalenčnih razredov podmnožica `A`, saj je vsak ekvivalenčni
   razred podmnožica `A`. Zagotovo pa je vsak `x ∈ A` v kakem ekvivalenčnem razredu,
   namreč `x ∈ [x]_A`. □

Torej vsaka ekvivalenčna relacija na `A` določa razdelitev mnnožice `A`, namreč na
ekvivalenčne razrede. Velja pa tudi obrat: vsaka razdelitev `S ⊆ P(A)` določa ekvivalenčno
relacijo na `A`, namreč `≃_S` definiran s predpisom

    x ≃_S y ⇔ ∃ B ∈ S . x ∈ B ∧ y ∈ B

Z besedami: `x` in `y` sta ekvivalentna, kadar sta v istem elementu razdelitve. Prazvzaprav
smo ugotovili, da imamo izomorfizem množic:

    { E ⊆ A × A | E je ekvivalenčna relacija } ≅ { S ⊆ P(A) | S je razdelitev A }

V eno smer izomorfizem ekvivalenčni relaciji `E` priredi njeno razdelitev, v drugo pa
razdelitvi priredimo ekvivalenčno relacijo, kakor smo to opisali zgoraj. (Premislite, da sta
ti preslikavi inverza.)


\subsection{Prerezi kvocientne preslikave in aksiom izbire}

Ekvivalenčni razred je natanko določen že z enim od svojih elementov, zato pogosto želimo
namesto ekvivalenčnih razredov navesti le njihove predstavnike.

**Definicija:** Naj bo `E` ekvivalenčna relacija na `A`. Množico `C ⊆ A`, ki vsak
ekvivalenčni razred relacije `E` seka natanko enkrat, imenujemo **izbor predstavnikov
(ekvivalenčnih razredov) za relacijo `E`**.

Izbor predstavnikov `C ⊆ A` za `E` določa preslikavo `c : A/E → A`, ki priredi
ekvivalenčnemu razredu `ξ` tisti `x ∈ ξ`, ki je element `C`:

    c : A/E → A
    c : ξ ↦ (ι x ∈ ξ . x ∈ C)

Preslikava `c : A/E → A` je *prerez* kvocientne preslikave `q_E : A → A/E`.

**Trditev:** Če je `s : A/E → A` prerez kvocientne preslikave `q_E : A → A/E`, potem je
njegova slika `s_*(A/E) = { c(ξ) | ξ ∈ A/E }` izbor predstavnikov za `E`.

Dokaz: Vaja. □

Ker izbor predstavnikov in prerez kvocientne preslikave določata drug drugega, včasih tudi
prerez imenujemo “izbor predstavnikov”.


**Primer:** Definirajmo `∼` na množici celih števil `Z` s predpisom

    a ∼ b ⇔ 7 | a - b

Torej sta števili `a` in `b` ekvivalentni, če dasta enak ostanek pri deljenju s 7,
na primer `13 ~ 20` in `¬ (13 ~ 15)`.

Ekvivalenčni razred števila `a` dobimo tako, da `a` prištejemo vse večkratnike števila `7`:

    [a]_∼ = { a + 7 · k | k ∈ ℤ }

Na primer,

    [13]_~ = { 7 · k + 13 | k ∈ ℤ }
           = { ..., -22, -15, -8, -1, 6, 13, 20, 27, 34, 41, ...}

Koliko pa je ekvivalenčnih razredov? Toliko, kot je ostankov pri deljenju s 7, torej 7.
Izbor predstavnikov za `~` je torej množica `{0, 1, 2, 3, 4, 5, 6}`, saj je vsako celo
število ekvivalentno natanko enemu od teh števil mo modulu `7`.

Ni pa to edini izbor! Tudi `{0, 1, 2, 3, 4, 5, 6, 13}` je izbor, prav tako pa `{-7, -6, -5, -4, -3, -2, -1}`.

(Konec primera.)

Ali ima vsaka ekvivalenčna relacija izbor predstavnikov? Da to vprašanje ni tako
enostavno, kot se zdi na prvi pogled, doma premislite o nalslednji nalogi.

**Naloga:** Na množici realnih števil `ℝ` definiramo relacijo `E` s predpisom

    x E y  ⇔  x - y ∈ ℚ

Se pravi, da sta števili ekvivalentni, če je njuna razlika racionalno število. Podajte kak
izbor predstavnikov za `E`.


**Izrek:** Naslednje izjave so ekvivalentne:

1. Vsaka surjektivna preslikava ima desni inverz (prerez).
2. Vsaka ekvivalenčna relacija ima izbor predstavnikov.
3. Vsaka družina nepraznih množic ima funkcijo izbire.
4. Produkt družine nepraznih množic je neprazen.

*Dokaz.*

`(1 ⇒ 2)`: Naj bo `E ⊆ A × A` ekvivalenčna relacija na `A`. Tedaj je `q_E : A → A/E`
surjektivna, zato ima po predpostavki (1) prerez, ki določa izbor predstavnikov.

`(2 ⇒ 3)`: Naj bo `A : I → Set` družina nepraznih množic. Naj bo `∼` ekvivalenčna relacija
na koproduktu `K := ∐_{i ∈ I} A_i`, porojena s prvo projekcijo `pr₁ : S → I`, t.j.,

    inᵢ(x) ∼ inⱼ(y) ⇔ i = j

Po predpostavki (2) obstaja izbor predstavnikov za `∼`, se pravi taka množica `C ⊆ K`, da
za vsak `u ∈ K` obstaja natanko en `v ∈ C`, da je `pr₁(u) = pr₁(v)`. Definirajmo `f : I →
⋃ A` s predpisom

    f(i) := tisti x ∈ A_i, za katerega je inᵢ(x) ∈ C

Očitno je `f` funkcija izbire za družino `A`, če je le dobro definirana:

* `f` je enolična, saj iz `inᵢ(x) ∈ C` in `inᵢ(y) ∈ C` sledi `inᵢ(x) = inⱼ(y)`.
* `f` je celovita: ker je `A_i` neprazna, obstaja `z ∈ A_i`, torej obstaja `v ∈ C`, da je
  `i = pr₁(inᵢ(z)) = pr₁(v)`, in je potemtakem `pr₂(v) ∈ A_i` element, za katerega velja
  `inᵢ(pr₂(v)) ∈ C`.

`(3 ⇒ 4)`: Elementi produkta so funkcije izbire, zato je produkt res neprazen, če obstaja
kaka funkcija izbire.

`(4 ⇒ 1)`: Naj bo `f : X → Y` surjektivna. Definirajmo družino `A : Y → Set` s
predpisom `A_y = f^*({y})`. Ker je `f` surjektivna, je `A` družina nepraznih
množic. Po predpostavki (4) je produkt te družine neprazen, torej vsebuje neko
funkcijo izbire `c : Y → ⋃ A_y`, se pravi, da je `f(c(y)) = y` za vsak `y ∈ Y`.
Opazimo še, da je `⋃ A = Y`, torej je `c` prerez `f`. □

Izbor prestavnikov je torej ekvivalenten še nekaterim drugim trditvam. Pa te veljajo? Za
to potrebujemo aksiom.

> **Aksiom izbire:** Vsaka družina nepraznih množic ima funkcijo izbire.

Se pravi, če je `A : I → Set` taka družina množica, da za vsak `i ∈ I` velja `A_i ≠ ∅`,
tedaj obstaja `f : I → ⋃ A`, za katerega je `f(i) ∈ A_i` za vse `i ∈ I`.

O aksiomu izbire bomo še govorili.

\subsection{Univerzalna lastnost kvocientne množice}

Naj bo `E` ekvivalenčna relacija na `A` in `B` množica. Pogosto želimo definirati
preslikavo

    f : A/E → B

s pomočjo preslikave `A → B`. Kdaj lahko to naredimo?

**Izrek:** Naj bo `E` ekvivalenčna relacija na `A` in `g : A → B` preslikava, ki je
*skladna* z `E`, kar pomeni da `g` slika ekvivalentne elemente v enake, se pravi `∀ x, y ∈
A . x E y ⇒ g(x) = g(y)`. Tedaj obstaja natanko ena preslikava `f : A/E → B`, da je
`f([x]_E) = g(x)` za vse `x ∈ A`, ali drugače povedano, `f ∘ q_E = g`.

*Dokaz.*

Dokažimo najprej, da imamo največ eno tako preslikavo. Denimo da za `f₁ : A/E → B` in
`f₂ : A/E → B` velja `f₁ ∘ q_E = f₂ ∘ q_E`. Ker je `q_E` surjektivna, je epi in jo smemo
krajšati na desni, od koder res sledi `f₁ = f₂`.

Sedaj dokažimo, da `f` obstaja. V ta namen naj bo `φ ⊆ A/E × B` relacija

    φ(ξ, y) ⇔ ∃ x ∈ A . x ∈ ξ ∧ g(x) = y

Trdimo, da je `φ` funkcijska relacija:

* enoličnost: če je `φ(ξ, y₁)` in `φ(ξ, y₂)`, potem obstajata `x₁, x₂ ∈ ξ`, da je `g(x₁) =
  y₁` in `g(x₂) = y₂`. Ker pa velja `x₁ E x₂` in je `g` skladna z `E`, sledi `y₁ = g(x1) =
  g(x₂) = y₂`.

* celovitost: naj bo `ξ ∈ A/E`. Tedaj obstaja `x ∈ ξ`. Očitno velja `g(ξ, g(x))`.

Naj bo `f : A/E → B` preslikava, ki je določena s funkcijsko relacijo `φ`. Za `x ∈ A`
velja `φ([x]_E, f([x]_E))`, od tod pa iz definicije `φ` sledi tudi `g(x) = f([x]_E)`. □


\section{Kanonična razčlenitev preslikave}

Naj bo `f : A → B` preslikava. Naj bo `∼_f` ekvivalenčna relacija na `A`, ki jo porodi
`f`, in `q_f : A → A/E` kanonična kvocientna preslikava (morali bi jo pisati `q_{∼_f}`,
kar je nečitljivo). Naj bo `i : f_*(A) → B` kanonična inkluzija slike `f` v kodomeno.
Preslikava `f : A → f_*(A)` je skladna s `∼_f`, zato obstaja (natanko ena) preslikava
`b_f : A/f → f_*(A)`, da velja `b_f([x]_∼) = f(x)`. Trdimo:

1. `f = i_f ∘ b_f ∘ q_f`
2. `q_f` je surjektivna, `b_f` je bijektivna in `i_f` je injektivna.

Računajmo: `f(x) = b_f([x]_~) = i_f(b_f([x]_~)) = i_f(b_f(q_f(x)))`, za vse `x ∈ A`, od
koder sledi prva trditev.

Vemo že, da je kanonična kvocientna preslikava surjektivna in kanonična inkluzija
injektivna. Ostane nam še bijektivnost preslikave `b_f`:

* `b_f` je injektivna: naj bosta `ξ, ζ ∈ A/(∼_f)` in denimo, da velja `b_f(ξ) = b_f(ζ)`.
  Obstajata `x, y ∈ A`, da je `ξ = [x]_∼` in `ζ = [y]_∼`. Velja

        f(x) = i_f(b_f(q_f(x))) = i_f(b_f(ξ)) = i_f(b_f(ζ)) = i_f(b_f(q_f(y))) = f(y)

  torej je `x ∼_f y` in zato `ξ = [x]_∼ = [y]_∼ = ζ`.

* `b_f` je surjektivna: naj bo `u ∈ f_*(A)`. Tedaj obstaja `x ∈ A`, da je `u = f(x)`.
  Vzemimo `ξ = [x]_E` in preverimo: `b_f(ξ) = b_f([x]_~) =f(x) = u`. □
