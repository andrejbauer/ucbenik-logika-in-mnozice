\documentclass[11pt,a4paper,twoside]{book}

%--------------------------------------------------------------------
%-- Velikost strani

%% A4 stran = 210mm x 297mm

\usepackage[a4paper,
            top=3cm,
            bottom=4cm,
            inner=4cm,
            outer=4cm
           ]{geometry}

\usepackage[T1]{fontenc}
\usepackage[utf8]{inputenc}
\usepackage[slovene]{babel}
\usepackage[colorlinks]{hyperref}

\usepackage{xcolor}
\usepackage{amsfonts,amssymb,amsmath,amsthm}

\usepackage[scaled=.95]{newpxtext}
\usepackage[vvarbb,smallerops]{newpxmath} % Matching math
\usepackage[scaled=.85]{beramono} % Monospace

\usepackage{textgreek} % Za grške črke v besedilu
\usepackage{phonetic} % Fonetični zapis

\usepackage{answers} % rešitve vaj

\usepackage{datetime2}
\usepackage{tablefootnote}

\usepackage{bbold}
\usepackage{fancyhdr}
\usepackage{mathpartir}
\usepackage{proof}
\usepackage{xypic}

\usepackage{tikz}
\usepackage{tkz-euclide}
\usetikzlibrary{calc}

\usepackage{booktabs}

\usepackage{enumitem}
\setlist{nosep}

%--------------------------------------------------------------------
% Stil poglavij
\usepackage[explicit]{titlesec}

\titleformat{\chapter}[hang]
  {\normalfont\huge\bfseries\raggedright} % format
  {\thechapter}                % label (number only)
  {1em}                        % space between number and title
  {#1}                         % title

% Adjust spacing before/after chapter titles
\titlespacing*{\chapter}{0pt}{50pt}{30pt}

%--------------------------------------------------------------------
%-- Barve hiper povezav

\hypersetup{
    colorlinks,
    linkcolor={red!50!black},
    citecolor={blue!50!black},
    urlcolor={blue!80!black}
}

%--------------------------------------------------------------------
%-- Okolja

{
  \theoremstyle{plain}

  \newtheorem{izrek}{Izrek}[chapter]
  \newtheorem{lema}[izrek]{Lema}
  \newtheorem{izjava}[izrek]{Izjava}
  \newtheorem{trditev}[izrek]{Trditev}
  \newtheorem{posledica}[izrek]{Posledica}
  \newtheorem{hipoteza}[izrek]{Hipoteza}
  \newtheorem{aksiom}[izrek]{Aksiom}
  \newtheorem{pravilo}[izrek]{Pravilo}
}

{
  \theoremstyle{definition}
  \newtheorem{definicija}[izrek]{Definicija}
  \newtheorem{zgled}[izrek]{Zgled}
  \newtheorem{opomba}[izrek]{Opomba}
  \newtheorem{vaja}[izrek]{Vaja}
}

\Newassociation{resitev}{Resitev}{resitve}
\renewcommand{\Resitevlabel}[1]{\emph{Re\v{s}itev~#1}}


%\newcommand{\qedsign}{{\vrule width 1ex height 1ex depth 0ex}}
%\newcommand{\qed}{\hfill\qedsign}

%--------------------------------------------------------------------
\newcommand{\nls}[1]{\guillemotright\textit{#1}\guillemotleft} % naravni jezik
\newcommand{\df}[1]{\emph{\textbf{#1}}}  % definiranec
\newcommand{\dfeq}{\mathrel{{:}{=}}}  % definicijska enačba

\newcommand{\konst}[1]{\mathrm{k}_{#1}} % konstantna preslikava


% Substitucija
\newcommand{\xsubst}[2]{#1[#2]}
\newcommand{\subst}[2]{(#1)[#2]}
\newcommand{\subto}{\,{\mapsto}\,}

%%% quantifiers
\newcommand{\all}[1]{\forall #1 \,.\,}
\newcommand{\some}[1]{\exists #1 \,.\,}
\newcommand{\exactlyone}[1]{\exists! #1 \,.\,}
\newcommand{\descr}[1]{\iota #1 \,.\,}

\newcommand{\defeq}{\mathbin{{:}{=}}}
\newcommand{\defiff}{\quad {{:}{\Longleftrightarrow}}\quad }

\newcommand{\brac}[1]{\dots $\langle$#1$\rangle$ \dots}

%\newcommand{\lam}[3]{\lambda #1 \,{\in}\, #2\,.\left(#3\right)}

% \newcommand{\uall}[2]{\forall\, #1\,.\left(#2\right)}
% \newcommand{\usome}[2]{\exists\, #1\,.\left(#2\right)}
% \newcommand{\uexactlyone}[3]{\exists!\, #1\,.\left(#2\right)}
% \newcommand{\ulam}[2]{\lambda #1 .\left(#2\right)}

% \newcommand{\xall}[3]{\forall\, #1 \,{\in}\, #2\,.\,#3}
% \newcommand{\xsome}[3]{\exists\, #1 \,{\in}\, #2\,.\,#3}
% \newcommand{\xexactlyone}[3]{\exists!\, #1 \,{\in}\, #2\,.\,#3}
% \newcommand{\xlam}[3]{\lambda #1 \,{\in}\, #2\,.\,#3}

% \newcommand{\xuall}[2]{\forall\, #1\,.\,#2}
% \newcommand{\xusome}[2]{\exists\, #1\,.\,#2}
% \newcommand{\xuexactlyone}[2]{\exists!\, #1,.\,#2}
% \newcommand{\xulam}[2]{\lambda #1 .\,#2}

% \newcommand{\tlam}[3]{\lambda #1 \,{:}\, #2\,.\,\left(#3\right)}
% \newcommand{\xtlam}[3]{\lambda #1 \,{:}\, #2\,.\,#3}

\newcommand{\iinn}{\qquad\text{in}\qquad}

% Implikacija
\newcommand{\lthen}{\Rightarrow}
\newcommand{\liff}{\Leftrightarrow}

% Sklep iz hipoteze
\newcommand{\hypoinfer}[2]{\begin{array}[b]{c}{#1}\\ \vdots\\{#2}\end{array}}

% Standardne množice
\newcommand{\NN}{\mathbb{N}}
\newcommand{\ZZ}{\mathbb{Z}}
\newcommand{\QQ}{\mathbb{Q}}
\newcommand{\RR}{\mathbb{R}}
\newcommand{\CC}{\mathbb{C}}

\newcommand{\one}{\mathbb{1}}
\newcommand{\two}{\mathbb{2}}
\newcommand{\unit}{(\,)}

%%% Množice
\newcommand{\set}[1]{\left\{#1\right\}}
\newcommand{\such}{\; \middle| \;}
\newcommand{\iso}{\cong}

%%%%%%  Projections and Injections
%%%%%%%%%%%%%%%%%%%%%%%%%%%%%%%%%%%%%%%%%%%%%%%%%%%%%%%%%%%%%
\newcommand{\inj}[1][]{\mathsf{in}_{#1}}
\newcommand{\pr}[1][]{\mathsf{pr}_{#1}}

\NewDocumentCommand{\fst}
{O{\empty} O{\empty}}
{\pr[1]^{{#1}\ifthenelse{\equal{#2}{}}{}{,}{#2}}}
\NewDocumentCommand{\snd}
{O{\empty} O{\empty}}
{\pr[2]^{{#1}\ifthenelse{\equal{#2}{}}{}{,}{#2}}}
\NewDocumentCommand{\inl}
{O{\empty} O{\empty}}
{\inj[1]^{{#1}\ifthenelse{\equal{#2}{}}{}{,}{#2}}}
\NewDocumentCommand{\inr}
{O{\empty} O{\empty}}
{\inj[2]^{{#1}\ifthenelse{\equal{#2}{}}{}{,}{#2}}}


%%% Naravna števila
\newcommand{\suc}[1]{{#1^{+}}}

\newcommand{\Set}{\mathsf{Set}}
\newcommand{\On}{\mathsf{On}}

%% Drevesa
\newcommand{\Tree}{\mathsf{Tree}}
\newcommand{\tree}[1]{\mathsf{tree}(#1)}
\newcommand{\emptyTree}{\mathsf{empty}}

%%% Operacije na množicah

\newcommand{\pow}[1]{\mathcal{P}(#1)}       % potenčna množica
\newcommand{\npow}[2]{\mathcal{P}_{#1}(#2)} % podmnožice dane velikosti

%% Preslikave
\newcommand{\id}[1][]{\mathrm{id}_{#1}}
\newcommand{\inv}[1]{{#1}^{-1}}
\newcommand{\restrict}[2]{#1{\big|}_{#2}}

\newcommand{\diag}[1][]{\Delta_{#1}}

\newcommand{\img}[1]{#1_{*}}
\newcommand{\invimg}[1]{#1^{*}}

%% Kombinatorika
\newcommand{\ceil}[1]{\lceil #1 \rceil}

%% Relacije
\newcommand{\transpose}[1]{#1^{\mathsf{T}}}
\newcommand{\compl}[1]{#1^{\mathsf{C}}}
\newcommand{\rel}[1]{\mathrel{#1}}



%--------------------------------------------------------------------
%-- Glava in dno

\pagestyle{fancyplain}

%\setlength{\headrulewidth}{0.2pt}
\addtolength{\headheight}{2pt}

\renewcommand{\chaptermark}[1]{\markboth{#1}{}}
\renewcommand{\sectionmark}[1]{\markright{\thesection\ #1}}

\lhead[\fancyplain{}{{\thepage}}]%
      {\fancyplain{}{{\rightmark}}}
\rhead[\fancyplain{}{{\leftmark}}]%
      {\fancyplain{}{\thepage}}
\cfoot{\footnotesize [verzija \today]}
\lfoot[]{}
\rfoot[]{}

%--------------------------------------------------------------------
% NASLOV

\author{Andrej Bauer \and Davorin Lešnik}
\title{Logika in množice \\ (Zapiski v nastajanju)}

\begin{document}

\maketitle

\cleardoublepage

%--------------------------------------------------------------------
% KAZALO
\pagestyle{fancyplain}

{
\renewcommand{\markboth}[2]{}
\tableofcontents
}

\cleardoublepage

%--------------------------------------------------------------------
% VSEBINA

\chapter*{Predgovor}
\addcontentsline{toc}{chapter}{Predgovor}

Pri predmetu Logika in množice v prvem letniku študija matematike na Fakulteti za matematiko in fiziko Univerze v Ljubljani se študenti učijo osnov matematičnega izražanja---kako beremo in pišemo matematično besedilo in simbolni zapis---hkrati pa spoznavajo temelje matematične logike in teorije množic.
%
Pred študenti matematike je torej težka naloga učiti se novo snov v novem žargonu.

Da bo učbenik v pomoč, bomo pri matematičnem izražanju bolj natančni, kot je to običajno za matematično besedilo.
Pojasnjevali bomo, kako matematiki pišejo, govorijo in razmišljajo v praksi ter marsikaj raje zapisali na dolgo, da bo začetniku bolj prijazno.
%
Bližnjice, ki jih ubirajo izkušeni matematiki, bomo vpeljali zlagoma, hkrati pa opozarjali na nedoslednosti, ki so
večinoma ostanki zgodovinskega razvoja matematike in ki se jim matematična tradicija stežka odreče.
%
Ne zamerite nam, če dobrohotno ponudimo še kak nasvet o študiju matematike.


\paragraph{Zahvala.}
%
Za pomoč pri urejanju zapiskov in opozarjanje na napake se zahvaljujeva študentkam in študentom:
%
Luka Debevc,
Milan Djaković,
Ema Grmšek,
Matija Fajfar,
Miha Gyergyek,
Peter Jereb,
Jan Kastelic,
Jan Malej,
Matej Marinko,
Jan Pantner,
Lev Rus,
Jakob Schrader,
Matija Sirk,
Matej Šafarič,
Gal Zmazek,
Marjetka Zupan in Patrik Žnidaršič.
%
Vse preostale napake so najina last.
\bigskip

\begin{flushright}
Andrej Bauer in Davorin Lešnik
\end{flushright}

\bigskip


%%% Local Variables: 
%%% mode: latex
%%% TeX-master: "lmn"
%%% End: 

%\chapter{Matematično izražanje}
\label{cha:matematicno-izrazanje}

Tako kot vsaka stroka ima tudi matematika svoj strokovni jezik, ki obsega matematične
simbole in izraze ter svojevrsten način izražanja. Matematiki stremimo k popolni
natančnosti in nedvoumnosti matematične misli. To je seveda le ideal, ki se mu bolj ali
manj približamo, dejanska matematična besedila pa pišemo ljudje za ljudi, zato ni nič
nenavadnega, da so prežeta s tradicijo in nepisanimi družbenimi dogovori, ki matematiko
oddaljijo od formalnega ideala, a jo tudi naredijo humano.
%
Pred študentom matematike je torej težka naloga, saj se mora hkrati z novo matematiko
učiti še nekoliko nenavadnega jezika. V pomoč se zato najprej posvetimo samo formi
matematičnega izražanja. In ne zamerite nam, če vam dobrohotno ponudimo še kak nasvet o
študiju matematike.

Matematično komuniciranje je raznoliko, saj je namenjeno različnim publikam in zato
posredovano na različne načine. Tako v raziskovalnem matematičnem članku ne bomo našli
pojasnil in izračunov, ki jih profesor matematike zahteva od svojih študentov. In verjetno
ni dveh matematikov, ki bi uporabljala povsem usklajen matematični zapis in izrazoslovje.
Kljub temu je matematični jezik skupen vsem matematikom in v večji meri poenoten.
Nesporazume, ki nastopijo zaradi različnih navad, pa lahko rešimo s pogovorom. Vsi
izkušeni matematiki vedo, da vedo zelo malo in zato vprašajo, ko česa ne vedo. To naj bo
torej prvi nasvet: vprašajte in če ne dobite odgovora, vprašajte še enkrat.

Ker je namen tega učbenika postaviti dobre osnove matematičnega izražanja in mišljenja,
bomo bolj natančni kot večina matematikov v praksi. Začetnik namreč potrebuje oporo v
natančnosti, kasneje, ko razume stvari bolje, pa lahko ubere bližnjice, ki jih bolj
izkušeni kolegi uporabljajo, ne da bi to sploh opazili. Sproti bomo opozarjali nanje,
kakor tudi na manjše nedoslednosti v matematični praksi, ki izhajajo iz zgodovinskega
razvoja matematike.

% * Sestavni deli besedila, brez podrobnih razlag, morda primeri, tu samo opozorimo na
%   raznovrstnost konceptov.
%    * spremno besedilo
%    * konstrukcije
%    * računi
%    * izjave (sinonimi, kako jih številčimo)
%    * dokazi (kako so označeni)
%    * definicije
%    * zgledi
%    * naloge in rešitve (namigi)
%    * formule in izrazi (kako jih številčimo in kako se nanje sklicujemo)
%    * citati in reference


% PRIMERI DRUŽBENIH DOGOVOROV

% $\vec{a}$ uporabljamo za oznako vektorja

% Pri algebri v prvem letniku na FMF je običaj, da se linearno preslikavo označi z
% $\mathcal{A}$, matriko, ki tej linearni preslikavi pripada, pa z $A$.

\section{Pisave in simboli}
\label{sec:pisave-in-simboli}

Matematična abeceda vsebuje precej več simbolov, kot zgolj običajne črke in števke. Nekatere že poznamo, na primer~$=$, $<$, $+$, $\emptyset$, $\cup$, $\cap$, $\int$ in tako naprej, precej jih še bomo spoznali. Poleg tega matematiki uporabljamo različne pisave, kot je prikazano v tabeli~\ref{tabela:oblike-crk}. Na tabli in v zvezku sicer težko ločimo med pokončno, odebeljeno in ležečo pisavo, ali med kaligrafsko in rokopisno, zato nabor pisav omejimo. V tiskanem besedilu se vedno držimo nekaterih pravil glede izbire pisav. Tako posamezne črke $a$, $b$, $c$, \ldots, $x$, $y$, $z$ pišemo v ležeči pisavi, imena elementarnih funkcij pa pokončno: $\sin$, $\cos$, $\log$, \ldots Šumnikov običajno ne uporabljamo. Včasih z uporabo znakov nakažemo povezavo med dvema objektoma: $f$ je funkcija in $F$ njen integral, $\mathcal{A}$ je linearna preslikava in $A$ njej pripadajoča matrika itd.

\begin{table}[ht]
\centering
\begin{tabular}{c|c}
\textbf{Pisava} & \textbf{Črke} \\
\hline
pokončna & $\mathrm{ABCDEFGHIJKLMNOPQRSTUVWXYZ}$ \\
odebeljena & $\mathbf{ABCDEFGHIJKLMNOPQRSTUVWXYZ}$ \\
ležeča & $ABCDEFGHIJKLMNOPQRSTUVWXYZ$ \\
kaligrafska & $\mathcal{ABCDEFGHIJKLMNOPQRSTUVWXYZ}$ \\
rokopisna & $\mathscr{ABCDEFGHIJKLMNOPQRSTUVWXYZ}$ \\
frakturna & $\mathfrak{ABCDEFGHIJKLMNOPQRSTUVWXYZ}$ \\
dvopoudarjena & $\mathbb{ABCDEFGHIJKLMNOPQRSTUVWXYZ}$
\end{tabular}
\caption{Pisave}\label{tabela:oblike-crk}
\end{table}

Črke lahko dodatno opremimo s črticami, vijugami, vektorskimi znaki, strešicami in podobno:
%
\begin{equation*}
 a \quad
 a' \quad
 \dot{a} \quad
 \bar{a} \quad
 \vec{a} \quad
 \tilde{a} \quad
 \hat{a} \quad
 \check{a}.
\end{equation*}
%
Uporabimo lahko tudi \emph{podpis} ali
\emph{nadpis}, ki je lahko črka, številka, ali kak drug simbol, na primer
%
\begin{equation*}
  a_i \quad
  a^i \quad
  a_1 \quad
  a_{\star} \quad
  a^{\dagger}.
\end{equation*}
%
Podpisu in nadpisu pogovorno pravimo tudi \emph{indeks} in \emph{eksponent}, a to ni
najbolj posrečena raba, ker se indeks lahko pojavi tudi v nadpisu ali kje drugje,
eksponent pa lahko pomeni tudi število, s katerim potenciramo.

Kljub temu obilju črk in oznak posežemo še po drugih abecedah, še posebej grški, zato se
jo čimprej naučite! Grške črke skupaj z njihovo izgovorjavo najdete v
tabeli~\ref{tabela:grska-abeceda}. Prostoročni zapis grških črk se boste naučili v
razredu.
%
Pa tudi to matematikom še ni dovolj! V teoriji množic uporabljamo še hebrejske črke
alef~$\aleph$, bet~$\beth$ in gimel~$\gimel$.

\begin{table}[ht]
\begin{center}
\begin{tabular}{cc|cc}
\multicolumn{2}{c|}{\textbf{Grška črka}} & \multicolumn{2}{c}{\textbf{Izgovorjava}} \\
\textit{velika} & \textit{mala} & \textit{v slovenščini} & \textit{v grščini} \\
\hline
A & $\alpha$ & alfa & alfa \\
B & $\beta$ & beta & vita \\
$\Gamma$ & $\gamma$ & gama & {\textgamma}ama \\
$\Delta$ & $\delta$ & delta & delta \\
E & $\epsilon$, $\varepsilon$ & epsilon & epsilon \\
Z & $\zeta$ & zeta & zita \\
H & $\eta$ & eta & ita \\
$\Theta$ & $\theta$, $\vartheta$ & {\scriptsize\textTheta}eta & {\scriptsize\textTheta}ita \\
I & $\iota$ & jota & jota \\
K & $\kappa$ & kapa & kapa \\
$\Lambda$ & $\lambda$ & lambda & lamda \\
M & $\mu$ & mi & mi \\
N & $\nu$ & ni & ni \\
$\Xi$ & $\xi$ & ksi & ksi \\
O & $\omicron$ & omikron & omikron \\
$\Pi$ & $\pi$, $\varpi$ & pi & pi \\
P & $\rho$, $\varrho$ & ro & ro \\
$\Sigma$ & $\sigma$, $\varsigma$ & sigma & si{\textgamma}ma \\
T & $\tau$ & ta\hill{u} & taf \\
$\Upsilon$ & $\upsilon$ & ipsilon & ipsilon \\
$\Phi$ & $\phi$, $\varphi$ & fi & fi \\
X & $\chi$ & hi & {\textchi}i \\
$\Psi$ & $\psi$ & psi & psi \\
$\Omega$ & $\omega$ & omega & ome{\textgamma}a \\
\end{tabular}
\end{center}
\par\medskip
\footnotesize{
Izgovorjava: \hill{u} je ustnični u (kot v besedi `pav');
{\textgamma} je cerkljanski `g' (nekaj med `g' in `h' --- vprašajte sošolce s tega območja);
{\scriptsize\textTheta} je angleški nezveneči `th' (kot v besedi `thing');
{\textchi} je nemški `ch' (kot v besedi `ich').}
\caption{Grška abeceda.}
\label{tabela:grska-abeceda}
\end{table}

In zakaj pravzaprav potrebujemo tako veliko število črk? Verjetno zato, ker je v
matematiki krajši zapis bolj učinkovit, saj zasede manj prostora na papirju, pa še hitreje
ga zapišemo in preberemo. Računalničarji imajo drugačne navade, saj pri njih velja, da naj
se uporablja opisna imena, ki razkrijejo pomen: kjer bi matematik in fizik uporabila~$m$
in~$a$, bi računalničar zapisal $\mathtt{masa\_delca}$ in $\mathtt{pospesek}$.

\section{Izrazi}
\label{sec:irazi}

Matematično besedilo je mešanica naravnega jezika in simbolnega zapisa. Delom besedila, ki
so napisani s simboli, pravimo \emph{simbolni izrazi} ali krajše kar \emph{izrazi}. Vsi
ste jih že videli, denimo
%
\begin{equation*}
  (3 + 4) \cdot 6 \qquad\quad
  \int_0^1 \frac{x}{1 + x^2} \, dx \qquad\quad
  a x^2 + b x + c = 0 \qquad\quad
  x > 0 \lor x \leq 0
\end{equation*}
%
Ste se kdaj vprašali, zakaj pravzaprav pišemo ulomke z vodoravno črto, integral z znakom
$\int$, zakaj ima množenje prednost pred seštevanjem in zakaj seštevamo od leve proti
desni, čeprav bi lahko tudi v drugi smeri? Odgovor je vedno isti: to so splošno sprejete
navade, ki so se izoblikovale v razvoju matematike. To niso matematične resnice, ampak
\emph{dogovori} med ljudmi, ki se jih držimo zato, ker so se izkazali za smiselne. Na
primer, integralski znak $\int$ je Leibniz\footnote{Gottfried Wilhelm von Leibniz
  (1646--1716) je bil nemški filozof, matematik, fizik, pravnik, zgodovinar, jezikoslovec,
  knjižničar in diplomat lužiško sorbskega porekla.} izpeljal iz črke S, ker je na
integral gledal kot na določene vrste vsoto (latinsko `summa').

Z vidika vsebine raznolikost matematičnega zapisa ni potrebna, saj bi lahko vse izraze
pisali na isti način. Namesto simbolov, kot so $+$, $-$ in $\sqrt{\ }$, bi lahko
uporabljali besede $\mathtt{plus}$, $\mathtt{minus}$, $\mathtt{sqrt}$ in jih zapisovali
kot preslikave. Tak zapis je preprost in enoten, saj se nam ni treba ukvarjati s
predponami, medponami in priponami ter z levim in desnim združevanjem. Uporablja se v
računalništvu, a kdo bi želel na tablo namesto $3 + \sqrt{5 - 4}$ zapisati
$\mathtt{plus(3, \mathtt{sqrt}(\mathtt{minus}(5, 4)))}$?

Ni vsako zaporedje znakov pravilen izraz. Denimo, $3 + ) x \cdot 4$ ni pravilen izraz, ker ima narobe postavljen zaklepaj.
%
Izraz je \df{pravilno formiran} ali \df{sintaktično pravilen}, če ustreza pravilom, ki določajo kako postavljamo oklepaje, vejice, pike, kako uporabljamo razne posebne simbole ($+$, $\lor$, $\int$) itd.
%
Natančna \df{sintaktična pravila} za pisanje matematičnih izrazov so precej zapletena, ker je matematični zapis raznovrsten
in se je razvijal skozi zgodovino. Mnoga že poznate (">vsak oklepaj mora imeti ustrezni zaklepaj"<, ">piše se $a + b$ in ne $a b +$"<), zato jih ne bomo podrobno obravnavali -- to je delo za računalničarje, ki želijo taka pravila implementirati. Posvetimo se raje pravilom in dogovorom za zapis izrazov, ki jih pogosto srečamo v matematiki.

\subsection{Predpone, medpone, pripone, nadnapisi in podnapisi}

Aritmetične operacije $+$, $-$, $\cdot$ in $/$ pišemo kot \df{medpone} ali \df{infiksne operacije}, tako da operacija
stoji med obema operandoma, na primer $x + y$. Kadar zapišemo operator za operand,
pravimo, da je \df{pripona} ali \df{postfiksna operacija}, na primer faktoriela~$x!$. Zapis operatorja je
\df{predpona} ali \df{prefiksna operacija}, če stoji pred operandom, na primer nasprotna vrednost~$-x$. Poleg teh
poznamo tudi druge zapise: potenciranje pišemo z eksponentom $x^y$, deljenje z ulomkom
$\frac{x}{y}$, kvadratni koren s posebnim simbolom $\sqrt{x}$ itn. Skrajni primer je zapis
množenja brez simbola, ko namesto $x \cdot y$ zapišemo kar $x y$.

Argumente operacije ali funkcije včasih zapišemo v \df{podnapis} ali \df{nadnapis}. Na primer, če je $a : \NN \to \RR$ preslikava, pogosto pišemo $a_i$ namesto $a(i)$.

\subsection{Prednost in združevanje}

Nekatere operacije imajo \df{prednost} ali \df{prioriteto} pred drugimi in nekatere \df{združujejo} ali \df{asociirajo} levo ali desno. Prednost pove, katera operacija pride prej na vrsto, kadar ni oklepajev:
potenciranje ima prednost pred množenjem in množenje pred seštevanjem. Operacija lahko
tudi združuje levo ali desno. Na primer, seštevanje $+$ združuje levo, zato je $5 + 2 + 1$
enako $(5 + 2) + 1$. Pri seštevanju to sicer ni pomembno, pri odštevanju pa moramo
upoštevati združevanje na levo: $5 - 2 - 1$ je enako $(5 - 2) - 1$ in ne $5 - (2 - 1)$.
Potenciranje združuje na desno, saj $2^{3^4}$ pomeni $2^{(3^4)}$. Nekatere operacije ne
združujejo in v takih primerih moramo uporabiti oklepaje.

Povejmo še to: nič ni narobe, če zapišemo več oklepajev, kot je to nujno potrebno. Izraza $3 \cdot 4 + 5$ in $(((3) \cdot 5) + 5)$ sta enakovredna.

\subsection{Implicitni argumenti, privzete vrednosti in preobteževanje}
\label{sec:implicitni-argumenti}

Argumente operacije lahko opustimo in od bralca pričakujemo, da bo pravilno uganil, kaj smo mislili. Pravimo, da so to
\df{implicitni argumenti}. Primer implicitnih argumentov smo že videli, ko smo zapisali prvo in drugo projekcijo $\fst$ in
$\snd$:
\begin{align*}
  \fst &: A \times B \to A, \\
  \snd &: A \times B \to A.
\end{align*}
%
Če bi bili zelo natančni, bi morali pri projekcijah zapisati tudi množici $A$ in $B$, ki tvorita kartezični produkt, na
primer nekaj takega kot $\fst^{A,B} : A \times B \to A$.
%
Ko torej vpeljemo novo zapis, lahko nekatere argumente razglasimo za \df{implicitne}, kar pomeni, da jih bomo opuščali,
kadar to ne pripelje do zmede.

\begin{vaja}
  Ali ima kompozicija preslikav $\circ$ implicitne argumente? Katere?
\end{vaja}

Argument operacije ima lahko \df{privzeto vrednost}. Na primer logaritem $x$ z osnovo $b$ zapišemo $\log_b x$. Če opustimo~$b$, se razume, da je mišljen desetiški logaritem, $\log x = \log_10 x$. Pravimo, da je privzeta vrednost osnove $b = 10$.

Simbol lahko tudi \df{preobtežimo}, da ima več pomenov, nato pa od bralca pričakujemo, da bo uganil, katerega smo
mislili. Na primer, $+$ uporabljamo za
%
seštevanje naravnih števil,
seštevanje celih števil,
seštevanje racionalnih števil,
seštevanje realnih števil,
seštevanje kompleksnih števil,
seštevanje vektorjev,
seštevanje matrik,
itd.
%
S preobteževanjem ne gre pretiravati, ker lahko pripelje do zmede. Običajno z istim simbolom označimo različne operacije, ki imajo kaj skupnega. Na primer, $+$ vedno uporabljamo le za operacijo, ki je komutativna, asociativna in ima nevtralni element.

\subsection{Izrazi predstavljajo drevesa}

Izrazi so zaporedja znakov, ki jih pišemo z leve na desno. A kje drugje na tem svetu bi jih pisali z desne na levo ali
navpično. Izrazi so le \emph{predstavitve} tako imenovanih \df{sintaktičnih dreves}. Na primer $((3 + x) \times y)^2$ predstavlja sintaktično drevo, pri čemer potenciranje predstavimo z znakom ${}^{\wedge}$:
%
\begin{center}
  \begin{tikzpicture}[level/.style={sibling distance=5em/#1},,level distance=2em,
    every node/.style = {align=center}, baseline=(current bounding box.center)
    ]
    \node {${}^{\wedge}$}
    child { node {$\times$}
      child { node {$+$}
        child { node {$3$} }
        child { node {$x$} }
      }
      child { node {$y$} }
    }
    child { node {$2$} } ;
  \end{tikzpicture}
\end{center}
%
O sintaktičnih drevesih ne bomo govorili, a jih omenimo, ker so pomembna iz dveh razlogov: sintaktična drevesa so
\emph{podatkovni tip}, s katerim v programu dejansko obdelujemo izraze; s pomočjo sintaktičnih dreves lahko simbolni zapis predstavimo kot posebno vrsto algebre, ki omogoča matematično obravnavo izrazov.


\section{Slike in diagrami}
\label{sec:slike-in-diagrami}

Matematiki uporabljamo tudi diagrame in slike, slednje predvsem v geometriji in analizi. Z
njimi lahko razjasnimo pojme in si pomagamo pri predstavi zapletenih pojmov in
konstrukcij, zato so nepogrešljivo orodje. To še posebej velja za poučevanje matematike.

Vendar pa moramo biti pri uporabi slik pazljivi, ker nas lahko zavedejo. V poduk podajmo ">dokaz"<, da so vsi trikotniki enakokraki!

\begin{izrek}
  Vsi trikotniki so enakokraki.
\end{izrek}

\begin{proof}
  Naj bo $\triangle ABC$ poljuben trikotnik, glej sliko~\ref{fig:trikotnik}.
  %
  \begin{figure}[ht]
    \centering
\tikzset{
    right angle quadrant/.code={
        \pgfmathsetmacro\quadranta{{1,1,-1,-1}[#1-1]}     % Arrays for selecting quadrant
        \pgfmathsetmacro\quadrantb{{1,-1,-1,1}[#1-1]}},
    right angle quadrant=1, % Make sure it is set, even if not called explicitly
    right angle length/.code={\def\rightanglelength{#1}},   % Length of symbol
    right angle length=4ex, % Make sure it is set...
    right angle symbol/.style n args={3}{
        insert path={
            let \p0 = ($(#1)!(#3)!(#2)$) in     % Intersection
                let \p1 = ($(\p0)!\quadranta*\rightanglelength!(#3)$), % Point on base line
                \p2 = ($(\p0)!\quadrantb*\rightanglelength!(#2)$) in % Point on perpendicular line
                let \p3 = ($(\p1)+(\p2)-(\p0)$) in  % Corner point of symbol
            (\p1) -- (\p3) -- (\p2)
        }
    }
}

\begin{center}
\begin{tikzpicture}[scale=0.5]

% Triangle vertices
\coordinate (A) at (3,10);
\coordinate (B) at (0,0);
\coordinate (C) at (8,2);
\coordinate (P) at (4,1);
\coordinate (Q) at (7/2,3);
\coordinate (R) at (243/218, 405/109);
\coordinate (S) at (1119/178, 422/89);

% Draw the perpendiculars from midpoints to opposite sides
\draw[thick] (S) -- (Q);
\draw[thick] (R) -- (Q);
\draw[thick] (P) -- (Q);
\draw[thick] (B) -- (Q);
\draw[thick] (C) -- (Q);
\draw[thick] (A) -- (Q);

% Draw triangle
\draw[very thick] (A) -- (B) -- (C) -- cycle;

% Perpendicular marks
\draw [right angle symbol={B}{A}{Q}];
\draw [right angle symbol={C}{A}{Q}];
\draw [right angle symbol={B}{C}{Q}];

\tkzMarkAngle[size=2,mark=|](B,A,Q)
\tkzMarkAngle[size=2,mark=|](Q,A,S)

% Points
\fill (A) circle (5pt) node[above] {A};
\fill (B) circle (5pt) node[below left] {B};
\fill (C) circle (5pt) node[below right] {C};
\fill (P) circle (5pt) node[below] {P};
\fill (S) circle (5pt) node[above right] {S};
\fill (R) circle (5pt) node[above left] {R};
\fill (Q) circle (5pt) node[above right] {Q};
\end{tikzpicture}
\end{center}
    \caption{Trikotnik $\triangle ABC$}
    \label{fig:trikotnik}
  \end{figure}
  %
  Naj bo $P$ središče stranice $BC$ ter $Q$ presečišče simetrale kota $\angle BAC$ in simetrale stranice $BC$.
  Naj bo~$R$ pravokotna projekcija točke~$Q$ na stranico~$AB$ in~$S$ pravokotna projekcija točke~$Q$ na stranico~$AC$.
  %
  Trikotnik~$\triangle BCQ$ je enakokrak z vrhom~$Q$, zato velja $BQ \cong CQ$.
  %
  Trikotnika $\triangle AQR$ in~$\triangle AQS$ sta podobna, saj skladna, ker imata skupno stranico in kota ob njej, torej velja $AR \cong AS$ in $QR \cong QS$.
  %
  Sklepamo, da sta tudi trikotnika $\triangle BQR$ in $\triangle CQS$ skladna, saj sta pravokotna trikotnika s skladno kateto in skladno hipotenuzo. Potemtakem sta skladni še preostali kateti, $RB \cong SC$, od koder izračunamo
  %
  \begin{equation*}
    AB \cong AR + RB \cong AS + SC \cong AC.
  \end{equation*}
  %
  Trikotnik $\triangle ABC$ je res enakokrak.
\end{proof}





%%% Local Variables:
%%% mode: latex
%%% TeX-master: "lmn"
%%% End:

\chapter{Množice in preslikave}

Pri predmetu Logika in množice se bomo učili, kako matematiki komuniciramo in razmišljamo. Spoznali bomo osnove logike
in teorije množic, tako iz povsem praktičnega vidika kot tudi matematičnega. Pri tem predmetu cenimo ne le matematično
razmišljanje, ampak tudi razmišljanje o matematiki.

Za uvod povejmo nekaj osnovnega o množicah in spoznajmo nekatere osnovne konstrukcije.

\section{Osnovno o množicah}

\subsection{Množice kot skupki elementov, relacija $\in$}

Naivno bi rekli, da je množica kakršnakoli zbirka ali skupek matematičnih objektov. Le-ti so lahko števila, funkcije,
množice, množice števil ipd., skratka karkoli.
%
Najbolj preprosti primeri množic so končne množice, katerih elemente naštejemo. Zapišemo jih na primer takole:
%
\begin{gather*}
  \set{1, 2, 3} \\
  \set{\sin, \cos, \tan} \\
  \set{\{1\}, \{2\}, \{3\}}.
\end{gather*}
%
Objektom, ki tvorijo množico, pravimo \textbf{elementi}. Na primer, elementi množice $\{1, \{4\}, 7/3\}$ so število $1$, množica $\{4\}$, in število $7/4$.

Kadar je $a$ element množice $M$, to zapišemo $a \in M$ in beremo ">$a$ je element $M$"<.

Ali sta množici $\{1, 4, 10\}$ in $\{4, 10, 1, 10\}$ enaki? Da, saj množice obravnavamo kot \emph{neurejene} skupke, v katerih ni pomembno, kolikokrat se pojavi kak element. Da vrstni red in število pojavitev nista pomembna, sledi iz \textbf{aksioma
ekstenzionalnosti}. Aksiom je matematična izjava, ki jo vzamemo za osnovno, se pravi, da je ne dokazujemo. Aksiomi opredeljujejo matematično teorijo, ki jo želimo študirati. Tako bom pri tem predmetu spoznali aksiome teorije množic, pri algebri aksiome za vektorski prostor in grupo itd.

\begin{aksiom}[Ekstenzionalnost množic]
  Množici sta enaki, če imata iste elemente.
\end{aksiom}

Povedano drugače: če je vsak element množice $A$ tudi element množice $B$ in je vsak element množice $B$ tudi element množice $A$, potem velja $A = B$.

Z uporabo ekstenzionalnosti, lahko \emph{dokažemo}, da sta $\{1, 4, 10\}$ in $\{4, 10, 1, 10\}$ enaki:
%
\begin{enumerate}
\item 
  Vsak element $\{1, 4, 10\}$ je tudi element $\{4, 10, 1, 10\}$:
  \begin{enumerate}
    \item velja $1 \in \{4, 10, 1, 10\}$
    \item velja $4 \in \{4, 10, 1, 10\}$
    \item velja $10 \in \{4, 10, 1, 10\}$
  \end{enumerate}
\item
Vsak element $\{4, 10, 1, 10\}$ je tudi element $\{1, 4, 10\}$:
  \begin{enumerate}
     \item velja $4 \in \{1, 4, 10\}$
     \item velja $10 \in \{1, 4, 10\}$
     \item velja $1 \in \{1, 4, 10\}$
     \item velja $10 \in \{1, 4, 10\}$
  \end{enumerate}
\end{enumerate}

Iz zgornjih dveh preverjanj z uporabo ekstenzionalnosti sledi, da $\{1, 4, 10\} = \{4, 10, 1, 10\}$.

\begin{naloga}
  Zapišite podroben dokaz, da sta množici $\{x, y\}$ in $\{y, x\}$ enaki.
\end{naloga}

\begin{opomba}
  Poznamo tudi skupke, pri katerih je pomembno, kolikokrat se pojavi vsak element. Imenujejo se \textbf{multimnožice}.
\end{opomba}

Opozorimo takoj, da v praksi pogosto uporabljamo zapise, ki niso povsem natančni. Takrat se zanašamo, da bodo ostali pravilni uganili, kaj imamo v mislih. Na primer, katere elemente vsebuje množica
%
\begin{equation*}
    \{1, 2, 3, ..., 2021\} \ ?
\end{equation*}
%
Verjetno bi vsi ">uganili"<, da so mišljena vsa naravna števila med $1$ in $2021$, ali ne? Zavedati se je treba, da zgornji zapis tega ne določa! Morda smo imeli v mislih vsa števila med $1$ in $2021$, ki pri deljenju s~$5$ ne dajo ostanka~$4$.

Pri tem predmetu bomo pogosto opozarjali na razne nejasnosti in nenatančne zapise, ki jih uporabljajo matematiki v praksi.
Ni mišljeno, da bi se pretvarjali, da je kaj narobe s ">človeško matematiko"<. Želimo se predvsem zavedati, kje se nejasnosti v praksi pojavljajo in kako bi jih lahko odpravili (tudi če jih v praksi dejansko ne odpravimo). Ko bo torej asistent pri analizi na tablo napisal
%
\begin{equation*}
    1, 2, 4, 8, \ldots
\end{equation*}
%a
imate tri možnosti:
%
\begin{enumerate}
  \item Ste zmedeni.
  \item Uganete, da ima v mislih potence števila 2.
  \item Vprašate, ali je $n$-ti člen število regij, na katerega lahko razdelimo prostor z $(n-1)$ ravninami?
\end{enumerate}
%
Sami se odločite, kakšen odnos želite vzpostaviti z asistentom.

\subsection{Prazna množica $\emptyset$}

Verjetno ni treba izgubljati besed o prazni množici. To je množica, ki nima nobenega elementa. Zapišemo jo $\emptyset$ ali $\{\}$.

\begin{naloga}
  Ali je kakšna razlika med $\{\}$ in $\{\emptyset\}$?
\end{naloga}


\subsection{Standardni enojec $\one$}

Množici, ki ima natanko en element, pravimo \textbf{enojec}.

Ali znamo pojasniti, kaj pomeni, da ima množica natanko en element, ne da bi pri tem omenili število $1$ ali katerokoli drugo število? Takole: množica $A$ ima natanko en element če velja:
%
\begin{enumerate}
\item obstaja $x \in A$ in
\item če je $x \in A$ in $y \in A$, potem $x = y$.
\end{enumerate}

\begin{naloga}
  Kako bi opredelili ">množica ima natanko dva elementa"< brez uporabe števil?
\end{naloga}

Pogosto bomo potrebovali kak enojec (že na naslednjih predavanjih). Seveda se ni težko domisliti enojca, na primer $\{42\}$ ali $\{\sin\}$. Da pa ne bomo vedno znova izgubljali časa z izbiro enojca, se dogovorimo da je \textbf{standardni enojec $\one$} množica $\{\unit\}$. To je zelo čudno, ker smo označili množico s številko\footnote{Ali ločite med ">števka"<, ">številka"< in ">število"<?} $1$ in ker je element standardnega enojca $\unit$, česar še nikoli nismo videli.

Glede oznake $\one$ povejmo, da imamo kot matematiki \emph{načelno svobodo} pri izbiri zapisa, a je smiselno in vljudno, da se ne zafrkavamo. Ali se torej predavatelj zafrkava, ko standardni enojec označi s številko $\one$? Ne, saj gresta ">ena"< in ">enojec"< lepo skupaj, poleg tega pa bomo na naslednjih predavanjih spoznali tudi matematične razloge za tak zapis.

Glede oznake $\unit$ se bo kmalu izkazalo, da je zapis smiseln, ker je $()$ pravzaprav ">urejena ničterica"<.

\subsection{Številske in ostale množice}

Seveda si bomo privoščili uporabo raznih množic, ki jih že poznate, kot so na primer številske množice $\NN$, $\ZZ$, $\QQ$, $\RR$ itd. Opozorimo pa na naslednji dilemo:
v osnovni in srednji šoli z $\NN$ označimo množico celih števil, ki so večja ali enaka $1$, vendar pa pogosto v matematiki, še posebej pa v logiki, tudi število $0$ obravnavamo kot naravno število. V takih primerih $\NN$ izenačuje množico celih števil, ki so večja ali enaka $0$.

Kaj je torej prav $\NN = \{0, 1, 2, \ldots\}$ ali $\NN = \{1, 2, 3, \ldots\}$? To je napačno vprašanje! Lahko vprašamo le ">kako se bomo dogovorili?"<. Pri tem predmetu se
dogovorimo, da je $0$ naravno število, ker vadimo ">matematično svobodo"<, imamo dobre matematične razloge, da $0$ uvrstimo med naravna števila, in ker je predavatelj tako zapovedal.

\begin{naloga}
  Zberite pogum in predavatelja vprašate, kakšni so ti dobri matematični razlogi, zaradi katerih je zapovedal, da je $0$ naravno število. Sledila bo filozofska razprava, ki vam bo pokvarila odmor.
\end{naloga}

\section{Konstrukcije množic}

Ena od osnovnih matematičnih aktivnosti so \textbf{konstrukcije}. Poznamo na primer geometrijske konstrukcije z ravnilom in šestilom. Ko računamo rešite enačbe, bi lahko rekli, da konstruiramo število, ki zadošča enačbi. Ko pišemo dokaz, konstruiramo objekt, iz katerega je razvidna resničnost neke izjave. Tudi računalniški programi so le matematični konstrukti.

Spoznajmo nekatere osnovne konstrukcije množic, se pravi, načine, kako iz množic naredimo nove množice.

\subsection{Zmnožek ali kartezični produkt}

\textbf{Urejeni par} $\pair{x,y}$ je matematični objekt, ki da dobimo tako, da združimo dva matematična objekta~$x$ in~$y$. V srednji šoli ste večinoma pisali urejene pare števil (ki ste jih imenovali ">koordinate"<). V urejenem paru je vrstni red \emph{pomemben}: urejena para $(1, 3)$ in $(3, 1)$ \emph{nista} enaka. (Množici $\{1, 3\}$ in $\{3, 1\}$ sta enaki.)

Urejeni par $\pair{x, y}$ ima \textbf{prvo komponento~$x$} in \textbf{drugo komponento~$y$}. Če imamo neki urejeni par~$u$, njegovi komponenti pišemo tudi $\fst(u)$ in $\snd(u)$. Velja torej:
%
\begin{equation*}
    \fst(x,y) = x
    \iinn
    \snd(x,y) = y.  
\end{equation*}
%
Simboloma $\mathsf{pr}_1$ in $\mathsf{pr}_2$ pravimo \textbf{prva} in \textbf{druga projekcija}. Običajne oznake za projekciji so tudi $\pi_1$ in $\pi_2$, v
programiranju $\mathtt{fst}$ in $\mathtt{snd}$, lahko pa tudi $\pi_0$ in $\pi_1$.

Spoznajmo sedaj \textbf{zmnožek} ali \textbf{kartezični produkt} množic $A$ in $B$. Opis nove konstrukcijo množic mora navesti zapis za konstruirano množico, katere elemente ima, in kdaj sta elementa konstruirane množice enaka:
%
\begin{enumerate}
\item Zmnožek množic $A$ in $B$ zapišemo $A \times B$.
\item Elementi množice $A \times B$ so urejeni pari $\pair{x, y}$, pri čemer je $x \in A$ in $y \in B$.
\item Enakost elementov (princip ekstenzionalnosti za pare): $u \in A \times B$ in $v \in A \times B$ sta enaka, če velja $\fst(u) = \fst(v)$ in $\snd(u) = \snd(v)$.
\end{enumerate}

\begin{primer}
Zmnožek množic $\{1,2,3\}$ in $\{\Box, \diamond\}$ je
%
\begin{equation*}
    \{1, 2, 3\} \times {\Box, \diamond} =
    \{\pair{1, \Box},
      \pair{2, \Box},
      \pair{3, \Box},
      \pair{1, \diamond},
      \pair{2, \diamond},
      \pair{3, \diamond}
     \}.
\end{equation*}
%
Iz principa ekstenzionalnosti za pare sledi, da je vrstni red v urejenem paru pomemben, saj $\pair{1, 3} \neq \pair{3, 1}$, ker $\fst(1,3) = 1 \neq 3 = \fst(3,1)$.
\end{primer}


\subsubsection{Zmnožek več množic}

Tvorimo lahko tudi zmnožek več množic. Na primer, zmnožek množic $A$, $B$ in $C$ je množica $A \times B \times C$, katerih elementi so \textbf{urejene trojke} $\pair{x, y, z}$, kjer je $x \in A$, $y \in B$ in $z \in C$. V tem primeru imamo tri projekcije $\mathsf{pr}_1$, $\mathsf{pr}_2$ in $\mathsf{pr}_3$. Podobno lahko tvorimo zmnožek štirih, petih, šestih, \dots množic.

\begin{naloga}
  Ali lahko tvorimo zmnožek ene množice? Kaj pa zmnožek nič množic?
\end{naloga}


\subsection{Vsota ali koprodukt}

Naslednja osnovna konstrukcija je \textbf{vsota} ali \textbf{koprodukt} množic $A$ in $B$:
%
\begin{enumerate}
\item vsoto množic $A$ in $B$ označimo z $A + B$,
\item elementi množice $A + B$ so $\inl{x}$ za $x \in A$ in $\inr{y}$ za $y in B$,
\item elementa $u \in A + B$ in $v \in A + B$ sta enaka, kadar velja
  %
  \begin{enumerate}
    \item bodisi za neki $a \in A$ velja $u = \inl{a} = v$,
    \item bodisi za neki $b \in B$ velja $u = \inr{b} = v$.
  \end{enumerate}
\end{enumerate}

\begin{primer}
Primeri vsote množic:
%
\begin{enumerate}

\item $\{1, 2, 3\} + \{\square, \diamond\} = \{\inl{1}, \inl{2}, \inl{3}, \inr{\Box}, \inr{\diamond}\}$

\item $\{a, b\} + \{b, c\} = \{\inl{a}, \inl{b}, \inr{b}, \inr{c}\}$

\item Vsota \emph{ni} unija! Po eni strani je
      $\{3, 5\} \cup \{3, 5\} = \{3, 5\}$ in po drugi
      $\{3, 5\} + \{3, 5\} = \{\inl{3}, \inl{5}, \inr{3}, \inr{5}\}$.
\end{enumerate}
\end{primer}

Vsoti pravimo tudi ">disjunktna unija"<, a se bomo temu izrazu izogibali, ker obravnavamo vsoto kot osnovno operacijo in ne kot poseben primer unije.

Oznakama $\mathsf{in}_1$ in $\mathsf{in}_2$ pravimo \textbf{prva in druga injekcija}. Uporabljajo se tudi oznake $\iota_1$ in $\iota_2$, v funkcijskem
programiranju $\mathtt{inl}$ in $\mathtt{inr}$, pa tudi $\iota_0$ in $\iota_1$. Pravzaprav ni pomembno, kakšne oznake uporabimo, poskrbeti moralo
le, da sta to različna simbola, s katerima razločimo elemente prvega in drugega sumanda.

Tvorimo lahko vsoto več množic, na primer $A + B + C$. V tem primeru imamo tri injekcije $\mathsf{in}_1$, $\mathsf{in}_2$ in $\mathsf{in}_3$.

\section{Preslikave ali funkcije}

Poleg množic so preslikave še en osnovni matematični pojem, ki mu bomo posvetili veliko pozornosti. \textbf{Preslikava} ali \textbf{funkcija} sestoji iz treh sestavin:
%
\begin{itemize}
\item množice, ki ji pravimo \textbf{domena},
\item množice, ki ji pravimo \textbf{kodomena},
\item \textbf{prirejanja}, ki vsakemu elementu domene priredi natanko en element kodomene.
\end{itemize}
%
Če je $f$ funkcija z domeno $A$ in kodomeno $B$, to zapišemo
%
\begin{equation*}
  f : A \to B
\end{equation*}
%
ali
%
\begin{equation*}
  \xymatrix{
    {A} \ar[rr]^{f} & & {B}
  }
\end{equation*}
%
Rišemo lahko tudi diagrame, ki prikazujejo več funkcij hkrati, na primer
%
\begin{equation*}
  \xymatrix{
    {A} \ar[r]^{f} &
    {B} \ar[r]^{g} &
    {C} \ar[d]^{h} \\
    & & {D}
  }
\end{equation*}
%
Ta diagram prikazuje tri preslikave: $f : A \to B$, $g : B \to C$ in $h : C \to D$.

V srednji šoli ste spoznavali posamične zvrsti funkcij, na primer linearne funkcije, trigonometrične funkcije, eksponentno funkcijo itd. Le-te so običajno slikale števila v števila, bile so \emph{številske funkcije}. Mi se bomo ukvarjali s preslikavami na splošno, se pravi s poljubnimi preslikavami med poljubnimi množicami.

\subsubsection{Princip ekstenzionalnosti preslikav}

\textbf{Princip ekstenzionalnosti za preslikave}, pove, kdaj sta dve funkciji enaki, namreč takrat, ko prirejata enake vrednosti:  če za preslikavi $f : A \to B$ in $g : C \to D$ velja $A = C$, $B = D$ in $f(x) = g(x)$ za vse $x \in A$, potem velja $f = g$.

Kasneje bomo videli, da princip ekstenzionalnosti za preslikave sledi iz principa ekstenzionalnosti za množice.

\subsection{Prirejanje in funkcijski predpisi}

Dejstvo, da mora prirejanje vsakemu elementu domene prirediti ">natanko en"< element kodomene, lahko izrazimo tako, da se izognemo uporabi števila ">ena"< ali kateregakoli števila. Poglejmo kako.

Prirejanje z domeno $A$ in kodomeno $B$ mora biti:
%
\begin{enumerate}
\item \textbf{celovito:} vsakemu $x \in A$ je prirejen vsaj en $y \in B$ (priredimo vsaj en element),
\item \textbf{enolično:} če sta $x \in A$ prirejena $y \in B$ in $z \in B$, potem velja $y = z$ (priredimo največ en element).
\end{enumerate}
%
Res, celovitost zagotavlja, da vsakemu elementu domene priredimo \emph{vsaj en} element kodomene, enoličnost pa zagotavlja, da priredimo \emph{kvečjemu enega}.

\begin{opomba}
  Pozor, celovitost \emph{ni} surjektivnost in enoličnost \emph{ni} injektivnost!
\end{opomba}

Kako pravzaprav podamo prirejanje? Kaj to pravzaprav je? Čez kak mesec bomo znali odgovoriti na to vprašanje natančno, zaenkrat pa le povejmo, da je prirejanje kakršnakoli metoda, tabela, postopek, prikaz, ali konstrukcija, ki zagotavlja celovitost in enoličnost prirejanja elementov kodomene elementom domene.

Običajni način za podajanje prirejanja je \textbf{funkcijski predpis}, ki ga pišemo
%
\begin{equation*}
  x \mapsto \cdots
\end{equation*}
%
Pri čemer za $\cdots$ na desni postavimo neki smiseln izraz, ki določa enolično vrednost za vsak $x$ iz domene. Spremenljivki~$x$ na levi pravimo \textbf{parameter}, izrazu $\cdots$ na desni pa \textbf{prirejena vrednost}.

\begin{primer}
  Primeri prirejanj:
  %
  \begin{itemize}
  \item prirejanje ">prištej 7 in kvadriraj"< zapišemo s funkcijskim predpisom  $x \mapsto (x + 7)^2$,
  \item prirejanje ">kvadriraj in prištej 7"< zapišemo s funkcijskim predpisom  $x \mapsto x^2 + 7$,
  \item prirejanje ">prištej kvadrat 7"< zapišemo s funkcijskim predpisom  $x \mapsto x + 7^2$.
  \end{itemize}
\end{primer}

\begin{opomba}
  Pozor: če podamo \emph{samo} funkcijski predpis brez domene in kodomene, še nismo podali preslikave! Preslikava sestoji iz \emph{treh} delov: domena, kodomena in prirejanje.
  Torej zgornji trije primeri \emph{ne} podajajo preslikav, ker nismo podali domen in
  kodomen.
\end{opomba}

Domeno, kodomeno in funkcijski predpis lahko zapišemo na različne načine:
%
\begin{align*}
  f &: \ZZ \to \NN \\
  f &: x \mapsto x^2 + 7
\end{align*}
%
ali
%
\begin{align*}
    f &: \ZZ \to \NN \\
    f(x) &\defeq x^2 + 7
\end{align*}
%
ali
%
\begin{align*}
    f &: \ZZ \to \NN \\
    f &= (x \mapsto x^2 + 7)
\end{align*}
%
Simbol $x$ je \textbf{vezana spremenljivka}, če jo preimenujemo, se predpis ne spremeni. Naslednji funkcijski predpisi so \emph{enaki}:
%
\begin{align*}
  x &\mapsto x^2 + 7 \\
  y &\mapsto y^2 + 7 \\
  \textit{banana} &\mapsto \textit{banana}^2 + 7
\end{align*}
%
Funkcijski predpis $x \mapsto 5 + x \cdot x + 2$ pa \emph{ni enak} zgornjim trem, čeprav vrača enake vrednosti in torej določa \emph{enako} funkcijo.

\subsubsection{Aplikacija ali uporaba}

Preslikavo $f : A \to B$ \textbf{uporabimo} ali \textbf{apliciramo} na elementu $a \in A$, da dobimo \textbf{vrednost} $f(a) \in B$. V izrazu $f(a)$ se imenuje $a$ \textbf{argument}
%
Kadar je $f$ podana s predpisom, izračunamo vrednost $f(a)$ tako, da $a$ vstavimo v predpis (vezano spremenljivo zamenjamo z argumentom~$A$).
%
O zamenjavi vezane spremenljivke z argumentom bomo več povedali v razdelku~\ref{sec:substitucija} o substituciji.

\begin{primer}
  Če je $f : \NN \to \NN$ podana s predpisom $f = (x \mapsto x^3 + 4)$, tedaj je $f(5)$ enako $5^3 + 4$. Lahko bi celo pisali
  %
  \begin{equation*}
    (x \mapsto x^3 + 4)(5) = 5^3 + 4.
  \end{equation*}
\end{primer}


\subsection{Eksponentna množica}

Tretja konstrukcija množic, ki jo bomo spoznali v uvodnem poglavju, je \textbf{eksponent} ali \textbf{eksponentna množica}:
%
\begin{enumerate}
\item eksponent množic $A$ in $B$ označimo $B^A$, in preberemo ">$B$ na $A$"<,
\item elementi $B^A$ so preslikave z domeno $A$ in kodomeno $B$,
\item preslikavi $f : A \to B$ in $g : A \to B$ sta enaki, če imate enake vrednosti: za
  vse $x \in A$ velja $f(x) = g(x)$, potem je $f = g$.
\end{enumerate}
%
Eksponent $B^A$ pišemo tudi $A \to B$. To pomeni, da bi lahko namesto $f : A \to B$ pisali tudi $f \in B^A$ ali celo $f \in A \to B$, vendar je ta zadnji zapis neobičajen.

\begin{primer}
Eksponent $\{1, 2\}^{\{a, b\}}$ ima štiri elemente:
%
\begin{equation*}
  \{1, 2\}^{\{a, b\}} =
  \{
     (a \mapsto 1, b \mapsto 1),
     (a \mapsto 1, b \mapsto 2),
     (a \mapsto 2, b \mapsto 1),
     (a \mapsto 2, b \mapsto 2)
  \}.
\end{equation*}
\end{primer}



%\chapter{Aritmetika množic}

Nadaljujmo s študijem splošnih preslikav.

\section{Preslikave in prazna množica}

Naj bo $A$ množica. Kaj vemo povedati o preslikavah $\emptyset \to A$?

Čez nekaj tednov bomo spoznali naslednji dejstvi, ki ju zaenkrat vzemimo v zakup:

\begin{itemize}
\item Vsaka izjava oblike ">za vsak element $\emptyset$ ..."< je resnična.
\item Vsaka izjava oblike ">obstaja element $\emptyset$ ..."< je neresnična.
\end{itemize}

Primeri resničnih izjav:
%
\begin{enumerate}
\item ">Vsak element prazne množice je sodo število"<
\item ">Vsak element prazne množice je liho število"<
\item ">Vsak element prazne množice je hkrati sodo in liho število"<
\item ">Vsak element prazne množice \dots"<
\end{enumerate}

Primeri neresničnih izjav:
%
\begin{enumerate}
\item ">Obstaja element prazne množice, ki je sodo število"<
\item ">Obstaja element prazne množice, ki je enak sam sebi"<
\item ">Obstaja element prazne množice, ki \dots"<
\end{enumerate}
%
Denimo, da imamo preslikave $f :\emptyset \to A$ in $g : \emptyset \to A$. Tedaj sta enaki, saj velja: ">za vsak element $x \in \emptyset$ velja $f(x) = g(x)$"<.
Torej imamo kvečjemu eno preslikavo $\emptyset \to A$. Ali pa imamo sploh kakšno? Da, pravimo ji \textbf{prazna preslikava}, ker je njeno prirejanje ">prazno"<, oziroma ga sploh ni treba podati (saj ni nobenega elementa domene $\emptyset$, ki bi mu morali prirediti kak element kodomene $A$).

Kaj pa preslikave $A \to \emptyset$?
%
Če je $A = \emptyset$, potem imamo natanko eno preslikavo $A \to \emptyset$, namreč prazno preslikavo, $\emptyset^A = \{ \textrm{prazna-preslikava} \}$.
%
Če $A$ vsebuje kak element, potem ni nobene preslikave $A \to \emptyset$, se pravi  $\emptyset^A = \emptyset$.

Zakaj ni preslikave $A \to \emptyset$, kadar $A$ vsebuje kak element? Denimo da je $x \in A$. Če bi bila kaka preslikava $f : A \to \emptyset$, bi
veljalo $f(x) \in \emptyset$, kar pa ni res. Torej take preslikave ni.

\begin{vaja}
  Koliko je preslikav $1 \to A$ in koliko je preslikav $A \to 1$?
  Ali je odgovor odvisen od~$A$?
\end{vaja}

\section{Identiteta in kompozicija}

Spoznajmo nekaj osnovnih preslikav in operacij na preslikavah.

\textbf{Identiteta} na $A$ je preslikava $\id[A] : A \to A$, podana s predpisom $x \mapsto x$.

\textbf{Kompozitum} preslikav
%
\begin{equation*}
  \xymatrix{
    {A} \ar[r]^f & {B} \ar[r]^g & {C}
  }
\end{equation*}
%
je preslikava $g \circ f : A \to C$, podana s predpisom $x \mapsto g(f(x))$.

\textbf{Kompozitum je asociativen:} za preslikave
%
\begin{equation*}
  \xymatrix{
    {A} \ar[r]^f & {B} \ar[r]^g & {C} \ar[r]^h & {D}
  }
\end{equation*}
%
velja $(h \circ g) \circ f = h \circ (g \circ f)$. Res, za vsak $x \in A$ velja
%
\begin{align*}
  ((h \circ g) \circ f)(x)
  &= (h \circ g) (f x) \\
  &=  h (g (f (x)) \\
  &= h ((g \circ f)(x)) \\
  &= (h \circ (g \circ f))(x),
\end{align*}
%
torej želena\footnote{Piše se ">želen"< in ne ">željen"<, ker je ">želen"< deležnik na
  ">n"< glagola ">želeti"<. V slovenščini ni glagola ">željeti"<. Hitro boste spoznali, da na FMF profesorji za matematiko radi popravljajo slovnico.} enačba sledi iz principa ekstenzionalnosti za funkcije.

\textbf{Identiteta je nevtralni element za kompozitum:} za vsako preslikavo $f : A \to B$ velja
%
\begin{equation*}
  \id[B] \circ f = f
  \iinn
  f \circ \id[A] = f.
\end{equation*}
%
To preverimo z uporabo ekstenzionalnosti za funkcije: za vsak $x \in A$ velja
%
\begin{equation*}
    (\id[B] \circ f)(x) = \id[B] (f(x)) = f(x)
\end{equation*}
%
in
\begin{equation*}
  (f \circ \id[A])(x) = f (\id[A](x)) = f(x).
\end{equation*}
%
Kompozicija $\circ$ in identiteta $\id$ se torej obnašata podobno kot nekatere operacije v algebri, na primer $+$ in $0$ ter $×$ in $1$.

\begin{vaja}
  Seštevanje je komutativno, velja $a + b = b + a$. Ali je kompozicija preslikav tudi komutativna?
\end{vaja}

\section{Funkcijski predpisi na zmnožku in vsoti}

Pogosto želimo definirati preslikavo, katere kodomena je zmnožek množic, denimo $f : A \times B \to C$. V takem primeru lahko
podamo funkcijski predpis takole:
%
\begin{equation*}
  (x, y) \mapsto \cdots
\end{equation*}
%
pri čemer je $x \in A$ in $y \in B$. To je dovoljeno, ker je vsak element domene $A \times B$ urejeni par $(x, y)$ za natanko določena $x \in A$ in $y \in B$.

\begin{zgled}

Preslikavo
%
\begin{align*}
  \RR \times \RR  &\to  \RR \\
  u &\mapsto  \fst(u)^2 + 3 \cdot \snd(u)
\end{align*}
%
lahko bolj čitljivo podamo s predpisom
%
\begin{align*}
  \RR \times \RR  &\to  \RR \\
  (x, y) &\mapsto  x^2 + 3 \cdot y
\end{align*}
\end{zgled}

\begin{zgled}
  Seveda lahko podobno podajamo tudi preslikave na zmnožkih več množic, denimo
  %
  \begin{align*}
  A \times B \times C &\to A \times A \\
  (a, b, c) &\mapsto (a, a)
  \end{align*}
  in
  \begin{align*}
  X \times (Y \times Z) &\to (X \times Y) \times Z \\
  (x, (y, z)) &\mapsto ((x, y), z).
  \end{align*}
\end{zgled}

Kako pa zapišemo funkcijski predpis funkcije z domeno $A + B$? V tem primeru je vsak element domene bodisi oblike
$\inl(x)$ za enolično določeni $x \in A$, bodisi oblike $\inr(y)$ za enolično določeni $y \in B$, zato funkcijski predpis podamo v dveh vrsticah:
%
\begin{align*}
    A + B &\to C \\
    \inl(x) &\mapsto \cdots \\
    \inr(y) &\mapsto \cdots
\end{align*}

\begin{zgled}
  Primer take preslikave je
  %
  \begin{align*}
    \RR + \ZZ &\to \RR \\
    \inl(x) &\mapsto x \\
    \inr(y) &\mapsto y + 3
  \end{align*}
  %
  Seveda lahko podobno podajamo tudi preslikave na vsotah več preslikav:
  %
  \begin{align*}
    A + B + C &\to \{u, v\} \\
    \inl(x) &\mapsto u \\
    \inr(y) &\mapsto u \\
    \inj[3](z) &\mapsto v
  \end{align*}
  in
  \begin{align*}
    A + (B + C) &\to \{u, v\} \\
    \inl(x) &\mapsto u \\
    \inr(\inl(y)) &\mapsto u \\
    \inr(\inr(y)) &\mapsto v
  \end{align*}
\end{zgled}

Zapisa za zmnožek in vsoto lahko tudi kombiniramo:
%
\begin{align*}
  (A \times B \times C) + (D \times E) &\to \{0, 1, 2\} \\
  \inl((a, b, c)) &\mapsto 1 \\
  \inr((d, e)) &\mapsto 2
\end{align*}
%
in
\begin{align*}
  (A + B) \times C &\to \{0, 1, 2\} \\
  (\inl(a), c) &\mapsto 0 \\
  (\inr(b), c) &\mapsto 1
\end{align*}
%
Izraz na levi strani $\mapsto$ sestoji iz vezanih spremenljivk in operacij, s katerimi gradimo elemente množic (urejeni par, kanonična injekcija). Imenuje se tudi \textbf{vzorec}. Predpis je podan pravilno, če so vzorci napisani tako, da vsak element domene ustreza natanko enemu vzorcu.
%
S tem zagotovimo, da predpis obravnava vse možne primere (celovitost) in da ne obravnava nobenega primera večkrat (enoličnost).


\section{Funkcijski predpis, podan po kosih}

Omenimo še en pogost način podajanja funkcij, namreč s predpisom po kosih.

\begin{zgled}
  Preslikava ">absolutno"< je definirana po kosih za negativna in nenegativna števila:
  %
  \begin{align*}
    \RR &\to \RR \\
    x &\mapsto
    \begin{cases}
      -x & \text{če $x < 0$,}\\
       x & \text{če $x \geq 0$.}
    \end{cases}
  \end{align*}
\end{zgled}

\begin{zgled}
  Preslikava ">predznak"< je definirana po kosih:
  %
  \begin{align*}
    \RR &\to \RR \\
    x &\mapsto
      \begin{cases}
        -1 & \text{če $x < 0$,}\\
        0 & \text{če $x = 0$,}\\
        1 & \text{če $x > 0$.}
      \end{cases}
  \end{align*}
\end{zgled}


Pri takem zapisu moramo paziti, da kosi skupaj pokrivajo domeno (vsi elementi domene so obravnavani) in da se kosi ne prekrivajo (vsak element domene je obravnavan natanko enkrat). Pravzaprav se smejo kosi prekrivati, a moramo v tem primeru preveriti, da se na skupnih delih skladajo, se pravi, da vsi kosi podajajo enake vrednosti na preseku.

\begin{zgled}
  Preslikavo ">absolutno"< bi lahko podali takole:
  %
  \begin{align*}
    \RR &\to \RR \\
    x &\mapsto
    \begin{cases}
      -x & \text{če $x < 0$,}\\
       x & \text{če $x \geq 0$.}
    \end{cases}
  \end{align*}
  %
  Kosa se prekrivata pri $x = 0$, vendar to ni težava, ker je $-0 = 0$.
\end{zgled}


\section{Nekatere preslikave na eksponentnih množicah}

Poglejmo si nekaj preslikav, ki slikajo iz in v eksponente množice.

\textbf{Evalvacija} ali \textbf{aplikacija} ali \textbf{uporaba} je preslikava, ki sprejme preslikavo in argument, ter preslikavo uporabi na argumentu:
%
\begin{align*}
  \mathsf{ev} &: B^A \times A \to B \\
  \mathsf{ev} &: (f, x) \mapsto f(x)
\end{align*}
%
Pravimo, da je $\mathsf{ev}$ \textbf{preslikava višjega reda}, ker slika preslikave v vrednosti.

\begin{zgled}
  Določeni integral $\int_0^1$ je funkcija višjega reda, ker
  sprejme funkcijo $[0,1] \to \RR$ in vrne realno število. Je torej preslikava
  $\RR^{[0,1]} \to \RR$, če se pretvarjamo, da lahko integriramo vsako funkcijo.
  Bolj pravilno bi bilo reči, da je $\int_0^1$ preslikava iz množice \emph{integrabilnih funkcij $[0,1] \to \RR$} v realna števila.
\end{zgled}

\noindent
Kompozitum preslikav je tudi preslikava višjega reda:
%
\begin{align*}
    {\circ} &: C^B \times B^A \to C^A \\
    {\circ} &: (g, f) \mapsto (x \mapsto g(f(x)))
\end{align*}
%
Tretja splošna preslikava višjega reda je ">currying"< (ali zna kdo to prevesi v slovenščino?):
%
\begin{align*}
  A^{(B \times C)} &\to (A^B)^C \\
  f &\mapsto (c \mapsto (b \mapsto f(b, c))).
\end{align*}
%
Pravzaprav je to izomorfizem, katerega inverz je ">uncurrying"<:
%
\begin{align*}
  (A^B)^C &\to A^{B \times C} \\
  g       &\mapsto ((b, c) \mapsto f(b)(c))
\end{align*}


\section{Izomorfizmi in aritmetika množic}

\subsection{Inverz}

\begin{definicija}
  Preslikava $f : A \to B$ je \textbf{inverz} preslikave $g : B \to A$, če velja $f \circ g = \id[B]$ in $g \circ f = \id[A]$.
\end{definicija}

\begin{vaja}
  Utemelji: če je $f$ inverz $g$, potem je $g$ inverz $f$.
\end{vaja}

\begin{zgled}
  Kub in kubični koren sta inverza
  %
  \begin{align*}
    \RR &\to \RR    &     \RR &\to \RR \\
    x &\mapsto x^3  &     y &\mapsto \sqrt[3]{y}
  \end{align*}
\end{zgled}

\begin{vaja}
  Naj bo~$S$ množica nenegativnih realnih števil, se pravi, $\RR_{\geq 0} = \{x \in \RR \mid x \geq 0\}$. Ali sta kvadriranje in kvadratni koren inverza?
  %
  \begin{align*}
    \RR &\to \RR_{\geq 0}    &     \RR_{\geq 0} &\to \RR \\
    x &\mapsto x^2           &     y &\mapsto \sqrt[2]{y}
  \end{align*}
  %
\end{vaja}

\begin{izjava}
  Če sta $f : A \to B$ in $g : A \to B$ oba inverza preslikave $h : B \to A$, potem je $f = g$.
\end{izjava}

\begin{dokaz}
  Denimo, da sta $f : A \to B$ in $g : A \to B$ inverza preslikave $h : B \to A$. Tedaj velja
  %
  \begin{equation*}
    f =
    f \circ \id[A] =
    f \circ (h \circ g) =
    (f \circ h) \circ g =
    \id[B] \circ g =
    g.
  \end{equation*}
\end{dokaz}

Ali znate utemeljiti vsakega od zgornjih korakov?

\begin{definicija}
  Preslikava, ki ima inverz, se imenuje \textbf{izomorfizem}.
\end{definicija}

Če je $f : A \to B$ izomorfizem, potem ima natanko en inverz $B \to A$, ki ga označimo $\inv{f}$.

\begin{zgled}
  Identiteta $\id[A] : A \to A$ je izomorfizem, saj je sama sebi inverz.
  Torej $\inv{\id[A]} = \id[A]$.
\end{zgled}

\begin{zgled}
  Eksponentna preslikava $\exp : \RR \to \RR_{> 0}$, $\exp : x \mapsto e^x$ je
  izomorfizem, njen inverz je naravni logaritem $\ln : \RR_{> 0} \to \RR$, torej $\inv{\exp} = \ln$.
\end{zgled}

\begin{zgled}
  Eksponentna preslikava $\exp : \RR \to \RR$ \emph{ni} izomorfizem.
\end{zgled}

\begin{izjava}
  Če sta $f : A \to B$ in $g : B \to C$ izomorfizma, potem je tudi $g \circ f : A \to C$ izomorfizem. Velja torej $\inv{(g \circ f)} = \inv{f} \circ \inv{g}$.
\end{izjava}

\begin{dokaz}
  Dokazati moramo, da ima $g \circ f$ inverz. Trdimo, da je $\inv{f} \circ \inv{g} : C \to A$ inverz preslikave $g \circ f$. Računajmo:
  %
  \begin{align*}
    (g \circ f) \circ (\inv{f} \circ \inv{g}) \\
     &= ((g \circ f) \circ \inv{f}) \circ \inv{g} \\
     &= (g \circ (f \circ \inv{f})) \circ \inv{g} \\
     &= (g \circ \id[B]) \circ \inv{g} \\
     &= g \circ \inv{g} \\
     &= \id[C].
  \end{align*}
  %
  Doma sami preverite, da velja tudi $(\inv{f} \circ \inv{g}) \circ (g \circ f) = \id[A]$.
\end{dokaz}

\subsection{Izomorfne množice}


\begin{definicija}
  Množici $A$ in $B$ sta \textbf{izomorfni}, če obstaja izomorfizem $f : A \to B$. Kadar sta $A$ in $B$ izomorfni, to zapišemo $A \iso B$.
\end{definicija}

\begin{izjava}
  \textbf{Izjava:} Za vse množice $A$, $B$ in $C$ velja:
  %
  \begin{enumerate}
    \item $A \iso A$,
    \item če $A \iso B$, potem $B \iso A$,
    \item če $A \iso B$ in $B \iso C$, potem $A \iso C$.
  \end{enumerate}
\end{izjava}

\begin{dokaz}
  %
  \begin{enumerate}
     \item $\id[A]$ je izomorfizem $A \to A$,
     \item če je $f : A \to B$ izomorfizem, potem je tudi $\inv{f} : B \to A$ izomorfizem,
     \item če je $f : A \to B$ izomorfizem in $g : B \to C$ izomorfizem, potem je $g \circ f : A \to C$ izomorfizem.
  \end{enumerate}
\end{dokaz}

\begin{zgled}
  $A \times B \iso B \times A$, ker imamo izomorfizem in njegov inverz
  %
  \begin{align*}
    A \times B  &\to  B \times A      &    B \times A  &\to  A \times B \\
    (x, y) &\mapsto  (y, x)           &    (b, a) &\mapsto  (a, b)
  \end{align*}
\end{zgled}

\subsection{Aritmetika množic}

Veljajo naslednji izomorfizmi, ki nas seveda spomnijo na zakone aritmetike, ki
veljajo za števila. Ali gre tu za kako globljo povezavo?
%
\begin{enumerate}
\item Vsota in $\emptyset$:
  \begin{enumerate}
    \item $A + \emptyset \iso A$
    \item $A + B \iso B + A$
    \item $(A + B) + C \iso A + (B + C)$
  \end{enumerate}

\item Zmnožek in $\one$:
  \begin{enumerate}
    \item $A \times \one \iso A$
    \item $A \times B \iso B \times A$
    \item $(A \times B) \times C \iso A \times (B \times C)$
  \end{enumerate}

\item Distributivnost:
  \begin{enumerate}
    \item $A \times (B + C) \iso (A \times B) + (A \times C)$
    \item $A \times \emptyset \iso \emptyset$
  \end{enumerate}

\item Eksponenti:
  \begin{enumerate}
    \item $A^\one \iso A$
    \item $\one^A \iso \one$
    \item $A^{\emptyset }\iso \one$
    \item $\emptyset^A \iso \emptyset , če  A \neq \emptyset$
    \item $A^{(B \times C)} \iso (A^B)^C$
    \item $A^{(B + C)} \iso A^B \times A^C$
    \item $(A \times B)^C \iso A^C \times B^C$
  \end{enumerate}
\end{enumerate}

\begin{vaja}
  Zapišite vseh 15 izomorfizmov, ki potrjujejo pravilnost zgornjega seznama.
\end{vaja}

\chapter{Simbolni zapis}

V matematiki uporabljamo \textbf{simbolni zapis} -- matematične objekte, konstrukcije in dokaze opišemo s pomočjo izrazov kot so
%
\begin{gather*}
  3 + 4\\
  x \mapsto x^2 + 3\\
  \all{x \in \RR} x^2 + x + 1 \geq 1/4
\end{gather*}
%
Matematično besedilo je mešanica naravnega jezika (slovenščine) in simbolnega zapisa. Načeloma bi lahko pisali matematiko \emph{samo} s simbolnim zapisom (kar dejansko počnemo, kadar matematiko \emph{formaliziramo} z računalnikom, a o tem kdaj drugič), a bi bilo to ljudem preveč nerazumljivo. V starih časih so uporabljali samo naravni jezik (latinščino), kar je bilo tudi zelo nerazumljivo.

Spoznajmo pravila simbolnega zapisa in se učimo razumeti, brati in pisati logične formule (matematične izjave, izražene s simbolnim zapisom).


\section{Izrazi}

\textbf{(Simbolni) izraz} je zaporedje znakov, ki predstavlja neki matematični pojem, na primer
%
\begin{gather*}
  3 + 5 \\
  S \cap (T \cup V) \\
  2 x y \leq x^2 + y^2
\end{gather*}
%
Izraz je \emph{pravilno formiran} ali \emph{sintaktično pravilen}, če ustreza pravilom, ki določajo kako postavljamo oklepaje,
vejice, pike, kako uporabljamo razne posebne simbole ($+$, $\lor$, $\int$) itd. Na primer, izraz $3 + ) x \cdot 4$ ni sintaktično
pravilen, ker ima narobe postavljen zaklepaj.

Natančna sintaktična pravila za pisanje matematičnih izrazov so precej zapletena, ker je matematični zapis raznovrsten
in se je razvijal skozi zgodovino. Na srečo skoraj vsa pravila že poznate (">vsak oklepaj mora imeti ustrezni zaklepaj"<,
">piše se $a + b$ in ne $a b +$"< ipd). Tu se ne bomo ukvarjali s podajanjem vseh pravil -- to je delo za računalničarje,
ki želijo taka pravila naprogramirati. Kljub temu pa velja omeniti nekatere pojme.


\subsection{Prefiksne, postfiksne in infiksne operacije}

V simbolnem zapisu uporabljamo \emph{operacije}, ki jih pišemo pred, za ali med argumente:
%
\begin{itemize}
\item \textbf{prefiksne operacije} so take, ki jih pišemo \emph{pred} argument:
  \begin{itemize}
  \item  $-x$ za nasprotno vrednost $x$,
  \item  $\neg P$ za negacijo izjave $P$,
  \end{itemize}
\item \textbf{infiksne operacije} so take, ki jih pišemo \emph{med} argumenta:
  \begin{itemize}
  \item aritmetične operacije $x + y$, $x - y$, $x \cdot y$ itn.
  \item logični vezniki $P \land Q$, $P \lor Q$, $P \lthen Q$ itn.
  \end{itemize}
\item \textbf{postfiksne operacije} so take, ki jih pišemo \emph{za} argument:
  \begin{itemize}
  \item $n!$ za faktorielo števila $n$.
  \end{itemize}
\end{itemize}
%
Včasih uporabljamo tudi druge oblike zapisa:
%
\begin{itemize}
\item \emph{potenciranje} $A^B$
\item \emph{ulomki} $\frac{a}{b}$
\item integrali $\int f(x) dx$ in vsote $\sum_{i = 0}^n a_i$,
\item zapis podmnožice $\set{x \in \RR \such x^2 + x > 2}$.
\end{itemize}
%
Operacija je lahko celo ">nevidna"<, oziroma jo pišemo kot presledek med argumentoma:
%
\begin{itemize}
\item $x \, y$ kot zmnožek $x$ in $y$,
\item $\sin x$ kot uporabe funkcije $\sin$ na argumentu~$x$.
\end{itemize}


\subsection{Oklepaji, prioriteta in asociiranost}

Z oklepaji ponazorimo, katera operacija ima prednost. Na primer, če ne bi imeli dogovora, da ima množenje prednost pred
seštevanjem, potem bi lahko izraz $3 + 4 \times 5$ razumeli kot $3 + (4 \times 5)$ ali kot $(3 + 4) \times 5$. Oklepajev ne smemo opustiti, kadar bi lahko prišlo do take zmede. Nikoli pa ne škodi, če zapišemo kak oklepaj več, kot je to potrebno (v mejah normale).

Da se izognemo pisanju oklepajev, se dogovorimo, da imajo nekatere operacije prednost pred ostalimi, kar so vas učili že
v osnovni šoli. Pravimo, da imajo operacije \textbf{prioriteto}. Operacija z višjo prioriteto ima prednost pred operacijo z
nižjo prioriteto.

\begin{primer}
  Množenje $\times$ ima višjo prioriteto kot seštevanje $+$ (to je \emph{dogovor} in ne matematično dejstvo).
  Podobno ima konjunkcija $\land$ višjo prioriteto kot disjunkcija $\lor$.
\end{primer}

Poleg prioritete imajo nekatere operacije tudi \textbf{asociiranost}. Kako naj razumemo izraz $8 - 3 - 2$, kot $(8 - 3) - 2$
ali kot $8 - (3 - 2)$? V šoli so vas učili, da je
%
\begin{equation*}
  A - B - C = (A - B) - C
\end{equation*}
%
Pravimo, da $-$ veže na levo oziroma da ima \textbf{levo asociiranost}. Ker beremo z leve na desno, ima večina operacij levo
asociiranost. Velja na primer
%
\begin{align*}
  A + B + C &= (A + B) + C \\
  A \times B \times C &= (A \times B) \times C.
\end{align*}
%
Morda bo kdo pripomnil, da itak velja $(A + B) + C = A + (B + C)$ in da zato ni pomembno, kako razumemo $A + B + C$. To je
res v preprostih primerih, ko vemo, da smo s $+$ označili seštevanje števil. Kaj pa, če s $+$ označimo kako drugo
preslikavo? Ali $(A + B) + C = A + (B + C)$ velja tudi v programskih jezikih, pri katerih lahko pride do prekoračitve
največjega možnega števila?

Primer operacije z desno asociiranostjo je implikacija: $P \lthen Q \lthen R$ je enako $P \lthen (Q \lthen R)$.


\subsection{Izrazi predstavljajo drevesa}

Izrazi so zaporedja znakov, ki jih pišemo z leve na desno. A kje drugje bi jih morda pisali z desne na levo ali
navpično. Izrazi so le \emph{predstavitve} tako imenovanih \textbf{sintaktičnih dreves}. Na primer $((3 + x) \times y)^2$ predstavlja sintaktično drevo, pri čemer potenciranje predstavimo z znakom ${}^{\wedge}$:
%
\begin{center}
  \begin{tikzpicture}[level/.style={sibling distance=5em/#1},,level distance=2em,
    every node/.style = {align=center}, baseline=(current bounding box.center)
    ]
    \node {${}^{\wedge}$}
    child { node {$\times$}
      child { node {$+$}
        child { node {$3$} }
        child { node {$x$} }
      }
      child { node {$y$} }
    }
    child { node {$2$} } ;
  \end{tikzpicture}
\end{center}
%
O sintaktičnih drevesih ne bomo govorili, a jih omenimo, ker so pomembna iz dveh razlogov: sintaktična drevesa so
\emph{podatkovni tip}, s katerim v programu dejansko obdelujemo izraze; s pomočjo sintaktičnih dreves lahko simbolni zapis
predstavimo kot posebno vrsto algebre, ki omogoča matematično obravnavo izrazov.


\subsection{Ostala sintaktična pravila}

Sintaktičnih pravil je še več, od katerih omenimo le nekatera.

\subsubsection{Podnapisi in nadnapisi}
\label{sec:podnapisi-nadnapisi}

Argumente operacije ali funkcije včasih zapišemo v \textbf{podnapis} ali \textbf{nadnapis}. Na primer, če je $a : \NN \to \RR$
preslikava, pogosto pišemo $a_i$ namesto $a(i)$.

\subsubsection{Implicitni argumenti}
\label{sec:implicitni-argumenti}

Argumente operacije lahko opustimo in od bralca pričakujemo, da bo pravilno uganil, kaj smo mislili. Pravimo, da so to
\textbf{implicitni argumenti}. Primer implicitnih argumentov smo že videli, ko smo zapisali prvo in drugo projekcijo $\fst$ in
$\snd$:
\begin{align*}
  \fst &: A \times B \to A, \\
  \snd &: A \times B \to B.
\end{align*}
%
Če bi bili zelo natančni, bi morali pri projekcijah zapisati tudi množici $A$ in $B$, ki tvorita kartezični produkt, na
primer nekaj takega kot $\fst^{A,B} : A \times B \to A$.
%
Ko torej vpeljemo nov zapis, lahko nekatere argumente razglasimo za \textbf{implicitne}, kar pomeni, da jih bomo opuščali,
kadar to ne pripelje do zmede.

\begin{naloga}
  Ali ima kompozicija preslikav $\circ$ implicitne argumente? Katere?
\end{naloga}

\subsubsection{Privzete vrednosti}
\label{sec:privzete-vrednosti}

Argument operacije ima lahko \textbf{privzeto vrednost}. Na primer logaritem $x$ z osnovo $b$ zapišemo $\log_b x$. Če opustimo~$b$, se razume, da je mišljen desetiški logaritem, $\log x = \log_{10} x$. Pravimo, da je privzeta vrednost osnove $b = 10$.

\subsubsection{Preobteževanje}
\label{sec:preobteevanje}

Simbol lahko tudi \textbf{preobtežimo}, da ima več pomenov, nato pa od bralca pričakujemo, da bo uganil, katerega smo
mislili. Na primer, $+$ uporabljamo za
%
seštevanje naravnih števil,
seštevanje celih števil,
seštevanje racionalnih števil,
seštevanje realnih števil,
seštevanje kompleksnih števil,
seštevanje vektorjev,
seštevanje matrik,
itd.
%
S preobteževanjem ne gre pretiravati, ker lahko pripelje do zmede. Običajno z istim simbolom označimo različne operacije, ki imajo kaj skupnega. Na primer, $+$ vedno uporabljamo le za operacijo, ki je komutativna, asociativna in ima nevtralni element.


\section{Logične formule}

Izrazi, ki označujejo števila, se imenujejo \textbf{aritmetični izrazi}.

Izrazi, ki označujejo matematične izjave, se imenujejo \textbf{logični izrazi} ali \textbf{logične formule}. Razumevanje, branje in pisanje le-teh zahteva kar nekaj treninga, zato se mu bomo posvetili tu in na vajah. Pravzaprav ne bomo vadili le razumevanja zapisa, ampak tudi, kako matematiki razmišljajo in razumejo drug drugega.

Tu o dokazih in pravilih dokazovanja še ne bomo govorili, bomo pa pojasnili intuitivni pomen logičnih operacij.

Računanje z logičnimi formulami delimo na:
%
\begin{itemize}
\item \textbf{izjavni račun} zaobjema logične veznike $\neg $, $\land$, $\lor$, $\lthen$, $\liff$,
\item \textbf{predikatni račun} zaobjema izjavni račun ter kvantifikatorja $\forall$ in $\exists$.
\end{itemize}


\subsection{Izjavni račun}

\textbf{Izjavni vezniki} so naslednje operacije:
%
\begin{itemize}

\item \textbf{resničnostni konstanti} $\bot$ in $\top$: beremo ju ">neresnica"> in ">resnica"<,

\item \textbf{negacija} $\neg$: izjavo $\neg A$ beremo ">$A$ ne velja"< ali ">ni res, da $A$"<,

\item \textbf{konjunkcija} $\land$: izjavo $A \land B$ beremo ">$A$ in $B$"<,

\item \textbf{disjunkcija} $\lor$: izjavo $A \lor B$ beremo ">$A$ ali $B$"<,

\item \textbf{implikacija} $\lthen$: izjavo $A \lthen B$ lahko beremo na več načinov:
  %
  \begin{itemize}
  \item ">Iz $A$ sledi $B$."<
  \item ">Če $A$, potem $B$."<
  \item ">$A$ samo če $B$."<
  \item ">$B$ sledi iz $A$."<
  \item ">$A$ je zadosten pogoj za $B$."<
  \item ">$B$ je potreben pogoj za $A$."<
  \end{itemize}
  %
\item \textbf{ekvivalenca} $\liff$: izjavo $A \liff B$ beremo
  %
  \begin{itemize}
  \item ">$A$ je ekvivalentno $B$."<
  \item ">$A$, če in samo če $B$."<
  \item ">$A$ natanko tedaj, ko $B$."<
  \item ">$A$ je zadosten in potreben pogoj za $B$."<
  \end{itemize}
\end{itemize}
%
Malo bolj neobičajna je:
%
\begin{itemize}
\item \textbf{ekskluzivna disjunkcija} $\oplus$: izjavo $A \oplus B$ beremo ">bodisi $A$ bodisi $B$"< ali ">$A$ ali $B$, vendar ne oba hkrati"<.
\end{itemize}
%
Prioriteta veznikov, od najvišje do najnižje:
%
\begin{itemize}
\item $\neg$,
\item $\land$,
\item $\lor$, $\oplus$,
\item $\lthen$, $\liff$.
\end{itemize}

\begin{primer}
  Izraz $\neg A \land B \lthen C \lor D$ beremo kot $((\neg  A) \land B) \lthen (C \lor D)$.
\end{primer}

Asociiranost veznikov:
%
\begin{itemize}
\item leva asociiranost: $\land$, $\lor$, $\oplus$,
\item desna asociiranost: $\lthen$.
\end{itemize}
%
Ekvivalenca $\liff$ nima asociiranosti, zato je zapis $A \liff B \liff C$ načeloma dvoumen, a v praksi pomeni $(A \liff B) \land (B \liff C)$.

\begin{opomba}
  Tudi zapis $x = y = z$ pravzaprav ni smiseln, saj sta $(x = y) = z$ in $x = (y = z)$ oba nesmiselna. V praksi $x = y = z$ pomeni $(x = y) \land (y = z)$. Pa še to: koliko enačb je izraženih z $a = b = c = d$? Tri! Toliko kot je enačajev.
\end{opomba}

\begin{opomba}
  Zapis $x \neq y \neq z$ je nejasen in se mu je bolje izogibati, saj zlahka pripelje do pomote, ker iz $x
  \neq y$ in $y \neq z$ ne sledi nujno $x \neq z$.
\end{opomba}

Glede razumevanja veznikov, omenimo:
%
\begin{itemize}
\item disjunkcija je \emph{inkluzivna}, kar pomeni, da je $A \lor B$ resnična izjava, če sta $A$ in $B$ resnični,
\item v implikaciji $A \lthen B$ se $A$ imenuje \textbf{antecedent} in $B$ \textbf{konsekvent}. Implikacija je veljavna, če je antecedent neveljaven,
\item ekvivalenco $A \liff B$ lahko razumemo kot okrajšavo za $(A \lthen B) \land (B \lthen A)$.
\end{itemize}

\subsection{Kvantifikatorja}

Matematične izjave vsebujejo fraze, kot so ">za vse"<, ">za neki"<, ">obstaja vsaj en"<, ">za natanko enega"< ipd. Le-te izrazimo s \textbf{kvantifikatorji}. Osnovna kvantifikatorja sta \textbf{univerzalni} in \textbf{eksistenčni}.


\subsubsection{Univerzalni kvantifikator $\forall$}

Formulo $\all {x \in A} \phi$ beremo:
%
\begin{itemize}
\item ">Za vsak $x$ iz $A$ velja $\phi$."<,
\item ">Vsi $x$ iz $A$ zadoščajo $\phi$."<,
\item ">$\phi$ za vse $x$ iz $A$."<
\end{itemize}
%
Pika pri tem nima nobenega posebnega pomena, pogosti so tudi zapisi
%
\begin{equation*}
  \forall x \in A \,,\, \phi
  \qquad\text{ali}\qquad
  \forall x : A \,,\, \phi
  \qquad\text{ali}\qquad
  (\forall x : A) \phi.
\end{equation*}
%
Nekateri matematiki pišejo po principu ">piši kao što govoriš"<
%
\begin{equation*}
  \phi, \forall x \in A
  \qquad\qquad \text{(">$\phi$ za vse $x$ iz $A$"<)}
\end{equation*}
%
Ta zapis odsvetujemo, ker ne deluje, ko kombiniramo več kvantifikatorjev hkrati.

Omenili smo že, da $\all{x \in \emptyset} \phi$ vedno velja. To bomo utemeljili v poglavju o pravilih sklepanja.

\subsubsection{Eksistenčni kvantifikator $\exists$}

Formulo $\some{x \in A} \phi$ beremo:
%
\begin{itemize}
\item ">Obstaja $x$ iz $A$ velja $\phi$."<
\item ">Obstaja vsaj en $x$ iz $A$ velja $\phi$."<
\item ">Za neki $x$ iz $A$ velja $\phi$."<
\item ">$\phi$ za neki $x$ iz $A$."<
\end{itemize}
%
S tem povemo, da obstaja \emph{eden ali več} takih~$x$. Na primer, izjava $\some{x \in \NN} x
< 3$ je veljavna, saj je $2$ naravno število, ki je manjše od $3$.


\subsubsection{Prioriteta $\forall$ in $\exists$}

Prioriteta kvantifikatorjev $\forall$ in $\exists$ je nižja od prioritete veznikov. Na primer:
%
\begin{itemize}
\item $\all{x \in A} \phi \land \psi$ je enako $\all{x \in A} (\phi \land \psi)$,
\item $\all{x \in \RR} x > 0 \lthen \phi$ je enako $\all{x \in \RR} (x > 0 \lthen \phi)$.
\end{itemize}
% 
Kvantifikator vedno zaobjame vse, kar zmore:
%
\begin{itemize}
\item $\all{x \in A} \phi \land \some{y \in B} \psi$ je enako $\all {x \in A} (\phi \land (\some{y \in B} \psi))$ in \emph{ni} enako $(\all{x \in A} \phi) \land (\some {y \in B} \psi)$,

\item $(P \land \all{x \in A} Q \lthen R) \lthen \some{y \in B} S$ je enako $(P \land \all{x \in A}  (Q \lthen R)) \lthen (\some{y \in B} S)$ in \emph{ni} enako 
  $(P \land (\all{x \in A} Q) \lthen R) \lthen (\some{y \in B} S)$
\end{itemize}

\subsubsection{Kombinacija $\forall$ in $\exists$}

Pozor, vrstnega reda kvantifikatorjev ne smemo mešati:
%
\begin{itemize}
\item $\all{x \in \RR} \some{y \in \RR} x < y$ pomeni ">vsako realno število je manjše od nekega realnega števila"< (kar je res),
\item $\some{x \in \RR} \all{y \in \RR} x < y$ pomeni ">obstaja najmanjše realno število"< (kar ni res).
\end{itemize}
%
To dejstvo bomo utrjevali na vajah. Zapomnite se, da morate biti tudi pri ostalih predmetih posebej pozorni na vrstni red
">za vsak"< in ">obstaja"<. Je profesorica pri analizi rekla ">za vsak $\epsilon > 0$ obstaja tak $\delta > 0$ da \dots"< ali je rekla ">obstaja tak $\delta > 0$ da za vsak $\epsilon > 0$ \dots"<? Če boste zamešali ti dve izjavi na ustnem izpitu iz analize, boste imeli pokvarjen dan, ali pa cele počitnice!


\subsubsection{Kvantifikator z dodatnim pogojem}

Pogosto kvantifikacijo kombiniramo z dodatnim pogojem, na primer:
%
\begin{itemize}
\item ">Obstaja \emph{liho} naravno število, ki ni deljivo s 7."<
\item ">Vsako \emph{sodo} naravno število je deljivo s 3."<
\end{itemize}
%
V prvem primeru je dodatni pogoj izražen z besedico ">liho"< in v drugem s ">sodo"<. Kako zapišemo take izjave s formulo, kam
vtaknemo dodatni pogoj? Izjavi pretvorimo po korakih:
%
\begin{itemize}
\item ">Obstaja liho naravno število, ki ni deljivo s 7."<
\item ">Obstaja naravno število, ki je liho, ki ni deljivo s 7."<
\item ">Obstaja naravno število, ki je liho in ki ni deljivo s 7."<
\item ">Obstaja $x$ iz $\NN$, da je $x$ lih in $x$ ni deljiv s 7."<
\item $\some{x \in \NN} (\text{$x$ je lih}) \land (\text{$x$ ni deljiv s 7})$
\item $\some{x \in \NN} (\some{y \in \NN} x = 2 y + 1) \land (\neg \some{z \in \NN} x = 7 z)$
\end{itemize}
%
In še druga izjava:
\begin{itemize}
\item ">Vsako sodo naravno število je deljivo s 3."<
\item ">Vsako naravno število, ki je sodo, je deljivo s 3."<
\item ">Za vsako naravno število velja, da če je sodo, potem je deljivo s 3."<
\item ">Za vsak $x$ iz $\NN$ velja, če je $x$ sod, potem je $x$ deljiv s $3$."<
\item $\all{x \in \NN} (\text{$x$ je sod}) \lthen (\text{$x$ je deljiv s 3})$
\item $\all{x \in \NN} (\some{y \in \NN} x = 2 y) \lthen (\some{z \in \NN} x = 3 z)$
\end{itemize}
%
Zapomnimo si: \textbf{dodatni pogoj pri $\exists$ izrazimo $\land$} in \textbf{dodatni pogoj pri $\forall$ izrazimo $\lthen$}.

Poglejmo še en primer, ko imamo več možnosti za zapis s formulo:
%
\begin{quote}
  ">Za vsako pozitivno realno število $x$ obstaja tako naravno število $n$, da je $x < n$."<
\end{quote}
%
Začetni del ">za vsako pozitivno realno število"< lahko zapišemo na več načinov:
%
\begin{itemize}
\item $\all{x \in \RR_{> 0}} \some{n \in \NN} x < n$,
\item $\all{x \in \set{y \in \RR \such y > 0}} \some{n \in \NN} x < n$,
\item $\all{x \in \RR} x > 0 \lthen \some{n \in \NN} x < n$,
\item $\all{x > 0} \some{n \in \NN} x < n$.
\end{itemize}
%
Pri prvem načinu moramo biti v naprej dogovorjeni, da $\RR_{> 0}$ označuje množico pozitivnih realnih števil.
Pri drugem načinu smo vstavili definicijo $\RR_{> 0}$, zato dogovor ni več potreben, a je zapis bolj nečitljiv.
Pri tretjem načinu smo predstavili pozitivnost kot dodatni pogoj.
Četrti način je najbolj čitljiv in se pogosto uporablja, a nam ne pove, ali je $x$ realno, celo, ali racionalno število.


\subsubsection{Vezane in proste spremenljivke}

V nekaterih izrazih nastopajo spremenljivke, ki so \textbf{vezane}. To pomeni, da je njihovo območje veljavnosti omejeno,
oziroma da so neke vrste ">lokalne spremenljivke"<. Spremenljivka, ki ni vezana, je \textbf{prosta}. Primeri:
%
\begin{itemize}
\item V funkcijskem predpisu $x \mapsto x^2 + y$ je $x$ vezan in $y$ prost.
\item V funkcijskem predpisu $(x,y) \mapsto x^2 + y$ sta $x$ in $y$ vezana.
\item V integralu $\int (x + a)^2 d x$ je $x$ vezan in $a$ prost.
\item V vsoti $\sum_{i=0}^n (i^2 + 1)$ je $i$ vezan in $n$ prost.
\item V formuli $\all{x \in \RR} x^3 + 3 x < 7$ je $x$ vezana spremenljivka.
\end{itemize}

Če vezano spremenljivko preimenujemo, se izraz ne spremeni. Funkcijska predpisa $x \mapsto a \cdot x^2 + 1$ in $y \mapsto a \cdot y^2 + 1$ sta \emph{enaka}. Vendar pozor, če vezano spremenljivko preimenujemo, za novo ime \emph{ne} smemo izbrati spremenljivke, ki se že pojavlja. Na primer, v integralu
%
\begin{equation*}
  \int_0^1 (a + x)^2 \, d x
\end{equation*}
%
smemo $x$ preimenovati v $t$, zato sta integrala enaka izraza (in imata tudi enako vrednost):
%
\begin{equation*}
  \int_0^1 (a + x)^2 \, d x  = \int_0^1 (a + t)^2 \, d t
\end{equation*}
%
Ne bi pa smeli $x$ preimenovati v $a$, saj bi dobili
%
\begin{equation*}
  \int (a + a)^2 \, d a
\end{equation*}
%
Pravimo, da se je prosta spremenljivka $a$ \emph{ujela} v integral.



\chapter{Definicije in dokazi}

\section{Enolični obstoj}

\subsection{Kvantifikatorje za enolični obstoj $\exists!$}
\label{sec:enolicni-obstoj}

S kvantifikatorje $\forall$ in $\exists$ lahko izrazimo tudi druge kvantifikatorje.
Na primer, ">obstajata vsaj dva elementa $x$ in $y$ iz $A$, da velja $\phi(x,y)$"< zapišemo
%
\begin{equation*}
    \some{x \in A} \some{y \in A} x \neq y \land \phi(x,y)
\end{equation*}
%
Kako pa izrazimo ">obstaja natanko en $x$ iz $A$, da velja $\phi(x)$"<? Takole:
%
\begin{equation*}
  (\some{x \in A} \phi(x)) \land \all{y z \in A} \phi(y) \land \phi(z) \lthen y = z
\end{equation*}
%
ali ekvivalentno
%
\begin{equation*}
    \some{x \in A} (\phi(x) \land \all{y \in A} \phi(y) \lthen x = y).
\end{equation*}
%
To okrajšamo $\exactlyone{x \in A} \phi(x)$ in beremo ">obstaja natanko en $x$ iz $A$, da velja $\phi(x)$"<.
%
Uporablja se tudi zapis $\exists^1 x \in A \,.\, \phi(x)$.

\subsection{Operator enoličnega opisa}

Če dokažemo, da obstaja natanko en $x \in A$, ki zadošča pogoju $\phi(x)$, potem se lahko nanj smiselno sklicujemo z ">tisti $x$ iz $A$, ki zadošča $\phi(x)$"<. Primeri:
%
\begin{itemize}
\item ">tisto realno število $x$, za katero je $x³ = 2$"<, namreč kubični koren 2,
\item ">tista množica $S$, ki nima nobenega elementa"<, namreč prazna množica.
\end{itemize}
%
Proti-primeri:
%
\begin{itemize}
\item ">tisto racionalno število $x$, za katero je $x² = 2$"<, saj takega števila ni,
\item ">tisto realno število $x$, za katero je $x² = 2$"<, ker sta dve taki števili,
\item ">tista množica $S$, ki ima natanko en element"<, ker je takih množic je zelo veliko.
\end{itemize}
%
To je lahko zelo koristen način za opredelitev matematičnih objektov, zato uvedemo zanj simbolni zapis. Če dokažemo
%
\begin{equation*}
    \exactlyone{x \in A} \phi(x)
\end{equation*}
%
potem lahko pišemo
%
\begin{equation*}
  \descr{x \in A} \phi(x),
  \qquad\qquad\text{">tisti $x \in A$, za katerega velja $\phi(x)$"<}
\end{equation*}
%
Torej velja
%
\begin{equation*}
  \phi(\descr{x \in A} \phi(x)).
\end{equation*}
%
Spremenljivka $x$ je \emph{vezana} v $\descr{x \in A} \phi(x)$.

\begin{zgled}
  Denimo, da še ne bi poznali simbola $\sqrt{}$ za kvadratne korene. Tedaj bi
  lahko kvadratni koren iz $2$ zapisali kot
  %
  \begin{equation*}
    \descr{x \in R} (x > 0 \land x^2 = 2)
  \end{equation*}
  %
  Še več, preslikavo $\sqrt{} : \RR_{\geq 0} \to \RR_{\geq 0}$ lahko definiramo takole:
  %
  \begin{equation*}
    \sqrt{} : x \mapsto (\descr{y \in \RR} (y \geq 0 \land y^2 = x)).
  \end{equation*}
\end{zgled}

\begin{vaja}
  Zapišite ">limita zaporedja $a : \NN \to \RR$"< z operatorjem $\iota$, pod predpostavko,
  da je $a$ konvergentno zaporedje. Najprej povejte z besedami ">limita zaporedja $a$ je
  tisti $x \in \RR$, ki \dots"<, nato pa zapišite še v obliki $\descr{x \in \RR} \dots$.
\end{vaja}


\begin{opomba}
  Ne pozabite: zapis $\descr{x \in A} \phi(x)$ je veljaven samo v primeru, da velja
  $\exactlyone{x \in A} \phi(x)$.
\end{opomba}


\section{Spremenljivke in definicije}

Preden v matematičnem besedilu uporabimo simbol ali spremenljivko, ga moramo \emph{vpeljati}. To pomeni, da moramo pojasniti, kakšen je pomen simbola. Poznamo dva osnovna načina za vpeljavo novih simbolov:
%
\begin{itemize}
\item \item Nov simbol $s$ lahko \textbf{definiramo} kot okrajšavo za neki drugi izraz ali logično formulo.
\item Nov simbol $s$ je (vezana ali prosta) \textbf{spremenljivka}, ki predstavlja neki (neznan, poljuben, nedoločen) element dane množice $A$.
\end{itemize}
%
V obeh primerih dodamo simbol~$s$ v \textbf{kontekst}, se pravi v spisek znanih simbolov. Če smo simbol uvedli le začasno (na primer v enem poglavju, ali v delu dokaza), ga iz konteksta odstranimo, ko ni več veljaven.

Matematiki zapisujejo definicije in vpeljujejo spremenljivke na razne načine.

\subsection{Vpeljava spremenljivke}

Če želimo vpeljati spremenljivko $x$, ki predstavlja neki poljuben ali neznani element množice $A$, zapišemo
%
\begin{quote}
  Naj bo $x \in A$.
\end{quote}
%
S tem postane $x$ veljavna spremenljivka, ki jo lahko uporabljamo. O njen vemo le to, da je element množice $A$ -- pravimo, da je $x$ \textbf{prosta spremenljivka}. V matematičnih besedilih boste zasledili tudi naslednje fraze, ki uvedejo prosto spremenljivko:
%
\begin{itemize}
\item ">Naj bo $x \in A$ poljuben."<
\item ">Obravnavajmo poljuben $x \in A$."<
\item ">Izberimo poljuben $x \in A$."<
\item ">Denimo, da imamo poljuben $x \in A$."<
\end{itemize}
%
Pozor, beseda ">izberimo"< bi komu dala misliti, da si lahko izbere neki konkretni~$x$, a to preprečuje beseda ">poljuben"<, ki jo matematik uporabi, kadar želi povedati, da je~$x$ neznana ali nedoločena (poljubna) vrednost.

\begin{vaja}
  Denimo, da učitelj reče ">Naj bo $n$ (poljubno) naravno število"<, nato pa vas vpraša ">Ali je $n$ sodo število?"<, kako boste odgovorili?
\end{vaja}

\subsection{Definicija simbola}

Definicija je v prvi vrsti \textbf{okrajšava} za neki izraz. Z njo uvedemo nov simbol~$s$ in mu pripišemo neko vrednost. Simbol $s$ je enak vrednosti, ki smo mu jo pripisali. Simbolni zapis za definicijo je
%
\begin{equation*}
  s \defeq \ldots
\end{equation*}
%
Na primer, v besedilu bi lahko napisali ">Naj bo $s := \sqrt{\log_2 7 + \pi/6}$."< S tem smo v kontekst dodali simbol $s$ in predpostavko $s = \sqrt{\log_2 7 + \pi/6}$. V matematičnih besedili boste zasledili tudi naslednje načine za definicijo:
%
\begin{itemize}
\item $s = \sqrt{\log_2 7 + \pi/6}$ (namesto $\defeq$ uporabimo $=$)
\item $s \cong \sqrt{\log_2 7 + \pi/6}$ (namesto $\defeq$ uporabimo $\cong$)
\item $s \triangleq \sqrt{\log_2 7 + \pi/6}$ (namesto $\defeq$ uporabimo $\triangleq$)
\end{itemize}
%
Kadar definiramo simbol tako, da mu priredimo funkcijski predpis, recimo
%
\begin{equation*}
  f \defeq (x \mapsto x^2 + 7)
\end{equation*}
%
to raje zapišemo kot
%
\begin{equation*}
  f(x) \defeq x^2 + 7.
\end{equation*}
%
Kadar definiramo simbol s pomočjo enoličnega obstoja, recimo
%
\begin{equation*}
  r \defeq \descr{x \in \RR} x^3 = 2
\end{equation*}
%
to raje zapišemo z besedami:
%
\begin{equation*}
  \text{Naj bo $r$ tisto realno število, ki zadošča $r^3 = 2$.}
\end{equation*}
%
Poglejmo še, kako definiramo okrajšave za logične formule. Denimo, da želimo s $\phi(x)$ označiti izjavo $\some{y \in \RR} y^2 = x + 1$. Glede na zgornji dogovor, zapišemo
%
\begin{equation*}
  \phi \defeq (x \mapsto (\some{y \in \RR} y^2 = x + 1))
\end{equation*}
%
ali
%
\begin{equation*}
  \phi(x) \defeq (\some{y \in \RR} y^2 = x + 1).
\end{equation*}
%
Vendar takega zapisa v praksi ne boste videli. Dosti bolj pogost je zapis
%
\begin{equation*}
  \phi(x) \defiff \some{y \in \RR} y^2 = x + 1
\end{equation*}
%
ali pa kar $\phi(x) \liff \some{y \in \RR} y^2 = x + 1$.

\subsection{Definicije novih matematičnih pojmov}

Kaj pa definicije novih pojmov, ki jih srečujete pri predavanjih, denimo pri analizi?

\begin{definicija}
  Zaporedje števil $a : \NN \to \RR$ je \textbf{neomejeno}, če za vsak $x \in \RR$ obstaja $i \in \NN$, da je $a_i > x$.
\end{definicija}

\noindent
S stališča simbolnega zapisa, je to le uvedba novega simbola $\mathsf{neomejeno}$:
%
\begin{equation*}
  \mathsf{neomejeno}(a) \defeq (\all{x \in \RR} \some{i \in \NN} a_i > x).
\end{equation*}
%
Seveda bistvo take definicije ni le krajši zapis izjave $\all{x \in \RR} \some{i \in \NN} a_i > x$, ampak uporabna vrednost pojma ">neomejeno zaporedje"<.


\section{Konstrukcije in dokazi}

Matematiki v sklopu svojih aktivnosti \emph{konstruiramo} matematične objekte:
%
\begin{itemize}
\item v geometriji so znane konstrukcije z ravnilom in šestilom,
\item računanje števk števila $\pi$ je konstrukcija približka,
\item reševanje enačbe, je konstrukcija števila z želeno lastnostjo,
\item konstruiramo lahko elemente množice, pogosto kar tako, da jih zapišemo, na primer $(2, \inl(3)) \in \NN \times (\ZZ + \ZZ)$.
\end{itemize}

Poleg tega \emph{dokazujemo} matematične izjave. Na dokaz lahko gledamo kot na konstrukcijo, saj je to le še ena zvrst matematičnega objekta. Ker pa so dokazi skoraj vedno zapisani v naravnem jeziku, jih matematiki pogosto dojemajo ločeno od ostalih matematičnih objektov (števila, preslikave, množice, ploskve, \dots).

Kaj pravzaprav je dokaz? V prvi vrsti je dokaz utemeljitev matematične izjave. Zgrajen je po točno določenih \emph{pravilih sklepanja}, ki jih lahko podamo formalno in jih tudi implementiramo na računalniku.\footnote{Kogar to zanima, si lahko ogleda ">\href{https://youtu.be/Z500sma3h90}{The dawn of formalized mathematics}"< (\href{https://www.icloud.com/keynote/0Gkr1yM7XY-31aQleWf-fiW7A}{prosojnice}) in se nauči uporabljati kak  \href{https://ncatlab.org/nlab/show/proof+assistant}{dokazovalni pomočnik} (v zadnjem času hitro napreduje \href{https://leanprover.github.io}{Lean}).
}

V praksi ljudje ne pišejo vseh podrobnosti v dokazu, ker bi bil tak dokaz nečitljiv in nerazumljiv. Pogosto podajo samo glavno idejo, iz katere lahko izkušeni matematik sam rekonstruira dokaz. Iz dobro napisanega dokaza se lahko naučimo marsikaj novega, poleg golega dejstva, da dokaza izjava velja.

Mi bomo vadili podrobno pisanje dokazov. Pri ostalih predmetih boste videli ">žive dokaze"<, ki imajo manj podrobnosti in so zapisani manj formalno. A vsi pravilni matematični dokazi se dajo zapisati na način, kot ga bomo predstavili mi (in celo zapisati povsem formalno z dokazovalnim pomočnikom).

\subsection{Kako pišemo dokaze}

Pravila sklepanja so kot pravila igre. Ne povedo, kako dobro igrati, samo kaj je dovoljeno. Seveda bomo hkrati s pravili sklepanja povedali nekaj namigov in nasvetov, kako dokaz poiščemo. A kot pri vsaki igri velja, da vaja dela mojstra.

Dokaz ima ugnezdeno strukturo: sestoji iz delov in pod-dokazov, ki sestoje iz nadaljnjih pod-dokazov itn., ki se zaključijo z osnovnimi dejstvi. Vsi ti kosi so s pomočjo pravil sklepanja zloženi v dokazno ">drevo"<.

Ko pišemo dokaz, moramo v vsakem trenutku poznati
%
\begin{itemize}
\item \textbf{cilj}: kaj trenutno dokazujemo in
\item \textbf{kontekst}: katere spremenljivke in predpostavke imamo trenutno na voljo.
\end{itemize}
%
Ko napravimo korak v dokazu, mora biti utemeljen z enim od pravil sklepanja. Dokaz je
popoln, ko smo utemeljili vse pod-dokaze, ki ga sestavljajo. Kot primer si poglejmo zelo podroben dokaz izjave
%
$(p \lor q) \land r \lthen (p \land r) \lor (q \land r)$.

\begin{center}
  \fbox{\parbox{0.6\textwidth}{
    Dokažimo $(p \lor q) \land r \lthen (p \land r) \lor (q \land r)$. \\
    \hbox{}\quad (1) Predpostavimo $(p \lor q) \land r.$ \\
    \hbox{}\quad (2) Zaradi (1) velja $p \lor q$. \\
    \hbox{}\quad (3) Zaradi (1) velja $r$. \\
    \hbox{}\quad Zaradi (2) lahko obravnavamo dva primera:\\
    \hbox{}\qquad \fbox{\parbox{0.5\textwidth}{
      (a) če velja $p$:\\
          \hbox{}\quad Dokažimo $(p \land r) \lor (p \land r)$.\\
          \hbox{}\quad Dokažimo levi disjunkt $p \land r$: \\
          \hbox{}\qquad (i) $p$ velja zaradi (a) \\
          \hbox{}\qquad (ii) $r$ velja zaradi (3).
    }} \\
    \hbox{}\qquad \fbox{\parbox{0.5\textwidth}{
      (b) če velja $q$:\\
          \hbox{}\quad Dokažimo $(p \land r) \lor (p \land r)$.\\
          \hbox{}\quad Dokažimo desni disjunkt $q \land r$: \\
          \hbox{}\qquad (i) $q$ velja zaradi (b) \\
          \hbox{}\qquad (ii) $r$ velja zaradi (3).
    }}
  }}
\end{center}
%
Dokaz bi bolj po človeško napisali takole:
%
\begin{quote}
  Predpostavimo $p \lor q$ in $r$. Če velja $p$, potem sledi $p \land r$ ter od tod $(p \land r) \lor (p \land r)$. Če pa velja $q$, sledi $q \land r$ ter spet $(p \land r) \lor (p \land r)$. $\Box$
\end{quote}
%
Ali pa kar takole:
%
\begin{quote}
  Očitno.
\end{quote}

Pravila sklepanja delimo na:
%
\begin{itemize}
\item \textbf{pravila vpeljave}, ki povedo, kako dokažemo izjavo, ter
\item \textbf{pravila uporabe}, ki povedo, kako lahko že znano izjavo uporabimo.
\end{itemize}
%
Poleg tega poznamo še pravila o zamenjavi:
%
\begin{itemize}
\item \textbf{zamenjava enakih izrazov}: izraz lahko vedno zamenjamo z njim enakim,
\item \textbf{zamenjava ekvivalentnih izjav}: izjavo vedno lahko zamenjamo z njej ekvivalentno.
\end{itemize}

Dokaz je skupek računskih korakov in sklepov, s katerimi utemeljimo izjavo. V vsakem
trenutku mora biti jasno, kaj dokazujemo, katere spremenljivke so veljavne in katere
predpostavke so na voljo.
%
Nekateri deli dokaza so samostojni pod-dokazi pomožnih izjav. Vse spremenljivke in
predpostavke, ki jih uvedemo v pod-dokazu, so na voljo izključno v pod-dokazu samem.

\subsection{Pravila vpeljave}
\label{sec:pravila-vpeljave}

S pravilom za vpeljavo \emph{neposredno} dokažemo izjavo. Za vsak veznik in kvantifikator ponazorimo, kako uporabimo
pripadajoče pravilo vpeljave.

\subsubsection{Konjunkcija}
%
\begin{quote}
  \sl
  Dokažimo $\phi \land \psi$.
  \begin{enumerate}
  \item Dokažimo $\phi$: \quad \brac{dokaz $\phi$}
  \item Dokažimo $\psi$: \quad \brac{dokaz $\psi$}
  \end{enumerate}
\end{quote}

\subsubsection{Disjunkcija}

Prvi način:
%
\begin{quote}
  \sl
  Dokažimo $\phi \lor \psi$.
  %
  \begin{itemize}
  \item[] Zadostuje dokazati levi disjunkt $\phi$: \quad \brac{dokaz $\phi$}
  \end{itemize}
\end{quote}
%
Drugi način:
%
\begin{quote}
  \sl
  Dokažimo $\phi \lor \psi$.
  %
  \begin{itemize}
  \item[] Zadostuje dokazati desni disjunkt $\psi$: \quad \brac{dokaz $\psi$}
  \end{itemize}
\end{quote}

\subsubsection{Implikacija}

\begin{quote}
  \sl
  Dokažimo $\phi \lthen \psi$:
  %
  \begin{itemize}
  \item[] Predpostavimo $\phi$. \\
        Dokažimo $\psi$: \quad \brac{dokaz $\psi$}
  \end{itemize}
\end{quote}

\subsubsection{Ekvivalenca}

\begin{quote}
  \sl
  Dokažimo $\phi \liff \psi$.
  \begin{enumerate}
  \item Dokažimo $\phi \lthen \psi$: \quad \brac{dokaz $\phi \lthen \psi$}
  \item Dokažimo $\psi \lthen \phi$: \quad \brac{dokaz $\psi \lthen \phi$}
  \end{enumerate}
\end{quote}

\subsubsection{Resnica}

Resnice $\top$ ni treba dokazovati, zapišemo ">očitno"<. \footnote{
V praksi $\top$ nastopi kot izjava, ki jo želimo dokazati, ko neko drugo izjavo poenostavimo. Primer: ko dokazujemo
$12^2 + 12^2 < 17^2$, najprej izračunamo, da je to ekvivalentno $288 < 289$, kar je ekvivalentno $\top$. S tem je dokaz
zaključen, saj smo dobili $\top$.}

\subsubsection{Neresnica}

Kadar dokazujemo $\bot$, pravimo, da ">iščemo protislovje"<.

\begin{quote}
  \sl
  Poiščimo protislovje.
  \begin{enumerate}
  \item Dokažimo $\phi$: \quad \brac{dokaz $\phi$}
  \item Dokažimo $\lnot \phi$: \quad \brac{dokaz $\lnot\phi$}
  \end{enumerate}
\end{quote}

\subsubsection{Negacija}

\begin{quote}
  \sl
  Dokažimo $\lnot \psi$:
  %
  \begin{itemize}
  \item[] Predpostavimo $\psi$. \\
          Poiščimo protislovje: \quad \dots
  \end{itemize}
\end{quote}
%
Opomba: ni nujno, da poiščemo protislovje med $\psi$ in $\lnot\psi$, vsako protislovje je sprejemljivo.

\subsubsection{Univerzalna izjava}

\begin{quote}
  \sl
  Dokažimo $\all{x \in A} \phi(x)$.
  %
  \begin{enumerate}
  \item[] Naj bo $x \in A$. \\
          Dokažemo $\phi(x)$: \quad \brac{dokaz $\phi(x)$}
  \end{enumerate}
\end{quote}
%
\textbf{Pozor:} spremenljivka $x$ mora biti \emph{sveža}, se pravi, da je ne uporabljamo nikjer drugje. Če jo, najprej izberemo svežo spremenljivko~$y$ in $x$ preimenujemo v~$y$.

\subsubsection{Eksistenčna izjava}

\begin{quote}
  \sl
  Dokažimo $\some{x \in A} \phi(x)$:
  %
  \begin{itemize}
  \item[] Podamo $x \mathbin{{:}{=}} \langle\text{izraz}\rangle$. \\
          Dokažemo $\langle\text{izraz}\rangle \in A$: \quad \dots \\
          Dokažemo $\phi(\langle\text{izraz}\rangle)$: \quad \dots
  \end{itemize}
\end{quote}
%
Opomba: $\langle\text{izraz}\rangle$ sme vsebovati vse proste spremenljivke, ki so trenutno na voljo ($x$ \emph{ni} na
voljo).

\subsection{Pravila uporabe}

Pravila uporabe nam povedo, kako iz predpostavk in že znanih dejstev izpeljemo nova dejstva.

\subsubsection{Konjunkcija} 

\begin{quote}
  \sl
  Vemo, da velja $\phi \land \psi$.\\
  Torej velja $\phi$.\\
  Torej velja $\psi$.
\end{quote}
%
Opomba: v praksi tega koraka ne delamo, ampak namesto predpostavke $\phi \land \psi$ kar takoj vpeljemo ločeni
predpostavki $\phi$ in $\psi$.


\subsubsection{Disjunkcija}
%
\begin{quote}
  \sl
  Dokažimo $\rho$.\\
  Vemo, da velja $\phi \lor \psi$, torej obravnavamo primera:\\
  %
  \begin{enumerate}
  \item Če velja $\phi$: \\
        Dokažemo $\rho$: \quad \brac{dokaz $\rho$}
  \item Če velja $\psi$: \\
        Dokažemo $\rho$: \quad \brac{dokaz $\rho$}
  \end{enumerate}
\end{quote}

\subsubsection{Implikacija}

\begin{quote}
  \sl
  Vemo, da velja $\phi \lthen \psi$.
  %
  \begin{itemize}
  \item[] Dokažimo $\phi$: \quad \brac{dokaz $\phi$}
  \end{itemize}
  %
  Torej velja tudi $\psi$.
\end{quote}

\subsubsection{Resnica}

Resnica ni uporabna kot predpostavka in jo lahko zavržemo.

\subsubsection{Neresnica}

\begin{quote}
  \sl
  Dokažimo $\rho$:
  \begin{itemize}
  \item[]
    Ugotovimo, da velja $\bot$.\\
    Ker iz neresnice sledi karkoli, velja $\rho$.
  \end{itemize}
\end{quote}


\subsubsection{Negacija}

Negacijo $\lnot \phi$ uporabimo tako, da dokažemo $\phi$ in zaključimo dokaz.
%
\begin{quote}
  \sl
  Dokažimo $\rho$.\\
  Vemo, da velja $\lnot\phi$.
  %
  \begin{itemize}
  \item Dokažimo $\phi$: \quad \brac{dokaz $\phi$}
  \end{itemize}
  %
  Torej velja $\rho$.
\end{quote}

\subsubsection{Univerzalna izjava}

\begin{quote}
  \sl
  Vemo, da velja $\all{x \in A} \phi(x)$.\\
  Vemo, da je $\langle\text{izraz}\rangle \in A$.\\
  Torej velja $\phi(\langle\text{izraz}\rangle)$.
\end{quote}

\subsubsection{Eksistenčna izjava}

\begin{quote}
  \sl
  Dokažimo $\rho$.\\
  Vemo, da velja $\some{x \in A} \phi(x)$.
  %
  \begin{enumerate}
  \item[] Imamo $x \in A$, za katerega velja $\phi(x)$.\\
       Dokažemo $\rho$: \quad \brac{dokaz $\rho$}
  \end{enumerate}
\end{quote}
%
\textbf{Pozor:} spremenljivka $x$ mora biti sveža, se pravi, da se ne pojavlja v $\rho$ ali kjerkoli drugje. Če se~$x$ pojavi kje drugje, ga moramo najprej nadomestiti s svežo spremenljivko~$y$.

\subsubsection{Izključena tretja možnost in dokaz s protislovjem}

\textbf{Pravilo izključene tretje možnosti} pravi, da vedno velja $\phi \lor \lnot\phi$ in ga uporabimo takole:
%
\begin{quote}
  \sl
  Dokažimo $\rho$.\\
  Velja $\phi \lor \lnot \phi$:
  %
  \begin{enumerate}
  \item Če velja $\phi$:\\
        Dokažemo $\rho$: \quad \dots
  \item Če velja $\lnot\phi$. \\
        Dokažemo $\rho$: \quad \dots
  \end{enumerate}
\end{quote}
%
\textbf{Dokaz s protislovjem} poteka takole:
%
\begin{quote}
  \sl
  Dokažimo $\rho$. Dokazujemo s protislovjem:
  %
  \begin{itemize}
  \item[] Predpostavimo $\lnot\rho$.\\
        Poiščimo protislovje: \quad \dots
  \end{itemize}
\end{quote}
%
Opomba: dokaz s protislovjem in pravilo vpeljave za negacijo sta \emph{različni}
pravili!


\chapter{Logika in pravila sklepanja (dodatno poglavje)}
\label{chap:logika}


\textbf{Opomba:} To poglavje je del učbenika v nastajanju in ni povsem v skladu s predavanji. Kljub temu ga vključujem v te zapiske, ker vsebuje precej koristnih nasvetov in misli.

%%%%%%%%%%%%%%%%%%%%%%%%%%%%%%%%%%%%%%%%%%%%%%%%%%%%%%%%%%%%%%%%%%%%%%
\section{Kaj je matematični dokaz?}
\label{sec:kaj-je-dokaz}

V srednji šoli se dijaki pri matematiki učijo, \emph{kako} se kaj
izračuna. Na univerzi imajo študentje matematike pred seboj
zahtevnejšo nalogo: poleg \emph{kako} morajo vedeti tudi \emph{zakaj}.
Od njih se pričakuje, da bodo računske postopke znali tudi utemeljiti,
ne pa samo slediti pravilom, ki jih je predpisal učitelj. Razumeti
morajo dokaze znamenitih izrekov in sami poiskati dokaze preprostih
izjav. Da bi se lažje spopadli s temi novimi nalogami, bomo prvi del
predmeta Logika in množice posvetili matematični infrastrukturi:
izjavam, pra\-vi\-lom sklepanja in dokazom. Učili se bomo, kako pišemo
dokaze, kako jih analiziramo in kako jih sami poiščemo.

Osrednji pojem matematične aktivnosti je \emph{dokaz}. Namen dokaza je
s pomočjo točno določenih in vnaprej dogovorjenih \emph{pravil
  sklepanja} utemeljiti neko matematično \emph{izjavo}. Načeloma mora
dokaz vsebovati vse podrobnosti in natanko opisati posamezne korake
sklepanja, ki privedejo do želene matematične izjave. Ker bi bili taki
dokazi zelo dolgi in bi vsebovali nezanimive podrobnosti, matematiki
običajno predstavijo samo oris ali glavno zamisel dokaza. Izkušenemu
matematiku to zadošča, saj zna oris sam dopolniti do pravega dokaza.
Matematik začetnik seveda potrebuje več podrobnosti. Poglejmo si
primer.

\begin{izrek}
  \label{izr:n3-n-deljivo-3}
  Za vsako naravno število $n$ je $n^3 - n$ deljivo s~$3$.
\end{izrek}

\noindent
Po kratkem premisleku bi izkušeni matematik zapisal:

\begin{quote}
  \begin{dokaz}
    Očitno.
  \end{dokaz}
\end{quote}

\noindent
To ni dokaz, izkušeni matematik nam le dopoveduje, da je (zanj) izrek
zelo lahek in da nima smisla izgubljati časa s pisanjem dokaza.
Začetnik, ki težko razume že sam izrek, bo ob takem ">dokazu"< seveda
zgrožen. Verjetno bo najprej preizkusil izrek na nekaj primerih:
%
\begin{align*}
  1^3 - 1 &= 0,\\
  2^3 - 2 &= 8 - 2 = 6,\\
  3^3 - 3 &= 27 - 4 = 24,\\
  4^3 - 4 &= 64 - 4 = 60.
\end{align*}
%
Res dobivamo večkratnike števila~$3$. Ali smo izrek s tem dokazali? Seveda ne!
Preizkusili smo le štiri primere, ostane pa jih še neskončno mnogo. Kdor
misli, da lahko iz nekaj primerov sklepa na splošno veljavnost, naj v poduk
vzame naslednjo nalogo.

\begin{vaja}
  Ali je $n^2 - n + 41$ praštevilo za vsako naravno število~$n$?
\end{vaja}

\noindent
Ko izkušenega matematika prosimo, da naj nam vsaj pojasni idejo dokaza,
zapiše:

\begin{quote}
  \begin{dokaz}
    Število $n^3 - n$ je zmnožek treh zaporednih naravnih števil.
  \end{dokaz}
\end{quote}

\noindent
To še vedno ni dokaz, ampak samo namig. Starejši študenti matematike pa
bi iz namiga morali znati sestaviti naslednji dokaz:

\begin{quote}
  \begin{dokaz}
    Ker je $n^3 - n = (n-1) \cdot n \cdot (n+1)$, je $n^3 - n$ zmnožek
    treh zaporednih naravnih števil, od katerih je eno deljivo s~$3$,
    torej je tudi $n^3 - n$ deljivo s~$3$.
  \end{dokaz}
\end{quote}

\noindent
(Mimogrede, s škatlico $\Box$ označimo konec dokaza.) Čeprav bi bila
večina matematikov s tem dokazom zadovoljna, bi morali za popoln dokaz
preveriti še nekaj podrobnosti:
%
\begin{enumerate}
\item Ali res velja $n^3 - n = (n-1) \cdot n \cdot (n+1)$?
\item Ali je res, da je izmed treh zaporednih naravnih števil eno
  vedno deljivo s~$3$?
\item Ali je res, da je zmnožek treh števil deljiv s~$3$, če je eno od
  števil deljivo s~$3$?
\end{enumerate}
%
S srednješolskim znanjem algebre ugotovimo, da je odgovor na prvo
vprašanje pritrdilen. Tudi odgovora na drugo in tretje vprašanje sta
očitno pritrdilna, mar ne? To pa ne pomeni, da ju ni treba dokazati.
Nasprotno, zgodovina matematike nas uči, da moramo prav ">očitne"<
izjave še posebej skrbno preveriti.

\begin{vaja}
  Kakšno je tvoje mnenje o resničnosti naslednjih izjav? Vprašaj
  starejše kolege, asistente in učitelje, kaj menijo oni. Ali znajo
  svoje mnenje utemeljiti z dokazi?
  %
  \begin{enumerate}
  \item Sodih števil je manj kot naravnih števil.
  \item Kroglo je mogoče razdeliti na pet delov tako, da lahko iz njih
    sestavimo dve krogli, ki sta enako veliki kot prvotna krogla.
  \item Sklenjena krivulja v ravnini, ki ne seka same sebe, razdeli
    ravnino na dve območji, eno omejeno in eno neomejeno.
  \item S krivuljo ne moremo prekriti notranjosti kvadrata.
  \item Če ravnino razdelimo na tri območja, potem zagotovo obstaja
    točka, ki je dvomeja in ni tromeja med območji.
  \end{enumerate}
\end{vaja}

\noindent
%
Vrnimo s k izreku~\ref{izr:n3-n-deljivo-3}. Če dokaz zapišemo preveč
podrobno, postane dolgočasen in ne\-ra\-zumljiv:

\begin{quote}
  \begin{dokaz}
    Naj bo $n$ poljubno naravno število. Tedaj velja
    %
    \begin{align*}
      n^3 - n
      &= n \cdot n^2 - n \cdot 1 \\
      &= n \cdot (n^2 - 1) \\
      &= n \cdot ((n + 1) \cdot (n - 1)) \\
      &= n \cdot ((n - 1) \cdot (n + 1)) \\
      &= (n \cdot (n - 1)) \cdot (n + 1) \\
      &= (n - 1) \cdot n \cdot (n + 1).
    \end{align*}
    %
    Vidimo, da je $n^3 - n$ zmnožek treh zaporednih naravnih števil.
    Dokažimo, da je eno od njih deljivo s~$3$. Število $n$ lahko
    enolično zapišemo kot $n = 3 k + r$, kjer je $k$ naravno število
    in $r = 0$, $r = 1$ ali $r = 2$. Obravavajmo tri primere:
    %
    \begin{itemize}
    \item če je $r = 0$, je $n = 3 k$, zato je $n$ deljiv s~$3$,
    \item če je $r = 1$, je $n - 1 = (3 k + 1) - 1 = 3 k + (1 - 1) = 3
      k + 0 = 3 k$, zato je $n-1$ deljiv s~$3$,
    \item če je $r = 2$, je $n + 1 = (3 k + 2) + 1 = 3 k + (2 + 1) = 3
      k + 3 = 3 k + 3 \cdot 1 = 3 (k +1)$, zato je $n+1$ deljiv s~$3$.
    \end{itemize}
    %
    Vemo torej, da je $n - 1$, $n$ ali $n + 1$ deljiv s~$3$.
    Obravnavamo tri primere:
    %
    \begin{itemize}
    \item Če je $n - 1$ deljiv s~$3$, tedaj  obstaja naravno število
      $k$, da je $n - 1 = 3 k$. V tem primeru je $(n - 1) n (n + 1) =
      (3 k) n (n + 1) = 3 (k n (n + 1))$, zato je $(n - 1) n (n + 1)$
      deljivo s~$3$.
    \item Če je $n$ deljiv s~$3$, tedaj obstaja naravno število $k$,
      da je $n = 3 k$. V tem primeru je $(n - 1) n (n + 1) = (n - 1)
      (3 k) n (n + 1) = (3 k) (n - 1) (n + 1) = 3 (k (n - 1) (n +
      1))$, zato je $(n - 1) n (n + 1)$ deljivo s~$3$.
    \item Če je $n + 1$ deljiv s~$3$, tedaj obstaja naravno število
      $k$, da je $n + 1 = 3 k$. V tem primeru je $(n - 1) n (n + 1) =
      (n - 1) n (3 k) = (n - 1) (3 k) n = (3 k) (n - 1) n = 3 (k (n -
      1) n)$, zato je $(n - 1) n (n + 1)$ deljivo s~$3$.
    \end{itemize}
    %
    V vsakem primeru je $(n - 1) n (n + 1)$ deljivo s~$3$. Ker smo
    dokazali, da je $n^3 = n = (n - 1) n (n + 1)$, je tudi $n^3 - n$
    deljivo s~$3$.
  \end{dokaz}
\end{quote}

\begin{vaja}
  S kolegi se igraj naslednjo igro.\footnote{%
    Igranje odsvetujemo zunaj prostorov Fakultete za matematiko in fiziko.}
  Prvi igralec v zgornjem dokazu poišče korak, ki ga je treba še dodatno
  utemeljiti. Drugi igralec ga utemelji. Nato prvi igralec poišče nov korak,
  ki ga je treba še dodatno utemeljiti in igra se ponovi. Zgubi tisti, ki se prvi naveliča igrati. Ali lahko igra traja neskončno dolgo?
\end{vaja}

Matematični dokaz ima dvojno vlogo. Po eni strani je utemeljitev matematične
izjave, zato mora biti čim bolj podroben. V idealnem primeru bi bil dokaz
zapisan tako, da bi lahko njegovo pravilnost preverili mehansko, z
računalnikom. Po drugi strani je dokaz sredstvo za komunikacijo idej med
matematiki, zato mora vsebovati ravno pravo mero podrobnosti. Mera pa je
odvisna od tega, komu je dokaz namenjen. Te socialne komponente se študenti
učijo skozi prakso v toku študija. Dokazu kot povsem matematičnemu pojmu pa se
bomo posvetili prav pri predmetu Logika in množice. Pojasnili bomo, kaj je
dokaz kot matematični konstrukt in kako ga zapišemo tako podrobno, da je res
mehansko preverljiv. Naučili se bomo tudi nekaj preprostih tehnik iskanja
dokazov, ki pa še zdaleč ne bodo zadostovale za reševanje zares zanimivih
matematičnih problemov, ki zahtevajo poglobljeno znanje, vztrajnost, kanček
talenta in nekaj sreče.


%%%%%%%%%%%%%%%%%%%%%%%%%%%%%%%%%%%%%%%%%%%%%%%%%%%%%%%%%%%%%%%%%%%%%%

\section{Simbolni zapis matematičnih izjav}
\label{sec:simbolni-zapis-izjav}

Matematična \textbf{izjava} je smiselno besedilo, ki izraža kako lastnost ali
razmerje med matematičnimi objekti (števili, liki, funkcijami, množicami
itn.). Primeri matematičnih izjav:
%
\begin{itemize}
\item $2 + 2 = 5$.
\item Točke $P$, $Q$ in $R$ so kolinearne.
\item Enačba $x^2 + 1 = 0$ nima realnih rešitev.
\item $a > 5$.
\item $\phi \lor \psi \lthen (\lnot \phi \lthen \psi)$.
\end{itemize}
%
Vidimo, da je lahko izjava resnična, neresnična, ali pa je resničnost
izjave \emph{odvisna} od vrednosti spremenljivk, ki nastopajo v njej.
Primeri besedila, ki niso matematične izjave:
%
\begin{itemize}
\item Ali je $2 + 2 = 5$?
\item Za vsak $x > 5$.
\item Študenti bi morali znati reševati diferencialne enačbe.
\item Od nekdaj lepe so Ljubljanke slovele, al lepše od Urške bilo ni nobene.
\item $\phi \lor ) \psi \lthen \psi$.
\end{itemize}
%
Matematične izjave običajno pišemo kombinirano v naravnem jeziku in z
matematični simboli, saj so tako najlažje razumljive ljudem. Za
potrebe matematične logike pa izjave pišemo \emph{samo} z
matematičnimi simboli. Tako zapisani izjavi pravimo \textbf{logična
  formula}. V ta namen moramo nadomestiti osnovne gradnike izjav, kot
so ">in"<, ">ali"< in ">za vsak"<, z \textbf{logičnimi operacijami}.
Le-te delimo na tri sklope. V prvi sklop sodita \textbf{logični
  konstanti}:
%
\begin{itemize}
\item resnica $\top$,
\item neresnica $\bot$.
\end{itemize}
%
V računalništvu resnico $\top$ pogosto označimo z $1$ ali \texttt{True} in
neresnico $\bot$ z $0$ ali \texttt{False}. Naslednji sklop so \textbf{logični
vezniki}, s katerimi sestavljamo nove izjave iz že danih:
%
\begin{itemize}
\item konjunkcija $\phi \land \psi$, beremo ">$\phi$ in $\psi$"<,
\item disjunkcija $\phi \lor \psi$, beremo ">$\phi$ ali $\psi$"<,
\item implikacija $\phi \lthen \psi$, beremo ">če $\phi$ potem $\psi$"<,
\item ekvivalenca $\phi \liff \psi$, beremo ">$\phi$ če, in samo če, $\psi$"< ali pa ">$\phi$ natanko tedaj, kadar~$\psi$"<,
\item negacija $\lnot \phi$, beremo ">ne $\phi$"<,
\end{itemize}
%
V tretji sklop sodita \textbf{logična kvantifikatorja}:
%
\begin{itemize}
\item univerzalni kvantifikator $\all{x \in S} \phi$, beremo ">za vse $x$
  iz $S$ velja $\phi$"<,
\item eksistenčni kvantifikator $\some{x \in S} \phi$, beremo ">obstaja
  $x$ v $S$, da velja $\phi$"<,
\end{itemize}
%
Pri tem je $S$ množica, razred\footnote{V poglavju~\ref{chap:mnozice}
  bomo spoznali razliko med množicami in razredi, zaenkrat si $S$
  predstavljamo kot množico.} ali tip spremenljivke~$x$. V praksi se
uporablja več inačic zapisa za kvantifikatorje, kot so ">$\forall x : S
.\, \phi$"<, ">$\forall x \in S : \phi$"< in ">$(\forall x \in S)
\phi$"<. Srečamo tudi zapis ">$\phi, \forall x \in S$"<, ki pa ga
odsvetujemo.

\textbf{Neomejena kvantifikatorja} $\all{x} \phi$ in
$\some{x} \phi$ se uporabljata, kadar je vnaprej znana množica $S$,
po kateri teče spremenljivka~$x$. V matematičnem besedilu je običajno
razvidna iz spremnega besedila, včasih pa je treba upoštevati
ustaljene navade: $n$ je naravno ali celo število, $x$ realno, $f$ je
funkcija ipd.

V uporabi so nekatere ustaljene okrajšave:
%
\begin{xalignat*}{3}
  &\some{x,y \in S} \phi,&
  &\text{pomeni}&
  &\some{x}{S} (\some{y}{S} \phi),\\
  %
  &\all{x \in S,y \in T} \phi,&
  &\text{pomeni}&
  &\all{x \in S}(\all{y \in T} \phi),\\
  %
  &\phi \liff \psi \liff \rho \liff \sigma&
  &\text{pomeni}&
  &(\phi \liff \psi) \land (\psi \liff \rho) \land (\rho \liff \sigma),\\
  %
  &f(x) = g(x) = h(x) = i(x)&
  &\text{pomeni}&
  &f(x) = g(x) \land g(x) = h(x) \land h(x) = i(x),\\
  &a \leq b < c \leq d&
  &\text{pomeni}&
  &a \leq b \land b < c \land c \leq d.
\end{xalignat*}
%
Nekatere okrajšave odsvetujemo. V nizu neenakosti naj gredo vse
primerjave v isto smer. Torej ne pišemo $a \leq b < c \geq d$, ker se
zlahka zmotimo in mislimo, da velja $a \geq d$. To bi morali zapisati
ločeno kot $a \leq b < c$ in $c \geq d$. Prav tako ne nizamo neenakosti,
saj premnogi iz $f(x) \neq g(x) \neq h(x)$ ">sklepajo"< $f(x) \neq
h(x)$, čeprav neenakost \emph{ni} tranzitivna relacija. Zapis $f(x) =
g(x) \neq h(x) = i(x)$ je v redu, saj ena sama neenakost ne povzroči
težav.

\begin{vaja}
  Zapiši $f(x) = g(x) \neq h(x) = i(x)$ brez okrajšav.
\end{vaja}

Povejmo še nekaj o pisanju oklepajev. Oklepaji povedo, katera
operacija ima prednost. Kadar manjkajo, moramo poznati dogovorjeno
\textbf{prioriteto} operacij. Na primer, ker ima množenje višjo
prioriteto kot seštevanje, je $5 \cdot 3 + 8$ enako $(5 \cdot 3) + 8$
in ne $5 \cdot (3 + 8)$. Tudi logične operacije imajo svoje
prioritete, ki pa niso tako splošno znane kot prioritete aritmetičnih
operacij. Zato bodite pazljivi, ko berete tuje besedilo.

Mi bomo privzeli naslednje prioritete logičnih operacij:
%
\begin{itemize}
\item negacija $\lnot$ ima prednost pred
\item konjunkcijo $\land$, ki ima prednost pred
\item disjunkcijo $\lor$, ki ima prednost pred
\item implikacijo $\lthen$, ki ima prednost pred
\item kvantifikatorjema $\forall$ in $\exists$.
\end{itemize}
%
Na primer:
%
\begin{itemize}
\item $\lnot \phi \lor \psi$ je isto kot $(\lnot \phi) \lor \psi$,
\item $\lnot \lnot \phi \lthen \phi$ je isto kot $(\lnot(\lnot\phi))
  \lthen \phi$,
\item $\phi \lor \psi \land \rho$ je isto kot $\phi \lor (\psi \land \rho)$,
\item $\phi \land \psi \lthen \phi \lor \psi$ je isto kot $(\phi
  \land \psi) \lthen (\phi \lor \psi)$,
\item $\all{x \in S} \phi \lthen \psi$ je isto kot $\all{x \in S} (\phi
    \lthen \psi)$,
\item $\some{x \in S} \phi \land \psi$ je isto kot $\some{x \in S} (\phi
    \land \psi)$.
\end{itemize}

V aritmetiki poznamo poleg prioritete operacij tudi \textbf{levo} in
\textbf{desno asociranost}. Denimo, seštevanje je levo asocirano,
ker beremo $5 + 3 + 7$ kot $(5 + 3) + 7$, saj najprej izračunamo $5 +
3$ in nato $8 + 7$. Pri seštevanju to sicer ni pomembno in bi lahko
seštevali tudi $3 + 7$ in nato $5 + 10$. Drugače je z odštevanjem,
kjer $5 - 3 - 7$ pomeni $(5 - 3) - 7$ in ne $5 - (3 - 7)$. Tudi za
logične operacije velja dogovor o njihovi asociranosti. Konjunkcija in
disjunkcija sta levo asocirani:
% 
\begin{align*}
  \phi \land \psi \land \rho
  &\qquad\text{pomeni}\qquad
  (\phi \land \psi) \land \rho,\\
  \phi \lor \psi \lor \rho
  &\qquad\text{pomeni}\qquad
  (\phi \lor \psi) \lor \rho.
\end{align*}
%
Za disjunkcijo in konjunkcijo sicer ni pomembno, kako postavimo
oklepaje, ker sta obe možnosti med seboj ekvivalentni, vendar je prav,
da natančno določimo, katera od njiju je mišljena. V logiki je
implikacija desno asocirana:
%
\begin{equation*}
  \phi \lthen \psi \lthen \rho
  \qquad\text{pomeni}\qquad
  \phi \lthen (\psi \lthen \rho).
\end{equation*}
%
Tu \emph{ni} vseeno, kako postavimo oklepaje, saj $\phi \lthen (\psi
\lthen \rho)$ in $(\phi \lthen \psi) \lthen \rho$ v splošnem nista
ekvivalentna. Vendar pozor! Ko matematiki, ki niso logiki, v
matematičnem besedilu zapišejo
%
\begin{equation*}
  \phi \lthen \psi \lthen \rho,
\end{equation*}
%
s tem skoraj vedno mislijo
%
\begin{equation*}
  (\phi \lthen \psi) \land (\psi \lthen \rho).
\end{equation*}
%
Zakaj? Zato ker je to priročen zapis, ki nakazuje zaporedje sklepov
">iz $\phi$ sledi $\psi$ in nato iz $\psi$ sledi $\rho$"<, še posebej,
če je zapisan v več vrsticah. Recimo, za nenegativni števili $x$ in
$y$ bi takole zapisali utemeljitev neenakosti med aritmetično in
geometrijsko sredino:
%
\begin{align*}
  & (x - y)^2 \geq 0 \lthen \\
  & x^2 - 2 x y + y^2 \geq 0 \lthen
  \tag{razstavimo}\\
  & x^2 + 2 x y + y^2 \geq 4 x y \lthen
  \tag{prištejemo $4 x y$}\\
  & (x + y)^2 \geq 4 x y \lthen
  \tag{faktoriziramo}\\
  & \frac{(x + y)^2}{4} \geq x y \lthen
  \tag{delimo s $4$}\\
  & \frac{x+y}{2} \geq \sqrt{x y}.
  \tag{korenimo}
\end{align*}
%ANDREJ: meni je Gordon rekel, da utemeljitve sledijo sklepu, torej so
% eno vrstico niže.
%
Matematiki radi celo spustijo znak $\lthen$ in preprosto vsak
naslednji sklep napišejo v svojo vrstico. Ker torej velja tak ustaljen
način pisanja zaporedja sklepov, je varneje pisati $\phi \lthen (\psi
\lthen \rho)$ brez oklepajev, da ne povzročamo zmede.
%ANDREJ: zadnjega stavka ne razumem.

%%%%%%%%%%%%%%%%%%%%%%%%%%%%%%%%%%%%%%%%%%%%%%%%%%%%%%%%%%%%%%%%%%%%%%

\section{Kako beremo in pišemo simbolni zapis}
\label{sec:simbolni-zapis}

Izjave, zapisane v simbolni obliki, ni težko prebrati. Na primer,
%
\begin{equation*}
  \all{x, y \in \RR}
    x^2 = 4 \land y^2 = 4 \lthen x = y,
\end{equation*}
%
preberemo:
%
\begin{quote}
  ">Za vse realne $x$ in $y$, če je $x^2$ enako $4$ in $y^2$ enako
  $4$, potem je $x$ enako $y$."<
\end{quote}
%
Več izkušenj pa je potrebnih, da \emph{razumemo} matematični pomen
take izjave, v tem primeru:
%
\begin{quote}
  ">Enačba $x^2 = 4$ ima največ eno realno rešitev."<
\end{quote}
%
Začetnik potrebuje nekaj vaje, da se navadi brati simbolni zapis. Tudi
prevod v obratno smer, iz besedila v simbolno obliko, ni enostaven,
zato povejmo, kako se prevede nekatere standardne fraze.

\subsubsection{">$\phi$ je zadosten pogoj za $\psi$."<}

To pomeni, da zadošča dokazati $\phi$ zato, da dokažemo $\psi$, ali v
simbolni obliki
%
\begin{equation*}
  \phi \lthen \psi.
\end{equation*}

\subsubsection{">$\phi$ je potreben pogoj za $\psi$."<}

To pomeni, da $\psi$ ne more veljati, ne da bi veljal~$\phi$. Z drugimi
besedami, če velja $\psi$, potem velja tudi $\phi$, kar se v simbolni obliki
zapiše
%
\begin{equation*}
  \psi \lthen \phi.
\end{equation*}

\subsubsection{">$\phi$ je zadosten in potreben pogoj za $\psi$."<}

To je kombinacija prejšnjih dveh primerov, ki trdi, da iz $\phi$ sledi
$\psi$ in iz $\psi$ sledi $\phi$, kar pa je ekvivalenca:
%
\begin{equation*}
  \phi \liff \psi.
\end{equation*}

\begin{vaja}
  Je ">$n$ je sod in $n > 2$"< \textbf{potreben} ali \textbf{zadosten}
  pogoj za ">$n$ ni praštevilo"<?
\end{vaja}


\subsubsection{">Naslednje izjave so ekvivalentne: $\phi$, $\psi$, $\rho$ in $\sigma$."<}

To pomeni, da sta vsaki dve izmed danih izjav ekvivalentni, se pravi
%
\begin{equation*}
  (\phi \liff \psi) \land (\phi \liff \rho) \land (\phi \liff \sigma) \land (\psi \liff \rho)
  \land (\psi \liff \sigma) \land (\rho \liff \sigma).
\end{equation*}
%
Ker je ekvivalenca tranzitivna relacija, ni treba obravnavati vseh
kombinacij, zadostujejo že tri, ki dane izjave ">povežejo"< med seboj:
%
\begin{equation*}
  (\phi \liff \psi) \land (\psi \liff \rho) \land (\rho \liff \sigma).
\end{equation*}
%
To pišemo krajše kar kot
%
\begin{equation*}
  \phi \liff \psi \liff \rho \liff \sigma,
\end{equation*}
%
čeprav je formalno gledano tako zapis nepravilen. V
razdelku~\ref{sec:ekvivalenca} bomo spoznali, kako se tako zaporedje
ekvivalenc dokaže s ciklom implikacij $\phi \lthen \psi \lthen \rho
\lthen \sigma \lthen \phi$.

\begin{vaja}
  Podaj konkretne primere izjav $\phi$, $\psi$ in $\rho$, iz katerih
  je razvidno, da izjava $(\phi \liff \psi) \land (\psi \liff \rho)$
  \emph{ni} ekvivalentna niti $(\phi \liff \psi) \liff \rho$ niti
  $\phi \liff (\psi \liff \rho)$.
\end{vaja}



\subsubsection{">Za vsak $x$ iz $S$, za katerega velja $\phi$, velja tudi
  $\psi$."<}

To lahko preberemo tudi kot ">Za vsak $x$ iz $S$, če zanj velja $\phi$,
potem velja $\psi$,"< kar je v simbolni obliki
%
\begin{equation*}
  \all{x \in S} \phi \lthen \psi.
\end{equation*}
%
Tudi izjave oblike ">vsi $\phi$-ji so $\psi$-ji"< so te oblike, denimo ">vsa
od dva večja praštevila so liha"< zapišemo
%
\begin{equation*}
  \all{n \in \NN} n > 2 \land \text{$n$ je praštevilo} \lthen \text{$n$ je lih}.
\end{equation*}

\begin{vaja}
  V simbolni obliki zapiši ">$n$ je lih"< in ">$n$ je praštevilo"<.
  Namig: $n$ je lih, kadar obstaja naravno število $k$, za katerega
  velja $n = 2 k + 1$, in $n$ je praštevilo, kadar \emph{ni} zmnožek
  dveh naravnih števil, ki sta obe večji od~$1$.
\end{vaja}


\subsubsection{">Enačba $f(x) = g(x)$ nima realne rešitve."<}

To lahko povemo takole: ni res, da obstaja $x \in \RR$, za katerega bi
veljalo $f(x) = g(x)$. S simboli zapišemo
%
\begin{equation*}
  \lnot \some{x \in \RR} f(x) = g(x).
\end{equation*}
%
Opozoriti velja, da iz same enačbe ne moremo vedno sklepati, kaj je
neznanka. V enačbi $a x^2 + b x + c = 0$ bi za neznanko lahko načeloma
imeli katerokoli od štirih spremenljivk $a$, $b$, $c$ in $x$, ali pa
kar vse. Večina matematikov bi sicer uganila, da je najverjetneje
neznanka $x$, vendar se v splošnem ne moremo zanašati na običaje in
uganjevanje, ampak moramo točno povedati, kateri simboli so
\textbf{neznanke} in kateri \textbf{parametri}.

\begin{vaja}
  Zapiši v simbolni obliki: ">Sistem enačb
  %
  \begin{align*}
    a_1 x + b_1 y &= c_1,\\
    a_2 x + b_2 y &= c_2
  \end{align*}
  %
  nima pozitivnih realnih rešitev $x, y$."<
\end{vaja}

\begin{vaja}
  Zapiši v simbolni obliki:
  \begin{enumerate}
  \item ">Enačba $f(x) = g(x)$ ima največ eno realno rešitev."<
  \item ">Enačba $f(x) = g(x)$ ima več kot eno realno rešitev."<
  \item ">Enačba $f(x) = g(x)$ ima natanko dve realni rešitvi."<
  \end{enumerate}
\end{vaja}


\subsubsection{">Brez izgube za splošnost."<}

V matematičnih besedilih najdemo frazo ">brez izgube za splošnost"<
kot v naslednjem primeru.

\begin{izrek}
  \label{izrek:abc-vsota-razlik-soda}
  Za vsa cela števila $a$, $b$ in $c$ je $|a-b|+|b-c|+|c-a|$ sodo
  število.
\end{izrek}

\begin{dokaz}
  Brez izgube za splošnost smemo predpostaviti $a \geq b \geq c$.
  Tedaj velja
  %
  \begin{equation*}
    |a-b| + |b-c| + |c-a| = (a - b) + (b - c) - (c - a) = 2 (a - c),
  \end{equation*}
  %
  kar je sodo število.
\end{dokaz}

Fraza ">brez izgube za splošnost"< nakazuje, da dokaz obravnava le eno
od večih možnosti. Načeloma bi morali obravnavati tudi ostale
možnosti, ki pa jih je pisec dokaza opustil, ker so bodisi zelo lahke
bodisi zelo podobne tisti, ki jo dokaz obravnava. Za začetnika je
najtežje dognati, katere so preostale možnosti in zakaj se je pisec
dokaza pravzaprav odločil zanje. Avtor zgornjega dokaza je verjetno
opazil, da števila $a$, $b$ in $c$ v izrazu $|a-b|+|b-c|+|c-a|$
nastopajo \emph{simetrično}: če jih premešamo, se izraz ne spremeni.
Denimo, ko zamenjamo $a$ in $b$, dobimo $|b-a|+|a-c|+|c-b|$, kar je
enako prvotnemu izrazu $|a-b|+|b-c|+|c-a|$. Torej lahko izmed šestih
možnosti
%
\begin{xalignat*}{3}
  & a \geq b \geq c,&
  & a \geq c \geq b,&
  & b \geq a \geq c,\\
  & b \geq c \geq a,&
  & c \geq a \geq b,&
  & c \geq b \geq a
\end{xalignat*}
%
obravnavamo le eno. Seveda pisanje dokazov, pri katerih večji del
dokaza opustimo, zahteva pazljivost in nekaj izkušenj.

\begin{vaja}
  Dokaži izrek~\ref{izrek:abc-vsota-razlik-soda} tako, da obravnavaš
  samo možnost $b \geq c \geq a$ in zraven dopišeš ">brez izgube za
  splošnost"<.
\end{vaja}


%%%%%%%%%%%%%%%%%%%%%%%%%%%%%%%%%%%%%%%%%%%%%%%%%%%%%%%%%%%%%%%%%%%%%%
\section{Definicije}
\label{sec:definicije}


Poznamo tri vrste definicij. Prva in najpreprostejša je definicija, ki
služi kot \textbf{okrajšava} za daljši izraz. To smemo vedno nadomestiti
s prvotnim izrazom. Na primer, funkcija ">hiperbolični tangens"<
$\tanh(x)$ je definirana kot
%
\begin{equation*}
  \tanh(x) = \frac{e^{2 x} - 1}{e^{2 x} + 1}.
\end{equation*}
%
Lahko bi shajali tudi brez zapisa $\tanh(x)$, vendar bi morali potem
povsod pisati daljši izraz $\frac{e^{2 x} - 1}{e^{2 x} + 1}$, kar bi
bilo nepregledno.

Druga vrsta definicije je vpeljava novega matematičnega pojma.
Študenti prvega letnika matematike spoznajo celo vrsto novih pojmov
(grupa, vektorski prostor, limita, stekališče, metrika itn.), s
katerimi si razširijo sposobnost matematičnega razmišljanja.
Matematiki cenijo dobre definicije in vpeljavo novih matematičnih
pojmov vsaj toliko, kot dokaze težkih izrekov.

Tretja vrsta definicije je \textbf{konstrukcija} matematičnega objekta s
pomočjo dokaza o enoličnem obstoju. O tem bomo povedali več v
razdelku~\ref{sec:enolicni-obstoj}.

\section{Pravila sklepanja in dokazi}
\label{sec:pravila-sklepanja-in-dokazi}


Povedali smo že, da je dokaz utemeljitev neke matematične izjave. V
razdelku~\ref{sec:kaj-je-dokaz} smo govorili o tem, da so dokazi
mešanica besedila in simbolov, ki jih matematiki uporabljajo tako za
utemeljitev matematičnih izjav, kakor tudi za razlago in podajanje
matematičnih idej. V tem razdelku se posvetimo \textbf{formalnim
  dokazom}, ki so logične konstrukcije namenjene mehanskemu
preverjanju pravilnosti izjav.

Za vsako logično operacijo bomo podali \textbf{formalna pravila
  sklepanja}, ki jih smemo uporabljati v formalnem dokazu. Pravilo
sklepanja shematsko zapišemo
%
\begin{equation*}
  \inferrule{\phi \\ \psi \\ \rho}{\sigma}
\end{equation*}
%
in ga preberemo: ">Če smo dokazali $\phi$, $\psi$ in $\rho$, smemo
sklepati $\sigma$."< Izjavam nad črto pravimo \textbf{hipoteze}, izjavi
pod črto pa \textbf{sklep}. Hipotez je lahko nič ali več, sklep mora
biti natanko en. Pravilo sklepanja brez hipotez se imenuje
\textbf{aksiom}.

Da bomo lahko pojasnili, kaj je dokaz, podajmo pravila sklepanja za
$\top$ in $\land$, ki jih bomo v naslednjem razdelku še enkrat bolj
pozorno obravnavali:
%
\begin{mathpar}
  \inferrule{\quad}{\top}
  %
  \and
  %
  \inferrule
  {\phi \\ \psi}
  {\phi \land \psi}
  %
  \and
  %
  \inferrule
  {\phi \land \psi}
  {\phi}
  %
  \and
  %
  \inferrule
  {\phi \land \psi}
  {\psi}  
\end{mathpar}
%
Po vrsti beremo:
%
\begin{itemize}
\item Velja $\top$.
\item Če velja $\phi$ in $\psi$, smemo sklepati $\phi \land \psi$.
\item Če velja $\phi \land \psi$, smemo sklepati $\phi$.
\item Če velja $\phi \land \psi$, smemo sklepati $\psi$.
\end{itemize}
%
Formalni dokaz ima drevesno obliko in prikazuje, kako iz danih
\textbf{hipotez} dokažemo neko \textbf{sodbo}. Pri dnu je zapisana izjava,
ki jo dokazujemo, nad njo pa dokaz. Vsako vejišče je eno od pravil
sklepanja. Vsaka veja se mora zaključiti z aksiomom ali s hipotezo.
Oglejmo si dokaz izjave $(\alpha \land \alpha) \land (\top
\land \beta)$ iz hipoteze $\beta \land \alpha$:
%
\begin{equation*}
  \inferrule{
    \inferrule{
      \inferrule{\beta \land \alpha}{\alpha}
      \\
      \inferrule{\beta \land \alpha}{\alpha}}
      {\alpha \land \alpha}
    \\
    \inferrule{
      \inferrule{ }{\top}
      \\
      \inferrule{\beta \land \alpha}{\beta}
    }{\top \land \beta}
  }{(\alpha \land \alpha) \land (\top \land \beta)}
\end{equation*}
%
Dokaz se razveji na dve veji, vsaka od njiju pa še na dve veji. Tako
pri vrhu dobimo štiri liste, od katerih se trije izjava $\beta \land
\alpha$ in en aksiom za $\top$.

\begin{vaja}
  Preveri, da je vsako vejišče v zgornjem dokazu res uporaba enega od
  zgoraj podanih pravil sklepanja.
\end{vaja}

V praksi matematično besedilo bolj ali manj odraža strukturo
formalnega dokaza, le da se besedilo ne veji, ampak so sestavni kosi
dokaza zloženi v zaporedje. Formalni dokazi so uporabni, kadar želimo
preveriti veljavnost najbolj osnovnih logičnih dejstev. Ni mišljeno,
da bi matematiki pisali ali preverjali velike formalne dokaze
pomembnih matematičnih izrekov, to je delo za račualnike. Formalna
pravila sklepanja in formalni dokazi so za matematike pomembni, ker
nam omogočajo, da natančno in v celoti povemo, kakšna so ">pravila
igre"< v matematiki.


\section{Izjavni račun}
\label{sec:izjavni-racun}

Izjavni račun je tisti del logike, ki govori o logičnih konstantah
$\bot$, $\top$ in o logičnih operacijah $\land$, $\lor$, $\lthen$,
$\liff$, $\lnot$. Za vsako od njih podamo formalna pravila sklepanja,
ki so dveh vrst. Pravila \textbf{vpeljave} povedo, kako se izjave
dokaže, pravila \textbf{uporabe} pa povedo, kako se že dokazane izjave uporabi.

\subsection{Konjunkcija}
\label{sec:konjunkcija}

Konjunkcija ima eno pravilo vpeljave in dve pravili uporabe:
%
\begin{mathpar}
  \inferrule
  {\phi \\ \psi}
  {\phi \land \psi}
  \and
  \inferrule
  {\phi \land \psi}
  {\phi}  
  %
  \and
  %
  \inferrule
  {\phi \land \psi}
  {\psi}
\end{mathpar}
%
Pravilo vpeljave pove, da konjunkcijo $\phi \land \psi$ dokažemo
tako, da dokažemo posebej $\phi$ in posebej $\psi$. Pravili uporabe pa
povesta, da lahko $\phi \land \psi$ ">razstavimo"< na izjavi~$\phi$
in~$\psi$.

V matematičnem besedilu je dokaz konjunkcije $\phi \land \psi$ zapisan
kot zaporedje dveh pod-dokazov:
%
\begin{quote}
  \it 
  %
  Dokazujemo $\phi \land \psi$:
  \begin{enumerate}
  \item (Dokaz $\phi$)
  \item (Dokaz $\psi$)
  \end{enumerate}
  Dokazali smo $\phi \land \psi$.
\end{quote}
%
Manj podroben dokaz ne vsebuje začetnega in končnega stavka, ampak
samo dokaza za $\phi$ in $\psi$. Bralec mora sam ugotoviti, da je s
tem dokazana izjava $\phi \land \psi$.

\subsection{Implikacija}
\label{sec:implikacija}

Preden zapišemo pravila sklepanja za implikacijo, si oglejmo primer
neformalnega dokaza.

\begin{izrek}
  Če je $x > 2$, potem je $x^3 + x + 7 > 16$.
\end{izrek}

\begin{dokaz}
  Predpostavimo, da velja $x > 2$. Tedaj je $x^3 > 2^3 = 8$, zato
  velja
  %
  \begin{equation*}
    x^3 + x + 7 > 8 + 2 + 7 = 17 > 16.
  \end{equation*}
  %
  Dokazali smo $x > 2 \lthen x^3 + x + 7 > 16$.
\end{dokaz}

\noindent
%
Prvi stavek dokaza z besedico ">predpostavimo"< uvede \textbf{začasno
  hipotezo} $x > 2$, iz katere nato izpeljemo posledico $x^3 + x + 7 >
16$. Implikacijo $\phi \lthen \psi$ torej dokažemo tako, da začasno
predpostavimo $\phi$ in dokažemo $\psi$. Tako pravilo vpeljave
zapišemo
%
\begin{equation*}
  \inferrule{\infer*{\psi}{[\phi]}}{\phi \lthen \psi}  
\end{equation*}
%
Zapis $[\phi]$ z oglatimi oklepaji pomeni, da $\phi$ ni prava, ampak
samo začasna hipoteza. Zapis
%
\begin{equation*}
  \infer*{\psi}{[\phi]}
\end{equation*}
%
pomeni ">dokaz izjave $\phi$ s pomočjo začasne hipoteze $\phi$."<

Pravilo uporabe za implikacijo se imenuje \textbf{modus ponens} in se
glasi
%
\begin{mathpar}
  \inferrule{\phi \lthen \psi \\ \phi}{\psi}
\end{mathpar}
%
V matematičnem besedilu se modus ponens pojavi kot uporaba že prej
dokazanega izreka izreka oblike $\phi \lthen \psi$.

\subsection{Disjunkcija}
\label{sec:disjunkcija}

Disjunkcija ima dve pravili vpeljave in eno pravilo uporabe:
%
\begin{mathpar}
  \inferrule
  {\phi}
  {\phi \lor \psi}
  \and
  \inferrule
  {\psi}
  {\phi \lor \psi}
  \and
  \inferrule
  {\phi \lor \psi \\ \infer*{\rho}{[\phi]} \\ \infer*{\rho}{[\psi]}}
  {\rho}
\end{mathpar}
%
Pravili sklepanja povesta, da lahko dokažemo disjunkcijo $\phi \lor
\psi$ tako, da dokažemo enega od disjunktov.

Pojasnimo še pravilo uporabe. Denimo, da bi radi dokazali $\rho$, pri
čemer že vemo, da velja $\phi \lor \psi$. Pravilo uporabe pravi, da je
treba obravnavati dva primera: iz začasne hipoteze $\phi$ je treba
dokazati $\rho$ in iz začasne hipoteze $\psi$ je treba dokazati
$\rho$.

Ponazorimo pravilo uporabe v dokazu izjave $(\alpha \lor \gamma) \land
(\beta \lor \gamma)$ iz hipoteze $(\alpha \land \beta) \lor \gamma$.
Dokazno drevo je precej veliko, v njem pa se dvakrat pojavi uporaba
disjunkcije:
%
\begin{equation*}
  \inferrule
  {\inferrule*
    {\inferrule*{}{(\alpha \land \beta) \lor \gamma}
      \\
      \inferrule*{
        \inferrule
        {[\alpha \land \beta]}
        {\alpha}
      }{\alpha \lor \gamma}
      \\
      \inferrule*{[\gamma]}{\alpha \lor \gamma}
    }
    {\alpha \lor \gamma}
    \\
    \inferrule*
    {\inferrule*{}{(\alpha \land \beta) \lor \gamma}
      \\
      \inferrule*{
        \inferrule
        {[\alpha \land \beta]}
        {\beta}
      }{\beta \lor \gamma}
      \\
      \inferrule*{[\gamma]}{\beta \lor \gamma}
    }
    {\beta \lor \gamma}
  }
  {(\alpha \lor \gamma) \land (\beta \lor \gamma)}
\end{equation*}
%
Poglejmo na primer levo vejo tega dokaza, desna je podobna:
%
\begin{equation*}
  \inferrule*
    {\inferrule*{}{(\alpha \land \beta) \lor \gamma}
      \\
      \inferrule*{
        \inferrule
        {[\alpha \land \beta]}
        {\alpha}
      }{\alpha \lor \gamma}
      \\
      \inferrule*{[\gamma]}{\alpha \lor \gamma}
    }
    {\alpha \lor \gamma}
\end{equation*}
%
Res je to uporaba disjunkcije $\phi \lor \psi$, kjer smo vzeli $\phi =
\alpha \land \beta$ in $\psi = \gamma$, dokazali pa smo izjavo $\rho =
\alpha \lor \gamma$.

\begin{vaja}
  Iz hipoteze $(\alpha \lor \gamma) \land (\beta \lor \gamma)$ dokaži
  $(\alpha \land \beta) \lor \gamma$.
\end{vaja}

V besedilu dokažemo disjunkcijo s pravilom za vpeljavo takole:
%
\begin{quote}
  \it
  %
  Dokazujemo $\phi \lor \psi$. Zadostuje dokazati $\phi$:
  \begin{enumerate}
  \item[] (Dokaz $\phi$.)
  \end{enumerate}
  %
  Dokazali smo $\phi \lor \psi$.
\end{quote}
%
Pravilo uporabe disjunkcije se v besedilu zapiše kot obravnava
primerov:
%
\begin{quote}
  \it
  %
  Dokazujemo $\rho$. To bomo dokazali z obravnavo primerov $\phi$ in
  $\psi$:
  \begin{enumerate}
  \item (Dokaz $\phi \lor \rho$)
  \item Predpostavimo, da velja $\phi$. (Dokaz $\rho$)
  \item Predpostavimo, da velja $\psi$. (Dokaz $\rho$)
  \end{enumerate}
  %
  Dokazali smo $\rho$.
\end{quote}
%
Še primer konkretnega dokaza, ki je tako napisan.

\begin{izrek}
  \label{izrek:x-3-5}
  Naj bo $x$ realno število. Če je $|x - 3| > 5$, potem je $x^4 > 15$.
\end{izrek}

\begin{dokaz}
  Dokazujemo $|x - 3| > 5 \lthen x^4 > 15$. Predostavimo $|x - 3| > 5$
  in dokažimo $x^4 > 15$. To bomo dokazali z obravavo primerov $x \leq
  3$ in $x \geq 3$:
  %
  \begin{enumerate}
  \item $x \leq 3 \lor x \geq 3$ velja, ker so realna števila linearno
    urejena z relacijo $\leq$.
  \item Predpostavimo $x \leq 3$. Tedaj je $x - 3 \leq 0$ in zato $|x
    - 3| = 3 - x$, od koder sledi $3 - x = |x - 3| > 5$, oziroma $x <
    -2$. Tako dobimo
    %
    \begin{equation*}
      x^4 > (-2)^4 = 16 > 15.
    \end{equation*}
  \item Predpostavimo $x \geq 3$. Tedaj je $x - 3 \geq 0$ in zato$|x -
    3| = x - 3$, od koder sledi $x - 3 = |x - 3| > 5$, oziroma $x >
    8$. Tako dobimo
    %
    \begin{equation*}
      x^4 > 8^4 = 4096 > 15.
    \end{equation*}
  \end{enumerate}
  %
  Iz predpostavke $|x - 3| > 5$ smo izpeljali $x^4 > 15$. S tem smo
  dokazali $|x - 3| > 5 \lthen x^4 > 15$.
\end{dokaz}

Težji del tega dokaza se skriva v izbiri disjunkcije. Kako je pisec
uganil, da je treba obravnavati primera $x \leq 3$ in $x \geq 3$?
Zakaj ni raje obravnaval $x < 3$ in $x \geq 3$, ali morda $x \leq 17$
in $x \geq 17$? Odgovor se skriva v definiciji absolutne vrednosti:
%
\begin{equation*}
  |a| =
  \begin{cases}
    a & \text{če je $a \geq 0$,}\\
    -a & \text{če je $a \leq 0$.}
  \end{cases}
\end{equation*}
%
Ker v izreku nastopa izraz $|x - 3|$, bo obravnava primerov $x - 3
\geq 0$ in $x - 3 \leq 0$ omogočila, da $|x - 3|$ poenostavimo enkrat
v $x - 3$ in drugič v $3 - x$. Seveda pa je $x - 3 \geq 0$
ekvivalentno $x \geq 3$ in $x - 3 \leq 0$ ekvivalentno $x \leq 3$.

\begin{vaja}
  Ali bi lahko izrek~\ref{izrek:x-3-5} dokazali tudi z obravnavo
  primerov $x < 3$ in $x \geq 3$?
\end{vaja}

\subsection{Resnica in neresnica}
\label{sec:resnica-neresnica}

Logična konstanta $\top$ označuje resnico. Kar je res, je res, in tega
ni treba posebej dokazovati. To dejstvo izraža aksiom
%
\begin{equation*}
  \inferrule{\qquad}{\top}
\end{equation*}
%
Logična konstanta $\top$ nima pravila uporabe, ker iz $\top$ ne moremo
sklepati nič koristnega.

Logična konstanta $\bot$ označuje neresnico. Ker se tega, kar ni res,
ne more dokazati, $\bot$ nima pravila vpeljave. Pravilo uporabe je
%
\begin{equation*}
  \inferrule{\quad\bot\quad}{\phi}
\end{equation*}
%
se imenuje \textbf{ex falso (sequitur) quodlibet}, kar pomeni ">iz
neresnice sledi karkoli"<.

V matematičnem besedilu se $\top$ in $\bot$ ne pojavljata pogosto, ker
matematiki izraze, v katerih se $\top$ in $\bot$ pojavita, vedno
poenostavijo s pomočjo ekvivalenc:
%
\begin{mathpar}
  \top \land \phi \liff \phi
  \and
  \top \lor \phi \liff \phi
  \and
  \bot \land \phi \liff \bot
  \and
  \bot \lor \phi \liff \phi
  \\
  (\top \lthen \phi) \liff \phi
  \and
  (\bot \lthen \phi) \liff \top
  \and
  (\phi \lthen \top) \liff \top
\end{mathpar}
%

\subsection{Ekvivalenca}
\label{sec:ekvivalenca}

Logična ekvivalenca $\phi \liff \psi$ je okrajšava za
%
\begin{equation*}
  (\phi \lthen \psi) \land (\psi \lthen \phi).
\end{equation*}
%
Ker je to konjunkcija (dveh implikacij), so pravila za vpeljavo in
uporabo ekvivalence samo poseben primer pravil sklepanja za
konjunkcijo:
%
\begin{mathpar}
  \inferrule
  {\phi \lthen \psi \\ \psi \lthen \phi}
  {\phi \liff \psi}
  \and
  \inferrule{\phi \liff \psi}{\phi \lthen \psi}
  \and
  \inferrule{\phi \liff \psi}{\psi \lthen \phi}
\end{mathpar}
%
V matematičnem besedilu ekvivalenco dokažemo takole:
%
\begin{quote}
  \it
  %
  Dokazujemo $\phi \liff \psi$:
  %
  \begin{enumerate}
  \item (Dokaz $\phi \lthen \psi$)
  \item (Dokaz $\psi \lthen \phi$)
  \end{enumerate}
  Dokazali smo $\phi \liff \psi$.
\end{quote}

Če sta izjavi $\phi$ in $\psi$ logično ekvivalentni, lahko eno
zamenjamo z drugo. To matematiki s pridom uporabljajo pri dokazovanju
izjav, čeprav pogosto sploh ne omenijo, katero ekvivalenco so
uporabili.

Kadar dokazujemo medsebojno ekvivalenco večih izjav $\phi_1$,
$\phi_2$, \ldots, $\phi_n$, zadostuje dokazati cikel implikacij
%
\begin{equation*}
  \phi_1 \lthen \phi_2 \lthen \cdots \lthen \phi_{n-1} \lthen \phi_n \lthen \phi_1.
\end{equation*}
%
(Ne spreglejte zadnje implikacije $\phi_n \lthen \phi_1$, ki zaključi
cikel). V besedilu to dokažemo:

\begin{quote}
  \it
  %
  Dokazujemo, da so izjave $\phi_1, \phi_2, \ldots, \phi_n$
  ekvivalentne:
  %
  \begin{enumerate}
  \item (Dokaz $\phi_1 \lthen \phi_2$)
  \item (Dokaz $\phi_2 \lthen \phi_3$)
  \item \dots
  \item (Dokaz $\phi_{n-1} \lthen \phi_n$)
  \item (Dokaz $\phi_n \lthen \phi_1$)
  \end{enumerate}
\end{quote}

\noindent
%
Seveda smemo pred samim dokazovanjem izjave $\phi_1, \ldots, \phi_n$
preurediti tako, da je zahtevane implikacije kar najlažje dokazati.
Dokaz lahko tudi razdelimo na dva ločena cikla implikacij
%
\begin{equation*}
  \phi_1 \lthen \cdots \lthen \phi_k \lthen \phi_1
\end{equation*}
%
in
%
\begin{equation*}
  \phi_{k+1} \lthen \cdots \lthen \phi_n \lthen \phi_{k+1}
\end{equation*}
%
in nato dokažemo še eno ekvivalenco $\phi_i \liff \phi_j$, pri čemer
je $\phi_i$ iz prvega in $\phi_j$ iz drugega cikla.

\subsection{Negacija}
\label{sec:negacija}


Negacija $\lnot\phi$ je definirana kot okrajšava za $\phi
\lthen \bot$. Iz pravil sklepanja za $\lthen$ in $\bot$ tako izpeljemo
pravili sklepanja za negacijo:
%
\begin{mathpar}
  \inferrule
  {\infer*{\bot}{[\phi]}}
  {\lnot \phi}
  %
  \and
  %
  \inferrule
  {\lnot\phi \\ \phi}
  {\psi}
\end{mathpar}
%
V besedilu dokazujemo $\lnot\phi$ takole:
%
\begin{quote}
  \it
  %
  Dokazujemo $\lnot\phi$.
  \begin{itemize}
  \item[] Predpostavimo $\phi$.
  \item[] (Dokaz $\bot$.)
  \end{itemize}
  Dokazali smo $\lnot\phi$.
\end{quote}
%
Tu ">Dokaz $\bot$"< pomeni, da iz danih predpostavk izpeljemo
protislovje. Mnogi matematiki menijo, da se takemu dokazu reče ">dokaz
s protislovjem"<, vendar to ni res. To je samo navaden dokaz negacije.
Dokazovanje s protislovjem bomo obravnavali v razdelku~\ref{sec:lem}.

Pravilo uporabe za $\lnot\phi$ v besedilu ni eksplicitno vidno, ampak
ga matematiki uporabijo, ko sredi dokaza, da velja $\psi$, izpeljejo
protislovje:
%
\begin{quote}
  \it
  %
  Dokazujemo $\psi$.
  %
  \begin{itemize}
  \item[] (Dokaz $\phi$.)
  \item[] (Dokaz $\lnot\phi$.)
  \end{itemize}
  %
  To je nesmisel, in ker iz nesmisla sledi karkoli, sledi $\psi$.
\end{quote}

\subsection{Aksiom o izključenem tretjem}
\label{sec:lem}

Aksiom o izključenem tretjem se glasi
%
\begin{equation*}
  \inferrule{ }{\phi \lor \lnot \phi}
\end{equation*}
%
Povedano z besedami, vsaka izjava je bodisi resnična bodisi
neresnična. Torej ni ">tretje možnosti"< za resničnostno vrednost
izjave $\phi$, od koder izhaja tudi ime aksioma.

Aksiom o izključenem tretjem omogoča \emph{posredne} dokaze izjav, od
katerih je najbolj znano \textbf{dokazovanje s protislovjem}: pri tem ne
utemeljimo izjave $\phi$, ampak utemeljimo, zakaj $\lnot\phi$
\emph{ne} velja. Natančneje povedano, izjavo $\phi$ zamenjamo z njej
ekvivalentno izjavo $\lnot\lnot\phi$ in dokažemo $\lnot\lnot\phi$.
Dokaz ekvivalence $\phi \liff \lnot\lnot\phi$ sestoji iz dokazov dveh
implikacij:
%
\begin{mathpar}
  \inferrule
  {\inferrule{
      \inferrule{[\lnot\phi] \\ [\phi]}{\bot}
    }
    {\lnot\lnot\phi}
  }
  {\phi \lthen \lnot\lnot\phi}
  %
  \and
  %
  \inferrule*
  {\inferrule*
    {\inferrule*{ }{\phi \lor \lnot\phi} \\
     [\phi] \\
     \inferrule*{
       \inferrule*{
         [\lnot\lnot\phi] \\ [\lnot\phi]
       }
       {\bot}
     }
     {\phi}
    }
    {\phi}
  }
  {\lnot\lnot\phi \lthen \phi}
\end{mathpar}
%
V dokazu $\lnot\lnot\phi \lthen \phi$ smo uporabili aksiom o
izključenem tretjem. V matematičnem besedilu se dokaz s protislovjem
glasi:
%
\begin{quote}
  \it
  %
  Dokažimo $\phi$ s protislovjem.
  %
  \begin{itemize}
  \item[] Predpostavimo, da bi veljalo $\lnot\phi$.
  \item[] (Dokaz neresnice $\bot$.)
  \end{itemize}
  %
  Ker torej $\lnot\phi$ pripelje do protislovja, velja $\phi$.
\end{quote}
%
Praviloma izvemo o vsebini matematične izjave~$\phi$ več, če jo
dokažemo neposredno. Dokazovanja s protislovjem zato ni smiselno
uporabljati vsepovprek, ampak le takrat, ko je zares potreben ali ko
nam zelo olajša dokazovanje.

Ostali načini za sestavljanje posrednih dokazov slonijo na
ekvivalencah
%
\begin{mathpar}
  (\phi \lor \psi) \liff \lnot (\lnot\phi \land \lnot\psi),\and
  (\phi \lor \psi) \liff (\lnot\phi \lthen \psi),\and
  (\phi \lthen \psi) \liff (\lnot\psi \lthen \lnot\phi),\and
  (\all{x \in S} \phi) \liff \lnot \some{x \in S} \lnot \phi,\and
  (\some{x \in S} \phi) \liff \lnot \all{x \in S} \lnot \phi.
\end{mathpar}
%
V vseh petih primerih implikacija $\lthen$ iz leve na desno velja brez
uporabe aksioma o izključenem tretjem. Za dokaz implikacij
$\Leftarrow$ iz desne na levo pa potrebujemo aksiom o izključenem
tretjem.

\begin{vaja}
  Sestavi formalne dokaze za zgornjih pet ekvivalenc. Pri dokazovanju
  ekvivalenc za $\forall$ in $\exists$ si pomagaj s pravili sklepanja
  iz razdelkov~\ref{sec:univerzalni-kvantifikator}
  in~\ref{sec:eksistencni-kvantifikator}.
\end{vaja}

Povejmo, kako zgornje ekvivalence uporabimo v besedilu za posredni
dokaz izjave:
%
\begin{itemize}
\item $(\phi \lor \psi) \liff \lnot (\lnot\phi \land \lnot\psi)$
  uporabimo takole:
  %
  \begin{quote}
    \it
    %
    Dokazujemo $\phi \lor \psi$.
    %
    \begin{itemize}
    \item[] Predpostavimo, da velja $\lnot\phi$ in $\lnot\psi$.
    \item[] (Dokaz neresnice $\bot$.)
    \end{itemize}
    %
    Ker torej nista $\phi$ in $\psi$ oba neresnična, je eden od njiju
    resničen. Dokazali smo $\phi \lor \psi$.
  \end{quote}
\item $(\phi \lor \psi) \liff (\lnot\phi \lthen \psi)$ uporabimo
  takole:
  %
  \begin{quote}
    \it
    %
    Dokazujemo $\phi \lor \psi$.
    %
    \begin{itemize}
    \item[] Predpostavimo $\lnot\phi$.
    \item[] (Dokaz $\psi$.)
    \end{itemize}
    %
    Če torej ne velja $\lnot\phi$, velja $\psi$. Torej velja vsaj
    eden, zato smo dokazali $\phi \lor \psi$.
  \end{quote}
\item $(\phi \lthen \psi) \liff (\lnot\psi \lthen \lnot\phi)$
  uporabimo takole:
  %
  \begin{quote}
    \it
    %
    Dokazujemo $\phi \lthen \psi$.
    %
    \begin{enumerate}
    \item Predpostavimo $\lnot\psi$.
    \item (Dokaz $\lnot\psi$.)
    \end{enumerate}
    %
    Dokazali smo, da iz $\phi$ sledi $\psi$.
  \end{quote}
\item $(\all{x \in S} \phi) \liff \lnot \some{x \in S} \lnot \phi$
  uporabimo takole:
  %
  \begin{quote}
    \it
    %
    Dokazujemo, da za vsak $x \in S$ velja $\phi$.
    %
    \begin{enumerate}
    \item Predpostavimo, da obstaja $x \in S$, za katerega $\phi$
      \emph{ne} velja.
    \item (Dokaz neresnice $\bot$.)
    \end{enumerate}
    %
    Predpostavka, da obstaja $x \in S$, za katerega $\phi$ ne velja,
    pripelje do protislovja. Torej za vsak $x \in S$ velja $\phi$.
  \end{quote}
\item $(\some{x \in S} \phi) \liff \lnot \all{x \in S} \lnot \phi$
  uporabimo takole:
  %
  \begin{quote}
    \it
    %
    Dokazujemo, da obstaja tak $x \in S$, za katerega velja $\phi$.
    %
    \begin{enumerate}
    \item Predpostavimo, da bi veljalo $\lnot\phi$ za vse $x \in S$.
    \item (Dokaz neresnice $\bot$.)
    \end{enumerate}
    %
    Predpostavka, da velja $\lnot\phi$ za vse $x \in S$, pripelje do
    protislovja. Torej obstaja $x \in S$, za katerega velja $\phi$.
  \end{quote}
\end{itemize}

Negacijo poljubne izjave $\phi$ tvorimo preprosto tako, da pred njo
postavimo $\lnot$. Vendar nam to ne pove dosti o matematični vsebini
negirane izjave. V večini primerov je negacijo lažje razumeti, če
simbol~$\lnot$ ">porinemo"< navznoter do osnovnih izjav z uporabo
naslednjih ekvivalenc:
%
\begin{align*}
  \lnot (\phi \land \psi) &\iff \lnot\phi \lor \lnot\psi \\
  \lnot (\phi \lor \psi) &\iff \lnot\phi \land \lnot\psi \\
  \lnot (\phi \lthen \psi) &\iff \phi \land \lnot\psi \\
  \lnot (\lnot \phi) &\iff \phi \\
  \lnot (\all{x \in S} \phi) &\iff \some{x \in S} \lnot\phi \\
  \lnot (\some{x \in S} \phi) &\iff \all{x \in S} \lnot\phi
\end{align*}

\begin{zgled}
  Denimo, da bi radi ovrgli izjavo
  % 
  \begin{quote}
    ">Vsako zaporedje pozitivnih realnih števil ima limito~$0$."<
  \end{quote}
  % 
  Da izjavo ovržemo, moramo dokazati njeno negacijo. Načeloma lahko
  negacijo tvorimo tako, da pred izjavo napišemo ">ni res, da velja
  \dots"<, a nam to ne pove, kako bi negacijo dokazali. Zapišimo
  prvotno izjavo v delni simbolni obliki:
  % 
  \begin{equation}
    \label{eq:pozitivno-limita-0}
    \all{a \in \RR^\NN}{\text{$(a_n)_n$ pozitivno zaporedje}
      \lthen \text{$0$ je limita zaporedja $(a_n)_n$}}.
  \end{equation}
  % 
  Zgornja pravila za računanje negacije nam povedo, da se
  $\lnot\forall$ spremeni v $\exists\lnot$ in da se nato implikacija
  oblike $\phi \lthen \psi$ spremeni v $\phi \land \lnot\psi$. Tako
  izrazimo negacijo izjave~\eqref{eq:pozitivno-limita-0}:
  % 
  \begin{equation*}
    \some{a \in \RR^\NN}{\text{$(a_n)_n$ pozitivno zaporedje}
      \land \lnot (\text{$0$ je limita zaporedja $(a_n)_n$})}.
  \end{equation*}
  % 
  To preberemo z besedami:
  % 
  \begin{quote}
    ">Obstaja tako zaporedje $(a_n)_n$, da je $(a_n)_n$ zaporedje
    pozitivnih števil in da $0$ ni limita zaporedja $(a_n)_n$."<
  \end{quote}
  %
  Če se še malo potrudimo, preberemo bolj razumljivo:
  % 
  \begin{quote}
    ">Obstaja tako zaporedje pozitivnih realnih števil, da $0$ ni
    njegova limita."<
  \end{quote}
  %
  S tem še nismo končali, saj je tudi ">Število $0$ ni limita
  zaporedja $(a_n)_n$"< negacija. Izjavo ">$0$ je limita zaporedja
  $(a_n)_n$"< najprej zapišemo simbolno:
  % 
  \begin{equation}
    \label{eq:limita-0}
    \all{\epsilon > 0}
      \some{m}{\NN}
        \all{n \geq m}
          |a_n - 0| < \epsilon.
  \end{equation}
  % 
  Z zgornjimi pravili za negiranje izračunamo negacijo
  izjave~\eqref{eq:limita-0}. Operacijo $\lnot$ postopoma ">porivamo"<
  navznoter:
  % 
  % 
  \begin{align*}
    \lnot \all{\epsilon > 0} \some{m \in \NN} \all{n \geq m} |a_n
          - 0| < \epsilon & \iff
    \\
    \some{\epsilon > 0} \lnot \some{m}{\NN} \all{n \geq m}
          |a_n - 0| < \epsilon &\iff
    \\
    \some{\epsilon > 0} \all{m}{\NN} \lnot \all{n \geq m}
          |a_n - 0| < \epsilon &\iff
    \\
    \some{\epsilon > 0} \all{m}{\NN} \some{n \geq m}
          \lnot (|a_n - 0| < \epsilon) &\iff
    \\
    \some{\epsilon > 0} \all{m}{\NN} \some{n \geq m}
          |a_n - 0| \geq \epsilon &\iff
    \\
    \some{\epsilon > 0} \all{m}{\NN} \some{n \geq m}
          a_n \geq \epsilon.
  \end{align*}
  % 
  V zadnjem koraku smo upoštevali, da za pozitivno število $a_n$ velja
  $|a_n - 0| = |a_n| = a_n$. Tako smo dobili podrobno zapisano
  negacijo prvotne izjave
  % 
  \begin{quote}
    ">Obstaja tako zaporedje pozitivnih števil $(a_n)_n$ in obstaja
    tak $\epsilon > 0$, da za vsak $m \in \NN$ obstaja $n \geq m$, za
    katerega velja $a_n > \epsilon$."<
  \end{quote}
  % 
  To izjavo pa znamo dokazati tako, da podamo konkreten primer
  zaporedja $(a_n)_n$ in konkretno vrednost $\epsilon$, ki zadoščata
  pogoju, denimo $a_n = 2 + n$ in $\epsilon = 1$. Res, če je $m \in
  \NN$ poljuben, lahko vzamemo kar $n = m$, saj potem velja $a_n = a_m
  = 2 + m > 1 = \epsilon$.

  Pričujoči primer smo zapisali zelo podrobno. Izkušeni matematik tega
  seveda ne bo pisal, saj bo izračunal negacijo prvotne izjave kar v
  glavi in takoj podal primer zaporedja, ki dokazuje, da prvotna
  izjava ne velja.
\end{zgled}

%%%%%%%%%%%%%%%%%%%%%%%%%%%%%%%%%%%%%%%%%%%%%%%%%%%%%%%%%%%%%%%%%%%%%%
\section{Predikatni račun}
\label{sec:predikatni-racun}

Predikatni račun je tisti del logike, ki obravnava predikate ter
kvantifikatorja~$\forall$ in~$\exists$.

Predikate tvorimo z logičnimi operacijami in kvantifikatorji iz
\textbf{osnovnih predikatov}. Katere osnovne predikate imamo na voljo,
je odvisno od snovi, ki jo obravnavamo.\footnote{Na primer, če
  obravnavamo ravninsko geometrijo, potem so osnovni predikati ">točka
  $x$ leži na premici $y$"<, ">premici $p$ in $q$ se sekata"< itn.}
Vedno imamo na voljo tudi \textbf{enakost} $x = y$, ki jo bomo
obravnavali v razdelku~\ref{sec:enakost}.

V osnovnih predikatih nastopajo \textbf{izrazi} ali \textbf{termi}. Katere
izraze lahko tvorimo je spet odvisno od tega, katere konstante in
operacije imamo na voljo. Na primer, če obravnavamo aritmetiko celih
števil, so na voljo operacije $+$, $-$, $\times$, če pa obravnavamo
realna števila, so na voljo operacije $+$, $-$, $\times$, $/$. V
izrazih vedno lahko nastopajo \textbf{spremenljivke}. Kadar uporabimo
spremenljivko, moramo povedati njen \textbf{tip} oziroma \textbf{množico}
vrednosti, ki jih lahko zavzame spremenljivka. Pogosto je tip
spremenljivke razviden iz spremnega besedila ali iz ustaljene uporabe:
$n$ se uporablja za naravno število, $x$ za realno, $f$ za funkcijo
ipd.

Ponazorimo pravkar definirane pojme s primerom. Predikat
%
\begin{equation*}
  0 < f(x) \land f(x) < \pi/4 \lthen \sin(2 f(x)) = 1/3
\end{equation*}
%
je sestavljen s pomočjo logičnih operacij $\land$ in $\lthen$ iz treh
osnovnih predikatov, zgrajenih iz osnovnih relacij $<$ in $=$,
%
\begin{mathpar}
  0 < f(x)
  \and
  f(x) < \pi/4
  \and
  \sin(2 f(x)) = 1/3,
\end{mathpar}
%
v katerih nastopa pet izrazov:
%
\begin{mathpar}
  0
  \and
  f(x)
  \and
  \pi/4
  \and
  \sin(2 f(x))
  \and
  1/3
\end{mathpar}
%
V teh izrazih nastopa spremenljivka $x$, katere tip je množica
realnih števil (to moramo uganiti) in spremenljivka $f$, ki označuje
funkcijo iz realnih v realna števila (tudi to moramo uganiti).
Nadalje, v izrazih nastopajo konstante $0$, $\pi$, $4$, $2$,
$1$ in $3$, operacija $\sin$ in operacija množenja.


%%%%%%%%%%%%%%%%%%%%%%%%%%%%%%%%%%%%%%%%%%%%%%%%%%
\subsection{Proste in vezane spremenljivke}
\label{sec:spremenljivke}

V predikatih in izrazih se pojavljajo spremenljivke. Pri tem moramo
ločiti med \textbf{prostimi} in \textbf{vezanimi} spremenljivkami. Oglejmo
si naslednja izraza in predikat:
%
\begin{equation*}
  \sum_{i=0}^{n} a_i,
  \qquad
  \int_0^1 f(t) \, d t,
  \qquad
  \forall x \in A .\, \phi(x) \;.
\end{equation*}
%
V vsoti je vezana spremenljivka $i$, spremenljivki $n$ in $a$ sta
prosti. To pomeni, da je $i$ neke vrste ">lokalna
spremenljivka"<,\footnote{Podobnost z lokalnimi spremenljivkami v
  programskih jezikih ni zgolj naključje. Lokalna spremenljivka in
  števec v zanki sta tudi primera vezanih spremenljivk v teoriji
  programskih jezikov.} katere veljavnost je samo znotraj vsote,
medtem ko sta spremenljiki $n$ in $a$ veljavni tudi zunaj samega
izraza. Podobno je v integralu $t$ vezana spremenljivka in $f$ prosta,
v izjavi na desni pa je vezana spremenljivka $x$, spremenljivki $A$ in
$\phi$ sta prosti.

Vezane spremenljivke so ">nevidne"< zunaj izraza in jih lahko vedno
preimenujemo, ne da bi spremenili pomen izraza (seveda se novo ime ne
sme mešati z ostalimi spremenljivkami, ki nastopajo v izrazu): izraza
$\int_0^1 f(t)\, d t$ in $\int_0^1 f(x)\, d x$ štejemo za
\emph{enaka}, ker se razlikujeta le v imenu vezane spremenljivke.
Spremenljivki, ki ni vezana, pravimo \textbf{prosta}. Izrazu, v katerem
ni prostih spremenljivk, pravimo \textbf{zaprt izraz}. Zaprta
logična izjava se imenuje \textbf{stavek}.

Pomembno se je zavedati, da vezana spremenljivka ">zunaj"< svojega
območja ne obstaja. Matematiki so glede tega precej površni in na
primer pišejo
%
\begin{equation*}
  \int x^2 \, d x = x^3/3 + C,
\end{equation*}
%
kar je strogo gledano nesmisel. Na levi strani v integralu stoji
vezana spremenljivka~$x$, ki je na desni ">pobegnila"< iz integrala.
Še več, če je $x \in \RR$ in $C \in \RR$, potem je izraz $x^3/3 + C$
\emph{število} (odvisno od vrednosti $x$ in $C$), saj je vsota dveh
realnih števil. Na desni strani bi morala stati oznaka za
\emph{funkcijo}, recimo
%
\begin{equation*}
  \int x^2 \, d x = (x \mapsto x^3/3 + C),
\end{equation*}
%
vendar tega v praksi nihče ne piše. Seveda pri vsem tem ostane še
vprašanje, kakšno vlogo ima v zgornjem izrazu~$C$. Pri analizi se
učimo, da je~$C$ ">poljubna konstanta"<. Poskusimo to razumeti
natančno s stališča logike. Besedico ">poljubno"< ponavadi razumemo
kot ">za vsak"<, vendar to ne gre, saj je
%
\begin{equation*}
  \all{C \in \RR} \int x^2 \, d x = (x \mapsto x^3/3 + C)
\end{equation*}
%
nesemisel. Če bi to bilo res, bi veljalo za $C = 1$ in za $C = 2$, od
koder bi dobili
%
\begin{equation*}
  (x \mapsto x^3/3 + 1) =
  \int x^2 \, d x =
  (x \mapsto x^3/3 + 2).
\end{equation*}
%
Potemtakem bi morali biti funkciji $(x \mapsto x^3/3 + 1)$ in $(x
\mapsto x^3/3 + 1)$ enaki, od koder sledi nesmisel $1 = 2$. Težave
nastopajo iz dejstva, da poskušamo nedoločeni integral razumeti kot
operacijo, ki slika funkcije v funkcije, kar ni. Nedoločeni integral
preslika funkcijo~$f$ v \emph{množico} vseh funkcij $F$, za katere
velja $F' = f$. Če bi to želeli zapisati zares pravilno, bi dobili
%
\begin{equation*}
  \int x^2 \, d x =
  \set{(x \mapsto x^3/3 + C) \such C \in \RR}.
\end{equation*}
%
Ali naj torej sklepamo, da so matematiki pravzaprav zelo površni pri
pisanju integralov? Da, s stališča formalne logike prav gotovo. Vendar
to ni nujno slabo: matematični zapis v praksi služi ljudem za
sporazumevanje in prav je, da si izberejo tak zapis, s katerim najbolj
učinkovito komunicirajo drug z drugim. Kljub temu pa se je treba
zavedati, kdaj gredo matematiki ">po bližnjici"< in ne zapišejo ali
povedo vsega dovolj natančno, da bi to bilo sprejemljivo za standarde,
ki jih postavlja formalna logika.


%%%%%%%%%%%%%%%%%%%%%%%%%%%%%%%%%%%%%%%%%%%%%%%%%%
\subsection{Substitucija}
\label{sec:substitucija}

\textbf{Substitucija} je osnovna sintaktična operacija, v kateri
\emph{proste} spremenljivke zamenjamo z izrazi. Zapis
%
\begin{equation*}
  \xsubst{e}{x_1 \subto e_1, \ldots, x_n \subto e_n}
\end{equation*}
%
pomeni: ">v izrazu $e$ \emph{hkrati} zamenjaj proste spremenljivke
$x_1$ z $e_1$, $x_2$ z $e_2$, \dots in $x_n$ z $e_n$."<  Na primer,
%
\begin{equation*}
  \subst{x^2 + y}{x \subto 3, y \subto 5, z \subto 12}
\end{equation*}
%
je enako $3^2 + 5$. Nič hudega ni, če se v substituciji omenja
spremenljivko $z$, ki se v izrazu $x^2 + y$ ne pojavi.

Ko naredimo substitucijo, moramo paziti, da se proste spremenljivke ne
">ujamejo"<. Denimo, da želimo v integralu
%
\begin{equation*}
  \int_0^1 \frac{x}{a - x^2} \; dx
\end{equation*}
%
parameter $a$ zamenjati z $y^2$. To naredimo s substitucijo
%
\begin{equation*}
  \xsubst{\left(\int_0^1 \frac{x}{a - x^2} \; dx\right)}{a \subto y^2} =
  \int_0^1 \frac{x}{y^2 - x^2} \; dx.
\end{equation*}
%
Vse lepo in prav. Kaj pa, če želimo $a$ zamenjati z $1 + x$? Ker je
spremenljivka $x$ vezana v integralu, \emph{ne smemo} delati takole:
%
\begin{equation*}
  \xsubst{\left(\int_0^1 \frac{x}{a - x^2} \; dx\right)}{a \subto x^2} =
  \int_0^1 \frac{x}{x^2 - x^2} \; dx ?!
\end{equation*}
%
Ker vstavljamo v integral spremenljivko $x$, moramo vezano
spremenljivko $x$ najprej preimenovati v kaj drugega, na primer $t$,
šele nato vstavimo:
%
\begin{equation*}
  \xsubst{\left(\int_0^1 \frac{x}{a - x^2}\; dt \right)}{a \subto x^2} =
  \xsubst{\left(\int_0^1 \frac{t}{a - t^2} \; dt\right)}{a \subto x^2} =
  \int_0^1 \frac{t}{x^2 - t^2} \; dt.
\end{equation*}
%
Podajmo še nekaj primerov substitucij:
%
\begin{align*}
  \subst{x + y + 1}{x \subto 2} &= 2 + y + 1 \;,
  \\
  \subst{x + y^2 + 1}{x \subto y, y \subto x} &= y + x^2 + 1 \;
  \\
  \subst{\subst{x + y^2 + 1}{x \subto y}}{y \subto x} &=
  x + x^2 + 1 \;,
  \\
  \textstyle
  \subst{x + \int_0^1 x \cdot y \;, d x}{x \subto 2}
  &= \textstyle  2 + \int_0^1 x \cdot y \;, d x \;,
  \\
  \textstyle
  \subst{\int_0^1 x \cdot y \; d x}{y \subto x^2}
  &= \textstyle \int_0^1 t \cdot x^2 \; d t \;.
\end{align*}
%
Ločiti je treba med hkratno in zaporedno substitucijo:
%
\begin{align*}
  \subst{x + y^2}{x \subto y, y \subto x} &= y + x^2
  \\
  \subst{\subst{x + y^2}{x \subto y}}{y \subto x} &=
  \subst{y + y^2}{y \subto x} = x + x^2
  \\
  \subst{\subst{x + y^2}{y \subto x}}{x \subto y} &=
  \subst{x + x^2}{x \subto y} = y + y^2.
\end{align*}
%

V nadaljevanju bomo obravnavali pravila sklepanja za univerzalne in
eksistenčne kvantifkatorje, v katerih se pojavi substitucija. Ker je
sam zapis za substitucijo nekoliko nepregleden, bomo uporabili
nekoliko manj pravilen, a bolj praktičen zapis. Denimo, da imamo
logično formulo $\phi$, v kateri se morda pojavi spremenljivka $x$, ni
pa to nujno. Tedaj pišemo $\phi(x)$. Če želimo zamenjati $x$ z izrazom
$e$, zapišemo $\phi(e)$. To je pravzaprav običajni zapis, kot ga
uporabljajo matematiki za zapis funkcij, mi pa smo ga uporabili za
zapis logičnih formul. Če bi uporabili zapis s substitucijo, bi
formulo označili samo s $\phi$ namesto s $\phi(x)$ in zamenjavo s
$\xsubst{\phi}{x \subto e}$ namesto s $\phi(e)$. Zakaj je ta bolj
priročen zapis hkrati manj pravilen? V formalni logiki strogo ločimo
med \emph{simbolnim zapisom} matematičnega pojma, ki je zaporedje
znakov na papirju, in njegovim \emph{pomenom}, ki je matematična
abstrakcija. Substitucija $\xsubst{\phi}{x \subto e}$ nam pove, kako
niz znakov $\phi$ predelamo v novi niz znakov, torej deluje na novoju
simbolnega zapisa. Ko pišemo $\phi(x)$ pa si že predstavljamo, da je
$\phi$ matematična funkcija, ki deluje na argumentu $x$. S tem nastopi
zmešnjava med simbolnim zapisom in pomenom. Dokler se zmešnjave
zavedamo, je vse v redu.

\subsection{Univerzalni kvantifikator}
\label{sec:univerzalni-kvantifikator}

Univerzalna kvantifikacija $\all{x \in S} \phi$ se prebere ">Za vse $x$
iz $S$ velja $\phi$."< Pravili sklepanja sta
%
\begin{mathpar}
  \inferrule
  {\infer*{\phi(x)}{[x \in S]}}{\all{x \in S} \phi(x)} \ \text{($x$ svež)}
  \and
  \inferrule{\all{x \in S} \phi(x) \\ e \in S}{\phi(e)}
\end{mathpar}
%
pri čemer je $x$ spremenljivka, $\phi(x)$ logična formula in $e$ poljuben izraz.

V besedilu dokažemo se pravilo vpeljave zapiše:
%
\begin{quote}
  \it
  %
  Dokazujemo $\all{x \in S} \phi(x)$:
  %
  \begin{itemize}
  \item[] Naj bo $x \in S$ poljuben.
  \item[] (Dokaz, da velja $\phi(x)$).
  \end{itemize}
  %
  Dokazali smo $\all{x \in S} \phi(x)$.
\end{quote}
%
Pravilo uporabe v besedilu ponavadi ni eksplicitno navedeno, če pa bi
ga že zapisali, bi šlo takole:
%
\begin{quote}
  \it
  %
  Dokazujemo, da velja $\phi(e)$:
  \begin{itemize}
  \item[] (Dokaz, da velja $\all{x \in S} \phi(x)$.)
  \item[] (Dokaz, da velja $e \in S$.)
  \end{itemize}
  %
  Torej velja $\phi(e)$.
\end{quote}


Ob pravilu vpeljave stoji stranski pogoj, da mora biti spremenljivka
$x$ ">sveža"<. To pomeni, da se $x$ ne sme pojavljati drugje v dokazu,
saj bi sicer lahko prišlo do zmešnjave med vezanimi in prostimi
spremenljivkami. V besedilu se dejstvo, da je $x$ svež izraža z
besedico ">poljuben"< ali ">katerikoli"<. Primer, kako gredo stvari
narobe, če ne pazimo in pomešamo spremenljivke:

\begin{izrek}[z napako v dokazu]
  Če je $x$ večji od~$42$, so vsa realna števila večja od~$23$.
\end{izrek}

\begin{dokaz}
  Denimo, da bi nekoliko nerodno zapisali izrek simbolno takole:
  %
  \begin{equation*}
    x > 42 \lthen \all{x \in \RR} x > 23.
  \end{equation*}
  %
  To je sicer dovoljeno, saj se prosti $x$, ki stoji zunaj $\forall$
  ni ujel, ni pa preveč smotrno, ker smo na dobri poti, da bomo
  zunanji prosti $x$ in vezanega znotraj $\forall$ pomešali. Res, če
  ne upoštevamo pravila, da mora biti $x$ svež, dobimo tale nepravi
  ">dokaz"<:
  %
  \begin{equation*}
    \inferrule*
    {
      \inferrule*
      {\inferrule*
        {[x > 42] \\ 42 > 23}
        {x > 23}
      }
      {\all{x \in \RR} x > 23}
    }
    {x > 42 \lthen \all{x \in \RR} x > 23}
  \end{equation*}
  %
  Pri pravilu za vpeljavo $\forall$ smo uporabili spremenljivko $x$,
  ki pa je že nastopala v začasni hipotezi $x > 42$. Z besedilom bi se
  isti dokaz glasil takole:
  %
  \begin{quote}
    ">Dokazujemo $x > 42 \lthen \all{x \in \RR} x > 23$. Predpostavimo,
    da velja $x > 42$ in dokažimo $\all{x \in \RR} x > 23$. Naj bo $x
    \in \RR$. Po predpostavki je $x > 42$ in ker je $42 > 23$, od tod
    sledi $x > 3$."<
  \end{quote}
  %
  Če bi izrek zapisali bolje kot $x > 42 \lthen \all{y \in \RR} y >
    23$, težav ne bi bilo, saj bi se prejšnji dokaz ">zataknil"<:
  %
  \begin{quote}
    ">Dokazujemo $x > 42 \lthen \all{y \in \RR} y > 23$. Predpostavimo,
    da velja $x > 42$ in dokažimo $\all{y \in \RR} y > 23$. Naj bo $y
    \in \RR$. (Kaj zdaj? Lahko sicer dokažemo $x > 23$, a zares bi
    morali dokazati $y > 23$, kar ne gre.)"<
  \end{quote}
\end{dokaz}

Pogoj, da mora biti spremenljivka $x$ v pravilu za vpeljavo ">sveža"<,
se v praksi kaže v tem, da pri uvajanju nove spremenljivke izberemo
zanjo novo ime, ki se še ni pojavilo v dokazu.


\subsection{Eksistenčni kvantifikator}
\label{sec:eksistencni-kvantifikator}

Eksistenčna kvantifikacija $\some{x \in S} \phi$ se prebere ">obstaja
$x$ iz $S$, za katerega velja $\phi$"< ali ">za neki $x$ iz $S$ velja
$\phi$."< Pravili sklepanja za eksistenčni kvantifikator se glasita
%
\begin{mathpar}
  \inferrule
  {\phi(e) \\ e \in S}
  {\some{x \in S} \phi(x)}
  \and
  \inferrule
  {\some{x \in S} \phi(x)
    \\
    \infer*{\psi}{[x \in S \land \phi(x)]}}
  {\psi}\ \text{($x$ svež)}
\end{mathpar}
%
kjer je $e$ poljuben izraz in $x$ spremenljivka. Pri tem mora biti $x$
v pravilu uporabe svež. V besedilu pravilo vpeljave uporabimo takole:
%
\begin{quote}
  \it
  %
  Dokazujemo $\some{x \in S} \phi(x)$:
  %
  \begin{enumerate}
  \item (Skonstruiramo element $e \in S$.)
  \item (Dokažemo, da velja $\phi(e)$.)
  \end{enumerate}
  %
  Dokazali smo $\some{x \in S} \phi(x)$.
\end{quote}
%
Pravilo uporabe pa se v besedilu izraža takole:
%
\begin{quote}
  \it
  %
  Dokazujemo $\psi$:
  %
  \begin{enumerate}
  \item (Dokaz izjave $\some{x \in S} \phi(x)$.)
  \item Predpostavimo, da za $x \in S$ velja $\phi(x)$:
    %
    \begin{itemize}
    \item[] (Dokaz izjave $\psi$.)
    \end{itemize}
  \end{enumerate}
  %
  Dokazali smo $\psi$.
\end{quote}

\subsubsection{Enolični obstoj}
\label{sec:enolicni-obstoj}

Poleg običajnega eksistenčnega kvantifikatorja $\exists$ poznamo tudi
\emph{enolični} eksistenčni kvantifikator $\exists!$. Izjavo
$\exactlyone{x}{S}{\phi}$ preberemo ">obstaja natanko en $x \in S$, za
katerega velja $\phi(x)$"<.

Enolični eksistenčni kvantifikator ni osnovni logični operator, ampak
je $\exactlyone{x}{S}{\phi}$ le okrajšava za
%
\begin{equation}
  \label{eq:uniqe-exists}
  \some{x \in S} \phi(x) \land (\all{y \in S} \phi(y) \lthen x = y).
\end{equation}
%
Z besedami preberemo to izjavo takole: ">obstaja $x$ iz $S$, za
katerega velja $\phi(x)$ in za vsak $y \in S$ za katerega velja
$\phi(y)$ sledi $x = y$"<. To je samo zapleten način, kako povedati,
da obstaja natanko en element množice~$S$, ki zadošča pogoju $\phi$.

Pravilo sklepanja za vpeljavo enoličnega obstoja izpeljemo
iz~\eqref{eq:uniqe-exists}:
%
\begin{equation*}
  \inferrule{
    e \in S
    \\
    \phi(e)
    \\
    \infer*{y = e}{y \in S \land \phi(y)}
  }
  {\exactlyone{x}{S}{\phi}}
\end{equation*}
%
V besedilu dokažemo enolični obstoj takole:
%
\begin{quote}
  \it
  %
  Dokazujemo, da obstaja natanko en $x \in S$, za katerega velja
  $\phi(x)$:
  %
  \begin{enumerate}
  \item Obstoj: (Konstrukcija elementa $e \in S$ in dokaz, da velja $\phi(x)$.)
  \item Enoličnost: denimo da za $y \in S$ velja $\phi(y)$:
    %
    \begin{itemize}
    \item[] (Dokaz, da je $e = y$).
    \end{itemize}
  \end{enumerate}
  %
  Dokazali smo $\exactlyone{x \in S} \phi(x)$.
\end{quote}

Če dokažemo enolični obstoj $\exactlyone{x \in S} \phi(x)$, lahko
vpeljemo novo konstanto $c$, ki označuje tisti element iz $S$, ki
zadošča pogoju~$\phi$, pri čemer moramo seveda paziti, da znaka $c$
nismo že prej uporabili za kak drug pomen. Nova konstanta~$c$ je
opredeljena s praviloma
%
\begin{mathpar}
  \inferrule{ }{\phi(c)}
  \and
  \inferrule{
    y \in S
    \\
    \phi(y)
  }
  {y = c}
\end{mathpar}
%
Če v formuli $\phi$ poleg spremenljivke $x$ nastopajo še druge proste
spremenljivke, denimo $y_1, \ldots, y_n$, potem je nova konstanta~$c$
v resnici \emph{funkcija} parametrov $y_1, \ldots, y_n$.

\subsection{Enakost in reševanje enačb}
\label{sec:enakost}

Enakost $=$ je osnovna relacija, ki zadošča naslednjim aksiomom in
pravilom sklepanja:
%
\begin{mathpar}
  \inferrule{ }{a = a}
  \and
  \inferrule{a = b}{b = a}
  \and
  \inferrule{a = b \\ b = c}{a = c}
  \and
  \inferrule{\phi(a) \\ a = b}{\phi(b)}
\end{mathpar}
%
Po vrsti so so pravilo \emph{refleksivnosti}, \emph{simetrije},
\emph{tranzitivnosti} in \emph{zamenjave}. Zaenkrat enakosti ne bomo
posvečali posebne pozornosti, saj jo v praksi študenti dobro
obvladajo.

V osnovni iz srednji šoli se učimo pravil za reševanje enačb: enačbi
smemo na obeh straneh prišteti ali odšteti poljuben izraz, pomnožiti
ali deliti smemo s poljubnim \emph{neničelnim} izrazom, ipd. Od kod
izhajajo ta pravila? Kaj sploh pomeni, da smo enačbo ">rešili"<? Ko
rešimo kvadratno enačbo
%
\begin{equation*}
  x^2 - 5 x + 6 = 0
\end{equation*}
%
običajno zapišemo rešitev takole:
%
\begin{equation*}
  x_1 = 2, \quad x_2 = 3.
\end{equation*}
%
Kako naj to razumemo iz stališča matematične logike? Treba je
pojasniti dvoje: kaj pomenita $x_1$ in $x_2$, saj v prvotni enačbi
nastopa spremenljivka $x$, ter kako naj razumemo vejico med izjavama
$x_1 = 2$ in $x_2 = 3$. Z indeksoma $1$ in $2$ štejemo rešitve enačbe
in sta v resnici nepotrebna,\footnote{Kako pa bi zapisali rešitve
  enačbe $x_1^2 - 5 x_1 + 6 x = 0$?} na kar kaže tudi dejstvo, da
pišemo $x = \ldots$ in ne $x_1 = \ldots$, kadar je rešitev ena sama.
Torej bi lahko rešitev zapisali kot
%
\begin{equation*}
  x = 2, \quad x = 3.
\end{equation*}
%
Sedaj pa je tudi jasno, da bi namesto vejice morala stati disjunkcija,
se pravi
%
\begin{equation*}
  x = 2 \lor x = 3.
\end{equation*}
%
Začetna enačba in tako zapisana rešitev sta logično ekvivalentni:
%
\begin{equation*}
  x^2 - 5 x + 6 = 0 \iff
  x = 2 \lor x = 3.
\end{equation*}
%
Povzemimo: reševanje enačbe je postopek, s katerim dano enačbo $f(x) =
g(x)$ prevedemo v njen \emph{logično ekvivalentno} obliko $x = a_1
\lor x = a_2 \lor \cdots \lor x = a_n$, iz katere so neposredno razvidne
rešitve enačbe.

Pravila za reševanje enačb torej niso nič drugega kot recepti, s
pomočjo katerih enačbo predelamo v njen \emph{ekvivalentno} obliko, ki
je korak bližje končni obliki, v kateri bi radi zapisali rešitev. To
pojasnjuje srednješolska pravila za reševanje enačb. Na primer, za
realna števila $a, b, c \in \RR$ vedno velja
%
\begin{equation*}
  a = b \lthen c \cdot a = c \cdot b,
\end{equation*}
%
medtem ko obratna implikacija
%
\begin{equation*}
  c \cdot a = c \cdot b \lthen a = b
\end{equation*}
%
za splošne $a$ in $b$ velja le v primeru, ko je $c \neq 0$. Ker pri
reševanju enačb potrebujemo implikacijo v obe smeri, srednješolce
učimo, da smejo enačbo množiti samo z od nič različnimi števili.

\begin{vaja}
  Kako bi srednješolcem pojasnil, od kod izvira pravilo za množenje
  enačbe z neničelnim številom?
\end{vaja}

\begin{vaja}
  Enačbo $f(x) = g(x)$ smo ">rešili"< z zaporedjem korakov
  %
  \begin{align*}
    f(x) = g(x) &\liff \\
    f_1(x) = g_1(x) &\liff \\
    \vdots & \\
    f_k(x) = g_k(x) &\lthen \\
    f_{k+1}(x) = g_{k+1}(x) &\liff \\
    \vdots & \\
    x = a_1 \lor \cdots \lor x = a_n
  \end{align*}
  %
  kjer smo v $k$-tem koraku namesto ekvivalence pomotoma naredili
  implikacijo. Smo s tem dobili preveč ali premalo rešitev prvotne
  enačbe?
\end{vaja}



%%% Local Variables: 
%%% mode: latex
%%% TeX-master: "lmn"
%%% End: 


\chapter{Boolova algebra}

\section{Resničnostne tabele}

Vsaka izjava ima \textbf{resničnostno vrednost}. Resničnostni vrednosti sta $\bot$
(resnica) in $\top$ (neresnica). Na primer, izjava $\bot \lor (\top \lthen \top)$ je resnična, njena resničnostna vrednost je $\top$. Izjava $2 + 2 = 5$ je neresnična, njena resničnostna vrednost je~$\bot$.

Kadar izjava vsebuje spremenljivke (pravimo jim tudi \emph{parametri}), je njena
resničnostna vrednost \emph{odvisna} od parametrov. Na primer, če sta $x, y \in \NN$ spremenljivki, je resničnostna vrednost izjave $x + 2 y < 3$ odvisna
od $x$ in $y$, kar lahko prikažemo z \textbf{resničnostno tabelo}:
%
\begin{center}
  \begin{tabular}{ccc}
    \toprule
    $x$ & $y$ & $x + 2 y < 3$ \\ \midrule
    $0$ & $0$ & $\top$ \\
    $0$ & $1$ & $\top$ \\
    $1$ & $0$ & $\top$ \\
    $2$ & $0$ & $\top$ \\
    $1$ & $1$ & $\bot$ \\
    $0$ & $2$ & $\bot$ \\
    $\vdots$ & $\vdots$ & $\vdots$ \\
    \bottomrule
  \end{tabular}
\end{center}
% 
Kot vidimo, je lahko resničnostna tabela neskončna. Bolj uporabne so končne resničnostne tabele, v katerih parametri zavzemajo vrednosti iz končne množice.

V izjavi lahko nastopajo tudi \textbf{izjavne spremenljivke} ali \textbf{izjavni simboli}, to se spremenljivke, ki zavzamejo vrednosti $\bot$ in $\top$.
Na primer, naj bo $\two = \set{\bot, \top}$ in $p, q \in \two$. Tedaj je $\neg p \lor q$ izjava, katere resničnostna tabela je
%
\begin{center}
  \begin{tabular}{ccc}
    \toprule
    $p$ & $q$ & $\neg p \lor q$ \\ \midrule
    $\bot$ & $\bot$ & $\top$ \\
    $\bot$ & $\top$ & $\top$ \\
    $\top$ & $\bot$ & $\bot$ \\
    $\top$ & $\top$ & $\top$ \\
    \bottomrule
  \end{tabular}
\end{center}

Izjava $\phi(p_1, \ldots, p_n)$, v kateri nastopajo izjavne spremenljivke $p_1, \ldots, p_n$ (in nobeni drugi parametri) določa preslikavo
%
\begin{equation*}
  \two \times \cdots \times \two \to \two
\end{equation*}
%
s predpisom
%
\begin{equation*}
    (p_1, \ldots, p_n) \mapsto \phi(p_1, \ldots, p_n)
\end{equation*}
%
Preslikavi, ki slika iz produkta $\two \times \cdots \times \two$ v $\two$ pravimo \textbf{Boolova preslikava}. Prikažemo jo lahko z resničnostno tabelo. Če ima preslikava~$n$ argumentov, ima tabela $2^n$ vrstic.


\subsection{Tavtologije}

Izjava je \textbf{tavtologija}, če je njena resničnostna vrednost $\top$ ne glede na
vrednosti parametrov. Premisli: kako iz resničnostne tabele razberemo, ali je
izjava tavtologija?

\begin{izrek}
  Naj bo $\phi$ izjava, v kateri nastopajo le izjavni simboli
  $p_1,\ldots,p_n$. Tedaj je $\phi$ tavtologija, če in samo če ima dokaz.
\end{izrek}

\begin{proof}
  Dokaz najdete v \cite{prijatelj92:_osnov}.
\end{proof}

\noindent
%
Izrek je pomemben, ker nam pove, da lahko dokazovanje izjav nadomestimo s preverjanjem resničnostnih tabel.

\begin{opomba}
  Izrek velja samo za izjave, ki jih sestavimo iz izjavnih simbolov, $\bot$, $\top$ in
  logičnih veznikov $\neg$, $\land$, $\lor$, $\lthen$, $\liff$. Za splošne izjave, ki vsebujejo tudi $\forall$ in $\exists$ izrek \emph{ne} velja. Lahko se namreč zgodi, da ima izjava neskončno resničnostni tabelo, v kateri so vse resničnostne vrednosti~$\top$, a izjava nima dokaza.
\end{opomba}

\subsection{Polni nabori}

Vsaka formula v izjavnem računu ima resničnostno tabelo. Ali lahko vsako tabelo
dobimo kot resničnostno tabelo neke formule? Na primer, ali obstaja formula,
katere resničnostna tabela se glasi
%
\begin{center}
  \begin{tabular}{ccc}
    \toprule
    $p$ & $q$ & ? \\ \midrule
    $\bot$ & $\bot$ & $\bot$ \\
    $\bot$ & $\top$ & $\top$ \\
    $\top$ & $\bot$ & $\top$ \\
    $\bot$ & $\bot$ & $\bot$ \\
    \bottomrule
  \end{tabular}
\end{center}
%
Odgovor je pritrdilen. Podajmo dva načina, kako tako izjavo izračunamo iz tabele.

\subsubsection{Disjunktivna oblika}
\label{sec:disjunktivna-oblika}

Za vsako vrstico v tabeli, ki ima vrednost $\top$ zapišemo konjunkcijo simbolov in
njihovih negacij, pri čemer negiramo tiste simbole, ki imajo v dani vrstici vrednost
$\bot$. Na primer, v zgornji tabeli imata druga in tretja vrstica vrednost $\top$, zanju
zapišemo konjunkciji:
%
\begin{itemize}
\item 2.~vrstica: $\neg p \land q$,
\item 3.~vrstica: $p \land \neg q$.
\end{itemize}
%
Nato tvorimo disjunkcijo tako dobljenih konjunkcij:
%
\begin{equation*}
  (\neg p \land q) \land (p \land \neg q).
\end{equation*}
%
Dobljena formula ima želeno resničnostno tabelo.

\subsubsection{Konjuktivna oblika}
\label{sec:konjuktivna-oblika}

Za vsako vrstico v tabeli, ki ima vrednost $\bot$ zapišemo
disjunkcijo simbolov in njihovih negacij, pri čemer negiramo tiste simbole, ki
imajo v dani vrstici vrednost $\top$. Na primer, v zgornji tabeli imata prva in
četrta vrstica vednost $\bot$, zanju zapišemo disjunkciji:
%
\begin{itemize}
\item 1.~vrstica: $p \lor q$
\item 4.~vrstica: $\neg p \lor \neg q$
\end{itemize}
%
Nato tvorimo konjunkcijo tako dobljenih disjunkcij:
%
\begin{equation*}
  (p \lor q) \land (\neg p \lor \neg q).
\end{equation*}
%
Zgornjo tabelo bi lahko dobili tudi kot resničnostno tabelo formule
%
\begin{equation*}
    p \liff q
\end{equation*}

\subsection{Polni nabori}
\label{sec:polni-nabori}

Vidimo, da lahko vsako resničnostno tabelo dobimo z uporabo veznikov $\neg$, $\lor$ in
$\land$. \textbf{Polni nabor} je tak izbor veznikov, s katerim lahko dobimo vsako
resničnostno tabelo.

Torej je $\neg$, $\lor$, $\land$ poln nabor. Lahko bi ga še zmanjšali na $\neg$, $\land$, saj lahko $p \lor q$ izrazimo kot $\neg p \land \neg q$.

Nabor $\land$, $\lor$ pa \emph{ni} poln, saj ne moremo dobiti resničnostne tabele
%
\begin{center}
  \begin{tabular}{cc}
    \toprule
    $p$ & ? \\ \midrule
    $\bot$ & $\top$ \\
    $\top$ & $\bot$ \\
    \bottomrule
  \end{tabular}
\end{center}
%
Res, če iz $p$, $\land$ in $\lor$ sestavimo poljubno formulo $\phi(p)$, na primer $(p \land (p \lor p)) \land p$, bo ta ekvivalentna~$p$ in bo zato veljalo $\phi(\top) = \top$, zgornja tabela pa zahteva $\phi(\top) = \bot$.


\section{Boolova algebra}

Ekvivalentni izjavi imata enake resničnostne vrednosti, torej lahko ekvivalenco
$\liff$ obravnavamo kar kot enakost, saj to tudi je, kar se tiče resničnostnih
vrednosti. Zato lahko namesto $p \liff q$ pišemo tudi $p = q$, če imamo v mislih le
resničnostne vrednosti.

\begin{opomba}
  Ekvivalentni izjavi imata lahko različen \emph{pomena}. Na primer,
  $\all{x, y \in R} x + y = y + x$ in
  $\all{\alpha \in R} \sin(2 \alpha) = 2 \cdot \cos \alpha \cdot \sin \alpha$ sta
  ekvivalentni, saj sta obe resnični, a ne moremo reči, da je njun pomen enak. (Predstavljate si, da bi bi vas v srednji šoli profesorica matematike vprašala adicijski izrek za $\sin$, vi pa bi odgovorili ">vrstni red seštevanja realnih števil ne vpliva na vrednost vsote"<.)
\end{opomba}


Za logične veznike veljajo \emph{algebrajska pravila}, se pravi enačbe, kakršne poznamo v algebri. Ta pravila lahko uporabljamo kot računska pravila, s katerimi lahko izjavo poenostavimo v ekvivalentno obliko. Pogosto je tako računanje bolj prikladno kot dokazovanje. Spodaj našteta pravila lahko preverimo tako, da zapišemo resničnostne tabele izjav in jih primerjamo.

Pravilom, ki veljajo za logične veznike, pravimo \textbf{Boolova algebra}.
Razdelimo jih po sklopih.

Pravila za konjunkcijo:
%
\begin{align}
  (p \land q) \land r &= p \land (q \land r) \tag{asociativnost $\land$} \\
  p \land q &= q \land p \tag{komutativnost $\land$} \\
  p \land p &= p \tag{idempotentnost $\land$} \\
  \top \land p &= p \tag{$\top$ je nevtralen za $\land$} \\
  \bot \land p &= \bot \tag{$\bot$ absorbira $\land$}
\end{align}
%
Pravila za disjunkcijo:
%
\begin{align}
  (p \lor q) \lor r &= p \lor (q \lor r) \tag{asociativnost $\lor$} \\
  p \lor q &= q \lor p \tag{komutativnost $\lor$} \\
  p \lor p &= p \tag{idempotentnost $\lor$} \\
  \bot \lor p &= p \tag{$\bot$ je nevtralen za $\lor$} \\
  \top \lor p &= \top \tag{$\top$ absorbira $\lor$}
\end{align}
%
Pravila za implikacijo:
%
\begin{align}
  (p \lthen q) &= (\neg q \lthen \neg p) \tag{kontrapozitivna oblika $\lthen$}\\
  (p \lthen q) &= \neg p \lor q \notag \\
  (\bot \lthen q) &= \top \notag \\
  (\top \lthen q) &= q \notag \\
  (p \lthen \bot) &= \neg p \notag \\
  (p \lthen \top) &= \top \notag
\end{align}
%
Kombinirana pravila:
%
\begin{align}
  \neg(p \land q) &= \neg p \lor \neg q \tag{de Morganovo pravilo za $\land$} \\
  \neg(p \lor q) &= \neg p \land \neg q \tag{de Morganovo pravilo za $\lor$} \\
  \neg(p \lthen q) &= \neg p \land q \notag \\
  p \land (p \lor q) &= p \tag{absorbcijsko pravilo za $\land$}\\
  p \lor (p \land q) &= p \tag{absorbcijsko pravilo za $\lor$} \\
  p \land (q \lor r) &= (p \land q) \lor (p \land r) \tag{distributivnost $\land$}\\
  p \lor (q \land r) &= (p \lor q) \land (p \lor r) \tag{distributivnost $\lor$}
\end{align}
%
Pravila za negacijo:
%
\begin{align}
  \neg \top  &= \bot \notag \\
  \neg \bot &= \top \notag \\
  \neg\neg p &= p \tag{negacija je involucija} \\
  p \lor \neg p &= \top \tag{izključena tretja možnost} \\
  p \land \neg p &= \bot \notag
\end{align}

Zapišimo še uporabna logična pravila za kvantifikatorje. Tokrat uporabimo $\liff$
namesto $=$, ker je to bolj običajno:
%
\begin{align*}
  (\all{x \in \emptyset} \phi(x))   &\iff   \top \\
  (\some{x \in \emptyset} \phi(x))   &\iff   \bot \\
  (\all{x \in \set{a}} \phi(x))   &\iff   \phi(a) \\
  (\some{x \in \set{a}} \phi(x))   &\iff   \phi(a) \\
  (\neg \all{x \in A} \phi(x))   &\iff   \some{x \in A} \neg \phi(x) \\
  (\neg \some{x \in A} \phi(x))   &\iff   \all{x \in A} \neg \phi(x) \\
  (\psi \lthen \all{x \in A} \phi(x))   &\iff   \all{x \in A} \psi \lthen \phi(x) \\
  (\psi \lor \all{x \in A} \phi(x))   &\iff   \all{x \in A} \psi \lor \phi(x) \\
  (\psi \land \some{x \in A} \phi(x))   &\iff   \some{x \in A} \psi \land \phi(x) \\
  (\all{u \in A \times B} \phi(u))   &\iff   \all{x \in A} \all{y \in B} \phi(x, y) \\
  (\some{u \in A \times B} \phi(u))   &\iff   \some{x \in A} \some{y \in B} \phi(x, y) \\
  (\all{u \in A + B} \phi(u))   &\iff   (\all{x \in A} \phi(\inl(x))) \land (\all{y \in B} \phi(\inr(y))) \\
  (\all{u \in A \cup B} \phi(u))   &\iff   (\all{x \in A} \phi(x)) \land (\all{y \in B} \phi(y)) \\
  (\some{u \in A + B} \phi(u))   &\iff   (\some{x \in A} \phi(\inl(x))) \lor (\some{y \in B} \phi(\inr(y))) \\
  (\some{u \in A \cup B} \phi(u))   &\iff   (\some{x \in A} \phi(x)) \lor (\some{y \in B} \phi(y)) \\
  (\all{u \in \set{x \in A \such \psi(x)}} \phi(u))   &\iff   \all{x \in A} \psi(x) \lthen \phi(x) \\
  (\some{u \in \set{x \in A \such \psi(x)}} \phi(u))   &\iff   \some{x \in A} \psi(x) \land \phi(x)
\end{align*}
%
Te ekvivalence je treba preveriti tako, da jih dokažemo.


\chapter{Podmnožice in potenčne množice}

\subsection{Definicija relacije $\subseteq$}

Pravimo, da je množica $S$ \textbf{podmnožica} množice $T$, pišemo $S \subseteq T$, ko velja $\all{x \in S} x \in T$. Pravimo tudi, da je $S$ \textbf{vsebovana} v $T$ in da je $T$ \textbf{nadmnožica}~$S$.

Vedno velja $\emptyset \subseteq S$ in $S \subseteq S$.

Princip ekstenzionalnosti za množice pravi:
%
\begin{equation*}
  S = T \iff (\all{x \in S} S \in T) \land (\all{y \in T} y \in S)
\end{equation*}
%
kar lahko zapišemo s podmnožicami:
%
\begin{equation*}
  S = T \iff S \subseteq T \land T \subseteq S.
\end{equation*}
%
Vsaka podmnožica $S \subseteq A$ opredeljuje neko lastnost elementov iz $A$: tisti
elementi, ki imajo opredeljeno lastnost, so v $S$, ostali pa ne.

\begin{zgled}
  Naj bo $P$ množica vseh praštevil, torej je $P \subseteq N$. Podmnožica $P$
  opredeljuje lastnost ">je praštevilo"<.
\end{zgled}


\subsection{Kako tvorimo podmnožice}

Če je $\phi(x)$ logična formula, v kateri nastopa spremenljivka $x \in A$, lahko tvorimo množico
%
\begin{equation*}
    \set{ x \in A \such \phi(x) }.
\end{equation*}
%
Pri tem je $x$ vezana spremenljivka. Za to množico velja:
%
\begin{equation*}
    a \in \set{ x \in A \such \phi(x) } \iff a \in A \land \phi(a).
\end{equation*}
%
Povedano z besedami: elementi množice $\set{ x \in A \such \phi(x) }$ so tisti elementi iz $A$, ki zadoščajo pogoju $\phi$.
%
Velja $\set{ x \in A \such \phi(x) } \subseteq A$, prav tako pa
\begin{equation*}
  \set{x \in A \mid \phi(x)} \subseteq \set{x \in A \mid \psi(x)} \iff
  \all{x \in A} \phi(x) \lthen \psi(x).
\end{equation*}


\subsection{Kanonična inkluzija}

Za podmnožico $S \subseteq T$ definiramo \textbf{kanonično inkluzijo} ali \textbf{kanonično vključitev} $i_S : S \to T$, s predpisom $i_S : x \mapsto x$. Pozor, to ni identiteta, razen v primeru $S = T$!.
Oznaka $i_S$ ni standardna, pravzaprav standardne oznake ni.

Če je $f : T \to U$ in $S \subseteq T$, pravimo kompozitumu $f \circ i_S$ \textbf{zožitev} preslikave $f$ na $S$, pišemo $\restrict{f}{S}$.


\section{Potenčna množica}

\subsection{Definicija potenčne množice}

Za vsako množico $A$ tvorimo množico $\pow{A}$, ki ji pravimo \textbf{potenčna množica}.
Elementi potenčne množice $\pow{A}$ so natanko podmnožice množice $A$:
%
\begin{equation*}
    S \in \pow{A} \iff S \subseteq A
\end{equation*}
%
Na primer $\pow{\emptyset} = \set{\emptyset}$ in
%
\begin{equation*}
  P(\set{a,b,c}) = \set{\set{}, \set{a}, \set{b}, \set{c}, \set{a,b}, \set{a,c}, \set{b,c}, \set{a,b,c}}.
\end{equation*}


\subsection{Karakteristične funkcije}

\textbf{Karakteristična funkcija} na množici $A$ je funkcija z domeno $A$ in kodomeno $\two$. Tu je $\two = \set{\bot, \top}$ množica resničnostnih vrednosti.

Eksponentna množica $\two^A$ je torej množica vseh karakterističnih funkcij na $A$.

\begin{opomba}
  Karakteristične funkcije se uporabljajo tudi v analizi, kjer jih
  običajno razumemo kot preslikave $A \to \set{0,1}$ namesto $A \to \set{\bot, \top}$. Ker sta množici $\set{\bot,\top}$ in $\set{0,1}$ izomorfni, to ni bistvena razlika.
\end{opomba}

Karakteristično funkcijo si lahko predstavljamo kot preslikavo, ki opredeljuje
neko lastnost elementov~$A$: tisti elementi, ki imajo opredeljeno lastnost, se
slikajo v $\top$, ostali pa v $\bot$.

\begin{zgled}
  Preslikava $p : \NN \to \two$, definirana s predpisom
  %
  \begin{equation*}
    p(n) = 
    \begin{cases}
      \top & \text{če je $n$ praštevilo}, \\
      \bot & \text{če $n$ ni praštevilo}.
    \end{cases}
  \end{equation*}
  %
  je karakteristična preslikava lastnosti ">je praštevilo"<. Lahko bi jo zapisali tudi takole:
  %
  \begin{equation*}
    p(n) = (n > 1 \land \all{k, m} n = k \cdot m \lthen k = 1 \lor m = 1).
  \end{equation*}
\end{zgled}


\subsection{Izomorfizem $\pow{A} \cong 2^A$}

Videli smo, da lahko neko lastnost elementov množice $A$ predstavimo bodisi s
podmnožico bodisi s karakteristično preslikavo. To nam da idejo, da med
podmnožicami $A$ in karakterističnimi preslikavami na $A$ obstaja neka zveza.

\begin{izrek}
  $\pow{A} \cong 2^A$.
\end{izrek}

\begin{dokaz}
  Definirajmo preslikavi
  %
  \begin{align*}
    \chi &: \pow{A} \to 2^A &
    \xi &: 2^A \to \pow{A} \\
    \chi_S(x) &\defeq
      \begin{cases}
        \top & \text{če $x \in S$,} \\
        \bot & \text{če $x \not\in S$,}
      \end{cases}
    &
    \xi_f &\defeq \set{x \in A \such f(x) = \top}.
  \end{align*}
  %
  Ta predpisa bi lahko krajše zapisali tudi takole:
  %
  \begin{align*}
  \chi_S(x) &\defeq (x \in S), &
  \xi_f &\defeq \set{x \in A \such f(x) }.
  \end{align*}
  %
  Preslikavi $\chi_S$ pravimo \textbf{karakteristična funkcija podmnožice $S$}.
  %
  Trdimo, da sta $\chi$ in $\xi$ inverza:
  %
  \begin{enumerate}
  \item 
    Dokažimo $\chi \circ \xi = \id[2^A]$. Uporabimo princip ekstenzionalnosti za preslikave.
    Naj bo $f \in 2^A$. Dokažimo, da je $\chi_{\xi_f} = f$.
    Uporabimo princip ekstenzionalnosti za preslikave. Naj bo $x \in A$:
    %
    \begin{equation*}
      \chi_{\xi_f}(x) = (x \in \xi_f) = f(x).
    \end{equation*}

  \item
    Dokažimo $\xi \circ \chi = id_{\pow{A}}$. Uporabimo princip ekstenzionalnosti za preslikave. Naj bo $S \in \pow{A}$. Dokažimo, da je $\xi_{\chi_S} = S$:
    %
    \begin{equation*}
      \xi_{\chi_S} = \set{x \in A \such \chi_S(x)} = \set{x \in A \such x \in S} = S.
    \end{equation*}
  \end{enumerate}
\end{dokaz}

\subsection{Boolova algebra podmnožic}

Podmnožice množice $A$ tvorijo Boolovo algebro za operacije presek $\cap$, unija $\cup$ in relativni komplement. Nevtralni element za unijo je $\emptyset$ in nevtralni element za presek je $A$.

Definirajmo tudi operacijo \textbf{simetrična razlika $\oplus$}, ki podmnožicama $S, T \in A$ priredi podmnožico
%
\begin{equation*}
  S \oplus T \defeq (S \setminus T) \cup (T \setminus S) = (S \cup T) \setminus (S \cap T).
\end{equation*}
%
Potenčna množica $\pow{A}$ je za operacijo $\oplus$ Abelova grupa.


\chapter{Razredi in družine}

\section{Russellov paradoks}

V prejšnji lekciji smo spoznali zapis podmnožice
%
\begin{equation*}
    \set{ x \in A \such \phi(x) },
\end{equation*}
%
ki tvori podmnožico $A$ vseh elementov, ki zadoščajo pogoju $x$. Ko je bila
teorija množic še v povojih, se je sama po sebi ponujala ideja, da bi lahko
opisali množice kot kakršnokoli zbirko stvari. Torej bi lahko pisali
%
\begin{equation*}
    \set{ x \such \phi(x) }
\end{equation*}
%
za množico vseh tistih stvari (objektov, matematičnih entitet), ki zadoščajo
pogoju~$\phi$. Se pravi, da bi veljalo
%
\begin{equation*}
    a \in \set{ x \such \phi(x) } \iff \phi(a)
\end{equation*}
%
A izkaže se, da ne moremo kar tako tvoriti povsem poljubnih množic objektov. To
je odkril znameniti filozof, logik in matematik Betrand Russell. Razmislek se po
njem imenuje \textbf{Russellov paradoks}. Le-ta je v matematiko vnesel pravo ">krizo
temeljev"<, iz katere se je v prvi polovici 20.~stoletja razvila logika in
temelji matematike, kot jih poznamo danes.

Russellov paradoks gre takole. Denimo, da bi lahko tvorili poljubne množice
objektov. Tedaj bi lahko tvorili tudi množico vseh množic, ki niso element same
sebe:
%
\begin{equation*}
    R \defeq \set{ S \such S \not\in S }
\end{equation*}
%
Sedaj bomo izpeljali protislovje tako, da bomo dokazali $R \not\in R$ in $R \in R$:
%
\begin{enumerate}
\item Dokažimo $R \not\in R$.
  %
  Denimo, da bi veljalo $R \in R$. Potem po definiciji $R$ velja $R \not\in R$, kar
  je v protislovju s predpostavko $R \in R$.

\item
  Dokažimo $R \in R$. V prvem koraku smo že dokazali $R \not\in R$, torej po
  definiciji $R$ velja $R \in R$.
\end{enumerate}
%
Kaj lahko storimo? Očitno je treba pazljivo nadzorovati dopustne konstrukcije
množic.

\section{Množice in razredi}

V sodobni teoriji množic Russellov paradoks razrešimo tako, da ločimo med dvema
različnima zvrstema zbirk ali skupkov elementov, namreč \textbf{množicami} in \textbf{razredi}.

Torej imamo opravka s tremi zvrstmi matematičnih objektov:
%
\begin{enumerate}
\item Elementi, ki niso množice (na primer naravna števila), pravimo jim \textbf{urelementi}.
\item Zbirke elementov, ki se imenujejo \textbf{množice}.
\item Zbirke elementov, ki se imenujejo \textbf{razredi}.
\end{enumerate}
%
Elementi množic so urelementi in množice. Enako velja za razrede.
%
V čem je torej razlika med množicami in razredi?
%
\emph{Množica je lahko element (druge množice ali razreda).
Razred ne more biti element (druge množice ali razreda).}
%
S tem želimo povedati, da je zapis
%
\begin{equation*}
    x \in Y
  \end{equation*}
%
\emph{neveljaven}, če je~$x$ razred. Se pravi, če je $x$ razred, potem sploh ne moremo govoriti o tem, ali je $x \in Y$ resnična izjava, saj ni izjava, ker ni izraz.

Vsaka množica je hkrati razred. Ni pa vsak razred tudi množica.

Razred je množica, če ga lahko skonstruiramo še na kak drug način s pomočjo
pravil za konstrukcije množic (kartezični produkti, vsote, eksponenti, unije,
preseki, podmnožice in vse ostale konstrukcije množic, ki jih bomo še spoznali).

\textbf{Pravi razred} je tak razred, ki ni množica.
%
Z zapisom
%
\begin{equation*}
  \set{ x \such \phi(x) }
\end{equation*}
%
definiramo \textbf{razred} vseh objektov, ki zadoščajo pogoju $\phi$. Se pravi, da velja
%
\begin{equation*}
    a \in \set{ x \such \phi(x) } \iff \phi(a).
\end{equation*}
%
Poglejmo nekaj primerov.

\begin{primer}
  \textbf{Russellov razred} $R \defeq \set{ S \such S \not\in S }$ vsebuje vse množice, ki niso element same sebe. Paradoks smo razrešili, saj je nesmiselno zapisati $R \in R$.
\end{primer}


\begin{primer}
  \textbf{Razred vseh množic}
  %
  \begin{equation*}
    V \defeq \set{ S \such \text{$S$ je množica} },
  \end{equation*}
% 
  ki ga označimo tudi s $\Set$. To je pravi razred. Res, če bi bil $V$ množica,
  potem bi lahko tvorili podmnožico
  %
  \begin{equation*}
    \set{ S \in V \such S \not\in S},
  \end{equation*}
  %
  ki ni nič drugega kot Russellov $R$. Tako bi spet dobili protislovje. Torej $V$
  ni množica.
\end{primer}

\begin{primer}
  Razred vseh enojcev $E \defeq \set{ S \such \exactlyone{x \in S} \top }$ je pravi razred. Res, če bi bil množica, potem bi bila množica tudi njegova unija $\bigcup E$, ki pa je enaka~$V$.
\end{primer}

\begin{primer}
  Zbirke vseh matematičnih struktur dane vrste pogosto tvorijo prave razrede. Na primer, razred vseh grup, razred vseh kolobarjev, vseh vektorskih prostorov itd.
\end{primer}

Z razredi lahko delamo tako kot z množicami: tvorimo unije, preseke in produkte
razredov, govorimo o podrazredih. Pri tem uporabljamo enake oznake za
operacije kot pri množicah. Paziti moramo le, da razreda nikoli ne uporabimo kot
element kake množice ali razreda. Na primer, če je $C$ razred, lahko tvorimo
">potenčni razred"> $\pow{C}$, ki vsebuje vse \emph{podmnožice} $C$:
%a
\begin{equation*}
    \pow{C} \defeq \set{ S \such S \in \Set \land S \subseteq C }.
\end{equation*}
%
Ne smemo pa tvoriti $\set{ D \such D \subseteq C }$, ker bi s tem $C$ postal element razreda $\set{D \such D \subseteq C}$.

\section{Družine množic}

Pogosto imamo opravka z zbirko množic. Če je zbirka končna, lahko množice preprosto
naštejemo in vsako od njih poimenujemo
%
\begin{align*}
    A &= \cdots \\
    B &= \cdots \\
    C &= \cdots
\end{align*}
%
Če je množic neskončno, jih morda lahko oštevilčimo:
%
\begin{align*}
    A_1 &= \cdots \\
    A_2 &= \cdots \\
    A_3 &= \cdots \\
    A_4 &= \cdots \\
        &\vdots
\end{align*}
%
A tu se zadeve še ne končajo, saj lahko v splošnem obravnavamo poljubno zbirko množic.
Takim zbirkam pravimo \textbf{družine množic}. Družina množic je \textbf{indeksirana} z elementi neke množice $I$, ki ji pravimo \textbf{indeksna množica}. Za vsak $i \in I$ imamo množico $A_i$,  kar lahko izrazimo tudi z naslednjo definicijo.

\begin{definicija}
  \textbf{Družina množic} je preslikava $I \to \Set$. Množici $I$ pravimo \textbf{indeksna množica} in njenim elementov \textbf{indeksi}.
\end{definicija}


\begin{primer}
  Končno zbirko množic lahko indeksiramo s končno množico. Denimo, da imamo
  množice $A$, $B$, $C$, $D$, $E$. Iz njih lahko tvorimo družino $S : I \to \Set$:
  %
  \begin{align*}
  I &= \set{1, 2, 3, 4, 5}, \\
  S_1 &= A,\\
  S_2 &= B,\\
  S_3 &= C,\\
  S_4 &= D,\\
  S_5 &= E.
  \end{align*}
\end{primer}

\begin{primer}
  Nihče nas ne sili, da morajo biti indeksi števila. V prejšnjem primeru bi lahko
  uporabili $I = \set{42, 13, \sqrt{2}, \emptyset, \RR}$ in definirali $S : I \to \Set$
  %
  \begin{align*}
  S_{42} &= A,\\
  S_{13} &= B,\\
  S_{\sqrt{2}} &= C,\\
  S_{\emptyset} &= D,\\
  S_{\RR} &= E.
  \end{align*}
\end{primer}


\begin{primer}
  Množice v družini se lahko ponavljajo. Skrajni primer je \textbf{konstantna družina}, v kateri so vse množice med seboj enake.
\end{primer}

\begin{primer}
  \textbf{Prazna družina} je družina množic, ki je indeksirana z $\emptyset$.
\end{primer}

\begin{primer}
  Prazno družino moramo ločiti od \textbf{družine praznih množic}
  %
  \begin{align*}
    I &\to \Set \\
    i &\mapsto \emptyset
  \end{align*}
\end{primer}

\begin{primer}
  \textbf{Neprazna družina} je družina indeksirana z neprazno množico.
  \textbf{Družina nepraznih množic} je družina, v kateri so vse množice neprazne.
  Torej velja:
  %
  \begin{itemize}
  \item Prazna družina je družina nepraznih množic.
  \item Družina praznih množic je lahko prazna družina (ko je indeksna množica $\emptyset$).
  \item Družina praznih množic je lahko neprazna družina (ko je indeksna množica neprazna).
  \end{itemize}
\end{primer}


\section{Konstrukcije in operacije z družinami množic}

Operacije $\times$, $+$, $\cap$ in $\cup$ lahko posplošimo tako, da namesto z dvema
množicama delujejo na poljubnem številu množic. V ta namen uporabimo družine
množic.

\subsection{Presek in unija družine}

Presek in unija družine $A : I \to \Set$ sta definirana takole:
%
\begin{align*}
  \bigcup_{i \in I} A_i &\defeq \set{ x \such \some{i \in I} x \in A_i },
  \\
  \bigcap_{i \in I} A_i &\defeq \set{ x \such \all{ i \in I} x \in A_i }.
\end{align*}
%
Pozor! Na desni strani imamo razred! Res se lahko zgodi, da dobimo pravi razred, namreč
kot presek prazne družine:
%
\begin{align*}
  \bigcap_{i \in \emptyset} A_i
  &= \set{ x \such \all{ i \in \emptyset} x \in A_i } \\
  &= \set { x \such \top } \\
  &= V.
\end{align*}
%
Kdaj pa dobimo množico? Presek neprazne družine je vedno množica. Res, če imamo
$k \in I$, potem velja
%
\begin{equation*}
    \bigcap_{i \in I} A_i = \set{ x \in A_k \such \all{i \in I} x \in A_i}.
\end{equation*}
%
Sedaj na desni ne stoji več razred, ampak podmnožica množice~$A_k$.

Kaj pa unija družine množic? Ali je množica? Izkaže se, da za to potrebujemo aksiom:

\begin{aksiom}
  Unija družine množic je množica.
\end{aksiom}


\subsection{Kartezični produkt družine}

\begin{definicija}
  \textbf{Funkcija izbire} za družino $A : I \to \Set$ je tako prirejanje, ki vsakemu indeksu $i \in I$ priredi natanko en element $f(i) \in A_i$.
 \end{definicija}

 \begin{primer}
   Primer: funkcija izbire za družino
   %
   \begin{align*}
   A &: \NN \to \Set \\
   A_n &\defeq \set{ x \in R \such 0 < x < 2^{-n} }
   \end{align*}
   %
   je na primer $f(n) \defeq 2^{-n - 1}$. To ni edina funkcija izbire za $A$, lahko bi vzeli tudi $f(n) \defeq 2^{-n} / 3$.
 \end{primer}

 \begin{definicija}
   \textbf{Kartezični produkt} družine $A : I \to \Set$ je množica vseh funkcij izbire družine~$A$:
   %
   \begin{equation*}
     \prod_{i \in I} A_i \defeq
     \set{f : I \to \textstyle\bigcup_{i \in I} A_i \such \all{i \in I} f(i) \in A_i}.
   \end{equation*}
   %
   Za vsak $j \in I$ imamo \textbf{$j$-to projekcijo}
   %
   \begin{align*}
    \pr[j] &:  \left(\textstyle\prod_{i \in I} A_i\right) \to A_j, \\
    \pr[j] &:  f \mapsto f(j).
   \end{align*}
 \end{definicija}

Običajni kartezični produkt dveh množic je poseben primer produkta množic, namreč družine
množic, ki je indeksirana z $I = \set{1, 2}$. Natančneje, velja
%
\begin{equation*}
  A \times B \cong \textstyle\prod_{i \in \set{1, 2}} C_i,
\end{equation*}
%
kjer je $C_1 = A$ in $C_2 = B$.

Tudi eksponentna množica je poseben primer produkta množic, saj velja
%
\begin{equation*}
  B^A \cong \textstyle\prod_{a \in A} B
\end{equation*}
%
Na desni imamo produkt konstantne družine množic
\begin{align*}
  A &\to \Set, \\
  a &\mapsto B.
\end{align*}


\subsection{Koprodukt ali vsota množic}

Vsoto množic posplošimo na koprodukt družine.

\begin{definicija}
  \textbf{Koprodukt} ali \textbf{vsota družine} $A : I \to \Set$ je množica
  $\sum_{i \in I} A_i$, katere elementi so $\inj[i](a)$ za $i \in I$ in $a \in A_i$.
  Preslikavi $\inj[k] : A_k \to \sum_{i \in I} A_i$ pravimo \textbf{$k$-ta injekcija}.

  Poleg tega definiramo še \textbf{projekciji}
  %
  \begin{align*}
    \fst (\inj[i](a)) &= i, \\
    \snd (\inj[i](a)) &= a.
  \end{align*}
  %
  Namesto $\sum$ se piše tudi $\coprod$.
\end{definicija}

Poseben primer koprodukta je vsota $A + B$, saj velja
%
\begin{equation*}
  A + B \cong \textstyle\sum_{k \in \set{1, 2}} C_k
\end{equation*}
%
kjer je
\begin{align*}
  C &: \set{1, 2} \to \Set \\
  C_1 &\defeq A, \\
  C_2 &\defeq B.
\end{align*}
%
Tudi kartezični produkt $A \times B$ je poseben primer koprodukta, saj velja
%
\begin{equation*}
    A \times B \cong \textstyle\sum_{a \in A} B
\end{equation*}
%
Na desni imamo tokrat koprodukt konstantne družine množic
%
\begin{align*}
  A &\to \Set, \\
  a &\mapsto B.
\end{align*}


\chapter{Lastnosti preslikav}

Mnogi ste v srednji šoli že spoznali osnovne lastnosti preslikav, kot so injektivnost, surjektivnost in bijektivnost preslikave. V tej lekciji ponovimo te pojme in jih povežemo še s pojmoma monomorfizem in epimorfizem, ki sta pomembna v algebri

\section{Osnovne lastnosti preslikav}

\subsection{Injektivna, surjektivna, bijektivna preslikava}

\begin{definicija}
  Preslikava $f : A \to B$ je
  %
  \begin{itemize}
  \item \textbf{injektivna}, ko velja $\all{x y \in A} f(x) = f(y) \lthen x = y$,
  \item \textbf{surjektivna}, ko velja $\all{y \in B} \some{x \in A} f(x) = y$,
  \item \textbf{bijektivna}, ko je surjektivna in injektivna.
  \end{itemize}
\end{definicija}

\begin{opomba}
  Pogosto vidimo definicijo injektivnosti, ki pravi, da $f$ slika različne elemente v različne vrednosti, se pravi $\all{x y \in A} x \neq y \lthen f(x) \neq f(y)$. Ta definicija je ekvivalentna naši, a jo ne priporočamo, ker je manj uporabna.
  Naša definicija namreč podaja recept, kako preverimo injektivnost: predpostavimo $f(x) = f(y)$ in od tod izpeljemo
  $x = y$ tako, da predelamo \emph{enačbo} $f(x) = f(y)$ v enačbo $x = y$. To je v splošnem lažje kot predelava \textbf{neenačb}.
\end{opomba}

\begin{vaja}
  Primerjaj definicijo injektivnosti in surjektivnosti z zahtevo, da mora biti prirejanje, ki določa preslikavo, enolično in celovito.
\end{vaja}


\subsection{Monomorfizmi in epimorfizmi}

\begin{definicija}
  Preslikava $f : A \to B$ je
  %
  \begin{itemize}
  \item \textbf{monomorfizem (mono)}, ko jo lahko krajšamo na levi:
    \begin{equation*}
      \all{C \in \Set} \all{g, h : C \to A} f \circ g = f \circ h \lthen g = h.
    \end{equation*}

   \item \textbf{epimorfizem (epi)}, ko jo lahko krajšamo na desni:
     \begin{equation*}
      \all{C \in \Set} \all{g, h : B \to C} g \circ f = h \circ f \lthen g = h.
    \end{equation*}
  \end{itemize}
\end{definicija}

Pojma monomorfizem in epimorfizem sta uporabna, ker nam omogočata, da \emph{krajšamo} funkcije, ki nastopajo v enačbah. Na vajah boste reševali naloge, kjer to pride prav.

\begin{izrek}
  \label{izr:epi-mono-comp}
  Naj bosta $f : A \to B$ in $g : B \to C$ preslikavi. Tedaj velja:
  %
  \begin{enumerate}
  \item Kompozicija monomorfizmov je monomorfizem.
  \item Kompozicija epimorfizmov je epimorfizem.
  \item Če je $g \circ f$ monomorfizem, je $f$ monomorfizem.
  \item Če je $g \circ f$ epimorfizem, je $g$ epimorfizem.
  \end{enumerate}
\end{izrek}

\begin{dokaz}
  \begin{enumerate}
  \item Naj bosta $f : A \to B$ in $g : B \to C$ monomorfizma. Dokazujemo, da je
    $g \circ f$ tudi monomorfizem. Naj bosta $h, k : D \to A$ preslikavi, za kateri velja
    $(g \circ f) \circ h = (g \circ f) \circ k$. Dokazujemo $h = k$. Ker je kompozicija
    preslikav asociativna, velja
    $g \circ (f \circ h) = (g \circ f) \circ h = (g \circ f) \circ k = g \circ (f \circ k)$.
    Ker je $g$ monomorfizem, ga smemo krajšati na levi, torej dobimo
    $f \circ h = f \circ k$. Ker je $f$ monomorfizem, ga smemo krajšati in dobimo želeno
    enakost $h = k$.

  \item Dokaz je podoben prejšnjemu, le vloga leve in desne strani se spremeni.

  \item Dokaz je podoben naslednjemu, le vloga leve in desne strani se spremeni.

  \item Naj bosta $f : A \to B$ in $g : B \to C$ preslikavi in $g \circ f$ epimorfizem.
    Dokazujemo, da je $g$ epimorfizem. Naj bosta $h, k : C \to D$ taki preslikavi, da
    velja $h \circ g = k \circ g$. Dokazujemo, da je $h = k$. Iz $h \circ g = k \circ h$
    sledi $(h \circ g) \circ f = (k \circ g) \circ f$. Če upoštevamo asociativnost
    kompozicije, dobimo $h \circ (g \circ f) = k \circ (g \circ f)$. Ker je $g \circ f$
    epimorfizem, ga smemo krajšati na desni, od koder dobimo želeno enakost $h = k$.
  \end{enumerate}
\end{dokaz}


\begin{izrek}
  Za preslikavo $f : A \to B$ velja:
  %
  \begin{enumerate}
  \item $f$ je monomorfizem, če in samo če je $f$ injektivna.
  \item $f$ je epimorfizem, če in samo če je $f$ surjektivna.
  \item $f$ je izomorfizem, če in samo če je $f$ bijektivna.
  \end{enumerate}
\end{izrek}


\begin{dokaz}
  \begin{enumerate}
  \item ($\Rightarrow$) Če je $f$ monomorfizem in $f(x) = f(y)$, tedaj je
  $(f \circ (u \mapsto x)) \unit = f(x) = f(y) = (f \circ (u \mapsto y)) \unit$, torej
  $(u \mapsto x) = (u \mapsto y)$ in sledi $x = y$.

  ($\Leftarrow$) Če je $f$ injektivna in $f \circ g = f \circ h$, potem je za vsak $x$
  $f(g(x)) = f(h(x))$, torej $g(x) = h(x)$ za vsak $x$, torej $g = h$.

  \item
    ($\Rightarrow$) Če je $f$ epimorfizem: obravnavajmo množico
    % 
    \begin{equation*}
      S = \set{ z \in B \such \some{x \in A} f(x) = z }
    \end{equation*}
    % a
    ter preslikavi $\chi_S : B \to 2$ in $(y \mapsto \top) : B \to \two$. Ker velja
    $\chi_S \circ f = (y \mapsto \top) \circ f$, sledi $\chi_S = (y \mapsto \top)$, torej $S = B$, kar je surjektivnost.

    ($\Leftarrow$) Če je $f$ surjektivna in $g \circ f = h \circ f$: naj bo $y \in B$. Obstaja $x \in A$, da je $f(x) = y$. Torej je $g(y) = g(f(x)) = h(f(x)) = h(y)$.
    Torej je $g = h$.

  \item ($\Rightarrow$) Če je $f$ izomorfizem, potem:
    %
    je $f$ epi, ker je $\id[B] = f \circ \inv{f}$ epi;
    je $f$ mono, ker je $\id[A] = \inv{f} \circ f$ mono.

    ($\Leftarrow$) Če je $f$ bijektivna, potem je njen inverz $\inv{f}$ definiran s predpisom
    %
    \begin{equation*}
      f(y) = \descr{x \in A} f(x) = y \qquad\qquad
      \text{">tisti $x$, ki ga $f$ slika v $y$"<}
    \end{equation*}
    %
    Dokazati je treba $\exactlyone{x \in A} f(x) = y$.
    To velja, saj $\some{x \in A} f(x) = y$ sledi iz surjektivnosti $f$ in
    $\all{x_1, x_2} f(x_1) = y \land f(x_2) = y \lthen x_1 = x_2$ iz injektivnosti~$f$.
  \end{enumerate}
\end{dokaz}

\subsection{Retrakcija in prerez}

Spoznajmo še pojem retrakcije in prereza. Na predavanjih bomo s sliko pojasnili, zakaj se tako imenujeta.

\begin{definicija}
  Če sta $f : A \to B$ in $g : B \to A$ taki preslikava, da velja $f \circ g = \id[B]$, pravimo:
  %
  \begin{itemize}
  \item $f$ je \textbf{retrakcija} ali \textbf{levi inverz} $g$,
  \item $g$ je \textbf{prerez} ali \textbf{desni inverz} $f$.
  \end{itemize}
\end{definicija}

\begin{vaja}
  Podajte primer retrakcije in prereza, ki \emph{nista} izomorfizma.
\end{vaja}

\begin{izrek}
  Retrakcija je epimorfizem, prerez je monomorfizem.
\end{izrek}

\begin{dokaz}
  Denimo, da velja $f \circ g = \id$, torej je $f$ retrakcija in $g$ prerez. Ker je identiteta monomorfizem, je po izreku \ref{izr:epi-mono-comp} tudi $g$ monomorfizem. In ker je identiteta epimorfizem, je po istem izreku $f$ epimorfizem.
\end{dokaz}

\section{Slike in praslike}

\subsection{Izpeljane množice}

Naj bo $f : A \to B$ preslikava. Tedaj definiramo \textbf{izpeljano množico}
%
\begin{equation*}
  \set{ f(x) \such x \in A }
  \defeq \set{ y \in B \such \some{x \in A} y = f(x) }.
\end{equation*}
%
ter \textbf{izpeljano množico s pogojem}
%
\begin{equation*}
  \set{ f(x) \such x \in A \such \phi(x) }
  \defeq \set{ y \in B \such \some{x \in A} \phi(x) \land y = f(x) }.
\end{equation*}
%
Običajno se piše izpeljano množico s pogojem kar
%
\begin{equation*}
  \set{ f(x) \such x \in A \land \phi(x) }.
\end{equation*}

\begin{zgled}
  Množica vseh kvadratov naravnih števil je izpeljana množica $\set{ n^2 \such n \in \NN }$.
\end{zgled}


\subsection{Slike in praslike}

\begin{definicija}
  Naj bo $f : A \to B$ preslikava:
  %
  \begin{itemize}
  \item \textbf{Praslika} podmnožice $S \subseteq B$ je $\invimg{f}(S) \defeq \set{ x \in A \such f(x) \in S }$.
  \item \textbf{Slika} podmnožice $T \subseteq A$ je $\img{f}(T) \defeq \set{ y \in B \such \some{x \in T} f(x) = y }$.
  \end{itemize}
\end{definicija}
%
Prasliki pravimo tudi \textbf{inverzna slika} in sliki tudi \textbf{direktna slika}.

Kot vidimo, lahko sliko zapišemo tudi kot izpeljano množico
%
\begin{equation*}
  \img{f}(T) \defeq \set{ f(x) \such x \in T }.
\end{equation*}
%
Običajni zapis za prasliko $\invimg{f}(S)$ je tudi $\inv{f}(S)$, vendar tega zapisa mi ne bomo uporabljali, ker napačno namiguje, da ima $f$ inverz. Boste pa ta zapis videli marsikje drugje, ker so matematiki konzervativni bitja, ki raje nekaj stoletij uporabljajo slab zapis, kot da bi spremenili svoje navade.

Običajni zapis za sliko $\img{f}(S)$ je tudi $f(S)$ ali $f[S]$. Predvsem $f(S)$ se uporablja v praksi, a tudi tega odsvetujemo. Kako naj pri takem zapisu ločimo med $f(x)$ in $\img{f}(\set{x})$?

\begin{definicija}
  \textbf{Zaloga vrednosti} preslikave $f : A \to B$ je slika domene, torej $\img{f}(A)$.
\end{definicija}


\subsection{Slike in praslike kot preslikave višjega reda}

Naj bo $f : A \to B$. Tedaj sta tudi $\invimg{f}$ in $\img{f}$ preslikavi.
%
Res, $\invimg{f} : \pow{B} \to \pow{A}$ je določena s predpisom $S \mapsto \set{ x \in A \such f(x) \in S }$, in
$\img{f} : \pow{A} \to \pow{B}$ je določena s predpisom $T \mapsto \set{ f(x) \such x \in T }$

Še več, tudi ">zgornja zvezdica $^{*}$"< in ">spodnja zvezdica $_{*}$"< sta preslikavi
%
\begin{equation*}
  {}^* : B^A \to \pow{A}^{\pow{B}} \qquad
  {}_* : B^A \to \pow{B}^{\pow{A}}
\end{equation*}
%
Ker slikata preslikave v preslikave, pravimo, da sta to \textbf{preslikavi višjega reda}. Primer preslikave višjega reda je tudi odvod, ki funkciji priredi njen odvod.


\subsection{Lastnosti slike in praslike}

\begin{izrek}
  Naj bo $f : A \to B$ preslikava:
  %
  \begin{itemize}
  \item praslike so monotone: če je $S \subseteq T \subseteq A$, potem je $\invimg{f}(S) \subseteq \invimg{f}(T)$
  \item slike so monotone: če je $X \subseteq Y \subseteq B$, potem je $\img{f}(X) \subseteq \img{f}(Y)$.
  \end{itemize}
\end{izrek}

\begin{dokaz}
  Dokaz pustimo za vajo.
\end{dokaz}

\begin{izrek}
  Praslike ohranjajo preseke in unije: za vse $f : A \to B$ in $S : I \to \pow{B}$ velja
  %
  \begin{equation*}
    \textstyle
    \invimg{f} (\bigcup_{i \in I} S_i) = \bigcup_{i \in I} \invimg{f}(S_i)
    \iinn
    \invimg{f} (\bigcap_{i \in I} S_i) = \bigcap_{i \in I} \invimg{f}(S_i).
  \end{equation*}
\end{izrek}

\begin{dokaz}
  Dokažimo prvo izjavo, druga je zelo podobna, le da $\exists$ zamenjamo z $\forall$.
  %
  Dokazujemo $\invimg{f} (\bigcup_{i \in I} S_i) \subseteq \bigcup_{i \in I} \invimg{f}(S_i)$.
  Naj bo $x \in \invimg{f} (\bigcup_{i \in I} S_i)$ in dokazujemo $x \in \bigcup_{j \in I} \invimg{f}(S_j)$.
  Ker je $f(x) \in \bigcup_{i \in I} S_i$ obstaja $k \in I$, da je $f(x) \in S_k$, torej je
  $x \in \invimg{f}(S_k) \subseteq \bigcup_{i \in I} \invimg{f}(S_i)$.
\end{dokaz}

\begin{izrek}
  Naj bo $f : A \to B$ in $T : I \to \pow{A}$. Tedaj je
  %
  \begin{equation*}
    \textstyle
     \img{f} (\bigcup_{i \in I} T_i) = \bigcup_{i \in I} \img{f}(T_i)
     \iinn
     \img{f} (\bigcap_{i \in I} T_i) \subseteq \bigcap_{i \in I} \img{f}(S_i).
  \end{equation*}
  %
\end{izrek}

\begin{dokaz}
  Dokaz prepustimo za vajo.
\end{dokaz}

\begin{vaja}
  Iz zgornjih dveh izrekov izpeljite naslednja dejstva:
  %
  \begin{align*}
    \invimg{f}(\emptyset) &= \emptyset, \\
    \img{f}(\emptyset) &= \emptyset, \\
    \invimg{f}(B) &= A, \\
    \invimg{f}(S \cup T) &= \invimg{f}(S) \cup \invimg{f}(T), \\
    \invimg{f}(S \cap T) &= \invimg{f}(S) \cap \invimg{f}(T).
  \end{align*}
  %
  Poleg tega imamo za $S \subseteq B$ še $\invimg{f}(\compl{S}) = \compl{(\invimg{f}(S))}$.
\end{vaja}


\chapter{Relacije}

\section{Predikati}

\textbf{Predikat} na množici $A$ opredeljuje kako lastnost elementov množice $A$. Če
je $P$ predikat na $A$ in $x \in A$, potem se je smiselno vprašati, ali $x$
zadošča predikatu $P$. Odgovor je resničnostna vrednost, ki jo označimo s $P(x)$.

\begin{primer}
  Na množici naravnih števil $\NN$ lahko obravnavamo predikat ">je sodo
  število"<. Tako na primer $4$ zadošča predikatu ">je sodo število"<, $7$ pa mu ne zadošča.
\end{primer}

Predikat $P$ na množici $A$ lahko predstavimo na dva načina:
%
\begin{itemize}
\item kot preslikavo $P : A \to \two$, ki slika $x \in A$ v resničnostno vrednost $P(x)$,
\item kot podmnožico $P \subseteq A$ tistih $x \in A$, za katere velja $P(x)$.
\end{itemize}
%
Oba načina predstavitve sta uporabna, spoznali pa smo že izomorfizem med njima,
saj velja $P(A) \iso \two^A$.

\section{Relacije}

Relacije s večmestni predikati. Se pravi, relacija $R$ opredeljujejo kako
lastnost urejenih večteric kartezičnega produkta $A_1 \times A_2 \times \cdots \times A_n$. Pravimo, da je $R$ \textbf{$n$-člena} ali \textbf{$n$-mestna relacija} na množicah $A_1, …, A_n$.

\begin{primer}
  Na množici točk v ravnini lahko obravnavamo relacijo kolinearnosti.
  To je trimestna relacija: točke $A$, $B$ in $C$ so kolinearne, kadar obstaja
  premica, ki vsebuje vse tri točke.
\end{primer}

Relacijo $R$ na množicah $A_1, \ldots, A_n$ lahko predstavimo na dva načina, podobno
kot predikate:
\begin{itemize}
\item kot preslikavo $R : A_1 \times A_2 \times \cdots \times A_n \to \two$,
\item kot podmnožico $R \subseteq A_1 \times A_2 \times \cdots \times A_n$.
\end{itemize}
%
Bolj običajna je predstavitev s podmnožicami, zato bomo dejstvo, da je $R$
relacija na množicah $A_1, \ldots, A_n$ zapisali kar kot $R \subseteq A_1 \times A_2 \times \cdots \times A_n$.
Za elemente $x_1 \in A_1, \ldots, x_n \in A_n$ dejstvo, da so v relaciji $R$ zapišemo
$R(x_1, \ldots, x_n)$, včasih pa tudi $(x_1, \ldots, x_n) \in R$.

Na množicah $A_1, \ldots, A_n$ lahko vedno definiramo:
%
\begin{itemize}
\item \textbf{prazno relacijo $\emptyset$}: nobeni elementi niso v prazni relaciji,
\item \textbf{univerzalno relacijo $A_1 \times A_2 \times \cdots \times A_n$}: vsi elementi so v univerzalni relaciji.
\end{itemize}
%
Univerzalna relacija se imenuje tudi \textbf{polna relacija}.

V praksi so najbolj pogoste \textbf{dvomestna relacije}, se pravi relacije na dveh
množicah, $R \subseteq A \times B$.
V tem primeru pravimo množici $A$ \textbf{domena} in $B$ \textbf{kodomena} relacije $R$, relaciji $R$ pa relacija med $A$ in $B$.

Pomembna relacija na množici $A$ je \textbf{enakost} ali \textbf{diagonala} na $A$:
%
\begin{equation*}
    \diag[A] \defeq \set{ (x, y) \in A \times A \such x = y }
\end{equation*}
%
Zakaj ji pravimo diagonala?

Izmed dvočlenih relacij so najbolj pogoste relacije, pri katerih se domena in
kodomena ujemata, torej $R \subseteq A \times A$. V tem primeru pravimo, da je $R$ \textbf{relacija na množici $A$}.

Denimo, da je $R \subseteq A \times B$ relacija, $x \in A$ in $y \in B$. Dejstvo, da sta $x$ in $y$ v relaciji $R$ zapišemo na enega od načinov
%
\begin{equation*}
  (x, y) \in R
  \qquad
  R(x, y)
  \qquad
  x \rel{R} y
\end{equation*}
%
Prvi zapis se uporablja, kadar je $R$ podana kot podmnožica $A \times B$, drugi kadar
podamo~$R$ z logično formulo. Tretji način je tudi pogost, še posebej kadar je
relacija označena s simbolom kot je $=$, $\neq$, $<$, $>$, $\sqsubseteq$, $\sim$ ipd.

Relacijo lahko predstavimo na več načinov, na primer z logično formulo, z resničnostno tabelo, ali z usmerjenim grafom.
%
Z grafom predstavimo $R \subseteq A \times A$ tako, da za vozlišča grafa vzamemo
elemente množice $A$, nato pa narišemo puščico od $x$ do $y$, kadar velja $x \rel{R} y$.


\section{Osnovne lastnosti relacij}

Relacije, ki so pomembne v matematični praksi imajo pogosto lastnosti, ki jih poimenujemo. Za relacijo $R \subseteq A \times A$ pravimo da je:
%
\begin{itemize}
  \item \textbf{refleksivna:} $\all{x \in A} x \rel{R} x$,
  \item \textbf{simetrična:} $\all{x, y \in A} x \rel{R} y \lthen y \rel{R} x$,
  \item \textbf{antisimetrična:} $\all{x, y \in A} x \rel{R} y \land y \rel{R} x \lthen x = y$,
  \item \textbf{tranzitivna:} $\all{x, y, z \in A} x \rel{R} y \land y \rel{R} z \lthen x \rel{R} z$,
  \item \textbf{irefleksivna:} $\all{x \in A} \lnot (x \rel{R} x)$,
  \item \textbf{asimetrična:} $\all{x, y \in A} x \rel{R} y \lthen \lnot (y \rel{R} x)$,
  \item \textbf{sovisna:} $\all{x, y \in A} x \neq y \lthen x \rel{R} y \lor y \rel{R} x$,
  \item \textbf{strogo sovisna:} $\all{x, y \in A} x \rel{R} y \lor y \rel{R} x$.
\end{itemize}
%

\begin{naloga}
  Kako iz usmerjenega grafa relacije razberemo refleksivnost in simetričnost? Kaj pa ostale lastnosti?
\end{naloga}

\section{Operacije na relacijah}

\subsection{Unija, presek in komplement relacij}

Ker so relacije pravzaprav podmnožice, lahko na njih uporabljamo operacije unija $\cup$,
presek $\cap$ in komplement $\compl{\Box}$. Denimo, da sta $R, S \subseteq A \times B$ relaciji. Tedaj velja:
%
\begin{align*}
  x \rel{(R \cup S)} y &\iff x \rel{R} y \lor x \rel{S} y, \\
  x \rel{(R \cap S)} y &\iff x \rel{R} y \land x \rel{S} y, \\
  x \compl{R} y &\iff \lnot (x \rel{R} y).
\end{align*}

\begin{primer}
  Za relacije enakosti in urejenost na realnih številih velja:
  %
  \begin{itemize}
  \item Komplement relacije enakosti $=$ je relacija neenakosti $\neq$.
  \item Unija relacij $<$ in $>$ na realnih številih je relacija $\neq$.
  \item Presek relacij $\leq$ in $\geq$ na realnih številih je relacija $=$.
  \end{itemize}
\end{primer}


\subsection{Transponirana relacija}

Dvojiške relacije lahko tudi \textbf{transponiramo}. Transponiranka relacije $R \subseteq A \times B$ je relacija $\transpose{R} \subseteq B \times A$, definirana s predpisom
%
\begin{equation*}
    y \transpose{R} x \defiff x \rel{R} y
\end{equation*}
%
ali ekvivalentno
%
\begin{equation*}
  \transpose{R} \defeq \set{ (y, x) \in B \times A \such x \rel{R} y }.
\end{equation*}
%
Očitno velja $\transpose{(\transpose{R})} = R$, torej je transponiranje \emph{involucija}.

\begin{primer}
  Transpozicija relacije $<$ na realnih številih $\RR$ je relacija $>$ na $\RR$.
  Komplement relacije $<$ na $\RR$ je relacija $\geq$ na $\RR$.
\end{primer}

\subsection{Kompozitum relacij}

Nadalje definiramo \textbf{kompozitum} relacij $R \subseteq A \times B$ in $S \subseteq B \times C$ kot relacijo $S \circ R \subseteq A \times C$, s predpisom
%
\begin{equation*}
    x \rel{(S \circ R)} z \defiff \some{y \in B} x \rel{R} y \land y \rel{S} z
\end{equation*}
%a
ali ekvivalentno
%
\begin{equation*}
  S \circ R \defeq
  \set{ (x, z) \in A \times C \such \some{y \in B} (x,y) \in R \land (y,z) \in S }.
\end{equation*}
%
Se pravi, da sta $x \in A$ in $z \in C$ v relaciji $S \circ R$, če sta preko $S$ in $R$
povezana s kakim elementom $y \in B$.

\begin{primer}
  Kompozitum relacij ">$x$ je otrok od $y$"< in ">$z$ je mati od $y$"< je relacija
  ">$z$ je babica od $x$"<.
\end{primer}

\begin{izrek}
  Komponiranje relacij je asociativno in diagonala je enota.
\end{izrek}

\begin{naloga}
  Zgornji izrek zapiši bolj natančno, da bo razvidno, kaj so domene in kodomene relacij.
\end{naloga}

\begin{dokaz}
  Najprej dokažimo asociativnost kompozicije.
  %
  Naj bo $R \subseteq A \times B$, $S \subseteq B \times C$ in $T \subseteq C \times D$ ter $a \in A$ in $d \in D$. Tedaj velja
  %
  \begin{align}
    a \rel{(T \circ (S \circ R))} d &\iff  \notag \\
    \some{c \in C} a \rel{(S \circ R)} c \land c \rel{T} d &\iff \notag \\
    \some{c \in C} (\some{b \in B} a \rel{R} b \land b \rel{S} c) \land c \rel{T} d \label{eq:comp-1}
  \end{align}
  %
  in
  %
  \begin{align}
    a \rel{((T \circ S) \circ R)} d &\iff \notag \\
    \some{b \in B} a \rel{R} b \land b \rel{(T \circ S)} d &\iff \notag \\
    \some{b \in B} a \rel{R} b \land (\some{c \in C} b \rel{S} c \land c \rel{T} d) \label{eq:comp-2}
  \end{align}
  %
  Torej je treba dokazati ekvivalenco izjav~\eqref{eq:comp-1} in~\eqref{eq:comp-2}, kar prepuščamo za vajo. Naj namignemo, da je treba pri dokazovanju ekvivalence uporabiti \emph{Frobeniusevo pravilo}
  %
  \begin{equation*}
    (\some{x \in X} p \land q(x)) \liff p \land \some{x \in X} q(x).
  \end{equation*}
  %
  V pravilu je $p$ formula, v kateri $x$ ne nastopa kot prosta spremenljivka.

  Dokažimo še, da je diagonala enota za kompozicijo: naj bo $R \subseteq A \times B$ ter $x \in A$ in $y \in B$. Tedaj velja
  %
  \begin{align*}
    x (\diag[B] \circ R) y &\iff \\
    \some{z \in B} x \rel{R} z \land z \diag[B] y  &\iff \\
    \some{z \in B} x \rel{R} z \land z = y &\iff \\
    x \rel{R} y
  \end{align*}
  %
  V zadnjem koraku smo uporabili ekvivalenco $(\some{u \in U} u = v \land P(v)) \liff P(v)$. Podobno dokažemo, da je diagonala desna enota.
\end{dokaz}

Kompozitum relacij ima torej podobne lastnosti kot kompozitum funkcij.

\subsection{Potenca relacije}

Za $n \in \NN$ definiramo \textbf{$n$-to potenco} relacije $R \subseteq A \times A$ kot relacijo $R^n \subseteq A \times A$ takole:
%
\begin{equation*}
    x R^n y \defiff
    \some{z_0, \ldots, z_n \in A}
    z_0 = x \land z_n = y \land \all{i \in {0, \ldots, n-1}} z_i \rel{R} z_{i+1}.
\end{equation*}
%
To je precej nečitljiva formula. Bolj razumljiva definicija je potenca kot $n$-kratni kompozitum relacije $R$ same s sabo:
%
\begin{equation*}
    R^n \defeq \underbrace{R \circ \cdots \circ R}_n
\end{equation*}
%
kjer se desni $R$ ponovi $n$-krat. Kaj dobimo, ko za $n$ vstavimo $0$? Enoto za kompozitum:
%
\begin{equation*}
    R^0 = \diag[A].
\end{equation*}

\section{Funkcijske relacije}

Funkcijo $f : A \to B$ smo definirali kot \emph{prirejanje} med elementi $A$ in $B$. A
kaj pravzaprav je ">prirejanje"<? Je to funkcijski predpis, program, kaj drugega?
Sedaj lahko povemo natančno: prirejanje, s katerim je podana funkcija, je
\emph{relacija} med elementi domene in kodomene.

\begin{definicija}
  Naj bo $f : A \to B$ funkcija. \textbf{Graf} funkcije $f$ je relacija
  $\Gamma_f \subseteq A \times B$, definirana s predpisom
  %
  \begin{equation*}
    x \,\Gamma_{\!f}\, y \liff f(x) = y
  \end{equation*}
  %
  ali ekvivalentno
  %
  \begin{equation*}
    \Gamma_{\!f} \defeq \set{ (x, y) \in A \times B \such f(x) = y }.
  \end{equation*}
\end{definicija}

Skratka, graf funkcije ni nič drugega kot njeno prirejanje.
%
Sedaj pa se vprašajmo: kakšnim pogojem mora zadoščati relacija $R \subseteq A \times B$, da je prirejanje za neko funkcijo? Odgovor poznamo: biti mora enolična in celovita.

\begin{definicija}
  Relacija $R \subseteq A \times B$ je \textbf{funkcijska relacija}, če je
  %
  \begin{itemize}
  \item \textbf{celovita:} $\all{x \in A} \some{y \in B} x \rel{R} y$ in
  \item \textbf{enolična:} $\all{x \in A} \all{y_1, y_2 \in B} x \rel{R} y_1 \land x \rel{R} y_2 \lthen y_1 = y_2$.
  \end{itemize}
  %
  Ekvivalentno oba pogoja skupaj zapišemo: $\all{x \in A} \exactlyone{y \in B} x \rel{R} y$.
\end{definicija}

Graf $\Gamma_{\!f} \subseteq A \times B$ funkcije $f : A \to B$ je vedno funkcijska relacija.
%
Funkcijska relacija $R \subseteq A \times B$ določa preslikavo $\phi_R : A \to B$ definirano s predpisom
%
\begin{equation*}
  \phi_R : x \mapsto \descr{y \in B} x \rel{R} y.
\end{equation*}
%
Če iz funkcije $f : A \to B$ tvorimo njen graf $\Gamma_{\!f}$, nato pa iz njega funkcijo
$\phi_{\Gamma_{\!f}} : A \to B$ dobimo nazaj prvotno funkcijo $f$. Obratno, če je $R$ funkcijska relacija, tedaj je $\Gamma_{\phi_R}$ enaka $R$. Torej imamo izomorfizem
%
\begin{equation*}
  B^A \iso \set{ R \in \pow{A \times B} \such \all{x \in A} \exactlyone{y \in B} x \rel{R} y }.
\end{equation*}
%

\begin{izjava}
  Kompozitum funkcij se ujema s kompozitumom relacij:
  $\Gamma_{g \circ f} = \Gamma_g \circ \Gamma_{\!f}$.
\end{izjava}

\begin{dokaz}
  Dokaz prepustimo za vajo, še prej pa morate izjavo zapisati bolj natančno: od
  kod in kam slikata preslikavi $f$ in $g$, kaj pomeni kompozitum na levi in kaj
  na desni?
\end{dokaz}


\section{Ovojnice relacij}

Pogosto imamo opravka z relacijo $R$, ki nima želene lastnosti (na primer ni
tranzitivna) mi pa želimo relacijo, ki to lastnost ima. Ali lahko $R$ kako
spremenimo, da bo imela želeno lastnost? Če to lahko naredimo na več načinov,
ali se eden od njih odlikuje?

\begin{definicija}
  Naj bo $R \subseteq A \times A$ relacija. Tedaj pravimo, da je relacija $T \subseteq A \times A$ \textbf{tranzitivna ovojnica} relacije $R$, če velja:
  %
  \begin{enumerate}
  \item $T$ je tranzitivna,
  \item $R \subseteq T$ in
  \item če je $S \subseteq A \times A$ tranzitivna in velja $R \subseteq S$, tedaj je $T \subseteq S$.
  \end{enumerate}
\end{definicija}

Povedano drugače: tranzitivna ovojnica relacije $R$ je \textsf{najmanjša} tranzitivna
relacija, ki vsebuje $R$. Zaenkrat ne vemo, ali ima vsaka relacija tranzitivno
ovojnico.

Izraz ">ovojnica"< uporabljamo, ker si lahko mislimo, da smo relacijo ovili
s tranzitivno relacijo tako, da se ji slednja čim bolj prilega. Namesto ">ovojnica"<
rečemo tudi \textbf{ogrinjača} ali \textbf{zaprtje}.

Poleg tranzitivne ovojnice lahko definiramo tudi druge ovojnice:
%
\begin{itemize}
  \item \textbf{Refleksivna ovojnica} relacije $R \subseteq A \times A$ je najmanjša refleksivna relacija, ki vsebuje $R$.
  \item \textbf{Simetrična ovojnica} relacije $R \subseteq A \times A$ je najmanjša simetrična relacija, ki vsebuje $R$.
  \item \textbf{Refleksivna tranzitivna ovojnica} relacije $R \subseteq A \times A$ je najmanjša refleksivna in tranzitivna relacija, ki vsebuje $R$.
\end{itemize}
%
Ali take ovojnice sploh obstajajo? Obravnavajmo le tranzitivne ovojnice, saj so
ostali dokazi zelo podobni. Ključno pri dokazu obstoja tranzitivne ovojnice je
naslednje dejstvo.

\begin{lema}
  Naj bo $A$ množica in $R : I \to P(A \times A)$ družina relacij na $A$. Če za
  vsak $i \in I$ velja, da je $R_i$ tranzitivna relacija, potem je tudi presek $\bigcap R$ tranzitivna relacija.
\end{lema}

\begin{dokaz}
  Iz definicije preseka družine množic (relacije so le posebne množice) sledi
  %
  \begin{equation*}
  x (\textstyle\bigcap R) y \liff \all{i \in I} x R_i y.
  \end{equation*}
  %
  Dokažimo, da je $\textstyle\bigcap R$ tranzitivna.
  Naj bodo $x, y, z \in A$ in denimo, da velja
  $x (\textstyle\bigcap R) y$ in $y (\textstyle\bigcap R) z$, kar je ekvivalentno
  %
  \begin{equation*}
    \all{i \in I} x R_i y
    \iinn
    \all{j \in I} y R_j z.
  \end{equation*}
  %
  Dokazati moramo $x (\textstyle\bigcap R) z$, kar je ekvivalentno
  %
  $\all{k \in I} x R_k z$.
  %
  Naj bo torej $k \in I$, dokazujemo $x R_k z$. Uporabimo $\all{i \in I} x R_i y$ pri $i = k$ in dobimo $x R_k y$.
  %
  Uporabimo $\all{j \in I} y R_j z$ pri $j = k$ in dobimo $y R_k z$.
  %
  Po predpostavki je $R_k$ tranzitivna relacija, torej velja $x R_k z$.
\end{dokaz}

\begin{izrek}
  Vsaka relacija ima enolično tranzitivno ovojnico.
\end{izrek}

\begin{dokaz}
  Najprej premislimo, da ima $R$ največ eno tranzitivno ovojnico: če sta
  $S$ in $T$ obe tranzitivni ovojnici $R$, potem iz definicije tranzitivne ovojnice
  sledi $S \subseteq T$ in $T \subseteq S$, torej velja $S = T$.

  Sedaj pokažimo, da $R$ ima tranzitivno ovojnico. Naj bo $R \subseteq A \times A$. Definirajmo množico relacij
  %
  \begin{equation*}
    D \defeq \set{ S \subseteq A \times A \such \text{$R \subseteq S$ in $S$ je tranzitivna} }.
  \end{equation*}
  %
  Trdimo, da je $\textstyle\bigcap D$ tranzitivna ovojnica relacije $R$.
  %
  Iz prejšnje leme sledi, da je $\textstyle\bigcap D$ tranzitivna.
  %
  Ker velja $R \subseteq S$ za vsak $S \in D$, seveda sledi $R \subseteq \textstyle\bigcap D$.
  %
  Če je $R \subseteq T$ in $T \subseteq A \times A$ tranzitivna relacija, tedaj velja $T \in D$, torej je $\bigcap D \subseteq T$.
\end{dokaz}


Po istem kopitu pokažemo, da ima vsaka relacija $R \subseteq A \times A$ tudi ostale
ovojnice. Je pa zgornji izrek neroden, ker nam dokaz ne poda uporabnega opisa
tranzitivne ovojnice. Povejmo, kako lahko razne ovojnice opišemo bolj
eksplicitno:
%
\begin{enumerate}
\item Refleksivna ovojnica relacije $R$ je relacija $R \cup \diag[A]$, se pravi, da
  relaciji $R$ dodamo še diagonalo.
\item 
  Simetrična ovojnica relacije $R$ je relacija $R \cup \transpose{R}$.
\item
  Tranzitivna ovojnica relacije $R$ je relacija $R^{+} \defeq \bigcup_{n \geq 1} R^n$, se pravi
  %
  \begin{equation*}
    R^{+} \defeq R \cup (R \circ R) \cup (R \circ R \circ R) \cup \cdots
  \end{equation*}
\item
  Refleksivna tranzitivna ovojnica relacije $R$ je relacija $R^{*} \defeq \bigcup_{n \geq 0} R^n$, se pravi
  %
  \begin{equation*}
  R^{*} \defeq \diag[A] \cup R \cup (R \circ R) \cup (R \circ R \circ R) \cup \cdots
  \end{equation*}
\end{enumerate}


\chapter{Ekvivalenčne relacije}

\section{Ekvivalenčne relacije}

\begin{definicija}
  Relacija $R \subseteq A \times A$ je \textbf{ekvivalenčna relacija}, če je refleksivna, tranzitivna in simetrična. Kadar velja $x \rel{R} y$, pravimo, da sta $x$ in $y$ \textbf{ekvivalentna} glede na~$R$.
\end{definicija}

\begin{opomba}
  Kdor reče ">ekvivalentna relacija"<, je noob. Kdor reče, da sta ">$x$ in $y$
  ekvivalenčna"<, je rookie.
\end{opomba}

Ekvivalenčne relacije se običajno označuje s simboli, ki so podobni znaku za enakost:
$\equiv$, $\sim$, $\simeq$, $\cong$.

\begin{primer}
  Primeri ekvivalenčnih relacij:
  \begin{enumerate}
    \item Relacija ">vzporednost"< med premicami v ravnini.
    \item Relacija ">skladnost"< med trikotniki v ravnini.
    \item Relacija ">podobnost"< med trikotniki v ravnini.
    \item Relacija ">isti ostanek pri deljenju s 7"< na množici $\NN$.
    \item Prazna relacija $\emptyset \subseteq A \times A$ je ekvivalenčna le v primeru, da je $A = \emptyset$.
    \item Polna relacija $A \times A$ je ekvivalenčna.
    \item Diagonala (enakost) je ekvivalenčna relacija.
  \end{enumerate}
\end{primer}

\subsection{Ekvivalenčna relacija porojena s preslikavo}

Posebej pomemben je primer ekvivalenčne relacije \textbf{porojene (ali inducirane) s preslikavo}:
naj bo $f : A \to B$ preslikava in definirajmo relacijo $\sim_f$ na $A$ s predpisom
%
\begin{equation*}
  x \sim_f y \liff f(x) = f(y)
\end{equation*}
%
Tedaj je $\sim_f$ ekvivalenčna relacija:
%
\begin{itemize}
\item refleksivnost: $x \sim_f x$ velja, ker velja $f(x) = f(x)$,
\item tranzitivnost: če je $x \sim_f y$ in $y \sim_f z$, potem je $f(x) = f(y)$ in $f(y) = f(z)$, torej $f(x) = f(z)$ in $x \sim_f z$,
\item simetričnost: če je $x \sim_f y$, potem je $f(x) = f(y)$, torej $f(y) = f(x)$ in $y \sim_f x$.
\end{itemize}
%
Ali je vsaka ekvivalenčna relacija porojena z neko preslikavo?

\begin{primer}
  Premici sta vzporedni natanko tedaj, ko imata enaka smerna vektorja. Če je
  torej $P$ množica vseh premic, $\RR^2$ množica vektorjev v ravnini, in $s : P \to \RR^2$
  preslikava, ki premici $P$ priredi njen enotski smerni vektor, ki leži v zgornji polravnini ali
  na pozitivnem delu osi $x$, tedaj velja
  \begin{equation*}
    p \parallel q \liff s(p) = s(q).
  \end{equation*}
  %
  Torej je vzporednost porojena s preslikavo $s$.
\end{primer}

\section{Ekvivalenčni razredi in kvocientne množice}

\begin{definicija}
  Naj bo $E \subseteq A \times A$ ekvivalenčna relacija. \textbf{Ekvivalenčni razred} elementa $x \in A$ je množica
  $[x]_E \defeq \set{ y \in A \such x \rel{E} y }$. Z besedami: ekvivalenčni razred~$x$ je množica vseh elementov, ki so mu
  ekvivalentni.
\end{definicija}

\begin{opomba}
  Kdor reče ">ekvivalentni razred"<, je newbie.
  Če pustimo šalo ob strani: ekvivalenčni razredi se tako imenujejo zaradi zgodovinskih razlogov. Beseda ">razred"< nakazuje dejstvo, da so imajo elementi ekvivalenčnega razredi vsi nekaj skupnega (">delavski razred"<, ">Tina Maze je razred zase"<) in ne, da niso množice (saj očitno so).
\end{opomba}

\begin{definicija}
  Naj bo $E \subseteq A \times A$ ekvivalenčna relacija. \textbf{Kvocientna ali faktorska množica} ali \textbf{kvocient} $A/E$ je množica vseh ekvivalenčnih razredov:
  %
  \begin{equation*}
    A/E \defeq \set{ \xi \in \pow{A} \such \some{x \in A} \xi = [x]_E }.
  \end{equation*}
  %
  Z izpeljanimi množicami lahko to zapišemo bolj razumljivo
  % 
  \begin{equation*}
    A/E = \set{ [x]_E \such x \in A }.
  \end{equation*}
  %
  \textbf{Kanonična kvocientna preslikava} $q_E : A \to A/E$ je preslikava, ki vsakemu elementu
  priredi njegov ekvivalenčni razred: $q_E(x) \defeq [x]_E$.
\end{definicija}

\begin{izrek}
  Vsaka ekvivalenčna relacija je porojena z neko preslikavo.
\end{izrek}

\begin{dokaz}
  Dokažimo, da je ekvivalenčna relacija porojena s svojo kvocientno preslikavo.

  Naj bo $E$ ekvivalenčna relacija na $A$. Najprej ugotovimo naslednje: za vse $x,
  y \in A$ velja
  %
  \begin{equation*}
    x \rel{E} y \liff [x]_E = [y]_E.
  \end{equation*}

  ($\lthen$) Če je $x \rel{E} y$ potem je $[x]_E \subseteq [y]_E$, ker iz $z \rel{E} x$ in $x \rel{E} y$ sledi $z \rel{E} y$.
  Podobno dokažemo $[y]_E \subseteq [x]_E$.

  ($\Leftarrow$) Če je $[x]_E = [y]_E$ potem je $y \in [y]_E = [x]_E$, torej po definiciji $[x]_E$
  dobimo $x \rel{E} y$.

  Sedaj izrek sledi zlahka: $q_E(x) = q_E(y) \liff [x]_E = [y]_E \liff x \rel{E} y$.
\end{dokaz}


\subsection{Razdelitev množice}

\begin{definicija}
  \textbf{Razdelitev} ali \textbf{particija} množice $A$ je množica nepraznih, paroma
  disjunktnih množic, ki tvorijo pokritje $A$ (kar pomeni, da je $A$ enaka njihovi uniji). Se
  pravi, to je množica $S \subseteq \pow{A}$, za katero velja:
  %
  \begin{enumerate}
  \item Elementi razdelitve so neprazni: $\all{B \in S} B \neq \emptyset$.
  \item Vsaka dva elementa razdelitve sta bodisi enaka bodisi disjunktna:
    %
    \begin{equation*}
      \all{B, C \in S} B = C \lor B \cap C = \emptyset.
    \end{equation*}
  \item Elementi razdelitve tvorijo pokritje $A$, se pravi $A = \bigcup S$.
  \end{enumerate}
\end{definicija}

\begin{primer}
  Primeri razdelitev:
  %
  \begin{enumerate}
  \item Navpične premice tvorijo razdelitev ravnine.
  \item Množici sodih in lihih števil tvorita razdelitev naravnih števil.
  \item Množica $\set{\set{1,2}, \set{3,5}, \set{4,6,7}}$ tvori razdelitev $\set{1,2,3,4,5,6,7}$.
  \item Množica $\set{\set{1,2,3,4,5,6,7}}$ tvori razdelitev $\set{1,2,3,4,5,6,7}$.
  \end{enumerate}
\end{primer}

\begin{izrek}
  Naj bo $E \subseteq A \times A$ ekvivalenčna relacija. Njeni ekvivalenčni razredi tvorijo
  razdelitev množice $A$.
\end{izrek}

\begin{dokaz}
  Dokažimo, da so ekvivalenčni razredi neprazni, paroma disjunktni in da tvorijo pokritje.

  Naj bo $\xi \in \pow{A}$ ekvivalenčni razred za $E$. Tedaj obstaja $x \in A$, da je $\xi = [x]_E$,
  torej je $x \in \xi$ in zato $\xi \neq \emptyset$.

  Naj bosta $\zeta, \xi \in \pow{A}$. Dokazali bomo $\zeta \cap \xi \neq \emptyset \lthen \zeta = \xi$. Če je $x \in \zeta \cap \xi$, potem velja $\zeta \subseteq \xi$ ker: naj bo $y \in \zeta$, tedaj je $y \rel{E} x$ in ker je $x \in \xi$ velja $y \in \xi$. Simetrično dokažemo $\xi \subseteq \zeta$.

  Očitno je unija vseh ekvivalenčnih razredov podmnožica $A$, saj je vsak ekvivalenčni razred podmnožica $A$. Zagotovo
  pa je vsak $x \in A$ v kakem ekvivalenčnem razredu, namreč $x \in [x]_E$.
\end{dokaz}

Torej vsaka ekvivalenčna relacija na $A$ določa razdelitev množice $A$, namreč na
ekvivalenčne razrede. Velja pa tudi obrat: vsaka razdelitev $S \subseteq \pow{A}$ določa ekvivalenčno
relacijo na $A$, namreč $\simeq_S$ definiran s predpisom
\begin{equation*}
    x \simeq_S y \defiff \some{B \in S} x \in B \land y \in B.
\end{equation*}
%
Z besedami: $x$ in $y$ sta ekvivalentna, kadar sta v istem elementu razdelitve. Pravzaprav
smo ugotovili, da imamo izomorfizem množic
%
\begin{equation*}
  \set{ E \subseteq A \times A \such \text{$E$ je ekvivalenčna relacija na $A$} } \iso
  \set{ S \subseteq \pow{A} \such \text{$S$ je razdelitev $A$} }.
\end{equation*}
%
V eno smer izomorfizem ekvivalenčni relaciji $E$ priredi njeno razdelitev, v drugo pa razdelitvi priredimo ekvivalenčno
relacijo, kakor smo to opisali zgoraj. (Premislite, da sta ti preslikavi inverza.)


\subsection{Prerezi kvocientne preslikave in aksiom izbire}

Ekvivalenčni razred je natanko določen že z enim od svojih elementov, zato pogosto želimo
namesto ekvivalenčnih razredov navesti le njihove predstavnike.

\begin{definicija}
  Naj bo $E$ ekvivalenčna relacija na $A$. Množico $C \subseteq A$, ki vsak
  ekvivalenčni razred relacije $E$ seka natanko enkrat, imenujemo \textbf{izbor predstavnikov}
  (ekvivalenčnih razredov) za relacijo $E$.
\end{definicija}

Izbor predstavnikov $C \subseteq A$ za $E$ določa preslikavo $c : A/E \to A$, ki priredi
ekvivalenčnemu razredu $\xi$ tisti $x \in \xi$, ki je element $C$:
%
\begin{align*}
  c &: A/E \to A \\
  c &: \xi \mapsto \descr{x \in \xi} x \in C
\end{align*}
%
Preslikava $c : A/E \to A$ je \emph{prerez} kvocientne preslikave $q_E : A \to A/E$.

\begin{izjava}
  Če je $s : A/E \to A$ prerez kvocientne preslikave $q_E : A \to A/E$, potem je
  njegova slika $\img{s}(A/E) = \set{ c(\xi) \such \xi \in A/E }$ izbor predstavnikov za $E$.
\end{izjava}

\begin{dokaz}
  Vaja.
\end{dokaz}

Ker izbor predstavnikov in prerez kvocientne preslikave določata drug drugega, včasih tudi
prerez imenujemo ">izbor predstavnikov"<.

\begin{primer}
  Definirajmo $\sim$ na množici celih števil $Z$ s predpisom
  %
  \begin{equation*}
    a \sim b \defiff 7 \mathrel{|} a - b.
  \end{equation*}
  %
  Torej sta števili $a$ in $b$ ekvivalentni, če dasta enak ostanek pri deljenju s~$7$,
  na primer $13 \sim 20$ in $\lnot (13 \sim 15)$.
  %
  Ekvivalenčni razred števila $a$ dobimo tako, da $a$ prištejemo vse večkratnike števila $7$:
  %
  \begin{equation*}
    [a]_{\sim} = \set{ a + 7 \cdot k \such k \in \ZZ }.
  \end{equation*}
  %
  Na primer,
  \begin{equation*}
    [13]_\sim = \set{ 7 \cdot k + 13 \such k \in \ZZ }
           = \set{ \ldots, -22, -15, -8, -1, 6, 13, 20, 27, 34, 41, \ldots}.
  \end{equation*}
  %
  Koliko pa je ekvivalenčnih razredov? Toliko, kot je ostankov pri deljenju s~$7$, torej sedem. Množica
  $\set{0, 1, 2, 3, 4, 5, 6}$ je izbor predstavnikov za $\sim$, saj je vsako celo število ekvivalentno natanko enemu od
  teh števil po modulu $7$.
  %
  Ni pa to edini izbor! Tudi $\set{0, 1, 2, 3, 4, 5, 13}$ je izbor in prav tako $\set{-7, -6, -5, -4, -3, -2, -1}$.
\end{primer}

Ali ima vsaka ekvivalenčna relacija izbor predstavnikov? Da to vprašanje ni tako
enostavno, kot se zdi na prvi pogled, doma premislite o naslednji nalogi.

\begin{naloga}
  Na množici realnih števil $\RR$ definiramo relacijo $E$ s predpisom
  %
  \begin{equation*}
    x \rel{E} y  \defiff  x - y \in \QQ.
  \end{equation*}
  %
  Se pravi, da sta števili ekvivalentni, če je njuna razlika racionalno število. Podajte kak
  izbor predstavnikov za $E$.
\end{naloga}

\begin{izrek}
  Naslednje izjave so ekvivalentne:
  %
  \begin{enumerate}
  \item Vsaka surjektivna preslikava ima desni inverz (prerez).
  \item Vsaka ekvivalenčna relacija ima izbor predstavnikov.
  \item Vsaka družina nepraznih množic ima funkcijo izbire.
  \item Produkt družine nepraznih množic je neprazen.
  \end{enumerate}
\end{izrek}

\begin{dokaz}
  ($1 \lthen 2$):
  %
  Naj bo $E \subseteq A \times A$ ekvivalenčna relacija na $A$. Tedaj je $q_E : A \to A/E$
  surjektivna, zato ima po predpostavki (1) prerez, ki določa izbor predstavnikov.

  ($2 \lthen 3$):
  %
  Naj bo $A : I \to \Set$ družina nepraznih množic. Naj bo $\sim$ ekvivalenčna relacija
  na koproduktu $K \defeq \sum_{i \in I} A_i$, porojena s prvo projekcijo $\fst : K \to I$, t.j.,
  %
  \begin{equation*}
    \inj[i](x) \sim \inj[j](y) \liff i = j.
  \end{equation*}
  %
  Po predpostavki (2) obstaja izbor predstavnikov za $\sim$, se pravi taka množica $C \subseteq K$, da
  za vsak $u \in K$ obstaja natanko en $v \in C$, da je $\fst(u) = \fst(v)$. Definirajmo $f : I \to
  \bigcup A$ s predpisom
  %
  \begin{equation*}
    f(i) \defeq \descr{x \in A_i} \inj[i](x) \in C
  \end{equation*}
  %
  Očitno je $f$ funkcija izbire za družino $A$, če je izraz na desni veljaven:
  %
  \begin{itemize}
  \item Enoličnost: iz $\inj[i](x) \in C$ in $\inj[i](y) \in C$ sledi $\inj[i](x) = \inj[i](y)$.
  \item Celovitost: ker je $A_i$ neprazna, obstaja $z \in A_i$, torej obstaja $v \in C$, da je
    $i = \fst(\inj[i](z)) = \fst(v)$, in je potemtakem $\snd(v) \in A_i$ element, za katerega velja
    $\inj[i](\snd(v)) \in C$.
  \end{itemize}

  ($3 \lthen 4$):
  %
  Elementi produkta so funkcije izbire, zato je produkt res neprazen, če obstaja
  kaka funkcija izbire.

  ($4 \lthen 1$):
  %
  Naj bo $f : X \to Y$ surjektivna. Definirajmo družino $A : Y \to \Set$ s
  predpisom $A_y = \invimg{f}(\set{y})$. Ker je $f$ surjektivna, je $A$ družina nepraznih
  množic. Po predpostavki (4) je produkt te družine neprazen, torej vsebuje neko
  funkcijo izbire $c : Y \to \bigcup A$, se pravi, da je $f(c(y)) = y$ za vsak $y \in Y$.
  Opazimo še, da je $\bigcup A = X$, torej je $c$ prerez $f$.
\end{dokaz}

Izbor predstavnikov je torej ekvivalenten še nekaterim drugim trditvam. Pa te veljajo? Za
to potrebujemo aksiom.

\begin{aksiom}[Aksiom izbire]
  Vsaka družina nepraznih množic ima funkcijo izbire.
\end{aksiom}

Se pravi, če je $A : I \to \Set$ taka družina množica, da za vsak $i \in I$ velja $A_i \neq \emptyset$,
tedaj obstaja $f : I \to \bigcup A$, za katerega je $f(i) \in A_i$ za vse $i \in I$.
%
O aksiomu izbire bomo še govorili.


\subsection{Univerzalna lastnost kvocientne množice}

Naj bo $E$ ekvivalenčna relacija na $A$ in $B$ množica. Pogosto želimo definirati
preslikavo
%
\begin{equation*}
    f : A/E \to B
\end{equation*}
%
s pomočjo preslikave $A \to B$. Kdaj lahko to naredimo?

\begin{izrek}
  Naj bo $E$ ekvivalenčna relacija na $A$ in $g : A \to B$ preslikava, ki je \emph{skladna} z $E$, kar pomeni da $g$
  slika ekvivalentne elemente v enake: $\all{x, y \in A} x \rel{E} y \lthen g(x) = g(y)$. Tedaj obstaja natanko ena
  preslikava $f : A/E \to B$, da je $f([x]_E) = g(x)$ za vse $x \in A$, ali drugače povedano, $f \circ q_E = g$.
\end{izrek}

\begin{dokaz}
  Dokažimo najprej, da imamo največ eno tako preslikavo. Denimo da za $f_1 : A/E \to B$ in
  $f_2 : A/E \to B$ velja $f_1 \circ q_E = f_2 \circ q_E$. Ker je $q_E$ surjektivna, je epi in jo smemo
  krajšati na desni, od koder res sledi $f_1 = f_2$.

  Sedaj dokažimo, da $f$ obstaja. V ta namen naj bo $\phi \subseteq A/E \times B$ relacija
  %
  \begin{equation*}
    \phi(\xi, y) \defiff \some{x \in A} x \in \xi \land g(x) = y.
  \end{equation*}
  %
  Trdimo, da je $\phi$ funkcijska relacija:
  %
  \begin{itemize}
  \item
    Enoličnost: če je $\phi(\xi, y_1)$ in $\phi(\xi, y_2)$, potem obstajata $x_1, x_2 \in \xi$, da je $g(x_1) = y_1$
    in $g(x_2) = y_2$. Ker pa velja $x_1 \rel{E} x_2$ in je $g$ skladna z $E$, sledi $y_1 = g(x_1) = g(x_2) = y_2$.

  \item  Celovitost: naj bo $\xi \in A/E$. Tedaj obstaja $x \in \xi$. Očitno velja $\phi(\xi, g(x))$.
  \end{itemize}
  %
  Naj bo $f : A/E \to B$ preslikava, ki je določena s funkcijsko relacijo $\phi$. Za $x \in A$
  velja $\phi([x]_E, f([x]_E))$, od tod pa iz definicije $\phi$ sledi tudi $g(x) = f([x]_E)$.
\end{dokaz}

\begin{opomba}
  Profesorja prosite, da pojasni ali sem zapiše, zakaj se reče ">univerzalna lastnost"< kvocientne množice.
\end{opomba}


\section{Kanonična razčlenitev preslikave}

Naj bo $f : A \to B$ preslikava. Naj bo $\sim_f$ ekvivalenčna relacija na $A$, ki jo porodi
$f$, in $q_f : A \to A/{\sim_f}$ kanonična kvocientna preslikava (morali bi jo pisati $q_{\sim_f}$,
kar je nečitljivo). Naj bo $i_f : \img{f}(A) \to B$ kanonična inkluzija slike $f$ v kodomeno.
Preslikava $f : A \to \img{f}(A)$ je skladna s $\sim_f$, zato obstaja (natanko ena) preslikava
$b_f : A/{\sim_f} \to \img{f}(A)$, da velja $b_f([x]_\sim) = f(x)$. Trdimo:
%
\begin{enumerate}
\item $f = i_f \circ b_f \circ q_f$ in
\item $q_f$ je surjektivna, $b_f$ je bijektivna in $i_f$ je injektivna.
\end{enumerate}
%
Računajmo: $f(x) = b_f([x]_\sim) = i_f(b_f([x]_\sim)) = i_f(b_f(q_f(x)))$, za vse $x \in A$, od
koder sledi prva trditev.

Vemo že, da je kanonična kvocientna preslikava surjektivna in kanonična inkluzija
injektivna. Ostane nam še bijektivnost preslikave $b_f$:
%
\begin{itemize}
\item $b_f$ je injektivna: naj bosta $\xi, \zeta \in A/{\sim_f}$ in denimo, da velja $b_f(\xi) = b_f(\zeta)$.
  Obstajata $x, y \in A$, da je $\xi = [x]_\sim$ in $\zeta = [y]_\sim$. Velja
  %
  \begin{equation*}
    f(x) = i_f(b_f(q_f(x))) = i_f(b_f(\xi)) = i_f(b_f(\zeta)) = i_f(b_f(q_f(y))) = f(y),
  \end{equation*}
  %
  torej je $x \sim_f y$ in zato $\xi = [x]_\sim = [y]_\sim = \zeta$.

  \item $b_f$ je surjektivna: naj bo $u \in \img{f}(A)$. Tedaj obstaja $x \in A$, da je $u = f(x)$.
  Vzemimo $\xi = [x]_\sim$ in preverimo: $b_f(\xi) = b_f([x]_\sim) =f(x) = u$.
\end{itemize}

\chapter{Relacije urejenosti}

\section{Relacije urejenosti}
\begin{definicija}
  Relacija $R \subseteq A \times A$ je:
  %
  \begin{enumerate}
  \item \textbf{šibka urejenost}, ko je refleksivna in tranzitivna,
  \item \textbf{delna urejenost}, ko je refleksivna, tranzitivna in antisimetrična,
  \item \textbf{linearna urejenost}, ko je delna urejenost in je strogo sovisna ($\all{x, y \in A} x \rel{R} y \lor y \rel{R} x$).
  \end{enumerate}
\end{definicija}

Za relacije urejenosti ponavadi uporabljamo simbole, ki spominjajo na znak $\leq$, kot so $\preceq$, $\subseteq$, $\sqsubseteq$ ipd.

\begin{primer}
  Primeri urejenosti:
  \begin{enumerate}
    \item Relacija deljivosti na naravnih številih je delna urejenost.
    \item Relacija deljivosti na celih številih je šibka urejenost, ni pa delna urejenost.
    \item Relacija $\leq$ na realnih številih je linearna urejenost.
    \item Relacija $\subseteq$ na $\pow{A}$ je delna urejenost. Za katere množice $A$ je linearna?
    \item Relacija $=$ je delna urejenost. Imenuje se tudi \textbf{diskretna urejenost}.
  \end{enumerate}
\end{primer}


\begin{definicija}
  V delni ureditvi $(P, {\leq})$ je \textbf{veriga} taka podmnožica $V \subseteq P$, ki je linearno urejena z relacijo~$\leq$, se pravi $\all{x, y \in V} x \leq y \lor y \leq x$. \textbf{Antiveriga} je taka podmnožica $A \subseteq P$, ki je diskretno urejena z relacijo~$\leq$, se pravi $\all{x, y \in A} x \leq y \lthen x = y$.
\end{definicija}

\begin{primer}
\end{primer}

\begin{primer}
  Primeri verig in antiverig:
  %
  \begin{itemize}
  \item Če je $(P, {\leq})$ linearno urejena, je vsaka njena podmnožica veriga. Na primer, vsaka podmnožica $\NN$ je veriga glede na~$\leq$.
  \item Potence števila $2$ tvorijo verigo v $\NN$ glede na relacijo deljivosti.
  \item Praštevila tvorijo antiverigo v $\NN$ glede na relacijo deljivosti.
  \item V $(\pow{\QQ}, {\subseteq})$ imamo neštevno verigo
    %
    $V = \set{S \in \pow(\QQ) \mid \text{$S$ je doljna množica}}$.
    %
    Množica $S \subseteq \QQ$ je \textbf{doljna}, če velja
    $\all{x y \in \QQ} x \leq y \land y \in \QQ \lthen x \in \QQ$.
    Res, vsak Dedekindov rez je doljna množica, le-teh pa je neštevno mnogo.
  \end{itemize}
\end{primer}



\subsection{Hassejev diagram}

Končno delno ureditev $(A, \leq)$ lahko predstavimo s \textbf{Hassejevim diagramom}: elemente
množice $A$ narišemo tako, da je $x$ pod $y$, kadar velja $x \leq y$. Nato povežemo vozlišči $x$ in $y$, če je $y$ neposredni naslednik $x$, se pravi, da velja $x \neq y$, $x \leq y$ in iz $x \leq z \leq y$ sledi $x = z \lor z = y$.

\begin{naloga}
  Narišite Hassejev diagram relacije deljivosti na množici $\set{0, 1, \dots, 10}$ ter
  Hassejev diagram relacije $\subseteq$ na množici $\pow(\{a,b,c\})$.
\end{naloga}

\begin{naloga}
  Kako v Hassejevem diagramu prepoznamo verigo? In kako prepoznamo antiverigo?
\end{naloga}


\subsection{Operacije na urejenostih}

\subsubsection{Obratna urejenost}

Če je $\leq$ delna urejenost na $P$ potem je tudi transponirana relacija $\geq$, definirana z
%
\begin{equation*}
    x \geq y \liff x \leq y,
\end{equation*}
%
delna urejenost na $P$. Če je $\leq$ linearna, je $\geq$ linearna.

\subsubsection{Produktna in leksikografska urejenost}

Naj bosta $(P, {\leq_P})$ in $(Q, {\leq_Q})$ delni urejenosti. Na kartezičnem produktu $P \times Q$ lahko definiramo dve urejenosti.

Prva je \textbf{produktna} urejenost
%
\begin{equation*}
  (x_1,y_1) \leq_{\times} (x_2,y_2) \defiff x_1 \leq_P x_2 \land y_1 \leq_Q y_2
\end{equation*}
%
in druga \textbf{leksikografska} urejenost
%
\begin{equation*}
  (x_1,y_1) \preceq_\mathrm{lex} (x_2,y_2)
  \defiff (x_1 \neq x_2 \land x_1 \leq_P x_2) \lor (x_1 = x_2 \land y_1 \leq_Q y_2).
\end{equation*}


\begin{naloga}
  Kako si predstavljamo produktno in leksikografsko ureditev na $[0,1] \times [0,1]$, če $[0,1]$ uredimo z običajno relacijo $\leq$? Na sliki označite območji
  %
  \begin{equation*}
    \set{(x,y) \in [0,1] \times [0, 1] \such (1/2,1/3) \leq_\times (x,y)}
  \end{equation*}
  %
  in
  %
  \begin{equation*}
    \set{(x,y) \in [0,1] \times [0, 1] \such (1/2,1/3) \leq_\mathrm{lex} (x,y)}.
  \end{equation*}
\end{naloga}

\begin{izjava}
  Produktna in leksikografska urejenosti sta delni urejenosti. Leksikografska urejenost linearnih urejenosti je linearna.
\end{izjava}

\begin{dokaz}
  Dejstvo, da je produktna urejenost refleksivna, tranzitivna in antisimetrična, pustimo za vajo. Preverimo, da je leksikografska urejenost $\leq_\mathrm{lex}$ delna urejenost.

  Dokaz, da je $\leq_\mathrm{lex}$ je refleksivna: za vsak $(x, y) \in P \times Q$ velja $x = x \land y \leq_Q y$, torej velja $(x, y) \sqsubseteq (x, y)$.

  Dokaz, da je $\leq_\mathrm{lex}$ je antisimetrična: naj bosta $(x_1,y_1), (x_2,y_2) \in P \times Q$ in denimo, da velja
  %
  \begin{equation*}
    (x_1, y_1) \leq_\mathrm{lex} (x_2, y_2) \land (x_2, y_2) \leq_\mathrm{lex} (x_1, y_1)
  \end{equation*}
  %
  To je ekvivalentno
  %
  \begin{align*}
  & (x_1 \neq x_2 \land x_1 \leq_P x_2 \land x_2 \neq x_1 \land x_2 \leq_P x_1) \lor {}\\
  & (x_1 \neq x_2 \land x_1 \leq_P x_2 \land x_2 = x_1 \land y_2 \leq_Q y_1) \lor {}\\
  & (x_1 = x_2 \land y_1 \leq_Q y_2 \land x_2 \neq x_1 \land x_2 \leq_P x_1) \lor {}\\
  & (x_1 = x_2 \land y_1 \leq_Q y_2 \land x_2 = x_1 \land y_2 \leq_Q y_1).
  \end{align*}
  %
  Če v zgornji formuli upoštevamo, da je $x_1 \neq x_2 \land x_1 = x_2$, vidimo, da sta drugi in tretji disjunkt ekvivalentna $\bot$, zato
  je izjava ekvivalentna:
  \begin{align*}
  &(x_1 \neq x_2 \land x_1 \leq_P x_2 \land x_2 \neq x_1 \land x_2 \leq_P x_1) \lor {}\\
  &(x_1 = x_2 \land y_1 \leq_Q y_2 \land x_2 = x_1 \land y_2 \leq_Q y_1).
  \end{align*}
  %
  A tudi prvi disjunkt je ekvivalenten $\bot$, ker iz $x_1 \leq_P x_2 \land x_2 \leq_P x_1$ sledi $x_1 = x_2$, saj je $\leq_P$ po predpostavki antisimetrična. Torej ostane samo zadnji disjunkt, ki je ekvivalenten
  \begin{equation*}
    x_1 = x_2 \land y_1 \leq_Q y_2 \land y_2 \leq_Q y_1.
  \end{equation*}
  %
  Ker je $\leq_Q$ antisimetrična, sledi $x_1 = x_2$ in $y_1 = y_2$, kar smo želeli dokazati.

  Dokaz, da je $\leq_\mathrm{lex}$ tranzitivna: naj bodo $(x_1,y_1), (x_2,y_2), (x_3, y_3) \in P \times Q$ in denimo, da velja
  %
  \begin{equation*}
    (x_1, y_1) \leq_\mathrm{lex} (x_2, y_2) \land (x_2, y_2) \leq_\mathrm{lex} (x_3, y_3).
  \end{equation*}
  %
  To je ekvivalentno
  %
  \begin{align*}
  & (x_1 \neq x_2 \land x_1 \leq_P x_2 \land x_2 \neq x_3 \land x_2 \leq_P x_3) \lor  {} \\
  & (x_1 \neq x_2 \land x_1 \leq_P x_2 \land x_2 = x_3 \land y_2 \leq_Q y_3) \lor {} \\
  & (x_1 = x_2 \land y_1 \leq_Q y_2 \land x_2 \neq x_3 \land x_2 \leq_P x_3) \lor {} \\
  & (x_1 = x_2 \land y_1 \leq_Q y_2 \land x_2 = x_3 \land y_2 \leq_Q y_3)
  \end{align*}
  %
  Obravnavajmo štiri primere in v vsakem od njih dokažimo $(x_1, y_1) \leq_\mathrm{lex} (x_3, y_3)$, se pravi
  $(x_1 \neq x_3 \land x_1 \leq_P x_3) \lor (x_1 = x_3 \land y_1 \leq_Q y_3)$:
  %
  \begin{enumerate}
  \item Če velja $x_1 \neq x_2 \land x_1 \leq_P x_2 \land x_2 \neq x_3 \land x_2 \leq_P x_3$: ker je $\leq$ tranzitivna sledi $x_1 \leq_P x_3$, poleg tega pa velja $x_1 \neq
    x_3$: če bi veljalo $x_1 = x_3$, bi iz predpostavk dobili $x_3 \leq_P x_2 \land x_2 \leq_P x_3$, od koder bi sledilo $x_2 = x_3$, kar je v
    protislovju s predpostavko $x_2 \neq x_3$.

  \item Če velja $x_1 \neq x_2 \land x_1 \leq_P x_2 \land x_2 = x_3 \land y_2 \leq_Q y_3$: ker je $x_2 = x_3$ iz prvih dveh predpostavk sledi $x_1 \neq x_3 \land x_1 \leq_P x_3$.

  \item Če velja $x_1 = x_2 \land y_1 \leq_Q y_2 \land x_2 \neq x_3 \land x_2 \leq_P x_3$: ker je $x_1 = x_2$ iz zadnjih dveh predpostavk sledi $x_1 \neq x_3 \land x_1 \leq_P x_3$.

  \item Če velja $x_1 = x_2 \land y_1 \leq_Q y_2 \land x_2 = x_3 \land y_2 \leq_Q y_3$: torej je $x_1 = x_3$ ker je $=$ tranzitivna in $y_1 \leq_Q y_3$ ker je $\leq_Q$ tranzitivna.
  \end{enumerate}
  %
  Nazadnje preverimo še, da je $\leq_\mathrm{lex}$ linearna, če sta $\leq$ in $\leq_Q$ linearni. Naj bosta $(x_1,y_1), (x_2,y_2) \in P \times Q$. Dokazati želimo
  %
  \begin{equation*}
    (x_1, y_1) \preceq (x_2, y_2) \lor (x_2, y_2) \preceq (x_1, y_1).
  \end{equation*}
  %
  To je ekvivalentno disjunkciji
  % 
  \begin{align*}
    & (x_1 \neq x_2 \land x_1 \leq_P x_2) \lor {} \\
    & (x_1 = x_2 \land y_1 \leq_Q y_2) \lor {} \\
    & (x_2 \neq x_1 \land x_2 \leq_P x_1) \lor {} \\
    & (x_2 = x_1 \land y_2 \leq_Q y_1),
  \end{align*}
  %
  kar je ekvivalentno
  %
  \begin{align*}
    & (x_1 \neq x_2 \land (x_1 \leq_P x_2 \lor x_2 \leq_P x_1)) \lor {} \\
    &(x_1 = x_2 \land (y_1 \leq_Q y_2 \lor y_2 \leq_Q y_1)).
  \end{align*}
  %
  Ker sta $\leq_P$ in $\leq_Q$ linearni, je to ekvivalentno
  %
  \begin{equation*}
    (x_1 \neq x_2 \land \top) \lor (x_1 = x_2 \land \top),
  \end{equation*}
  %
  kar je ekvivalentno
  \begin{equation*}
    (x_1 \neq x_2) \lor (x_1 = x_2).
  \end{equation*}
  %
  To pa drži po zakonu o izključeni tretji možnosti. S tem je linearnost $\leq_\mathrm{lex}$, dokazana.
\end{dokaz}

\subsubsection{Vsota urejenosti}

Naj bosta $(P, \leq_P)$ in $(Q, \leq_Q)$ delni urejenosti. Na vsoti $P + Q$ lahko
definiramo urejenost $\leq_{+}$ s predpisom:
%
\begin{equation*}
  u \leq_{+} v \defiff
  \begin{aligned}[t]
    & (\some{x, y \in P} u = \inl(x) \land v = \inl(y) \land x \leq_P y) \lor {} \\
    & (\some{s, t \in Q} u = \inr(s) \land v = \inr(t) \land s \leq_Q t).
  \end{aligned}
\end{equation*}

\subsubsection{Zaporedna vsota urejenosti}

Naj bosta $(P, \leq_P)$ in $(Q, \leq_Q)$ delni urejenosti. Na vsoti $P + Q$ lahko definiramo urejenost $\leq_{\to}$ s predpisom:
%
\begin{equation*}
  u \leq_{\to} v \defiff
  \begin{aligned}[t]
    &(\some{x, y \in P} u = \inl(x) \land v = \inl(y) \land x \leq_P y) \lor {} \\
    &(\some{x \in P} \some{s \in Q} u = \inl(x) \land v = \inr(s)) \lor {} \\
    &(\some{s, t \in Q} u = \inr(s) \land v = \inr(t) \land s \leq_Q t).
  \end{aligned}
\end{equation*}
%
Torej so vsi elementi $P$ pred vsemi elementi $Q$. Zaporedna vsota linearnih urejenosti je linearna.


\subsubsection{Potenca urejenosti}

Naj bo $(P, \leq)$ delna urejenost in $A$ množica. Na eksponentni množici $P^A$ lahko definiramo urejenost $\preceq$ s predpisom:
%
\begin{equation*}
  f \preceq g \defiff \all{x \in A} f(x) \leq g(x).
\end{equation*}

\begin{naloga}
  Ali je $\preceq$ linearna, kadar je $\leq$ linearna?
\end{naloga}


\subsubsection{Delna urejenost, inducirana s šibko ureditvijo}

Naj bo $(P, \leq)$ šibka ureditev. Relacija $\sim$ na $P$, definirana s predpisom
%
\begin{equation*}
  x \sim y \defiff x \leq y \land y \leq x,
\end{equation*}
%
je ekvivalenčna relacija. Na kvocientu $P/{\sim}$ lahko definiramo relacijo $\preceq$ s
predpisom
%
\begin{equation*}
  [x] \preceq [y] \defiff x \leq y.
\end{equation*}
%
Treba je preveriti, da je relacija dobro definirana, saj smo uporabili predstavnike ekvivalenčnih razredov. Se pravi, ali velja
\begin{equation*}
  x \sim x' \land y \sim y' \lthen (x \leq y \liff x' \leq y') ?
\end{equation*}
%
Pa preverimo. Denimo, da velja $x, y, x', y' \in P$ in $x \sim x'$ in $y \sim y'$.
Torej velja
\begin{equation*}
  x \leq x' \land x' \leq x \land y \leq y' \land y' \land x.
\end{equation*}
%
Sedaj dokažimo $x \leq y \liff x' \leq y'$:
%
\begin{enumerate}
\item Če velja $x \leq y$ potem $x' \leq x \leq y \leq y'$.
\item Če velja $x' \leq y'$, potem $x \leq x' \leq y' \leq y$.
\end{enumerate}
%
Torej je $\preceq$ dobro definirana.

\begin{izjava}
  Relacija, ki je inducirana s šibko ureditvijo, je delna ureditev.
\end{izjava}

\begin{dokaz}
  Refleksivnost in tranzitivnost $\preceq$ sledita iz refleksivnosti in tranzitivnosti~$\leq$. Preverimo antisimetričnost: denimo, da velja $[x] \leq [y]$ in $[y] \leq [x]$. Tedaj velja $x \leq y$ in $y \leq x$, torej velja $x \sim y$ in $[x] = [y]$.
\end{dokaz}

\begin{primer}
  Obravnavajmo cela števila $\ZZ$ in deljivost $\mid$, ki je šibka
  ureditev. Za vse $k, m \in \ZZ$ velja
  \begin{equation*}
    k \sim m \liff k \mid m \land m \mid k \liff |k| = |m|.
  \end{equation*}
  %
  Torej je $\ZZ/{\sim} \cong \NN$, kjer izomorfizem preslika $[k] \mapsto |k|$. Delna ureditev na $\ZZ/{\sim}$ inducirana z deljivostjo je spet deljivost (ko jo prenesemo iz $\ZZ/{\sim}$ na $\NN$ s pomočjo izomorfizma).
\end{primer}


\subsection{Monotone preslikave}

\begin{definicija}
  Preslikava $f : P \to Q$ med delnima urejenostma $(P, {\leq_P})$ in $(Q, {\leq_Q})$ je
  \textbf{monotona} (ali \textbf{naraščajoča}), ko velja $\all{x, y \in P} x \leq_P y \lthen f(x) \leq_Q f(y)$.
\end{definicija}

\begin{definicija}
  Preslikava $f : P \to Q$ med delnima urejenostma $(P, \leq_P)$ in $(Q, \leq_Q)$ je
  \textbf{antitona} (ali \textbf{padajoča}), ko velja $\all{x, y \in P} x \leq_P y \lthen f(y) \leq_Q f(x)$.
\end{definicija}

\begin{opomba}
  V analizi ">monotona"< pomeni ">monotona ali antitona"<. To ni nič
  čudnega, ker ">dan"< tudi pomeni ">dan in noč">.
\end{opomba}

\begin{izrek}
  Kompozicija monotonih preslikav je monotona. Identiteta je monotona.
\end{izrek}

\begin{dokaz}
  Naj bosta $f : P \to Q$ in $g : Q \to R$ monotoni preslikavi med delnimi
  urejenostmi $(P, {\leq_P})$, $(Q, {\leq_Q})$ in $(R, {\leq_R})$. Če je $x \leq_P y$, potem je zaradi monotonosti $f$ tudi $f(x) \leq_Q f(y)$, nato pa je zaradi monotonosti $g$ spet $g(f(x)) \leq_R g(f(y))$. Identiteta je očitno monotona.
\end{dokaz}

\begin{primer}
  Primeri monotonih preslikav:
  \begin{enumerate}
    \item Konstantna preslikava je monotona.
    \item Seštevanje ${+} : \RR \times \RR \to \RR$ je monotona operacija glede na produktno ureditev na $\RR \times \RR$.
    \item Množenje ${\times} : \RR \times \RR \to \RR$ ni monotona operacija.
  \end{enumerate}
\end{primer}


\subsection{Meje}

\begin{definicija}
  Naj bo $(P, {\leq})$ delna urejenost, $S \subseteq P$ in $x \in P$:
  \begin{itemize}

  \item $x$ je \textbf{spodnja meja} podmnožice $S$, ko velja $\all{y \in S} x \leq y$,

  \item $x$ je \textbf{zgornja meja} podmnožice $S$, ko velja $\all{y \in S} y \leq x$,

  \item $x$ je \textbf{infimum} ali \textbf{največja spodnja meja} ali \textbf{natančna spodnja meja} podmnožice $S$, ko je spodnja meja $S$ in velja: za vse $y \in P$, če je $y$ spodnja meja
    $S$, potem je $y \leq x$,

  \item $x$ je \textbf{supremum} ali \textbf{najmanjša zgornja meja} ali \textbf{natančna zgornja meja} podmnožice $S$, ko je zgornja meja $S$ in velja: za vse $y \in P$, če je $y$ zgornja meja $S$, potem je $x \leq y$,

  \item $x$ je \textbf{minimalni element} podmnožice $S$, ko velja $x \in S$ in $\all{y \in S} y \leq x \lthen x = y$,

  \item $x$ je \textbf{maksimalni element} podmnožice $S$, ko velja $x \in s$ in
      $\all{x \in S} x \leq y \lthen x = y$,

  \item $x$ je \textbf{najmanjši} ali \textbf{prvi} element ali \textbf{minimum} podmnožice $S$, ko velja $x \in S$ in $\all{y \in S} x \leq y$,

  \item $x$ je \textbf{največji} ali \textbf{zadnji} element ali \textbf{maksimum} podmnožice $S$, ko velja $x \in S$ in $\all{y \in S} y \leq x$.
\end{itemize}
\end{definicija}

\begin{opomba}
  Minimalni element ni isto kot minimum (in maksimalni element ni isto kot maksimum).
\end{opomba}

Kadar govorimo o ">prvem elementu"< ali ">maksimalnem elementu"< in ne povemo, na
katero podmnožico se nanaša element, imamo običajno v mislih kar celotno delno
ureditev.

\begin{izrek}
  Naj bo $(P, {\leq})$ delna urejenost in $S \subseteq P$. Tedaj ima $S$ največ en
  infimum in največ en supremum, ki ju zapišemo $\inf S$ ter $\sup S$, kadar obstajata.
\end{izrek}

\begin{dokaz}
  Denimo, da sta $x$ in $y$ oba infimum $S$. Ker je $y$ spodnja meja za
  $S$ in $x$ njen infimum, velja $y \leq x$. Podobno velja $x \leq y$, torej $x = y$. Za
  supremum je dokaz podoben.
\end{dokaz}

\begin{primer}
  Supremum končne neprazne množice $S \subseteq \NN$ za relacijo deljivosti $\mid$
  je najmanjši skupni večkratnik elementov iz $S$. Infimum je največji skupni
  delitelj. Kaj pa, če je $S$ prazna ali neskončna?
\end{primer}

\subsection{Mreže}

\begin{definicija}
  Naj bo $(P, {\leq})$ delna urejenost:
  %
  \begin{enumerate}
  \item $(P, \leq)$ je \textbf{mreža}, ko imata vsaka dva elementa $x, y \in P$ infimum in supremum.

  \item $(P, \leq)$ je \textbf{omejena mreža}, ko ima vsaka končna podmnožica $P$ infimum in supremum.

  \item $(P, \leq)$ je \textbf{polna mreža}, ko ima vsaka podmnožica $P$ infimum in supremum.
  \end{enumerate}
  %
  Infimum in supremum elementov $x$ in $y$ pišemo $x \land y$ in $x \lor y$.
\end{definicija}

\begin{izrek}
  Delna urejenost $(P, {\leq})$ je omejena mreža natanko tedaj, ko ima
  najmanjši element in največji element, ter imata vsaka dva elementa infimum in supremum.
\end{izrek}

\begin{dokaz}
  Denimo, da je $(P, \leq)$ omejena mreža. Tedaj $P$ ima najmanjši element, namreč
  $\sup \emptyset$, in največji element, namreč $\inf \emptyset$. Infimum in supremum $x$ in $y$ sta seveda $\inf \set{x, y}$ in $\sup \set{x, y}$.

  Denimo, da ima $P$ najmanjši element $\bot_P$ in največji element $\top_P$, vsaka dva
  elementa pa imata infimum in supremum. Naj bo $S \subseteq P$ končna množica:
  %
  \begin{enumerate}
  \item če je $S = \emptyset$, potem je $\inf S = \top_P$ in $\sup S = \bot_P$,
  \item če je $S = \set{x_1, \ldots, x_n}$ za $n > 0$, potem je $\inf S = \inf \set{x_1, \ldots, x_{n-1}} \lor x_n$ in $\sup S = \sup \set{x_1, \ldots, x_{n-1}} \lor x_n$.
  \end{enumerate}
\end{dokaz}

\begin{primer}
  Primeri mrež:
  %
  \begin{enumerate}
  \item Množica $\two = \set{\bot, \top}$ je omejena mreža za relacijo $\lthen$.
  \item Relacija deljivosti na množici pozitivnih naravnih števil je omejena mreža.
  \item Potenčna množica $\pow{A}$, urejena z $\subseteq$, je polna mreža.
  \item Zaprti interval $[a,b]$, urejen z $\leq$, je polna mreža.
  \item Realna števila $R$, urejena z $\leq$,
  \end{enumerate}
\end{primer}


\chapter{Indukcija in dobra osnovanost}

\section{Dobra osnovanost}

\subsection{Indukcija na naravnih številih}

Poznamo že indukcijo na naravnih številih. Zapišemo jo lahko na dva načina,
kjer naslednika števila $n$ označimo $\suc{n}$:
%
\begin{enumerate}
\item Kot aksiom o predikatih na naravnih številih:
  %
  \begin{equation*}
  \phi(0) \land (\all{n \in \NN} \phi(n) \lthen \phi(\suc{n})) \lthen \all{m \in \NN} \phi(m)
  \end{equation*}

\item Kot lastnost podmnožic naravnih števil:
  %
  \begin{equation*}
    \all{S \in \pow{\NN}} 0 \in S \land (\all{k \in \NN} k \in S \lthen \suc{k} \in S) \lthen S = \NN
  \end{equation*}
\end{enumerate}
%
Uporabljali bomo verzijo s podmnožicami. Najprej jo predelajmo v ekvivalentno obliko:
%
\begin{align}
  &\all{S \in \pow{\NN}} 0 \in S \land (\all{k \in \NN} k \in S \lthen \suc{k} \in S) \lthen S = \NN \tag{$\liff$} \\
  &\all{S \in \pow{\NN}} 0 \in S \land (\all{m \in \NN} (\all{k \in \NN} \suc{k} = m \lthen k \in S) \lthen m \in S) \lthen S = \NN \tag{$\liff$} \\
  &\all{S \in \pow{\NN}} (\all{m \in \NN} (\all{k \in \NN} \suc{k} = m \lthen k \in S) \lthen m \in S) \lthen S = \NN. \notag
\end{align}
%
Kaj smo dosegli? Bazo indukcije in indukcijski korak smo združili v eno samo predpostavko
%
\begin{equation}
  \label{eq:ind-N}
  \all{m \in \NN} (\all{k \in \NN} \suc{k} = m \lthen k \in S) \lthen m \in S
\end{equation}
%
Če vstavimo $m \defeq 0$, dobimo:
%
\begin{align}
  &(\all{k \in \NN} \suc{k} = 0 \lthen k \in S) \lthen 0 \in S \tag{$\liff$} \\
  &(\all{k \in \NN} \bot \lthen k \in S) \lthen 0 \in S \tag{$\liff$} \\
  &(\all{k \in \NN} \top) \lthen 0 \in S \tag{$\liff$} \\
  &\top \lthen 0 \in S \tag{$\liff$} \\
  &0 \in S \notag
\end{align}
%
Če vstavimo $m \defeq \suc{n}$ dobimo:
%
\begin{align}
  &(\all{k \in \NN} \suc{k} = \suc{n} \lthen k \in S) \lthen \suc{n} \in S \tag{$\liff$} \\
  &(\all{k \in \NN} k = n \lthen k \in S) \lthen \suc{n} \in S \tag{$\liff$} \\
  &n \in S \lthen \suc{n} \in S \notag
\end{align}
%
To pa sta ravno običajna pogoja za indukcijo.

Ali lahko izrazimo indukcijo na naravnih številih tudi brez operacije naslednik?
Da, s pomočjo relacije $<$:
%
\begin{equation*}
    \all{S \in \pow{\NN}} (\all{m \in \NN} (\all{k \in \NN} k < m \lthen k \in S) \lthen m \in S) \lthen S = \NN
\end{equation*}
%
Temu principu pravimo tudi \textbf{krepka indukcija}, z besedami jo povemo takole: za podmnožico $S \subseteq \NN$ velja
$S = \NN$, če za vse $m \in \NN$ velja ">če so vsa števila manjša od $m$ v $S$, potem je tudi $m$ v $S$"<.

Denimo, da $S$ res ima dano lastnost. Ali je $0 \in S$? Da, ker za vse predhodnike $0$ velja, da
so $S$ (saj jih ni). Ali je $1 \in S$? Da, saj za vse predhodnike $1$ velja, da so v $S$. Ali je $2 \in
S$? Da, saj za vse predhodnike $2$ velja, da so v $S$. In tako naprej.


\subsection{Dobra osnovanost}

Princip indukcije na naravnih številih posplošimo, pri čemer izhajamo iz principa indukcije, izraženega s pomočjo lastnosti~\eqref{eq:ind-N}, v kateri relacijo ">neposredni predhodnik"< nadomestimo s splošno relacijo.

\begin{definicija}
  Relacija $R \subseteq A \times A$ je \textbf{dobro osnovana}, kadar velja
  %
  \begin{equation}
    \label{eq:ind-wf}%
    \all{S \in P(A)} (\all{y \in A} (\all{x \in A} x \rel{R} y \lthen x \in S) \lthen y \in S) \lthen S = A.
  \end{equation}
  %
  Množici $S \subseteq A$, ki zadošča pogoju
  %
  \begin{equation*}
  \all{y \in A} (\all{x \in A} x \rel{R} y \lthen x \in S) \lthen y \in S
  \end{equation*}
  %
  pravimo \textbf{$R$-progresivna} množica ali, da je $S$ \textbf{progresivna za $R$}.
\end{definicija}

Pogoj \eqref{eq:ind-wf} je \emph{indukcijski predpis} za dobro osnovano relacijo~$R$. Nekatere relacije temu predpisu zadoščajo in druge ne. Na primer, relacija ">neposredni predhodnik"< na $\NN$ mu zadošča, saj v tem primeru dobimo običajno indukcijo na~$\NN$.

\begin{primer}
  Preverimo, da je relacija ">neposredni predhodnik"< $P$ na množici $A = \set{0, 1, \ldots, 42}$ dobro osnovana.
  Natančneje, govorimo o relaciji
  %
  \begin{equation*}
    m \rel{P} n \defiff m + 1 = n.
  \end{equation*}
  %
  Naj bo $S \subseteq A$ progresivna množica, torej zadošča
  %
  \begin{equation*}
    \all{y \in A} (\all{x \in A} x + 1 = y \lthen x \in S) \lthen y \in S.
  \end{equation*}
  %
  Če vstavimo $y = 0$, dobimo
  %
  \begin{equation*}
    (\all{x \in A} x + 1 = 0 \lthen x \in S) \lthen 0 \in S,
  \end{equation*}
  %
  kar je ekvivalentno $0 \in S$. Torej je $0 \in S$. Nato vstavimo $y = 1$ in dobimo
  %
  \begin{equation*}
    (\all{x \in A} x + 1 = 1 \lthen x \in S) \lthen 1 \in S,
  \end{equation*}
  %
  kar se poenostavi v $0 \in S \lthen 1 \in S$. Ker smo že dokazali $0 \in S$, sledi tudi $1 \in S$. V naslednjem koraku vstavimo $y = 2$, poenostavimo in dobimo $1 \in S \lthen 2 \in S$, torej $2 \in S$. Tako nadaljujemo do $y = 42$ in ugotovimo, da res velja $S = A$. S tem smo pokazali, da je $P$ dobro osnovana. Seveda ni bistveno, da smo uporabili $42$.
\end{primer}

\subsection{Dvojiška drevesa}

Naravna števila $\NN$ so \textbf{induktivno definirana množica}. To pomeni, da elemente $\NN$
opredelimo s pravili, ki povedo, kako se gradi naravna števila:
%
\begin{itemize}
\item $0 \in \NN$,
\item če je $n \in \NN$, potem je $\suc{n} \in \NN$.
\end{itemize}
%
Množica $\NN$ vsebuje natanko tiste elemente, ki jih lahko zgradimo s pomočjo teh pravil:
%
\begin{equation*}
    0, 0^{+}, 0^{++}, 0^{+++}, 0^{++++}, \ldots
\end{equation*}
%
Tu sta $0$ in $\suc{{}}$ mišljena kot simbolni oznaki, podobno kot $\inl$ in $\inr$ v definiciji vsote množic. Dejstvo,
da $\NN$ vsebuje natanko tiste elemente, ki jih lahko zgradimo s pomočjo $0$ in $\suc{{}}$ ni nič drugega kot indukcija
na~$\NN$.

Podobno lahko definiramo tudi druge induktivne množice, ki tudi zadoščajo principu indukcije.
%
Na primer, \textbf{dvojiška drevesa} so induktivno definirana množica $\Tree$ s predpisoma:
%
\begin{itemize}
\item $\emptyTree \in \Tree$,
\item če je $t_1 \in \Tree$ in $t_2 \in \Tree$, potem je $\tree{t_1, t_2} \in \Tree$
\end{itemize}
%
Z besedami: drevo je bodisi prazno, bodisi je sestavljeno iz dveh \textbf{poddreves}. Ali znamo
našteti vsa drevesa, ali še bolje, jih narisati?
%
\begin{align*}
    & \emptyTree, \\
    & \tree{\emptyTree, \emptyTree} \\
    & \tree{\emptyTree, \tree{\emptyTree, \emptyTree}}, \\
    & \tree{\tree{\emptyTree, \emptyTree}, \emptyTree}, \\
    & \tree{\tree{\emptyTree, \emptyTree}, \tree{\emptyTree, \emptyTree}}, \\
    & \vdots
\end{align*}
%
Definirajmo relacijo $R \subseteq \Tree \times \Tree$ s predpisom:
%
\begin{equation*}
  t \rel{R} s \defiff \some{u \in \Tree} s = \tree{t, u} \lor s = \tree{u, t}.
\end{equation*}
%
To je relacija ">neposredno poddrevo"<. Je dobro osnovana, česar ne bomo dokazali, porodi pa naslednji princip indukcije za dvojiška drevesa.

\begin{izjava}[Indukcija za dvojiška drevesa]
  Naj bo $S \subseteq \Tree$ podmnožica dreves, za katero velja:
  %
  \begin{itemize}
  \item prazno drevo je v $S$,
  \item za vsa drevesa $t_1$ in $t_2$ velja: če je $t_1 \in S$ in $t_2 \in S$, potem je $\tree{t_1, t_2} \in S$.
  \end{itemize}
  %
  Tedaj je $S = \Tree$.
\end{izjava}

Princip povejmo še s pomočjo predikatov.

\begin{izjava}[Indukcija za dvojiška drevesa]
  Naj bo $\phi$ predikat na dvojiških drevesih, za katerega velja:
  %
  \begin{itemize}
  \item baza indukcije: $\phi(\emptyTree)$
  \item indukcijski korak: za vsa drevesa $t_1$ in $t_2$, če velja $\phi(t_1)$ in $\phi(t_2)$, potem
    $\phi(\tree{t_1, t_2})$.
  \end{itemize}
  %
  Tedaj $\all{t \in \Tree} \phi(t)$.
\end{izjava}

Kot vidimo, imamo v indukcijskem koraku \emph{dve} indukcijski predpostavki, ker ima vsako
sestavljeno drevo dve poddrevesi.


\subsubsection{Dobra osnovanost in padajoče verige}

Kako pa bi dobili kak proti-primer, se pravi, relacijo, ki ni dobra osnovanost? Poiskati
moramo kako lastnost, ki jo imajo vse dobre osnovanosti, nato pa relacijo, ki te lastnosti nima.

\begin{definicija}
  Naj bo $R \subseteq A \times A$ relacija na $A$. \textbf{Padajoča veriga} za relacijo $R$
	je zaporedje $a : \NN \to A$, za katerega velja $\all{i \in \NN} a_{i+1} \rel{R} a_i$.
\end{definicija}

Se pravi, da je padajoča veriga zaporedje, za katerega velja
%
\begin{equation*}
  \cdots a_4 \rel{R} a_3 \rel{R} a_2 \rel{R} a_1 \rel{R} a_0
\end{equation*}
%
\textbf{Cikel} za relacijo~$R$ je končna podmnožica $\set{a_0, \ldots, a_n} \subseteq A$ da velja
%
\begin{equation*}
  a_0 \rel{R} a_1 \rel{R} \cdots \rel{R} a_n \rel{R} a_0.
\end{equation*}
%
Iz cikla dobimo padajočo verigo, tako da cikel ponavljamo v nedogled:
%
\begin{equation*}
  \cdots \rel{R} a_0 \rel{R} \cdots \rel{R} a_n
         \rel{R} a_0 \rel{R} \cdots \rel{R} a_n \rel{R} a_0.
\end{equation*}

\begin{lema}
  V dobri osnovanosti ni ciklov in ni padajočih verig.
\end{lema}

\begin{dokaz}
  Dovolj je pokazati, da ni padajočih verig, saj iz cikla dobimo padajočo verigo.
  Denimo, da je $a : \NN \to A$ padajoča veriga za $R \subseteq A \times A$. Dokazali bomo, da $R$ ni dobro
  osnovana. Se pravi, da moramo poiskati $R$-progresivno podmnožico $S \subseteq A$, za katero velja
  $S \neq A$. Vzemimo $S \defeq A \setminus \set{ a_i \mid i \in \NN}$. Očitno velja $S \neq A$, saj 
  $a_0 \not\in S$. Preverimo, da je $S$ progresivna, se pravi, da je
  %
  \begin{equation*}
    \all{y \in A} (\all{x \in A} x \rel{R} y \lthen x \in S) \lthen y \in S.
  \end{equation*}
  %
  Naj bo $y \in A$ in denimo, da velja
  \begin{equation}
    \label{eq:verige}
    \all{x \in A} x \rel{R} y \lthen x \in S
  \end{equation}
  %
  Dokazati moramo $y \in S$. Obravnavamo dve možnosti:
  %
  \begin{itemize}
  \item če $y \in S$, potem seveda sledi $y \in S$.
	\item če $y \not\in S$, potem obstaja $i \in \NN$, da je $y = a_i$. Ker je $a_{i+1} \rel{R} a_i$, iz
    predpostavke~\eqref{eq:verige} sledi $y = a_i \in S$.
  \end{itemize}
  Torej v vsakem primeru velja $y \in S$.
\end{dokaz}

\begin{primer}
  Sedaj lahko zlahka priskrbimo kak proti-primer. Na primer, cela števila $\ZZ$ z relacijo $R \subseteq \ZZ \times \ZZ$
  %
  \begin{equation*}
    a \rel{R} b \defiff a + 1 = b
  \end{equation*}
  %
  niso dobro osnovana, ker imajo padajočo verigo
  %
  \begin{equation*}
    \cdots \rel{R} (-3) \rel{R} (-2) \rel{R} (-1) \rel{R} 0
  \end{equation*}
  %
  Prav tako ni dobro osnovana relacija $<$ na intervalu $[0,1]$, ker imamo padajočo verigo
  $n \mapsto 2^{-n}$.
\end{primer}

\section{Dobra urejenost}

Posplošimo sedaj še krepko indukcijo na naravnih številih. Tokrat bomo najprej posplošili
strogo urejenost $<$.

\subsection{Stroge urejenosti}

\begin{definicija}
  Relacija $R \subseteq A \times A$ je \textbf{stroga urejenost}, če je
  %
  \begin{itemize}
  \item irefleksivna: $\all{x \in A} \lnot (x \rel{R} x)$ in
  \item tranzitivna: $\all{x, y, z \in A} x \rel{R} y \land y \rel{R} z \lthen x \rel{R} z$.
  \end{itemize}
  %
  Stroga urejenost je \textbf{linearna}, če je še
  %
  \begin{itemize}
  \item sovisna: $\all{x, y \in A} x \rel{R} y \lor x = y \lor y \rel{R} x$.
  \end{itemize}
  %
  Za stroge urejenosti uporabljamo simbole $<$, $\subset$, $\prec$, $\sqsubset$ ipd.
\end{definicija}

Relaciji $<$ in $\leq$ na številih sta med seboj povezani, saj denimo za realna števila velja
%
\begin{equation*}
  x < y \iff x \leq y \land x \neq y
\end{equation*}
%
in
%
\begin{equation}
  \label{eq:leq-iff-lteq}
  %
  x \leq y \iff x < y \lor x = y
\end{equation}
%
To velja v splošnem. Stroga urejenost $<$ na množici $A$ porodi delno urejenost $\leq$ na $A$,
definirano s predpisom:
%
\begin{equation*}
    x \leq y  \defiff x = y \lor x \leq y.
\end{equation*}
%
V obratno smer, delna urejenost $\sqsubseteq$ določa strogo urejenost $\sqsubset$, definirano s predpisom
%
\begin{equation}
  \label{eq:leq-iff-neqlt}
  a \sqsubset b  \defiff  a \neq b \land a \sqsubseteq b.
\end{equation}
%
Seveda je treba preveriti naslednja dejstva, ki jih postimo za vajo:
%
\begin{itemize}
\item če je $<$ stroga urejenost, potem je $\leq$ definirana s \eqref{eq:leq-iff-lteq} delna urejenost
\item če je $\sqsubseteq$ delna urejenost, potem je $\sqsubset$ definirana s \eqref{eq:leq-iff-neqlt} stroga urejenost.
\end{itemize}
%
Tako lahko prehajamo med delno in strogo urejenostjo.

\subsection{Dobra ureditev}

\begin{definicija}
  Relacija je \textbf{dobra ureditev}, če je dobro osnovana in stroga linearna ureditev.
\end{definicija}

\begin{izrek}
  Relacija je dobra ureditev natanko tedaj, ko je dobro osnovana in sovisna.
\end{izrek}

\begin{dokaz}
  V eno smer je ekvivalenca očitna, zato dokažimo samo obratno smer. Denimo, da je
  $R \subseteq A \times A$ dobro osnovana in sovisna relacija. Dokazujemo, da je dobra ureditev, se pravi,
  da potrebujemo še irefleksivnost in tranzitivnost $R$.

  Relacija $R$ je irefleksivna: če bi veljalo $x \rel{R} x$ za $x \in A$, potem $R$ ne bi bila dobro
  osnovana, ker bi vsebovala padajočo verigo $\cdots x \rel{R} x \rel{R} x$.

  Relacija $R$ je tranzitivna: denimo, da velja $x \rel{R} y$ in $y \rel{R} z$. Dokazujemo $x \rel{R} z$. Ker je $R$
  sovisna, velja $x \rel{R} z$ ali $x = z$ ali $z \rel{R} x$. Pokažimo, da $x = z$ in $z \rel{R} x$ nista
  možna:
  %
  \begin{itemize}
  \item Če je $x = z$, potem velja $x \rel{R} y$ in $y \rel{R} x$, torej $x$ in $y$ tvorita cikel, a
    $R$ je dobro osnovana, zato to ni možno.
  \item Če velja $z \rel{R} x$, potem dobimo cikel $x \rel{R} y \rel{R} z \rel{R} x$, kar spet ni možno.
  \end{itemize}
\end{dokaz}

\begin{lema}
  \label{lem:nepr-min-veriga}%
  Denimo, da je $<$ stroga urejenost na neprazni množici~$B$. Če $B$ nima
  $\leq$-minimalnega elementa, potem ima padajočo verigo.
\end{lema}

\begin{dokaz}
  Denimo, da $B$ nima minimalnega elementa, torej
  %a
  \begin{equation*}
    \lnot \some{x \in B} \all{y \in B} y \leq x \lthen y = x.
  \end{equation*}
  %
  To je ekvivalentno
  %
  \begin{equation*}
    \all{x \in B} \some{y \in B} y \leq x \land y \neq x
  \end{equation*}
  %
  kar je ekvivalentno
  \begin{equation}
    \label{eq:lema-min-elem}
    \all{x \in B} \some{y \in B} y < x.
  \end{equation}
  %
  Padajočo verigo $b : \NN \to B$ definiramo z zaporedjem izbir: ker je $B$ neprazna, lahko izberemo
  neki element $b(0) \in B$. Denimo, da smo za neki $i \in \NN$ že izbrali elemente $b(0), \ldots, b(i)$
  tako, da velja
  %
  \begin{equation*}
    b(i) < b(i-1) < \ldots < b(1) < b(0).
  \end{equation*}
  %
  Ker $B$ nima minimalnega elementa, $b(i)$ ni minimalni, torej po \eqref{eq:lema-min-elem} obstaja tak $y \in B$, da je $y < b(i)$. Torej lahko izberemo $b(i+1) \in B$, da velja $b(i+1) < b(i)$.
\end{dokaz}

\begin{opomba}
  V zgornjem dokazu smo uporabili \emph{aksiom odvisne izbire}, ki je poseben primer
  aksioma izbire in o katerem bomo še govorili.
\end{opomba}

\begin{izrek}
  \label{izr:dobr-osn-iff}
  Naj bo $\sqsubset$ relacija na $A$. Tedaj so ekvivalentne naslednje izjave:
  %
  \begin{enumerate}
  \item \label{it:dobr-osn-1}%
    $\sqsubset$ je dobro osnovana,
  \item \label{it:dobr-osn-2}%
    vsaka \emph{neprazna} $S \subseteq A$ ima $\sqsubseteq$-minimalni element,
  \item \label{it:dobr-osn-3}%
    $\sqsubset$ nima padajoče verige.
  \end{enumerate}
\end{izrek}

\begin{dokaz}
  ($1 \lthen 2$)
  %
  Denimo, da je $S \subseteq A$ neprazna. Če uporabimo \eqref{it:dobr-osn-1} na $A \setminus S$ dobimo
  %
  \begin{equation*}
    (\all{y \in A} (\all{x \in A} x \sqsubset y \lthen x \in A \setminus S) \lthen y \in A \setminus S) \lthen A \setminus S = A.
  \end{equation*}
  %
  Ker je $S$ neprazna, dobimo zaporedje ekvivalentnih izjav:
  \begin{align*}
    &(\all{y \in A} (\all{x \in A} x \sqsubset y \lthen x \in A \setminus S) \lthen y \in A \setminus S) \lthen \bot
    \tag{$\liff$} \\
    &\lnot (\all{y \in A} (\all{x \in A} x \sqsubset y \lthen x \in A \setminus S) \lthen y \in A \setminus S)
    \tag{$\liff$} \\
    &\some{y \in A} (\all{x \in A} x \sqsubset y \lthen x \in A \setminus S) \land y \not\in A \setminus S
    \tag{$\liff$} \\
    &\some{y \in A} (\all{x \in A} x \sqsubset y \lthen x \not\in S) \land y \in S
    \tag{$\liff$} \\
    &\some{y \in S} \all{x \in A} x \sqsubset y \lthen x \not\in S
    \tag{$\liff$} \\
    &\some{y \in S} (\all{x \in A} x \sqsubset y \lthen x \not\in S) \notag
  \end{align*}
  %
  Torej obstaja element $y \in S$ z lastnostjo, da pod njim ni nobenega elementa iz
  $S$, kar pa pomeni, da je $y$ iskani minimalni element.

  ($2 \lthen 3$) Denimo, da je $a : \NN \to A$ padajoča veriga. Tedaj slika $\set{ a(n) \mid n \in \NN }$ ne bi imela
  minimalnega elementa, v nasprotju z \eqref{it:dobr-osn-2}.

  ($3 \lthen 1$) Denimo, da je $S \subseteq A$ progresivna. Trdimo, da množica $C \defeq A \setminus S$ nima
  minimalnega elementa. Če bi bil $c \in C$ minimalni v $C$, bi to pomenilo
  %
  \begin{equation*}
    \all{x \in A} x \sqsubset c \lthen x \not\in C,
  \end{equation*}
  %
  kar je ekvivalentno
  %
  \begin{equation*}
    \all{x \in A} x \sqsubset c \lthen x \in S.
  \end{equation*}
  %
  Ker je $S$ progresivna, od tod sledi $c \in S$, kar ni mogoče.
  %
  Dokazati moramo, da je $C$ prazna. Če ne bi bila, bi lahko uporabili lemo~\ref{lem:nepr-min-veriga} in dobili padajočo verigo v $A$, kar je v nasprotju s \eqref{it:dobr-osn-3}.
\end{dokaz}

\begin{izrek}
  \label{izr:dobra-urejenost-karakterizacija}
  Naj bo $\sqsubset$ stroga urejenost na $A$. Tedaj so ekvivalentne naslednje izjave:
  %
  \begin{enumerate}
  \item[(1)] $\sqsubset$ je dobro urejena,
  \item[(2)] vsaka \emph{neprazna} množica $S \subseteq A$ ima $\sqsubset$-prvi element: to je tak $x \in S$, da velja
    $\all{y \in S} x \neq y \lthen x \sqsubset y$.
  \item[(3)] $A$ nima $\sqsubset$-padajoče verige in $\sqsubset$ je sovisna.
  \end{enumerate}
\end{izrek}

\begin{dokaz}
  Za nalogo predelajte dokaz prejšnjega izreka v dokaz tega izreka.
\end{dokaz}

\begin{primer}
  Primeri dobro urejenih množic:
  %
  \begin{enumerate}
  \item Končna množica $\set{0, \ldots, n}$ urejena z relacijo $<$.

  \item Naravna števila $\NN$ urejena z relacijo $<$.

  \item Če sta $(P, \leq_P)$ in $(Q, \leq_Q)$ dobri urejenosti, potem je dobro urejena tudi $P + Q$ z relacijo
    $\sqsubseteq$, ki~$P$ postavi pred~$Q$:
    %
    \begin{equation*}
      u \sqsubseteq v \defiff
      \begin{aligned}[t]
        &(\some{x \in P} \some{y \in Q} u = \inl(x) \land v = \inr(y)) \lor {}\\
        &(\some{x \in P} \some{y \in P} u = \inl(x) \land v = \inl(y) \lor x \leq_P y) \lor {}\\
        &(\some{x \in Q} \some{y \in Q} u = \inr(x) \land v = \inr(y) \lor x \leq_Q y).
      \end{aligned}
    \end{equation*}

  \item
    S prejšnjim primerom lahko seštevamo dobre urejenosti, na primer $\NN + \set{0, 1, 2}$ je dobra
    urejenost
    %
    \begin{equation*}
      \inl{0} < \inl{1} < \inl{2} < \cdots < \inr{0} < \inr{1} < \inr{2}.
    \end{equation*}
  \end{enumerate}
\end{primer}

\section{Ordinalna števila}
\label{sec:ordinalna-tevila}

Dobra urejenost na množici~$A$ postavi njene elemente v vrsto (strogo linearno urejenost), ki nima padajočih verig.
Končno množico lahko dobro uredimo na več načinov, na primer elemente $\set{0, 1, 2, \ldots, n-1}$ lahko postavimo v vrsto na $n!$ načinov. Množico vseh naravnih števil lahko postavimo v vrsto brez padajočih verig vsaj na tri načine,
%
\begin{equation*}
  0, 1, 2, 3, 4, 5, \ldots, n, n + 1, \ldots
\end{equation*}
%
in
%
\begin{equation*}
  1, 0, 3, 2, 5, 4, \ldots, 2 n + 1, 2 n, \ldots
\end{equation*}
%
in
%
\begin{equation*}
  0, 2, 4, 6, 8, \ldots, 1, 3, 4, 5, \ldots
\end{equation*}
%
Zdi se, da sta prvi in drugi način ">isti tip"< urejenosti in se razlikujeta od tretjega. Res, v tretji vrsti ima $1$ neskončno predhodnikov, v prvi in drugi pa takega elementa ni. Govorimo o naslednjem pojmu.

\begin{definicija}
  Dobri ureditvi $(P, {\leq_P})$ in $(Q, {\leq_Q})$ \textbf{izomorfni}, če obstajata monotoni preslikavi $f : P \to Q$ in $Q \to P$, da velja $f \circ g = \id[Q]$ in $g \circ f = \id[P]$.
\end{definicija}

Seveda je izomorfnost ekvivalenčna relacija, ki je definirana na pravem razredu vseh dobrih urejenosti. 
Koristno bi bilo imeti kak izbor predstavnikov zanjo, saj bi lahko z njimi merili ">dolžino"< dobre urejenosti. Takim predstavnikom pravimo \textbf{ordinalna števila}. A kako bi jih dobili? Pri 19.~letih je \href{https://en.wikipedia.org/wiki/John_von_Neumann}{John von Neumann} predlagal:
%
\begin{quote}
  \emph{">Ordinalno število je množica svojih predhodnikov, urejeno z relacijo $\in$."<}
\end{quote}
%
Poglejmo, kako deluje njegova ideja:
%
\begin{itemize}

\item Končna ordinalna števila sovpadajo z naravnimi števili:
  %
  \begin{align*}
    0 &\defeq \emptyset \\
    1 &\defeq \set{0} = \set{\emptyset} \\
    2 &\defeq \set{0, 1} = \set{\emptyset, \set{\emptyset}} \\
    3 &\defeq \set{0, 1, 2} = \set{\emptyset, \set{\emptyset}, \set{\emptyset, \set{\emptyset}}} \\
      &\vdots
  \end{align*}

\item Množica vseh končnih ordinalnih števil je prvo neskončno ordinalno število
  %
  \begin{equation*}
    \omega = \set{0, 1, 2, 3, \ldots}.
  \end{equation*}

\item Številu $\omega$ sledijo
  %
  \begin{align*}
    \omega + 1 &\defeq \set{0, 1, 2, \ldots, \omega} \\
    \omega + 2 &\defeq \set{0, 1, 2, \ldots, \omega, \omega + 1} \\
    \omega + 3 &\defeq \set{0, 1, 2, \ldots, \omega, \omega + 1, \omega + 2} \\
               &\vdots \\
    \omega + \omega &\defeq \set{0, 1, 2, \ldots, \omega, \omega + 1, \omega + 2, \ldots} \\
    \omega + \omega + 1 &\defeq \set{0, 1, 2, \ldots, \omega, \omega + 1, \omega + 2, \ldots, \omega + \omega} \\
               &\vdots
  \end{align*}
\end{itemize}

\begin{naloga}
  Kako bi si predstavljali naslednje ordinale: $\omega + \omega + \omega$, $\omega \cdot \omega$, $\omega^3$, $\omega^\omega$?
\end{naloga}

Von Neumann je imel pravo idejo, a pušča kanček dvoma, ker je definicija ordinalnega števila \emph{rekurzivna} (se nanaša sama nase). Če se malce potrudimo, da lahko von Neumannove ordinale opredelimo neposredno.

\begin{definicija}
  Množica $z$ je \textbf{tranzitivna}, če iz $x \in y$ in $y \in z$ sledi $x \in z$.
\end{definicija}

\noindent
%
Poimenovanje je smiselno, saj je pogoj v definiciji ravno tranzitivnost relacije~$\in$.
Ekvivalentno lahko pogoj izrazimo takole: množica $z$ je tranzitivna, če iz $y \in z$ sledi $y \subseteq z$.

\begin{primer}
  Množica $\set{\emptyset, \set{\emptyset}, \set{\set{\emptyset}}}$ je tranzitivna, niso pa vsi njeni elementi tranzitivne množice, saj $\set{\set{\emptyset}}$ ni tranzitivna, ker $\emptyset \in \set{\emptyset} \in \set{\set{\emptyset}}$ vendar $\emptyset \not\in \set{\set{\emptyset}}$.
\end{primer}

\begin{naloga}
  Dokažite, da so ekvivalentni pogoji:
  %
  \begin{enumerate}
  \item $A$ je tranzitivna množica,
  \item $\bigcup A \subseteq A$,
  \item $A \subseteq \pow{A}$.
  \end{enumerate}
\end{naloga}

Sedaj lahko zapišemo definicijo von Neumannovih ordinalov, ki ni rekurzivna.

\begin{definicija}
  \label{def:von-neuman-ordinal}
  \textbf{(Von Neumannov) ordinal} je tranzitivna množica, ki je z relacijo $\in$ dobro urejena.
\end{definicija}

Razred vseh von Neumannovih ordinalov označimo z $\On$ (v angleščini ">ordinal number"<). To je pravi razred, česar ne bomo dokazali. Kogar zanima dokaz, naj poišče ">Burali-Fortijev paradoks"<, ki je celo starejši od Russellovega paradoksa.

\begin{naloga}
  Poiščite množico, ki \emph{ni} tranzitivna in je dobro urejena z relacijo $\in$.
\end{naloga}

Ali definicija~\ref{def:von-neuman-ordinal} res sovpada z idejo, da je ordinal množica svojih prednikov? To potrjuje naslednja izjava.

\begin{izjava}
  Če je $\alpha$ ordinal in $\beta \in \alpha$, potem je $\beta$ ordinal.
\end{izjava}

\begin{dokaz}
  Ker je $\alpha$ tranzitivna množica, je $\beta \subseteq \alpha$, zato je $\beta$ z relacijo~$\in$ dobro urejen. Dokazati moramo še, da je $\beta$ tranzitivna množica. Denimo, da je $\gamma \in \beta$.
  Tedaj je $\gamma \in \alpha$ in ker je $\alpha$ z $\in$ linearno urejen, velja bodisi $\gamma \in \beta$ bodisi $\gamma = \beta$ bodisi $\beta \in \gamma$. A ker druga in tretja možnost ne prideta v poštev, saj bi dobili cikel $\gamma \in \beta \in \gamma$, velja prva, kar smo želeli dokazati.
\end{dokaz}

\begin{naloga}
  V zgornjem dokazu smo uporabili naslednje dejstvo: če je $(P, {<})$ dobra ureditev in $Q \subseteq P$, tedaj je $Q$ z relacijo $<$ zoženo na~$Q$ tudi dobra ureditev. Zapišite dokaz.
\end{naloga}

Brez dokaza navedimo, da so von Neumannovi ordinali izbor predstavnikov za dobre urejenosti.

\begin{izrek}
  Vsaka dobra ureditev je izomorfna natanko enemu von Neumannovemu ordinalu.
\end{izrek}



\chapter{Moč množic}

V tej lekciji bomo govorili o velikosti množic, končnih množicah in neskončnih množicah.

\section{Aksiom odvisne izbire}

Kasneje bom potrebovali inačico aksioma izbire, ki se glasi:

\begin{aksiom}[Odvisna izbira]
  Naj bo $A$ neprazna množica in $R \subseteq A \times A$ celovita relacija, se pravi
  $\all{x \in A} \some{y \in A} x \, R \, y$.
  %
  Tedaj obstaja tako zaporedje $a : \NN \to A$, da za vse $n \in N$ velja $a_n \, R \,  a_{n+1}$.
\end{aksiom}

Aksiom odvisne izbire sledi iz aksioma izbire, česar tu ne bomo dokazali.

Aksiom odvisne izbire se v praksi uporabi, kadar želimo konstruirati zaporedje $a : \NN \to A$, pri čemer sta izpolnjena dva pogoja:
%
\begin{enumerate}
\item za vsak člen zaporedja $a_n$ imamo na voljo eno ali več izbir,
\item izbire za člen $a_{n+1}$ so odvisne od tega, kaj smo izbrali za $a_n$.
\end{enumerate}
%
Primer uporabe bomo videli v nadaljevanju.

\section{Končne množice}

Kako bi definirali pojem ">končna množica"<?

\begin{definicija}
  Za vsako naravno število $n \in \NN$, naj bo
  \textbf{standardna končna množica} $[n] = \set{k \in \NN \such k < n}$.
\end{definicija}

\noindent
Torej velja
%
\begin{align*}
  [0] &= \{\} \\
  [1] &= \{0\} \\
  [2] &= \{0, 1\} \\
  [3] &= \{0, 1, 2\} \\
      &\vdots
\end{align*}

\begin{definicija}
  Množica je \textbf{končna}, če je izomorfna kaki standardni končni množici.
\end{definicija}


Velja naslednje (ne bomo dokazali): če je $A \iso [m]$ in $A \iso [n]$, potem je $m = n$. Torej za končno
množico~$A$ obstaja natanko en $n \in \NN$, da velja $A \iso [n]$. Temu $n$ pravimo \textbf{moč} množice $A$,
saj nam pove, koliko elementov ima $A$. Moč končne množice $A$ označimo z $|A|$.

Za moči končnih množic velja
%
\begin{align*}
  |[n]| &= n, \\
  |A \times B| &= |A| \times |B|, \\
  |A + B| &= |A| + |B|, \\
  |B^A| &= |B|^{|A|}.
\end{align*}
%
Zgornje enačbe je treba razumeti pravilno: na levi nastopajo $\times$, $+$ in potenciranje kot operacije na množicah, na desni pa kot operacije na naravnih številih.

Za unijo velja \textbf{pravilo vključitve in izključitve}:
%
\begin{equation*}
 |A \cup B| = |A| + |B| - |A \cap B|.
\end{equation*}
%
Pravilo se tako imenuje, ker smo pri štetju elementov $A \cup B$ \emph{vključili} elemente $A$ in $B$, nato pa \emph{izključili} elemente preseka $A \cap B$, da jih ne bi šteli dvakrat.
%
Pravilo vključitve in izključitve za tri množice se glasi
%
\begin{equation*}
  |A \cup B \cup C| = |A| + |B| + |C| - |A \cap B| - |B \cap C| - |C \cap A| + |A \cap B \cap C|.
\end{equation*}

\begin{vaja}
  Zapišite pravilo vključitve in izključitve za unijo $A_1 \cup A_2 \cup \cdots \cup A_n$.
\end{vaja}


\section{Neskončne množice}

\begin{definicija}
  Množica je \textbf{neskončna}, če ni končna.
\end{definicija}

\begin{izrek}
  Množica $A$ je neskončna natanko tedaj, ko obstaja injektivna preslikava $\NN \to A$.
\end{izrek}

\begin{proof}

($\lthen$)
%
Denimo, da $A$ ni končna.
Injektivno preslikavo $e : \NN \to A$ definiramo s pomočjo aksioma odvisne izbire.
Ker $A$ ni izomorfna $[0]$, ni prazna, torej obstaja $e(0) \in A$.
Denimo, da smo že definirali $e$ kot injektivno preslikavo $[n] \to A$.
Tedaj jo lahko razširimo na injektivno preslikavo $e : [n+1] \to A$ takole: ker $e$ ni surjektivna (če bi bila, bi veljalo $A \iso [n]$ in $A$ bi bila končna), obstaja $x \in A$, ki ni v sliki $e$.
Sedaj \emph{izberemo} $e(n) \in A$, ki ni v sliki.
Tako dobimo $e : \NN \to A$, ki je injektivna.

($\Leftarrow$)
%
Denimo, da obstaja injektivna preslikava $e : \NN \to A$.
Če bi za neki $n$ veljalo $A \iso [n]$, bi imeli izomorfizem $f : A \to [n]$.
Tedaj bi bil kompozitum $f \circ e : \NN \to [n]$ injektivna preslikava, ta pa ne obstaja (dokaz opustimo).
\end{proof}

\subsection{Moč množic}

Tudi neskončnim množicam želimo prirediti moč, se pravi, neko mero velikosti. Preden pa nam bo to uspelo, se najprej naučimo primerjati velikost množic, ne da bi pri tem govorili o ">številu elementov"<.

\begin{definicija}
  Množici $A$ in $B$ imata enako moč, sta \textbf{ekvipolentni}, kadar sta izomorfni.
\end{definicija}

Ekvipolentnost in izomorfnost sta torej sinonima, ki pa se uporabljata v različnih situacijah. O ekvipolentnosti govorimo, ko imamo v mislih velikost množic ali število elementov. Izomorfnost je širši pojem, ki se uporablja tudi v algebri, topologiji in povsod, kjer imamo opravka z matematičnimi strukturami, in pomeni ">enakovredna struktura"<.

Spomnimo se, da je izomorfnost in torej tudi ekvipolentnost ekvivalenčna relacija.
Torej lahko tvorimo ekvivalenčne razrede glede na ekvipolentnost: vsaki množici $A$ priredimo razred vseh množic, ki so jih ekvipolentne:
%
\begin{itemize}
\item $[\emptyset]_{\iso} = \{ \emptyset \}$,
\item $[\set{ \unit }]_{\iso}$ je \emph{pravi razred} vseh enojcev,
\item $[\set{0, 1}]_{\iso}$ je \emph{pravi razred} vseh množic z dvema elementoma,
\item itd.
\end{itemize}
%
Dejstvo, da so razredi glede na izomorfnost pravi razredi in ne množice, je precej nerodna reč, saj z njimi ne moremo udobno delati (potrebovali bi ">super razrede"<, katerih elementi so razredi).
Izognemo se jim tako, da namesto z razredi delamo z izborom predstavnikov.

Pravzaprav smo ta trik že uporabili, ko smo govorili o moči končnih množic, ko smo za predstavnike ekvipolnenčnih razredov končnih množic izbrali standardne končne množice. Le-te nam lahko služijo kot ">števila"<, s katerimi opišemo moči končnih množic, saj med standardno končno množico $[n]$ in številom $n$ ni bistvene razlike. (Še več, kasneje bomo videli, da lahko naravna števila obravnavamo tako, da dejansko so standardne končne množice!)

Kako bi torej izbrali predstavnike razredov za ekvipolentnost za vse množice?
Če bi nam to uspelo, bi take predstavnike lahko uporabili kot števila, imenujejo se \textbf{kardinalna števila}, s katerimi bi merili moč množic.

\begin{definicija}
  \textbf{Kardinalno število} je tako ordinalno število $\kappa$, za katerega velja $|\alpha| < |\kappa|$ za vse $\alpha \in \kappa$.
\end{definicija}

\begin{zgled}
  Tu ne bomo dokazali, da je vsaka množica ekvipolentna natanko enemu kardinalnemu številu. Raje si poskušajmo predstavljati kardinalna števila:
  % 
  \begin{itemize}
  \item Končni ordinali, ki so seveda kar naravna števila, so kardinalna števila, saj je naravno število strogo večje od
    svojih predhodnikov.
  \item Ordinal $\omega = \NN = \set{0, 1, 2, \ldots}$, ki vsebuje vse končne ordinale, je kardinalno število. Označujemo ga tudi z $\aleph_0$.
  \item Ordinal $\omega + 1 = \set{0, 1, 2, \ldots, \omega}$ \emph{ni} kardinalno število, saj je ekvipolenten~$\omega$. Prav tako so ordinali
    % 
    \begin{equation*}
      \omega + 2, \omega + 3, \ldots, \omega + \omega, \ldots, \omega + \omega + \omega, \ldots, \omega^2, \omega^3
    \end{equation*}
    % 
    vsi ekvipolentni $\omega$, zato niso kardinali. Pravzaprav si je precej težko predstavljati ordinal, katerega moč je strogo večja od $\omega$.
  \end{itemize}
\end{zgled}

Vsaki množici $A$ torej priredimo nekega predstavnika razreda $[A]_{\iso}$, ki ga označimo $|A|$ in ga imenujemo \textbf{moč} množice~$A$. Za končne množice so to kar naravna števila, za splošne množice pa so to kardinalna števila.

Moči množic lahko primerjamo med seboj, čeprav ne vemo, kaj točno naravna števila so!

\begin{definicija}
  Naj bosta $A$ in $B$ poljubni množici. Pravimo:
  %
  \begin{enumerate}
  \item $A$ ima enako moč kot $B$, pišemo $|A| = |B|$, ko obstaja bijektivna preslikava $A \to B$.
  \item $A$ ima moč manjšo ali enako $B$, pišemo $|A| \leq |B|$, ko obstaja injektivna preslikava $A \to B$.
  \item $A$ ima moč manjšo kot $B$, pišemo $|A| < |B|$, če velja $|A| \leq |B|$ in $|A| \neq |B|$.
  \end{enumerate}
\end{definicija}

\begin{izrek}
  \label{izr:leq-iff-empty-or-onto}
  $|A| \leq |B|$ natanko tedaj, ko je $A = \emptyset$ ali obstaja surjekcija $B \to A$.
\end{izrek}

\begin{proof}
  Denimo, da je $f : A \to B$ injektivna in $A \neq \emptyset$. Torej obstaja neki $a \in A$.
  Definiramo preslikavo $g : B \to A$ takole:
  %
  \begin{equation*}
    g(y) = x  \defiff f(x) = y \lor (y \not\in f_{*}(A) \land x = a).
  \end{equation*}
  %
  Povedano malo drugače:
  %
  \begin{equation*}
    g(y) =
    \begin{cases}
      f^{-1}(y) & \text{če $y \in f_{*}(A)$,} \\
      a         & \text{če $y \not\in f_{*}(A)$.}
    \end{cases}
  \end{equation*}
  %
  Ker velja $g \circ f = \id[A]$, je $g$ retrakcija in zato surjektivna.

  Obratno, denimo, da je $A$ prazna ali obstaja surjekcija $f : B \to A$. Če je $A$
  prazna, je edina preslikava $\emptyset \to B$ injektivna. Če je $f : B \to A$ surjektivna,
  ima prerez (zakaj?), ki je injektivna preslikava.
\end{proof}


\subsection{Cantorjev izrek}

\begin{izrek}[Cantor]
  $|A| < |\pow{A}|$.
\end{izrek}

\begin{proof}
  Najprej dokažimo $|A| \leq |\pow{A}|$. Iščemo injektivno preslikava $f : A \to \pow{A}$. Vzemimo $f(x) = \{x\}$. Zlahka preverimo, da je $f$ res injektivna.

  Sedaj dokazujemo, da ne obstaja bijekcija $A \to \pow{A}$. Dokazali bomo, da ne obstaja surjekcija $A \to \pow{A}$, kar zadostuje. Denimo, da je $g : A \to \pow{A}$ poljubna preslikava. Trdimo, da $g$ ni surjekcija. Res, podmnožica
  %
  \begin{equation*}
    S = \set{x \in A \mid x \not\in g(x) }
  \end{equation*}
  %
  ni v sliki $g$. Če bi bila, bi za neki $y \in A$ veljalo $g(y) = S$, a to bi vodilo v protislovje:
  % 
  \begin{itemize}
  \item velja $y \not\in S$: če $y \in S$ potem $y \not\in g(y) = S$ po definiciji $S$,
  \item velja $\lnot (y \not\in S)$: če $y \not\in S$ potem $y \not\in g(y) = S$.
  \end{itemize}
\end{proof}


\subsection{Števne in neštevne množice}

Kot smo že povedali, moč množice $\NN$ označimo z $\aleph_0$.

\begin{definicija}
  Množica $A$ je \textbf{števna}, če velja velja $|A| \leq \aleph_0$.
\end{definicija}

\begin{definicija}
  Množica $A$ je \textbf{neštevna}, če ni števna.
\end{definicija}

\begin{izrek}
  Za vsako množico $A$ so ekvivalentne naslednje izjave:
  %
  \begin{enumerate}
  \item $A$ je števna.
  \item Obstaja injektivna preslikava $A \to \NN$.
  \item $A$ je prazna ali obstaja surjektivna preslikava $\NN \to A$.
  \item Obstaja surjektivna preslikava $\NN \to \one + A$.
  \item $A$ je končna ali izomorfna $\NN$.
  \end{enumerate}
\end{izrek}

\begin{proof}
%
\begin{itemize}
\item[$(1 \lthen 2)$]
%
Če je $A$ števna, velja $|A| \leq \aleph_0 = |\NN|$, torej obstaja injektivna $A \to
\NN$ po definiciji relacije $\leq$.

\item[$(2 \lthen 3)$]
%
To sledi neposredno iz Izreka~\ref{izr:leq-iff-empty-or-onto}.

\item[$(3 \lthen 4)$]
%
Denimo, da je $A$ prazna ali obstaja surjektivna preslikava $\NN \to A$:
%
\begin{enumerate}
\item
  Če je $A = \emptyset$, potem seveda obstaja surjektivna preslikava $\NN \to \one + A$, in sicer
  $n \mapsto \inl \unit$.
\item 
  Če obstaja surjektivna preslikava $f : \NN \to A$, potem lahko definiramo surjektivno
  preslikavo $g : \NN \to \one + A$ s predpisom
  %
  \begin{equation*}
    g(n) =
    \begin{cases}
      \inl \unit      &\text{če $n = 0$,}\\
      \inr (f(n-1))   &\text{če $n > 0$.}
    \end{cases}
  \end{equation*}
\end{enumerate}

\item[$(4 \lthen 5)$]
%
Denimo, da obstaja surjektivna preslikava $r : \NN \to \one + A$.
Dokazali bomo, da je $A$ izomorfna $\NN$, če ni končna.
Predpostavimo torej, da $A$ ni končna.
Preslikava $r$ ima prerez $s : \one + A \to \NN$, ki je seveda injektivna preslikava.
Preslikav $s \circ \inr : A \to \NN$ je kompozitum injektivnih preslikav, zato je injektivna.
Ker $A$ ni končna, obstaja tudi injektivna preslikava $\NN \to A$.
Po izreku Cantor-Schröder-Bernstein, ki ga bomo dokazali spodaj, je torej $A$ izomorfna $\NN$.

\item[$(5 \lthen 1)$]
%
Če je $A$ končna, je števna, ker očitno velja $A = |[n]| \leq |\NN| = \aleph_0$.
Če je $A$ izomorfna $\NN$, potem velja $|A| = |\NN| \leq |\NN| = \aleph_0$.
\end{itemize}
\end{proof}

\begin{izrek}
  $\NN \times \NN \iso \NN$.
\end{izrek}

\begin{proof}
  Za vajo, poiščite dokaz v zapiskih iz analize ali na internetu.
\end{proof}

\begin{definicija}
  \textbf{Števna družina} je družina $A : I \to \Set$, katere indeksna množica~$I$ je števna.
\end{definicija}

\begin{izrek}
  Unija števne družine števnih množic je števna.
\end{izrek}

\begin{proof}
  Izrek bomo dokazali le za primer, ko je indeksna množica~$\NN$.
  %
  Najprej obravnavajmo unijo družine $A : \NN \to \Set$, kjer je $A_n$ števna za vse $n \in \NN$.
  Za vsak $n \in \NN$ obstaja surjektivna preslikava $\NN \to A_n + \one$. Po aksiomu izbire obstaja funkcija izbire
  %
  \begin{equation*}
    e \in \prod_{n \in \NN} \set{f : \NN \to A_n + \one \such \text{$f$ surjekcija}}.
  \end{equation*}
  %
  Definiramo $e' : \NN \times \NN \to \one + \bigcup_{n \in \NN} A_n$ s predpisom
  %
  \begin{equation*}
    e'(n, k) = e(n)(k).
  \end{equation*}
  %
  Trdimo, da je $e'$ surjekcija iz $\NN \times \NN$ na $\one + \bigcup_{n \in \NN} A_n$.
\end{proof}


\subsection{Cantor-Schröder-Bernsteinov izrek in zakon trihotomije}

\begin{izrek}[Cantor-Schröder-Bernstein]
  Če obstajata injektivni preslikava $A \to B$ in $B \to A$, potem obstaja bijektivna preslikava $A \to B$.
\end{izrek}

\begin{proof}
  Definirajmo družino $C : \NN \to \mathsf{Set}$ takole:
  %
  \begin{align*}
    C_0 &= A \setminus g_{*}(B), \\
    C_{n+1} &= g_{*}(f_{*}(C_n).
  \end{align*}
  %
  Naj bo $D = \bigcup_{n \in \NN} C_n$. Očitno je $C_n \subseteq A$ za vse
  $n \in \NN$, zato velja tudi $D \subseteq A$.

  Ker je $g$ injektivna, je bijekcija kot preslikava $g : B \to g_{*}(B)$, zato
  obstaja inverz $g^{-1} : g_{*}(B) \to B$. Trdimo, da velja
  $A \setminus D \subseteq g_{*}(B)$. Res, če velja $x \in A \setminus D$, tedaj
  $x \not\in D$ in zato $x \not\in C_0 = A \setminus g_{*}(B)$, od koder sledi
  $x \in g_{*}(B)$. Od tod sledi, da lahko $g^{-1}$ uporabimo na
  $x \in A \setminus D$.

  Definirajmo $h : A \to B$ s predpisom
  % 
  \begin{equation*}
    h(x) =
    \begin{cases}
      f(x), & \text{če $x \in D$,} \\
      g^{-1}(x) &\text{če $x \in A \setminus D$.}
    \end{cases}
  \end{equation*}
  %
  Dokažimo, da je $h$ injektivna preslikava.
  Denimo, da za $x, y \in A$ velja $h(x) = h(y)$. Obravnavamo štiri primere:
  %
  \begin{enumerate}
  \item Če je $x \in D$ in $y \in D$, potem je $f(x) = h(x) = h(y) = f(y)$ in
    zato $x = y$, saj je~$f$ injektivna.
  \item Če je $x \in A \setminus D$ in $y \in A \setminus D$, potem je
    $g^{-1}(x) = h(x) = h(y) = g^{-1}(y)$ in zato $x = y$, saj je $g^{-1}$
    injektivna.
  \item Če je $x \in D$ in $y \in A \setminus D$, potem je
    $f(x) = h(x) = h(y) = g^{-1}(y)$, zato je $y = g(g^{-1}(y)) = g(f(x))$.
    Obstaja $n \in \NN$, da je $x \in C_n$, od tod pa sledi
    $y = g(f(x)) \in C_{n+1} \subseteq D$, kar je v protislovju z
    $y \in A \setminus D$. Torej se ta primer sploh ne more zgoditi.
  \item Če je $x \in A \setminus D$ in $y \in D$, je razmislek kot v prejšnjem
    primeru, le da zamenjamo vlogi~$x$ in~$y$.
  \end{enumerate}

  Preveriti moramo še, da je $h$ surjektivna preslikava. Naj bo $z \in B$.
  Poiskati moramo tak $x \in A$, da velja $h(x) = z$. Obravnavamo dva primera:
  %
  \begin{enumerate}
  \item Če $z \in f_{*}(D)$, potem obstaja $x \in D$, da je $f(x) = y$, s tem pa
    velja tudi $h(x) = f(x) = z$.
  \item Če velja $z \not\in f_{*}(D)$, potem vzamemo $x = g(z)$. Preverimo, da
    velja $h(x) = z$.

    Najprej dokažimo $x \not\in D$. Če bi namreč veljalo $x \in D$, potem bi
    obstajal $n \in \NN$, da je $x \in C_n$. Poleg tega
    $x = g(z) \not\in A \setminus g_{*}(B) = C_0$, zato velja $n > 0$. Se pravi,
    da obstaja $y \in C_{n-1}$, da je $g(z) = x = g(f(y))$. Ker je $g$
    injektivna, sledi $z = f(y)$, kar je v nasprotju z predpostavko
    $z \not\in f_{*}(D)$. Torej res velja $x \not\in D$.

    Ker $x \not\in D$, velja $h(x) = g^{-1}(x) = g^{-1}(g(z)) = z$, kar smo
    želeli dokazati.
  \end{enumerate}
\end{proof}

\begin{posledica}
  Če $|A| \leq |B|$ in $|B| \leq |A|$, potem $|A| = |B|$.
\end{posledica}

\begin{proof}
  To sledi neposredno iz izreka CSB in definicije $\leq$.
\end{proof}

Brez dokaza omenimo še, da velja \textbf{zakon trihotomije}: za vsaki množici $A$ in $B$
velja
%
\begin{equation*}
  |A| < |B| \lor |A| = |B| \lor |B| < |A|.
\end{equation*}
%
Relacija $\leq$ torej uredi moči množic linearno.



\subsection{Moč kontinuuma in Cantorjeva hipoteza}

Na vajah boste spoznali, da ima množica realnih števil $\RR$ enako moč kot potenčna množica $\pow{\NN}$. Moči $\RR$ in $\pow{\NN}$ pravimo \textbf{moč kontinuuma} (ker je ">kontinuum"< tudi staro ime za $\RR$). Že Georg Cantor, utemeljitelj teorije množic, je postavil naslednji domnevo:
%
\begin{quote}
  \emph{\textbf{Cantorjeva hipoteza:} Vsaka neštevna podmnožica realnih števil je izomorfna~$\RR$.}
\end{quote}
%
Povedano, z drugimi besedami, po moči ni nobene množice strogo med $\NN$ in $\RR$. Cantorjeva hipoteza je ostala odprta dlje časa. Dokončno je Cohen pred dobrega pol stoletja dokazal naslednje:

\begin{izrek}[Cohen]
  Iz Zermelo-Fraenkelovih aksiomov teorije množic Cantorjeve hipoteze ne moremo niti dokazati niti ovreči.
\end{izrek}

Pravimo, da je Cantorjeva hipoteza \emph{neodvisna} od aksiomov teorije množic. Poznamo še posplošeno Cantorjevo hipotezo, ki se glasi:
%
\begin{quote}
  \textbf{Posplošena Cantorjeva hipoteza:}
  %
  Če je $|A| \leq |B| \leq |\pow{A}|$, potem je $|B| = |A|$ ali $|B| = |\pow{A}|$.
\end{quote}
%
Tudi posplošena Cantorjeva hipoteza je neodvisna od aksiomov teorije množic.
Danes vemo zelo veliko o tej hipotezi in poznamo še mnoge druge izjave o množicah, ki so neodvisne od Zermelo-Fraenkelovih aksiomov teorije množic.
Ti veljajo za nekakšno uradno različico teorije množic in jih bomo obravnavali na naslednjih predavanjih.


%%% Local Variables:
%%% mode: latex
%%% TeX-master: "lmn"
%%% End:


\chapter{Aksiomatska teorija množic}

\section{Kodiranje matematičnih objektov z množicami}

Z množicami smo izrazili številne matematične objekte, na primer:
%
\begin{itemize}
\item ordinalna števila smo predstavili kot množice svojih predhodnikov,
\item preslikavo $f : A \to B$ lahko izrazimo kot funkcijsko relacijo med $A$ in $B$, torej kot
  podmnožico $A \times B$,
\item kvocientna množica $A/R$ je množica ekvivalenčnih razredov, ekvivalenčni razredi so spet
  množice,
\end{itemize}
%
Ali je možno vse matematične objekte predstaviti z množicami? Da!

\subsection{Urejeni pari}

Par $(x, y)$ lahko predstavimo z množico $\set{\set{x}, \set{x,y}}$. Tako dobimo
%
\begin{equation*}
  A \times B \defeq \set{ \set{\set{x}, \set{x,y}} \mid x \in A \land y \in B }.
\end{equation*}


\subsection{Vsota}

Elemente vsote $A + B$ kodiramo takole:
%
\begin{align*}
  \inl(x) &\defeq (x, 0) = \set{\set{x}, \set{x, \emptyset}}, \\
  \inr(x) &\defeq (x, 1) = \set{\set{x}, \set{x, \set{\emptyset}}}.
\end{align*}


\subsection{Naravna števila}

Kot smo že videli, lahko ordinalna števila kodiramo kot množice svojih predhodnikov, poseben primer pa so naravna števila, ki so končni ordinali.

Kako pa kodiramo operacijo naslednik? Definirajmo preslikavo \textbf{naslednik} ${}^{+} : \Set \to \Set$,
%
\begin{equation*}
  \suc{x} \defeq x \cup \set{x}.
\end{equation*}
%
Če si predstavljamo, da je $x$ število, tedaj so elementi $\suc{x}$ predhodniki~$x$ in še $x$, kar je ravno naslednik~$x$.
Naravna števila res dobimo tako, da na $\emptyset$ uporabljamo naslednik $\suc{{}}$:
%
\begin{align*}
    0 &= \emptyset \\
    1 &= \suc{0} = \set{ 0 } = \set{\emptyset} \\
    2 &= \suc{1} = \set{0, 1} = \set{\emptyset, \set{\emptyset}} \\
    3 &= \suc{2} = \set{0, 1, 2} = \set{\emptyset, \set{\emptyset}, \set{\emptyset, \set{\emptyset}}} \\
    4 &= \suc{3} = \set{0, 1, 2, 3} =
       \set{\emptyset, \set{\emptyset}, \set{\emptyset, \set{\emptyset}},
            \set{\emptyset, \set{\emptyset}, \set{\emptyset, \set{\emptyset}}}} \\
      &\vdots
\end{align*}

\subsection{Cela števila}

Cela števila so kvocient $\NN \times \NN$:
%
\begin{equation*}
    \ZZ \defeq (\NN \times \NN)/{\sim},
\end{equation*}
%
kjer je
%
\begin{equation*}
  (a,b) \sim (c,d) \defiff a + d = c + b.
\end{equation*}
%
Urejeni par $(a, b)$ predstavlja razliko števil $a$ in $b$.


\subsection{Racionalna števila}

Racionalna števila so kvocient:
%
\begin{equation*}
  \QQ = (\ZZ \times \set{n \in \NN \such n > 0})/{\approx},
\end{equation*}
%
kjer je
%
\begin{equation*}
    (a,m) \approx (b,n) \defiff a n = b m.
\end{equation*}
%
Urejeni par $(a, n)$ predstavlja kvocient števil $a$ in $n$.

\subsection{Realna števila}

Realno število je Dedekindov rez, torej podmnožica $\QQ$. Reze ste obravnavali pri Analizi, tako da jih na tem mestu ne
bomo obnavljali.

In tako naprej. Ne pravimo, da je kodiranje vseh matematičnih objektov z množicami vedno
dobra ideja, vendar pa je dejstvo, da je to možno, pomembno spoznanje osnov matematike. Iz
njega na primer sledi tole: če je teorija množic neprotislovna, potem je neprotislovna
tudi vsa matematika, ki jo lahko kodiramo z množicami (torej več ali manj vsa običajna
matematika).



\section{Zermelo-Fraenkelovi aksiomi}

Aksiomi opredeljujejo množice brez urelementov (">\emph{Vse} je množica"<). Za aksiomatizacijo razredov bi morali zapisati drugačne aksiome, kot so na primer von Neumann-Bernays-Gödelovi aksiomi.

\begin{description}

\item[Ekstenzionalnost:] množici $A$ in $B$, ki imata iste elemente, sta enaki.

\item[Neurejeni par]: za vsak $x$ in $y$ je $\set{x, y}$ množica, ki vsebuje natanko $x$ in $y$:
  %
  \begin{equation*}
    \all{x y z} z \in \set{x, y} \liff z = x \lor z = y
  \end{equation*}
  %
  Okrajšava: $\set{x} = \set{x, x}$.

\item[Unija:] za vsako množico $A$ je $\bigcup A$ množica, ki vsebuje natanko vse
  elemente množic iz $A$:
  %
  \begin{equation*}
    \all{A x} x \in \bigcup A \liff \some{B \in A} x \in B.
  \end{equation*}

\item[Prazna množica:] množica $\emptyset$ nima elementa:
  %
  \begin{equation*}
  \all{x} x \not\in \emptyset.
  \end{equation*}

\item[Neskončna množica] obstaja množica, ki vsebuje $\emptyset$ in je zaprta za operacijo naslednik
  ($\suc{x} = x \cup \set{x}$):
  %
  \begin{equation*}
    \some{A} \emptyset \in A \land \all{x \in A} \suc{x} \in A.
  \end{equation*}

\item[Podmnožica:] za vsako množico $A$ in formulo $\phi$ je $\set{x \in A \mid \phi(x)}$
  množica, ki vsebuje natanko vse element iz $A$, ki zadoščajo $\phi$:
  %
  \begin{equation*}
    \all{y} y \in \{x \in A \mid \phi(x)\} \liff \phi(y).
  \end{equation*}

\item[Potenčna množica:] za vsako množico $A$ je $\pow{A}$ množica, ki vsebuje
  natanko vse njene podmnožice:
  %
  \begin{equation*}
    \all{S} S \in \pow{A} \liff S \subseteq A.
  \end{equation*}

\item[Zamenjava] če je $A$ množica in $f : A \to \Set$ preslikava, je
  %
  $
    \img{f}(A) = \set{ y \mid \some{x \in A} y = f(x) }
  $
  %
  množica.

\item[Dobra osnovanost:] relacija ${\in} \subseteq \Set \times \Set$ je dobro osnovana.

\item[Aksiom izbire:] vsaka družina nepraznih množic ima funkcijo izbire.
\end{description}


\section{Kumulativna hierarhija}

Če lahko vse matematične objekte kodiramo z množicami, potem lahko na razred
vseh množic $\Set$ gledamo kot na celotni matematični svet. Razred $\Set$ ima
zanimivo strukturo, ki ji pravimo \textbf{kumulativna hierarhija}. Namreč, s pomočjo
Zermelo-Fraenkelovih aksiomov lahko tvorimo vse množice iz $\emptyset$ z
operacijama potenčna množica in unija. Postopek je \textbf{transfiniten} (neskončen), ima pa toliko korakov, kot je ordinalnih števil:
%
\begin{align*}
  V_0 &= \emptyset \\
  V_1 &= \pow{V_0} = \set{\emptyset} \\
  V_2 &= \pow{V_1} = \set{\emptyset, \set{\emptyset}} \\
  V_3 &= \pow{V_2} = \set{\emptyset, \set{\emptyset}, \set{\set{\emptyset}}, \set{\emptyset, \set{\emptyset}}} \\
      &\vdots \\
  V_\omega &= \textstyle\bigcup_{k < \omega} V_k \\
  V_{\omega+1} &= \pow{V_\omega} \\
  V_{\omega+2} = &\pow{V_{\omega+1}} \\
  &\vdots \\
  V_{\omega + \omega} &= \textstyle\bigcup_{\alpha < \omega + \omega} V_\alpha \\
  &\vdots
\end{align*}
%
Splošna formula se glasi $V_\alpha = \textstyle\bigcup_{\beta < \alpha} \pow{V_\beta}$.


\begin{naloga}
  Koliko elementov ima $V_5$?
\end{naloga}

Bistvo kumulativne hierarhije je, da zaobjame vse množice.

\begin{izrek}[Kumulativna hierarhija]
  $\Set = \bigcup_{\alpha \in \On} V_\alpha$.
\end{izrek}

\begin{dokaz}
  Dokaz opustimo, povejmo le, da je za izrek bistven aksiom o dobro osnovanosti. Le-ta nam zagotavlja, da se vsaka padajoča $\in$-veriga konča z~$\emptyset$.
\end{dokaz}


% \subsection{Zakon trihotomije}

% V tem razadelku podamo še oris dokaza, da je aksiom izbire ekvivalenten zakonu trihotomije.

% \textbf{Definicija:} Naj bo $(P, <)$ dobra urejenost. Podmnožica $I ⊆ P$ je **začetni
% segment**, če je doljna množica: iz $x < y$ in $y ∈ I$ sledi $x ∈ I$.

% \textbf{Definicija:} Naj bosta $(P, <_P)$ in $(Q, <_Q)$ dobri urejenosti. Pravimo, da
% je preslikava $e : P \to Q$ **vložitev**, kadar velja:

% 1. $e$ je strogo monotona in
% 2. slika $e_{P}$ je začetni segment v $Q$.

% Vložitev je injektivna preslikava.

% **Lemma 1:** Naj bosta $(P, <_P)$ in $(Q, <_Q)$ dobri urejenosti. Če obstaja
% injektivna preslikava $P \to Q$, potem obstaja tudi vložitev $P \to Q$.

% Dokaz: opuščen.

% **Lemma 2:** Naj bosta $(P, <_P)$ in $(Q, <_Q)$ dobri urejenosti. Tedaj bodisi
% obstaja vložitev $P \to Q$ ali vložitev $Q \to P$.

% Dokaz: opuščen.

% \textbf{Izrek:} Aksiom izbire je ekvivalenten zakonu trihotomije: za vse množice $X$ in $Y$ velja
% $|X| ≤ |Y|$ ali $|Y| ≤ |X|$.

% Dokaz:

% Najprej predpostavimo, da velja aksiom izbire. Naj bosta $X$ in $Y$ množici. Ker
% velja aksiom izbire, lahko $X$ in $Y$ dobro uredimo, denimo z relacijama $<_X$
% in $<_Y$. Iz zgornje leme sledi, da obstaja vložitev $X \to Y$ ali $Y \to X$.
% Ker so vložitve injektivne, torej velja $|X| ≤ |Y|$ ali $|Y| ≤ |X|$.

% Predpostavimo zdaj, da za vse množice $X$ in $Y$ velja $|X| ≤ |Y|$ ali $|Y| ≤
% |X|$. Dokazali bomo, da lahko vsako množico dobro uredimo, iz česar sledi aksiom izbire.

\section{Aksiom izbire}

Za konec povejmo še nekaj več o aksiomu izbire in Zornovi lemi, ki mu je ekvivalentna. Le-ta se uporablja v algebri.

\begin{lema}[Zornova lema]
  Če ima v delni urejenosti $(P, {\leq})$ vsaka veriga zgornjo mejo,
  potem ima $P$ maksimalni element.
\end{lema}

\begin{dokaz}
  Dokaz se naslanja na aksiom izbire in Bourbaki-Wittov izrek o negibnih točkah (glej
  spodaj). Naj bo $C$ množica vseh verig v $P$. Uredimo jo z $\subseteq$. Na njej definiramo preslikavo
  $f : C \to C$, ki razširi verigo, če ni maksimalna, sicer je ne spremeni (tu uporabimo
  izbiro):
  %
  \begin{itemize}
  \item Če je $V \in C$ maksimalna veriga v $P$ (glede na $\subseteq$), definiramo $f(V) \defeq V$.
  \item Če $V \in C$ ni maksimalna veriga v $P$, potem obstaja tak $x \in P \setminus V$, da je $V
    \cup \set{x}$ spet veriga. V tem primeru \emph{izberemo} tak $x$ in definiramo $f(V) \defeq V
    \cup \set{x}$.
  \end{itemize}
  %
  Po izreku Bourbaki-Witt ima $f$ negibno vrednost $V \in C$. Ta $V$ je maksimalna
  veriga $V$, saj bi sicer veljalo, da je $V = f(V) = V \cup \set{x}$ za neki $x \not\in V$,
  kar ni možno. Naj bo $m$ zgornja meja za verigo $V$. Trdimo, da je $m$
  maksimalni element v $P$: denimo, da velja $m \leq y$ za $y \in P$. Ker je $V \cup \set{y}$
  veriga, ki vsebuje maksimalno verigo $V$, sledi $V = V \cup \set{y}$, od tod pa $y \in V$
  ter $y \leq m$. Torej je $m = y$ in $m$ je res maksimalni element.
\end{dokaz}

\begin{definicija}
  Naj bo $(P, \leq)$ delna ureditev. Preslikava $f : P \to P$ je \textbf{progresivna}, ko
  velja $x \leq f(x)$ za vsak $x \in P$.
\end{definicija}

\begin{opomba}
  Progresivna preslikava ni nujno monotona -- poiščite primer!
\end{opomba}

\begin{izrek}[Bourbaki-Witt]
  Naj bo $(P, {\leq})$ neprazna delna ureditev, v kateri ima vsaka veriga zgornjo mejo in $f : P \to P$ progresivna
  preslikava. Tedaj ima $f$ negibno točko: to je tak $x \in P$, da velja $f(x) = x$.
\end{izrek}

\begin{dokaz}
  Dokaz opustimo.
\end{dokaz}

\begin{izrek}
  V teoriji množic \emph{brez} aksioma izbire so naslednje izjave ekvivalentne:
  %
  \begin{enumerate}
  \item Aksiom izbire
  \item Zornova lema
  \item Princip dobre urejenosti: vsaka množica ima dobro ureditev.
  \end{enumerate}
\end{izrek}

\begin{dokaz}
  ($1 \lthen 2$) Glej Zornovo lemo.

  ($2 \lthen 3$) Skica dokaza: naj bo $A$ poljubna množica, ki jo želimo dobro urediti.
  %
  Definirajmo \emph{delne} dobre ureditve množice $A$: to so pari $(B,R)$, kjer je $B \subseteq A$
  in $R \subseteq B \times B$ dobra ureditev na $B$. Za delni dobri ureditvi $(B,R)$ in
  $(C,Q)$ pravimo, da je $(C,Q)$ \emph{razširitev} $(B,R)$, kadar velja $B \subseteq C$, $R \subseteq Q$ in
  še, da je $B$ začetni segment v $C$, kar pomeni
  %
  \begin{equation*}
    \all{x y \in C} x \rel{Q} y \land y \in B \lthen x \in B.
  \end{equation*}
  %
  Kadar je $(C,Q)$ razširitev $(B,R)$, pišemo $(B,R) \preceq (C,Q)$. Naj bo $P$ množica vseh delnih
  dobrih ureditev množice $A$,
  %
  \begin{equation*}
    P \defeq \set{ (B, R) \mid \text{$B \subseteq A$ in $R \subseteq B \times B$ in $R$ je dobra ureditev $B$}},
  \end{equation*}
  %
  urejena z relacijo $\preceq$. Očitno je $\preceq$ delna ureditev. Trdimo, da imajo verige v
  $P$ zgornje meje glede na $\preceq$: če je $V \subseteq P$ veriga dobro urejenih podmnožic
  $A$, je njena zgornja meja $(D,S)$ kar unija po komponentah:
  %
  \begin{align*}
    D &\defeq \bigcup \set{B \mid (B, R) \in V} \\
    S &\defeq \bigcup \set{R \mid (B, R) \in V}.
  \end{align*}
  %
  Preverimo, da velja $(D,S) \in P$. Očitno je $(D,S)$ stroga linearna ureditev
  (vaja). Denimo, da bi v $(D,S)$ imeli neskončno padajočo verigo
  %
  \begin{equation*}
    \cdots \rel{S} x_3 \rel{S} x_2 \rel{S} x_1 \rel{S} x_0.
  \end{equation*}
  %
  Obstaja $(B,R) \in V$, da je $x_0 \in B$. Potem bi bila $x_0, x_1, x_2, x_3, \ldots$
  padajoča veriga v $(B,R)$, kar ni možno, saj je $(B,R)$ dobro urejena. Res, ker
  je $x_i \in D$, obstaja $(C,Q) \in V$, da je $x_i \in C$. Če velja $(B,R) \preceq (C,Q)$, potem
  $x_i \in B$ po definiciji $\preceq$. Če velja $(C,Q) \preceq (B,R)$, potem $x_i \in B$, ker velja
  $C \subseteq B$. Torej je $(D,S)$ res delna ureditev $P$.

  Preverimo še, da velja $(B,R) \preceq (D,S)$ za vsak $(B,R) \in V$. Denimo, da je $y \in D$,
  $x \in B$ in $y \rel{S} x$. Obstaja $(C,Q) \in V$, da je $y \in C$. Če velja $(C,Q) \preceq (B,R)$,
  potem $y \in C \subseteq B$. Če pa velja $(B,R) \preceq (C,Q)$, potem je $y \in B$ po definiciji $\preceq$.

  Po Zornovi lemi obstaja maksimalni element $(B,R)$ v $P$. Trdimo, da je $B = A$. Če bi namreč
  obstajal $x \in B \setminus A$, bi lahko razširili $(B,R)$ na večjo dobro ureditev tako, da bi dodali $x$
  na konec $B$:
  %
  \begin{align*}
    & (B \cup \set{x}, R') \\
    & y \rel{R'} z \defiff z = x \lor y \rel{R} z.
  \end{align*}
  %
  To ni možno, ker je $(B,R)$ maksimalna delna ureditev. Torej je res $A = B$ in
  našli smo dobro ureditev $A$.

  $(3 \lthen 1)$ Naj bo $A : I \to \Set$ družina nepraznih množic. Naj bo $\prec$ dobra ureditev
  na uniji $\bigcup A$. Ker so vse množice $A_i$ neprazne, ima vsaka od njih prvi element
  glede na $\prec$. Torej lahko definiramo funkcijo izbire $f$ s predpisom
  $f(i) \defeq \text{">prvi element $A_i$"<}$.
\end{dokaz}

\begin{izrek}
  Vsak vektorski prostor ima bazo.
\end{izrek}

\begin{dokaz}
  Naj bo $L$ vektorski prostor. Definiramo množico
  %
  \begin{equation*}
    P \defeq \set{ B \subseteq L \mid \text{$B$ je linearno neodvisna} }.
  \end{equation*}
  %
  Množico $P$ delno uredimo z relacijo $\subseteq$. Trdimo, da imajo verige v $P$ zgornje
  meje: zgornja meja verige $V \subseteq P$, je kar njena unija $\bigcup_{B \in V} B$. Seveda je
  treba preveriti, da je unija verige linearno neodvisnih množic spet linearno
  neodvisna (vaja). Po Zornovi lemi obstaja maksimalni element v $P$, torej
  maksimalna linearno neodvisna množica $B$ v $L$. To pa je seveda vektorska baza
  za $L$.
\end{dokaz}



% \chapter{Množice}
\label{chap:mnozice}

V drugem delu predmeta bomo spoznali osnove teorije množic. Najprej pa
se bomo posvetili še naravnim številom in Peanovim aksiomom.

%%%%%%%%%%%%%%%%%%%%%%%%%%%%%%%%%%%%%%%%%%%%%%%%%%%%%%%%%%%%%%%%%%%%%%
\section{Naravna števila}
\label{sec:naravna-stevila}

Naravna števila
%
\begin{equation*}
  0, 1, 2, 3, 4, 5, 6, 7, 8, 9, 10, 11, 12, \ldots
\end{equation*}
%
vsi že dobro poznamo iz osnovne šole.\footnote{V teh zapiskih in v
  logiki nasploh vzamemo za prvo naravno število $0$. V osnovni šoli
  in drugje pa ponavadi za prvo naravno število jemljemo~$1$.} V tem
razdelku pokažimo, kako uvedemo naravna števila kot formalno teorijo v
logiki. V splošnem \emph{formalna teorija} opisuje neko matematično
strukturo ali družino struktur in je podana z osnovnimi simboli
(konstantami in operacijami), aksiomi in pravili sklepanja.

Teorijo naravnih števil, ki jo imenujemo tudi \emph{Peanova
  aritmetika}, sestoji iz konstante $0$, enočlene operacije
naslednik~$\suc{n}$ ter dvočlenih operacij seštevanje~$m + n$ in
množenje~$m \cdot n$. Množenje ia prednost pred seštevanjem, se pravi,
da je $k + m \cdot n = k + (m \cdot n)$ in ne $(k + m) \cdot n$.
Aksiomi in pravila sklepanja se glasijo:
%
\begin{enumerate}
  \item Nič ni naslednik:
  %
  \begin{equation*}
    \inferrule{ }{\suc{n} \neq 0}    
  \end{equation*}
  %
  \item Če sta naslednika enaka, sta števili enaki:
  %
  \begin{equation*}
    \inferrule{\suc{m} = \suc{n}}{m = n}
  \end{equation*}
  %
  \item Pravili za seštevanje:
  %
  \begin{mathpar}
    \inferrule{ }{0 + n = n}
    \and
    \inferrule{ }{\suc{m} + n = (m + n\suc{)}}    
  \end{mathpar}
  %
  \item Pravili za množenje:
  %
  \begin{mathpar}
    \inferrule{ }{0 \cdot n = 0}
    \and
    \inferrule{ }{\suc{m} \cdot n = m \cdot n + n}
  \end{mathpar}
  %
  \item Princip indukcije:
  %
  \begin{equation*}        
    \inferrule{\phi(0) \\ \xall{m}{\NN}{\phi(m) \lthen \phi(\suc{m})}}{\phi(n)}
  \end{equation*}
\end{enumerate}
%
Pri običajnem računanju z naravnimi števili uporabljamo vse znanje, ki
smo ga pridobili v šoli. Ko pa obravnavamo naravna števila kot
formalno teorijo, smemo uporabljati \emph{samo} konstante in simbole,
ki jih vpeljemo v teoriji, in se sklicevati \emph{samo} na Peanove
aksiome. Denimo, ker teorija ne vpelje simbolov $1$ in $2$, ju ne
smemo uporabljati, razen če ju prej definiramo kot okrajšavi za
$\suc{0}$ in $\suc{(\suc{0})}$. Prav tako ne smemo omenjati odštevanja
števil, ker to ni ena od operacij $+$ in $\cdot$, ne smemo govoriti o
parnosti števil, ne da bi prej ta pojem definirali, itn. Tudi osnovne
lastnosti seštevanja in množenja, kot sta komutativnost in
asociativnost, ne smemo uporabiti, če ju prej ne dokažemo. Matematiki
so seveda preverili, da vse običajne lastnosti števil dejansko sledijo
iz Peanovih aksiomov.

Glavno orodje pri dokazovanju lastnosti naravnih števil je princip
indukcije. V besedilu ga uporabimo takole:
%
\begin{quote}
  \em
  %
  Dokazujemo $\phi(n)$ z indukcijo po~$n$:
  %
  \begin{enumerate}
    \item Baza indukcije: (Dokaz, da velja $\phi(0)$.)
    \item Indukcijski korak: denimo, da za naravno število $m$ velja
      $\phi(m)$. (Dokaz, da velja $\phi(\suc{m})$.)
  \end{enumerate}
\end{quote}
%
Za zgled dokažimo, da je seštevanje komutativno. To naredimo v nekaj
korakih.

\begin{izjava}
  \label{izjava:peano-n-plus-0}
  Za vsako naravno število $m$ velja $m + 0 = m$.
\end{izjava}

\begin{proof}
  Dokazujemo z indukcijo. Baza indukcije: $0 + 0 = 0$ po enem od
  Peanovih aksiomov.
  %
  Indukcijski korak: denimo, da za naravno število $k$ velja $k + 0 =
  k$. Tedaj je $\suc{k} + 0 = \suc{(k + 0)} = \suc{k}$, kjer smo v
  prvem koraku uporabili enega od Peanovih aksiomov in v drugem
  indukcijsko predpostavko.
\end{proof}


\begin{izjava}
  \label{izjava:peano-m-plus-suc-n}
  Za vsaki naravni števili $m$ in $n$ velja $m + \suc{n} = \suc{(m + n)}$.
\end{izjava}

\begin{proof}
  Izjavo dokažemo z indukcijo po $m$.
  Baza indukcije: $0 + \suc{n} = \suc{n} = \suc{(0 + n)}$.
  %
  Indkucijski korak: denimo, da za naravno število $k$ velja $k +
  \suc{n} = \suc{(k + n)}$. Tedaj je
  %
  \begin{equation*}
    \suc{k} + \suc{n} = 
    \suc{(k + \suc{n})} =
    \suc{{\suc{(k + n)}}} =
    \suc{(\suc{k} + n)}.
  \end{equation*}
  %
\end{proof}

\begin{izjava}
  Za vsaki naravni števili $m$ in $n$ velja $m + n = n + m$.
\end{izjava}

\begin{proof}
  Izjavo dokažemo z indukcijo po $m$.
  Baza indukcije: $0 + n = n = n + 0$, kjer smo v prvem koraku uporabili Peanov aksiom in v drugem Izjavo~\ref{izjava:peano-n-plus-0}.
  %
  Indukcijski korak: denimo, da za naravno število $k$ velja $k + n = n + k$. Tedaj je
  %
  \begin{equation*}
    \suc{k} + n =
    \suc{(k + n)} =
    \suc{(n + k)} =
    n + \suc{k}.
  \end{equation*}
  %
  V prvem koraku smo uporabili Peanov aksiom, v drugem indukcijsko predpostavko, v tretjem pa Izjavo~\ref{izjava:peano-m-plus-suc-n}.
\end{proof}

%%%%%%%%%%%%%%%%%%%%%%%%%%%%%%%%%%%%%%%%%%%%%%%%%%%%%%%%%%%%%%%%%%%%%%
\section{Množice}
\label{sec:naivne-mnozice}

Množice so osnovni gradniki matematičnih objektov in struktur. V tem razdelku obravnavamo množice \emph{naivno}, se pravi s pomočjo neformalnih razlag. Samo formalno teorijo množic in aksiome bomo obravnavali v razdelku~\ref{sec:zfc}.

Množico si predstavljamo kot skupek ali zbirko poljubnih objektov, jim pravimo \emph{elementi} množice. Dejstvo, da je $x$ element množice $A$ zapišemo $x \in A$. Če $x$ ni element $S$, pišemo $x \not\in S$ kot okrajšavo za $\lnot (x \in S)$. Množica ni odvisna od tega, kako jo opišemo ali skonstruiramo, ampak le od tega, kateri elementi so v njej. To dejstvo izraža \emph{aksiomom o ekstenzionalnosti}, ki pravi, da sta množici $A$ in $B$ enaki natanko tedaj, ko vsebujeta iste elemente, kar zapišemo s formulo kot
%
\begin{equation*}
  A = B \liff \uall{x}{x \in A \liff x \in B}.
\end{equation*}
%
Množice gradimo iz osnovnih množic s pomočjo operacij.

\subsection{Osnovne množice}
\label{sec:osnovne-mnozice}

Najpreprostejša osnovna množica je \emph{prazna množica}, ki jo označimo z $\emptyset$. Dejstvo, da prazna množica ne vsebuje nobenih elementov izrazimo z aksiomom o prazni množici,
%
\begin{equation*}
  \xuall{x}{x \not\in \emptyset}.
\end{equation*}
%
V zvezi s prazno množico omenimo, da za vsako izjavo $\phi$ velja
%
\begin{equation*}
  \xall{x}{\emptyset}{\phi},
\end{equation*}
%
kar dokažemo takole: naj bo $x \in \emptyset$ poljuben. Ker velja $x \not\in \emptyset$, je to protislovje, od koder smemo sklepati $\phi$. Podobno za vsako izjavo $\phi$ velja
%
\begin{equation*}
  \lnot\xsome{x}{\emptyset}{\phi}.
\end{equation*}

\begin{vaja}
  Za katere množice $S$ velja $(\xall{x}{S}{\phi(x)}) \lthen   \xsome{x}{S}{\phi(x)}$?
\end{vaja}

Za osnovno množico vzamemo tudi množico naravnih števil~$\NN$, ki smo jo že spoznali v razdelku~\ref{sec:naravna-stevila}.

\subsection{Konstrukcije množic}
\label{sec:konstrukcije-mnozic}

Iz osnovnih množic lahko konstruiramo nove s pomočjo naslednjih operacij.

\subsubsection{Končne množice}
\label{sec:koncne-mnozice}

Naj bodo $a_1, \ldots, a_n$ poljubni objekti. Tedaj lahko tvorimo množico
%
\begin{equation*}
  \set{a_1, a_2, \ldots, a_n}
\end{equation*}
%
ki sestoji iz naštetih elementov, to je
%
\begin{equation*}
  \uall{x}{x \in \set{a_1, \ldots, a_n} \liff
    x = a_1 \lor \cdots x = a_n}.
\end{equation*}
%
Poseben primer take množice je \emph{enojec} $\set{a}$, za katerega velja
%
\begin{equation*}
  \uall{x}{x \in \set{a} \liff x = a}.
\end{equation*}


\begin{vaja}
  Ali je $\set{a, b} = \set{b, a}$? Ali je $\set{a, a, b} = \set{a, b}$? Uporabi aksiom o ekstenzionalnosti.
\end{vaja}

\subsubsection{Unija in presek}
\label{sec:unija-presek}

Družina množic.

Unije, preseki.

\subsubsection{Podmnožica}
\label{sec:podmnozica}

Podmnožica (separacija).

\subsubsection{Potenčna množica}
\label{sec:potencna-mnozica}

Potenčna množica.

\subsubsection{Kartezični produkt}
\label{sec:kartezicni-produkt}

Kartezični produkt. Produkt s prazno.

\subsubsection{Eksponentna množica}
\label{sec:eksponentna-mnozica}

Eksponentna množica. Eksponent s prazno.

\subsubsection{Vsota}
\label{sec:vsota-mnozic}

Disjunktna unija.

\subsubsection{Razlika in komplement}
\label{sec:vsota-mnozic}


\section{Funkcije}
\label{sec:funkcije}


Funkcija, neformalna definicija.

Kompozitum, asociativnost kompozituma. Identiteta.

Inverz funkcije. Inverz je enoličen, če obstaja.

Slika in inverzna slika.

Kdaj obstaja inverz? Surjektivna, injektivna, bijektivna funckija.

Epi in mono.

Sekcija in retrakcija.

Sekcija je mono, retrakcija je epi.

Standardne bijekcije za vsoto, produkt in eksponent.

\section{Relacije}
\label{sec:relacije}

Definicija relacije.

Nasprotna relacija. Komplement, unija.

\subsection{Funkcijske relacije}
\label{sub:funkcijske_relacije}


\subsection{Ekvivalenčne relacije}
\label{sub:ekvivalencne_relacije}


%Definirali smo pojem ekvivalenčne relacije in kvocienta množice po
%ekvivalenčni relaciji. Pokazali smo razne primere. Dokazali smo, da
%smemo definirati $f : A/{\sim} \to B$ na kvocientu tako, da definiramo
%$g : A \to B$, ki je skladen s~$\sim$.




Defincije. Primeri.

Ekvivalenčna relacija, generirana z relacijo.

Faktorska množica. Kako definiramo preslikavo na faktorski množici.

Kanonični razcep funkcije.

\subsection{Delna ureditev}
\label{sub:delna_ureditev}

Definicija delne ureditve. Primeri.

Linearna ureditev. Stroga linearna ureditev. Veriga.

Zgornja meja, spodna meja, infimum, supremum, maksimum, minimum, minimalni element, maksimalni element.



%%% Local Variables: 
%%% mode: latex
%%% TeX-master: "lmn"
%%% End: 



%--------------------------------------------------------------------
% BIBLIOGRAFIJA

\bibliographystyle{alpha}
\addcontentsline{toc}{chapter}{\numberline{}Literatura}
\markboth{}{Literatura}

{
\raggedright
\renewcommand{\markboth}[2]{}
\bibliography{literatura}
}


\end{document}

%%% Local Variables: 
%%% mode: latex
%%% TeX-master: t
%%% End: 
