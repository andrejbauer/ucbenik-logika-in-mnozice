\documentclass[11pt,a4paper]{book}

\usepackage[T1]{fontenc}
\usepackage[utf8]{inputenc}
\usepackage[slovene]{babel}
\usepackage[colorlinks]{hyperref}
\usepackage{lmodern}
\usepackage{xcolor}
\usepackage{amsfonts,amssymb,amsmath}
\usepackage{bbold}
\usepackage{fancyhdr}
\usepackage{theorem}
\usepackage{mathpartir}
\usepackage{proof}
\usepackage{xypic}
\usepackage{tikz}
\usepackage{booktabs}
\usepackage{enumitem}

\setlist{nosep}

%--------------------------------------------------------------------
%-- Barve hiper povezav

\hypersetup{
    colorlinks,
    linkcolor={red!50!black},
    citecolor={blue!50!black},
    urlcolor={blue!80!black}
}

%--------------------------------------------------------------------
%-- Okolja

{
  \theorembodyfont{\itshape}

  \newtheorem{izrek}{Izrek}[chapter]
  \newtheorem{lema}[izrek]{Lema}
  \newtheorem{izjava}[izrek]{Izjava}
  \newtheorem{posledica}[izrek]{Posledica}
  \newtheorem{hipoteza}[izrek]{Hipoteza}
  \newtheorem{aksiom}[izrek]{Aksiom}
}

{
  \theorembodyfont{\rmfamily}
  \newtheorem{definicija}[izrek]{Definicija}
  \newtheorem{primer}[izrek]{Primer}
  \newtheorem{opomba}[izrek]{Opomba}
  \newtheorem{naloga}[izrek]{Naloga}
}

\newcommand{\qedsign}{{\vrule width 1ex height 1ex depth 0ex}}
\newcommand{\qed}{\hfill\qedsign}

\newenvironment{dokaz}{
  \goodbreak\par
  \textit{Dokaz.}%
}{%
  \nopagebreak
  \qed
  \medskip
  \goodbreak
}

%--------------------------------------------------------------------
%%%%%%%%%%%%%%%%%%%%%%%%%%%%%%%%%%%%%%%%%%%%%%%%%%%%%%%%%%%%%%%%%%%%%%%%%%%%%%%%%%%%%%%%%%%%%%%%%%%%%%%%%%%%%%%%%%%%%%
%%%  Commands
%%%%%%%%%%%%%%%%%%%%%%%%%%%%%%%%%%%%%%%%%%%%%%%%%%%%%%%%%%%%%%%%%%%%%%%%%%%%%%%%%%%%%%%%%%%%%%%%%%%%%%%%%%%%%%%%%%%%%%


%%%%%%  Auxiliary
%%%%%%%%%%%%%%%%%%%%%%%%%%%%%%%%%%%%%%%%%%%%%%%%%%%%%%%%%%%%%
\newcommand{\sizedescriptor}[2]
{
\ifthenelse{\equal{#1}{0}}{}{
\ifthenelse{\equal{#1}{1}}{\big}{
\ifthenelse{\equal{#1}{2}}{\Big}{
\ifthenelse{\equal{#1}{3}}{\bigg}{
\ifthenelse{\equal{#1}{4}}{\Bigg}{
#2}}}}}
}

\newcommand{\someref}{{\small\textcolor{blue}{[\textbf{ref.}]}}}
\newcommand{\intermission}{\bigskip\medskip}
\newcommand{\qt}[1]{{\quotedblbase}{#1}{‘‘}}  % text in quotation marks
\newcommand{\nls}[1]{\qt{\textit{#1}}}  % sentence in a natural language

\definecolor{andrejcolor}{rgb}{0.7,0,0.7}
\definecolor{davorincolor}{rgb}{0,0.45,1}
\definecolor{markocolor}{rgb}{0.7,0.4,0}
\definecolor{matijacolor}{rgb}{0,0.6,0.4}

\newcommand{\andrej}[1]{{\small\textcolor{andrejcolor}{(#1 \ \mbox{--Andrej})}}}
\newcommand{\davorin}[1]{{\small\textcolor{davorincolor}{(#1 \ \mbox{--Davorin})}}}
\newcommand{\marko}[1]{{\small\textcolor{markocolor}{(#1 \ \mbox{--Marko})}}}
\newcommand{\matija}[1]{{\small\textcolor{matijacolor}{(#1 \ \mbox{--Matija})}}}

\definecolor{notecolor}{rgb}{0.6,0.5,0.7}
\newcommand{\note}[1]{{\small\textcolor{notecolor}{(#1)}}}
\newcommand{\alert}[1]{{\small\textcolor{red}{\textbf{#1}}}}


%%%%%%  Logical Quantifiers, λ- and ι-Expressions
%%%%%%%%%%%%%%%%%%%%%%%%%%%%%%%%%%%%%%%%%%%%%%%%%%%%%%%%%%%%%

%  no parenthesis (add x in front of the name of the command)
\NewDocumentCommand{\xall}
{m O{\empty} m}
{\forall\, {#1} \ifthenelse{\equal{#2}{}}{}{\in{#2}} \,.\, {#3}}
\NewDocumentCommand{\xsome}
{m O{\empty} m}
{\exists\, {#1} \ifthenelse{\equal{#2}{}}{}{\in{#2}} \,.\, {#3}}
\NewDocumentCommand{\xexactlyone}
{m O{\empty} m}
{\exists\;\!!\, {#1} \ifthenelse{\equal{#2}{}}{}{\in{#2}} \,.\, {#3}}
\NewDocumentCommand{\xlam}
{m O{\empty} m O{\empty}}
{\lambda{#1} \ifthenelse{\equal{#2}{}}{}{\in{#2}} \,.\, {#3} \ifthenelse{\equal{#4}{}}{}{\in{#4}}}
\NewDocumentCommand{\xthat}
{m O{\empty} m}
{\iota{#1} \ifthenelse{\equal{#2}{}}{}{\in{#2}} \,.\, {#3}}

%  with parenthesis -- the first optional argument is the size (values 0-4)
\NewDocumentCommand{\all}
{O{auto} m O{\empty} m}
{\xall{#2}[#3]{\sizedescriptor{#1}{\left}( {#4} \sizedescriptor{#1}{\right})}}
\NewDocumentCommand{\some}
{O{auto} m O{\empty} m}
{\xsome{#2}[#3]{\sizedescriptor{#1}{\left}( {#4} \sizedescriptor{#1}{\right})}}
\NewDocumentCommand{\exactlyone}
{O{auto} m O{\empty} m}
{\xexactlyone{#2}[#3]{\sizedescriptor{#1}{\left}( {#4} \sizedescriptor{#1}{\right})}}
\NewDocumentCommand{\lam}
{O{auto} m O{\empty} m O{\empty}}
{\xlam{#2}[#3]{\sizedescriptor{#1}{\left}( {#4} \sizedescriptor{#1}{\right})}[#5]}
\NewDocumentCommand{\that}
{O{auto} m O{\empty} m}
{\xthat{#2}[#3]{\sizedescriptor{#1}{\left}( {#4} \sizedescriptor{#1}{\right})}}


%%%%%%  Logic
%%%%%%%%%%%%%%%%%%%%%%%%%%%%%%%%%%%%%%%%%%%%%%%%%%%%%%%%%%%%%
\newcommand{\tvs}{\Omega}  % set of truth values
\newcommand{\true}{\top}  % truth
\newcommand{\false}{\bot}  % falsehood
\newcommand{\etrue}{\bm{\top}}  % emphasized truth
\newcommand{\efalse}{\bm{\bot}}  % emphasized falsehood
\newcommand{\impl}{\Rightarrow}  % implication sign
\newcommand{\revimpl}{\Leftarrow}  % reverse implication sign
\newcommand{\lequ}{\Leftrightarrow}  % equivalence sign
\newcommand{\xor}{\mathbin{\veebar}}  % exclusive disjunction sign
\newcommand{\shf}{\mathbin{\uparrow}}  % Sheffer connective
\newcommand{\luk}{\mathbin{\downarrow}}  % Łukasiewicz connective


%%%%%%  Sets
%%%%%%%%%%%%%%%%%%%%%%%%%%%%%%%%%%%%%%%%%%%%%%%%%%%%%%%%%%%%%
%  \set{1, 2, 3}         ->  {1, 2, 3}
%  \set{a \in X}{a < 1}  ->  {a ∈ X | a < 1}
\NewDocumentCommand{\set}
{O{auto} m G{\empty}}
{\sizedescriptor{#1}{\left}\{ {#2} \ifthenelse{\equal{#3}{}}{}{ \; \sizedescriptor{#1}{\middle}| \; {#3}} \sizedescriptor{#1}{\right}\}}
%\newcommand{\vsubset}{\Mapstochar\cap}
%\newcommand{\finseq}[1]{{#1}^*}
\newcommand{\pst}{\mathcal{P}}
\renewcommand{\complement}[1]{{#1}^C}


%%%%%%  Number Sets, Intervals
%%%%%%%%%%%%%%%%%%%%%%%%%%%%%%%%%%%%%%%%%%%%%%%%%%%%%%%%%%%%%
\newcommand{\NN}{\mathbb{N}}
\newcommand{\ZZ}{\mathbb{Z}}
\newcommand{\QQ}{\mathbb{Q}}
\newcommand{\RR}{\mathbb{R}}
\newcommand{\CC}{\mathbb{C}}
\newcommand{\intoo}[3][\RR]{{#1}_{(#2, #3)}}
\newcommand{\intcc}[3][\RR]{{#1}_{[#2, #3]}}
\newcommand{\intoc}[3][\RR]{{#1}_{(#2, #3]}}
\newcommand{\intco}[3][\RR]{{#1}_{[#2, #3)}}


%%%%%%  Maps and Relations
%%%%%%%%%%%%%%%%%%%%%%%%%%%%%%%%%%%%%%%%%%%%%%%%%%%%%%%%%%%%%
\newcommand{\id}[1][]{\textrm{Id}_{#1}}  % identity map
\newcommand{\argbox}{{\;\!\fbox{\phantom{M}}\;\!}}  % box for a function argument
\newcommand{\rstr}[1]{\left.{#1}\right|}  % map restriction
\newcommand{\im}{\mathrm{im}}  % map image
\newcommand{\parto}{\mathrel{\rightharpoonup}}  % partial mapping sign
\NewDocumentCommand{\rel}
{O{\empty} O{\empty}}
{\ifthenelse{\equal{#1}{}}{\mathscr{R}}{{#1} \mathrel{\mathscr{R}} {#2}}}  % a relation
\NewDocumentCommand{\srel}
{O{\empty} O{\empty}}
{\ifthenelse{\equal{#1}{}}{\mathscr{S}}{{#1} \mathrel{\mathscr{S}} {#2}}}  % a second relation
\newcommand{\dom}{\mathrm{dom}}  % domain
\newcommand{\cod}{\mathrm{cod}}  % codomain
\newcommand{\dd}[1]{D_{#1}}  % domain of definition
\newcommand{\rn}[1]{Z_{#1}}  % range
\newcommand{\graph}[1]{\Gamma_{#1}}  % graph of a (partial) function
\NewDocumentCommand{\img}  % image
{O{\empty} m G{\empty}}
{{#2}_*\ifthenelse{\equal{#3}{}}{}{\!\sizedescriptor{#1}{\left}( {#3} \sizedescriptor{#1}{\right})}}
\NewDocumentCommand{\pim}  % preimage
{O{\empty} m G{\empty}}
{{#2}^*\ifthenelse{\equal{#3}{}}{}{\!\sizedescriptor{#1}{\left}( {#3} \sizedescriptor{#1}{\right})}}
\newcommand{\ec}[2][]{[\:\!{#2}\:\!]_{#1}}  % equivalence class
\newcommand{\transposed}[1]{\widehat{#1}}


%%%%%%  Misc.
%%%%%%%%%%%%%%%%%%%%%%%%%%%%%%%%%%%%%%%%%%%%%%%%%%%%%%%%%%%%%
\newcommand{\df}[1]{\emph{\textbf{#1}}}  % defined notion
\newcommand{\oper}{\mathop{\circledast}}  % symbol for a general operation
\newcommand{\ism}{\cong}  % isomorphic
\newcommand{\equ}{\sim}  % equivalent
\newcommand{\dfeq}{\mathrel{\mathop:}=}  % definitional equality
\newcommand{\revdfeq}{=\mathrel{\mathop:}}  % reverse definitional equality
\newcommand{\isdefined}[1]{{#1}\!\downarrow}  % given value is defined
\newcommand{\kleq}{\simeq}  % Kleene equality
\newcommand{\claim}[3]{{#1} \;\colon\; \frac{#2}{#3}}  % claim, divided on context, assumptions, conclusions
\newcommand{\unit}{\mathord{\bm{*}}}  % element in a generic singleton
\NewDocumentEnvironment{implproof}  % proof of an implication
{O{\empty} G{\empty} O{=>} G{\empty}}
{
\begin{description}
\item[\quad$\sizedescriptor{#1}{\left}({#2}
\ifthenelse{\equal{#3}{=>}}{\impl}{
\ifthenelse{\equal{#3}{<=}}{\revimpl}{
\ifthenelse{\equal{#3}{->}}{\rightarrow}{
\ifthenelse{\equal{#3}{<-}}{\leftarrow}{
#3
}}}} {#4}\sizedescriptor{#1}{\right})$]\ \vspace{0.3em}\\
}
{
\end{description}
}


%%%%%%%%%%%%%%%%%%%%%%%%%%%%%%%%%%%%%%%%%%%%%%%%%%%%%%%%%%%%%%%%%%%%%%%%%%%%%%%%%%%%%%%%%%%%%%%%%%%%%%%%%%%%%%%%%%%%%%


%--------------------------------------------------------------------
%-- Velikost strani

%% A4 stran = 210mm x 297mm
%% sirino besedila nastavimo na 170mm, visino na 247mm

\setlength{\textwidth}{15cm}
\setlength{\textheight}{224mm}

\setlength{\topmargin}{0cm}
\setlength{\evensidemargin}{0cm}
\setlength{\oddsidemargin}{\paperwidth}
\addtolength{\oddsidemargin}{-\textwidth}
\addtolength{\oddsidemargin}{-2in}

%--------------------------------------------------------------------
%-- Glava in dno

\pagestyle{fancyplain}

%\setlength{\headrulewidth}{0.2pt}
%\addtolength{\headheight}{2pt}

\renewcommand{\chaptermark}[1]{\markboth{#1}{}}
\renewcommand{\sectionmark}[1]{\markright{\thesection\ #1}}

\lhead[\fancyplain{}{{\thepage}}]%
      {\fancyplain{}{{\rightmark}}}
\rhead[\fancyplain{}{{\leftmark}}]%
      {\fancyplain{}{\thepage}}
\cfoot{\footnotesize [verzija \today]}
\lfoot[]{}
\rfoot[]{}

%--------------------------------------------------------------------
% NASLOV

\author{Andrej Bauer}
\title{Logika in množice \\ ZAPISKI V NASTAJANJU}

\begin{document}

\maketitle

\cleardoublepage

%--------------------------------------------------------------------
% KAZALO
\pagestyle{fancyplain}

{
\renewcommand{\markboth}[2]{}
\tableofcontents
}

\cleardoublepage

%--------------------------------------------------------------------
% VSEBINA

\chapter*{Predgovor}
\addcontentsline{toc}{chapter}{Predgovor}

Pri predmetu Logika in množice v prvem letniku študija matematike na Fakulteti za matematiko in fiziko Univerze v Ljubljani se študenti učijo osnov matematičnega izražanja---kako beremo in pišemo matematično besedilo in simbolni zapis---hkrati pa spoznavajo temelje matematične logike in teorije množic.
%
Pred študenti matematike je torej težka naloga učiti se novo snov v novem žargonu.

Da bo učbenik v pomoč, bomo pri matematičnem izražanju bolj natančni, kot je to običajno za matematično besedilo.
Pojasnjevali bomo, kako matematiki pišejo, govorijo in razmišljajo v praksi ter marsikaj raje zapisali na dolgo, da bo začetniku bolj prijazno.
%
Bližnjice, ki jih ubirajo izkušeni matematiki, bomo vpeljali zlagoma, hkrati pa opozarjali na nedoslednosti, ki so
večinoma ostanki zgodovinskega razvoja matematike in ki se jim matematična tradicija stežka odreče.
%
Ne zamerite nam, če dobrohotno ponudimo še kak nasvet o študiju matematike.


\paragraph{Zahvala.}
%
Za pomoč pri urejanju zapiskov in opozarjanje na napake se zahvaljujeva študentkam in študentom:
%
Luka Debevc,
Milan Djaković,
Ema Grmšek,
Matija Fajfar,
Miha Gyergyek,
Peter Jereb,
Jan Kastelic,
Jan Malej,
Matej Marinko,
Jan Pantner,
Lev Rus,
Jakob Schrader,
Matija Sirk,
Matej Šafarič,
Gal Zmazek,
Marjetka Zupan in Patrik Žnidaršič.
%
Vse preostale napake so najina last.
\bigskip

\begin{flushright}
Andrej Bauer in Davorin Lešnik
\end{flushright}

\bigskip


%%% Local Variables: 
%%% mode: latex
%%% TeX-master: "lmn"
%%% End: 

\chapter{Osnovni podatki o predmetu}

\paragraph{Gradivo}
%
Osnovni podatki o predmetu in gradivo je na \href{https://ucilnica.fmf.uni-lj.si/}{spletni učilnici}, kjer najdete:
\begin{itemize}
\item povezavo do video posnetkov predavanj in zapiskov s table,
\item naloge z vaj, ki so objavljenje vnaprej,
\item prejšnje kolokvije in izpite,
\item povezavo na Discord server za predmet.
\end{itemize}

\paragraph{Izpitni režim}
%
Predmet opravite z izpitom, ki ima dva dela:
%
\begin{enumerate}
\item \textbf{pisni izpit}
\item \textbf{ustni izpit}
\end{enumerate}
%
Namesto pisnega izpita lahko opravite dva kolokvija (s povprečno oceno obeh kolokvijev skupaj vsaj 50\%). Na ustni izpit pridete šele, ko ste opravili pisni izpit. Če ustnega izpita ne opravite, vam pisni izpit propade in ga
morate ponovno opravljati.

\chapter{Osnovno o množicah in preslikavah}
\label{cha:mnozice-in-preslikave}


Temeljni gradniki sodobne matematike so \df{množice}, ki so skupki ali zbirke matematičnih
objektov, lahko spet množic. Vsaka množica sestoji iz \df{elementov} in je z njimi
natančno določena.
%
Kadar je $a$ element množice $M$, to zapišemo $a \in M$ in beremo ``$a$ je element~$M$'' ali ``$a$ pripada~$M$''.`
Slišali boste tudi. Odsvetujemo ``$a$ je vsebovan v~$M$'', ker že $A \subseteq M$ beremo ``$A$ je vsebovan v $M$''.

Ideja množice kot poljubne zbirke elementov je zavajajoče preprosta, kar so na lastni koži
izkusili matematiki na prelomu iz 19.~v 20.~stoletje. Takrat so že vedeli, da so množice zelo
uporabne in da lahko iz njih tvorimo razne vrste matematičnih objektov. A znameniti
matematik in filozof Bertrand Russell je odkril paradoks, ki se imenuje po njem, in gre
takole. Naj bo~$R$ množica vseh množic, ki niso element same sebe. Ali $R$ je element~$R$?
Če je $R$ element $R$, potem iz definicije $R$ sledi, da $R$ ni element $R$. In če $R$ ni
element $R$, spet iz definicije $R$ sledi, da $R$ je element $R$. Torej $R$ hkrati je in
ni svoj element, kar je protislovje! Russellov paradoks ste morda že spoznali v
priljubljeni različici, ki govori o vaškem brivcu, ki brije vse vaščane, ki ne brijejo
samih sebe.

Russellov paradoks je povzročil pravo krizo v temeljih matematike. Ker so bile množice
nepogrešljivo orodje, jih niso hoteli kar zavreči, po drugi strani pa je bilo treba
preprečiti Russellov in druge paradokse, ki so jih še odkrili. Bertrand Russell je
predlagal rešitev, ki jo je poimenoval \df{teorija tipov}. Russellova teorija tipov je
pomembno vplivala na nadaljni razvoj temeljev matematike, sodobna teorija tipov pa je
pomembno orodje v računalništvu. Tako kot množice so bili tipi skupki elementov, a so
tvorili neskončno hierarhijo, v kateri so bili elementi tipa vedno iz nižjega nivoja
hierarhije kot tip, ki so mu pripadali. Za potrebe večine matematike zadostuje že
preprostejša dvoslojna hierarhija množic in \df{razredov}. Množice smejo biti elementi
množic in razredov, razredi pa ne. Russellov paradoks izgine, ker je $R$ razred vseh
tistih množic, ki niso same svoj element. Vprašanje, ali je $R$ element samega sebe, tako
postane nesmiselno, saj $R$ ni množica. A zaenkrat odložimo podrobnejšo obravnavo razredov
in se raje posvetimo osnovnima pojmoma, množica in preslikava.

V splošni razpravi o množicah, ki bi presegala meje matematične vede, bi se opirali na
zgodovinski in družbeni kontekst, jezikovni izvor in rabo besed `množica', `skupek' in
`zbirka', kognitivno analizo, eksperimente, filozofijo itn. Vsi ti vidiki so za matematike
izjemo koristni, saj iz takih ``pred-matematičnih'' obravnav črpamo sveže zamisli in
matematiko naredimo zares uporabno. Ko pa delujemo znotraj matematike, zunanje vplive
odmislimo in se zanašamo le še na pravila logičnega sklepanja in matematične zakone, da ne
prihaja do nejasnosti in dvomljivih sklepov.

Kot matematiki lahko ustvarimo takšen ali drugačen pojem množice in pri tem imamo popolno
svobodo. Se množica lahko spreminja ali vedno vsebuje iste elemente? Je pomemben vrsti red
elementov v množici? Sme množica biti element same sebe? Ali morajo biti elementi množice
izračunljivi? To so vprašanja, ki nimajo enoznačnega odgovora. In res je znanih več med
seboj nezdružljivih zvrsti teorije množic, ki matematično opredeljujejo različne vidike
običajnega razumevanja besede `množica'. Mi bomo spoznali ``standardno'' teorijo množic,
ki jo uporablja velika večina matematikov.


\section{Pravilo ekstenzionalnosti}
\label{sec:nacelo-ekstenzionalnosti}

Zamisel, da je množica natančno določena s svojimi elementi, izrazimo z matematičnim
zakonom, ki mu pravimo \df{pravilo ekstenzionalnosti}:

\begin{pravilo}[Ekstenzionalnost množic]
  Množici sta enaki, če vsebujeta iste elemente.
\end{pravilo}

Kaj pravzaprav pomeni, da je to ``pravilo'', ``matematični zakon'' ali ``načelo''? So ga
razglasili v parlementu, je to zakon narave, ali morda dogma, ki jo je razglasil profesor
na predavanjih? Bodo tisti, ki pravila ekstenzionalnosti ne spoštujejo, deležni Lešnikove
masti? Ne. Matematični zakoni so \emph{dogovori}, nekakšna pravila matematične igre. V
zgodovinskem razvoju matematike so se uveljavili tisti dogovori, ki so bili uporabni v
naravoslovju in tehniki, ali pa so v njih matematiki videli notranjo lepoto in lastno
uporabno vrednost.

Pravkar smo se dogovorili, da bomo obravnavali matematične objekte množice, ki vsebujejo
elemente in da zanje velja pravilo ekstenzionalnosti. Namesto besed `množica' in `element'
bi lahko izbrali tudi kaki drugi besedi, denimo `zbor' in `član', ali celo `morje' in
`riba', s čimer se matematična vsebina pojmov ne bi čisto nič spremenila, čeprav ne gre
preveč izzivati svojih stanovskih kolegic in kolegov. Strukturo, lastnosti in povezave med
matematičnimi objekti namreč določajo dogovorjeni matematični zakoni in ne besede, s
katerimi jih poimenujemo.

Še enkrat poudarimo, da ima vsakdo, še posebej pa mladi um, popolno svobodo matematičnega
ustvarjanja. Želite razmišljati o drugačnih množicah, ki ne zadoščajo pravilu
ekstenzionalnsti? Ali pa o številih, ki zadoščajo zakonu $x + x = 0$? O geometriji, v
kateri skozi točko lahko potegnemo dve vzporednici k dani premici? Kar dajte! Pri tem vas
le prosimo, celo zahtevamo, da razmišljate temeljito, vztrajno in globoko, da ste iskreni
do sebe in ostalih ter da svoje zamisli in spoznanja predstavite na matematikom razumljiv
način.

Vrnimo se k našim množicam. Pravilo ekstenzionalnosti nam pove, da lahko množico podamo
tako, da natančno opredelimo njene elemente. A to ne pomeni, da množica obstaja, brž ko jo
lahko natančno opredelimo! To je pot, ki vodi naravnost do Russelovega paradoksa, saj so
elementi paradoksalne množice~$R$ natančno opredeljeni. Potrebujemo dodatna pravila, ki
določajo dopustne \df{konstrukcije množic}. Izbrati jih moramo previdno, da se izognemo
težavam.

\section{Končne množice}
\label{sec:koncne-mnozice}

Posebej preprosta konstrukcija množic združi končen nabor matematičnih objektov v množico.
Na primer, če so $a$, $b$ in $c$ matematični objekti, potem lahko tvorimo množico
%
\begin{equation*}
  \set{a, b, c}
\end{equation*}
%
katere objekti so natanko $a$, $b$ in $c$. To pomeni, da za vsak matematični objekt~$x$
velja
%
\begin{equation*}
  \text{$x \in \set{a, b, c}$, če in samo če $x = a$ ali $x = b$ ali $x = c$.}
\end{equation*}
%
Fraza ``če in samo če'' tu pomeni, da velja dvoje:
%
\begin{enumerate}
\item Če $x = a$ ali $x = b$ ali $x = c$, potem $x \in \set{a, b, c}$.
\item Če $x \in \set{a, b, c}$, potem $x = a$ ali $x = b$ ali $x = c$.
\end{enumerate}
%
Tako nam na primer prva trditev zagotavlja $1+1 \in \set{1, 2, 3}$, ker velja
vsaj ena od možnosti: $1 + 1 = 1$ ali $1 + 1 = 2$ ali $1 + 1 = 3$. Iz druge trditve sledi, da
$5 \in \set{1, 2, 3}$ ne velja, ker ne velja nobena od možnosti: $5 = 1$ ali $5 = 2$ ali
$5 = 3$.

Splošna konstrukcija končnih množic poteka takole.

\begin{pravilo}
  \label{pravilo:koncna-mnozica}
  Za vse objekte $a$, $b$, \dots, $z$ je $\set{a, b, \ldots, z}$ množica, katere elementi
  so natanko objekti $a$, $b$, \dots, $z$.
\end{pravilo}

Za trenutek ustavimo tok misli in opozorimo, da zapis s tropičjem `$\ldots$' ni dovolj
natančen, saj dopušča dvoumnosti. Denimo, so elementi množice
%
\begin{equation*}
  \set{3, 5, 7, \ldots, 31},
\end{equation*}
%
liha števila med $3$ in $31$, ali samo praštevila? Zapis res ni dovolj natančen. Kljub
temu tak zapis v praksi uporabljamo, ker v praksi bralec večinoma pravilno ugane, kaj je
bilo mišljeno, saj imamo ljudje zelo podobne sposobnosti prepoznavanja vzorcev. Z
matematičnega vidika pa to ni dopustno, saj lahko tropičje \emph{vedno} razumemo na več
načinov. (Ne verjamete? Naslednji člen v zaporedju $1, 2, 3, \ldots$ je seveda~$5$, ker je
naslednji člen vsota prejšnjih dveh, kot v Fibonaccijevem zaporedju.)

Kot smo že omenili, želimo pojem množice, pri kateri vrstni red elementov ni pomemben.
Pričakujemo, da lahko dokažemo enakost množic $\set{1, 2}$ in $\set{2, 1}$. Pa je to res? Velja ena
od treh možnosti:
%
\begin{enumerate}
\item Iz pravila ekstenzionalnosti in konstrukcije množic $\set{1, 2}$ in $\set{2, 1}$ sledi, da sta enaki.
\item Iz pravila ekstenzionalnosti in konstrukcije množic $\set{1, 2}$ in $\set{2, 1}$ sledi, da nista enaki.
\item Pravilo ekstenzionalnosti in konstrukcije množic $\set{1, 2}$ in $\set{2, 1}$ ne določajo, ali sta enaki.
\end{enumerate}
%
V prvem primeru želimo videti dokaza. V drugem primeru bi bili v zagati, saj 
bi izbrana matematična pravila imela neželene posledice. V tretjem primeru bi morali
dodati še kakšne nove zakone o množicah. Na srečo obvelja prva možnost.

\begin{trditev}
  Množici $\set{1, 2}$ in $\set{2, 1}$ sta enaki.
\end{trditev}

\begin{dokaz}
  Dokaz, ki ga bomo zapisali je izjemno podroben in ga v praksi matematik ne bi zapisal,
  saj je z njegovim branjem več dela, kot če bi ga poustvarili sami. Ker pa želimo pokazati, da
  tudi najbolj trivialna dejstva lahko dokažemo, ga zapišimo.

  Izhajati smemo izključno iz naslednji dejstev:
  %
  \begin{itemize}
  \item pravilo ekstenzionalnosti,
  \item $x \in \set{1, 2}$, če in samo če $x = 1$ ali $x = 2$,
  \item $x \in \set{2, 1}$, če in samo če $x = 2$ ali $x = 1$.
  \end{itemize}
  %
  Najprej uporabimo pravilo ekstenzionalnosti, ki zagotavlja, da sta $\set{1, 2}$ in
  $\set{2, 1}$ enaki, če imata iste elemente. Dokažimo torej, da imata iste elemente. To
  naredimo v dveh korakih:
  %
  \begin{enumerate}
  \item Za vsak element $\set{1, 2}$ dokažemo, da je element $\set{2, 1}$.
    Naj bo $x \in \set{1, 2}$. Iz definicije množice $\set{1, 2}$
    sledi, da je $x = 1$ ali $x = 2$. Obravnavamo dva podprimera:
    %
    \begin{enumerate}
    \item Primer $x = 1$: iz $x = 1$ sledi, da je $x = 2$ ali $x = 1$, zato je $x \in \set{2, 1}$.
    \item Primer $x = 2$: iz $x = 2$ sledi, da je $x = 2$ ali $x = 1$, zato je $x \in \set{2, 1}$.
    \end{enumerate}
    %
  \item Za vsak element $\set{2, 1}$ dokažemo, da je element $\set{1, 2}$.

    Ta korak je povsem podoben prvemu, le da je treba povsod zamenjati~$1$ in~$2$.
    Matematik bi zato na tem mestu zapisal, da je drugi korak podoben prevemu in dokaz
    zaključil. A tega tokrat ne bomo storili in bomo zapisali popoln dokaz.

    Naj bo $x \in \set{2, 1}$. Iz definicije množice $\set{2, 1}$ sledi, da je $x = 2$ ali
    $x = 1$. Obravnavamo dva primera:
    %
    \begin{enumerate}
    \item Primer $x = 2$: iz $x = 2$ sledi, da je $x = 1$ ali $x = 2$, zato je $x \in \set{1, 2}$.
    \item Primer $x = 1$: iz $x = 1$ sledi, da je $x = 1$ ali $x = 2$, zato je $x \in \set{1, 2}$. \qedhere
    \end{enumerate}
    %
  \end{enumerate}
\end{dokaz}

Mimogrede, kvadratek označuje konec dokaza. Imenuje se tudi ``Halmos'' po matematiku
Paulu Halmosu, ki ga je prvi uporabljal. S podobnim razmislekom, ki ga prepuščamo za vajo,
lahko dokažemo, da ni pomembno, ali se element pojavi enkrat ali večkrat.

\begin{vaja}
  Podrobno dokažite, da sta množici $\set{1, 1, 2}$ in $\set{1, 2}$ enaki.
\end{vaja}

V prejšnji nalogi smo zapisali $\set{1, 1, 2}$. Pa je to sploh dovoljeno?
Pravilo~\ref{pravilo:koncna-mnozica} pravi, da lahko iz objektov $a, b, c, \ldots, z$
tvorimo končno množico $\set{a, b, \ldots, z}$. Nikjer ne piše, da smeta biti $a$ in $b$
enaka, zato je upravičeno vprašanje, ali je dovoljeno za $a$ in $b$ vzeti~$1$. V
matematiki vse razumemo dobesedno. V pravilu~\ref{pravilo:koncna-mnozica} piše ``Za vse
objekte'', torej imamo povsem proste roke. Povedano z drugimi besedami, množico
$\set{1, 1, 2}$ smemo tvoriti, ker nikjer ne piše, da morajo biti elementi različni.

V zvezi s pravilom~\ref{pravilo:koncna-mnozica} se pojavljajo še drugi dvomi. Ali smemo
tvoriti množico, ki ima več elementov, kot je črk abecede? Ali bi bilo pravilo še vedno
isto, če bi namesto ``$a, b, \ldots, z$'' zapisali ``$a, b, \ldots, j$''? Ali smemo
tvoriti množico z nič elementi? Če namreč vstavimo nič elementov, se pravilo glasi ``Za
vse objekte je $\set{\,}$ množica, katere elementi so natanko objekti,'' kar je vsaj
nenavadno. Iz nesrečnega tropičja se res ne vidi, kaj je in kaj ni dovoljeno. Če poškilite
v razdelek~\ref{sec:aksiomi-teorije-mnozic}, kjer so našteti ``uradni'' aksiomih teorije
množic, tam pravila o končnih množicah ne boste našli, saj sledi iz treh bolj osnovnih
pravil.

\begin{pravilo}
  \label{pravilo:prazna-mnozica}
  \df{Prazna množica} $\emptyset$ je množica, ki nima elementov.
\end{pravilo}

\begin{pravilo}
  \label{pravilo:neurejeni-dvojec}
  Za vsak $x$ in $y$ je \df{(neurejeni) par} ali \df{dvojec} $\set{x, y}$ množica, katere
  elementa sta natanko $x$ in $y$.
\end{pravilo}

\begin{pravilo}
  \label{pravilo:unija}
  Za vsaki množici $A$ in $B$ je \df{unija $A \cup B$} množica, ki ima za elemente
  natanko vse objekte, ki so element $A$ ali element $B$.
\end{pravilo}

V pravilu~\ref{pravilo:neurejeni-dvojec} smo besedo ``neurejeni'' zapisali v oklepaju, kar
pomeni, da beseda pravzaprav ni pombembna in bi jo lahko tudi izpustili. Se pravi, da
``neurejeni dvojec'' in ``dvojec'' pomenita isto. V primeru nejasnosti raje uporabimo
daljšo obliko.

Tri nova pravila skupaj nadomestijo pravilo~\ref{pravilo:koncna-mnozica} in odstranijo
marsikateri dvom o uporabi. Prvo pravilo pojasni, da lahko tvorimo množico brez elementov.
Poleg oznake $\emptyset$ je za prazno množico smiselno uporabiti tudi zapis $\set{\,}$.

Drugo pravilo pove, kako lahko tvorimo množico z dvema elementoma, pa tudi z enim.
Spomnimo se, pravila je treba brati dobesedno: za $x$ in $y$ bi lahko vzeli dvakrat isti
objekt~$z$ in tvorili množico $\set{z, z}$, ki ima natanko elementa $z$ in $z$. To je
pravzaprav množica z enim samim elementom $z$, zato ji pravimo tudi \df{enojec} in jo
zapišemo~$\set{z}$.

Tretje pravilo nam omogoča, da tvorimo večje množice. Denimo, množico z elementi $a$, $b$,
$c$ lahko tvorimo kot unijo
%
\begin{equation*}
  \set{a, b} \cup \set{c}.
\end{equation*}
%
To ni edini način, enako množico lahko dobimo na več načinov:
%
\begin{equation*}
  (\set{a} \cup \set{b}) \cup \set{c}
  \quad\text{ali}\quad
  \set{b} \cup \set{c, a}
  \quad\text{ali}\quad
  \set{a,c,a} \cup \set{b,c}
  \quad\text{itn.}
\end{equation*}
%
Seveda bi morali dokazati, da so vse te množice enake, a tega ne bomo storili.

Pogosto nam bo prišlo prav, da bomo imeli pri roki množico z enim elementom, pri čemer nam
bo vseeno, kaj ta element je. V ta namen postavimo pravilo, ki zagotavlja obstoj množice z
enim elementom.

\begin{pravilo}
  \label{pravilo:enojec}
  \df{Standardni enojec} je množica~$\one$, katere edini element je~$\unit$.
\end{pravilo}

Morda se zdi nenavadno, da množico označimo s številom, a ta občutek bo hitro izginil, ko
bomo računali z množicami. Pravaprav bi lahko prazno množico označili z nič $\mathbf{0}$,
in nekateri matematiki to dejansko počnejo.

Edini element množice $\one$ smo označili z nenavadnim zapisom $\unit$. Na tem mestu ne
bomo pojasnili, zakaj pišemo tako, radovedneži pa lahko pogledajo v
razdelek~\ref{sec:algebra-mnozic}. Mimogrede, seveda velja $\one = \set{\unit}$.

Pravilo~\ref{pravilo:enojec} ni nujno potrebno, saj lahko tvorimo veliko različnih enojcev
kar sami $\set{\emptyset}$, $\set{42}$, $\set{\set{\emptyset}}$ itn. Ali je kateri od njih
``prvi med enakimi'' in bi ga lahko uporabljali kot ``standardni'' enojec? Ker je odgovor
v veliki meri stvar osebnega mnenja, je bolje, da razglasimo pravilo, ki ustoliči
standardni enojec. S prazno množico nimamo podobnih težav, saj je ena sama.

% \subsection{Druge množice}

% \andrej{To ne paše sem, ker bi bilo tu dosti bolj naravno nadaljevati s preslikavami.
%  To bomo prestavili na mesto, kjer bo dejansko prišlo prav.}

% Množice, s katerimi v matematiki delamo, tipično vsebujejo števila, ali pa so vsaj na tak ali drugačen način izpeljane iz številskih množic. Spomnimo se standardnih oznak najpogosteje uporabljanih številskih množic.
% \begin{center}
% \begin{tabular}{|cc|}
% \hline
% \textbf{Množica} & \textbf{Oznaka} \\
% \hline
% množica naravnih števil & $\NN$ \\
% množica celih števil & $\ZZ$ \\
% množica racionalnih števil & $\QQ$ \\
% množica realnih števil & $\RR$ \\
% množica kompleksnih števil & $\CC$ \\
% \hline
% \end{tabular}
% \end{center}

% Nekateri $0$ vzamejo za naravno število, nekateri ne. To je v celoti stvar dogovora, kaj pomeni pojem ``naravno število''. Za nas bo prišlo bolj prav, če ničlo štejemo kot element množice naravnih števil, torej $\NN = \set{0, 1, 2, 3, \ldots}$.

\section{Preslikave}

Temelj matematike ne tvorijo le množice, ampak tudi drugi matematični pojmi. Prvi izmed
njih je \df{preslikava}, oziroma s tujko \df{funkcija}.\footnote{Nekateri uporabljajo
  izraz ``funkcija'' samo za tiste preslikave, ki slikajo v realna ali kompleksna števila,
  vendar to navado izpodriva računalništvo, saj funkcije v programskih jezikih nimajo
  omejitev. Dandanes večina matematikov besedo ``funkcija'' obravnava kot sopomenko besede
  ``preslikava'' in tako jo bomo uporabljali tudi mi.} V srednji šoli ste že spoznali
nekatere preslikave, kot so na primer linearne preslikave, trigonometrijske funkcije,
logaritem itd. Nas pa ne bodo zanimale posamezne preslikave, ali posebne lastnosti
preslikav, ampak preslikave na splošno.

Vsaka preslikava ima tri sestavne dele: \df{domeno} ali \df{začetno množico},
\df{kodomeno} ali \df{ciljno množico} in \df{predpis}. Domeni se pogosto reče tudi
\df{definicijsko območje}. Če govorimo o preslikavi, ki ima domeno~$X$ in kodomeno~$Y$, to
ponazorimo s puščico med $X$ in $Y$, takole
%
\begin{equation*}
  \xymatrix{
    {X} \ar[r] &
    {Y}
  }
\end{equation*}
%
Če želimo preslikavo poimenovati, na primer $f$, zapišemo
%
\begin{equation*}
  \xymatrix{
   {f : X} \ar[r] &
    {Y}
  }
  \qquad\text{ali}\qquad
  \xymatrix{
   {X} \ar[r]^{f} &
   {Y}
  }
\end{equation*}
%
Pravimo, da je \df{$f$ preslikava iz $X$ v $Y$}. Zapis nad puščico je prikladen, kadar
imamo opravka z večimi preslikavami, ki jih predstavimo z diagramom. Na primer,
%
\begin{equation*}
  \xymatrix{
    {X} \ar[r] &
    {Y} \ar[r]^{f} &
    {Z}  &
    {W} \ar[l]_{g}
  }
\end{equation*}
%
nam pove, da imamo opravka z (neimenovano) preslikavo iz $X$ v $Y$, s preslikavo $f$ iz
$Y$ v $Z$ in s preslikavo $g$ is $W$ v $Z$. Diagrami so lahko še precej bolj zapleteni.

Tretji del preslikave je predpis, ki določa, kako elemente domene preslikamo v elemente
kodomene. Kaj pravzaprav to pomeni? Možnih je več odgovorov. V srednji šoli predpis
enačimo z matematično formulo, ki spremenljivko preslika v vrednost, na primer $x$ slika v
$2 \sin(x + \pi/4)$. S simboli to zapišemo
%
\begin{equation*}
  x \mapsto 2 \sin(x + \pi/4).
\end{equation*}
%
in preberemo ``$x$ se slika v dvakrat sinus od $x$ plus pi četrtin.''
%
Matematiki smo natančni, zato ne mešamo uporabe puščic $\to$ in $\mapsto$. Navadna puščica
se uporablja pri oznaki domene in kodomene, repata pa v predpisu. V računalništvu besedo
`predpis' razumemo kot `programska koda' in o preslikavah razmišljajo kar kot o
algoritmih --- tudi to je eden od možnih pogledov na preslikave.

V teoriji množic razumemo besedo `predpis' kot kakršnokoli prirejanje med elementi množic
domene~$X$ in kodomene~$Y$, mora pa veljati:
%
\begin{itemize}
\item \df{celovitost}: vsakemu elementu iz $X$ je prirejen vsaj en element iz $Y$,
\item \df{enoličnost}: če sta elementu $x$ prirejena $y \in Y$ in $z \in Y$, potem $y = z$.
\end{itemize}

Za vsako množico~$A$ je \df{identiteta} na~$A$ preslikava
%
\begin{equation*}
  \id[A] : A \to A
\end{equation*}
%
ki poljubnemu elementu $x \in A$ priredi~$x$. To je celovito prirejanje, saj vsak
$x \in A$ ima prirejeni element, namreč kar $x$, je pa tudi enolično: če sta $y_1$ in
$y_2$ prirejena $x \in A$, potem sta oba enaka~$x$ in zato enaka drug drugemu.

Za vsaki množici $A$ in $B$ ter $b \in B$ \df{konstantna preslikava}
%
\begin{equation*}
  \konst{b} : A \to B
\end{equation*}
%
priredi vsakemu elementu iz~$A$ element~$b$. Sami premislite, da je tako prirejanje
celovito in enolično.

\subsection{Funkcijski predpisi}
\label{sec:funkcijski-predpisi}

Predpise lahko podamo na različne načine, najbolj pogost pa je \df{funkcijski predpis}, ki
se mu še posebej posvetimo in se ob njem naučimo nekaj natančnosti. Funkcijski predpis ima
obliko
%
\begin{equation*}
  x \mapsto \cdots,
\end{equation*}
%
ki smo jo že videli maloprej. Na desni, lahko namesto $\cdots$ zapišemo izraz, v katerem
se sme pojaviti simbol~$x$, denimo
%
\begin{equation*}
  x \mapsto 1 + x^2.
\end{equation*}
%
S funkcijskim predpisom zapišemo identiteto in konstantno preslikavo takole:
%
\begin{align*}
  \id[A] &: A \to A
  &
  \konst{b} &: A \to B
  \\
  \id[A] &: x \mapsto x
  &
  \konst{b} &: x \mapsto b.
\end{align*}

Ni nujno, da se~$x$ pojavi, denimo $x \mapsto 42$ vsakemu elementu iz domene priredi
število $42$. V funkcijskem predpisu se smejo pojaviti tudi drugi simboli, ki jim
pravimo \df{parametri}. Tako je
%
\begin{equation*}
  x \mapsto a \cdot x + b
\end{equation*}
%
funkcijski predpis s parametroma $a$ in $b$, ki elementu $x$ priredi element $a \cdot x + b$.

Spremenljivka $x$ nima v naprej določene vrednosti, pač pa kaže, kam lahko vstavimo
elemente domene. Pravimo, da je $x$ \df{vezana spremenljivka}, kar pomeni, da je veljavna
le v funkcijskem predpisu, nanj je vezana, in da ni pomembno, s katerim simbolom jo
označimo. Tako sta funkcijska predpisa
%
\begin{equation*}
  x \mapsto 1 + x^2
  \qquad\text{in}\qquad
  a \mapsto 1 + a^2
\end{equation*}
%
enaka in lahko bi celo pisali $\Box \mapsto 1 + \Box^2$ ali
$\heartsuit \mapsto 1 + \heartsuit^2$.

V funkcijskem predpisu mora na levi stati en sam simbol, ki na desni kaže, kam je treba
vstaviti element iz domene. Tako
%
\begin{equation*}
  \sin(x) \mapsto \cos(2 x),
  \qquad
  3 + 2 \mapsto 5
  \qquad\text{in}\qquad
  \sin(x) \mapsto 2 \cdot \sin(x)
\end{equation*}
%
\emph{niso} veljavni funkcijski predpisi.

Seveda dopuščamo možnost, da se vezana spremenljivka pojavi enkrat, večkrat ali sploh ne.
Funkcijska predpisa
%
%
\begin{equation*}
  x \mapsto 42
  \qquad\text{in}\qquad
  x \mapsto x \cdot \sin(x)
\end{equation*}
%
sta torej veljavna.

Če želimo preslikavo z danim funkcijskim predpisom poimenovati, na primer $f$, zapišemo
%
\begin{equation*}
  f : x \mapsto 1 + x^2.
\end{equation*}
%
To preberemo ``$f$ slika $x$ v ena plus $x$ na kvadrat.'' Običajna sta tudi zapisa
%
\begin{equation*}
  f(x) = 1 + x^2
  \qquad\text{in}\qquad
  f(x) \dfeq 1 + x^2.
\end{equation*}
%
Funkcijske predpise je podrobno prvi preučeval Alonzo Church,\footnote{Alonzo Church
  (1903--1995) je bil ameriški matematik in logik, ki je pomembno prispeval k razvoju
  logike in teoretičnega računalništva. Njegov študent, Dana Stewarta Scott, je imel
  študenta Marka Petkovška in Andreja Bauerja, slednji pa je imel študenta Davorina
  Lešnika.} ki je uporabljal zapis
%
\begin{equation*}
  \lambda x \,.\, 1 + x^2
\end{equation*}
%
in teorijo funkcijskih predpisov poimenoval \df{$\lambda$-račun}. V logiki se je njegov
zapis obdržal in se uveljavil tudi v programski jezikih:
%
\begin{itemize}
\item v Pythonu pišemo \verb|lambda x: 1 + x ** 2|,
\item v Haskellu pišemo \verb|\x -> 1 + x ** 2| in
\item v OCamlu pišemo \verb|fun x => 1 + x * x|.
\end{itemize}
%
Predvsem v programiranju funkcijskim predpisom pravijo tudi \df{anonimne} ali \df{brezimne
  preslikave}.

Nekateri starejši zapisi funkcijskih predpisov so slabi, a jih ljudje vztrajno
uporabljajo. Opozorimo le na en slab zapis, ki povzroča precej preglavic, ne da bi se
matematiki tega zares zavedali. Funkcijski predpis mora določati vezano spremenljivko,
sicer ne vemo, kako vstaviti vrednosti. Na žalost jo matematiki pogosto izpustijo skupaj
z $\mapsto$ in pišejo $1 + x^2$ namesto $x \mapsto 1 + x^2$.
%
Težava je v tem, da se lahko v funkcijskem predpisu pojavi več kot en simbol. Če vam na primer nekdo poe, da ima v mislih funkcijski predpis
%
\begin{equation*}
  a \cdot x + b
\end{equation*}
%
boste zaradi ustaljenih navad v šolskem sistemu vsi mislili, da je mišljeno $x \mapsto a \cdot x + b$.
%
A pravzprav bi lahko bilo tudi $a \mapsto a \cdot x + b$ ali $b \mapsto a \cdot x + b$ ali celo
$t \mapsto a \cdot x + b$! Namreč, nič ni narobe s funkcijskim predpisom, v katerem se
pojavijo dodatni simboli.

Morda pa lahko vezano spremenljivko in $\mapsto$ brez škode izpustimo, če v izrazu nastopa
samo en simbol, denimo $1 + x^2?$
%
A spet bi zabredli v težave. Je $42$ število ali funkcijski predpis $x \mapsto 42$? Je
$1 + x^2$ funkcijski predpis $x \mapsto 1 + x^2$ ali $a \mapsto 1 + x^2$?

Velikokrat površno rečemo, da funkcijski predpis podaja preslikavo. To ni res, saj smo že
prej povedali, da ima vsaka preslikava tri sestavne dele: domeno, kodomeno in prirejanje.
Res, če ne poznamo domene, ne moremo preveriti, ali je funkcijski predpis celovit. Denimo,
funkcijski predpis
%
\begin{equation*}
  x \mapsto \frac{x}{x^2 - 2}
\end{equation*}
%
ni celovit, če je domena množica realnih števil, in je celovit, če je domena množica
racionalnih števil. Tudi kodomeno moramo poznati, sicer ne moremo določiti nekaterih
lastnosti preslikave, kot je na primer surjektivnost, glej
razdelek~\ref{razdelek:injektivnost-in-surjektivnost}.



\subsection{Ostali načini podajanja preslikav}
\label{sec:ostali-predpisi}

Funkcijski predpisi niso edini način za podajanje prirejanja, zato omenimo še nekatere
druge.

Preslikavo s končno domeno lahko podamo s tabelo, na primer:
%
\begin{center}
  $f : \set{1, 2, 3, 5} \to \set{10, 20, 30}$

  \medskip

  \begin{tabular}{|c|c|} \hline
    1 & 10 \\ \hline
    2 & 10 \\ \hline
    3 & 20 \\ \hline
    5 & 10 \\ \hline
  \end{tabular}
\end{center}
%
To seveda pomeni, da $f$ elementu $1$ priredi $10$, $2$ priredi $10$, $3$ priredi $20$ in $5$
priredi $10$. Tabelo lahko predstavimo na različne načine, lahko kar naštejemo vsa prirejanja:
%
\begin{align*}
  f(1) &\dfeq 10 \\
  f(2) &\dfeq 10 \\
  f(3) &\dfeq 20 \\
  f(5) &\dfeq 10.
\end{align*}
%
Tudi
%
\begin{align*}
  1 &\mapsto 10 \\
  2 &\mapsto 10 \\
  3 &\mapsto 20 \\
  5 &\mapsto 10.
\end{align*}
%
je še vedno le tabela, ki prikazuje prirejanje. Ne sme nas motiti dejstvo, da smo
$\mapsto$ uporabili za naštevanje prirejanj, namesto za funkcijski predpis.

Preslikava je lahko določena tudi z opisom računskega postopka, pravimo mu \df{algoritem},
s pomočjo katerega izračunamo vrednost preslikave pri danem argumentu. Paziti moramo, da je
opis postopka res natančen in nedvoumen, lahko ga kar zapišemo kot program. Teoretični
računalničar bi pripomnil, da je treba pri tem izbrati programski jezik, ki ima ustrezno
matematično definicijo.

Preslikave lahko podamo tudi tako, da opišemo pogoje, pri katerih je element kodomene
prirejen elementu domene. Na primer, lahko bi definirali preslikavo $g : \NN \to \ZZ$ z
zahtevo, da naravnemu številu $n \in \NN$ priredi celo število $k \in \ZZ$, kadar velja
%
\begin{equation*}
  k^2 \leq n < (k+1)^2.
\end{equation*}
%
To prirejanje je veljavno, če je celovito in enolično, česar ne bomo preverjali, lahko pa
poskusite sami. Nekaj prirejanj $g$ prikazuje naslednja razpredelnica:
%
\begin{align*}
0 &\mapsto 0   &   4 &\mapsto 2   &    8  &\mapsto 2   &   12 &\mapsto 3 \\
1 &\mapsto 1   &   5 &\mapsto 2   &    9  &\mapsto 3   &   13 &\mapsto 3 \\
2 &\mapsto 1   &   6 &\mapsto 2   &    10 &\mapsto 3   &   14 &\mapsto 3 \\
3 &\mapsto 1   &   7 &\mapsto 2   &    11 &\mapsto 3   &   15 &\mapsto 3
\end{align*}
%
Ali znate z besedami opisati preslikavo~$g$?

Osnovne načine podajanja preslikav bomo spoznali skupaj s
konstrukcijami množic.
%
V splošnem je lahko preslikava podana s precej zapleteno konstrukcijo, ki zahteva veliko
preverjanja in dokazovanja.

\subsection{Aplikacija in substitucija}
\label{sec:aplikacija-in-subsitucija}

Do sedaj smo se ukvarjali s tem, kako preslikavo podamo, zdaj pa se vprašajmo, kako lahko
preslikavo uporabimo. Če je $f : X \to Y$ preslikava iz $X$ v $Y$ in je $x \in X$, potem
lahko \df{$f$ uporabimo na $x$} in dobimo \df{vrednost} preslikave~$f$ pri
\df{argumentu}~$x$, to je tisti edini element $Y$, ki ga~$f$ priredi~$x$. Vrednost $f$
pri~$x$ zapišemo
%
\begin{equation*}
  f(x)
  \qquad\text{ali}\qquad
  f\,x
\end{equation*}
%
in preberemo ``$f$ od $x$'' ali ``$f$ pri $x$''. Izraz $f(x)$, oziroma $f\,x$, se imenuje
\df{aplikacija}. Večinoma se uporablja zapis z oklepaji, a ne vedno: navajeni smo pisati
$\ln 2$ in $\sin \alpha$ namesto $\ln(2)$ in $\sin(\alpha)$. Oklepaje izpuščamo tudi v
nekaterih programskih jezikih in občasno v algebri.

V analizi je uveljavljen še en zapis za aplikacijo, ki se uporablja za zaporedja. Namreč,
zaporedje ni nič drugega kot preslikava $a : \NN \to \RR$ iz naravnih v realna števila.
Aplikacijo $a(n)$, ki označuje $n$-ti člen zaporedja, ponavadi pišemo~$a_n$, torej
argument podpišemo.

Preslikavo lahko uporabimo na argumentu tudi, če je nismo poimenovali. Na primer,
preslikavo $\RR \to \RR$, podano s funkcijskim predpisom
%
\begin{equation*}
  x \mapsto 1 + x^2
\end{equation*}
%
uporabimo na argumentu~$3$:
%
\begin{equation*}
  (x \mapsto 1 + x^2)(3).
\end{equation*}
%
Se vam zdi tak zapis nenavaden? Verjetno, a pomislite, zakaj: ker so vas vzgojili, da
je treba vse preslikave vedno poimenovali in se nanje skliceval z njihovim imenom.\footnote{Če bi veljalo enako tudi yza števila, vam v srednji šoli ne bi pustili pisati kar $3 + 5$, nujno bi bilo poimenovanje $a \dfeq 3 + 5$. Tudi trikotnika ne bi smeli narisati, ne da bi mu dali simbolno ime.}
%
Navkljub dobri vzgoji, bomo s preslikavami delali enako kot s števili, vektorji in
ostalimi matematičnimi objekti. Računalničarji radi rečejo, da
je treba tudi preslikave obravnavati kot ``enakopravne državljane''.

Kako pravzaprav določimo vrednost funkcije pri danem argumentu? To je odvisno od tega,
kako je podano prirejanje. Če imamo tabelarični prikaz, poiščemo argument v levem stolpcu
in pogledamo v pripadajoči desni stolpec. Če je preslikava podana s funkcijskim predpisom, argument
vstavimo v predpis. Na primer, če je $f : \RR \to \RR$ podana s funkcijskim predpisom
%
\begin{equation*}
  f(x) = 1 + x^2,
\end{equation*}
%
potem je vrednost $f(3)$ enaka $1 + 3^2$, kar je seveda enako~$10$, a to zahteva dodaten
račun, ki nas v tem trenutku ne zanima. Pravimo, da smo simbol~$x$ \df{zamenjali} ali
\df{substituirali} s~$3$, oziroma da smo~$3$ \df{vstavili} v~$f$ namesto~$x$. Seveda lahko
vstavimo argument neposredno v funkcijski predpis, zato je aplikacija
%
\begin{equation*}
  (x \mapsto 1 + x^2)(3)
\end{equation*}
%
spet enaka $1 + 3^2$.

Preslikavo smemo uporabiti na poljubnem elementu domene, ki je lahko zapisan na bolj ali
manj zapleten način, pri čemer gre še vedno samo za zamenjavo. Na primer, v zgornjo
preslikavo~$f$ lahko vstavimo $3 + 4$ in dobimo $1 + (3 + 4)^2$ ali pa za neki $u \in \RR$
vstavimo $u + 2$ in dobimo $1 + (u + 2)^2$. V razdelku~\ref{sec:eksponent} bomo spoznali
še dodatna pravila za vstavljanje izrazov, ki se vrtijo okoli vezanih spremenljivk.


\subsection{Pravilo ekstenzionalnosti preslikav}

Podobno kot za množice tudi za preslikave velja pravilo ekstenzionalnosti, ki pravi, da sta preslikavi enaki, če imata enako domeno in kodomeno ter prirejata argumentom enake vrednosti.


\begin{pravilo}[Ekstenzionalnost preslikav]
  Preslikavi sta enaki, če imata enaki domeni in kodomeni ter imata za vse argumente
  enaki vrednosti.
\end{pravilo}

Natančneje, če sta $f : A \to B$ in $g : C \to D$ preslikavi in velja $A = C$, $B = D$ ter
za vsak $x \in A$ velja $f(x) = g(x)$, tedaj velja $f = g$.

Takoj opozorimo na razliko med
%
\begin{equation*}
  f(x) = g(x)
  \qquad\text{in}\qquad
  f = g
\end{equation*}
%
saj bi marsikdo trdil, da med njima ni razlike. Levi izraz pravi, da sta $f(x)$ in $g(x)$
enaka elementa množice $C$, desni pa da sta $f$ in~$g$ enaki preslikavi iz $A$ v $B$. Na
sploh je treba razlikovati med $f$ in $f(x)$, saj to nikakor nista enaka objekta: prvi je
preslikava, drugi pa vrednost te preslikave pri~$x$. Verjetno nihče ne bi trdil, da je
preslikava $\cos$ isto kot $\cos \frac{\pi}{4}$, ali ne? Isti razmislek veleva, da
$\cos x$ ni isto kot $\cos$, če tudi si mislimo, da je $x$ poljuben. Zmeda izhaja iz
neprimernega zapisa preslikav. Če bi že od malih nog pravilno uporabljali funkcijske
predpise, bi seveda vedeli, da pravilo ekstenzionalnosti za preslikave zagotavlja enakost
~$\cos$ in $x \mapsto \cos x$, oba pa sta različna od $\cos x$, ki sploh ni preslikava,
ampak neko realno število. Čeprav je število $\cos x$ odvisno od parametra~$x$, je še
vedno le število.

V bran tradicionalnemu zapisu pa moramo vseeno povedati, da se lahko \emph{dogovorimo} za
nekoliko napačen zapis, če to ne povzroča zmede. S tem se izognemu preveč birokratskemu
pisanju nebistvenih podrobnosti in lahko bistveno izboljšamo komunikacijo in razumevanje
med izkušenimi matematiki. A začetnikom priporočamo, da v dobrobit boljšega razumevanja
snovi vsaj na začetku študija raje vztrajajo pri doslednem zapisu.

Vrnimo se še k pravilu ekstenzionalnosti preslikav. Ali ni pravzaprav očitno, da sta
preslikavi enaki, če imata enaki domeni, kodomeni in vrednosti? Morda res, a to ni razlog,
da tega ne bi eksplicitno zapisali. Vsak matematik vam ve povedati kako zgodbo o tem,
kako se je v dokazu skrivala napako ravno tam, kjer je bilo nekaj ``očitno''. Poleg tega
pa si lahko predstavljamo razmere, v katerih je smiselno razlikovati med dvema
preslikavama, ki imata vedno enake vrednosti, denimo v programiranju, kjer je učinkovitost
zelo pomembna.

%% STAR MATERIAL OD DAVORINA. Preveriti, kaj od tega je treba dati v besedilo, in kam.

% Množice ne obstajajo ločene ena od druge pač pa so med sabo povezane s
% \df{preslikavami} oziroma s tujko \df{funkcijami}.  Posamična preslikava slika elemente ene
% množice po določenem predpisu v elemente druge množice.

% Če je $f$ preslikava, ki slika iz množice $X$ v množico $Y$, to zapišemo
% %
% \begin{equation*}
%   f : X \to Y.
% \end{equation*}
% %
% Rečemo, da je množica~$X$ \df{začetna množica} ali \df{domena} preslikave~$f$, množica~$Y$
% pa je \df{ciljna množica} ali \df{kodomena} preslikave $f$.


% Običaj je, da predpis preslikave podamo s pomočjo spremenljivke, tipično z oznako $x$. Na primer, če je $f$ preslikava kvadriranja, njen predpis zapišemo kot
% \[f(x) = x^2.\]
% Na tem mestu je potrebno poudariti več reči.
% \begin{itemize}
% \item
% Velikokrat površno rečemo, da zgornji predpis podaja preslikavo. To ni povsem res --- to je zgolj predpis preslikave. Za to, da preslikavo v celoti podamo, je potrebno navesti tri stvari: poleg predpisa še domeno in kodomeno. Vse to je del informacije o preslikavi.

% To se jasno pokaže, če začnemo razmišljati o lastnostih preslikav. Se še spomnite iz srednje šole, kaj pomeni, da je preslikava surjektivna? (Bomo ponovili v razdelku~\ref{razdelek:injektivnost-in-surjektivnost}.) Če vzamemo, da preslikava $f$ zadošča zgornjemu predpisu in jo obravnavamo kot preslikavo $f\colon \RR \to \RR$, ni surjektivna, če jo obravnavamo recimo kot preslikavo $f\colon \RR_{\geq 0} \to \RR_{\geq 0}$, pa je.
% \item
% Za spremenljivko $x$ velja isto, kot smo razpravljali že v prejšnjem razdelku pri lastnostih elementov množic: spremenljivka $x$ nima vnaprej določene vrednosti, pač pa predstavlja mesto, kamor lahko vstavimo poljubno vrednost. Seveda je potem vseeno, če vzamemo kakšno drugo črko ali čisto drug simbol: $f(y) = y^2$ določa isti predpis kot $f(x) = x^2$; prav tako $f(\heartsuit) = \heartsuit^2$. Se pravi, tudi v tem primeru gre za nemo spremenljivko. Če si torej izberemo neko vrednost, jo lahko vstavimo na mesto spremenljivke in izračunamo vrednost dobljenega izraza, npr.~$f(3) = 3^2 = 9$ oziroma $f(2\pi) = (2\pi)^2 = 4\pi^2$. Predstavljajte si, da je spremenljivka pravzaprav škatlica, kamor lahko vstavite vrednost, torej
% \[f(\argbox) = \argbox^2.\]
% \item
% Alternativen način zapisa $f(x) = x^2$ je
% \[f\colon x \mapsto x^2.\]
% Pazimo: navadna puščica $\to$ podaja domeno in kodomeno, kot razloženo zgoraj. Repata puščica $\mapsto$ pa za posamičen element domene pove, v kateri element kodomene se preslika.

% Zapis z repato puščico je še posebej uporaben, kadar želimo podati preslikavo, ne da bi nam bilo potrebno izbrati ime zanjo. Na primer, realno funkcijo kvadriranja lahko v celoti podamo takole:
% \begin{align*}
% \RR &\to \RR \\
% x &\mapsto x^2
% \end{align*}
% (prva vrstica pove domeno in kodomeno, druga pa predpis). Tako podanim preslikavam potem rečemo \df{brezimne preslikave} (s tujko \df{anonimne funkcije}). Kasneje (v razdelku~\ref{razdelek:brezimne-preslikave}) bomo spoznali bolj strnjen zapis takih preslikav, ki je še posebej primeren za izvajanje operacij med preslikavami; takrat bomo takšno funkcijo zapisali kot $\lam{x \in \RR} x^2$.
% \end{itemize}

% \note{Sklop (kompozicija, kompozitum) preslikav. Identiteta kot enota za sklapljanje. Razčlenitev (dekompozicija, faktorizacija) preslikav.}

% \davorin{Definirati moramo tudi oznako $\set{f(x)}{x \in X}$, kar je druge vrste oznaka kot prej definirana $\set{x \in X}{\phi(x)}$. Se gremo primerjavo s Pythonom (razlika med \texttt{\{f(x) for x in X\}} in \texttt{\{x if phi(x)\}})? Smo matematični hipsterji in uvedemo oznako $\{f(x) \,|\, x \in X \,|\, \phi(x)\}$, ki ustreza \texttt{\{f(x) for x in X if phi(x)\}}, kar bi tudi prišlo prav?}

% Zaenkrat smo imeli primere, ko je bil prepis preslikave dan z eno samo spremenljivko, npr.~$f(x) = x^2$. Zelo pogoste so pa tudi \df{preslikave več spremenljivk}, npr.~$f(x, y) = x^2 + y^2$. Že osnovne računske operacije so take --- na primer, pri seštevanju vzamemo \emph{dva} podatka in vrnemo rezultat (vsoto).

% V takem primeru je smiselno reči: domena preslikave sestoji iz \df{dvojic} ali \df{parov} števil. Pri seštevanju je to, katero število je prvo, katero pa drugo, sicer nepomembno, pri kakšni drugi operaciji (npr.~že odštevanju), pa je, zato posebej zahtevajmo: gre za \df{urejene dvojice} (\df{pare}). Urejeno dvojico elementov $a$ in $b$ (v tem vrstem redu) po dogovoru zapišemo kot $(a, b)$. Vrednosti $a$ in $b$ imenujemo \df{komponenti} tega para; natančneje, $a$ je \df{prva komponenta}, $b$ pa \df{druga komponenta}.

% Če imamo dve množici $A$ in $B$, tedaj množico vseh urejenih dvojic, katerih prva komponenta je element iz $A$, druga komponenta pa element iz $B$, označimo $A \times B$ in imenujemo \df{zmnožek} ali \df{produkt} množic $A$ in $B$. Glede na to, da obstaja mnogo operacij, ki se imenujejo ``produkt'' (poznate že vsaj produkt števil, produkt števila z vektorjem, skalarni produkt vektorjev in vektorski produkt vektorjev, obstaja pa jih še precej več), je koristno produkt množic posebej poimenovati, da ga ločimo od drugih: zanj se je uveljavil izraz \df{kartezični produkt} (izhaja iz imena Cartesius, tj.~latinske različice priimka Renéja Descarta\footnote{René Descartes (1596 -- 1650) je bil francoski filozof, matematik in znanstvenik.}).

% Seštevanje potemtakem lahko razumemo kot preslikavo $+\colon \RR \times \RR \to \RR$. V tem smislu še vedno gre za preslikavo, ki dan vhodni podatek preslika v neki rezultat, le da je vhodni podatek dvojica števil, ne pa zgolj eno število. Kadar imamo produkt več enakih faktorjev, ga lahko (kot običajno) zapišemo v obliki potence; pisali bi lahko tudi $+\colon \RR^2 \to \RR$.

% Seveda nismo omejeni na preslikave samo ene ali dveh spremenljivk. Nič nam ne preprečuje definirati recimo $f(x, y, z) = 2x + y - 3z$. Smiselna domena te preslikave setoji iz \df{urejenih trojic} števil. V splošnem, če jemljemo elemente iz množic $A$, $B$, $C$, tedaj se množica vseh takih trojic označi z $A \times B \times C$. Prejšnji predpis določa potem preslikavo $f\colon \RR \times \RR \times \RR \to \RR$ (oziroma krajše $f\colon \RR^3 \to \RR$).

% Spremenljivk je lahko še več; poleg dvojic in trojic tako dobimo še četverice, peterice, šesterice\ldots V splošnem takšna končna zaporedja elementov imenujemo \df{urejene večterice}. Tudi število spremenljivk je lahko označeno s črko; na primer, preslikava, ki računa povprečje $n$ števil (kjer $n \in \NN_{\geq 1}$), je dana kot
% \begin{align*}
% \RR^n &\to \RR \\
% (x_1, x_2, \ldots, x_n) &\mapsto \frac{x_1 + x_2 + \ldots + x_n}{n}
% \end{align*}
% (če hočemo poudariti, da imajo naše večterice natanko $n$ komponent, jih imenujemo $n$-terice). Nadlega pri tem je sicer spet dvoumnost tropičja. Deloma jo je možno odpraviti tako, da celotno večterico označimo z eno spremenljivko. Pogosta izbira zapisa je $f(\mathbf{x})$ ali $f(\vec{x})$ (razlog za to je, da lahko večterico vidimo kot vektor).

% Marsikdaj želimo delati ne samo z eno preslikavo, pač pa s celo množico preslikav naenkrat. Zato uvedemo: množica vseh preslikav, ki slikajo iz $X$ v $Y$, se označi kot $Y^X$; temu se reče \df{eksponent} množic $X$ in $Y$ (\note{na primernem mestu kasneje} bomo razložili, od kod ta oznaka).

% \begin{zgled}
% Množico vseh preslikav, ki realna števila slikajo nazaj v realna števila, označimo z $\RR^\RR$. Če nas zanimajo realne preslikave, ki so definirana samo na intervalu $\intoo{-1}{1}$, opazujemo množico $\RR^{\intoo{-1}{1}}$. Definiramo lahko preslikavo
% \begin{align*}
% \RR^{\intoo{-1}{1}} &\to \RR \\
% f &\mapsto f(0),
% \end{align*}
% ki preslikavam priredi njihovo vrednost v točki $0$. Ta preslikava torej ima za argumente (tj.~vnose) celotne preslikave in ne števila! Sama po sebi je element množice $\RR^{\RR^{\intoo{-1}{1}}}$.
% \end{zgled}

% \begin{zgled}
% Za poljubne množice $A$, $B$, $C$ lahko definiramo sledečo preslikavo, katere argumenti so pari preslikav.
% \begin{align*}
% B^A \times C^B &\to C^A \\
% (f, g) &\mapsto g \circ f
% \end{align*}
% \end{zgled}


% \davorin{Glede na to, da gre za slovenski učbenik, dajem izrazu `preslikava' prednost pred izrazom `funkcija'. Seveda pa sem pojasnil tudi slednji izraz (v prvem poglavju).}

% \note{Uvod. Definicijsko območje in zaloga vrednosti \davorin{morda dodamo kot možno ime za zalogo vrednosti še prevod angleške besede `range', se pravi `razpon'?}. Zožitve (tako domene kot kodomene); oznake za to so $\rstr{f}_A$, $\rstr{f}^B$, $\rstr{f}_A^B$. Izvrednotenje (evalvacija) preslikave (če ne bomo tega pojasnili že pri eksponentih množic).}

\section{Osnovne konstrukcije množic}

Množice lahko \df{tvorimo} ali \df{konstruiramo} iz drugih množic na različne načine. V
tem poglavju bomo spoznali tri osnovne konstrukcije, ostale pa kasneje, ko bomo že nekaj
vedeli o logiki.

Takoj se zastavi vprašanje, kako sploh opisati novo konstrukcijo množic. Pravilo
ekstenzionalnosti pove, da je množica opredeljena s svojimi elementi. Torej moramo
pojasniti, kaj so elementi nove množice, se pravi, kako jih vpeljemo, kaj lahko z njimi
počnemo in kakšne so njihove zakonitosti. Natančneje, novo konstrukcijo množic
določajo naslednja pravila:
%
\begin{enumerate}
\item pravilo \df{tvorbe}, ki vpelje novo množico,
\item pravila \df{vpeljave} elementov, ki podajo operacije, s katerimi gradimo elemente,
\item pravila \df{uporabe}, ki podajo opreacije, s katerimi razgradimo ali uporabimo elemente,
\item \df{enačbe}, ki opredeljujejo zakonitosti, ki veljajo za operacije vpeljave in uporabe.
\end{enumerate}
%
Najbolje je, da si postopek ogledamo na primeru.

\subsection{Zmnožek}
\label{sec:zmnozek}

Najprej obravnavajmo zmnožek ali kartezični produkt.

\begin{pravilo}[Tvorba zmnožka]
  \label{pravilo:zmnozek-tvorba}
  Za vsaki množici $A$ in $B$ je $A \times B$ množica, ki se imenuje \df{zmnožek} ali
  \df{kartezični produkt} $A$ in $B$.
\end{pravilo}

\noindent
%
Pravilo tvorbe pove, da lahko tvorimo novo množico $A \times B$, ne pove pa, kakšne
elemente ima. To je vsebina naslednjih dveh pravil, ki povesta, kako sestavimo in
razstavimo elemente zmnožka.

\begin{pravilo}[Vpeljava urejenih parov]
  \label{pravilo:zmnozek-vpeljava}
  %
  Za vse $a \in A$ in $b \in B$ je $(a, b) \in A \times B$. Element $(a, b)$ imenujemo
  \df{urejeni par}.
\end{pravilo}

\begin{pravilo}[Uporaba urejenih parov]
  \label{pravilo:zmnozek-uporaba}
    %
  Za vsak $p \in A \times B$ je $\fst(p) \in A$ \df{prva projekcija} in $\snd(p) \in B$
  \df{druga projekcija} elementa~$p$.
\end{pravilo}

Nazadnje podamo še enačbe.

\begin{pravilo}[Računsko pravilo za urejene pare]
  \label{pravilo:zmnozek-racunanje}
  Za vse $a \in A$, $b \in B$ velja $\fst(a, b) = a$ in $\snd(a, b) = b$.
\end{pravilo}

\begin{pravilo}[Ekstenzionalnost urejenih parov]
  \label{pravilo:zmnozek-ekstenzionalnost}
  Za vse $p, q \in A \times B$ velja: če $\fst(p) = \fst(q)$ in $\snd(p) = \snd(q)$,
  potem $p = q$.
\end{pravilo}

\noindent
%
Računsko pravilo se tako imenuje, ker lahko z njim poenostavljamo izraze, drugo pa je
pravilo ekstenzionalnosti, ker pravi, da je urejeni par določen s prvo in drugo projekcijo.

Kadar imamo opravka z večimi zmnožki, na primer $A \times B$ in $C \times D$, bi lahko
prišlo do zmede glede projekcij. Takrat jih opremimo še z dodatnimi oznakami množic, da
razločimo projekciji $\fst[A][B] : A \times B \to A$ in $\fst[C][D] : C \times D \to C$,
in podobno za~$\snd$.

Malo bolj naivna konstrukcija zmnožka bi se glasila takole: kartezični produkt
$A \times B$ je množica vseh urejenih parov $(a, b)$, kjer je $a \in A$ in $b \in B$. A
taka konstrukcija ni popolna, saj ne pove, kaj lahko z urejenim parom počnemo. Kako naj
vemo, da iz $(a, b)$ lahko izluščimo $a$ in $b$, in kako preverimo, ali sta dva urejena
para enaka? Če takih zadev ne določimo, bi lahko kdo mislil, da je urejeni par kaka druga
operacija, denimo seštevanje, unija, ali kdovekaj.

Dejstvo, da je vsak element zmnožka množic urejen par, in to celo na en sam način, lahko
dokažemo.

\begin{trditev}
  Naj bosta $A$ in $B$ množici. Za vsak element $p \in A \times B$ obstaja natanko en
  $a \in A$ in natanko en $b \in B$, da velja $p = (a, b)$.
\end{trditev}

\begin{dokaz}
  Naj bosta $A$ in $B$ množici in $p \in A \times B$. Najprej pokažimo, da $p$ res je enak
  nekemu urejenemu paru, namreč
  %
  \begin{equation*}
    p = (\fst(p), \snd(p)).
  \end{equation*}
  %
  Uporabimo pravilo ekstenzionalnosti za pare, ki nam zagotavlja to enačbo, če dokažemo
  %
  \begin{equation*}
    \fst(p) = \fst(\fst(p), \snd(p))
    \qquad\text{in}\qquad
    \snd(p) = \snd(\fst(p), \snd(p)).
  \end{equation*}
  %
  Ti dve enačbi pa veljata, ker sta primerka računskih pravil za pare.

  Preveriti moramo še, da je $(\fst(p), \snd(p))$ edini urejeni par, ki je enak~$p$.
  Povedano z drugimi besedami, dokazati moramo: če je $p = (a, b)$ za neki $a \in A$ in
  $b \in B$, potem velja $a = \fst(p)$ in $b = \snd(p)$. Pa denimo, da bi za neki
  $a \in A$ in $B \in B$ veljalo $p = (a,b)$. Tedaj bi lahko uporabili računska pravila za
  pare in dobili
  %
  \begin{equation*}
    \fst(p) = \fst(a, b) = a
    \qquad\text{in}\qquad
    \snd(p) = \snd(a, b) = b,
  \end{equation*}
  %
  kar smo želeli dokazati.
\end{dokaz}

Trditev je prikladna, ko želimo podati funkcijsko pravilo za preslikavo, katere domena je
zmnožek množic. Primer take preslikave je
%
\begin{gather*}
  \RR \times \RR \to \RR \\
  p \mapsto \fst(p) + \snd(p)^2 \cdot \fst(p).
\end{gather*}
%
Ta zapis je precej nepregleden, a sledili smo navodilu, da mora stati na levi strani
funkcijskega predpisa simbol. Prejšnja trditev nam zagotavlja, da lahko vsak element
$\RR \times \RR$ na en sam način izrazimo kot urejeni par $(x, y)$, in zato ne bo nič
narobe, če zapišemo ta isti funkcijski predpis bolj pregledno tako, da upoštevamo, da
je $p$ enak $(x, y)$ za enolično določena $x$ in $y$:
%
\begin{gather*}
  \RR \times \RR \to \RR \\
  (x, y) \mapsto x + y^2 \cdot x.
\end{gather*}
%
Če bi funkcijo poimenovali, denimo $f$, bi dobili običajni zapis:
%
\begin{gather*}
  f : \RR \times \RR \to \RR \\
  f(x, y) \dfeq x + y^2 \cdot x.
\end{gather*}
%
Za tako preslikavo pravimo, da je ``funkcija dveh spremenljivk'', ker si mislimo, da smo
podali argumenta $x$ in $y$ ločeno drug od drugega. A lahko rekli tudi, da je to funkcija
ene spremenljivke, ki jo uporabimo na urejenem paru:
%
\begin{gather*}
  f : \RR \times \RR \to \RR \\
  f(p) \dfeq \fst(p) + \snd(p)^2 \cdot \fst(p).
\end{gather*}


Poleg zmnožka dveh množic bi lahko tvorili tudi zmnožek treh ali več množic. Pravila bodo
podobna kot za zmnožek dveh množic, le da bi namesto urejenih parov tvorili \df{urejene
  večterice} in da bi imeli več projekcij. Za vsako projekcijo bi zapisali eno računsko
pravilo, princip ekstenzionalnosti pa bi bil tudi podoben tistemu za urejene pare.
Podorobnosti prepustimo za vajo.


\subsection{Vsota}
\label{sec:vsota}

Spoznali smo že unijo $A \cup B$ množic $A$ in $B$, ki vsebuje tiste elemente, ki so v $A$
ali v $B$. Če imata $A$ in $B$ skupne elemente, bodo ti v uniji seveda nastopili samo
enkrat. V skranjem primeru dobimo $A \cup A = A$. Včasih pa želimo združiti množici tako,
da ne pride do prekrivanja. Taka konstrukcija je \df{vsota} $A + B$ množic $A$ in $B$.
Prekrivanje preprečimo tako, da elemente, ki jih je prispevala~$A$ označimo z eno oznako,
tiste, ki jih je prispevala~$B$, pa z drugo.

\begin{pravilo}[Vsota]
  \label{vsota:tvorba}
  Za vsaki množici $A$ in $B$ je $A + B$ množica, ki se imenuje \df{vsota} ali
  \df{koprodukt} množic $A$ in $B$.
\end{pravilo}

\begin{pravilo}[Vpeljava elementov vsote]
  \label{vsota:vpeljava}
  Za vsaki množici $A$ in $B$ velja:
  %
  \begin{enumerate}
  \item za vsak $a \in A$ je $\inl(a) \in A + B$,
  \item za vsak $b \in B$ je $\inr(b) \in A + B$.
  \end{enumerate}
\end{pravilo}

S pravilom vpeljave smo pojasnili, da uporabljamo oznaki $\inl$ in $\inr$, prvo za
elemente iz~$A$ in drugo za elemente iz~$B$. Oznakama pravimo tudi
\df{injekciji}\footnote{Ni pomembno, kako poimenujemo oznaki, da sta le
  različni. V funkcijskem programiranju, kjer poznamo vsote podatkovnih tipov, programer
  sam določi, kakšne oznake bo uporabljal za injekcije.} in sta preslikavi
%
\begin{equation*}
  \iota_1 : A \to A + B
  \qquad\text{and}\qquad
  \iota_2 : B \to A + B.
\end{equation*}
%
Kadar imamo opravka z večimi vsotami, na primer $A + B$ in $C + D$, bi lahko prišlo do
zmede glede oznak. Takrat injekcije opremimo še z dodatnimi oznakami množic, da razločimo
injekciji $\inl[A][B] : A \to A + B$ in $\inl[C][D] : C \to C + D$, in podobno za~$\inr$.

Potrebujemo še pravili za uporabo in enakost elementov vsote, ki ju združimo v eno samo
pravilo.

\begin{pravilo}
  \label{vsota:uporaba}
  Za vsaki množici $A$ in $B$ in za vsak $u \in A + B$, bodisi obstaja natanko en
  $a \in A$, da je $u = \inl(a)$, bodisi obstaja natanko en $b \in B$, da je
  $u = \inr(b)$.
\end{pravilo}

V zgornjem pravilu fraza ``bodisi \dots bodisi \dots'' pomeni, da velja prva ali druga možnost, a ne obe hkrati.
%
S tem smo v $A + B$ res ločili elemente $A$ od elementov $B$, saj velja $\inl(a) \neq \inr(b)$, tudi ko je $A = B$ in $a = b$. 
%
Fraza ``natanko en $a \in A$'' pove, da iz $u = \inl(a_1)$ in $u = \inl(a_2)$ sledi $a_1 = a_2$.
Povedano drugače, če velja $\inl(a_1) = \inl(a_2)$, potem je $a_1 = a_2$. Podobno iz
$\inr(b_1) = \inr(b_2)$ sledi $b_1 = b_2$.
%
Podajmo prepost primer, ki verjetno marsikaj pojasni:
%
\begin{equation*}
  \set{a, b, c} + \set{a, d, e} =
  \set{\inl(a), \inl(b), \inl(c), \inr(a), \inr(d), \inr(e)}.
\end{equation*}

Kako definiramo preslikavo $A + B \to C$? Ker je vsak element domene $A + B$ bodisi
$\inl(a)$ za neki $a \in A$ bodisi $\inr(b)$ za neki $b \in B$, \emph{obravnavamo oba
  primera}. Tako funkcijski zapis za preslikavo $A + B \to C$ zapišemo kot
%
\begin{equation*}
  u \mapsto
  \begin{cases}
    \cdots a \cdots & \text{če $u = \inl(a)$,}\\
    \cdots b \cdots & \text{če $u = \inr(b)$,}
  \end{cases}
\end{equation*}
%
kjer smemo v $\cdots a \cdots$ zapisati izraz, ki vsebuje simbol~$a$, in v
$\cdots b \cdots$ izraz, ki vsebuje simbol~$b$. Ker je tak zapis nekoliko neroden, se
dogovorimo, da ga lahko zapišemo tudi s \emph{večdelnim} funkcijskim predpisom:
%
\begin{align*}
  \inl(a) &\mapsto \cdots a \cdots, \\
  \inr(b) &\mapsto \cdots b \cdots.
\end{align*}
%
Če želimo preslikavo poimenovati, zapišemo
%
\begin{align*}
  f &: A + B \to C, \\
  f(\inl(a)) &\dfeq \cdots a \cdots \\
  f(\inr(b)) &\dfeq \cdots b \cdots.
\end{align*}
%
Vsi ti zapisi res določajo celovito in enolično prirejanje, saj nam pravila za vsoto
zagotavljajo, da vedno obvelja natanko en primer. Na sploh lahko podamo funkcijski zapis z
večimi primeri, če le pazimo, da obravnavamo vse možnosti, in da se le-te ne prekrivajo.
Na primer, predpis
%
\begin{align*}
  (A + B) \times C &\to B + A \\
  (\inl[A][B](a), c) &\mapsto \inr[B][A](a) \\
  (\inr[A][B](b), c) &\mapsto \inl[B][A](b)
\end{align*}
%
je celovit in enoličen, medtem ko predpis
%
\begin{align*}
  (A \times A) + B &\to A \\
  \inl(a_1, a_2) &\mapsto a_2
\end{align*}
%
ni veljaven, ker ni celovit, saj manjka primer $\inr(b) \mapsto \cdots$.

Poleg vsote dveh množic bi lahko tvorili vsoto treh ali več množic. Pravila bi bila
podobna, le da bi imeli več injekcij in več primerov.

\subsection{Eksponent}
\label{sec:eksponent}

Denimo, da sta $A$ in $B$ množici. Tedaj lahko obravnavamo preslikave
%
\begin{equation*}
  A \to B
\end{equation*}
%
z domeno $A$ in kodomeno $B$. Ali vse take preslikave tvorijo množico? Russellov paradoks
nas je izučil, da moramo pazljivo postaviti pravila za konstrukcije množic, nato pa jih
strogo držati. Pravila, ki smo jih podali do sedaj, ne zagotavljajo knostrukcije množic
vseh preslikav iz $A$ v $B$. Potrebujemo novo pravilo.

\begin{pravilo}[Eksponent]
  Za vsaki množici $A$ in $B$ ima \df{eksponent} ali \df{eksponentna množica $B^A$} za
  elemente natanko vse preslikave iz~$A$ v~$B$.
\end{pravilo}

Potemtakem je zapis $f : A \to B$ enakovreden zapisu $f \in B^A$.

Pravila, ki opredeljujejo elemente množice $B^A$ smo že spoznali. Pravilo vpeljave pravi,
da je preslikava podana z domeno, kodomeno ter celovitim in enoličnim prirejanjem med
njima. Pravilo uporabe je kar aplikacija: če je $f \in B^A$ in $a \in A$, lahko tvorimo
$f(a) \in B$. Tudi računsko pravilo za preslikave smo že spoznali, saj je to kar pravilo
zamenjave: funkcijski predpis uporabimo na argumentu tako, da vezano spremenljivko v
predpisu zamenjamo z argumentom. In ekstenzionalnost preslikav pove, kdaj sta dve
preslikavi enaki.

Preslikavi, ki sprejme kot argument preslikavo, pravimo \df{funkcional} ali \df{preslikava
  višjega reda}. Primer take preslikave je \df{kompozicija}:
%
\begin{align*}
  {\circ} &: C^B \times B^A \to C^A \\
  {\circ} &: (g, f) \mapsto (x \mapsto g(f(x))).
\end{align*}
%
Pišemo jo kot operacijo, torej $g \circ f$ namesto ${\circ}(g, f)$. V zgornjem zapisu smo
uporabili eksponente, a v tem primeru je bolj pregleden diagram:
%
\begin{equation*}
  \xymatrix{
    {A}
    \ar[r]^{f}
    \ar@/_2ex/[rr]_{g \circ f}
    &
    {B}
    \ar[r]^{g}
    &
    {C}
  }
\end{equation*}
%
Zakaj smo $\circ$ definirali tako, da kompozicijo $f$ in $g$ pišemo $g \circ f$ namesto
$f \circ g$? Ker si je mnogo lažje zapomniti računsko pravilo
%
\begin{equation*}
  (g \circ f)(x) = g(f(x)),
\end{equation*}
%
ki velja z našo definicijo, kot pa $(f \circ g)(x) = g(f(x))$, kar bi veljalo, če bi
zamenjali vlogi~$f$ in~$g$.

\begin{trditev}
  \parbox{0pt}{}
  %
  \begin{enumerate}
  \item Identiteta je nevtralna za kompozicijo: $\id[B] \circ f = f = f \circ \id[A]$.
  \item Kompozicija je asociativna: $(h \circ g) \circ f = h \circ (g \circ f)$.
  \end{enumerate}
\end{trditev}

\begin{dokaz}
  Trditev je zapisana pomanjkljivo, saj ne piše, kaj so $A$, $B$,$ f$ in~$g$. Avtorja
  trditve bi lahko vprašali, kaj je hotel povedati, a je bolje, da poskusimo to razvozlati
  sami, ker je to odlična vaja iz razumevanja matematičnih besedil.

  Takoj vidimo, da je $A$ množica, sicer zapis $\id[A]$ ne bi bil smiselen, in podobno je
  tudi $B$ množica. Simboli $f$, $g$ in $h$ zagotovo označujejo preslikave, saj nastopajo
  v kompoziciji. Kaj pa njihove domene in kodomene? Preslikava $f$ mora imeti domeno $A$,
  sicer ne bi bilo dovoljeno komponirati $f \circ \id[A]$, in mora imeti kodomeno $B$,
  sicer ne bi bilo dovoljeno komponirati $\id[B] \circ f$. Ostaneta še domeni in kodomeni
  preslikav $g$ in~$h$. Kompozicija $g \circ f$ kaže, da mora biti domena $g$ enaka
  kodomeni~$f$, torej $B$. Kompozicija $h \circ g$ pa pove, da je kodomena $C$ enaka
  domeni $h$. Če vse to zložimo v diagram, dobimo
  %
  \begin{equation*}
    \xymatrix{
      {A} \ar[r]^{f}
      &
      {B} \ar[r]^{g}
      &
      {\text{?}} \ar[r]^{h}
      &
      {\text{?}}
    }
  \end{equation*}
  %
  Trditev moramo razumeti tako, da bo čim bolj splošna in smiselna. Torej bomo za neznani
  množici vzeli kar poljubni množici $C$ in $D$:
  %
  \begin{equation*}
    \xymatrix{
      {A} \ar[r]^{f}
      &
      {B} \ar[r]^{g}
      &
      {C} \ar[r]^{h}
      &
      {D}
    }
  \end{equation*}
  %
  Preverimo, ali smo trditev pravilno razumeli. Ko vstavimo podrobnosti, se prvi del
  glasi: ``Za vse množice $A$ in $B$ ter preslikavo $f : A \to B$ velja
  $\id[B] \circ f = f = f \circ \id[A]$.''
  %
  Ker je to smiselna izjava, jo dokažimo. Enakost preslikav se dokaže z ekstenzionalnostjo
  preslikav, torej preverimo, ali imajo $\id[B] \circ f$, $f$ in $f \circ \id[A]$ enako
  vrednost za poljuben $x \in A$:
  %
  \begin{align*}
    (\id[B] \circ f)(x) &= \id[B] (f (x)) = f (x), \\
    f (x) &= f(x), \\
    (f \circ \id[A])(x) &= f (\id[A](x)) = f(x).
  \end{align*}
  %
  Zapišimo podrobno še drugi del: ``Za vse množice $A$, $B$, $C$ in $D$ ter preslikave
  $f : A \to B$, $g : B \to C$ in $h : C \to D$ velja
  $(h \circ g) \circ f = h \circ (g \circ f)$. To spet dokažemo tako, da uporabimo levo in
  desno stran enačbe na poljubnem $x \in A$:
  %
  \begin{align*}
    ((h \circ g) \circ f)(x) &= (h \circ g)(f(x)) = h(g(f(x))) \\
    (h \circ (g \circ f))(x) &= h((g \circ f)(x)) = h(g(f(x))). \qedhere
  \end{align*}
\end{dokaz}


Kompozicijo smo zapisali z \emph{vgnezdenim} funkcijskim predpisom, ki argumentu priredi
preslikavo, ki je spet podana s funkcijskim predpisom. V splošnem je vgnezdeni funkcijski
predpis oblike
%
\begin{align*}
  A &\mapsto C^B \\
  a &\mapsto (b \mapsto \cdots),
\end{align*}
%
kjer se lahko v $\cdots$ pojavita~$a$ in~$b$. Na tak zapis se je treba navaditi, a je zelo
prikladen, še posebej v funkcijskem programiranju. V matematiki ni zelo pogost, a mi se ga
ne bomo bali.

Pri računanju s preslikavami višjega reda včasih hkrati obravnavamo več funkicjskih
predpisov in lahko pride do zmede, če za vse uporabimo isto vezano spremenljivko. Na
primer, kompozitum preslikav
%
\begin{equation*}
  \begin{aligned}
    \RR &\to \RR \\
    x &\mapsto x^2 - 4
  \end{aligned}
  %
  \qquad\text{in}\qquad
  %
  \begin{aligned}
    \RR &\to \RR \\
    x &\mapsto 2 - x
  \end{aligned}
\end{equation*}
%
bi lahko izračunali takole:
%
\begin{align*}
  (x \mapsto x^2  - 4) \circ (x \mapsto 2 - x)
  &= (x \mapsto (x \mapsto x^2 - 4) ((x \mapsto 2 - x) x)) \\
  &= (x \mapsto (x \mapsto x^2 - 4) (2 - x)) \\
  &= (x \mapsto (2 - x)^2 - 4) \\
  &= (x \mapsto x^2 - 4 x).
\end{align*}
%
Tu imamo tri pojavitve $x$, ki bi jih morali ločiti, ker vsaka nastopa kot vezana
spremenljivka v svojem funkcijskem predpisu. Še posebej nejasen je računski korak
$(x \mapsto (x \mapsto x^2 - 4) (2 - x)) = (x \mapsto (2 - x)^2 - 4)$, ko vezano
spremenljivko~$x$ v funkcijskem predpisu zamenjamo z izrazom $2 - x$, ki tudi vsebuje~$x$.
To sta dva različna $x$-a! Spomnimo se, da lahko vezane spremenljivke vedno preminujemo.
Ponovimo račun, a tokrat tako, da imajo različni funkcijski predpisi različne vezane
spremenljivke. Kompozitum
%
\begin{equation*}
  \begin{aligned}
    \RR &\to \RR \\
    y &\mapsto y^2 - 4
  \end{aligned}
  %
  \qquad\text{in}\qquad
  %
  \begin{aligned}
    \RR &\to \RR \\
    z &\mapsto 2 - z
  \end{aligned}
\end{equation*}
%
izračunamo takole:
%
\begin{align*}
  (y \mapsto y^2  - 4) \circ (z \mapsto 2 - z)
  &= (x \mapsto (y \mapsto y^2 - 4) ((z \mapsto 2 - z) x)) \\
  &= (x \mapsto (y \mapsto y^2 - 4) (2 - x)) \\
  &= (x \mapsto (2 - x)^2 - 4) \\
  &= (x \mapsto x^2 - 4 x).
\end{align*}
%
To je dosti bolj pregledno. Da ne bo prihajalo do zapletov z vezanimi spremenljivkami, se
dogovorimo: \emph{kadar imamo opravka z večimi vezanimi spremenljivkami, jih vedno
  preimenujemo tako, da so med seboj različne.}

Funkcionale srečamo v analizi in funkcijskem programiranju. Limita zaporedja je
funkcional, ker sprejme kot argument zaporedje realnih števil, se pravi element $\RR^\NN$,
in mu priredi realno število. Odvod je funkcional, ki sprejme element $\RR^\RR$ in mu
priredi element $\RR^\RR$. Če smo povsem natančni, limita kot preslikava $\RR^\NN \to \RR$
ni celovit funkcional, ker nekatera zaporedja ne konvergirajo. Prav tako odvod kot
preslikave $\RR^\RR \to \RR^\RR$ ni celovit, ker nekatere preslikave niso odvedljive.
Preslikavam, ki niso celovite, pravimo \emph{delne} in o njih bomo več povedali v
razdelku~\ref{delne-preslikve}.

\section{Izomorfizem množic}
\label{sec:izomorfizem-mnozic}

Ko otrok prvič spozna pojem števila, je ta zanimiv sam po sebi. Z vnemo šteje do sto in se
rad pogovarja se o tem, koliko je en miljon. Sčasoma se radovednost osredotoči na
aritmetične operacije in, če ima mladenič ali mladenka v sebi matematično žilico, na
\emph{zakonitosti} števil: množenje z~$1$ nima učinka, vrstni red seštevanja ni pomemben
itd. Ali tudi operacijam na množicah, ki smo jih spoznali do sedaj, vladajo kakšne
podobne zakonitosti?

Za števili $a$ in $b$ velja $a \cdot b = b \cdot a$. Nekaj podobnega velja tudi za množici
$A$ in $B$ in njuna zmnožka $A \times B$ in $B \times A$. V splošnem sicer nista enaka, a
sta v nekem smislu enakovredna, ker lahko par $(x, y) \in A \times B$ pretvorimo v par
$(y, x) \in B \times A$ in obratno. Ta razmislek vodi do pojma izomorfizma.

% Pojasnilo: izomorfnost $A \cong B$ je struktura, ki je naravno podana z dvema
% preslikavama $A \to B$ in $B \to A$ ter dvema enačbama med njima. Zato tu zapišemo
% definicijo, ki hkrati uvede vse te pojme.

\begin{definicija}
  Množici $A$ in $B$ sta \df{izomorfni} in pišemo $A \cong B$, kadar obstajata preslikavi
  %
  \begin{equation*}
    f : A \to B
    \qquad\text{in}\qquad
    g : B \to A,
  \end{equation*}
  %
  za kateri velja
  %
  \begin{equation*}
    g \circ f = \id[A]
    \qquad\text{in}\qquad
    f \circ g = \id[B].
  \end{equation*}
  %
  Pravimo, da je~$f$ \df{izomorfizem} med~$A$ in~$B$ in da je~$g$ \df{inverz} ali
  \df{obrat}~$f$.
\end{definicija}

Preverimo, da velja $A \times B \cong B \times A$ za poljubni množici $A$ in $B$. To
storimo tako, da zapišemo preslikavi med zmnožkoma in preverimo, da tvorita
izomorfizem:\footnote{Držimo se pravila, da nikoli ne uporabimo iste vezane spremenljivke
  dvakrat, zato pravilo za $f$ zapišemo z $x$ in $y$ in pravilo za $g$ z $v$ in $u$.
  Marsikdo bi oba funkcisjka predpisa zapisal z $x$ in $y$, torej
  $f : (x, y) \mapsto (y, x)$ in $g : (y, x) \mapsto (x, y)$. To zmede nekatere študente,
  ker mislijo, da ``sta je $x$ v definiciji $f$ isti kot v definiciji $g$'', karkoli že
  naj bi to pomenilo. Poudarimo še enkat: vezana spremenljivka v funkcijskem predpisu nima
  nikakršne zveze z nobeno drugo pojavitvijo iste spremenljivke kje druge.}
%
\begin{align*}
  f &: A \times B \to B \times A &
  g &: B \times A \to A \times B \\
  f &: (x, y) \mapsto (y, x) &
  g &: (v, u) \mapsto (u, v).
\end{align*}
%
Treba je preveriti, da velja $g \circ f = \id[A \times B]$ in
$f \circ g = \id[B \times A]$. To naredimo z uporabo ekstenzionalnosti preslikav, ki pravi
da $g \circ f = \id[A]$ velja, če velja $(g \circ f)(a,b) = \id[A](a,b)$ za vse $a \in A$
in $b \in B$, in podobno za $f \circ g$. Obravnavajmo torej poljubna $a \in A$ in
$b \in B$ in izračunajmo:
%
\begin{equation*}
  (g \circ f)(a, b) =
  g (f (a, b)) = g (b, a) = (a, b).
\end{equation*}
%
Na podoben način preverimo $f \circ g = \id[B \times A]$.

\begin{zgled}\label{zgled:logaritmiranje-je-obratno-od-eksponenciranja}
  Primere izmorfizmov poznamo že iz srednje šole. Naj bo $\RR$ množica vseh realnih števil
  (glej razdelek~\ref{sec:realna-stevila}) in $\RR_{>0}$ množica vseh pozitivnih realnih
  števil. Tedaj logaritem in eksponentna funkcija,
  %
  \begin{equation*}
    \log : \RR_{>0} \to \RR
    \qquad\text{in}\qquad
    \exp : \RR \to \RR_{>0}
  \end{equation*}
  %
  tvorita izomorfizem, saj za $x \in \RR$ velja $\log (\exp x) = x$ in za $y \in \RR_{>0}$
  velja $\exp (\log y) = y$. Eksponentna funkcija seštevanje slika v množenje:
  $\exp 0 = 1$ in $\exp (x + y) = \exp x \cdot \exp y$, zato ni samo izomorfizem med
  množicama, ampak celo izomorfizem med grupama $(\RR, {+}, 0)$ in
  $(\RR_{>0}, {\cdot}, 1)$.

  Če ne veste, kaj je grupa in izomorfizem grup, nikar ne obupavajte. Vsak matematik se v
  vsakdanjem delu nenehno srečuje z neznanimi pojmi. Veste, da je znameniti profesor
  France Križanič\footnote{France Križanič (1928--2002), slovenski matematik} v enega od
  svojih učbenikov zapisal, da naj tisti, ki mu je branje dokazov odveč, ravna tako kot Du
  Fu:\footnote{Du Fu (712--770 pr.~n.~š), kitajski pesnik}
  %
  \begin{center}
    \begin{tabular}{l}
      Ko berem knjige,\\
      z vinom se krepčam\\
      in znak preskočim,\\
      če ga ne poznam.
    \end{tabular}
  \end{center}
\end{zgled}

\begin{vaja}
  Odkorakajte v knjižnico, izposodite si knjigo profesorja Križaniča in jo preberite.
\end{vaja}

Dokažimo nekaj osnovih lastnosti izmorfnosti in izomorizmov. Tokrat ne bomo zapisali
podrobnih dokazov. Za vajo jih dopolnite do tolikšnih podrobnosti, da boste sami sebe
prepričali, da trditve držijo.

\begin{trditev}
  Če je $f : A \to B$ izomorfizem med množicama $A$ in $B$ ter sta preslikavi
  $g : B \to A$ in $h : B \to A$ obe obrata~$f$, potem je $g = h$.
\end{trditev}

\begin{dokaz}
  Ker je $g$ obrat $f$, velja
  %
  \begin{equation*}
    g \circ f = \id[A]
    \qquad\text{in}\qquad
    f \circ g = \id[B],
  \end{equation*}
  %
  in ker je $h$ obrat $f$, velja
  %
  \begin{equation*}
    h \circ f = \id[A]
    \qquad\text{in}\qquad
    f \circ h = \id[B].
  \end{equation*}
  %
  Dokazati moramo, da iz teh štirih predpostavk sledi $g = h$, kar storimo z naslednjim
  računom:
  %
  \begin{align*}
    g
    &= \id[A] \circ g \tag{kompozicija z $\id[A]$ nima učinka} \\
    &= (h \circ f) \circ g \tag{predpostavka $h \circ f = \id[A]$} \\
    &= h \circ (f \circ g) \tag{kompozicija je asociativna} \\
    &= h \circ \id[B] \tag{predpostavka $f \circ g = \id[B]$} \\
    &= h. \tag{kompozicija z $\id[B]$ nima učinka}
  \end{align*}
\end{dokaz}

Če je $f : A \to B$ izomorfizem, potem ima natanko en obrat, ki ga označimo $\inv{f}$. Če
$f$ ni izomorfizem, zapis $f^{-1}$ ni veljaven izraz.

Oznaka za obrat je nekoliko nerodna, ker se prekriva z zapisom za obratno vrednost
števila: če je $x \in \RR$ neničelno realno število, potem je $\inv{x}$ tisto realno
število, za katerega velja $x \cdot \inv{x} = 1$. Torej moramo paziti: če je
$f : \RR \to \RR$ izomorfizem in $x \in \RR$, je $\inv{(f(x))}$ obrat števila $f(x)$,
medtem ko je $\inv{f}(x)$ število, ki ga dobimo, ko obrat preslikave $f$ uporabimo na~$x$.
Sami premislite, kaj je $\inv{(\inv{f}(x))}$.

\begin{vaja}
  Podajte primer izomorfizma $f : \RR \to \RR$ in števila $x \in \RR$, da velja
  $\inv{f}(x) = \inv{(f(x))}$. Nato podajte še primer, ko velja
  $\inv{f}(x) \neq \inv{(f(x))}$.
\end{vaja}

\begin{vaja}
  Ozrimo se še enkrat na dokaz prejšnje trditve. Ali smo uporabili vse štiri predpostavke?
  Zapišite \emph{bolj splošno trditev}, se pravi tako, ki navede samo tiste predpostavke,
  ki jih res potrebujemo v dokazu.
\end{vaja}

\begin{trditev}
  Za vse izmorfizme $f : A \to B$ in $g : B \to C$ velja
  %
  \begin{equation*}
    \inv{(\inv{f})} = f
    \qquad\text{in}\qquad
    \inv{(g \circ f)} = \inv{f} \circ \inv{g}.
  \end{equation*}
\end{trditev}

\begin{dokaz}
  Dokaz prepuščamo za vajo. Pozor, v desni enakosti se je zamenjal vrstni red $f$ in $g$!
  Nadalje opazimo še to: zapisali smo $\inv{(\inv{f})}$ in $\inv{(g \circ f)}$, ne da bi
  predhodno preverili, ali sta $\inv{f}$ in $g \circ f$ izomorfizma. Torej morate v dokazu
  najprej preveriti, da je sta $\inv{f}$ in $g \circ f$ izomorfizma, če sta $f$ in $g$
  izomorfizma.
\end{dokaz}

\begin{trditev}
  Za vse množice $A$, $B$ in $C$ velja:
  %
  \begin{enumerate}
  \item $A \cong A$,
  \item če $A \cong B$, potem $B \cong A$,
  \item če $A \cong B$ in $B \cong C$, potem $A \cong C$.
  \end{enumerate}
\end{trditev}

\begin{dokaz}
  \parbox{0pt}{}
  %
  \begin{enumerate}
  \item $\id[A]$ je izomorfizem iz $A$ v $A$, ki je sam svoj obrat,
  \item če je $f : A \to B$ izomorfizem iz $A$ v $B$, potem je $\inv{f}$ izomorfizem iz
    $B$ v $A$ in $f$,
  \item če je $f : A \to B$ izomorfizem iz $A$ v $B$ in $g : B \to C$ izomorfizem iz $B$ v
    $C$, potem je $g \circ f$ izomorfizem iz $A \to C$. \qedhere
  \end{enumerate}
\end{dokaz}

\begin{trditev}
  Preslikava ima največ en inverz.
\end{trditev}

\begin{vaja}
  Pogosto rečemo, da sta seštevanje in odštevanje obratni operaciji. Strogo vzeto, ti dve
  operaciji nista obratni kot preslikavi, saj obe slikata (recimo, da ju gledamo na
  realnih številih) $\RR \times \RR \to \RR$, tj.~ne slikata v nasprotnih smereh. Ugotovi,
  v kakšnem smislu točno sta seštevanje in odštevanje obratni, tj.~kateri dve preslikavi
  sta pravzaprav druga drugi obratni.
\end{vaja}

% TODO Izomorfnost je kongruenca za produkt, vsoto in eksponent.

\section{Algebra množic}
\label{sec:algebra-mnozic}

Kot že veste, seštevanje, množenje in potenciranje števil zadoščajo naslednjim
algebrajskim zakonom:
%
\begin{align*}
  a + 0 &= a                   &     a \cdot 1 &= a \\
  a + b &= b + a               &     a \cdot b &= b \cdot a \\
  a + (b + c) &= (a + b) + c   &     a \cdot (b \cdot c) &= (a \cdot b) \cdot c \\[1ex]
  0 \cdot a &= 0                           &   1^a &= 1 \\
  (a + b) \cdot c &= a \cdot c + b \cdot c &   (a \cdot b)^c &= a^c \cdot b^c \\[1ex]
  a^0 &= 1                     &   a^1 &= a \\
  a^{b + c} &= a^b \cdot a^c   &   a^{b \cdot c} &= (a^b)^c \\[1ex]
  0^a &= 0 \quad \text{če $a \neq 0$.}
\end{align*}
%
Že prej smo opazili, da je zakon $a \cdot b = b \cdot a$ podoben izomorfizmu
$A \times B \cong B \times A$. Kaj pa ostali zakoni?

\begin{izrek}
  \label{izrek:algebra-mnozic}
  Za vse množice $A$, $B$ in $C$ velja:
  %
  \begin{align*}
    A + \emptyset &\cong A                   &     A \times \one &\cong A \\
    A + B &\cong B + A               &     A \times B &\cong B \times A \\
    A + (B + C) &\cong (A + B) + C   &     A \times (B \times C) &\cong (A \times B) \times C \\[1ex]
    \emptyset \times A &\cong \emptyset                           &   \one^A &\cong \one \\
    (A + B) \times C &\cong A \times C + B \times C &   (A \times B)^C &\cong A^C \times B^C \\[1ex]
    A^\emptyset &\cong \one                     &   A^\one &\cong A \\
    A^{B + C} &\cong A^B \times A^C   &   A^{B \times C} &\cong (A^B)^C \\[1ex]
    \emptyset^A &\cong \emptyset \quad \text{če $A \neq \emptyset$.}
  \end{align*}
\end{izrek}

Izrek ni sam sebi namen, ampak je v njem nauk: \emph{z množicami lahko računamo}, tako kot
s števili. Preostanek razdelka je posvečen dokazu izreka.

\subsubsection{Asociativnost}
\label{sec:asociativnost}

Za ogrevanje dokažimo asociativnost zmnožkov,
$A \times (B \times C) \cong (A \times B) \times C$. Splošni element
$A \times (B \times C)$ je urejeni par oblike $(x, (y, z))$, kjer je $x \in A$, $y \in B$
in $z \in C$, med tem ko je splošni element $(A \times B) \times C$ oblike $((u, v), w)$,
kjer je $u \in A$, $v \in B$ in $w \in C$. Izomorfizmov ni težko zapisati:
%
\begin{align*}
  f &:  A \times (B \times C) \to (A \times B) \times C &
  g &: (A \times B) \times C \to A \times (B \times C) \\
  f &: (x, (y, z)) \mapsto ((x, y), z) &
  g &: ((u, v), w) \mapsto (u, (v, w)).
\end{align*}
%
Preverimo, da je $g$ obrat $f$. Za vse $x \in A$, $y \in B$ in $z \in C$ velja:
%
\begin{equation*}
  g(f(x, (y, z))) = g((x, y), z) = (x, (y, z))
\end{equation*}
%
in za vse $u \in A$, $v \in B$ in $w \in C$ velja
%
\begin{equation*}
  f(g((u, v), w)) = f(u, (v, w)) = ((u, v), w).
\end{equation*}
%
Tudi asociativnost vsote, $A + (B + C) \cong (A + B) + C$ ni nič bolj zapletena, le da
imamo opravka z injekcijami in obravnavanjem primerov. Najprej zapišimo izomorfizma s
popolnoma natančnim zapisom, kjer vse injekcije opremimo z oznakami množic:
%
\begin{align*}
  f &:  A + (B + C) \to (A + B) + C &
  g &: (A + B) + C \to A + (B + C) \\
  f &: \inl[A][B+C](x)             \mapsto \inl[A+B][C](\inl[A][B](x)) &
  g &: \inl[A+B][C](\inl[A][B](u)) \mapsto \inl[A][B+C](u)\\
  f &: \inr[A][B+C](\inl[B][C](y)) \mapsto \inl[A+B][C](\inr[A][B](y)) &
  g &: \inl[A+B][C](\inr[A][B](v)) \mapsto \inr[A][B+C](\inl[B][C](v))  \\
  f &: \inr[A][B+C](\inr[B][C](z)) \mapsto \inr[A+B][C](z) &
  g &: \inr[A+B][C](w)              \mapsto \inr[A][B+C](\inr[B][C](w))
\end{align*}
%
Isti zapis brez oznak množic je precej bolj čitljiv:
%
\begin{align*}
  f &:  A + (B + C) \to (A + B) + C &
  g &: (A + B) + C \to A + (B + C) \\
  f &: \inl(x)             \mapsto \inl(\inl(x)) &
  g &: \inl(\inl(u)) \mapsto \inl(u)\\
  f &: \inr(\inl(y)) \mapsto \inl(\inr(y)) &
  g &: \inl(\inr(v)) \mapsto \inr(\inl(v))  \\
  f &: \inr(\inr(z)) \mapsto \inr(z) &
  g &: \inr(w)              \mapsto \inr(\inr(w))
\end{align*}
%
Ali vidite, zakaj matematiki cenimo kratek in pregleden zapis? Preveč podrobnosti lahko
zakrije bistvo ideje. Preverjanje, da je $g$ obrat $f$, prepustimo tistim, ki radi veliko
pišejo.

\subsubsection{Preslikave in enojec}
\label{sec:preslikave-enojec}

Preslikavi
%
\begin{align*}
  f &: A \times \one \to A &
  g &: A \to A \times \one \\
  f &: (x, u) \mapsto x &
  g &: y \mapsto (y, \unit)
\end{align*}
%
tvorita izomorfizem $A \times \one \cong A$, saj za vsak $a \in A$ in $t \in \one$ velja,
upoštevaje da so vsi elementi~$\one$ enaki~$\unit$,
%
\begin{equation*}
  g(f(a, t)) = g(a) = (a, \unit) = (a, t)
  \qquad\text{in}\qquad
  f(g(a)) = f(a, t) = a.
\end{equation*}
%
Lahko bi rekli, da je $\one$ nevtralni element za zmnožek \emph{do izomorfizma natančno},
s čimer povemo, da ne velja \emph{enakost} $A \times \one = A$, ampak le
\emph{izomorfizem} $A \times \one \cong A$. Na tem mestu lahko tudi pojasnimo nenavadni
zapis edinega elementa~$\one$. Elementi zmnožka dveh množic so urejene dvojice, zmnožka
treh množic urejene trojice itd. Zmnožek nič množic je nevtralni element za množenje,
torej so njegovi elementi urejen ničterice, oziroma urejena ničterica~$\unit$, ker je ena
sama.

Izomorfizma $A^\one \cong A$ ni težko zapisati:
%
\begin{align*}
  f &: A^\one \to A &
  g &: A \to A^\one \\
  f &: h \mapsto h(\unit) &
  g &: x \mapsto (y \mapsto x)
\end{align*}
%
Preverimo, da je $g$ inverz $f$. Za vsak $x \in A$ velja
%
\begin{equation*}
  f(g(x)) = f(y \mapsto x) = x,
\end{equation*}
%
zato je $f \circ g = \id[A]$. Za vsak $h \in A^\one$ velja
%
\begin{equation*}
  g(f(h)) = g(h(\unit)) = (y \mapsto h(\unit)).
\end{equation*}
%
Ali sta $h$ in $y \mapsto h(\unit)$ enaki preslikavi? Kot vsakič, uporabimo
ekstenzionalnost preslikav, le da je tokrat še posebej preprosta: preslikavi z
domeno~$\one$ sta enaki, če imata enako vrednost pri argumentu $\unit$, saj je to edini
element~$\one$. Torej je $h = (y \mapsto h(\unit))$, saj velja
%
\begin{equation*}
  (y \mapsto h(\unit))(\unit) = h(\unit).
\end{equation*}
%
Izomorfnost $A$ in $A^\one$ pravzaprav pove nekaj zanimivega: preslikave $\one \to A$
lahko obravnavamo kot elemente~$A$ in obratno.


\subsubsection{Preslikave in prazna množica}
\label{sec:presl-prazna-mnozica}

Lotimo se izomorfizmov, v katere je vpletena prazna množica. Tu se ne moremo več zanašati
le na prirojen občutek za logiko, saj s prazno množico nimamo vsakdanjih izkušenj, oziroma
jo obravnavamo kot posebnost. Kako bi odgovorili na vprašanje, ali so vsi elementi prazne
množice praštevila? Pravilni odgovor je ``da''. In hkrati so vsi elementi prazne množice
sestavljena števila. Zakaj je to res bomo spoznali v
razdelku~\ref{sec:logika-prazna-mnozica}, ko bomo podrobno obravnavali pravila sklepanja.
Zaenkrat si zapomnimo, da je pravilna vsaka izjava ``za vse elemente prazne množice velja
\dots''. Pravimo, da je taka izjava \emph{na prazno izpolnjena}

Začnimo z vprašanjem, ali lahko tvorimo kako preslikavo $\emptyset \to A$. Najprej
ugotovimo, da so vse preslikave $\emptyset \to A$ enake. Res, za $f, g : \emptyset \to A$
velja $f = g$ natanko tedaj, ko za vse $x \in \emptyset$ velja $f(x) = g(x)$. A ravnokar
smo povedali, da je vsaka izjava oblike ``za vse $x \in \emptyset$ \dots'' veljavna. Pa
imamo kako preslikavo $\emptyset \to A$? Odgovor je pritrdilen, če lahko podamo kako
celovito in enolično prirejanje med elementi~$\emptyset$ in~$A$. Ker sta celovitost in
enoličnost spet izavi oblike ``za vse $x \in \emptyset$ \dots'', sta na prazno izpolnjena,
zato bo zadoščalo kakršnokoli prirejanje, denimo: nobenemu elementu ne priredimo nobenega
elementa. S tem smo utemeljili naslednjo trditev.

\begin{trditev}
  Za vsako množico $A$ obstaja natanko ena preslikava $\emptyset \to A$.
\end{trditev}

Edini preslikavi $\emptyset \to A$ pravimo \df{prazna preslikava}. S tem smo utemeljili
$A^\emptyset \cong \one$, saj izomorfizem prazni preslikavi priredi
$\unit$, njegov obrat pa priredi $\unit$ prazno preslikavo.

\subsubsection{Izomorfizmi in eksponenti}
\label{sec:izomorfizmi-in-eksponenti}

Nazadnje se posvetimo še zakonu $A^{B \times C} \cong (A^B)^C$.
Preverimo, da preslikavi\footnote{Saj ste se že naučili grške črke, ali ne?}
%
\begin{align*}
  \Lambda &: A^{B \times C} \to (A^B)^C
  &
  \Theta &: (A^B)^C \to A^{B \times C}
  \\
  \Lambda &: f \mapsto (c \mapsto (b \mapsto f(b, c)))
  &
  \Theta &: g \mapsto ((b, c) \mapsto g(c)(b))
\end{align*}
%
tvorita izomorfizem. Za vse $f \in A^{B \times C}$, $x \in B$ in $y \in C$ velja
%
\begin{align*}
  \Theta(\Lambda(f))(x, y)
  &= ((b, c) \mapsto \Lambda(f)(c)(b)) (x, y)  \\
  &= \Lambda(f)(y)(x) \\
  &= (c \mapsto (b \mapsto f(b, c)))(y)(x) \\
  &= (b \mapsto f(b, y)(x) \\
  &= f(x, y),
\end{align*}
%
zato je $\Theta(\Lambda(f)) = f$. Prav tako za vse $g \in (A^B)^C$ in $x \in B$ in
$y \in C$ velja
%
\begin{align*}
  \Lambda(\Theta(g))(y)(x)
  &= (c \mapsto (b \mapsto \Theta(g)(b, c)))(y)(x) \\
  &= (b \mapsto \Theta(g)(b, y))(x) \\
  &= \Theta(g)(x, y) \\
  &= ((b, c) \mapsto g(c)(b)) (x, y) \\
  &= g(y)(x)
\end{align*}
%
in zato $\Lambda(\Theta(g)) = g$.
%
Preslikavi $\Lambda(f)$ pravimo \df{transpozicija} preslikave~$f$, in prav tako preslikavi
$\Theta(g)$ pravimo transpozicija preslikave~$g$.

Izomorfizem $A^{B \times C} \cong (A^B)^C$ je zanimiv, ker pove, da lahko preslikavo dveh
argumentov vedno prevedemo na preslikavo enega argumenta. Natančneje, če je
$f : B \times C \to A$ preslikava dveh argumentov, je njena transpozicija
$\Lambda(f) : C \to A^B$ preslikava enega argumenta, njena vrednost pa je preslikava, ki
pričakuje še en argument. To dejstvo se s pridom izkorišča v funkcijskem programiranju:
namesto, da bi definirali preslikavo $f : B \times C \to A$, ki sprejme urejeni par
$(b, c)$ in vrne vrednost $f(b,c)$, raje definiramo enakovredno preslikavo
$\tilde{f} : B \to C \to A$, ki sprejme $b$ in vrne preslikavo $\tilde{f}(b)$, ta pa
sprejme še $c$ in vrne vrednost $\tilde{f}(b)(c)$.


% TODO Potence A^n in binomski izrek.

% TODO zmnozek, vsota in eksponent zo kongruenca za izomorfizme

% Pri asociativnosti produkta obravnavamo $A_1 \times A_2 \times \cdots \times A_n$ in
% enojec kot produkt nič množic. Podobno za vsote.

% Tu je treba pojasniti, zakaj pišemo $\unit$ za element $\one$.


% \section{Kar je že Davorin napisal}

% Interval realnih števil podamo s krajiščema intervala v oklepajih --- okrogli oklepaji ( ) označujejo odprtost intervala (krajišče ni vključeno v interval), oglati oklepaji [ ] pa zaprtost (krajišče je vključeno). Tako se npr.~interval realnih števil od $0$ do $1$, ki ne vsebuje krajišč, označi z $(0, 1)$, če jih vsebuje, pa z $[0, 1]$.

% Včasih pridejo prav tudi intervali na drugih množicah kot $\RR$. Zato se dogovorimo, da bomo intervale označevali tako, da podamo množico, ob kateri v indeksu zapišemo krajišči v oklepajih, npr.~$\intco[\NN]{1}{5} = \set{1, 2, 3, 4}$. Realna intervala iz prejšnjega odstavka tako zapišemo kot $\intoo{0}{1}$ in $\intcc{0}{1}$.

% Če interval v katero smer gre v nedogled, preprosto zapišemo množico z ustreznim simbolom za urejenost in krajiščem v indeksu. Na primer, $\RR_{> 0}$ označuje množico pozitivnih realnih števil, $\RR_{\geq 0}$ pa množico nenegativnih realnih števil.

% Primerjave med elementi, kot npr.~pravkar podani $>$ in $\geq$, imenujemo \df{relacije} (podrobneje jih bomo spoznali v poglavju~\ref{poglavje:relacije}). Zgornji zapis bomo uporabljali tudi za druge vrste relacij, ne samo za relacije urejenosti. Na primer, množico vseh neničelnih realnih števil zapišemo kot $\RR_{\neq 0}$.

% \davorin{To bi vsaj bil moj predlog. Na ta način se izognemo dvoumnostim (kar je namen). Na primer, kaj pomeni $\forall\, a > 0$? Če zapišemo $\forall\, a \in \NN_{> 0}$ ali $\forall\, a \in \RR_{> 0}$, je jasno. Razlog, da matematiki ``goljufajo'' in pridejo skozi brez tega, je (napol dogovorjena in ponotranjena, ampak arbitrarna) izbira črk; vsak izkušen matematik ve, da $\forall\, \epsilon > 0$ pomeni $\forall\, \epsilon \in \RR_{> 0}$. Dodaten problem je, da kasneje uporabljamo urejene pare, ki jih vsi na naši fakulteti pišejo z okroglimi oklepaji. Poskusimo se izogniti zmedi, ali $(a, b)$ pomeni urejeni par ali odprti interval. Če se ne strinjate, popravite in pustite komentar.}

% Če imamo dan neki element in neko množico, potem pripadnost tega elementa tej množici izrazimo s simbolom $\in$. Na primer, da je štiri naravno število, zapišemo $4 \in \NN$ (beri: ``štiri pripada množici naravnih števil'').

% Elementi množic lahko zadoščajo raznim lastnostim. Na primer, recimo, da $\phi$ označuje lastnost ``biti manj od pet''; to potem zapišemo
% \[\phi(x) \ = \ \ x < 5.\]
% V tem primeru $x$ imenujemo \df{spremenljivka}, saj ne gre za točno določeno vrednost, pač pa predstavlja splošno število (recimo, da se dogovorimo, da s $\phi$ označujemo lastnost na realnih številih).

% Tovrstne lastnosti nam omogočajo, da iz neke množice odberemo elemente z dano lastnostjo in na ta način dobimo novo množico, ki je podmnožica prejšnje. Množico vseh realnih števil, ki so manjša od pet, zapišemo na naslednji način.
% \[\set{x \in \RR}{x < 5}\]
% Seveda, ker je primerjava s števili zelo pogosta lastnost, je uporabno, če uvedemo krajše oznake, ki povejo isto; že prej smo se dogovorili, da tako množico označimo z $\RR_{< 5}$. Za povsem splošne lastnosti pa ne bomo imeli vnaprej dogovorjenih oznak, zato je dobro, da poznamo splošni zapis. Torej, če je $X$ poljubna množica in $\phi$ poljubna lastnost njenih elementov, tedaj podmnožico, ki vsebuje točno tiste elemente množice $X$, ki zadoščajo lastnosti $\phi$, označimo takole.
% \[\set[1]{x \in X}{\phi(x)}\]

% Pri tem se zavedajmo: ni pomembno, da spremenljivko označimo ravno z $x$. Zapis
% \[\set[1]{y \in X}{\phi(y)}\]
% še vedno označuje isto množico. V vsakem primeru gre za množico vseh elementov iz $X$ z lastnostjo $\phi$. Pravzaprav sploh ni nujno, da uporabimo črko; poslužimo se lahko kateregakoli simbola (ki mu nismo predtem že predpisali določenega pomena). Taisto množico lahko zapišemo tudi $\set{\heartsuit \in X}{\phi(\heartsuit)}$.

% Kadar imamo spremenljivko, ki jo lahko preimenujemo, ne da bi spremenili pomen izraza, jo imenujemo \df{nema spremenljivka}. Takšne primere že dobro poznate; na primer, integral $\int_0^1 x^2 \,dx$ se ne spremeni, če preimenujete spremenljivko in zapišete $\int_0^1 y^2 \,dy$.

% \begin{zgled}
% Kako bi zapisali množico vseh sodih naravnih števil? Spomnimo se, da je število sodo, kadar je deljivo z $2$. Za $n \in \NN$ to zapišemo takole: $2 \divides n$ (beri: ``dve deli $n$''). Množica sodih naravnih števil se potem zapiše kot
% \[\set[1]{n \in \NN}{2 \divides n}.\]
% \end{zgled}


\section{Vaje}

\begin{vaja}
Kaj veste povedati o množici~$A$, če zanjo velja, da so vsi njeni elementi enaki?
\begin{resitev}
Množica~$A$ ima kvečjemu en element, tj.~množica~$A$ je bodisi prazna bodisi enojec. Tudi: množica~$A$ je podmnožica kakega enojca oz.~edina preslikava $A \to \one$ je injektivna.
\end{resitev}
\end{vaja}

\begin{vaja}
  Pravilo ekstenzionalnosti preslikav bi lahko zapisali tudi takole:
  %
  \begin{quote}
    Preslikavi $f : A \to B$ in $g : C \to D$ sta enaki, če velja $A = C$, $B = D$ in za
    vse $x_1, x_2 \in A$ velja, da iz $x_1 = x_2$ sledi $f(x_1) = g(x_2)$.
  \end{quote}
  %
  Dokažite, da je ta različica enakovredna običajnem pravilu ekstenzionalnosti.
\end{vaja}

\begin{vaja}
  Zapišite pravila za zmnožek treh množic. Nato premislite še, kako bi podali pravila za
  zmnožek $n$ množic, kjer je~$n$ naravno število.
\end{vaja}

\begin{vaja}
  Naštejte vse elemente množice $\one + \one + \one$.
\end{vaja}

\begin{vaja}
  Preveri tiste izomorfnosti iz izreka~\ref{izrek:algebra-mnozic}, ki jih v
  razdelku~\ref{sec:algebra-mnozic} nismo utemeljili.
\end{vaja}

\chapter{Aritmetika množic}

Nadaljujmo s študijem splošnih preslikav.

\section{Preslikave in prazna množica}

Naj bo $A$ množica. Kaj vemo povedati o preslikavah $\emptyset \to A$?

Čez nekaj tednov bomo spoznali naslednji dejstvi, ki ju zaenkrat vzemimo v zakup:

\begin{itemize}
\item Vsaka izjava oblike ">za vsak element $\emptyset$ ..."< je resnična.
\item Vsaka izjava oblike ">obstaja element $\emptyset$ ..."< je neresnična.
\end{itemize}

Primeri resničnih izjav:
%
\begin{enumerate}
\item ">Vsak element prazne množice je sodo število"<
\item ">Vsak element prazne množice je liho število"<
\item ">Vsak element prazne množice je hkrati sodo in liho število"<
\item ">Vsak element prazne množice \dots"<
\end{enumerate}

Primeri neresničnih izjav:
%
\begin{enumerate}
\item ">Obstaja element prazne množice, ki je sodo število"<
\item ">Obstaja element prazne množice, ki je enak sam sebi"<
\item ">Obstaja element prazne množice, ki \dots"<
\end{enumerate}
%
Denimo, da imamo preslikave $f :\emptyset \to A$ in $g : \emptyset \to A$. Tedaj sta enaki, saj velja: ">za vsak element $x \in \emptyset$ velja $f(x) = g(x)$".
Torej imamo kvečjemu eno preslikavo $\emptyset \to A$. Ali pa imamo sploh kakšno? Da, pravimo ji \textbf{prazna preslikava}, ker je njeno prirejanje ">prazno"<, oziroma ga sploh ni treba podati (saj ni nobenega elementa domene $\emptyset$, ki bi mu morali prirediti kak element kodomene $A$).

Kaj pa preslikave $A \to \emptyset$?
%
Če je $A = \emptyset$, potem imamo natanko eno preslikavo $A \to \emptyset$, namreč prazno preslikavo, $\emptyset^A = \{ \textrm{prazna-preslikava} \}$.
%
Če $A$ vsebuje kak element, potem ni nobene preslikave $A \to \emptyset$, se pravi  $\emptyset^A = \emptyset$.

Zakaj ni preslikave $A \to \emptyset$, kadar $A$ vsebuje kak element? Denimo da je $x \in A$. Če bi bila kaka preslikava $f : A \to \emptyset$, bi
veljalo $f(x) \in \emptyset$, kar pa ni res. Torej take preslikave ni.

\begin{naloga}
  Koliko je preslikav $1 \to A$ in koliko je preslikav $A \to 1$?
  Ali je odgovor odvisen od~$A$?
\end{naloga}

\section{Identiteta in kompozicija}

Spoznajmo nekaj osnovnih preslikav in operacij na preslikavah.

\textbf{Identiteta} na $A$ je preslikava $id[A] : A \to A$, podana s predpisom $x \mapsto x$.

\textbf{Kompozitum} preslikav
%
\begin{equation*}
  \xymatrix{
    {A} \ar[r]^f & {B} \ar[r]^g & {C}
  }
\end{equation*}
%
je preslikava $g \circ f : A \to C$, podana s predpisom $x \mapsto g(f(x))$.

\textbf{Kompozitum je asociativen:} za preslikave
%
\begin{equation*}
  \xymatrix{
    {A} \ar[r]^f & {B} \ar[r]^g & {C} \ar[r]^h & {D}
  }
\end{equation*}
%
velja $(h \circ g) \circ f = h \circ (g \circ f)$. Res, za vsak $x \in A$ velja
%
\begin{align*}
  ((h \circ g) \circ f)(x)
  &= (h \circ g) (f x) \\
  &=  h (g (f (x)) \\
  &= h ((g \circ f)(x)) \\
  &= (h \circ (g \circ f))(x),
\end{align*}
%
torej želena\footnote{Piše se ">želen"< in ne ">željen"<, ker je ">želen"< deležnik na
  ">n"< glagola ">želeti"<. V slovenščini ni glagola ">željeti"<. Hitro boste spoznali, da na profesorji za matematiko radi popravljajo slovnico.} enačba sledi iz principa ekstenzionalnosti za funkcije.

\textbf{Identiteta je nevtralni element za kompozitum:} za vsako preslikavo $f : A \to B$ velja
%
\begin{equation*}
  \id[B] \circ f = f
  \iinn
  f \circ id[A] = f.
\end{equation*}
%
To preverimo z uporabo ekstenzionalnosti za funkcije: za vsak $x \in A$ velja
%
\begin{equation*}
    (\id[B] \circ f)(x) = \id[B] (f(x)) = f(x)
\end{equation*}
%
in
\begin{equation*}
  (f \circ id[A])(x) = f (id[A](x)) = f(x).
\end{equation*}
%
Kompizicija $\circ$ in identiteta $id$ se torej obnašata podobno kot nekatere operacije v algebri, na primer $+$ in $0$ ter $×$ in $1$.

\begin{naloga}
  Seštevanje je komutativno, velja $a + b = b + a$. Ali je kompozicija preslikav tudi komutativna?
\end{naloga}

\section{Funkcijski predpisi na zmnožku in vsoti}

Pogosto želimo definirati preslikavo, katere kodomena je zmnožek množic, denimo $f : A \times B \to C$. V takem primeru lahko
podamo funkcijski predpis takole:
%
\begin{equation*}
  (x, y) \mapsto \cdots
\end{equation*}
%
pri čemer je $x \in A$ in $y \in B$. To je dovoljeno, ker je vsak element domene $A \times B$ urejeni par $(x, y)$ za natanko določena $x \in A$ in $y \in B$.

\begin{primer}

\textbf{Primer:} Preslikavo
%
\begin{align*}
  \RR \times \RR  &\to  \RR \\
  u &\mapsto  \fst{u}² + 3 \cdot \snd{u}
\end{align*}
%
lahko bolj čitljivo podamo s predpisom
%
\begin{align*}
  \RR \times \RR  &\to  \RR \\
  (x, y) &\mapsto  x^2 + 3 \cdot y
\end{align*}
\end{primer}

\begin{primer}
  Seveda lahko podobno podajamo tudi preslikave na zmnožkih večih preslika, denimo
  %
  \begin{align*}
  A \times B \times C &\to A \times A \\
  (a, b, c) &\mapsto (a, a)
  \end{align*}
  in
  \begin{align*}
  X \times (Y \times Z) &\to (X \times Y) \times Z \\
  (x, (y, z)) &\mapsto ((x, y), z).
  \end{align*}
\end{primer}

Kako pa zapišemo funkcijski predpis funkcije z domeno $A + B$? V tem primeru je vsak element domene bodisi oblike
$\inl{x}$ za enolično določeni $x \in A$, bodisi oblike $\inr{y}$ za enolično določeni $y \in B$, zato funkcijski predpis podamo v dveh vrsticah:
%
\begin{align*}
    A + B &\to C \\
    \inl{x} &\mapsto \cdots \\
    \inr{y} &\mapsto \cdots
\end{align*}

\begin{primer}
  Primer take preslikave je
  %
  \begin{align*}
    \RR + \ZZ &\to \RR \\
    \inl{x} &\mapsto x \\
    \inr{y} &\mapsto y + 3
  \end{align*}
  %
  Seveda lahko podobno podajamo tudi preslikave na vsotah večih preslikav:
  %
  \begin{align*}
    A + B + C &\to \{u, v\} \\
    \inl{x} &\mapsto u \\
    \inr{y} &\mapsto u \\
    \mathsf{in}_3(z) &\mapsto v
  \end{align*}
  in
  \begin{align*}
    A + (B + C) &\to \{u, v\} \\
    \inl{x} &\mapsto u \\
    \inr{\inl{y}} &\mapsto u \\
    \inr{\inr{y}} &\mapsto v
  \end{align*}
\end{primer}

Zapisa za zmnožek in vsoto lahko tudi kombiniramo:
%
\begin{align*}
  (A \times B \times C) + (D \times E) &\to \{0, 1, 2\} \\
  \inl{(a, b, c)} &\mapsto 1 \\
  \inr{(d, e)} &\mapsto 2
\end{align*}
%
in
\begin{align*}
  (A + B) \times C &\to \{0, 1, 2\} \\
  (\inl{a}, c) &\mapsto 0 \\
  (\inr{b}, c) &\mapsto 1
\end{align*}
%
Izraz na levi strani $\mapsto$ sestoji iz vezanih spremelnjivk in operacij, s katerimi gradimo elemente množic (urejeni par, kanonična injekcija). Imenuje se tudi \textbf{vzorec}. Predpis je podan pravilno, če so vzorci napisani tako, da vsak element domene ustreza natanko enemu vzorcu.
%
S tem zagotovimo, da predpis obravnava vse možne primere (celovitost) in da ne obravnava nobenega primera večkrat (enoličnost).


\subsection{Funkcijski predpis, podan po kosih}

Omenimo še en pogost način podajanja funkcij, namreč s predisom po kosih.

\begin{primer}
  Preslikava ">absolutno"< je definirana po kosih za negativna in nenegativna števila:
  %
  \begin{align*}
    \RR &\to \RR \\
    x &\mapsto
    \begin{cases}
      -x & \text{če $x < 0$,}\\
       x & \text{če $x \geq 0$.}
    \end{cases}
  \end{align*}
\end{primer}

\begin{primer}
  Preslikava ">predznak"< je definirana po kosih:
  %
  \begin{align*}
    \RR &\to \RR \\
    x &\mapsto
      \begin{cases}
        -1 & \text{če $x < 0$,}\\
        0 & \text{če $x = 0$,}\\
        1 & \text{če $x \geq 0$.}
      \end{cases}
  \end{align*}
\end{primer}


Pri takem zapisu moramo paziti, da kosi skupaj pokrivajo domeno (vsi elementi domene so obravnavani) in da se kosi se ne prekrivajo (vsak element domene je obravnavan natanko enkrat). Pravzaprav se smejo kosi prekrivati, a moramo v tem primeru preveriti, da se na skupnih delih skladajo, se pravi, da vsi kosi podajajo enake vrednosti na preseku.

\begin{primer}
  Preslikavo ">absolutno"< bi lahko podali takole:
  %
  \begin{align*}
    \RR &\to \RR \\
    x &\mapsto
    \begin{cases}
      -x & \text{če $x < 0$,}\\
       x & \text{če $x \geq 0$.}
    \end{cases}
  \end{align*}
  %
  Kosa se prekrivata pri $x = 0$, vendar to ni težava, ker je $-0 = 0$.
\end{primer}


\section{Nekatere preslikave na eksponentnih množicah}

Poglejmo si nekaj preslikav, ki slikajo iz in v eksponente množice.

\textbf{Evalvacija} ali \textbf{aplikacija} ali \textbf{uporaba} je preslikava, ki sprejme preslikavo in argument, ter preslikavo uporabi na argumentu:
%
\begin{align*}
  \mathsf{ev} &: B\^A \times A \to B \\
  \mathsf{ev} &: (f, x) \mapsto f(x)
\end{align*}
%
Pravimo, da je $\mathsf{ev}$ \textbf{preslikava višjega reda}, ker slika preslikave v vrednosti.

\begin{primer}
  Določeni integral $\int_0^1$ je funkcija višjega reda, ker
  sprejme funkcijo $[0,1] \to \RR$ in vrne realno število. Je torej preslikava
  $\RR^{[0,1]} \to \RR$, če se pretvarjamo, da lahko integriramo vsako funkcijo.
  Bolj pravilno bi bilo reči, da je $\int_0^1$ preslikava iz množice \emph{itegrabilnih funkcij $[0,1] \to \RR$} v realna števila.
\end{primer}

Kompozitum preslikav je tudi preslikava višjega reda:
%
\begin{align*}
    {\circ} &: C^B \times B^A \to C^A \\
    {\circ} &: (g, f) \mapsto (x \mapsto g(f(x)))
\end{align*}
%
Tretja splošna preslikava višjega reda je ">currying"< (ali zna kdo to prevesi v slovenščino?):
%
\begin{align*}
  A^(B \times C) &\to (A^B)^C \\
  f &\mapsto (c \mapsto (b \mapsto f(b, c))).
\end{align*}
%
Pravzaprav je to izomorfizem, katerega inverz je ">uncurrying"<:
%
\begin{align*}
  (A^B)^C &\to A^{B \times C} \\
  g       &\mapsto ((b, c) \mapsto f(b)(c))
\end{align*}


\section{Izomorfizmi in artimetika množic}

\subsection{Inverz}

\begin{definicija}
  Preslikava $f : A \to B$ je \textbf{inverz} preslikave $g : B \to A$, če velja $f \circ g = \id[B]$ in $g \circ f = \id[A]$.
\end{definicija}

\begin{naloga}
  Utemelji: če je $f$ inverz $g$, potem je $g$ inverz $f$.
\end{naloga}

\begin{primer}
  Kub in kubični koren sta inverza
  %
  \begin{align*}
    \RR &\to \RR    &     \RR &\to \RR \\
    x &\mapsto x^3  &     y &\mapsto \sqrt[3]{y}
  \end{align*}
\end{primer}

\begin{naloga}
  Naj bo~$S$ množica nenegativnih realnih števil, se pravi, $\RR_{\geq 0} = \{x \in \RR \mid x \geq 0\}$. Ali sta kvadriranje in kvadratni koren inverza?
  %
  \begin{align*}
    \RR &\to \RR_{\geq 0}    &     \RR_{\geq 0} &\to \RR \\
    x &\mapsto x^2           &     y &\mapsto \sqrt[2]{y}
  \end{align*}
  %
\end{naloga}

\begin{izjava}
  Če sta $f : A \to B$ in $g : A \to B$ oba inverza preslikave $h : B \to A$, potem je $f = g$.
\end{izjava}

\begin{dokaz}
  Denimo, da sta $f : A \to B$ in $g : A \to B$ inverza preslikave $h : B \to A$. Tedaj velja
  %
  \begin{equation*}
    f =
    f \circ \id[A] =
    f \circ (h \circ g) =
    (f \circ h) \circ g =
    \id[B] \circ g =
    g.
  \end{equation*}
\end{dokaz}

Ali znate utemeljiti vsakega od zgornjih korakov?

\begin{definicija}
  Presikava, ki ima inverz, se imenuje \textbf{izomorfizem}.
\end{definicija}

Če je $f : A \to B$ izomorfizem, potem ima natanko en inverz $B \to A$, ki ga označimo $\inv{f}$.

\begin{primer}
  Identiteta $id[A] : A \to A$ je izomorfizem, saj je sama sebi inverz.
  Torej $\inv{id[A]} = \id[A]$.
\end{primer}

\begin{primer}
  Eksponentna preslikava $\exp : \RR \to \RR_{> 0}$, $exp : x \mapsto e^x$ je
  izomorfizem, njen inverz je naravni logaritem $\ln : \RR_{> 0} \to \RR$, torej $\inv{\exp} = \ln$.
\end{primer}

\begin{primer}
  Eksponentna preslikava $\exp : \RR \to \RR$ \emph{ni} izomorfizem.
\end{primer}

\begin{izjava}
  Če sta $f : A \to B$ in $g : B \to C$ izomorfizma, potem je tudi $g \circ f : A \to C$ izomorfizem. Velja torej $\inv{(g \circ f)} = \inv{f} \circ \inv{g}$.
\end{izjava}

\begin{dokaz}
  Dokazati moramo, da ima $g \circ f$ inverz. Trdimo, da je $\inv{f} \circ \inv{g} : C \to A$ inverz preslikave $g \circ f$. Računajmo:
  %
  \begin{align*}
    (g \circ f) \circ (\inv{f} \circ \inv{g}) \\
     &= ((g \circ f) \circ \inv{f}) \circ \inv{g} \\
     &= (g \circ (f \circ \inv{f})) \circ \inv{g} \\
     &= (g \circ \id[B]) \circ \inv{g} \\
     &= g \circ \inv{g} \\
     &= \id[C].
  \end{align*}
  %
  Doma sami preverite, da velja tudi $(\inv{f} \circ \inv{g}) \circ (g \circ f) = \id[A]$.
\end{dokaz}

\subsection{Izomorfne množice}


\begin{definicija}
  Množici $A$ in $B$ sta \textbf{izomorfni}, če obstaja izomorfizem $f : A \to B$. Kadar sta $A$ in $B$ izomorfni, to zapišemo $A \iso B$.
\end{definicija}

\begin{izjava}
  \textbf{Izjava:} Za vse množice $A$, $B$ in $C$ velja:
  %
  \begin{enumerate}
    \item $A \iso A$,
    \item če $A \iso B$, potem $B \iso A$,
    \item če $A \iso B$ in $B \iso C$, potem $A \iso C$.
  \end{enumerate}
\end{izjava}

\begin{dokaz}
  %
  \begin{enumerate}
     \item $id[A]$ je izomorfizem $A \to A$,
     \item če je $f : A \to B$ izomorfizem, potem je tudi $\inv{f} : B \to A$ izomorfizem,
     \item če je $f : A \to B$ izomorfizem in $g : B \to C$ izomorfizem, potem je $g \circ f : A \to C$ izomorfizem.
  \end{enumerate}
\end{dokaz}

\begin{primer}
  $A \times B \iso B \times A$, ker imamo izmorfizem in njegov inverz
  %
  \begin{align*}
    A \times B  &\to  B \times A      &    B \times A  &\to  A \times B \\
    (x, y) &\mapsto  (y, x)           &    (b, a) &\mapsto  (a, b)
  \end{align*}
\end{primer}

\subsection{Aritmetika množic}

Veljajo naslednji izomorfizmi, ki nas seveda spomnijo na zakone aritmetike, ki
veljajo za števila. Ali gre tu za kako globjo povezavo?
%
\begin{enumerate}
\item Vsota in $\emptyset$:
  \begin{enumerate}
    \item $A + \emptyset \iso A$
    \item $A + B \iso B + A$
    \item $(A + B) + C \iso A + (B + C)$
  \end{enumerate}

\item Zmnožek in $\one$:
  \begin{enumerate}
    \item $A \times 1 \iso A$
    \item $A \times B \iso B \times A$
    \item $(A \times B) \times C \iso A \times (B \times C)$
  \end{enumerate}

\item Distributivnost:
  \begin{enumerate}
    \item $A \times (B + C) \iso (A \times B) + (A \times C)$
    \item $A \times \emptyset \iso \emptyset$
  \end{enumerate}

\item Eksponenti:
  \begin{enumerate}
    \item $A^1 \iso A$
    \item $1^A \iso 1$
    \item $A \emptyset \iso 1$
    \item $\emptyset^A \iso \emptyset če A \neq \emptyset$
    \item $A^(B \times C) \iso (A^B)^C$
    \item $A^(B + C) \iso A^B \times A^C$
    \item $(A \times B)^C \iso A^C \times B^C$
  \end{enumerate}
\end{enumerate}

\begin{naloga}
  Zapišite vseh 15 izomorfizmov, ki potrjujejo pravilnost zgornjega seznama.
\end{naloga}

\chapter{Simbolni zapis}

V matematiki uporabljamo \textbf{simbolni zapis} -- matematične objekte, konstrukcije in dokaze opišemo s pomočjo izrazov kot so
%
\begin{gather*}
  3 + 4\\
  x \mapsto x^2 + 3\\
  \all{x \in \RR} x^2 + x + 1 \geq 1/4
\end{gather*}
%
Matematično besedilo je mešanica naravnega jezika (slovenščine) in simbolnega zapisa. Načeloma bi lahko pisali matematiko \emph{samo} s simbolnim zapisom (kar dejansko počnemo, kadar matematiko \emph{formaliziramo} z računalnikom, a o tem kdaj drugič), a bi bilo to ljudem preveč nerazumljivo. V starih časih so uporabljali samo naravni jezik (latinščino), kar je bilo tudi zelo nerazumljivo.

Spoznajmo pravila simbolnega zapisa in se učimo razumeti, brati in pisati logične formule (matematične izjave, izražene s simbolnim zapisom).


\section{Izrazi}

\textbf{(Simbolni) izraz} je zaporedje znakov, ki predstavlja neki matematični pojem, na primer
%
\begin{gather*}
  3 + 5 \\
  S \cap (T \cup V) \\
  2 x y \leq x^2 + y^2
\end{gather*}
%
Izraz je \emph{pravilno formiran} ali \emph{sintaktično pravilen}, če ustreza pravilom, ki določajo kako postavljamo oklepaje,
vejice, pike, kako uporabljamo razne posebne simbole ($+$, $\lor$, $\int$) itd. Na primer, izraz $3 + ) x \cdot 4$ ni sintaktično
pravilen, ker ima narobe postavljen zaklepaj.

Natančna sintaktična pravila za pisanje matematičnih izrazov so precej zapletena, ker je matematični zapis raznovrsten
in se je razvijal skozi zgodovino. Na srečo skoraj vsa pravila že poznate (">vsak oklepaj mora imeti ustrezni zaklepaj"<,
">piše se $a + b$ in ne $a b +$"< ipd). Tu se ne bomo ukvarjali s podajanjem vseh pravil -- to je delo za računalničarje,
ki želijo taka pravila naprogramirati. Kljub temu pa velja omeniti nekatere pojme.


\subsection{Prefiksne, postfiksne in infiksne operacije}

V simbolnem zapisu uporabljamo \emph{operacije}, ki jih pišemo pred, za ali med argumente:
%
\begin{itemize}
\item \textbf{prefiksne operacije} so take, ki jih pišemo \emph{pred} argument:
  \begin{itemize}
  \item  $-x$ za nasprotno vrednost $x$,
  \item  $\neg P$ za negacijo izjave $P$,
  \end{itemize}
\item \textbf{infiksne operacije} so take, ki jih pišemo \emph{med} argumenta:
  \begin{itemize}
  \item aritmetične operacije $x + y$, $x - y$, $x \cdot y$ itn.
  \item logični vezniki $P \land Q$, $P \lor Q$, $P \lthen Q$ itn.
  \end{itemize}
\item \textbf{postfiksne operacije} so take, ki jih pišemo \emph{za} argument:
  \begin{itemize}
  \item $n!$ za faktorielo števila $n$.
  \end{itemize}
\end{itemize}
%
Včasih uporabljamo tudi druge oblike zapisa:
%
\begin{itemize}
\item \emph{potenciranje} $A^B$
\item \emph{ulomki} $\frac{a}{b}$
\item integrali $\int f(x) dx$ in vsote $\sum_{i = 0}^n a_i$,
\item zapis podmnožice $\set{x \in \RR \such x^2 + x > 2}$.
\end{itemize}
%
Operacija je lahko celo ">nevidna"<, oziroma jo pišemo kot presledek med argumentoma:
%
\begin{itemize}
\item $x \, y$ kot zmnožek $x$ in $y$,
\item $\sin x$ kot uporabe funkcije $\sin$ na argumentu~$x$.
\end{itemize}


\subsection{Oklepaji, prioriteta in asociiranost}

Z oklepaji ponazorimo, katera operacija ima prednost. Na primer, če ne bi imeli dogovora, da ima množenje prednost pred
seštevanjem, potem bi lahko izraz $3 + 4 \times 5$ razumeli kot $3 + (4 \times 5)$ ali kot $(3 + 4) \times 5$. Oklepajev ne smemo opustiti, kadar bi lahko prišlo do take zmede. Nikoli pa ne škodi, če zapišemo kak oklepaj več, kot je to potrebno (v mejah normale).

Da se izognemo pisanju oklepajev, se dogovorimo, da imajo nekatere operacije prednost pred ostalimi, kar so vas učili že
v osnovni šoli. Pravimo, da imajo operacije \textbf{prioriteto}. Operacija z višjo prioriteto ima prednost pred operacijo z
nižjo prioriteto.

\begin{primer}
  Množenje $\times$ ima višjo prioriteto kot seštevanje $+$ (to je \emph{dogovor} in ne matematično dejstvo).
  Podobno ima konjunkcija $\land$ višjo priroriteto kot disjunkcija $\lor$.
\end{primer}

Poleg prioritete imajo nekatere operacije tudi \textbf{asociiranost}. Kako naj razumemo izraz $8 - 3 - 2$, kot $(8 - 3) - 2$
ali kot $8 - (3 - 2)$? V šoli so vas učili, da je
%
\begin{equation*}
  A - B - C = (A - B) - C
\end{equation*}
%
Pravimo, da $-$ veže na levo oziroma da ima \textbf{levo asociiranost}. Ker beremo z leve na desno, ima večina operacij levo
asociiranost. Velja na primer
%
\begin{align*}
  A + B + C &= (A + B) + C \\
  A \times B \times C &= (A \times B) \times C.
\end{align*}
%
Morda bo kdo pripomnil, da itak velja $(A + B) + C = A + (B + C)$ in da zato ni pomembno, kako razumemo $A + B + C$. To je
res v preprostih primerih, ko vemo, da smo s $+$ označili seštevanje števil. Kaj pa, če s $+$ označimo kako drugo
preslikavo? Ali $(A + B) + C = A + (B + C)$ velja tudi v programiskih jezikih, pri katerih lahko pride do prekoračitve
največjega možnega števila?

Primer operacije z desno asociiranostjo je implikacija: $P \lthen Q \lthen R$ je enako $P \lthen (Q \lthen R)$.


\subsection{Izrazi predstavljajo drevesa}

Izrazi so zaporedja znakov, ki jih pišemo z leve na desno. A kje drugje bi jih morda pisali z desne na levo ali
navpično. Izrazi so le \emph{predstavitve} tako imenovanih \textbf{sintaktičnih dreves}. Na primer $((3 + x) \times y)^2$ predstavlja sintaktično drevo, pri čemer potenciranje predstavimo z znakom ${}^{\wedge}$:
%
\begin{center}
  \begin{tikzpicture}[level/.style={sibling distance=5em/#1},,level distance=2em,
    every node/.style = {align=center}, baseline=(current bounding box.center)
    ]
    \node {${}^{\wedge}$}
    child { node {$\times$}
      child { node {$+$}
        child { node {$3$} }
        child { node {$x$} }
      }
      child { node {$y$} }
    }
    child { node {$2$} } ;
  \end{tikzpicture}
\end{center}
%
O sintaktičnih drevesih ne bomo govorili, a jih omenimo, ker so pomembna iz dveh razlogov: sintaktična drevesa so
\emph{podatkovni tip}, s katerim v programu dejansko obdelujemo izraze; s pomočjo sintaktičnih dreves lahko simbolni zapis
predstavimo kot posebno vrsto algebre, ki omogoča matematično obravnavo izrazov.


\subsection{Ostala sintaktična pravila}

Sintaktičnih pravil je še več, od katerih omenimo le nekatera.

\subsubsection{Podnapisi in nadnapisi}
\label{sec:podnapisi-nadnapisi}

Argumente operacije ali funkcije včasih zapišemo v \textbf{podnapis} ali \textbf{nadnapis}. Na primer, če je $a : \NN \to \RR$
preslikava, pogosto pišemo $a_i$ namesto $a(i)$.

\subsubsection{Implicitni argumenti}
\label{sec:implicitni-argumenti}

Argumente operacije lahko opustimo in od bralca pričakujemo, da bo pravilno uganil, kaj smo mislili. Pravimo, da so to
\textbf{implicitni argumenti}. Primer implicitnih argumetov smo že videli, ko smo zapisali prvo in drugo projekcijo $\fst$ in
$\snd$:
\begin{align*}
  \fst &: A \times B \to A, \\
  \snd &: A \times B \to A.
\end{align*}
%
Če bi bili zelo natančni, bi morali pri projekcijah zapisati tudi množici $A$ in $B$, ki tvorita kartezični produkt, na
primer nekaj takega kot $\fst^{A,B} : A \times B \to A$.
%
Ko torej vpeljemo novo zapis, lahko nekatere argumente razglasimo za \textbf{implicitne}, kar pomeni, da jih bomo opuščali,
kadar to ne pripelje do zmede.

\begin{naloga}
  Ali ima kompozicija preslikav $\circ$ implicitne argumente? Katere?
\end{naloga}

\subsubsection{Privzete vrednosti}
\label{sec:privzete-vrednosti}

Argument operacije ima lahko \textbf{privzeto vrednost}. Na primer logaritem $x$ z osnovo $b$ zapišemo $\log_b x$. Če opustimo~$b$, se razume, da je mišljen desetiški logaritem, $\log x = \log_10 x$. Pravimo, da je privzeta vrednost osnove $b = 10$.

\subsubsection{Preobteževanje}
\label{sec:preobteevanje}

Simbol lahko tudi \textbf{preobtežimo}, da ima več pomenov, nato pa od bralca pričakujemo, da bo uganil, katerega smo
mislili. Na primer, $+$ uporabljamo za
%
seštevanje naravnih števil,
seštevanje celih števil,
seštevanje racionalnih števil,
seštevanje realnih števil,
seštevanje kompleksnih števil,
seštevanje vektorjev,
seštevanje matrik,
itd.
%
S preobteževanjem ne gre pretiravati, ker lahko pripelje do zmede. Običajno z istim simbolom označimo različne operacije, ki imajo kaj skupnega. Na primer, $+$ vedno uporabljamo le za operacijo, ki je komutativna, asociativna in ima nevtralni element.


\section{Logične formule}

Izrazi, ki označujejo števila, se imenujejo \textbf{aritmetični izrazi}.

Irazi, ki označujejo matematične izjave, se imenujejo \textbf{logični izrazi} ali \textbf{logične formule}. Razumevanje, branje in pisanje le-teh zahteva kar nekaj treninga, zato se mu bomo posvetili tu in na vajah. Pravzaprav ne bomo vadili le razumevanja zapisa, ampak tudi, kako matematiki razmišljajo in razumejo drug drugega.

Tu o dokazih in pravilih dokazovanja še ne bomo govorili, bomo pa pojasnili intuitivni pomen logičnih operacij.

Računanje z logičnimi formulami delimo na:
%
\begin{itemize}
\item \textbf{izjavni račun} zaobjema logične veznike $\neg $, $\land$, $\lor$, $\lthen$, $\liff$,
\item \textbf{predikatni račun} zaobjema izjavni račun, ter kvantifikatorja $\forall$ in $\exists$.
\end{itemize}


\subsection{Izjavni račun}

\textbf{Izjavni vezniki} so naslednje operacije:
%
\begin{itemize}

\item \textbf{resničnostni konstanti} $\bot$ in $\top$: beremo ju ">neresnica"> in ">resnica"<,

\item \textbf{negacija} $\neg$: izjavo $\neg A$ beremo ">$A$ ne velja"< ali ">ni res, da $A$"<,

\item \textbf{konjunkcija} $\land$: izjavo $A \land B$ beremo ">$A$ in $B$"<,

\item \textbf{disjunkcija} $\lor$: izjavo $A \lor B$ beremo ">$A$ ali $B$"<,

\item \textbf{implikacija} $\lthen$: izjavo $A \lthen B$ lahko beremo na več načinov:
  %
  \begin{itemize}
  \item ">Iz $A$ sledi $B$."<
  \item ">Če $A$, potem $B$."<
  \item ">$A$ samo če $B$."<
  \item ">$B$ sledi iz $A$."<
  \item ">$A$ je zadosten pogoj za $B$."<
  \item ">$B$ je potreben pogoj za $A$."<
  \end{itemize}
  %
\item \textbf{ekvivalenca} $\liff$: izjavo $A \liff B$ beremo
  %
  \begin{itemize}
  \item ">$A$ je ekvivalentno $B$."<
  \item ">$A$, če in samo če $B$."<
  \item ">$A$ natanko tedaj, ko $B$."<
  \item ">$A$ je zadosten in potreben pogoj za $B$."<
  \end{itemize}
\end{itemize}
%
Malo bolj neobičajna je:
%
\begin{itemize}
\item \textbf{ekskluzivna disjunkcija} $\oplus$: izjava $A \oplus B$ beremo ">bodisi $A$ bodisi $B$"< ali ">$A$ ali $B$, vendar ne oba hkrati"<.
\end{itemize}
%
Prioriteta veznikov, od najvišje do najnižje:
%
\begin{itemize}
\item $\neg$,
\item $\land$,
\item $\lor$, $\oplus$,
\item $\lthen$, $\liff$.
\end{itemize}

\begin{primer}
  Izraz $\neg A \land B \lthen C \lor D$ beremo kot $((\neg  A) \land B) \lthen (C \lor D)$.
\end{primer}

Asociranost veznikov:
%
\begin{itemize}
\item leva asociiranost: $\land$, $\lor$, $\oplus$,
\item desna asociiranost: $\lthen$.
\end{itemize}
%
Ekvivalenca $\liff$ nima asociiranosti, zato je zapis $A \liff B \liff C$ načeloma dvoumen, a v praksi pomeni $(A \liff B) \land (B \liff C)$.

\begin{opomba}
  Tudi zapis $x = y = z$ pravzaprav ni smiselen, saj sta $(x = y) = z$ in $x = (y = z)$ oba nesmiselna. V praksi $x = y = z$ pomeni $(x = y) \land (y = z)$. Pa še to: koliko enačb je izraženih z $a = b = c = d$? Tri! Toliko kot je enačajev.
\end{opomba}

\begin{opomba}
  Zapis $x \neq y \neq z$ je nejasen in se mu je bolje izogibati, saj zlahka pripelje do pomote, ker iz $x
  \neq y$ in $y \neq z$ ne sledi nujno $x \neq z$.
\end{opomba}

Glede razumevanja veznikov, omenimo:
%
\begin{itemize}
\item disjunkcija je \emph{inkluzivna}, kar pomeni, da je $A \lor B$ resnična izjava, če sta $A$ in $B$ resnični,
\item v implikaciji $A \lthen B$ se $A$ imenuje \textbf{antecedent} in $B$ \textbf{konsekvent}. Implikacija je veljavna, če je antecedent neveljaven,
\item ekvivalenco $A \liff B$ lahko razumemo kot okrajšavo za $(A \lthen B) \land (B \lthen A)$.
\end{itemize}

\subsection{Kvantifikatorja}

Matematične izjave vsebujejo fraze, kot so ">za vse"<, ">za neki"<, ">obstaja vsaj en"<, ">za natanko enega"< ipd. Le-te izrazimo s \textbf{kvantifiaktorji}. Osnovna kvantifikatorja sta \textbf{univerzalni} in \textbf{eksistenčni}.


\subsubsection{Univerzalni kvantifikator $\forall$}

Formulo $\all {x \in A} \phi$ beremo:
%
\begin{itemize}
\item ">Za vsak $x$ iz $A$ velja $\phi$."<,
\item ">Vsi $x$ iz $A$ zadoščajo $\phi$."<,
\item ">$\phi$ za vse $x$ iz $A$."<
\end{itemize}
%
Pika pri tem nima nobenega posebnega pomena, pogosti so tudi zapisi
%
\begin{equation*}
  \forall x \in A \,,\, \phi
  \qquad\text{ali}\qquad
  \forall x : A \,,\, \phi
  \qquad\text{ali}\qquad
  (\forall x : A) \phi.
\end{equation*}
%
Nekateri matematiki pišejo po principu ">piši kao što govoriš"<
%
\begin{equation*}
  \phi, \forall x \in A
  \qquad\qquad \text{(">$\phi$ za vse $x$ iz $A$"<)}
\end{equation*}
%
Ta zapis odsvetujemo, ker ne deluje, ko kombiniramo več kvantifikatorjev hkrati.

Omenili smo že, da $\all{x \in \emptyset} \phi$ vedno velja. To bomo utemeljili v poglavju o pravilih sklepanja.

\subsubsection{Eksistenčni kvantifikator $\exists$}

Formulo $\some{x \in A} \phi$ beremo:
%
\begin{itemize}
\item ">Obstaja $x$ iz $A$ velja $\phi$."<
\item ">Obstaja vsaj en $x$ iz $A$ velja $\phi$."<
\item ">Za neki $x$ iz $A$ velja $\phi$."<
\item ">$\phi$ za neki $x$ iz $A$."<
\end{itemize}
%
S tem povemo, da obstaja \emph{eden ali več} takih~$x$. Na primer, izjava $\some{x \in \NN} x
< 3$ je veljavna, saj je $2$ naravno število, ki je manjša od $3$.


\subsubsection{Prioriteta $\forall$ in $\exists$}

Prioriteta kvantifikatorjev $\forall$ in $\exists$ je nižja od prioritete veznkov. Na primer:
%
\begin{itemize}
\item $\all{x \in A} \phi \land \psi$ je enako $\all{x \in A} (\phi \land \psi)$,
\item $\all{x \in \RR} x > 0 \lthen \phi$ je enako $\all{x \in \RR} (x > 0 \lthen \phi)$.
\end{itemize}
% 
Kvantifikator vedno zaobjame vse, kar zmore:
%
\begin{itemize}
\item $\all{x \in A} \phi \land \exists{y \in B} \psi$ je enako $\all {x \in A} (\phi \land (\exists y \in B . \psi))$ in \emph{ni} enako $(\all{x \in A} \phi) \land (\exists {y \in B} \psi)$,

\item $(P \land \all{x \in A} Q \lthen R) \lthen \some{y \in B} S$ je enako $(P \land \all{x \in A}  (Q \lthen R)) \lthen (\some{y \in B} S)$ in \emph{ni} enako 
  $(P \land (\all{x \in A} Q) \lthen R) \lthen (\some{y \in B} S)$
\end{itemize}

\subsubsection{Kombinacija $\forall$ in $\exists$}

Pozor, vrstnega reda kvantifikatorjev ne smemo mešati:
%
\begin{itemize}
\item $\all{x \in \RR} \some{y \in \RR} x < y$ pomeni ">vsako realno število je manjše od nekega realnega števila"< (kar je res),
\item $\some{x \in \RR} \all{y \in \RR} x < y$ pomeni ">obstaja najmanjše realno število"< (kar ni res).
\end{itemize}
%
To dejstvo bomo utrjevali na vajah. Zapomnite se, da morate biti tudi pri ostalih predmetih posebej pozorni na vrstni red
">za vsak"< in ">obstaja"<. Je profesorica pri analizi rekla ">za vsak $\epsilon > 0$ obstaja tak $\delta > 0$ da \dots"< ali je rekla ">obstaja tak $\delta > 0$ da za vsak $\epsilon > 0$ \dots"<? Če boste zamešali ti dve izjavi na ustnem izpitu iz analize, boste imeli pokvarjen dan, ali pa cele počitnice!


\subsubsection{Kvantifikator z dodatnim pogojem}

Pogosto kvantifikacijo kombiniramo z dodatnim pogojem, na primer:
%
\begin{itemize}
\item ">Obstaja \emph{liho} naravno število, ki ni deljivo s 7."<
\item ">Vsako \emph{sodo} naravno število je deljivo s 3."<
\end{itemize}
%
V prvem primeru je dodatni pogoj izražen z besedico ">liho"< in v drugem s ">sodo"<. Kako zapišemo take izjave s formulo, kam
vtaknemo dodatni pogoj? Izjavi pretvorimo po korakih:
%
\begin{itemize}
\item ">Obstaja liho naravno število, ki ni deljivo s 7."<
\item ">Obstaja naravno število, ki je liho in ki ni deljivo s 7."<
\item ">Obstaja naravno število, ki je liho in deljivo s 7."<
\item ">Obstaja $x$ iz $\NN$, da je $x$ lih in $x$ je deljiva s 7."<
\item $\some{x \in \NN} (\text{$x$ je lih}) \land (\text{$x$ je deljiv s 7})$
\item $\some{x \in \NN} (\some{y \in \NN} x = 2 y + 1) \land (\some{z \in \NN} y = 7 z)$
\end{itemize}
%
In še druga izjava:
\begin{itemize}
\item ">Vsako sodo naravno število je deljivo s 3."<
\item ">Vsako naravno število, ki je sodo, je deljivo s 3."<
\item ">Za vsako naravno število velja, da če je sodo, potem je deljivo s 3."<
\item ">Za vsak $x$ iz $\NN$ velja, če je $x$ sod, potem je $x$ deljiv s $3$."<
\item $\forall{x \in \NN} \text{$x$ sod} \lthen \text{$x$ deljiv s 3}$
\item $\forall{x \in \NN} (\some{y \in \NN} x = 2 y) \lthen (\some{z \in \NN} x = 3 z)$
\end{itemize}
%
Zapomnimo si: \textbf{dodatni pogoj pri $\exists$ izrazimo $\land$} in \textbf{dodatni pogoj pri $\forall$ izrazimo $\lthen$}.

Poglejmo še en primer, ko imamo več možnosti za zapis s formulo:
%
\begin{quote}
  ">Za vsako pozitivno realno število $x$ obstaja tako naravno število $n$, da je $x < n$."<
\end{quote}
%
Začetni del ">za vsako pozitivno realno število"< lahko zapišemo na več načinov:
%
\begin{itemize}
\item $\all{x \in \RR_{> 0}} \some{n \in \NN} x < n$,
\item $\all{x \in \set{y \in \RR \such y > 0}} \some{n \in \NN} x < n$,
\item $\all{x \in \RR} x > 0 \lthen \some{n \in \NN} x < n$,
\item $\all{x > 0} \some{n \in \NN} x < n$.
\end{itemize}
%
Pri prvem načinu moramo biti v naprej dogovorjeni, da $\RR_{> 0}$ označuje množico pozitivnih realnih števil.
Pri drugem načinu smo vstavili definicijo $\RR_{> 0}$, zato dogovor ni več potreben, a je zapis bolj nečitljiv.
Pri tretjem načinu smo predstavili pozitivnost kot dodatni pogoj.
Četrti način je najbolj čitljiv in se pogosto uporablja, a nam ne pove, ali je $x$ realno, celo, ali racionalno število.


\subsubsection{Vezane in proste spremenljivke}

V nekaterih izrazih nastopajo spremenljivke, ki so \textbf{vezane}. To pomeni, da je njihovo območje veljavnosti imejeno,
oziroma da so neke vrste ">lokalne spremenljivke"<. Spremenljivka, ki ni vezana, je \textbf{prosta}. Primeri:
%
\begin{itemize}
\item V funkcijskem predpisu $x \mapsto x^2 + y$ je $x$ vezan in $y$ prost.
\item V funkcijskem predpisu $(x,y) \mapsto x^2 + y$ sta $x$ in $y$ vezana.
\item V integralu $\int (x + a)^2 d x$ je $x$ vezan in $a$ prost.
\item V vsoti $\sum_{i=0}^n (i^2 + 1)$ je $i$ vezan in $n$ prost.
\item V formuli $\all{x \in \RR} x^3 + 3 x < 7$ je $x$ vezana spremenljivka.
\end{itemize}

Če vezano spremenljivko preimenujemo, se izraz ne spremeni. Funkcijska predpisa $x \mapsto a \cdot x^2 + 1$ in $y \mapsto y \cdot y^2 + 1$ sta \emph{enaka}. Vendar pozor, če vezano spremenljivko preimenujemo, za novo ime \emph{ne} smemo izbrati spremnljivke, ki se že pojavlja. Na primer, v integralu
%
\begin{equation*}
  \int_0^1 (a + x)^2 \, d x
\end{equation*}
%
smemo $x$ preimenovati v $t$, zato sta integrala enaka izraza (in imata tudi enako vrednost):
%
\begin{equation*}
  \int_0^1 (a + x)^2 \, d x  = \int_0^1 (a + t)^2 \, d t
\end{equation*}
%
Ne bi pa smeli $x$ preimenovati v $a$, saj bi dobili
%
\begin{equation*}
  \int (a + a)^2 \, d a
\end{equation*}
%
Pravimo, da se je prosta spremenljivka $a$ \emph{ujela} v integral.



\chapter{Enolični obstoj}

\section{Kvantifikator za enolični obstoj `∃!`}

S kvantifikatorjema `∀` in `∃` lahko izrazimo tudi druge kvantifikatorje.
Na primer, "Obstajata vsaj dva elementa `x` in `y` iz `A`, da velja `ϕ` lahko zapišemo

    ∃ x ∈ A . ∃ y ∈ A . x ≠ y ∧ ϕ

**Naloga:** S pomočjo `∀` in `∃` zapišite izjavo »Obstajata natanko dva elementa `x` in `y` iz `A`, da velja `ϕ`«.

Kako pa izrazimo »obstaja natanko en `x` iz `A`, da velja `ϕ(x)`«? Takole:

    (∃ x ∈ A . φ(x)) ∧ ∀ x y ∈ A . φ(x) ∧ φ(y) ⇒ x = y

ali ekvivalentno

    ∃ x ∈ A . φ(x) ∧ ∀ x ∈ A . φ(y) ⇒ x = y

To okrajšamo `∃! x ∈ A . φ(x)` in beremo »Obstaja natanko en `x` iz `A`, da velja `ϕ(x)`«.

Uporablja se tudi zapis `∃¹ x ∈ A . φ(x)`.

\section{Operator enoličnega opisa}

Če dokažemo, da obstaja natanko en `x ∈ A`, ki zadošča pogoju `ϕ(x)`, potem se lahko nanj smiselno sklicujemo z »tisti
`x` iz `A`, ki zadošča `ϕ(x)`«.

Primeri:

* »tisto realno število `x`, za katero je `x³ = 2`« (kubični koren 2)
* »tista množica `S`, ki nima nobenega elementa« (prazna množica)

Proti-primeri:

* »tisto racionalno število `x`, za katero je `x² = 2` (takega števila ni)
* »tisto realno število `x`, za katero je `x² = 2` (dve taki števili sta)
* »tista množica `S`, ki ima natanko en element` (takih množic je zelo veliko)

To je lahko zelo koristen način za opredelitev matematičnih objektov, zato uvedemo zanj simbolni zapis.
Če dokažemo

    ∃! x ∈ A . φ(x)

potem lahko pišemo

    ι x ∈ A . φ(x)

kar preberemo: "tisti `x ∈ A`, za katerega velja `φ(x)`". Torej velja:

    φ(ι x ∈ A . φ(x))

Spremenljivka `x` je *vezana* v `ι x ∈ A . φ(x)`.


**Primer:** Denimo, da še ne bi poznali simbola `√` za kvadatne korene. Tedaj bi
lahko kvadratni koren iz `2` zapisali kot

    ι x ∈ R . (x > 0 ∧ x^2 = 2)

Še več, preslikavo `√ : [0,∞) → [0,∞)` lahko definiramo s pomočjo zapisa `ι`:

    √ : x ↦ (ι y ∈ ℝ . (y ≥ 0 ∧ y² = x))


**Naloga:** Zapiši »limita zaporedja `a : ℕ → ℝ`« z operatorjem `ι` (pod predpostavko, da je `a` konvergentno
zaporedje). Najprej povej z besedami »limita zaporedja `a` je tisti `x ∈ ℝ`, ki ...«, nato pa zapiši še v obliki `ι x ∈ ℝ . ⋯`.

*Pomni:* zapis `ι x ∈ A . ϕ(x)` je dopusten samo v primeru, da velja `∃! x ∈ A . φ(x)`.


\chapter{Spremenljivke in definicije}


Preden v matematičnem ebsedilu uporabimo simbol ali spremenljivko, ga moramo *vpeljati*. To pomeni, da moramo pojasniti, kakšen je pomen simbola. Poznamo dva osnovna načina za vpeljavo novih simbolov:

* Nov simbol `s` lahko **definiramo** kot okrajšavo za neki drugi izraz (ki je lahko tudi logična formula).
* Nov simbol `s` je (prosta) **spremenljivka**, ki predstavlja neki (neznano, poljubno, nedoločeno) element dane množice `A`.

V obeh primerih dodamo simbol `s` v kontekst, se pravi v spisek znanih simbolov. Če smo simbol uvedli le začasno (na
primer v enem poglavju, ali v delu dokaza), ga iz konteksta odstranimo, ko ni več veljaven.

Matematiki zapisujejo definicije in vpeljujejo spremenljivke na razne načine.

\section{Vpeljava spremenljivke}

Če želimo vpeljavi spremljivko `x`, ki predstavlja neki poljuben ali neznani element množice `A`, zapišemo

    Naj bo x ∈ A.

S tem postane `x` veljavna spremenljivka, ki jo lahko uporabljamo. O njen vemo le to, da je element množice `A` – pravimo, da je `x` *prosta* spremenljivka. V matematičnih besedilih boste zasledili tudi naslednje fraze:

* »Naj bo `x ∈ A` poljuben.«
* »Obravnavajmo poljuben `x ∈ A`.«
* »Izberimo poljuben `x ∈ A`.«
* »Denimo, da imamo poljuben `x ∈ A`.«

Pozor, beseda "izberimo" bi komu dala misliti, da si lahko izbere neki konkretni `x`, a to preprečuje beseda "poljuben", ki jo matematik uporabi, kadar želi povedati, da je `x` neznana (poljubna) vrednost.

**Naloga:** Denimo, da učitelj reče »Naj bo `n` (poljubno) naravno število«, nato pa vas vpraša »Ali je `n` sodo število?« Kako boste odgovorili?


\section{Definicija simbola}

Definicija je v prvi vrsti **okrajšava** za neki izraz. Z njo uvedemo nov simbol `s` in mu pripišemo neko vrednost.
Simbol `s` je enak vrednosti, ki smo mu jo pripisali. Simbolni zapis za definicijo je

    s := ⋯

Na primer, v besedilu bi lahko napisali "Naj bo `s := √(log₂ 7 + π/6)`." S tem smo v kontekst dodali simbol `s` in
predpostavko, da je `s` enak `√(log₂ 7 + π/6)`. V matematičnih besedili boste zasledili tudi naslednje načine za definicijo:

* `s = √(log₂ 7 + π/6)` (namesto `:=` uporabimo `=`)
* `s ≡ √(log₂ 7 + π/6)` (namesto `:=` uporabimo `≡`)
* `s =̂ √(log₂ 7 + π/6)` (namesto `:=` uporabimo `=̂`)

Kadar definiramo simbol tako, da mu priredimo funkcijski predpis, recimo

    f := (x ↦ x² + 7)

to raje zapišemo kot

    f(x) := x² + 7

Kadar definiramo simbol s pomočjo enoličnega obstoja, recimo

    r := ι x ∈ ℝ . x³ = 2

to raje zapišemo z besedami:

    Naj bo r tisto realno število, ki zadošča r³ = 2.

Poglejmo še, kako definiramo okrajšave za logične formule. Defnimo, da želimo s `ϕ(x)` označiti izjavo `∃ y ∈ ℝ . y² =
x + 1`. Glede na zgornji dogovor, zapišemo

    ϕ := (x ↦ (∃ y ∈ ℝ . y² = x + 1))

ali

    ϕ(x) := (∃ y ∈ ℝ . y² = x + 1)

Vendar takega zapisa v praksi ne boste videli. Dosti bolj pogost je zapis

    ϕ(x) :⇔ ∃ y ∈ ℝ . y² = x + 1

ali pa kar `ϕ(x) ⇔ ∃ y ∈ ℝ . y² = x + 1`.


\subsection{Definicije novih matematičnih pojmov}

Kaj pa definicije novih pojmov, ki jih srečujete pri predavanjih, denimo pri analizi? Na primer:

**Definicija:** Zaporedje števil `a : ℕ → ℝ` je *neomejeno*, če za vsak `x ∈ ℝ` obstaja `i ∈ ℕ`, da je `aᵢ > x`.

S stališča simbolnega zapisa, je to le uveba novega simbola `neomejeno`:

    neomejeno(a) := (∀ x ∈ ℝ . ∃ i ∈ ℕ . aᵢ > x)

Seveda bistvo take definicije ni le krajši zapis izjave `(∀ x ∈ ℝ . ∃ i ∈ ℕ . aᵢ > x)`, ampak uporabna vrednost pojma
"neomejeno zaporedje".


\chapter{Konstrukcije in dokazi}

Matematiki v sklopu svojih aktivnosti **konstruiramo** matematične objekte:

* v geometriji so znane konstrukcije z ravnilom in šestilom
* računanje števk števila `π` je konstrukcija približka
* reševanje enačbe, je konstrukcija števila z želeno lastnostjo
* konstruiramo lahko elemente množice, pogosto kar tako, da jih zapišemo, na primer `(2, \inl(3)) ∈ ℕ × (ℤ + ℤ)`

Poleg tega **dokazujemo** matematične izjave. Na dokaz lahko gledamo kot na konstrukcijo, saj je to le še ena zvrst
matematičnega objekta. Ker pa so dokazi skoraj vedno zapisani v naravnem jeziku, jih matematiki pogosto dojemajo ločeno
od ostalih matematičnih objekotv (števila, preslikave, množice, ploskve, ...).

Kaj pravzaprav je dokaz? V prvi vrsti je dokaz utemeljitev matematične izjave. Zgrajen je po točno določenih *pravilih
sklepanja*, ki jih lahko podamo formalno in jih tudi implementiramo na računalniku, s čimer se pri tem predmetu ne bomo
ubadali, a kogar to zanima, si lahko ogleda »[The dawn of formalized mathematics](https://youtu.be/Z500sma3h90)«
([prosojnice](https://www.icloud.com/keynote/0Gkr1yM7XY-31aQleWf-fiW7A#The_Dawn_of_Formalized_Mathematics)) in se nauči
uporabljati kak [dokazovalni pomočnik](https://ncatlab.org/nlab/show/proof+assistant) (v zadnjem času hitro napreduje [Lean](https://leanprover.github.io)).

V praksi ljudje ne pišejo vseh podrobnosti v dokazu, ker bi bil tak dokaz nečitljiv in nerazumljiv. Pogosto podajo samo
glavno idejo, iz katere lahko izkušeni matematik sam rekonstruira dokaz. Iz dobro napisanega dokaza se lahko naučimo
marsikaj novega, poleg goleda dejstva, da dokaza izjava velja.

Mi bomo vadili podrobno pisanje dokazov. Pri ostalih predmetih boste videli "žive dokaze", ki imajo manj podrobnosti in so zapisani manj formalno. A vsi pravilni matematični dokazi se dajo zapisati na način, kot ga bomo predstavili mi (in celo zapisati povsem formalon z dokazovalnim pomočnikom).

\section{Kako pišemo dokaze}

Pravila sklepanja so kot pravila igre. Ne povedo, kako dobro igrati, samo kaj je dovoljeno. Seveda bomo hkrati s pravili
sklepanja povedali nekaj namigov in nasvetov, kako dokaz poiščemo. A kot pri vsaki igri velja, da vaja dela mojstra.

Dokaz ima vgnezdeno strukturo: sestoji iz delov in pod-dokazov, ki sestojijo iz nadaljnih pod-dokazov itn., ki se
zaključijo z osnovnimi dejstvi. Vsi ti kosi so s pomočjo pravil sklepanja zloženi v dokazno "drevo".

Ko pišemo dokaz, moramo v vsakem trenutku poznati

* **cilj**: kaj trenutno dokazujemo in
* **kontekst**: katere spremenljivke in predpostavke imamo trenutno na voljo.

Ko napravimo korak v dokazu, mora biti utemeljen z enim od pravil sklepanja. Dokaz je
popoln, ko smo utemeljili vse pod-dokaze, ki ga sestavljajo.

\subsubsection{Primer}

**Izjava:** `(p ∨ q) ∧ r ⇒ (p ∧ r) ∨ (q ∧ r).`

Dokaz:

```
Dokažimo (p ∨ q) ∧ r ⇒ (p ∧ r) ∨ (q ∧ r)
  Predpostavimo (p ∨ q) ∧ r.                         (1)
  Zaradi (1) velja p ∨ q.                            (2)
  Zaradi (1) velja r.                                (3)
  Dokažimo (p ∧ r) ∨ (p ∧ r).
     Zaradi (2) lahko obravnavamo dva primera:
         1. Če velja p:                              (4)
             Dokažimo (p ∧ r) ∨ (p ∧ r)
                Dokažimo p ∧ r:
                     1.1. p velja zaradi (4)
                     1.2. r velja zaradi (3)
         2. Če velja q:                              (5)
             Dokažimo (p ∧ r) ∨ (q ∧ r)
                Dokažimo q ∧ r:
                     1.1. q velja q zaradi (5)
                     1.2. r velja r zaradi (3)
Konec dokaza.
```

Dokaz bi bolj po človeško napisali takole:

> *Dokaz.* Predpostavimo `p ∨ q` in `r`. Če velja `p`, potem sledi `p ∧ r` ter od tod `(p ∧ r) ∨ (p ∧ r)`. Če pa velja `q`, sledi `q ∧ r` ter spet `(p ∧ r) ∨ (p ∧ r)`. □

Ali pa kar takole:

> *Dokaz:* Očitno.

\section{Pravila sklepanja}

Pravila sklepanja delimo na dve vrsti:

* **pravila vpeljave** povedo, kako dokažemo izjavo
* **pravila uporabe** povedo, kako lahko že znano izjavo uporabimo v dokazu neke druge izjave

Poleg tega poznamo še pravila o zamenjavi:

* **zamenjava enakih izrazov**: izraz lahko vedno zamenjamo z njim enakim
* **zamenjava ekvivalentnih izjav**: izjavo vedno lahko zamenjamo z njej ekvivalentno

Pravila sklepanja so zapisana v priloženi datoteki.



\chapter{Logika in pravila sklepanja (dodatno poglavje)}
\label{chap:logika}


\textbf{Opomba:} To poglavje je del učbenika v nastajanju in ni povsem v skladu s predavanji. Kljub temu ga vključujem v te zapiske, ker vsebuje precej koristnih nasvetov in misli.

%%%%%%%%%%%%%%%%%%%%%%%%%%%%%%%%%%%%%%%%%%%%%%%%%%%%%%%%%%%%%%%%%%%%%%
\section{Kaj je matematični dokaz?}
\label{sec:kaj-je-dokaz}

V srednji šoli se dijaki pri matematiki učijo, \emph{kako} se kaj
izračuna. Na univerzi imajo študentje matematike pred seboj
zahtevnejšo nalogo: poleg \emph{kako} morajo vedeti tudi \emph{zakaj}.
Od njih se pričakuje, da bodo računske postopke znali tudi utemeljiti,
ne pa samo slediti pravilom, ki jih je predpisal učitelj. Razumeti
morajo dokaze znamenitih izrekov in sami poiskati dokaze preprostih
izjav. Da bi se lažje spopadli s temi novimi nalogami, bomo prvi del
predmeta Logika in množice posvetili matematični infrastrukturi:
izjavam, pra\-vi\-lom sklepanja in dokazom. Učili se bomo, kako pišemo
dokaze, kako jih analiziramo in kako jih sami poiščemo.

Osrednji pojem matematične aktivnosti je \emph{dokaz}. Namen dokaza je
s pomočjo točno določenih in vnaprej dogovorjenih \emph{pravil
  sklepanja} utemeljiti neko matematično \emph{izjavo}. Načeloma mora
dokaz vsebovati vse podrobnosti in natanko opisati posamezne korake
sklepanja, ki privedejo do želene matematične izjave. Ker bi bili taki
dokazi zelo dolgi in bi vsebovali nezanimive podrobnosti, matematiki
običajno predstavijo samo oris ali glavno zamisel dokaza. Izkušenemu
matematiku to zadošča, saj zna oris sam dopolniti do pravega dokaza.
Matematik začetnik seveda potrebuje več podrobnosti. Poglejmo si
primer.

\begin{izrek}
  \label{izr:n3-n-deljivo-3}
  Za vsako naravno število $n$ je $n^3 - n$ deljivo s~$3$.
\end{izrek}

\noindent
Po kratkem premisleku bi izkušeni matematik zapisal:

\begin{quote}
  \begin{proof}
    Očitno.
  \end{proof}
\end{quote}

\noindent
To ni dokaz, izkušeni matematik nam le dopoveduje, da je (zanj) izrek
zelo lahek in da nima smisla izgubljati časa s pisanjem dokaza.
Začetnik, ki težko razume že sam izrek, bo ob takem ">dokazu"< seveda
zgrožen. Verjetno bo najprej preizkusil izrek na nekaj primerih:
%
\begin{align*}
  1^3 - 1 &= 0,\\
  2^3 - 2 &= 8 - 2 = 6,\\
  3^3 - 3 &= 27 - 4 = 24,\\
  4^3 - 4 &= 64 - 4 = 60.
\end{align*}
%
Res dobivamo večkratnike števila~$3$. Ali smo izrek s tem dokazali? Seveda ne!
Preizkusili smo le štiri primere, ostane pa jih še neskončno mnogo. Kdor
misli, da lahko iz nekaj primerov sklepa na splošno veljavnost, naj v poduk
vzame naslednjo nalogo.

\begin{vaja}
  Ali je $n^2 - n + 41$ praštevilo za vsako naravno število~$n$?
\end{vaja}

\noindent
Ko izkušenega matematika prosimo, da naj nam vsaj pojasni idejo dokaza,
zapiše:

\begin{quote}
  \begin{proof}
    Število $n^3 - n$ je zmnožek treh zaporednih naravnih števil.
  \end{proof}
\end{quote}

\noindent
To še vedno ni dokaz, ampak samo namig. Starejši študenti matematike pa
bi iz namiga morali znati sestaviti naslednji dokaz:

\begin{quote}
  \begin{proof}
    Ker je $n^3 - n = (n-1) \cdot n \cdot (n+1)$, je $n^3 - n$ zmnožek
    treh zaporednih naravnih števil, od katerih je eno deljivo s~$3$,
    torej je tudi $n^3 - n$ deljivo s~$3$.
  \end{proof}
\end{quote}

\noindent
(Mimogrede, s škatlico $\Box$ označimo konec dokaza.) Čeprav bi bila
večina matematikov s tem dokazom zadovoljna, bi morali za popoln dokaz
preveriti še nekaj podrobnosti:
%
\begin{enumerate}
\item Ali res velja $n^3 - n = (n-1) \cdot n \cdot (n+1)$?
\item Ali je res, da je izmed treh zaporednih naravnih števil eno
  vedno deljivo s~$3$?
\item Ali je res, da je zmnožek treh števil deljiv s~$3$, če je eno od
  števil deljivo s~$3$?
\end{enumerate}
%
S srednješolskim znanjem algebre ugotovimo, da je odgovor na prvo
vprašanje pritrdilen. Tudi odgovora na drugo in tretje vprašanje sta
očitno pritrdilna, mar ne? To pa ne pomeni, da ju ni treba dokazati.
Nasprotno, zgodovina matematike nas uči, da moramo prav ">očitne"<
izjave še posebej skrbno preveriti.

\begin{vaja}
  Kakšno je tvoje mnenje o resničnosti naslednjih izjav? Vprašaj
  starejše kolege, asistente in učitelje, kaj menijo oni. Ali znajo
  svoje mnenje utemeljiti z dokazi?
  %
  \begin{enumerate}
  \item Sodih števil je manj kot naravnih števil.
  \item Kroglo je mogoče razdeliti na pet delov tako, da lahko iz njih
    sestavimo dve krogli, ki sta enako veliki kot prvotna krogla.
  \item Sklenjena krivulja v ravnini, ki ne seka same sebe, razdeli
    ravnino na dve območji, eno omejeno in eno neomejeno.
  \item S krivuljo ne moremo prekriti notranjosti kvadrata.
  \item Če ravnino razdelimo na tri območja, potem zagotovo obstaja
    točka, ki je dvomeja in ni tromeja med območji.
  \end{enumerate}
\end{vaja}

\noindent
%
Vrnimo s k izreku~\ref{izr:n3-n-deljivo-3}. Če dokaz zapišemo preveč
podrobno, postane dolgočasen in ne\-ra\-zumljiv:

\begin{quote}
  \begin{proof}
    Naj bo $n$ poljubno naravno število. Tedaj velja
    %
    \begin{align*}
      n^3 - n
      &= n \cdot n^2 - n \cdot 1 \\
      &= n \cdot (n^2 - 1) \\
      &= n \cdot ((n + 1) \cdot (n - 1)) \\
      &= n \cdot ((n - 1) \cdot (n + 1)) \\
      &= (n \cdot (n - 1)) \cdot (n + 1) \\
      &= (n - 1) \cdot n \cdot (n + 1).
    \end{align*}
    %
    Vidimo, da je $n^3 - n$ zmnožek treh zaporednih naravnih števil.
    Dokažimo, da je eno od njih deljivo s~$3$. Število $n$ lahko
    enolično zapišemo kot $n = 3 k + r$, kjer je $k$ naravno število
    in $r = 0$, $r = 1$ ali $r = 2$. Obravavajmo tri primere:
    %
    \begin{itemize}
    \item če je $r = 0$, je $n = 3 k$, zato je $n$ deljiv s~$3$,
    \item če je $r = 1$, je $n - 1 = (3 k + 1) - 1 = 3 k + (1 - 1) = 3
      k + 0 = 3 k$, zato je $n-1$ deljiv s~$3$,
    \item če je $r = 2$, je $n + 1 = (3 k + 2) + 1 = 3 k + (2 + 1) = 3
      k + 3 = 3 k + 3 \cdot 1 = 3 (k +1)$, zato je $n+1$ deljiv s~$3$.
    \end{itemize}
    %
    Vemo torej, da je $n - 1$, $n$ ali $n + 1$ deljiv s~$3$.
    Obravnavamo tri primere:
    %
    \begin{itemize}
    \item Če je $n - 1$ deljiv s~$3$, tedaj  obstaja naravno število
      $k$, da je $n - 1 = 3 k$. V tem primeru je $(n - 1) n (n + 1) =
      (3 k) n (n + 1) = 3 (k n (n + 1))$, zato je $(n - 1) n (n + 1)$
      deljivo s~$3$.
    \item Če je $n$ deljiv s~$3$, tedaj obstaja naravno število $k$,
      da je $n = 3 k$. V tem primeru je $(n - 1) n (n + 1) = (n - 1)
      (3 k) n (n + 1) = (3 k) (n - 1) (n + 1) = 3 (k (n - 1) (n +
      1))$, zato je $(n - 1) n (n + 1)$ deljivo s~$3$.
    \item Če je $n + 1$ deljiv s~$3$, tedaj obstaja naravno število
      $k$, da je $n + 1 = 3 k$. V tem primeru je $(n - 1) n (n + 1) =
      (n - 1) n (3 k) = (n - 1) (3 k) n = (3 k) (n - 1) n = 3 (k (n -
      1) n)$, zato je $(n - 1) n (n + 1)$ deljivo s~$3$.
    \end{itemize}
    %
    V vsakem primeru je $(n - 1) n (n + 1)$ deljivo s~$3$. Ker smo
    dokazali, da je $n^3 = n = (n - 1) n (n + 1)$, je tudi $n^3 - n$
    deljivo s~$3$.
  \end{proof}
\end{quote}

\begin{vaja}
  S kolegi se igraj naslednjo igro.\footnote{%
    Igranje odsvetujemo zunaj prostorov Fakultete za matematiko in fiziko.}
  Prvi igralec v zgornjem dokazu poišče korak, ki ga je treba še dodatno
  utemeljiti. Drugi igralec ga utemelji. Nato prvi igralec poišče nov korak,
  ki ga je treba še dodatno utemeljiti in igra se ponovi. Zgubi tisti, ki se prvi naveliča igrati. Ali lahko igra traja neskončno dolgo?
\end{vaja}

Matematični dokaz ima dvojno vlogo. Po eni strani je utemeljitev matematične
izjave, zato mora biti čim bolj podroben. V idealnem primeru bi bil dokaz
zapisan tako, da bi lahko njegovo pravilnost preverili mehansko, z
računalnikom. Po drugi strani je dokaz sredstvo za komunikacijo idej med
matematiki, zato mora vsebovati ravno pravo mero podrobnosti. Mera pa je
odvisna od tega, komu je dokaz namenjen. Te socialne komponente se študenti
učijo skozi prakso v toku študija. Dokazu kot povsem matematičnemu pojmu pa se
bomo posvetili prav pri predmetu Logika in množice. Pojasnili bomo, kaj je
dokaz kot matematični konstrukt in kako ga zapišemo tako podrobno, da je res
mehansko preverljiv. Naučili se bomo tudi nekaj preprostih tehnik iskanja
dokazov, ki pa še zdaleč ne bodo zadostovale za reševanje zares zanimivih
matematičnih problemov, ki zahtevajo poglobljeno znanje, vztrajnost, kanček
talenta in nekaj sreče.


%%%%%%%%%%%%%%%%%%%%%%%%%%%%%%%%%%%%%%%%%%%%%%%%%%%%%%%%%%%%%%%%%%%%%%

\section{Simbolni zapis matematičnih izjav}
\label{sec:simbolni-zapis-izjav}

Matematična \textbf{izjava} je smiselno besedilo, ki izraža kako lastnost ali
razmerje med matematičnimi objekti (števili, liki, funkcijami, množicami
itn.). Primeri matematičnih izjav:
%
\begin{itemize}
\item $2 + 2 = 5$.
\item Točke $P$, $Q$ in $R$ so kolinearne.
\item Enačba $x^2 + 1 = 0$ nima realnih rešitev.
\item $a > 5$.
\item $\phi \lor \psi \lthen (\lnot \phi \lthen \psi)$.
\end{itemize}
%
Vidimo, da je lahko izjava resnična, neresnična, ali pa je resničnost
izjave \emph{odvisna} od vrednosti spremenljivk, ki nastopajo v njej.
Primeri besedila, ki niso matematične izjave:
%
\begin{itemize}
\item Ali je $2 + 2 = 5$?
\item Za vsak $x > 5$.
\item Študenti bi morali znati reševati diferencialne enačbe.
\item Od nekdaj lepe so Ljubljanke slovele, al lepše od Urške bilo ni nobene.
\item $\phi \lor ) \psi \lthen \psi$.
\end{itemize}
%
Matematične izjave običajno pišemo kombinirano v naravnem jeziku in z
matematični simboli, saj so tako najlažje razumljive ljudem. Za
potrebe matematične logike pa izjave pišemo \emph{samo} z
matematičnimi simboli. Tako zapisani izjavi pravimo \textbf{logična
  formula}. V ta namen moramo nadomestiti osnovne gradnike izjav, kot
so ">in"<, ">ali"< in ">za vsak"<, z \textbf{logičnimi operacijami}.
Le-te delimo na tri sklope. V prvi sklop sodita \textbf{logični
  konstanti}:
%
\begin{itemize}
\item resnica $\top$,
\item neresnica $\bot$.
\end{itemize}
%
V računalništvu resnico $\top$ pogosto označimo z $1$ ali \texttt{True} in
neresnico $\bot$ z $0$ ali \texttt{False}. Naslednji sklop so \textbf{logični
vezniki}, s katerimi sestavljamo nove izjave iz že danih:
%
\begin{itemize}
\item konjunkcija $\phi \land \psi$, beremo ">$\phi$ in $\psi$"<,
\item disjunkcija $\phi \lor \psi$, beremo ">$\phi$ ali $\psi$"<,
\item implikacija $\phi \lthen \psi$, beremo ">če $\phi$ potem $\psi$"<,
\item ekvivalenca $\phi \liff \psi$, beremo ">$\phi$ če, in samo če, $\psi$"< ali pa ">$\phi$ natanko tedaj, kadar~$\psi$"<,
\item negacija $\lnot \phi$, beremo ">ne $\phi$"<,
\end{itemize}
%
V tretji sklop sodita \textbf{logična kvantifikatorja}:
%
\begin{itemize}
\item univerzalni kvantifikator $\all{x \in S} \phi$, beremo ">za vse $x$
  iz $S$ velja $\phi$"<,
\item eksistenčni kvantifikator $\some{x \in S} \phi$, beremo ">obstaja
  $x$ v $S$, da velja $\phi$"<,
\end{itemize}
%
Pri tem je $S$ množica, razred\footnote{V poglavju~\ref{chap:mnozice}
  bomo spoznali razliko med množicami in razredi, zaenkrat si $S$
  predstavljamo kot množico.} ali tip spremenljivke~$x$. V praksi se
uporablja več inačic zapisa za kvantifikatorje, kot so ">$\forall x : S
.\, \phi$"<, ">$\forall x \in S : \phi$"< in ">$(\forall x \in S)
\phi$"<. Srečamo tudi zapis ">$\phi, \forall x \in S$"<, ki pa ga
odsvetujemo.

\textbf{Neomejena kvantifikatorja} $\all{x} \phi$ in
$\some{x} \phi$ se uporabljata, kadar je vnaprej znana množica $S$,
po kateri teče spremenljivka~$x$. V matematičnem besedilu je običajno
razvidna iz spremnega besedila, včasih pa je treba upoštevati
ustaljene navade: $n$ je naravno ali celo število, $x$ realno, $f$ je
funkcija ipd.

V uporabi so nekatere ustaljene okrajšave:
%
\begin{xalignat*}{3}
  &\some{x,y \in S} \phi,&
  &\text{pomeni}&
  &\some{x}{S} (\some{y}{S} \phi),\\
  %
  &\all{x \in S,y \in T} \phi,&
  &\text{pomeni}&
  &\all{x \in S}(\all{y \in T} \phi),\\
  %
  &\phi \liff \psi \liff \rho \liff \sigma&
  &\text{pomeni}&
  &(\phi \liff \psi) \land (\psi \liff \rho) \land (\rho \liff \sigma),\\
  %
  &f(x) = g(x) = h(x) = i(x)&
  &\text{pomeni}&
  &f(x) = g(x) \land g(x) = h(x) \land h(x) = i(x),\\
  &a \leq b < c \leq d&
  &\text{pomeni}&
  &a \leq b \land b < c \land c \leq d.
\end{xalignat*}
%
Nekatere okrajšave odsvetujemo. V nizu neenakosti naj gredo vse
primerjave v isto smer. Torej ne pišemo $a \leq b < c \geq d$, ker se
zlahka zmotimo in mislimo, da velja $a \geq d$. To bi morali zapisati
ločeno kot $a \leq b < c$ in $c \geq d$. Prav tako ne nizamo neenakosti,
saj premnogi iz $f(x) \neq g(x) \neq h(x)$ ">sklepajo"< $f(x) \neq
h(x)$, čeprav neenakost \emph{ni} tranzitivna relacija. Zapis $f(x) =
g(x) \neq h(x) = i(x)$ je v redu, saj ena sama neenakost ne povzroči
težav.

\begin{vaja}
  Zapiši $f(x) = g(x) \neq h(x) = i(x)$ brez okrajšav.
\end{vaja}

Povejmo še nekaj o pisanju oklepajev. Oklepaji povedo, katera
operacija ima prednost. Kadar manjkajo, moramo poznati dogovorjeno
\textbf{prioriteto} operacij. Na primer, ker ima množenje višjo
prioriteto kot seštevanje, je $5 \cdot 3 + 8$ enako $(5 \cdot 3) + 8$
in ne $5 \cdot (3 + 8)$. Tudi logične operacije imajo svoje
prioritete, ki pa niso tako splošno znane kot prioritete aritmetičnih
operacij. Zato bodite pazljivi, ko berete tuje besedilo.

Mi bomo privzeli naslednje prioritete logičnih operacij:
%
\begin{itemize}
\item negacija $\lnot$ ima prednost pred
\item konjunkcijo $\land$, ki ima prednost pred
\item disjunkcijo $\lor$, ki ima prednost pred
\item implikacijo $\lthen$, ki ima prednost pred
\item kvantifikatorjema $\forall$ in $\exists$.
\end{itemize}
%
Na primer:
%
\begin{itemize}
\item $\lnot \phi \lor \psi$ je isto kot $(\lnot \phi) \lor \psi$,
\item $\lnot \lnot \phi \lthen \phi$ je isto kot $(\lnot(\lnot\phi))
  \lthen \phi$,
\item $\phi \lor \psi \land \rho$ je isto kot $\phi \lor (\psi \land \rho)$,
\item $\phi \land \psi \lthen \phi \lor \psi$ je isto kot $(\phi
  \land \psi) \lthen (\phi \lor \psi)$,
\item $\all{x \in S} \phi \lthen \psi$ je isto kot $\all{x \in S} (\phi
    \lthen \psi)$,
\item $\some{x \in S} \phi \land \psi$ je isto kot $\some{x \in S} (\phi
    \land \psi)$.
\end{itemize}

V aritmetiki poznamo poleg prioritete operacij tudi \textbf{levo} in
\textbf{desno asociranost}. Denimo, seštevanje je levo asocirano,
ker beremo $5 + 3 + 7$ kot $(5 + 3) + 7$, saj najprej izračunamo $5 +
3$ in nato $8 + 7$. Pri seštevanju to sicer ni pomembno in bi lahko
seštevali tudi $3 + 7$ in nato $5 + 10$. Drugače je z odštevanjem,
kjer $5 - 3 - 7$ pomeni $(5 - 3) - 7$ in ne $5 - (3 - 7)$. Tudi za
logične operacije velja dogovor o njihovi asociranosti. Konjunkcija in
disjunkcija sta levo asocirani:
% 
\begin{align*}
  \phi \land \psi \land \rho
  &\qquad\text{pomeni}\qquad
  (\phi \land \psi) \land \rho,\\
  \phi \lor \psi \lor \rho
  &\qquad\text{pomeni}\qquad
  (\phi \lor \psi) \lor \rho.
\end{align*}
%
Za disjunkcijo in konjunkcijo sicer ni pomembno, kako postavimo
oklepaje, ker sta obe možnosti med seboj ekvivalentni, vendar je prav,
da natančno določimo, katera od njiju je mišljena. V logiki je
implikacija desno asocirana:
%
\begin{equation*}
  \phi \lthen \psi \lthen \rho
  \qquad\text{pomeni}\qquad
  \phi \lthen (\psi \lthen \rho).
\end{equation*}
%
Tu \emph{ni} vseeno, kako postavimo oklepaje, saj $\phi \lthen (\psi
\lthen \rho)$ in $(\phi \lthen \psi) \lthen \rho$ v splošnem nista
ekvivalentna. Vendar pozor! Ko matematiki, ki niso logiki, v
matematičnem besedilu zapišejo
%
\begin{equation*}
  \phi \lthen \psi \lthen \rho,
\end{equation*}
%
s tem skoraj vedno mislijo
%
\begin{equation*}
  (\phi \lthen \psi) \land (\psi \lthen \rho).
\end{equation*}
%
Zakaj? Zato ker je to priročen zapis, ki nakazuje zaporedje sklepov
">iz $\phi$ sledi $\psi$ in nato iz $\psi$ sledi $\rho$"<, še posebej,
če je zapisan v več vrsticah. Recimo, za nenegativni števili $x$ in
$y$ bi takole zapisali utemeljitev neenakosti med aritmetično in
geometrijsko sredino:
%
\begin{align*}
  & (x - y)^2 \geq 0 \lthen \\
  & x^2 - 2 x y + y^2 \geq 0 \lthen
  \tag{razstavimo}\\
  & x^2 + 2 x y + y^2 \geq 4 x y \lthen
  \tag{prištejemo $4 x y$}\\
  & (x + y)^2 \geq 4 x y \lthen
  \tag{faktoriziramo}\\
  & \frac{(x + y)^2}{4} \geq x y \lthen
  \tag{delimo s $4$}\\
  & \frac{x+y}{2} \geq \sqrt{x y}.
  \tag{korenimo}
\end{align*}
%ANDREJ: meni je Gordon rekel, da utemeljitve sledijo sklepu, torej so
% eno vrstico niže.
%
Matematiki radi celo spustijo znak $\lthen$ in preprosto vsak
naslednji sklep napišejo v svojo vrstico. Ker torej velja tak ustaljen
način pisanja zaporedja sklepov, je varneje pisati $\phi \lthen (\psi
\lthen \rho)$ brez oklepajev, da ne povzročamo zmede.
%ANDREJ: zadnjega stavka ne razumem.

%%%%%%%%%%%%%%%%%%%%%%%%%%%%%%%%%%%%%%%%%%%%%%%%%%%%%%%%%%%%%%%%%%%%%%

\section{Kako beremo in pišemo simbolni zapis}
\label{sec:simbolni-zapis}

Izjave, zapisane v simbolni obliki, ni težko prebrati. Na primer,
%
\begin{equation*}
  \all{x, y \in \RR}
    x^2 = 4 \land y^2 = 4 \lthen x = y,
\end{equation*}
%
preberemo:
%
\begin{quote}
  ">Za vse realne $x$ in $y$, če je $x^2$ enako $4$ in $y^2$ enako
  $4$, potem je $x$ enako $y$."<
\end{quote}
%
Več izkušenj pa je potrebnih, da \emph{razumemo} matematični pomen
take izjave, v tem primeru:
%
\begin{quote}
  ">Enačba $x^2 = 4$ ima največ eno realno rešitev."<
\end{quote}
%
Začetnik potrebuje nekaj vaje, da se navadi brati simbolni zapis. Tudi
prevod v obratno smer, iz besedila v simbolno obliko, ni enostaven,
zato povejmo, kako se prevede nekatere standardne fraze.

\subsubsection{">$\phi$ je zadosten pogoj za $\psi$."<}

To pomeni, da zadošča dokazati $\phi$ zato, da dokažemo $\psi$, ali v
simbolni obliki
%
\begin{equation*}
  \phi \lthen \psi.
\end{equation*}

\subsubsection{">$\phi$ je potreben pogoj za $\psi$."<}

To pomeni, da $\psi$ ne more veljati, ne da bi veljal~$\phi$. Z drugimi
besedami, če velja $\psi$, potem velja tudi $\phi$, kar se v simbolni obliki
zapiše
%
\begin{equation*}
  \psi \lthen \phi.
\end{equation*}

\subsubsection{">$\phi$ je zadosten in potreben pogoj za $\psi$."<}

To je kombinacija prejšnjih dveh primerov, ki trdi, da iz $\phi$ sledi
$\psi$ in iz $\psi$ sledi $\phi$, kar pa je ekvivalenca:
%
\begin{equation*}
  \phi \liff \psi.
\end{equation*}

\begin{vaja}
  Je ">$n$ je sod in $n > 2$"< \textbf{potreben} ali \textbf{zadosten}
  pogoj za ">$n$ ni praštevilo"<?
\end{vaja}


\subsubsection{">Naslednje izjave so ekvivalentne: $\phi$, $\psi$, $\rho$ in $\sigma$."<}

To pomeni, da sta vsaki dve izmed danih izjav ekvivalentni, se pravi
%
\begin{equation*}
  (\phi \liff \psi) \land (\phi \liff \rho) \land (\phi \liff \sigma) \land (\psi \liff \rho)
  \land (\psi \liff \sigma) \land (\rho \liff \sigma).
\end{equation*}
%
Ker je ekvivalenca tranzitivna relacija, ni treba obravnavati vseh
kombinacij, zadostujejo že tri, ki dane izjave ">povežejo"< med seboj:
%
\begin{equation*}
  (\phi \liff \psi) \land (\psi \liff \rho) \land (\rho \liff \sigma).
\end{equation*}
%
To pišemo krajše kar kot
%
\begin{equation*}
  \phi \liff \psi \liff \rho \liff \sigma,
\end{equation*}
%
čeprav je formalno gledano tako zapis nepravilen. V
razdelku~\ref{sec:ekvivalenca} bomo spoznali, kako se tako zaporedje
ekvivalenc dokaže s ciklom implikacij $\phi \lthen \psi \lthen \rho
\lthen \sigma \lthen \phi$.

\begin{vaja}
  Podaj konkretne primere izjav $\phi$, $\psi$ in $\rho$, iz katerih
  je razvidno, da izjava $(\phi \liff \psi) \land (\psi \liff \rho)$
  \emph{ni} ekvivalentna niti $(\phi \liff \psi) \liff \rho$ niti
  $\phi \liff (\psi \liff \rho)$.
\end{vaja}



\subsubsection{">Za vsak $x$ iz $S$, za katerega velja $\phi$, velja tudi
  $\psi$."<}

To lahko preberemo tudi kot ">Za vsak $x$ iz $S$, če zanj velja $\phi$,
potem velja $\psi$,"< kar je v simbolni obliki
%
\begin{equation*}
  \all{x \in S} \phi \lthen \psi.
\end{equation*}
%
Tudi izjave oblike ">vsi $\phi$-ji so $\psi$-ji"< so te oblike, denimo ">vsa
od dva večja praštevila so liha"< zapišemo
%
\begin{equation*}
  \all{n \in \NN} n > 2 \land \text{$n$ je praštevilo} \lthen \text{$n$ je lih}.
\end{equation*}

\begin{vaja}
  V simbolni obliki zapiši ">$n$ je lih"< in ">$n$ je praštevilo"<.
  Namig: $n$ je lih, kadar obstaja naravno število $k$, za katerega
  velja $n = 2 k + 1$, in $n$ je praštevilo, kadar \emph{ni} zmnožek
  dveh naravnih števil, ki sta obe večji od~$1$.
\end{vaja}


\subsubsection{">Enačba $f(x) = g(x)$ nima realne rešitve."<}

To lahko povemo takole: ni res, da obstaja $x \in \RR$, za katerega bi
veljalo $f(x) = g(x)$. S simboli zapišemo
%
\begin{equation*}
  \lnot \some{x \in \RR} f(x) = g(x).
\end{equation*}
%
Opozoriti velja, da iz same enačbe ne moremo vedno sklepati, kaj je
neznanka. V enačbi $a x^2 + b x + c = 0$ bi za neznanko lahko načeloma
imeli katerokoli od štirih spremenljivk $a$, $b$, $c$ in $x$, ali pa
kar vse. Večina matematikov bi sicer uganila, da je najverjetneje
neznanka $x$, vendar se v splošnem ne moremo zanašati na običaje in
uganjevanje, ampak moramo točno povedati, kateri simboli so
\textbf{neznanke} in kateri \textbf{parametri}.

\begin{vaja}
  Zapiši v simbolni obliki: ">Sistem enačb
  %
  \begin{align*}
    a_1 x + b_1 y &= c_1,\\
    a_2 x + b_2 y &= c_2
  \end{align*}
  %
  nima pozitivnih realnih rešitev $x, y$."<
\end{vaja}

\begin{vaja}
  Zapiši v simbolni obliki:
  \begin{enumerate}
  \item ">Enačba $f(x) = g(x)$ ima največ eno realno rešitev."<
  \item ">Enačba $f(x) = g(x)$ ima več kot eno realno rešitev."<
  \item ">Enačba $f(x) = g(x)$ ima natanko dve realni rešitvi."<
  \end{enumerate}
\end{vaja}


\subsubsection{">Brez izgube za splošnost."<}

V matematičnih besedilih najdemo frazo ">brez izgube za splošnost"<
kot v naslednjem primeru.

\begin{izrek}
  \label{izrek:abc-vsota-razlik-soda}
  Za vsa cela števila $a$, $b$ in $c$ je $|a-b|+|b-c|+|c-a|$ sodo
  število.
\end{izrek}

\begin{proof}
  Brez izgube za splošnost smemo predpostaviti $a \geq b \geq c$.
  Tedaj velja
  %
  \begin{equation*}
    |a-b| + |b-c| + |c-a| = (a - b) + (b - c) - (c - a) = 2 (a - c),
  \end{equation*}
  %
  kar je sodo število.
\end{proof}

Fraza ">brez izgube za splošnost"< nakazuje, da dokaz obravnava le eno
od večih možnosti. Načeloma bi morali obravnavati tudi ostale
možnosti, ki pa jih je pisec dokaza opustil, ker so bodisi zelo lahke
bodisi zelo podobne tisti, ki jo dokaz obravnava. Za začetnika je
najtežje dognati, katere so preostale možnosti in zakaj se je pisec
dokaza pravzaprav odločil zanje. Avtor zgornjega dokaza je verjetno
opazil, da števila $a$, $b$ in $c$ v izrazu $|a-b|+|b-c|+|c-a|$
nastopajo \emph{simetrično}: če jih premešamo, se izraz ne spremeni.
Denimo, ko zamenjamo $a$ in $b$, dobimo $|b-a|+|a-c|+|c-b|$, kar je
enako prvotnemu izrazu $|a-b|+|b-c|+|c-a|$. Torej lahko izmed šestih
možnosti
%
\begin{xalignat*}{3}
  & a \geq b \geq c,&
  & a \geq c \geq b,&
  & b \geq a \geq c,\\
  & b \geq c \geq a,&
  & c \geq a \geq b,&
  & c \geq b \geq a
\end{xalignat*}
%
obravnavamo le eno. Seveda pisanje dokazov, pri katerih večji del
dokaza opustimo, zahteva pazljivost in nekaj izkušenj.

\begin{vaja}
  Dokaži izrek~\ref{izrek:abc-vsota-razlik-soda} tako, da obravnavaš
  samo možnost $b \geq c \geq a$ in zraven dopišeš ">brez izgube za
  splošnost"<.
\end{vaja}


%%%%%%%%%%%%%%%%%%%%%%%%%%%%%%%%%%%%%%%%%%%%%%%%%%%%%%%%%%%%%%%%%%%%%%
\section{Definicije}
\label{sec:definicije}


Poznamo tri vrste definicij. Prva in najpreprostejša je definicija, ki
služi kot \textbf{okrajšava} za daljši izraz. To smemo vedno nadomestiti
s prvotnim izrazom. Na primer, funkcija ">hiperbolični tangens"<
$\tanh(x)$ je definirana kot
%
\begin{equation*}
  \tanh(x) = \frac{e^{2 x} - 1}{e^{2 x} + 1}.
\end{equation*}
%
Lahko bi shajali tudi brez zapisa $\tanh(x)$, vendar bi morali potem
povsod pisati daljši izraz $\frac{e^{2 x} - 1}{e^{2 x} + 1}$, kar bi
bilo nepregledno.

Druga vrsta definicije je vpeljava novega matematičnega pojma.
Študenti prvega letnika matematike spoznajo celo vrsto novih pojmov
(grupa, vektorski prostor, limita, stekališče, metrika itn.), s
katerimi si razširijo sposobnost matematičnega razmišljanja.
Matematiki cenijo dobre definicije in vpeljavo novih matematičnih
pojmov vsaj toliko, kot dokaze težkih izrekov.

Tretja vrsta definicije je \textbf{konstrukcija} matematičnega objekta s
pomočjo dokaza o enoličnem obstoju. O tem bomo povedali več v
razdelku~\ref{sec:enolicni-obstoj}.

\section{Pravila sklepanja in dokazi}
\label{sec:pravila-sklepanja-in-dokazi}


Povedali smo že, da je dokaz utemeljitev neke matematične izjave. V
razdelku~\ref{sec:kaj-je-dokaz} smo govorili o tem, da so dokazi
mešanica besedila in simbolov, ki jih matematiki uporabljajo tako za
utemeljitev matematičnih izjav, kakor tudi za razlago in podajanje
matematičnih idej. V tem razdelku se posvetimo \textbf{formalnim
  dokazom}, ki so logične konstrukcije namenjene mehanskemu
preverjanju pravilnosti izjav.

Za vsako logično operacijo bomo podali \textbf{formalna pravila
  sklepanja}, ki jih smemo uporabljati v formalnem dokazu. Pravilo
sklepanja shematsko zapišemo
%
\begin{equation*}
  \inferrule{\phi \\ \psi \\ \rho}{\sigma}
\end{equation*}
%
in ga preberemo: ">Če smo dokazali $\phi$, $\psi$ in $\rho$, smemo
sklepati $\sigma$."< Izjavam nad črto pravimo \textbf{hipoteze}, izjavi
pod črto pa \textbf{sklep}. Hipotez je lahko nič ali več, sklep mora
biti natanko en. Pravilo sklepanja brez hipotez se imenuje
\textbf{aksiom}.

Da bomo lahko pojasnili, kaj je dokaz, podajmo pravila sklepanja za
$\top$ in $\land$, ki jih bomo v naslednjem razdelku še enkrat bolj
pozorno obravnavali:
%
\begin{mathpar}
  \inferrule{\quad}{\top}
  %
  \and
  %
  \inferrule
  {\phi \\ \psi}
  {\phi \land \psi}
  %
  \and
  %
  \inferrule
  {\phi \land \psi}
  {\phi}
  %
  \and
  %
  \inferrule
  {\phi \land \psi}
  {\psi}  
\end{mathpar}
%
Po vrsti beremo:
%
\begin{itemize}
\item Velja $\top$.
\item Če velja $\phi$ in $\psi$, smemo sklepati $\phi \land \psi$.
\item Če velja $\phi \land \psi$, smemo sklepati $\phi$.
\item Če velja $\phi \land \psi$, smemo sklepati $\psi$.
\end{itemize}
%
Formalni dokaz ima drevesno obliko in prikazuje, kako iz danih
\textbf{hipotez} dokažemo neko \textbf{sodbo}. Pri dnu je zapisana izjava,
ki jo dokazujemo, nad njo pa dokaz. Vsako vejišče je eno od pravil
sklepanja. Vsaka veja se mora zaključiti z aksiomom ali s hipotezo.
Oglejmo si dokaz izjave $(\alpha \land \alpha) \land (\top
\land \beta)$ iz hipoteze $\beta \land \alpha$:
%
\begin{equation*}
  \inferrule{
    \inferrule{
      \inferrule{\beta \land \alpha}{\alpha}
      \\
      \inferrule{\beta \land \alpha}{\alpha}}
      {\alpha \land \alpha}
    \\
    \inferrule{
      \inferrule{ }{\top}
      \\
      \inferrule{\beta \land \alpha}{\beta}
    }{\top \land \beta}
  }{(\alpha \land \alpha) \land (\top \land \beta)}
\end{equation*}
%
Dokaz se razveji na dve veji, vsaka od njiju pa še na dve veji. Tako
pri vrhu dobimo štiri liste, od katerih se trije izjava $\beta \land
\alpha$ in en aksiom za $\top$.

\begin{vaja}
  Preveri, da je vsako vejišče v zgornjem dokazu res uporaba enega od
  zgoraj podanih pravil sklepanja.
\end{vaja}

V praksi matematično besedilo bolj ali manj odraža strukturo
formalnega dokaza, le da se besedilo ne veji, ampak so sestavni kosi
dokaza zloženi v zaporedje. Formalni dokazi so uporabni, kadar želimo
preveriti veljavnost najbolj osnovnih logičnih dejstev. Ni mišljeno,
da bi matematiki pisali ali preverjali velike formalne dokaze
pomembnih matematičnih izrekov, to je delo za račualnike. Formalna
pravila sklepanja in formalni dokazi so za matematike pomembni, ker
nam omogočajo, da natančno in v celoti povemo, kakšna so ">pravila
igre"< v matematiki.


\section{Izjavni račun}
\label{sec:izjavni-racun}

Izjavni račun je tisti del logike, ki govori o logičnih konstantah
$\bot$, $\top$ in o logičnih operacijah $\land$, $\lor$, $\lthen$,
$\liff$, $\lnot$. Za vsako od njih podamo formalna pravila sklepanja,
ki so dveh vrst. Pravila \textbf{vpeljave} povedo, kako se izjave
dokaže, pravila \textbf{uporabe} pa povedo, kako se že dokazane izjave uporabi.

\subsection{Konjunkcija}
\label{sec:konjunkcija}

Konjunkcija ima eno pravilo vpeljave in dve pravili uporabe:
%
\begin{mathpar}
  \inferrule
  {\phi \\ \psi}
  {\phi \land \psi}
  \and
  \inferrule
  {\phi \land \psi}
  {\phi}  
  %
  \and
  %
  \inferrule
  {\phi \land \psi}
  {\psi}
\end{mathpar}
%
Pravilo vpeljave pove, da konjunkcijo $\phi \land \psi$ dokažemo
tako, da dokažemo posebej $\phi$ in posebej $\psi$. Pravili uporabe pa
povesta, da lahko $\phi \land \psi$ ">razstavimo"< na izjavi~$\phi$
in~$\psi$.

V matematičnem besedilu je dokaz konjunkcije $\phi \land \psi$ zapisan
kot zaporedje dveh pod-dokazov:
%
\begin{quote}
  \it 
  %
  Dokazujemo $\phi \land \psi$:
  \begin{enumerate}
  \item (Dokaz $\phi$)
  \item (Dokaz $\psi$)
  \end{enumerate}
  Dokazali smo $\phi \land \psi$.
\end{quote}
%
Manj podroben dokaz ne vsebuje začetnega in končnega stavka, ampak
samo dokaza za $\phi$ in $\psi$. Bralec mora sam ugotoviti, da je s
tem dokazana izjava $\phi \land \psi$.

\subsection{Implikacija}
\label{sec:implikacija}

Preden zapišemo pravila sklepanja za implikacijo, si oglejmo primer
neformalnega dokaza.

\begin{izrek}
  Če je $x > 2$, potem je $x^3 + x + 7 > 16$.
\end{izrek}

\begin{proof}
  Predpostavimo, da velja $x > 2$. Tedaj je $x^3 > 2^3 = 8$, zato
  velja
  %
  \begin{equation*}
    x^3 + x + 7 > 8 + 2 + 7 = 17 > 16.
  \end{equation*}
  %
  Dokazali smo $x > 2 \lthen x^3 + x + 7 > 16$.
\end{proof}

\noindent
%
Prvi stavek dokaza z besedico ">predpostavimo"< uvede \textbf{začasno
  hipotezo} $x > 2$, iz katere nato izpeljemo posledico $x^3 + x + 7 >
16$. Implikacijo $\phi \lthen \psi$ torej dokažemo tako, da začasno
predpostavimo $\phi$ in dokažemo $\psi$. Tako pravilo vpeljave
zapišemo
%
\begin{equation*}
  \inferrule{\infer*{\psi}{[\phi]}}{\phi \lthen \psi}  
\end{equation*}
%
Zapis $[\phi]$ z oglatimi oklepaji pomeni, da $\phi$ ni prava, ampak
samo začasna hipoteza. Zapis
%
\begin{equation*}
  \infer*{\psi}{[\phi]}
\end{equation*}
%
pomeni ">dokaz izjave $\phi$ s pomočjo začasne hipoteze $\phi$."<

Pravilo uporabe za implikacijo se imenuje \textbf{modus ponens} in se
glasi
%
\begin{mathpar}
  \inferrule{\phi \lthen \psi \\ \phi}{\psi}
\end{mathpar}
%
V matematičnem besedilu se modus ponens pojavi kot uporaba že prej
dokazanega izreka izreka oblike $\phi \lthen \psi$.

\subsection{Disjunkcija}
\label{sec:disjunkcija}

Disjunkcija ima dve pravili vpeljave in eno pravilo uporabe:
%
\begin{mathpar}
  \inferrule
  {\phi}
  {\phi \lor \psi}
  \and
  \inferrule
  {\psi}
  {\phi \lor \psi}
  \and
  \inferrule
  {\phi \lor \psi \\ \infer*{\rho}{[\phi]} \\ \infer*{\rho}{[\psi]}}
  {\rho}
\end{mathpar}
%
Pravili sklepanja povesta, da lahko dokažemo disjunkcijo $\phi \lor
\psi$ tako, da dokažemo enega od disjunktov.

Pojasnimo še pravilo uporabe. Denimo, da bi radi dokazali $\rho$, pri
čemer že vemo, da velja $\phi \lor \psi$. Pravilo uporabe pravi, da je
treba obravnavati dva primera: iz začasne hipoteze $\phi$ je treba
dokazati $\rho$ in iz začasne hipoteze $\psi$ je treba dokazati
$\rho$.

Ponazorimo pravilo uporabe v dokazu izjave $(\alpha \lor \gamma) \land
(\beta \lor \gamma)$ iz hipoteze $(\alpha \land \beta) \lor \gamma$.
Dokazno drevo je precej veliko, v njem pa se dvakrat pojavi uporaba
disjunkcije:
%
\begin{equation*}
  \inferrule
  {\inferrule*
    {\inferrule*{}{(\alpha \land \beta) \lor \gamma}
      \\
      \inferrule*{
        \inferrule
        {[\alpha \land \beta]}
        {\alpha}
      }{\alpha \lor \gamma}
      \\
      \inferrule*{[\gamma]}{\alpha \lor \gamma}
    }
    {\alpha \lor \gamma}
    \\
    \inferrule*
    {\inferrule*{}{(\alpha \land \beta) \lor \gamma}
      \\
      \inferrule*{
        \inferrule
        {[\alpha \land \beta]}
        {\beta}
      }{\beta \lor \gamma}
      \\
      \inferrule*{[\gamma]}{\beta \lor \gamma}
    }
    {\beta \lor \gamma}
  }
  {(\alpha \lor \gamma) \land (\beta \lor \gamma)}
\end{equation*}
%
Poglejmo na primer levo vejo tega dokaza, desna je podobna:
%
\begin{equation*}
  \inferrule*
    {\inferrule*{}{(\alpha \land \beta) \lor \gamma}
      \\
      \inferrule*{
        \inferrule
        {[\alpha \land \beta]}
        {\alpha}
      }{\alpha \lor \gamma}
      \\
      \inferrule*{[\gamma]}{\alpha \lor \gamma}
    }
    {\alpha \lor \gamma}
\end{equation*}
%
Res je to uporaba disjunkcije $\phi \lor \psi$, kjer smo vzeli $\phi =
\alpha \land \beta$ in $\psi = \gamma$, dokazali pa smo izjavo $\rho =
\alpha \lor \gamma$.

\begin{vaja}
  Iz hipoteze $(\alpha \lor \gamma) \land (\beta \lor \gamma)$ dokaži
  $(\alpha \land \beta) \lor \gamma$.
\end{vaja}

V besedilu dokažemo disjunkcijo s pravilom za vpeljavo takole:
%
\begin{quote}
  \it
  %
  Dokazujemo $\phi \lor \psi$. Zadostuje dokazati $\phi$:
  \begin{enumerate}
  \item[] (Dokaz $\phi$.)
  \end{enumerate}
  %
  Dokazali smo $\phi \lor \psi$.
\end{quote}
%
Pravilo uporabe disjunkcije se v besedilu zapiše kot obravnava
primerov:
%
\begin{quote}
  \it
  %
  Dokazujemo $\rho$. To bomo dokazali z obravnavo primerov $\phi$ in
  $\psi$:
  \begin{enumerate}
  \item (Dokaz $\phi \lor \rho$)
  \item Predpostavimo, da velja $\phi$. (Dokaz $\rho$)
  \item Predpostavimo, da velja $\psi$. (Dokaz $\rho$)
  \end{enumerate}
  %
  Dokazali smo $\rho$.
\end{quote}
%
Še primer konkretnega dokaza, ki je tako napisan.

\begin{izrek}
  \label{izrek:x-3-5}
  Naj bo $x$ realno število. Če je $|x - 3| > 5$, potem je $x^4 > 15$.
\end{izrek}

\begin{proof}
  Dokazujemo $|x - 3| > 5 \lthen x^4 > 15$. Predostavimo $|x - 3| > 5$
  in dokažimo $x^4 > 15$. To bomo dokazali z obravavo primerov $x \leq
  3$ in $x \geq 3$:
  %
  \begin{enumerate}
  \item $x \leq 3 \lor x \geq 3$ velja, ker so realna števila linearno
    urejena z relacijo $\leq$.
  \item Predpostavimo $x \leq 3$. Tedaj je $x - 3 \leq 0$ in zato $|x
    - 3| = 3 - x$, od koder sledi $3 - x = |x - 3| > 5$, oziroma $x <
    -2$. Tako dobimo
    %
    \begin{equation*}
      x^4 > (-2)^4 = 16 > 15.
    \end{equation*}
  \item Predpostavimo $x \geq 3$. Tedaj je $x - 3 \geq 0$ in zato$|x -
    3| = x - 3$, od koder sledi $x - 3 = |x - 3| > 5$, oziroma $x >
    8$. Tako dobimo
    %
    \begin{equation*}
      x^4 > 8^4 = 4096 > 15.
    \end{equation*}
  \end{enumerate}
  %
  Iz predpostavke $|x - 3| > 5$ smo izpeljali $x^4 > 15$. S tem smo
  dokazali $|x - 3| > 5 \lthen x^4 > 15$.
\end{proof}

Težji del tega dokaza se skriva v izbiri disjunkcije. Kako je pisec
uganil, da je treba obravnavati primera $x \leq 3$ in $x \geq 3$?
Zakaj ni raje obravnaval $x < 3$ in $x \geq 3$, ali morda $x \leq 17$
in $x \geq 17$? Odgovor se skriva v definiciji absolutne vrednosti:
%
\begin{equation*}
  |a| =
  \begin{cases}
    a & \text{če je $a \geq 0$,}\\
    -a & \text{če je $a \leq 0$.}
  \end{cases}
\end{equation*}
%
Ker v izreku nastopa izraz $|x - 3|$, bo obravnava primerov $x - 3
\geq 0$ in $x - 3 \leq 0$ omogočila, da $|x - 3|$ poenostavimo enkrat
v $x - 3$ in drugič v $3 - x$. Seveda pa je $x - 3 \geq 0$
ekvivalentno $x \geq 3$ in $x - 3 \leq 0$ ekvivalentno $x \leq 3$.

\begin{vaja}
  Ali bi lahko izrek~\ref{izrek:x-3-5} dokazali tudi z obravnavo
  primerov $x < 3$ in $x \geq 3$?
\end{vaja}

\subsection{Resnica in neresnica}
\label{sec:resnica-neresnica}

Logična konstanta $\top$ označuje resnico. Kar je res, je res, in tega
ni treba posebej dokazovati. To dejstvo izraža aksiom
%
\begin{equation*}
  \inferrule{\qquad}{\top}
\end{equation*}
%
Logična konstanta $\top$ nima pravila uporabe, ker iz $\top$ ne moremo
sklepati nič koristnega.

Logična konstanta $\bot$ označuje neresnico. Ker se tega, kar ni res,
ne more dokazati, $\bot$ nima pravila vpeljave. Pravilo uporabe je
%
\begin{equation*}
  \inferrule{\quad\bot\quad}{\phi}
\end{equation*}
%
se imenuje \textbf{ex falso (sequitur) quodlibet}, kar pomeni ">iz
neresnice sledi karkoli"<.

V matematičnem besedilu se $\top$ in $\bot$ ne pojavljata pogosto, ker
matematiki izraze, v katerih se $\top$ in $\bot$ pojavita, vedno
poenostavijo s pomočjo ekvivalenc:
%
\begin{mathpar}
  \top \land \phi \liff \phi
  \and
  \top \lor \phi \liff \phi
  \and
  \bot \land \phi \liff \bot
  \and
  \bot \lor \phi \liff \phi
  \\
  (\top \lthen \phi) \liff \phi
  \and
  (\bot \lthen \phi) \liff \top
  \and
  (\phi \lthen \top) \liff \top
\end{mathpar}
%

\subsection{Ekvivalenca}
\label{sec:ekvivalenca}

Logična ekvivalenca $\phi \liff \psi$ je okrajšava za
%
\begin{equation*}
  (\phi \lthen \psi) \land (\psi \lthen \phi).
\end{equation*}
%
Ker je to konjunkcija (dveh implikacij), so pravila za vpeljavo in
uporabo ekvivalence samo poseben primer pravil sklepanja za
konjunkcijo:
%
\begin{mathpar}
  \inferrule
  {\phi \lthen \psi \\ \psi \lthen \phi}
  {\phi \liff \psi}
  \and
  \inferrule{\phi \liff \psi}{\phi \lthen \psi}
  \and
  \inferrule{\phi \liff \psi}{\psi \lthen \phi}
\end{mathpar}
%
V matematičnem besedilu ekvivalenco dokažemo takole:
%
\begin{quote}
  \it
  %
  Dokazujemo $\phi \liff \psi$:
  %
  \begin{enumerate}
  \item (Dokaz $\phi \lthen \psi$)
  \item (Dokaz $\psi \lthen \phi$)
  \end{enumerate}
  Dokazali smo $\phi \liff \psi$.
\end{quote}

Če sta izjavi $\phi$ in $\psi$ logično ekvivalentni, lahko eno
zamenjamo z drugo. To matematiki s pridom uporabljajo pri dokazovanju
izjav, čeprav pogosto sploh ne omenijo, katero ekvivalenco so
uporabili.

Kadar dokazujemo medsebojno ekvivalenco večih izjav $\phi_1$,
$\phi_2$, \ldots, $\phi_n$, zadostuje dokazati cikel implikacij
%
\begin{equation*}
  \phi_1 \lthen \phi_2 \lthen \cdots \lthen \phi_{n-1} \lthen \phi_n \lthen \phi_1.
\end{equation*}
%
(Ne spreglejte zadnje implikacije $\phi_n \lthen \phi_1$, ki zaključi
cikel). V besedilu to dokažemo:

\begin{quote}
  \it
  %
  Dokazujemo, da so izjave $\phi_1, \phi_2, \ldots, \phi_n$
  ekvivalentne:
  %
  \begin{enumerate}
  \item (Dokaz $\phi_1 \lthen \phi_2$)
  \item (Dokaz $\phi_2 \lthen \phi_3$)
  \item \dots
  \item (Dokaz $\phi_{n-1} \lthen \phi_n$)
  \item (Dokaz $\phi_n \lthen \phi_1$)
  \end{enumerate}
\end{quote}

\noindent
%
Seveda smemo pred samim dokazovanjem izjave $\phi_1, \ldots, \phi_n$
preurediti tako, da je zahtevane implikacije kar najlažje dokazati.
Dokaz lahko tudi razdelimo na dva ločena cikla implikacij
%
\begin{equation*}
  \phi_1 \lthen \cdots \lthen \phi_k \lthen \phi_1
\end{equation*}
%
in
%
\begin{equation*}
  \phi_{k+1} \lthen \cdots \lthen \phi_n \lthen \phi_{k+1}
\end{equation*}
%
in nato dokažemo še eno ekvivalenco $\phi_i \liff \phi_j$, pri čemer
je $\phi_i$ iz prvega in $\phi_j$ iz drugega cikla.

\subsection{Negacija}
\label{sec:negacija}


Negacija $\lnot\phi$ je definirana kot okrajšava za $\phi
\lthen \bot$. Iz pravil sklepanja za $\lthen$ in $\bot$ tako izpeljemo
pravili sklepanja za negacijo:
%
\begin{mathpar}
  \inferrule
  {\infer*{\bot}{[\phi]}}
  {\lnot \phi}
  %
  \and
  %
  \inferrule
  {\lnot\phi \\ \phi}
  {\psi}
\end{mathpar}
%
V besedilu dokazujemo $\lnot\phi$ takole:
%
\begin{quote}
  \it
  %
  Dokazujemo $\lnot\phi$.
  \begin{itemize}
  \item[] Predpostavimo $\phi$.
  \item[] (Dokaz $\bot$.)
  \end{itemize}
  Dokazali smo $\lnot\phi$.
\end{quote}
%
Tu ">Dokaz $\bot$"< pomeni, da iz danih predpostavk izpeljemo
protislovje. Mnogi matematiki menijo, da se takemu dokazu reče ">dokaz
s protislovjem"<, vendar to ni res. To je samo navaden dokaz negacije.
Dokazovanje s protislovjem bomo obravnavali v razdelku~\ref{sec:lem}.

Pravilo uporabe za $\lnot\phi$ v besedilu ni eksplicitno vidno, ampak
ga matematiki uporabijo, ko sredi dokaza, da velja $\psi$, izpeljejo
protislovje:
%
\begin{quote}
  \it
  %
  Dokazujemo $\psi$.
  %
  \begin{itemize}
  \item[] (Dokaz $\phi$.)
  \item[] (Dokaz $\lnot\phi$.)
  \end{itemize}
  %
  To je nesmisel, in ker iz nesmisla sledi karkoli, sledi $\psi$.
\end{quote}

\subsection{Aksiom o izključenem tretjem}
\label{sec:lem}

Aksiom o izključenem tretjem se glasi
%
\begin{equation*}
  \inferrule{ }{\phi \lor \lnot \phi}
\end{equation*}
%
Povedano z besedami, vsaka izjava je bodisi resnična bodisi
neresnična. Torej ni ">tretje možnosti"< za resničnostno vrednost
izjave $\phi$, od koder izhaja tudi ime aksioma.

Aksiom o izključenem tretjem omogoča \emph{posredne} dokaze izjav, od
katerih je najbolj znano \textbf{dokazovanje s protislovjem}: pri tem ne
utemeljimo izjave $\phi$, ampak utemeljimo, zakaj $\lnot\phi$
\emph{ne} velja. Natančneje povedano, izjavo $\phi$ zamenjamo z njej
ekvivalentno izjavo $\lnot\lnot\phi$ in dokažemo $\lnot\lnot\phi$.
Dokaz ekvivalence $\phi \liff \lnot\lnot\phi$ sestoji iz dokazov dveh
implikacij:
%
\begin{mathpar}
  \inferrule
  {\inferrule{
      \inferrule{[\lnot\phi] \\ [\phi]}{\bot}
    }
    {\lnot\lnot\phi}
  }
  {\phi \lthen \lnot\lnot\phi}
  %
  \and
  %
  \inferrule*
  {\inferrule*
    {\inferrule*{ }{\phi \lor \lnot\phi} \\
     [\phi] \\
     \inferrule*{
       \inferrule*{
         [\lnot\lnot\phi] \\ [\lnot\phi]
       }
       {\bot}
     }
     {\phi}
    }
    {\phi}
  }
  {\lnot\lnot\phi \lthen \phi}
\end{mathpar}
%
V dokazu $\lnot\lnot\phi \lthen \phi$ smo uporabili aksiom o
izključenem tretjem. V matematičnem besedilu se dokaz s protislovjem
glasi:
%
\begin{quote}
  \it
  %
  Dokažimo $\phi$ s protislovjem.
  %
  \begin{itemize}
  \item[] Predpostavimo, da bi veljalo $\lnot\phi$.
  \item[] (Dokaz neresnice $\bot$.)
  \end{itemize}
  %
  Ker torej $\lnot\phi$ pripelje do protislovja, velja $\phi$.
\end{quote}
%
Praviloma izvemo o vsebini matematične izjave~$\phi$ več, če jo
dokažemo neposredno. Dokazovanja s protislovjem zato ni smiselno
uporabljati vsepovprek, ampak le takrat, ko je zares potreben ali ko
nam zelo olajša dokazovanje.

Ostali načini za sestavljanje posrednih dokazov slonijo na
ekvivalencah
%
\begin{mathpar}
  (\phi \lor \psi) \liff \lnot (\lnot\phi \land \lnot\psi),\and
  (\phi \lor \psi) \liff (\lnot\phi \lthen \psi),\and
  (\phi \lthen \psi) \liff (\lnot\psi \lthen \lnot\phi),\and
  (\all{x \in S} \phi) \liff \lnot \some{x \in S} \lnot \phi,\and
  (\some{x \in S} \phi) \liff \lnot \all{x \in S} \lnot \phi.
\end{mathpar}
%
V vseh petih primerih implikacija $\lthen$ iz leve na desno velja brez
uporabe aksioma o izključenem tretjem. Za dokaz implikacij
$\Leftarrow$ iz desne na levo pa potrebujemo aksiom o izključenem
tretjem.

\begin{vaja}
  Sestavi formalne dokaze za zgornjih pet ekvivalenc. Pri dokazovanju
  ekvivalenc za $\forall$ in $\exists$ si pomagaj s pravili sklepanja
  iz razdelkov~\ref{sec:univerzalni-kvantifikator}
  in~\ref{sec:eksistencni-kvantifikator}.
\end{vaja}

Povejmo, kako zgornje ekvivalence uporabimo v besedilu za posredni
dokaz izjave:
%
\begin{itemize}
\item $(\phi \lor \psi) \liff \lnot (\lnot\phi \land \lnot\psi)$
  uporabimo takole:
  %
  \begin{quote}
    \it
    %
    Dokazujemo $\phi \lor \psi$.
    %
    \begin{itemize}
    \item[] Predpostavimo, da velja $\lnot\phi$ in $\lnot\psi$.
    \item[] (Dokaz neresnice $\bot$.)
    \end{itemize}
    %
    Ker torej nista $\phi$ in $\psi$ oba neresnična, je eden od njiju
    resničen. Dokazali smo $\phi \lor \psi$.
  \end{quote}
\item $(\phi \lor \psi) \liff (\lnot\phi \lthen \psi)$ uporabimo
  takole:
  %
  \begin{quote}
    \it
    %
    Dokazujemo $\phi \lor \psi$.
    %
    \begin{itemize}
    \item[] Predpostavimo $\lnot\phi$.
    \item[] (Dokaz $\psi$.)
    \end{itemize}
    %
    Če torej ne velja $\lnot\phi$, velja $\psi$. Torej velja vsaj
    eden, zato smo dokazali $\phi \lor \psi$.
  \end{quote}
\item $(\phi \lthen \psi) \liff (\lnot\psi \lthen \lnot\phi)$
  uporabimo takole:
  %
  \begin{quote}
    \it
    %
    Dokazujemo $\phi \lthen \psi$.
    %
    \begin{enumerate}
    \item Predpostavimo $\lnot\psi$.
    \item (Dokaz $\lnot\psi$.)
    \end{enumerate}
    %
    Dokazali smo, da iz $\phi$ sledi $\psi$.
  \end{quote}
\item $(\all{x \in S} \phi) \liff \lnot \some{x \in S} \lnot \phi$
  uporabimo takole:
  %
  \begin{quote}
    \it
    %
    Dokazujemo, da za vsak $x \in S$ velja $\phi$.
    %
    \begin{enumerate}
    \item Predpostavimo, da obstaja $x \in S$, za katerega $\phi$
      \emph{ne} velja.
    \item (Dokaz neresnice $\bot$.)
    \end{enumerate}
    %
    Predpostavka, da obstaja $x \in S$, za katerega $\phi$ ne velja,
    pripelje do protislovja. Torej za vsak $x \in S$ velja $\phi$.
  \end{quote}
\item $(\some{x \in S} \phi) \liff \lnot \all{x \in S} \lnot \phi$
  uporabimo takole:
  %
  \begin{quote}
    \it
    %
    Dokazujemo, da obstaja tak $x \in S$, za katerega velja $\phi$.
    %
    \begin{enumerate}
    \item Predpostavimo, da bi veljalo $\lnot\phi$ za vse $x \in S$.
    \item (Dokaz neresnice $\bot$.)
    \end{enumerate}
    %
    Predpostavka, da velja $\lnot\phi$ za vse $x \in S$, pripelje do
    protislovja. Torej obstaja $x \in S$, za katerega velja $\phi$.
  \end{quote}
\end{itemize}

Negacijo poljubne izjave $\phi$ tvorimo preprosto tako, da pred njo
postavimo $\lnot$. Vendar nam to ne pove dosti o matematični vsebini
negirane izjave. V večini primerov je negacijo lažje razumeti, če
simbol~$\lnot$ ">porinemo"< navznoter do osnovnih izjav z uporabo
naslednjih ekvivalenc:
%
\begin{align*}
  \lnot (\phi \land \psi) &\iff \lnot\phi \lor \lnot\psi \\
  \lnot (\phi \lor \psi) &\iff \lnot\phi \land \lnot\psi \\
  \lnot (\phi \lthen \psi) &\iff \phi \land \lnot\psi \\
  \lnot (\lnot \phi) &\iff \phi \\
  \lnot (\all{x \in S} \phi) &\iff \some{x \in S} \lnot\phi \\
  \lnot (\some{x \in S} \phi) &\iff \all{x \in S} \lnot\phi
\end{align*}

\begin{zgled}
  Denimo, da bi radi ovrgli izjavo
  % 
  \begin{quote}
    ">Vsako zaporedje pozitivnih realnih števil ima limito~$0$."<
  \end{quote}
  % 
  Da izjavo ovržemo, moramo dokazati njeno negacijo. Načeloma lahko
  negacijo tvorimo tako, da pred izjavo napišemo ">ni res, da velja
  \dots"<, a nam to ne pove, kako bi negacijo dokazali. Zapišimo
  prvotno izjavo v delni simbolni obliki:
  % 
  \begin{equation}
    \label{eq:pozitivno-limita-0}
    \all{a \in \RR^\NN}{\text{$(a_n)_n$ pozitivno zaporedje}
      \lthen \text{$0$ je limita zaporedja $(a_n)_n$}}.
  \end{equation}
  % 
  Zgornja pravila za računanje negacije nam povedo, da se
  $\lnot\forall$ spremeni v $\exists\lnot$ in da se nato implikacija
  oblike $\phi \lthen \psi$ spremeni v $\phi \land \lnot\psi$. Tako
  izrazimo negacijo izjave~\eqref{eq:pozitivno-limita-0}:
  % 
  \begin{equation*}
    \some{a \in \RR^\NN}{\text{$(a_n)_n$ pozitivno zaporedje}
      \land \lnot (\text{$0$ je limita zaporedja $(a_n)_n$})}.
  \end{equation*}
  % 
  To preberemo z besedami:
  % 
  \begin{quote}
    ">Obstaja tako zaporedje $(a_n)_n$, da je $(a_n)_n$ zaporedje
    pozitivnih števil in da $0$ ni limita zaporedja $(a_n)_n$."<
  \end{quote}
  %
  Če se še malo potrudimo, preberemo bolj razumljivo:
  % 
  \begin{quote}
    ">Obstaja tako zaporedje pozitivnih realnih števil, da $0$ ni
    njegova limita."<
  \end{quote}
  %
  S tem še nismo končali, saj je tudi ">Število $0$ ni limita
  zaporedja $(a_n)_n$"< negacija. Izjavo ">$0$ je limita zaporedja
  $(a_n)_n$"< najprej zapišemo simbolno:
  % 
  \begin{equation}
    \label{eq:limita-0}
    \all{\epsilon > 0}
      \some{m}{\NN}
        \all{n \geq m}
          |a_n - 0| < \epsilon.
  \end{equation}
  % 
  Z zgornjimi pravili za negiranje izračunamo negacijo
  izjave~\eqref{eq:limita-0}. Operacijo $\lnot$ postopoma ">porivamo"<
  navznoter:
  % 
  % 
  \begin{align*}
    \lnot \all{\epsilon > 0} \some{m \in \NN} \all{n \geq m} |a_n
          - 0| < \epsilon & \iff
    \\
    \some{\epsilon > 0} \lnot \some{m}{\NN} \all{n \geq m}
          |a_n - 0| < \epsilon &\iff
    \\
    \some{\epsilon > 0} \all{m}{\NN} \lnot \all{n \geq m}
          |a_n - 0| < \epsilon &\iff
    \\
    \some{\epsilon > 0} \all{m}{\NN} \some{n \geq m}
          \lnot (|a_n - 0| < \epsilon) &\iff
    \\
    \some{\epsilon > 0} \all{m}{\NN} \some{n \geq m}
          |a_n - 0| \geq \epsilon &\iff
    \\
    \some{\epsilon > 0} \all{m}{\NN} \some{n \geq m}
          a_n \geq \epsilon.
  \end{align*}
  % 
  V zadnjem koraku smo upoštevali, da za pozitivno število $a_n$ velja
  $|a_n - 0| = |a_n| = a_n$. Tako smo dobili podrobno zapisano
  negacijo prvotne izjave
  % 
  \begin{quote}
    ">Obstaja tako zaporedje pozitivnih števil $(a_n)_n$ in obstaja
    tak $\epsilon > 0$, da za vsak $m \in \NN$ obstaja $n \geq m$, za
    katerega velja $a_n > \epsilon$."<
  \end{quote}
  % 
  To izjavo pa znamo dokazati tako, da podamo konkreten primer
  zaporedja $(a_n)_n$ in konkretno vrednost $\epsilon$, ki zadoščata
  pogoju, denimo $a_n = 2 + n$ in $\epsilon = 1$. Res, če je $m \in
  \NN$ poljuben, lahko vzamemo kar $n = m$, saj potem velja $a_n = a_m
  = 2 + m > 1 = \epsilon$.

  Pričujoči primer smo zapisali zelo podrobno. Izkušeni matematik tega
  seveda ne bo pisal, saj bo izračunal negacijo prvotne izjave kar v
  glavi in takoj podal primer zaporedja, ki dokazuje, da prvotna
  izjava ne velja.
\end{zgled}

%%%%%%%%%%%%%%%%%%%%%%%%%%%%%%%%%%%%%%%%%%%%%%%%%%%%%%%%%%%%%%%%%%%%%%
\section{Predikatni račun}
\label{sec:predikatni-racun}

Predikatni račun je tisti del logike, ki obravnava predikate ter
kvantifikatorja~$\forall$ in~$\exists$.

Predikate tvorimo z logičnimi operacijami in kvantifikatorji iz
\textbf{osnovnih predikatov}. Katere osnovne predikate imamo na voljo,
je odvisno od snovi, ki jo obravnavamo.\footnote{Na primer, če
  obravnavamo ravninsko geometrijo, potem so osnovni predikati ">točka
  $x$ leži na premici $y$"<, ">premici $p$ in $q$ se sekata"< itn.}
Vedno imamo na voljo tudi \textbf{enakost} $x = y$, ki jo bomo
obravnavali v razdelku~\ref{sec:enakost}.

V osnovnih predikatih nastopajo \textbf{izrazi} ali \textbf{termi}. Katere
izraze lahko tvorimo je spet odvisno od tega, katere konstante in
operacije imamo na voljo. Na primer, če obravnavamo aritmetiko celih
števil, so na voljo operacije $+$, $-$, $\times$, če pa obravnavamo
realna števila, so na voljo operacije $+$, $-$, $\times$, $/$. V
izrazih vedno lahko nastopajo \textbf{spremenljivke}. Kadar uporabimo
spremenljivko, moramo povedati njen \textbf{tip} oziroma \textbf{množico}
vrednosti, ki jih lahko zavzame spremenljivka. Pogosto je tip
spremenljivke razviden iz spremnega besedila ali iz ustaljene uporabe:
$n$ se uporablja za naravno število, $x$ za realno, $f$ za funkcijo
ipd.

Ponazorimo pravkar definirane pojme s primerom. Predikat
%
\begin{equation*}
  0 < f(x) \land f(x) < \pi/4 \lthen \sin(2 f(x)) = 1/3
\end{equation*}
%
je sestavljen s pomočjo logičnih operacij $\land$ in $\lthen$ iz treh
osnovnih predikatov, zgrajenih iz osnovnih relacij $<$ in $=$,
%
\begin{mathpar}
  0 < f(x)
  \and
  f(x) < \pi/4
  \and
  \sin(2 f(x)) = 1/3,
\end{mathpar}
%
v katerih nastopa pet izrazov:
%
\begin{mathpar}
  0
  \and
  f(x)
  \and
  \pi/4
  \and
  \sin(2 f(x))
  \and
  1/3
\end{mathpar}
%
V teh izrazih nastopa spremenljivka $x$, katere tip je množica
realnih števil (to moramo uganiti) in spremenljivka $f$, ki označuje
funkcijo iz realnih v realna števila (tudi to moramo uganiti).
Nadalje, v izrazih nastopajo konstante $0$, $\pi$, $4$, $2$,
$1$ in $3$, operacija $\sin$ in operacija množenja.


%%%%%%%%%%%%%%%%%%%%%%%%%%%%%%%%%%%%%%%%%%%%%%%%%%
\subsection{Proste in vezane spremenljivke}
\label{sec:spremenljivke}

V predikatih in izrazih se pojavljajo spremenljivke. Pri tem moramo
ločiti med \textbf{prostimi} in \textbf{vezanimi} spremenljivkami. Oglejmo
si naslednja izraza in predikat:
%
\begin{equation*}
  \sum_{i=0}^{n} a_i,
  \qquad
  \int_0^1 f(t) \, d t,
  \qquad
  \forall x \in A .\, \phi(x) \;.
\end{equation*}
%
V vsoti je vezana spremenljivka $i$, spremenljivki $n$ in $a$ sta
prosti. To pomeni, da je $i$ neke vrste ">lokalna
spremenljivka"<,\footnote{Podobnost z lokalnimi spremenljivkami v
  programskih jezikih ni zgolj naključje. Lokalna spremenljivka in
  števec v zanki sta tudi primera vezanih spremenljivk v teoriji
  programskih jezikov.} katere veljavnost je samo znotraj vsote,
medtem ko sta spremenljiki $n$ in $a$ veljavni tudi zunaj samega
izraza. Podobno je v integralu $t$ vezana spremenljivka in $f$ prosta,
v izjavi na desni pa je vezana spremenljivka $x$, spremenljivki $A$ in
$\phi$ sta prosti.

Vezane spremenljivke so ">nevidne"< zunaj izraza in jih lahko vedno
preimenujemo, ne da bi spremenili pomen izraza (seveda se novo ime ne
sme mešati z ostalimi spremenljivkami, ki nastopajo v izrazu): izraza
$\int_0^1 f(t)\, d t$ in $\int_0^1 f(x)\, d x$ štejemo za
\emph{enaka}, ker se razlikujeta le v imenu vezane spremenljivke.
Spremenljivki, ki ni vezana, pravimo \textbf{prosta}. Izrazu, v katerem
ni prostih spremenljivk, pravimo \textbf{zaprt izraz}. Zaprta
logična izjava se imenuje \textbf{stavek}.

Pomembno se je zavedati, da vezana spremenljivka ">zunaj"< svojega
območja ne obstaja. Matematiki so glede tega precej površni in na
primer pišejo
%
\begin{equation*}
  \int x^2 \, d x = x^3/3 + C,
\end{equation*}
%
kar je strogo gledano nesmisel. Na levi strani v integralu stoji
vezana spremenljivka~$x$, ki je na desni ">pobegnila"< iz integrala.
Še več, če je $x \in \RR$ in $C \in \RR$, potem je izraz $x^3/3 + C$
\emph{število} (odvisno od vrednosti $x$ in $C$), saj je vsota dveh
realnih števil. Na desni strani bi morala stati oznaka za
\emph{funkcijo}, recimo
%
\begin{equation*}
  \int x^2 \, d x = (x \mapsto x^3/3 + C),
\end{equation*}
%
vendar tega v praksi nihče ne piše. Seveda pri vsem tem ostane še
vprašanje, kakšno vlogo ima v zgornjem izrazu~$C$. Pri analizi se
učimo, da je~$C$ ">poljubna konstanta"<. Poskusimo to razumeti
natančno s stališča logike. Besedico ">poljubno"< ponavadi razumemo
kot ">za vsak"<, vendar to ne gre, saj je
%
\begin{equation*}
  \all{C \in \RR} \int x^2 \, d x = (x \mapsto x^3/3 + C)
\end{equation*}
%
nesemisel. Če bi to bilo res, bi veljalo za $C = 1$ in za $C = 2$, od
koder bi dobili
%
\begin{equation*}
  (x \mapsto x^3/3 + 1) =
  \int x^2 \, d x =
  (x \mapsto x^3/3 + 2).
\end{equation*}
%
Potemtakem bi morali biti funkciji $(x \mapsto x^3/3 + 1)$ in $(x
\mapsto x^3/3 + 1)$ enaki, od koder sledi nesmisel $1 = 2$. Težave
nastopajo iz dejstva, da poskušamo nedoločeni integral razumeti kot
operacijo, ki slika funkcije v funkcije, kar ni. Nedoločeni integral
preslika funkcijo~$f$ v \emph{množico} vseh funkcij $F$, za katere
velja $F' = f$. Če bi to želeli zapisati zares pravilno, bi dobili
%
\begin{equation*}
  \int x^2 \, d x =
  \set{(x \mapsto x^3/3 + C) \such C \in \RR}.
\end{equation*}
%
Ali naj torej sklepamo, da so matematiki pravzaprav zelo površni pri
pisanju integralov? Da, s stališča formalne logike prav gotovo. Vendar
to ni nujno slabo: matematični zapis v praksi služi ljudem za
sporazumevanje in prav je, da si izberejo tak zapis, s katerim najbolj
učinkovito komunicirajo drug z drugim. Kljub temu pa se je treba
zavedati, kdaj gredo matematiki ">po bližnjici"< in ne zapišejo ali
povedo vsega dovolj natančno, da bi to bilo sprejemljivo za standarde,
ki jih postavlja formalna logika.


%%%%%%%%%%%%%%%%%%%%%%%%%%%%%%%%%%%%%%%%%%%%%%%%%%
\subsection{Substitucija}
\label{sec:substitucija}

\textbf{Substitucija} je osnovna sintaktična operacija, v kateri
\emph{proste} spremenljivke zamenjamo z izrazi. Zapis
%
\begin{equation*}
  \xsubst{e}{x_1 \subto e_1, \ldots, x_n \subto e_n}
\end{equation*}
%
pomeni: ">v izrazu $e$ \emph{hkrati} zamenjaj proste spremenljivke
$x_1$ z $e_1$, $x_2$ z $e_2$, \dots in $x_n$ z $e_n$."<  Na primer,
%
\begin{equation*}
  \subst{x^2 + y}{x \subto 3, y \subto 5, z \subto 12}
\end{equation*}
%
je enako $3^2 + 5$. Nič hudega ni, če se v substituciji omenja
spremenljivko $z$, ki se v izrazu $x^2 + y$ ne pojavi.

Ko naredimo substitucijo, moramo paziti, da se proste spremenljivke ne
">ujamejo"<. Denimo, da želimo v integralu
%
\begin{equation*}
  \int_0^1 \frac{x}{a - x^2} \; dx
\end{equation*}
%
parameter $a$ zamenjati z $y^2$. To naredimo s substitucijo
%
\begin{equation*}
  \xsubst{\left(\int_0^1 \frac{x}{a - x^2} \; dx\right)}{a \subto y^2} =
  \int_0^1 \frac{x}{y^2 - x^2} \; dx.
\end{equation*}
%
Vse lepo in prav. Kaj pa, če želimo $a$ zamenjati z $1 + x$? Ker je
spremenljivka $x$ vezana v integralu, \emph{ne smemo} delati takole:
%
\begin{equation*}
  \xsubst{\left(\int_0^1 \frac{x}{a - x^2} \; dx\right)}{a \subto x^2} =
  \int_0^1 \frac{x}{x^2 - x^2} \; dx ?!
\end{equation*}
%
Ker vstavljamo v integral spremenljivko $x$, moramo vezano
spremenljivko $x$ najprej preimenovati v kaj drugega, na primer $t$,
šele nato vstavimo:
%
\begin{equation*}
  \xsubst{\left(\int_0^1 \frac{x}{a - x^2}\; dt \right)}{a \subto x^2} =
  \xsubst{\left(\int_0^1 \frac{t}{a - t^2} \; dt\right)}{a \subto x^2} =
  \int_0^1 \frac{t}{x^2 - t^2} \; dt.
\end{equation*}
%
Podajmo še nekaj primerov substitucij:
%
\begin{align*}
  \subst{x + y + 1}{x \subto 2} &= 2 + y + 1 \;,
  \\
  \subst{x + y^2 + 1}{x \subto y, y \subto x} &= y + x^2 + 1 \;
  \\
  \subst{\subst{x + y^2 + 1}{x \subto y}}{y \subto x} &=
  x + x^2 + 1 \;,
  \\
  \textstyle
  \subst{x + \int_0^1 x \cdot y \;, d x}{x \subto 2}
  &= \textstyle  2 + \int_0^1 x \cdot y \;, d x \;,
  \\
  \textstyle
  \subst{\int_0^1 x \cdot y \; d x}{y \subto x^2}
  &= \textstyle \int_0^1 t \cdot x^2 \; d t \;.
\end{align*}
%
Ločiti je treba med hkratno in zaporedno substitucijo:
%
\begin{align*}
  \subst{x + y^2}{x \subto y, y \subto x} &= y + x^2
  \\
  \subst{\subst{x + y^2}{x \subto y}}{y \subto x} &=
  \subst{y + y^2}{y \subto x} = x + x^2
  \\
  \subst{\subst{x + y^2}{y \subto x}}{x \subto y} &=
  \subst{x + x^2}{x \subto y} = y + y^2.
\end{align*}
%

V nadaljevanju bomo obravnavali pravila sklepanja za univerzalne in
eksistenčne kvantifkatorje, v katerih se pojavi substitucija. Ker je
sam zapis za substitucijo nekoliko nepregleden, bomo uporabili
nekoliko manj pravilen, a bolj praktičen zapis. Denimo, da imamo
logično formulo $\phi$, v kateri se morda pojavi spremenljivka $x$, ni
pa to nujno. Tedaj pišemo $\phi(x)$. Če želimo zamenjati $x$ z izrazom
$e$, zapišemo $\phi(e)$. To je pravzaprav običajni zapis, kot ga
uporabljajo matematiki za zapis funkcij, mi pa smo ga uporabili za
zapis logičnih formul. Če bi uporabili zapis s substitucijo, bi
formulo označili samo s $\phi$ namesto s $\phi(x)$ in zamenjavo s
$\xsubst{\phi}{x \subto e}$ namesto s $\phi(e)$. Zakaj je ta bolj
priročen zapis hkrati manj pravilen? V formalni logiki strogo ločimo
med \emph{simbolnim zapisom} matematičnega pojma, ki je zaporedje
znakov na papirju, in njegovim \emph{pomenom}, ki je matematična
abstrakcija. Substitucija $\xsubst{\phi}{x \subto e}$ nam pove, kako
niz znakov $\phi$ predelamo v novi niz znakov, torej deluje na novoju
simbolnega zapisa. Ko pišemo $\phi(x)$ pa si že predstavljamo, da je
$\phi$ matematična funkcija, ki deluje na argumentu $x$. S tem nastopi
zmešnjava med simbolnim zapisom in pomenom. Dokler se zmešnjave
zavedamo, je vse v redu.

\subsection{Univerzalni kvantifikator}
\label{sec:univerzalni-kvantifikator}

Univerzalna kvantifikacija $\all{x \in S} \phi$ se prebere ">Za vse $x$
iz $S$ velja $\phi$."< Pravili sklepanja sta
%
\begin{mathpar}
  \inferrule
  {\infer*{\phi(x)}{[x \in S]}}{\all{x \in S} \phi(x)} \ \text{($x$ svež)}
  \and
  \inferrule{\all{x \in S} \phi(x) \\ e \in S}{\phi(e)}
\end{mathpar}
%
pri čemer je $x$ spremenljivka, $\phi(x)$ logična formula in $e$ poljuben izraz.

V besedilu dokažemo se pravilo vpeljave zapiše:
%
\begin{quote}
  \it
  %
  Dokazujemo $\all{x \in S} \phi(x)$:
  %
  \begin{itemize}
  \item[] Naj bo $x \in S$ poljuben.
  \item[] (Dokaz, da velja $\phi(x)$).
  \end{itemize}
  %
  Dokazali smo $\all{x \in S} \phi(x)$.
\end{quote}
%
Pravilo uporabe v besedilu ponavadi ni eksplicitno navedeno, če pa bi
ga že zapisali, bi šlo takole:
%
\begin{quote}
  \it
  %
  Dokazujemo, da velja $\phi(e)$:
  \begin{itemize}
  \item[] (Dokaz, da velja $\all{x \in S} \phi(x)$.)
  \item[] (Dokaz, da velja $e \in S$.)
  \end{itemize}
  %
  Torej velja $\phi(e)$.
\end{quote}


Ob pravilu vpeljave stoji stranski pogoj, da mora biti spremenljivka
$x$ ">sveža"<. To pomeni, da se $x$ ne sme pojavljati drugje v dokazu,
saj bi sicer lahko prišlo do zmešnjave med vezanimi in prostimi
spremenljivkami. V besedilu se dejstvo, da je $x$ svež izraža z
besedico ">poljuben"< ali ">katerikoli"<. Primer, kako gredo stvari
narobe, če ne pazimo in pomešamo spremenljivke:

\begin{izrek}[z napako v dokazu]
  Če je $x$ večji od~$42$, so vsa realna števila večja od~$23$.
\end{izrek}

\begin{proof}
  Denimo, da bi nekoliko nerodno zapisali izrek simbolno takole:
  %
  \begin{equation*}
    x > 42 \lthen \all{x \in \RR} x > 23.
  \end{equation*}
  %
  To je sicer dovoljeno, saj se prosti $x$, ki stoji zunaj $\forall$
  ni ujel, ni pa preveč smotrno, ker smo na dobri poti, da bomo
  zunanji prosti $x$ in vezanega znotraj $\forall$ pomešali. Res, če
  ne upoštevamo pravila, da mora biti $x$ svež, dobimo tale nepravi
  ">dokaz"<:
  %
  \begin{equation*}
    \inferrule*
    {
      \inferrule*
      {\inferrule*
        {[x > 42] \\ 42 > 23}
        {x > 23}
      }
      {\all{x \in \RR} x > 23}
    }
    {x > 42 \lthen \all{x \in \RR} x > 23}
  \end{equation*}
  %
  Pri pravilu za vpeljavo $\forall$ smo uporabili spremenljivko $x$,
  ki pa je že nastopala v začasni hipotezi $x > 42$. Z besedilom bi se
  isti dokaz glasil takole:
  %
  \begin{quote}
    ">Dokazujemo $x > 42 \lthen \all{x \in \RR} x > 23$. Predpostavimo,
    da velja $x > 42$ in dokažimo $\all{x \in \RR} x > 23$. Naj bo $x
    \in \RR$. Po predpostavki je $x > 42$ in ker je $42 > 23$, od tod
    sledi $x > 3$."<
  \end{quote}
  %
  Če bi izrek zapisali bolje kot $x > 42 \lthen \all{y \in \RR} y >
    23$, težav ne bi bilo, saj bi se prejšnji dokaz ">zataknil"<:
  %
  \begin{quote}
    ">Dokazujemo $x > 42 \lthen \all{y \in \RR} y > 23$. Predpostavimo,
    da velja $x > 42$ in dokažimo $\all{y \in \RR} y > 23$. Naj bo $y
    \in \RR$. (Kaj zdaj? Lahko sicer dokažemo $x > 23$, a zares bi
    morali dokazati $y > 23$, kar ne gre.)"<
  \end{quote}
\end{proof}

Pogoj, da mora biti spremenljivka $x$ v pravilu za vpeljavo ">sveža"<,
se v praksi kaže v tem, da pri uvajanju nove spremenljivke izberemo
zanjo novo ime, ki se še ni pojavilo v dokazu.


\subsection{Eksistenčni kvantifikator}
\label{sec:eksistencni-kvantifikator}

Eksistenčna kvantifikacija $\some{x \in S} \phi$ se prebere ">obstaja
$x$ iz $S$, za katerega velja $\phi$"< ali ">za neki $x$ iz $S$ velja
$\phi$."< Pravili sklepanja za eksistenčni kvantifikator se glasita
%
\begin{mathpar}
  \inferrule
  {\phi(e) \\ e \in S}
  {\some{x \in S} \phi(x)}
  \and
  \inferrule
  {\some{x \in S} \phi(x)
    \\
    \infer*{\psi}{[x \in S \land \phi(x)]}}
  {\psi}\ \text{($x$ svež)}
\end{mathpar}
%
kjer je $e$ poljuben izraz in $x$ spremenljivka. Pri tem mora biti $x$
v pravilu uporabe svež. V besedilu pravilo vpeljave uporabimo takole:
%
\begin{quote}
  \it
  %
  Dokazujemo $\some{x \in S} \phi(x)$:
  %
  \begin{enumerate}
  \item (Skonstruiramo element $e \in S$.)
  \item (Dokažemo, da velja $\phi(e)$.)
  \end{enumerate}
  %
  Dokazali smo $\some{x \in S} \phi(x)$.
\end{quote}
%
Pravilo uporabe pa se v besedilu izraža takole:
%
\begin{quote}
  \it
  %
  Dokazujemo $\psi$:
  %
  \begin{enumerate}
  \item (Dokaz izjave $\some{x \in S} \phi(x)$.)
  \item Predpostavimo, da za $x \in S$ velja $\phi(x)$:
    %
    \begin{itemize}
    \item[] (Dokaz izjave $\psi$.)
    \end{itemize}
  \end{enumerate}
  %
  Dokazali smo $\psi$.
\end{quote}

\subsubsection{Enolični obstoj}
\label{sec:enolicni-obstoj}

Poleg običajnega eksistenčnega kvantifikatorja $\exists$ poznamo tudi
\emph{enolični} eksistenčni kvantifikator $\exists!$. Izjavo
$\exactlyone{x}{S}{\phi}$ preberemo ">obstaja natanko en $x \in S$, za
katerega velja $\phi(x)$"<.

Enolični eksistenčni kvantifikator ni osnovni logični operator, ampak
je $\exactlyone{x}{S}{\phi}$ le okrajšava za
%
\begin{equation}
  \label{eq:uniqe-exists}
  \some{x \in S} \phi(x) \land (\all{y \in S} \phi(y) \lthen x = y).
\end{equation}
%
Z besedami preberemo to izjavo takole: ">obstaja $x$ iz $S$, za
katerega velja $\phi(x)$ in za vsak $y \in S$ za katerega velja
$\phi(y)$ sledi $x = y$"<. To je samo zapleten način, kako povedati,
da obstaja natanko en element množice~$S$, ki zadošča pogoju $\phi$.

Pravilo sklepanja za vpeljavo enoličnega obstoja izpeljemo
iz~\eqref{eq:uniqe-exists}:
%
\begin{equation*}
  \inferrule{
    e \in S
    \\
    \phi(e)
    \\
    \infer*{y = e}{y \in S \land \phi(y)}
  }
  {\exactlyone{x}{S}{\phi}}
\end{equation*}
%
V besedilu dokažemo enolični obstoj takole:
%
\begin{quote}
  \it
  %
  Dokazujemo, da obstaja natanko en $x \in S$, za katerega velja
  $\phi(x)$:
  %
  \begin{enumerate}
  \item Obstoj: (Konstrukcija elementa $e \in S$ in dokaz, da velja $\phi(x)$.)
  \item Enoličnost: denimo da za $y \in S$ velja $\phi(y)$:
    %
    \begin{itemize}
    \item[] (Dokaz, da je $e = y$).
    \end{itemize}
  \end{enumerate}
  %
  Dokazali smo $\exactlyone{x \in S} \phi(x)$.
\end{quote}

Če dokažemo enolični obstoj $\exactlyone{x \in S} \phi(x)$, lahko
vpeljemo novo konstanto $c$, ki označuje tisti element iz $S$, ki
zadošča pogoju~$\phi$, pri čemer moramo seveda paziti, da znaka $c$
nismo že prej uporabili za kak drug pomen. Nova konstanta~$c$ je
opredeljena s praviloma
%
\begin{mathpar}
  \inferrule{ }{\phi(c)}
  \and
  \inferrule{
    y \in S
    \\
    \phi(y)
  }
  {y = c}
\end{mathpar}
%
Če v formuli $\phi$ poleg spremenljivke $x$ nastopajo še druge proste
spremenljivke, denimo $y_1, \ldots, y_n$, potem je nova konstanta~$c$
v resnici \emph{funkcija} parametrov $y_1, \ldots, y_n$.

\subsection{Enakost in reševanje enačb}
\label{sec:enakost}

Enakost $=$ je osnovna relacija, ki zadošča naslednjim aksiomom in
pravilom sklepanja:
%
\begin{mathpar}
  \inferrule{ }{a = a}
  \and
  \inferrule{a = b}{b = a}
  \and
  \inferrule{a = b \\ b = c}{a = c}
  \and
  \inferrule{\phi(a) \\ a = b}{\phi(b)}
\end{mathpar}
%
Po vrsti so so pravilo \emph{refleksivnosti}, \emph{simetrije},
\emph{tranzitivnosti} in \emph{zamenjave}. Zaenkrat enakosti ne bomo
posvečali posebne pozornosti, saj jo v praksi študenti dobro
obvladajo.

V osnovni iz srednji šoli se učimo pravil za reševanje enačb: enačbi
smemo na obeh straneh prišteti ali odšteti poljuben izraz, pomnožiti
ali deliti smemo s poljubnim \emph{neničelnim} izrazom, ipd. Od kod
izhajajo ta pravila? Kaj sploh pomeni, da smo enačbo ">rešili"<? Ko
rešimo kvadratno enačbo
%
\begin{equation*}
  x^2 - 5 x + 6 = 0
\end{equation*}
%
običajno zapišemo rešitev takole:
%
\begin{equation*}
  x_1 = 2, \quad x_2 = 3.
\end{equation*}
%
Kako naj to razumemo iz stališča matematične logike? Treba je
pojasniti dvoje: kaj pomenita $x_1$ in $x_2$, saj v prvotni enačbi
nastopa spremenljivka $x$, ter kako naj razumemo vejico med izjavama
$x_1 = 2$ in $x_2 = 3$. Z indeksoma $1$ in $2$ štejemo rešitve enačbe
in sta v resnici nepotrebna,\footnote{Kako pa bi zapisali rešitve
  enačbe $x_1^2 - 5 x_1 + 6 x = 0$?} na kar kaže tudi dejstvo, da
pišemo $x = \ldots$ in ne $x_1 = \ldots$, kadar je rešitev ena sama.
Torej bi lahko rešitev zapisali kot
%
\begin{equation*}
  x = 2, \quad x = 3.
\end{equation*}
%
Sedaj pa je tudi jasno, da bi namesto vejice morala stati disjunkcija,
se pravi
%
\begin{equation*}
  x = 2 \lor x = 3.
\end{equation*}
%
Začetna enačba in tako zapisana rešitev sta logično ekvivalentni:
%
\begin{equation*}
  x^2 - 5 x + 6 = 0 \iff
  x = 2 \lor x = 3.
\end{equation*}
%
Povzemimo: reševanje enačbe je postopek, s katerim dano enačbo $f(x) =
g(x)$ prevedemo v njen \emph{logično ekvivalentno} obliko $x = a_1
\lor x = a_2 \lor \cdots \lor x = a_n$, iz katere so neposredno razvidne
rešitve enačbe.

Pravila za reševanje enačb torej niso nič drugega kot recepti, s
pomočjo katerih enačbo predelamo v njen \emph{ekvivalentno} obliko, ki
je korak bližje končni obliki, v kateri bi radi zapisali rešitev. To
pojasnjuje srednješolska pravila za reševanje enačb. Na primer, za
realna števila $a, b, c \in \RR$ vedno velja
%
\begin{equation*}
  a = b \lthen c \cdot a = c \cdot b,
\end{equation*}
%
medtem ko obratna implikacija
%
\begin{equation*}
  c \cdot a = c \cdot b \lthen a = b
\end{equation*}
%
za splošne $a$ in $b$ velja le v primeru, ko je $c \neq 0$. Ker pri
reševanju enačb potrebujemo implikacijo v obe smeri, srednješolce
učimo, da smejo enačbo množiti samo z od nič različnimi števili.

\begin{vaja}
  Kako bi srednješolcem pojasnil, od kod izvira pravilo za množenje
  enačbe z neničelnim številom?
\end{vaja}

\begin{vaja}
  Enačbo $f(x) = g(x)$ smo ">rešili"< z zaporedjem korakov
  %
  \begin{align*}
    f(x) = g(x) &\liff \\
    f_1(x) = g_1(x) &\liff \\
    \vdots & \\
    f_k(x) = g_k(x) &\lthen \\
    f_{k+1}(x) = g_{k+1}(x) &\liff \\
    \vdots & \\
    x = a_1 \lor \cdots \lor x = a_n
  \end{align*}
  %
  kjer smo v $k$-tem koraku namesto ekvivalence pomotoma naredili
  implikacijo. Smo s tem dobili preveč ali premalo rešitev prvotne
  enačbe?
\end{vaja}



%%% Local Variables: 
%%% mode: latex
%%% TeX-master: "lmn"
%%% End: 


\chapter{Boolova algebra}

\section{Resničnostne tabele}

Vsaka izjava ima \textbf{resničnostno vrednost}. Resničnostni vrednosti sta $\bot$
(neresnica) in $\top$ (resnica). Na primer, izjava $\bot \lor (\top \lthen \top)$ je resnična, njena resničnostna vrednost je $\top$. Izjava $2 + 2 = 5$ je neresnična, njena resničnostna vrednost je~$\bot$.

Kadar izjava vsebuje spremenljivke (pravimo jim tudi \emph{parametri}), je njena
resničnostna vrednost \emph{odvisna} od parametrov. Na primer, če sta $x, y \in \NN$ spremenljivki, je resničnostna vrednost izjave $x + 2 y < 3$ odvisna
od $x$ in $y$, kar lahko prikažemo z \textbf{resničnostno tabelo}:
%
\begin{center}
  \begin{tabular}{ccc}
    \toprule
    $x$ & $y$ & $x + 2 y < 3$ \\ \midrule
    $0$ & $0$ & $\top$ \\
    $0$ & $1$ & $\top$ \\
    $1$ & $0$ & $\top$ \\
    $2$ & $0$ & $\top$ \\
    $1$ & $1$ & $\bot$ \\
    $0$ & $2$ & $\bot$ \\
    $\vdots$ & $\vdots$ & $\vdots$ \\
    \bottomrule
  \end{tabular}
\end{center}
% 
Kot vidimo, je lahko resničnostna tabela neskončna. Bolj uporabne so končne resničnostne tabele, v katerih parametri zavzemajo vrednosti iz končne množice.

V izjavi lahko nastopajo tudi \textbf{izjavne spremenljivke} ali \textbf{izjavni simboli}, to se spremenljivke, ki zavzamejo vrednosti $\bot$ in $\top$.
Na primer, naj bo $\two = \set{\bot, \top}$ in $p, q \in \two$. Tedaj je $\neg p \lor q$ izjava, katere resničnostna tabela je
%
\begin{center}
  \begin{tabular}{ccc}
    \toprule
    $p$ & $q$ & $\neg p \lor q$ \\ \midrule
    $\bot$ & $\bot$ & $\top$ \\
    $\bot$ & $\top$ & $\top$ \\
    $\top$ & $\bot$ & $\bot$ \\
    $\top$ & $\top$ & $\top$ \\
    \bottomrule
  \end{tabular}
\end{center}

Izjava $\phi(p_1, \ldots, p_n)$, v kateri nastopajo izjavne spremenljivke $p_1, \ldots, p_n$ (in nobeni drugi parametri) določa preslikavo
%
\begin{equation*}
  \two \times \cdots \times \two \to \two
\end{equation*}
%
s predpisom
%
\begin{equation*}
    (p_1, \ldots, p_n) \mapsto \phi(p_1, \ldots, p_n)
\end{equation*}
%
Preslikavi, ki slika iz produkta $\two \times \cdots \times \two$ v $\two$ pravimo \textbf{Boolova preslikava}. Prikažemo jo lahko z resničnostno tabelo. Če ima preslikava~$n$ argumentov, ima tabela $2^n$ vrstic.


\subsection{Tavtologije}

Izjava je \textbf{tavtologija}, če je njena resničnostna vrednost $\top$ ne glede na
vrednosti parametrov. Premisli: kako iz resničnostne tabele razberemo, ali je
izjava tavtologija?

\begin{izrek}
  Naj bo $\phi$ izjava, v kateri nastopajo le izjavni simboli
  $p_1,\ldots,p_n$. Tedaj je $\phi$ tavtologija, če in samo če ima dokaz.
\end{izrek}

\begin{dokaz}
  Dokaz najdete v \cite{prijatelj92:_osnov}.
\end{dokaz}

\noindent
%
Izrek je pomemben, ker nam pove, da lahko dokazovanje izjav nadomestimo s preverjanjem resničnostnih tabel.

\begin{opomba}
  Izrek velja samo za izjave, ki jih sestavimo iz izjavnih simbolov, $\bot$, $\top$ in
  logičnih veznikov $\neg$, $\land$, $\lor$, $\lthen$, $\liff$. Za splošne izjave, ki vsebujejo tudi $\forall$ in $\exists$ izrek \emph{ne} velja. Lahko se namreč zgodi, da ima izjava neskončno resničnostni tabelo, v kateri so vse resničnostne vrednosti~$\top$, a izjava nima dokaza.
\end{opomba}

\subsection{Polni nabori}

Vsaka formula v izjavnem računu ima resničnostno tabelo. Ali lahko vsako tabelo
dobimo kot resničnostno tabelo neke formule? Na primer, ali obstaja formula,
katere resničnostna tabela se glasi
%
\begin{center}
  \begin{tabular}{ccc}
    \toprule
    $p$ & $q$ & ? \\ \midrule
    $\bot$ & $\bot$ & $\bot$ \\
    $\bot$ & $\top$ & $\top$ \\
    $\top$ & $\bot$ & $\top$ \\
    $\top$ & $\top$ & $\bot$ \\
    \bottomrule
  \end{tabular}
\end{center}
%
Odgovor je pritrdilen. Podajmo dva načina, kako tako izjavo izračunamo iz tabele.

\subsubsection{Disjunktivna oblika}
\label{sec:disjunktivna-oblika}

Za vsako vrstico v tabeli, ki ima vrednost $\top$ zapišemo konjunkcijo simbolov in
njihovih negacij, pri čemer negiramo tiste simbole, ki imajo v dani vrstici vrednost
$\bot$. Na primer, v zgornji tabeli imata druga in tretja vrstica vrednost $\top$, zanju
zapišemo konjunkciji:
%
\begin{itemize}
\item 2.~vrstica: $\neg p \land q$,
\item 3.~vrstica: $p \land \neg q$.
\end{itemize}
%
Nato tvorimo disjunkcijo tako dobljenih konjunkcij:
%
\begin{equation*}
  (\neg p \land q) \lor (p \land \neg q).
\end{equation*}
%
Dobljena formula ima želeno resničnostno tabelo.

\subsubsection{Konjuktivna oblika}
\label{sec:konjuktivna-oblika}

Za vsako vrstico v tabeli, ki ima vrednost $\bot$ zapišemo
disjunkcijo simbolov in njihovih negacij, pri čemer negiramo tiste simbole, ki
imajo v dani vrstici vrednost $\top$. Na primer, v zgornji tabeli imata prva in
četrta vrstica vrednost $\bot$, zanju zapišemo disjunkciji:
%
\begin{itemize}
\item 1.~vrstica: $p \lor q$
\item 4.~vrstica: $\neg p \lor \neg q$
\end{itemize}
%
Nato tvorimo konjunkcijo tako dobljenih disjunkcij:
%
\begin{equation*}
  (p \lor q) \land (\neg p \lor \neg q).
\end{equation*}
%
Zgornjo tabelo bi lahko dobili tudi kot resničnostno tabelo formule
%
\begin{equation*}
    p \oplus q
\end{equation*}

\subsection{Polni nabori}
\label{sec:polni-nabori}

Vidimo, da lahko vsako resničnostno tabelo dobimo z uporabo veznikov $\neg$, $\lor$ in
$\land$. \textbf{Polni nabor} je tak izbor veznikov, s katerim lahko dobimo vsako
resničnostno tabelo.

Torej je $\neg$, $\lor$, $\land$ poln nabor. Lahko bi ga še zmanjšali na $\neg$, $\land$, saj lahko $p \lor q$ izrazimo kot $\neg (\neg p \land \neg q)$.

Nabor $\land$, $\lor$ pa \emph{ni} poln, saj ne moremo dobiti resničnostne tabele
%
\begin{center}
  \begin{tabular}{cc}
    \toprule
    $p$ & ? \\ \midrule
    $\bot$ & $\top$ \\
    $\top$ & $\bot$ \\
    \bottomrule
  \end{tabular}
\end{center}
%
Res, če iz $p$, $\land$ in $\lor$ sestavimo poljubno formulo $\phi(p)$, na primer $(p \land (p \lor p)) \land p$, bo ta ekvivalentna~$p$ in bo zato veljalo $\phi(\top) = \top$, zgornja tabela pa zahteva $\phi(\top) = \bot$.


\section{Boolova algebra}

Ekvivalentni izjavi imata enake resničnostne vrednosti, torej lahko ekvivalenco
$\liff$ obravnavamo kar kot enakost, saj to tudi je, kar se tiče resničnostnih
vrednosti. Zato lahko namesto $p \liff q$ pišemo tudi $p = q$, če imamo v mislih le
resničnostne vrednosti.

\begin{opomba}
  Ekvivalentni izjavi imata lahko različen \emph{pomena}. Na primer,
  $\all{x, y \in R} x + y = y + x$ in
  $\all{\alpha \in R} \sin(2 \alpha) = 2 \cdot \cos \alpha \cdot \sin \alpha$ sta
  ekvivalentni, saj sta obe resnični, a ne moremo reči, da je njun pomen enak. (Predstavljate si, da bi bi vas v srednji šoli profesorica matematike vprašala adicijski izrek za $\sin$, vi pa bi odgovorili ">vrstni red seštevanja realnih števil ne vpliva na vrednost vsote"<.)
\end{opomba}


Za logične veznike veljajo \emph{algebrajska pravila}, se pravi enačbe, kakršne poznamo v algebri. Ta pravila lahko uporabljamo kot računska pravila, s katerimi lahko izjavo poenostavimo v ekvivalentno obliko. Pogosto je tako računanje bolj prikladno kot dokazovanje. Spodaj našteta pravila lahko preverimo tako, da zapišemo resničnostne tabele izjav in jih primerjamo.

Pravilom, ki veljajo za logične veznike, pravimo \textbf{Boolova algebra}.
Razdelimo jih po sklopih.

Pravila za konjunkcijo:
%
\begin{align}
  (p \land q) \land r &= p \land (q \land r) \tag{asociativnost $\land$} \\
  p \land q &= q \land p \tag{komutativnost $\land$} \\
  p \land p &= p \tag{idempotentnost $\land$} \\
  \top \land p &= p \tag{$\top$ je nevtralen za $\land$} \\
  \bot \land p &= \bot \tag{$\bot$ absorbira $\land$}
\end{align}
%
Pravila za disjunkcijo:
%
\begin{align}
  (p \lor q) \lor r &= p \lor (q \lor r) \tag{asociativnost $\lor$} \\
  p \lor q &= q \lor p \tag{komutativnost $\lor$} \\
  p \lor p &= p \tag{idempotentnost $\lor$} \\
  \bot \lor p &= p \tag{$\bot$ je nevtralen za $\lor$} \\
  \top \lor p &= \top \tag{$\top$ absorbira $\lor$}
\end{align}
%
Pravila za implikacijo:
%
\begin{align}
  (p \lthen q) &= (\neg q \lthen \neg p) \tag{kontrapozitivna oblika $\lthen$}\\
  (p \lthen q) &= \neg p \lor q \notag \\
  (\bot \lthen q) &= \top \notag \\
  (\top \lthen q) &= q \notag \\
  (p \lthen \bot) &= \neg p \notag \\
  (p \lthen \top) &= \top \notag
\end{align}
%
Kombinirana pravila:
%
\begin{align}
  \neg(p \land q) &= \neg p \lor \neg q \tag{de Morganovo pravilo za $\land$} \\
  \neg(p \lor q) &= \neg p \land \neg q \tag{de Morganovo pravilo za $\lor$} \\
  \neg(p \lthen q) &= p \land \neg q \notag \\
  p \land (p \lor q) &= p \tag{absorbcijsko pravilo za $\land$}\\
  p \lor (p \land q) &= p \tag{absorbcijsko pravilo za $\lor$} \\
  p \land (q \lor r) &= (p \land q) \lor (p \land r) \tag{distributivnost $\land$}\\
  p \lor (q \land r) &= (p \lor q) \land (p \lor r) \tag{distributivnost $\lor$}
\end{align}
%
Pravila za negacijo:
%
\begin{align}
  \neg \top  &= \bot \notag \\
  \neg \bot &= \top \notag \\
  \neg\neg p &= p \tag{negacija je involucija} \\
  p \lor \neg p &= \top \tag{izključena tretja možnost} \\
  p \land \neg p &= \bot \notag
\end{align}

Zapišimo še uporabna logična pravila za kvantifikatorje. Tokrat uporabimo $\liff$
namesto $=$, ker je to bolj običajno:
%
\begin{align*}
  (\all{x \in \emptyset} \phi(x))   &\iff   \top \\
  (\some{x \in \emptyset} \phi(x))   &\iff   \bot \\
  (\all{x \in \set{a}} \phi(x))   &\iff   \phi(a) \\
  (\some{x \in \set{a}} \phi(x))   &\iff   \phi(a) \\
  (\neg \all{x \in A} \phi(x))   &\iff   \some{x \in A} \neg \phi(x) \\
  (\neg \some{x \in A} \phi(x))   &\iff   \all{x \in A} \neg \phi(x) \\
  (\psi \lthen \all{x \in A} \phi(x))   &\iff   \all{x \in A} \psi \lthen \phi(x) \\
  (\psi \lor \all{x \in A} \phi(x))   &\iff   \all{x \in A} \psi \lor \phi(x) \\
  (\psi \land \some{x \in A} \phi(x))   &\iff   \some{x \in A} \psi \land \phi(x) \\
  (\all{u \in A \times B} \phi(u))   &\iff   \all{x \in A} \all{y \in B} \phi(x, y) \\
  (\some{u \in A \times B} \phi(u))   &\iff   \some{x \in A} \some{y \in B} \phi(x, y) \\
  (\all{u \in A + B} \phi(u))   &\iff   (\all{x \in A} \phi(\inl(x))) \land (\all{y \in B} \phi(\inr(y))) \\
  (\all{u \in A \cup B} \phi(u))   &\iff   (\all{x \in A} \phi(x)) \land (\all{y \in B} \phi(y)) \\
  (\some{u \in A + B} \phi(u))   &\iff   (\some{x \in A} \phi(\inl(x))) \lor (\some{y \in B} \phi(\inr(y))) \\
  (\some{u \in A \cup B} \phi(u))   &\iff   (\some{x \in A} \phi(x)) \lor (\some{y \in B} \phi(y)) \\
  (\all{u \in \set{x \in A \such \psi(x)}} \phi(u))   &\iff   \all{x \in A} \psi(x) \lthen \phi(x) \\
  (\some{u \in \set{x \in A \such \psi(x)}} \phi(u))   &\iff   \some{x \in A} \psi(x) \land \phi(x)
\end{align*}
%
Te ekvivalence je treba preveriti tako, da jih dokažemo.


\chapter{Podmnožice in potenčne množice}

\subsection{Definicija relacije $\subseteq$}

Pravimo, da je množica $S$ \textbf{podmnožica} množice $T$, pišemo $S \subseteq T$, ko velja $\all{x \in S} x \in T$. Pravimo tudi, da je $S$ \textbf{vsebovana} v $T$ in da je $T$ \textbf{nadmnožica}~$S$.

Vedno velja $\emptyset \subseteq S$ in $S \subseteq S$.

Princip ekstenzionalnosti za množice pravi:
%
\begin{equation*}
  S = T \iff (\all{x \in S} S \in T) \land (\all{y \in T} y \in S)
\end{equation*}
%
kar lahko zapišemo s podmnožicami:
%
\begin{equation*}
  S = T \iff S \subseteq T \land T \subseteq S.
\end{equation*}
%
Vsaka podmnožica $S \subseteq A$ opredeljuje neko lastnost elementov iz $A$: tisti
elementi, ki imajo opredeljeno lastnost, so v $S$, ostali pa ne.

\begin{primer}
  Naj bo $P$ množica vseh praštevil, torej je $P \subseteq N$. Podmnožica $P$
  opredeljuje lastnost ">je praštevilo"<.
\end{primer}


\subsection{Kako tvorimo podmnožice}

Če je $\phi(x)$ logična formula, v kateri nastopa spremenljivka $x \in A$, lahko tvorimo množico
%
\begin{equation*}
    \set{ x \in A \such \phi(x) }.
\end{equation*}
%
Pri tem je $x$ vezana spremenljivka. Za to množico velja:
%
\begin{equation*}
    a \in \set{ x \in A \such \phi(x) } \iff a \in A \land \phi(a).
\end{equation*}
%
Povedano z besedami: elementi množice $\set{ x \in A \such \phi(x) }$ so tisti elementi iz $A$, ki zadoščajo pogoju $\phi$.
%
Velja $\set{ x \in A \such \phi(x) } \subseteq A$, prav tako pa
\begin{equation*}
  \set{x \in A \mid \phi(x)} \subseteq \set{x \in A \mid \psi(x)} \iff
  \all{x \in A} \phi(x) \lthen \psi(x).
\end{equation*}


\subsection{Kanonična inkluzija}

Za podmnožico $S \subseteq T$ definiramo \textbf{kanonično inkluzijo} ali \textbf{kanonično vključitev} $i_S : S \to T$, s predpisom $i_S : x \mapsto x$. Pozor, to ni identiteta, razen v primeru $S = T$!.
Oznaka $i_S$ ni standardna, pravzaprav standardne oznake ni.

Če je $f : T \to U$ in $S \subseteq T$, pravimo kompozitumu $f \circ i_S$ \textbf{zožitev} preslikave $f$ na $S$, pišemo $\restrict{f}{S}$.


\section{Potenčna množica}

\subsection{Definicija potenčne množice}

Za vsako množico $A$ tvorimo množico $\pow{A}$, ki ji pravimo \textbf{potenčna množica}.
Elementi potenčne množice $\pow{A}$ so natanko podmnožice množice $A$:
%
\begin{equation*}
    S \in \pow{A} \iff S \subseteq A
\end{equation*}
%
Na primer $\pow{\emptyset} = \set{\emptyset}$ in
%
\begin{equation*}
  P(\set{a,b,c}) = \set{\set{}, \set{a}, \set{b}, \set{c}, \set{a,b}, \set{a,c}, \set{b,c}, \set{a,b,c}}.
\end{equation*}


\subsection{Karakteristične funkcije}

\textbf{Karakteristična funkcija} na množici $A$ je funkcija z domeno $A$ in kodomeno $\two$. Tu je $\two = \set{\bot, \top}$ množica resničnostnih vrednosti.

Eksponentna množica $\two^A$ je torej množica vseh karakterističnih funkcij na $A$.

\begin{opomba}
  Karakteristične funkcije se uporabljajo tudi v analizi, kjer jih
  običajno razumemo kot preslikave $A \to \set{0,1}$ namesto $A \to \set{\bot, \top}$. Ker sta množici $\set{\bot,\top}$ in $\set{0,1}$ izomorfni, to ni bistvena razlika.
\end{opomba}

Karakteristično funkcijo si lahko predstavljamo kot preslikavo, ki opredeljuje
neko lastnost elementov~$A$: tisti elementi, ki imajo opredeljeno lastnost, se
slikajo v $\top$, ostali pa v $\bot$.

\begin{primer}
  Preslikava $p : \NN \to \two$, definirana s predpisom
  %
  \begin{equation*}
    p(n) = 
    \begin{cases}
      \top & \text{če je $n$ praštevilo}, \\
      \bot & \text{če $n$ ni praštevilo}.
    \end{cases}
  \end{equation*}
  %
  je karakteristična preslikava lastnosti ">je praštevilo"<. Lahko bi jo zapisali tudi takole:
  %
  \begin{equation*}
    p(n) = (\some{k, m \in \NN} k \geq 2 \land m \leq 2 \land n = k \cdot m).
  \end{equation*}
\end{primer}


\subsection{Izomorfizem $\pow{A} \cong 2^A$}

Videli smo, da lahko neko lastnost elementov množice $A$ predstavimo bodisi s
podmnožico bodisi s karakteristično preslikavo. To nam da idejo, da med
podmnožicami $A$ in karakterističnimi preslikavami na $A$ obstaja neka zveza.

\begin{izrek}
  $\pow{A} \cong 2^A$.
\end{izrek}

\begin{dokaz}
  Definirajmo preslikavi
  %
  \begin{align*}
    \chi &: \pow{A} \to 2^A &
    \xi &: 2^A \to \pow{A} \\
    \chi_S(x) &\defeq
      \begin{cases}
        \top & \text{če $x \in S$,} \\
        \bot & \text{če $x \not\in S$,}
      \end{cases}
    &
    \xi_f &\defeq \set{x \in A \such f(x) = \top}.
  \end{align*}
  %
  Ta predpisa bi lahko krajše zapisali tudi takole:
  %
  \begin{align*}
  \chi_S(x) &\defeq (x \in S), &
  \xi_f &\defeq \set{x \in A \such f(x) }.
  \end{align*}
  %
  Preslikavi $\chi_S$ pravimo \textbf{karakteristična funkcija podmnožice $S$}.
  %
  Trdimo, da sta $\chi$ in $\xi$ inverza:
  %
  \begin{enumerate}
  \item 
    Dokažimo $\chi \circ \xi = \id[2^A]$. Uporabimo princip ekstenzionalnosti za preslikave.
    Naj bo $f \in 2^A$. Dokažimo, da je $\chi_{\xi_f} = f$.
    Uporabimo princip ekstenzionalnosti za preslikave. Naj bo $x \in A$:
    %
    \begin{equation*}
      \chi_{\xi_f}(x) = (x \in \xi_f) = f(x).
    \end{equation*}

  \item
    Dokažimo $\xi \circ \chi = id_{\pow{A}}$. Uporabimo princip ekstenzionalnosti za preslikave. Naj bo $S \in \pow{A}$. Dokažimo, da je $\xi_{\chi_S} = S$:
    %
    \begin{equation*}
      \xi_{\chi_S} = \set{x \in A \such \chi_S(x)} = \set{x \in A \such x \in S} = S.
    \end{equation*}
  \end{enumerate}
\end{dokaz}

\subsection{Boolova algebra podmnožic}

Podmnožice množice $A$ tvorijo Boolovo algebro za operacije presek $\cap$, unija $\cup$ in relativni komplement. Nevtralni element za unijo je $\emptyset$ in nevtralni element za presek je $A$.

Definirajmo tudi operacijo \textbf{simetrična razlika $\oplus$}, ki podmnožicama $S, T \in A$ priredi podmnožico
%
\begin{equation*}
  S \oplus T \defeq (S \setminus T) \cup (T \setminus S) = (S \cup T) \setminus (S \cap T).
\end{equation*}
%
Potenčna množica $\pow{A}$ je za operacijo $\oplus$ Abelova grupa.


\chapter{Razredi in družine}

\section{Russellov paradoks}

V prejšnji lekciji smo spoznali zapis podmnožice

    { x ∈ A | φ(x) }

ki tvori podmnožico `A` vseh elementov, ki zadoščajo pogoju `x`. Ko je bila
teorija množic še v povojih, se je sama po sebi ponujala ideja, da bi lahko
opisali množice kot "kakršnokoli zbirko stvari. Torej bi lahko pisali

    { x | φ(x) }

za množico vseh tistih stvari (objektov, matematičnih entitet), ki zadoščajo
pogoju `φ`. Se pravi, da bi veljalo

    a ∈ { x | φ(x) } ⇔ φ(a)

A izkaže se, da ne moremo kar tako tvoriti povsem poljubnih množic objektov. To
je odkril znameniti filozof, logik in matematik Betrand Russell. Razmislek se po
njem imenuje *Russellov paradoks*. Le-ta je v matematiko vnesel pravo "krizo
temeljev", iz katere se je v prvi polovici 20. stoletja razvila logika in
temelji matematike, kot jih poznamo danes.

Russellov paradoks gre takole. Denimo, da bi lahko tvorili poljubne množice
objektov. Tedaj bi lahko tvorili tudi množico vseh množic, ki niso element same
sebe:

    R := { S | S ∉ S }

Sedaj bomo izpeljali protislovje tako, da bomo dokazali `R ∈ R` in `R ∉ R`:

1. Dokažimo `R ∉ R`.

   Denimo, da bi veljalo `R ∈ R`. Potem po definiciji `R` velja `R ∉ R`, kar
   je v protislovju s predpostavko `R ∈ R`.

2. Dokažimo `R ∈ R`. V prvem koraku smo že dokazali `R ∉ R`, torej po
   definiciji `R` velja `R ∈ R`.

Kaj lahko storimo? Očitno je treba pazljivo nadzorovati dopustne konstrukcije
množic.

\section{Množice in razredi}

V sodobni teoriji množic Russellov paradoks razrešimo tako, da ločimo med dvema
različnima zvrstema zbirk ali skupkov elementov, namreč **množicami** in **razredi**.

Torej imamo opravka s tremi zvrstmi matematičnih objektov:

1. Elemnti, ki niso množice (na primer naravna števila), pravimo jim **urelementi**.
2. Zbirke elementov, ki se imenujejo **množice**.
3. Zbirke elementov, ki se umenujejo **razredi**.

Elementi množic so urelementi in množice. Enako velja za razrede.

V čem je torej razlika med množicami in razredi?

> Množica je lahko element (druge množice ali razreda)
> Razred ne more biti element (druge množice ali razreda).

S tem želimo povedati, da je zapis

    x ∈ Y

*neveljaven*, če je `x` razred. Se pravi, če je `x` razred, potem `x ∈ Y` sploh
ni veljaven izraz. Ne moremo govoriti o tem, da je resničen ali neresničen, saj
sploh ni smiselen.

Vsaka množica je hkrati razred. Ni pa vsak razred tudi množica.

Razred je množica, če ga lahko skonstruiramo še na kak drug način s pomočjo
pravil za konstrukcije množic (kartezični produkti, vsote, eksponenti, unije,
preseki, podmnožice in vse ostale konstrukcije množic, ki jih bomo še spoznali).

**Pravi razred** je tak razred, ki ni množica.

Z zapisom

    { x | φ(x) }

definiramo *razred* vseh objektov, ki zadoščajo pogoju `φ`. Se pravi, da velja

    a ∈ { x | φ(x) } ⇔ φ(a)

Poglejmo nekaj primerov.

**Russellov razred** `R := { S | S ∉ S }` vsebuje vse množice, ki niso element
same sebe. Paradoks smo razrešili, saj je nesmiselno zapisati `R ∈ R`.

**Razred vseh množic**

    V := { S | S je množica }

ki ga označimo tudi s `Set`. To je pravi razred. Res, če bi bil `V` množica,
potem bi lahko tvorili podmnožico

    { S ∈ V | S ∉ S }

ki ni nič drugega kot Russellov `R`. Tako bi spet dobili protislovje. Torej `V`
ni množica.

Ostali primeri, v katere se ne bomo poglabljali:

* razred vseh enojcev `{ S | ∃! x ∈ S . ⊤ }`
* razred vseh grup
* razred vseh vektorskih prostorov

Z razredi lahko delamo tako kot z množicami: tvorimo unije, preseke in produkte
razredov, govorimo po podrazredih). Pri tem uporabljamo enake oznake za
operacije kot pri množicah. Paziti moramo le, da razreda nikoli ne uporabimo kot
element kake množice ali razreda. Na primer, če je `C` razred, lahko tvorimo
"potenčni razred" `P(C)`, ki vsebuje vse *podmnožice* `C`:

    P(C) := { S | S ∈ Set ∧ S ⊆ C }

Ne smemo pa tvoriti `{ D | D ⊆ C }`, ker bi s tem `C` postal element razreda `{D | D ⊆ C}`.

\section{Družine množic}

Pogosto imamo opravka z zbirko množic. Če je zbirka končna, lahko množice preprosto
naštejemo in vsako od njih poimenujemo

    A = ...
    B = ...
    C = ...

Če je množic neskončno, jih morda lahko oštevilčimo:

    A_1 = ...
    A_2 = ...
    A_3 = ...
    A_4 = ...
    ...

A tu se zadeve še ne končajo, saj lahko v splošnem obravnavamo poljubno zbirko množic.
Takim zbirkam pravimo *družine množic*. Družina množic je *indeksirana* z elementi
neke množice `I`, ki ji pravimo *indeksna množica*. Za vsak `i ∈ I` imamo množico

    A_i = ...

To lahko izrazimo tudi takole:

> **Definicija:** **Družina množic** je preslikava `I → Set`. Množici `I` pravimo *indeksna*
> množica.

Primeri družin:

1. Končno zbirko množic lahko indeksiramo s končno množico. Denimo, da imamo
   množice `A`, `B`, `C`, `D`, `E`. Iz njih lahko tvorimo družino `S`

        I = {1, 2, 3, 4, 5}

        S_1 = A
        S_2 = B
        S_3 = C
        S_4 = D
        S_5 = E.

2. Množice v družini se lahko ponavljajo. V prejšnjem primeru bi lahko na primer
   imeli `A = C` in bi tako veljalo `S_1 = S_3`. Skrajni primer je *konstantna* družina,
   v kateri so vse množice med seboj enake.

3. *Prazna* družina je družina množic, ki je indeksirana `∅`.

4. Prazno družino moramo ločiti od družine praznih množic

        I → Set
        i ↦ ∅

5. Neprazna družina je družina indeksirana z neprazno množico.
   Družina nepraznih množic je družina, v kateri so vse množice neprazne:

      * Prazna družina je družina nepraznih množic.
      * Družina praznih množic je lahko prazna družina (ko je indeksna množica `∅`)
      * Družina praznih množic je lahko neprazna družina (ko je indeksna množica nerazna).

\section{Konstrukcije in operacije z družinami množic}

Operacije `×`, `+`, `∩` in `∪` lahko posplošimo tako, da namesto z dvema
množicama delujejo na poljubnem številu množic. V ta namen uporabimo družine
množic.

\subsection{Presek in unija družine}

Presek in unija družine `A : I → Set` je definirana takole:

    ⋃_(i ∈ I) A_i = { x | ∃ i ∈ I . x ∈ A_i }

    ⋂_(i ∈ I) A_i = { x | ∀ i ∈ I . x ∈ A_i }

Pozor! Na desni strani imamo razred! Res se lahko zgodi, da dobimo pravi razred, denimo
kot presek prazne družine:

    ⋂_(i ∈ ∅) A_i = { x | ∀ i ∈ ∅ . x ∈ A_i } = { x | ⊤ } = V

Kdaj pa dobimo množico? Presek neprazne družine je vedno množica. Res, če imamo
`k ∈ I`, potem velja

    ⋂_(i ∈ ∅) A_i = { x ∈ A_k | ∀ i ∈ ∅ . x ∈ A_i }

Sedaj na desni ne stoji več razred, ampak podmnožica množice `A_k`.

Kaj pa unija družine množic? Ali je množica, ali bi lahko dobili pravi razred, denimo `V`,
kot unijo družine množic? Izkaže se, da za to potrebujemo aksiom:

**Aksiom:** Unija družine množic je množica.

\subsection{Kartezični produkt}

Denimo, da imam družino množic `A : I → Set`.

**Funkcija izbire f** za `A` je prirejanje, ki vsakemu indeksu `i ∈ I` priredi neki element
`f(i) ∈ A_i` iz `A_i`.

Primer: funkcija izbire za družino

     A : N → Set
     A_n = { x ∈ R | 0 < x < 2^(-n) }

je na primer `f(n) = 2^(-n - 1) ali pa f(n) = 2^(-n) / 3`. To ni edina funkcija izbire za `A`.

**Kartezični produkt** družine `A : I → Set` je množica

    ∏_(i ∈ I) A_i

katere elementi so funkcije izbire za `A`. To je nova konstrukcija množice.

Za vsak `j ∈ I` imamo **`j`-to projekcijo**

    pr_j :  ∏_(i ∈ I) A_i → A_j
    pr_j :  f ↦ f(j)

Običajni kartezični produkt dveh množic je poseben primer produkta množic, namreč družine
množic, ki je indeksirana z `I = {1, 2}`. Natančneje, velja

  `A × B ≅ ∏_(i ∈ {1, 2}) C_i`

  kjer je `C_1 = A` in `C_2 = B`.

Tudi eksponentna množica je poseben primer produkta množic, saj velja

    B^A ≅ ∏_(a ∈ A) B

Na desni imamo produkt konstantne družine množic

    A → Set
    a ↦ B

\subsection{Koprodukt ali vsota množic}

Vsoto množic posplošimo na koprodukt družine. Za dano družino `A : I → Set` tvorimo množico

    ∑_(i ∈ I) A_i

Elementi koprodukta so oblike

    in_k(a)

kjer je `k ∈ I` in `a ∈ A_k`. Preslikavi

    in_k : A_k → ∑_(i ∈ I) A_i

pravimo **`k`-ta injekcija**.

Namesto `∑` se piše tudi `∐`.

TODO: prva in druga projekcija.

Poseben primer koprodukta je vsota `A + B`, saj velja

    A + B ≅ ∑_(k ∈ {1, 2}) C_k

kjer je

    C : {1, 2} → Set
    C_1 = A
    C_2 = B.

Kartezični produkt `A × B` je tudi poseben primer koprodukta, saj velja

    A × B ≅ ∑_{a ∈ A} B

Na desni imamo tokrat koprodukt konstantne družine množic

    A → Set
    a ↦ B




\chapter{Lastnosti preslikav}

Mnogi ste v srednji šoli že spoznali osnovne lastnosti preslikav, kot so injektivnost, surjektivnost in bijektivnost
preslikave. V tej lekciji ponovimo te pojme in jih povežemo še s pojmoma monomorfizem in epimorfizem, ki sta pomembna v
algebri

\section{Osnovne lastnosti preslikav}

\subsection{Injektivna, surjektivna, bijektivna preslikava}

**Definicija:** Preslikava `f : A → B` je

* **injektivna**, ko velja `∀ x y ∈ A . f(x) = f(y) ⇒ x = y`
* **surjektivna**, ko velja `∀ y ∈ B . ∃ x ∈ A . f(x) = y`
* **bijektivna**, ko je surjektivna in injektivna

*Opomba:* Pogosto vidimo definicijo injektivnosti, ki pravi, da `f` slika različne elemente v različne vrednosti, se
pravi `∀ x y ∈ A . x ≠ y ⇒ f(x) ≠ f(y)`. Ta definicija je ekvivalentna naši, a jo ne priporočamo, ker je manj uporabna.
Naša definicija namreč podaja recept, kako preverimo injektivnost: predpostavimo `f(x) = f(y)` in od tod izpeljemo
`x = y` tako, da predelamo *enačbo* `f(x) = f(y)` v enačbo `x = y`. To je v splošnem lažje kot predelava *neenačb* v
*neenačbe*.

**Naloga:** primerjaj definicijo injektivnosti z zahtevo, da mora biti prirejanje, ki določa preslikavo, enolično.

**Naloga:** primerjaj definicijo surjektivnost z zahtevo, da mora biti prirejanje, ki določa preslikavo, celovito.


\subsection{Monomorfizmi in epimorfizmi}

**Definicija:** Preslikava `f : A → B` je

* **monomorfizem (mono)**, ko jo lahko krajšamo na levi:
  `∀ C ∈ Set ∀ g, h : C → A . f ∘ g = f ∘ h ⇒ g = h`

* **epimorfizem* (epi)**, ko jo lahko krajšamo na desni:
  `∀ C ∈ Set ∀ g, h : B → C . g ∘ f = h ∘ f ⇒ g = h`

Pojma monomorfizem in epimorfizem sta uporabna, ker nam omogočata, da *krajšamo* funkcije, ki nastopajo v enačbah. Na
vajah boste reševali naloge, kjer to pride prav.

**Izrek 1:** Naj bosta `f : A → B` in `g : B → C` preslikavi.

1. Kompozicija monomorfizmov je monomorfizem.
2. Kompozicija epimorfizmom je epimorfizem.
3. Če je `g ∘ f` monomorfizem, je `f` monomorfizem.
4. Če je `g ∘ f` epimorfizem, je `g` epimorfizem.

*Dokaz:*

1. Naj bosta `f : A → B` in `g : B → C` monomorfizma. Dokazujemo, da je `g ∘ f` tudi monomorfizem.
   Naj bosta `h, k : D → A` preslikavi, za kateri velja `(g ∘ f) ∘ h  = (g ∘ f) ∘ k`. Dokazujemo `h = k`.
   Ker je kompozicija preslikav asociativna, velja `g ∘ (f ∘ h) = (g ∘ f) ∘ h  = (g ∘ f) ∘ k g ∘ (f ∘ k)`.
   Ker je `g` monomorfizem, ga smemo krajšati na levi, torej dobimo `f ∘ h = f ∘ k`. Ker je `f` monomorfizem,
   ga smemo krajšati in dobimo želeno enakost `h = k`.

2. Dokaz je podoben 1, le vloga leve in desne strani se spremeni (vaja).

3. Dokaz je podoben 4, le vloga leve in desne strani se spremeni (vaja).

4. Naj bosta `f : A → B` in `g : B → C` preslikavi in `g ∘ f` epimorfizem. Dokazujemo, da
   je `g` epimorfizem. Naj bosta `h, k : C → D` taki preslikavi, da velja `h ∘ g = k ∘ g`.
   Dokazujemo, da je `h = k`. Iz `h ∘ g = k ∘ h` sledi `(h ∘ g) ∘ f = (k ∘ g) ∘ f`. Če
   upoštevamo asociativnost kompozicije, dobimo `h ∘ (g ∘ f) = k ∘ (g ∘ f)`. Ker je `g ∘
   f` epimorfizem, ga smemo krajšati na desni, od koder dobimo želeno enakost `h = k`.
□

**Izrek 2:** Za preslikavo `f : A → B` velja

1. `f` je monomorfizem ⇔ `f` je injektivna
2. `f` je epimorfizem ⇔ `f` je surjektivna
3. `f` je izomorfizem ⇔ `f` je bijektivna

*Dokaz:*

1. Če je `f` monomorfizem in `f(x) = f(y)`, tedaj je
   `(f ∘ (u ↦ x)) () = f(x) = f(y) = (f ∘ (u ↦ y)) ()`, torej
   `(u ↦ x) = (u ↦ y) torej x = y`.

   Če je `f` injektivna in `f ∘ g = f ∘ h`, potem je za vsak `x`
   `f(g(x)) = f(h(x))`, torej `g(x) = h(x)` za vsak `x`, torej `g = h`.

2. Če je `f` epimorfizem: obravnavajmo množico

        S = { z ∈ B | ∃ x ∈ A . f(x) = z }

   ter preslikavi `χ_S : B → 2` in `(y ↦ ⊤) : B → 2`. Ker velja
   `χ_S ∘ f = (y ↦ ⊤) ∘ f`, sledi `χ_S = (y ↦ ⊤)`, torej `S = B`,
   kar je surjektivnost.

   Če je `f` surjektivna in `g ∘ f = h ∘ f`: naj bo `y ∈ B`. Obstaja
   `x ∈ A`, da je `f(x) = y`. Torej je

        g(y) = g(f(x)) = h(f(x)) = h(y).

   Torej je `g = h`.

3. Če je `f` izomorfizem, potem

    * `f` je epi, ker je `id_B = f ∘ f⁻¹` epi
    * `f` je mono, ker je `id_A = f⁻¹ ∘ f` mono

   Če je `f` bijektivna, potem je njen inverz `f⁻¹` definiran s predpisom

    `f(y) = ι x ∈ A . f(x) = y`      "tisti x, ki ga f slika v y"

   Dokazati je treba `∃! x . f(x) = y:`

   * `∃ x . f(x) = y` je surjektivnost `f`
   * `∀ x₁ x₂ . f(x₁) = y ∧ f(x₂) = y ⇒ x₁ = x₂` sledi iz injektivnosti `f`
□

\subsection{Retrakcija in prerez}

Spoznajmo še pojem retrakcije in prereza. Na predavanjih bomo s sliko pojasnili, zakaj se tako imenujeta.

**Definicija:** Če sta `f : A → B` in `g : B → A` taki preslikava, da velja `f ∘ g = id_B`, pravimo:

* `f` je **levi** inverz `g`
* `g` je **desni** inverz `f`
* `g` je **prerez** preslikave `f`
* `f` je **retrakcija** iz `B` na `A`

Opomba: retrakcija in prerez *ni* isto kot izomorfizem!

**Izrek 3:** Retrakcija je epimorfizem, prerez je monomorfizem.

*Dokaz:*

Denimo, da velja `f ∘ g = id`, torej je `f` retrakcija in `g` prerez. Ker je `id`
monomorfizem, je po izreku 1 tudi `g` monomorfizem. In ker je `id` epimorfizem, je po
istem izreku `f` monomorfizem. □


\section{Slike in praslike}

\subsection{Izpeljane množice}

Naj bo `f : A → B` preslikava. Tedaj definiramo **izpeljano množico**

    { f(x) | x ∈ A } := { y ∈ B | ∃ x ∈ A . y = f(x) }

ter **izpeljano množico s pogojem**

    { f(x) | x ∈ A | φ(x) } := { y ∈ B | ∃ x ∈ A . φ(x) ∧ y = f(x) }

Običajno se piše izpeljano množico s pogojem kar

    { f(x) | x ∈ A ∧ φ(x) }

*Primer:* Množica vseh kvadratov naravnih števil je izpeljana množica `{ n² | n ∈ N }`.

\subsection{Slike in praslike}

**Definicija:** Naj bo `f : A → B` preslikava:

1. **Praslika** podmnožice `S ⊆ B` je `f^*(S) := { x ∈ A | f(x) ∈ S }`.
2. **Slika** podmnožice `T ⊆ A` je `f_*(T) := { y ∈ B | ∃ x ∈ A . f(x) = y }`.

Kot vidimo, lahko sliko zapišemo tudi kot izpeljano množico

    f_*(T) := { f(x) | x ∈ T }

Običajni zapis za prasliko `f^*(S)` je tudi `f⁻¹(S)`, vendar tega zapisa mi ne bomo uporabljali, ker napačno namiguje, da ima `f` inverz. Boste pa ta zapis videli marsikje drugje, ker so matematiki pravzaprav precej konzervativni in ne marajo sprememb.

Običajni zapis za sliko `f_*(S)` je tudi `f(S)` ali `f[S]`. Predvsem `f(S)` se uporablja v praksi, a tudi tega odsvetujemo. Kako naj pri takem zapisu ločimo med `f(x)` in `f_*({x})`?

**Zaloga vrednosti** je slika domene, torej `f_*(B)`.

\subsection{Slike in praslike kot preslikave višjega reda}

Naj bo `f : A → B`. Tedaj sta tudi `f^*` in `f_*` preslikavi:

* `f^* : P(B) → P(A)` je določena s predpisom `S ↦ { x ∈ A | f(x) ∈ S }`
* `f_* : P(A) → P(B)` je določena s predpisom `T ↦ { f(x) | x ∈ T }`

Še več, tudi "zgoraj zvezdica `^*`" in "spodaj zvezdica `_*`" sta preslikavi

    ^* : B^A → P(A)^P(B)
    _* : B^A → P(B)^P(A)

Ker slikata preslikave v preslikave, pravimo, da sta to preslikavi *višjega reda*. Primer preslikave višjega reda je tudi odvod, ki funkciji priredi njen odvod.

\subsection{Lastnosti slike in praslike}

**Izrek 4:** Naj bo `f : A → B` preslikava:

* praslike so monotone: če je `S ⊆ T ⊆ A`, potem je `f_*(S) ⊆ f_*(T)`
* slike so monotone: če je `X ⊆ Y ⊆ B`, potem je `f^*(X) ⊆ f^*(Y)`.

*Dokaz:* Vaja.

**Izrek 5:** Prasike ohranjajo preseke in unije: za vse `f : A → B` in `S : I → P(B)` velja

* `f^* (⋃_{i ∈ I} S_i) = ⋃_{i ∈ I} f^*(S_i)`
* `f^* (⋂_{i ∈ I} S_i) = ⋂_{i ∈ I} f^*(S_i)`

*Dokaz:* Dokažimo prvo izjavo, druga je zelo podobna, le da `∃` zamenjamo z `∀`.

Dokazujemo `f^* (⋃_{i ∈ I} S_i) ⊆ ⋃_{i ∈ I} f^*(S_i)`.
Naj bo `x ∈ f^* (⋃_{i ∈ I} S_i)` in dokazujemo `x ∈ ⋃_{j ∈ I} f^*(S_j)`.
Ker je `f x ∈ ⋃_{i ∈ I} S_i` obstaja `k ∈ I`, da je `f x ∈ S_k`, torej je
`x ∈ f^* S_k ⊆ ⋃_{i ∈ I} f^*(S_i)`. □

**Izrek 6:** Naj bo `f : A → B` in `T : I → P(A)`. Tedaj je

* `f_* (⋃_{i ∈ I} T_i = ⋃_{i ∈ I} f_*(T_i)`.
* `f_* (∩_{i ∈ I} T_i) ⊆ ⋂_{i ∈ I} f_*(S_i)`.

*Dokaz:* Vaja.

**Naloga:** Iz zgornjih dveh izrekov izpeljite naslednja dejstva:

* `f^*(∅) = ∅`
* `f_*(∅) = ∅`
* `f^*(B) = A`
* `f^*(S ∪ T) = f^*(S) ∪ f^*(T)`
* `f^*(S ∩ T) = f^*(S) ∩ f^*(T)`

Poleg tega imamo za `S ⊆ B`

    f^*(Sᶜ) = (f^*(S))ᶜ.


\chapter{Relacije}

\section{Predikati}

\textbf{Predikat} na množici $A$ opredeljuje kako lastnost elementov množice $A$. Če
je $P$ predikat na $A$ in $x \in A$, potem se je smiselno vprašati, ali $x$
zadošča predikatu $P$. Odgovor je resničnostna vrednost, ki jo označimo s $P(x)$.

\begin{primer}
  Na množici naravnih števil $\NN$ lahko obravnavamo predikat ">je sodo
  število"<. Tako na primer $4$ zadošča predikatu ">je sodo število"<, $7$ pa mu zadošča.
\end{primer}

Predikat $P$ na množici $A$ lahko predstavimo na dva načina:
%
\begin{itemize}
\item kot preslikavo $P : A \to \two$, ki slika $x \in A$ v resničnostno vrednost $P(x)$,
\item kot podmnožico $P \subseteq A$ tistih $x \in A$, za katere velja $P(x)$.
\end{itemize}
%
Oba načina predstavitve sta uporabna, spoznali pa smo že izomorfizem med njima,
saj velja $P(A) \iso \two^A$.

\section{Relacije}

Relacije s večmestni predikati. Se pravi, relacija $R$ opredeljujejo kako
lastnost urejenih večteric kartezičnega produkta $A_1 \times A_2 \times \cdots \times A_n$. Pravimo, da je $R$ \textbf{$n$-člena} ali \textbf{$n$-mestna relacija} na množicah $A_1, …, A_n$.

\begin{primer}
  Na množici točk v ravnini lahko obravnavamo relacijo kolinearnosti.
  To je trimestna relacija: točke $A$, $B$ in $C$ so kolinearne, kadar obstaja
  premica, ki vsebuje vse tri točke.
\end{primer}

Relacijo $R$ na množicah $A_1, \ldots, A_n$ lahko predstavimo na dva načina, podobno
kot predikate:
\begin{itemize}
\item kot preslikavo $R : A_1 \times A_2 \times \cdots \times A_n \to \two$,
\item kot podmnožico $R \subseteq A_1 \times A_2 \times \cdots \times A_n$.
\end{itemize}
%
Bolj običajna je predstavitev s podmnožicami, zato bomo dejstvo, da je $R$
relacija na množicah $A_1, \ldots, A_n$ zapisali kar kot $R \subseteq A_1 \times A_2 \times \cdots \times A_n$.
Za elemente $x_1 \in A_1, \ldots, x_n \in A_n$ dejstvo, da so v relaciji $R$ zapišemo
$R(x_1, \ldots, x_n)$, včasih pa tudi $(x_1, \ldots, x_n) \in R$.

Na množicah $A_1, \ldots, A_n$ lahko vedno definiramo:
%
\begin{itemize}
\item \textbf{prazno relacijo $\emptyset$}: nobeni elementi niso v prazni relaciji,
\item \textbf{univerzalno relacijo $A_1 \times A_2 \times \cdots \times A_n$}: vsi elementi so v univerzalni relaciji.
\end{itemize}
%
Univerzalna relacija se imenuje tudi \textbf{polna relacija}.
%
V praksi so najbolj pogoste \textbf{dvomestna relacije}, se pravi relacije na dveh
množicah, $R \subseteq A \times B$.
V tem primeru pravimo množici $A$ \textbf{domena} in $B$ \textbf{kodomena} relacije $R$.

Pomembna relacija na množici $A$ je \textbf{enakost} ali \textbf{diagonala} na $A$:
%
\begin{equation*}
    \diag[A] \defeq \set{ (x, y) \in A \times A \such x = y }
\end{equation*}
%
Zakaj ji pravimo diagonala?

Izmed dvočelnih relacij so najbolj pogoste relacije, pri katerih se domena in
kodomena ujemata, torej $R \subseteq A \times A$. V tem primeru pravimo, da je $R$ \textbf{relacija na množici $A$}.

Denimo, da je $R \subseteq A \times B$ relacija, $x \in A$ in $y \in B$. Dejstvo, da sta $x$ in $y$ v relaciji $R$ zapišemo na enega od načinov
%
\begin{equation*}
  (x, y) \in R
  \qquad
  R(x, y)
  \qquad
  x R y
\end{equation*}
%
Prvi zapis se uporablja, kadar je $R$ podana kot podmnožica $A \times B$, drugi kadar
podamo~$R$ z logično formulo. Tretji način je tudi pogost, še posebej kadar je
relacija označena s simbolom kot je $=$, $\neq$, $<$, $>$, $\sqsubseteq$, $\sim$ ipd.

Relacijo lahko predstavimo na več načinov, na primer z logično formulo, z resničnostno tabelo, ali z usmerjenim grafom.
%
Z grafom predstavimo $R \subseteq A \times A$ tako, da za vozlišča grafa vzamemo
elemente množice $A$, nato pa narišemo puščico od $x$ do $y$, kadar velja $x R y$.


\section{Osnovne lastnosti relacij}

Relacije, ki so pomembne v matematični praksi imajo pogosto lastnosti, ki jih poimenujemo. Za relacijo $R \subseteq A \times A$ pravimo da je:
%
\begin{itemize}
  \item \textbf{refleksivna:} $\all{x \in A} x R x$,
  \item \textbf{simetrična:} $\all{x, y \in A} x R y \lthen y R x$,
  \item \textbf{antisimetrična:} $\all{x, y \in A} x R y \land y R x \lthen x = y$,
  \item \textbf{tranzitivna:} $\all{x, y, z \in A} x R y \land y R z \lthen x R z$,
  \item \textbf{irefleksivna:} $\all{x \in A} \lnot (x R x)$,
  \item \textbf{asimetrična:} $\all{x, y \in A} x R y \lthen \lnot (y R x)$,
  \item \textbf{sovisna:} $\all{x, y \in A} x \neq y \lthen x R y \lor y R x$,
  \item \textbf{strogo sovisna:} $\all{x, y \in A} x R y \lor y R x$.
\end{itemize}
%

\begin{naloga}
  Kako iz usmerjenga grafa relacije razberemo refleksivnost in simetričnost? Kaj pa ostale lastnosti?
\end{naloga}

\section{Operacije na relacijah}

\subsection{Unija, presek in komplement relacij}

Ker so relacije pravzaprav podmnožice, lahko na njih uporabljamo operacije unija $\cup$,
presek $\cap$ in komplement $\compl{\Box}$. Denimo, da sta $R, S \subseteq A \times B$ relaciji. Tedaj velja:
%
\begin{align*}
  x (R \cup S) y &\iff x R y \lor x S y, \\
  x (R \cap S) y &\iff x R y \land x S y, \\
  x \compl{R} y &\iff \lnot (x R y).
\end{align*}

\begin{primer}
  Za relacije enakosti in urejenost na realnih števlih velja:
  %
  \begin{itemize}
  \item Komplement relacije enakosti $=$ je relacija neenakosti $\neq$.
  \item Unija relacij $<$ in $>$ na realnih številih je relacija $\neq$.
  \item Presek relacij $\leq$ in $\geq$ na realnih številih je relacija $=$.
  \end{itemize}
\end{primer}


\subsection{Transponirana relacija}

Dvojiške relacije lahko tudi \textbf{transponiramo}. Transponiranka relacije $R \subseteq A \times B$ je relacija $\transpose{R} \subseteq B \times A$, definirana s predpisom
%
\begin{equation*}
    y \transpose{R} x \defiff x R y
\end{equation*}
%
ali ekvivalentno
%
\begin{equation*}
  \transpose{R} \defeq \set{ (y, x) \in B \times A \such x R y }.
\end{equation*}
%
Očitno velja $\transpose{(\transpose{R})} = R$, torej je transponiranje \emph{involucija}.

\begin{primer}
  Transpozicija relacije $<$ na realnih številih $\RR$ je relacija $>$ na $\RR$.
  Komplement relacije $<$ na $\RR$ je relacija $\geq$ na $\RR$.
\end{primer}

\subsection{Kompozitum relacij}

Nadalje definiramo \textbf{kompozitum} relacij $R \subseteq A \times B$ in $S \subseteq B \times C$ kot relacijo $S \circ R \subseteq A \times C$, s predpisom
%
\begin{equation*}
    x (S \circ R) z \defiff \some{y \in B} x R y \land y S z
\end{equation*}
%a
ali ekvivalentno
%
\begin{equation*}
  S \circ R \defeq
  \set{ (x, z) \in A \times C \such \some{y \in B} (x,y) \in R \land (y,z) \in S }.
\end{equation*}
%
Se pravi, da sta $x \in A$ in $z \in C$ v relaciji $S \circ R$, če sta preko $S$ in $R$
povezana s kakim elementom $y \in B$.

\begin{primer}
  Kompozitum relacij ">$x$ je otrok od $y$"< in ">$z$ je mati od $y$"< je relacija
  ">$z$ je babica od $x$"<.
\end{primer}

\begin{izrek}
  Komponiranje relacij je asociativno in diagonala je enota.
\end{izrek}

\begin{naloga}
  Zgornji izrek zapiši bolj natančno, da bo razvidno, kaj so domene in kodomene relacij.
\end{naloga}

\begin{dokaz}
  Najprej dokažimo asociativnost kompozicije.
  %
  Naj bo $R \subseteq A \times B$, $S \subseteq B \times C$ in $T \subseteq C \times D$ ter $a \in A$ in $d \in D$. Tedaj velja
  %
  \begin{align}
    a (T \circ (S \circ R)) d &\iff  \notag \\
    \some{c \in C} a (S \circ R) c \land c T d &\iff \notag \\
    \some{c \in C} (\some{b \in B} a R b \land b R c) \land c T d \label{eq:comp-1}
  \end{align}
  %
  in
  %
  \begin{align}
    a ((T \circ S) \circ R) d &\iff \notag \\
    \some{b \in B} a R b \land b (T \circ S) d &\iff \notag \\
    \some{b \in B} a R b \land (\some{c \in C} b S c \land c T d) \label{eq:comp-2}
  \end{align}
  %
  Torej je treba dokazati ekvivalenco izjav~\eqref{eq:comp-1} in~\eqref{eq:comp-2}, kar prepuščamo za vajo. Naj namignemo, da je treba pri dokazovanju ekvivalence uporabiti \emph{Frobeniuseva pravilo}
  %
  \begin{equation*}
    (\some{x \in X} p \land q(x)) \liff p \land \some{x \in X} q(x).
  \end{equation*}
  %
  V pravilu je $p$ formula, v kateri $x$ ne nastopa kot prosta spremenljivka.

  Dokažimo še, da je diagonala enota za kompozicijo: naj bo $R \subseteq A \times B$ ter $x \in A$ in $y \in B$. Tedaj velja
  %
  \begin{align*}
    x (\diag[B] \circ R) y &\iff \\
    \some{z \in B} x R y \land y \diag[B] z  &\iff \\
    \some{z \in B} x R y \land y = z &\iff \\
    x R y
  \end{align*}
  %
  V zadnjem koraku smo uporabili ekvivalenco $(\some{u \in U} u = v \land P(v)) \liff P(v)$. Podobno dokažemo, da je diagonala desna enota.
\end{dokaz}

Kompozitum relacij ima torej podobne lastnosti kot kompozitum funkcij.

\subsection{Potenca relacije}

Za $n \in \NN$ definiramo \textbf{$n$-to potenco} relacije $R \subseteq A \times A$ kot relacijo $R^n \subseteq A \times A$ takole:
%
\begin{equation*}
    x R^n y \defiff
    \some{z_0, \ldots, z_n \in A}
    z_0 = x \land z_n = y \land \all{i \in {0, \ldots, n-1}} z_i R z_{i+1}.
\end{equation*}
%
To je precej nečitljiva formula. Bolj razumljiva definicija je potenca kot $n$-kratni kompozitum relacije $R$ same s sabo:
%
\begin{equation*}
    R^n \defeq \underbrace{R \circ \cdots \circ R}_n
\end{equation*}
%
kjer se desni $R$ ponovi $n$-krat. Kaj dobimo, ko za $n$ vstavimo $0$? Enoto za kompozitum:
%
\begin{equation*}
    R^0 = \diag[A].
\end{equation*}

\section{Funkcijske relacije}

Funkcijo $f : A \to B$ smo definirali kot \emph{prirejanje} med elementi $A$ in $B$. A
kaj pravzaprav je ">prierjanje"<? Je to funkcijski predpis, program, kaj drugega?
Sedaj lahko povemo natančno: prirejanje, s katerim je podana funkcija, je
\emph{relacija} med elementi domene in kodomene.

\begin{definicija}
  Naj bo $f : A \to B$ funkcija. \textbf{Graf} funkcije $f$ je relacija
  $\Gamma_f \subseteq A \times B$, definirana s predpisom
  %
  \begin{equation*}
    x \,\Gamma_{\!f}\, y \liff f(x) = y
  \end{equation*}
  %
  ali ekvivalentno
  %
  \begin{equation*}
    \Gamma_{\!f} \defeq \set{ (x, y) \in A \times B \such f(x) = y }.
  \end{equation*}
\end{definicija}

Skratka, graf funkcije ni nič drugega kot njeno prirejanje.
%
Sedaj pa se vprašajmo: kakšnim pogojem mora zadoščati relacija $R \subseteq A \times B$, da je prirejanje za neko funkcijo? Odgovor poznamo: biti mora enolična in celovita.

\begin{definicija}
  Relacija $R \subseteq A \times B$ je \textbf{funkcijska relacija}, če je
  %
  \begin{itemize}
  \item \textbf{celovita:} $\all{x \in A} \some{y \in B} x R y$ in
  \item \textbf{enolična:} $\all{x \in A} \all{y_1, y_2 \in B} x R y_1 \land x R y_2 \lthen y_1 = y_2$.
  \end{itemize}
  %
  Ekvivalentno oba pogoja skupaj zapišemo: $\all{x \in A} \exactlyone{y \in B} x R y$.
\end{definicija}

Graf $\Gamma_{\!f} \subseteq A \times B$ funkcije $f : A \to B$ je vedno funkcijska relacija.
%
Funkcijska relacija $R \subseteq A \times B$ določa preslikavo $\phi_R : A \to B$ definirano s predpisom
%
\begin{equation*}
  \phi_R : x \mapsto \descr{y \in B} x R y.
\end{equation*}
%
Če iz funkcije $f : A \to B$ tvorimo njen graf $\Gamma_{\!f}$, nato pa iz njega funckijo
$\phi_{\Gamma_{\!f}} : A \to B$ dobimo nazaj prvotno funkcijo $f$. Obratno, če je $R$ funkcijska relacija, tedaj je $\Gamma_{\phi_R}$ enaka $R$. Torej imamo izomorfizem
%
\begin{equation*}
  B^A \iso \set{ R \in \pow{A \times B} \such \all{x \in A} \exactlyone{y \in B} x R y }.
\end{equation*}
%

\begin{izjava}
  Kompozitum funkcij se ujema s kompozitumom relacij:
  $\Gamma_{g \circ f} = \Gamma_g \circ \Gamma_{\!f}$.
\end{izjava}

\begin{dokaz}
  Dokaz prepustimo za vajo, še prej pa morate izjavo zapisati bolj natančno: od
  kod in kam slikata preslikavi $f$ in $g$, kaj pomeni kompozitum na levi in kaj
  na desni?
\end{dokaz}


\section{Ovojnice relacij}

Pogosto imamo opravka z relacijo $R$, ki nima želene lastnosti (na primer ni
tranzitivna) mi pa želimo relacijo, ki to lastnost ima. Ali lahko $R$ kako
spremenimo, da bo imela želeno lastnost? Če to lahko naredimo na več načinov,
ali se eden od njih odlikuje?

\begin{definicija}
  Naj bo $R \subseteq A \times A$ relacija. Tedaj pravimo, da je relacija $T \subseteq A \times A$ \textbf{tranzitivna ovojnica} relacije $R$, če velja:
  %
  \begin{enumerate}
  \item $T$ je tranzitivna,
  \item $R \subseteq T$ in
  \item če je $S \subseteq A \times A$ tranzitivna in velja $R \subseteq S$, tedaj je $T \subseteq S$.
  \end{enumerate}
\end{definicija}

Povedano drugače: tranzitivna ovojnica relacije $R$ je \textsf{najmanjša} tranzitivna
relacija, ki vsebuje $R$. Zaenkrat ne vemo, ali ima vsaka relacija tranzitivno
ovojnico.

Izraz ">ovojnica"< uporabljamo, ker si lahko mislimo, da smo relacijo ovili
s tranzitivno relacijo tako, da se ji slednja čim bolj prilega. Namesto ">ovojnica"<
rečemo tudi \textbf{ogrinjača} ali \textbf{zaprtje}.

Poleg tranzitivne ovojnice lahko definiramo tudi druge ovojnice:
%
\begin{itemize}
  \item \textbf{Refleksivna ovojnica} relacije $R \subseteq A \times A$ je najmanjša refleksivna relacija, ki vsebuje $R$.
  \item \textbf{Simetrična ovojnica} relacije $R \subseteq A \times A$ je najmanjša simetrična relacija, ki vsebuje $R$.
  \item \textbf{Refleksivna tranzitivna ovojnica} relacije $R \subseteq A \times A$ je najmanjša refleksivna in tranzitivna relacija, ki vsebuje $R$.
\end{itemize}
%
Ali take ovojnice sploh obstajajo? Obravnavajmo le tranzitivne ovojnice, saj so
ostali dokazi zelo podobni. Ključno pri dokazu obstoja tranzitivne ovojnice je
naslednje dejstvo.

\begin{lema}
  Naj bo $A$ množica in $R : I \to P(A \times A)$ družina relacij na $A$. Če za
  vsak $i \in I$ velja, da je $R_i$ tranzitivna relacija, potem je tudi presek $\bigcap R$ tranzitivna relacija.
\end{lema}

\begin{dokaz}
  Iz definicije preseka družine množic (relacije so le posebne množice) sledi
  %
  \begin{equation*}
  x (\textstyle\bigcap R) y \liff \all{i \in I} x R_i y.
  \end{equation*}
  %
  Dokažimo, da je $\textstyle\bigcap R$ tranzitivna.
  Naj bodo $x, y, z \in A$ in denimo, da velja
  $x (\textstyle\bigcap R) y$ in $y (\textstyle\bigcap R) z$, kar je ekvivalentno
  %
  \begin{equation*}
    \all{i \in I} x R_i y
    \iinn
    \all{j \in I} y R_j z.
  \end{equation*}
  %
  Dokazati moramo $x (\textstyle\bigcap R) z$, kar je ekvivalentno
  %
  $\all{k \in I} x R_k z$.
  %
  Naj bo torej $k \in I$, dokazujemo $x R_k z$. Uporabimo $\all{i \in I} x R_i y$ pri $i = k$ in dobimo $x R_k y$.
  %
  Uporabimo $\all{j \in I} y R_j z$ pri $j = k$ in dobimo $y R_k z$.
  %
  Po predpostavki je $R_k$ tranzitivna relacija, torej velja $x R_k z$.
\end{dokaz}

\begin{izrek}
  Vsaka relacija ima enolično tranzitivno ovojnico.
\end{izrek}

\begin{dokaz}
  Najprej premislimo, da ima $R$ največ eno tranzitivno ovojnico: če sta
  $S$ in $T$ obe tranzitvni ovojnici $R$, potem iz definicije tranzitivne ovojnice
  sledi $S \subseteq T$ in $T \subseteq S$, torej velja $S = T$.

  Sedaj pokažimo, da $R$ ima tranzitivno ovojnico. Naj bo $R \subseteq A \times A$. Definirajmo množico relacij
  %
  \begin{equation*}
    D \defeq \set{ S \subseteq A \times A \such \text{$R \subseteq S$ in $S$ je tranzitivna} }.
  \end{equation*}
  %
  Trdimo, da je $\textstyle\bigcap D$ tranzitivna ovojnica relacije $R$.
  %
  Iz prejšnje leme sledi, da je $\textstyle\bigcap D$ tranzitivna.
  %
  Ker velja $R \subseteq S$ za vsak $S \in D$, seveda sledi $R \subseteq \textstyle\bigcap D$.
  %
  Če je $R \subseteq T$ in $T \subseteq A \times A$ tranzitivna relacija, tedaj velja $T \in D$, torej je $T \subseteq \textstyle\bigcap D$.
\end{dokaz}


Po istem kopitu pokažemo, da ima vsaka relacija $R \subseteq A \times A$ tudi ostale
ovojnice. Je pa zgornji izrek neroden, ker nam dokaz ne poda uporabnega opisa
tranzitivne ovojnice. Povejmo, kako lahko razne ovojnice opišemo bolj
eksplicitno:
%
\begin{enumerate}
\item Refleksivna ovojnica relacije $R$ je relacija $R \cup \diag[A]$, se pravi, da
  relaciji $R$ dodamo še diagonalo.
\item 
  Simetrična ovojnica relacije $R$ je relacija $R \cup \transpose{R}$.
\item
  Tranzitivna ovojnica relacije $R$ je relacija $R^{+} \defeq \bigcup_{n \geq 1} R^n$, se pravi
  %
  \begin{equation*}
    R^{+} \defeq R \cup (R \circ R) \cup (R \circ R \circ R) \cup \cdots
  \end{equation*}
\item
  Refleksivna tranzitivna ovojnica relacije $R$ je relacija $R^{*} \defeq \bigcup_{n \geq 0} R^n$, se pravi
  %
  \begin{equation*}
  R^{*} \defeq \diag[A] \cup R \cup (R \circ R) \cup (R \circ R \circ R) \cup \cdots
  \end{equation*}
\end{enumerate}

\chapter{Ekvivalenčne relacije}

\section{Ekvivalenčne relacije}

\begin{definicija}
  Relacija $R \subseteq A \times A$ je \textbf{ekvivalenčna relacija}, če je refleksivna, tranzitivna in simetrična. Kadar velja $x \rel{R} y$, pravimo, da sta $x$ in $y$ \textbf{ekvivalentna} glede na~$R$.
\end{definicija}

\begin{opomba}
  Kdor reče ">ekvivalentna relacija"<, je noob. Kdor reče, da sta ">$x$ in $y$
  ekvivalenčna"<, je rookie.
\end{opomba}

Ekvivalenčne relacije se običajno označuje s simboli, ki so podobni znaku za enakost:
$\equiv$, $\sim$, $\simeq$, $\cong$.

\begin{zgled}
  Primeri ekvivalenčnih relacij:
  \begin{enumerate}
    \item Relacija ">vzporednost"< med premicami v ravnini.
    \item Relacija ">skladnost"< med trikotniki v ravnini.
    \item Relacija ">podobnost"< med trikotniki v ravnini.
    \item Relacija ">isti ostanek pri deljenju s 7"< na množici $\NN$.
    \item Prazna relacija $\emptyset \subseteq A \times A$ je ekvivalenčna le v primeru, da je $A = \emptyset$.
    \item Polna relacija $A \times A$ je ekvivalenčna.
    \item Diagonala (enakost) je ekvivalenčna relacija.
  \end{enumerate}
\end{zgled}

\subsection{Ekvivalenčna relacija porojena s preslikavo}

Posebej pomemben je primer ekvivalenčne relacije \textbf{porojene (ali inducirane) s preslikavo}:
naj bo $f : A \to B$ preslikava in definirajmo relacijo $\sim_f$ na $A$ s predpisom
%
\begin{equation*}
  x \sim_f y \liff f(x) = f(y)
\end{equation*}
%
Tedaj je $\sim_f$ ekvivalenčna relacija:
%
\begin{itemize}
\item refleksivnost: $x \sim_f x$ velja, ker velja $f(x) = f(x)$,
\item tranzitivnost: če je $x \sim_f y$ in $y \sim_f z$, potem je $f(x) = f(y)$ in $f(y) = f(z)$, torej $f(x) = f(z)$ in $x \sim_f z$,
\item simetričnost: če je $x \sim_f y$, potem je $f(x) = f(y)$, torej $f(y) = f(x)$ in $y \sim_f x$.
\end{itemize}
%
Ali je vsaka ekvivalenčna relacija porojena z neko preslikavo?

\begin{zgled}
  Premici sta vzporedni natanko tedaj, ko imata enaka smerna vektorja. Če je
  torej $P$ množica vseh premic, $\RR^2$ množica vektorjev v ravnini, in $s : P \to \RR^2$
  preslikava, ki premici $P$ priredi njen enotski smerni vektor, ki leži v zgornji polravnini ali
  na pozitivnem delu osi $x$, tedaj velja
  \begin{equation*}
    p \parallel q \liff s(p) = s(q).
  \end{equation*}
  %
  Torej je vzporednost porojena s preslikavo $s$.
\end{zgled}

\section{Ekvivalenčni razredi in kvocientne množice}

\begin{definicija}
  Naj bo $E \subseteq A \times A$ ekvivalenčna relacija. \textbf{Ekvivalenčni razred} elementa $x \in A$ je množica
  $[x]_E \defeq \set{ y \in A \such x \rel{E} y }$. Z besedami: ekvivalenčni razred~$x$ je množica vseh elementov, ki so mu
  ekvivalentni.
\end{definicija}

\begin{opomba}
  Kdor reče ">ekvivalentni razred"<, je newbie.
  Če pustimo šalo ob strani: ekvivalenčni razredi se tako imenujejo zaradi zgodovinskih razlogov. Beseda ">razred"< nakazuje dejstvo, da so imajo elementi ekvivalenčnega razredi vsi nekaj skupnega (">delavski razred"<, ">Tina Maze je razred zase"<) in ne, da niso množice (saj očitno so).
\end{opomba}

\begin{definicija}
  Naj bo $E \subseteq A \times A$ ekvivalenčna relacija. \textbf{Kvocientna ali faktorska množica} ali \textbf{kvocient} $A/E$ je množica vseh ekvivalenčnih razredov:
  %
  \begin{equation*}
    A/E \defeq \set{ \xi \in \pow{A} \such \some{x \in A} \xi = [x]_E }.
  \end{equation*}
  %
  Z izpeljanimi množicami lahko to zapišemo bolj razumljivo
  % 
  \begin{equation*}
    A/E = \set{ [x]_E \such x \in A }.
  \end{equation*}
  %
  \textbf{Kanonična kvocientna preslikava} $q_E : A \to A/E$ je preslikava, ki vsakemu elementu
  priredi njegov ekvivalenčni razred: $q_E(x) \defeq [x]_E$.
\end{definicija}

\begin{izrek}
  Vsaka ekvivalenčna relacija je porojena z neko preslikavo.
\end{izrek}

\begin{proof}
  Dokažimo, da je ekvivalenčna relacija porojena s svojo kvocientno preslikavo.

  Naj bo $E$ ekvivalenčna relacija na $A$. Najprej ugotovimo naslednje: za vse $x,
  y \in A$ velja
  %
  \begin{equation*}
    x \rel{E} y \liff [x]_E = [y]_E.
  \end{equation*}

  ($\lthen$) Če je $x \rel{E} y$ potem je $[x]_E \subseteq [y]_E$, ker iz $z \rel{E} x$ in $x \rel{E} y$ sledi $z \rel{E} y$.
  Podobno dokažemo $[y]_E \subseteq [x]_E$.

  ($\Leftarrow$) Če je $[x]_E = [y]_E$ potem je $y \in [y]_E = [x]_E$, torej po definiciji $[x]_E$
  dobimo $x \rel{E} y$.

  Sedaj izrek sledi zlahka: $q_E(x) = q_E(y) \liff [x]_E = [y]_E \liff x \rel{E} y$.
\end{proof}


\subsection{Razdelitev množice}

\begin{definicija}
  \textbf{Razdelitev} ali \textbf{particija} množice $A$ je množica nepraznih, paroma
  disjunktnih množic, ki tvorijo pokritje $A$ (kar pomeni, da je $A$ enaka njihovi uniji). Se
  pravi, to je množica $S \subseteq \pow{A}$, za katero velja:
  %
  \begin{enumerate}
  \item Elementi razdelitve so neprazni: $\all{B \in S} B \neq \emptyset$.
  \item Vsaka dva elementa razdelitve sta bodisi enaka bodisi disjunktna:
    %
    \begin{equation*}
      \all{B, C \in S} B = C \lor B \cap C = \emptyset.
    \end{equation*}
  \item Elementi razdelitve tvorijo pokritje $A$, se pravi $A = \bigcup S$.
  \end{enumerate}
\end{definicija}

\begin{zgled}
  Primeri razdelitev:
  %
  \begin{enumerate}
  \item Navpične premice tvorijo razdelitev ravnine.
  \item Množici sodih in lihih števil tvorita razdelitev naravnih števil.
  \item Množica $\set{\set{1,2}, \set{3,5}, \set{4,6,7}}$ tvori razdelitev $\set{1,2,3,4,5,6,7}$.
  \item Množica $\set{\set{1,2,3,4,5,6,7}}$ tvori razdelitev $\set{1,2,3,4,5,6,7}$.
  \end{enumerate}
\end{zgled}

\begin{izrek}
  Naj bo $E \subseteq A \times A$ ekvivalenčna relacija. Njeni ekvivalenčni razredi tvorijo
  razdelitev množice $A$.
\end{izrek}

\begin{proof}
  Dokažimo, da so ekvivalenčni razredi neprazni, paroma disjunktni in da tvorijo pokritje.

  Naj bo $\xi \in \pow{A}$ ekvivalenčni razred za $E$. Tedaj obstaja $x \in A$, da je $\xi = [x]_E$,
  torej je $x \in \xi$ in zato $\xi \neq \emptyset$.

  Naj bosta $\zeta, \xi \in \pow{A}$. Dokazali bomo $\zeta \cap \xi \neq \emptyset \lthen \zeta = \xi$. Če je $x \in \zeta \cap \xi$, potem velja $\zeta \subseteq \xi$ ker: naj bo $y \in \zeta$, tedaj je $y \rel{E} x$ in ker je $x \in \xi$ velja $y \in \xi$. Simetrično dokažemo $\xi \subseteq \zeta$.

  Očitno je unija vseh ekvivalenčnih razredov podmnožica $A$, saj je vsak ekvivalenčni razred podmnožica $A$. Zagotovo
  pa je vsak $x \in A$ v kakem ekvivalenčnem razredu, namreč $x \in [x]_E$.
\end{proof}

Torej vsaka ekvivalenčna relacija na $A$ določa razdelitev množice $A$, namreč na
ekvivalenčne razrede. Velja pa tudi obrat: vsaka razdelitev $S \subseteq \pow{A}$ določa ekvivalenčno
relacijo na $A$, namreč $\simeq_S$ definiran s predpisom
\begin{equation*}
    x \simeq_S y \defiff \some{B \in S} x \in B \land y \in B.
\end{equation*}
%
Z besedami: $x$ in $y$ sta ekvivalentna, kadar sta v istem elementu razdelitve. Pravzaprav
smo ugotovili, da imamo izomorfizem množic
%
\begin{equation*}
  \set{ E \subseteq A \times A \such \text{$E$ je ekvivalenčna relacija na $A$} } \iso
  \set{ S \subseteq \pow{A} \such \text{$S$ je razdelitev $A$} }.
\end{equation*}
%
V eno smer izomorfizem ekvivalenčni relaciji $E$ priredi njeno razdelitev, v drugo pa razdelitvi priredimo ekvivalenčno
relacijo, kakor smo to opisali zgoraj. (Premislite, da sta ti preslikavi inverza.)


\subsection{Prerezi kvocientne preslikave in aksiom izbire}

Ekvivalenčni razred je natanko določen že z enim od svojih elementov, zato pogosto želimo
namesto ekvivalenčnih razredov navesti le njihove predstavnike.

\begin{definicija}
  Naj bo $E$ ekvivalenčna relacija na $A$. Množico $C \subseteq A$, ki vsak
  ekvivalenčni razred relacije $E$ seka natanko enkrat, imenujemo \textbf{izbor predstavnikov}
  (ekvivalenčnih razredov) za relacijo $E$.
\end{definicija}

Izbor predstavnikov $C \subseteq A$ za $E$ določa preslikavo $c : A/E \to A$, ki priredi
ekvivalenčnemu razredu $\xi$ tisti $x \in \xi$, ki je element $C$:
%
\begin{align*}
  c &: A/E \to A \\
  c &: \xi \mapsto \descr{x \in \xi} x \in C
\end{align*}
%
Preslikava $c : A/E \to A$ je \emph{prerez} kvocientne preslikave $q_E : A \to A/E$.

\begin{izjava}
  Če je $s : A/E \to A$ prerez kvocientne preslikave $q_E : A \to A/E$, potem je
  njegova slika $\img{s}(A/E) = \set{ c(\xi) \such \xi \in A/E }$ izbor predstavnikov za $E$.
\end{izjava}

\begin{proof}
  Vaja.
\end{proof}

Ker izbor predstavnikov in prerez kvocientne preslikave določata drug drugega, včasih tudi
prerez imenujemo ">izbor predstavnikov"<.

\begin{zgled}
  Definirajmo $\sim$ na množici celih števil $Z$ s predpisom
  %
  \begin{equation*}
    a \sim b \defiff 7 \mathrel{|} a - b.
  \end{equation*}
  %
  Torej sta števili $a$ in $b$ ekvivalentni, če dasta enak ostanek pri deljenju s~$7$,
  na primer $13 \sim 20$ in $\lnot (13 \sim 15)$.
  %
  Ekvivalenčni razred števila $a$ dobimo tako, da $a$ prištejemo vse večkratnike števila $7$:
  %
  \begin{equation*}
    [a]_{\sim} = \set{ a + 7 \cdot k \such k \in \ZZ }.
  \end{equation*}
  %
  Na primer,
  \begin{equation*}
    [13]_\sim = \set{ 7 \cdot k + 13 \such k \in \ZZ }
           = \set{ \ldots, -22, -15, -8, -1, 6, 13, 20, 27, 34, 41, \ldots}.
  \end{equation*}
  %
  Koliko pa je ekvivalenčnih razredov? Toliko, kot je ostankov pri deljenju s~$7$, torej sedem. Množica
  $\set{0, 1, 2, 3, 4, 5, 6}$ je izbor predstavnikov za $\sim$, saj je vsako celo število ekvivalentno natanko enemu od
  teh števil po modulu $7$.
  %
  Ni pa to edini izbor! Tudi $\set{0, 1, 2, 3, 4, 5, 13}$ je izbor in prav tako $\set{-7, -6, -5, -4, -3, -2, -1}$.
\end{zgled}

Ali ima vsaka ekvivalenčna relacija izbor predstavnikov? Da to vprašanje ni tako
enostavno, kot se zdi na prvi pogled, doma premislite o naslednji nalogi.

\begin{vaja}
  Na množici realnih števil $\RR$ definiramo relacijo $E$ s predpisom
  %
  \begin{equation*}
    x \rel{E} y  \defiff  x - y \in \QQ.
  \end{equation*}
  %
  Se pravi, da sta števili ekvivalentni, če je njuna razlika racionalno število. Podajte kak
  izbor predstavnikov za $E$.
\end{vaja}

\begin{izrek}
  Naslednje izjave so ekvivalentne:
  %
  \begin{enumerate}
  \item Vsaka surjektivna preslikava ima desni inverz (prerez).
  \item Vsaka ekvivalenčna relacija ima izbor predstavnikov.
  \item Vsaka družina nepraznih množic ima funkcijo izbire.
  \item Produkt družine nepraznih množic je neprazen.
  \end{enumerate}
\end{izrek}

\begin{proof}
  ($1 \lthen 2$):
  %
  Naj bo $E \subseteq A \times A$ ekvivalenčna relacija na $A$. Tedaj je $q_E : A \to A/E$
  surjektivna, zato ima po predpostavki (1) prerez, ki določa izbor predstavnikov.

  ($2 \lthen 3$):
  %
  Naj bo $A : I \to \Set$ družina nepraznih množic. Naj bo $\sim$ ekvivalenčna relacija
  na koproduktu $K \defeq \sum_{i \in I} A_i$, porojena s prvo projekcijo $\fst : S \to I$, t.j.,
  %
  \begin{equation*}
    \inj[i](x) \sim \inj[j](y) \liff i = j.
  \end{equation*}
  %
  Po predpostavki (2) obstaja izbor predstavnikov za $\sim$, se pravi taka množica $C \subseteq K$, da
  za vsak $u \in K$ obstaja natanko en $v \in C$, da je $\fst(u) = \fst(v)$. Definirajmo $f : I \to
  \bigcup A$ s predpisom
  %
  \begin{equation*}
    f(i) \defeq \descr{x \in A_i} \inj[i](x) \in C
  \end{equation*}
  %
  Očitno je $f$ funkcija izbire za družino $A$, če je izraz na desni veljaven:
  %
  \begin{itemize}
  \item Enoličnost: iz $\inj[i](x) \in C$ in $\inj[i](y) \in C$ sledi $\inj[i](x) = \inj[j](y)$.
  \item Celovitost: ker je $A_i$ neprazna, obstaja $z \in A_i$, torej obstaja $v \in C$, da je
    $i = \fst(\inj[i](z)) = \fst(v)$, in je potemtakem $\snd(v) \in A_i$ element, za katerega velja
    $\inj[i](\snd(v)) \in C$.
  \end{itemize}

  ($3 \lthen 4$):
  %
  Elementi produkta so funkcije izbire, zato je produkt res neprazen, če obstaja
  kaka funkcija izbire.

  ($4 \lthen 1$):
  %
  Naj bo $f : X \to Y$ surjektivna. Definirajmo družino $A : Y \to \Set$ s
  predpisom $A_y = \invimg{f}(\set{y})$. Ker je $f$ surjektivna, je $A$ družina nepraznih
  množic. Po predpostavki (4) je produkt te družine neprazen, torej vsebuje neko
  funkcijo izbire $c : Y \to \bigcup A$, se pravi, da je $f(c(y)) = y$ za vsak $y \in Y$.
  Opazimo še, da je $\bigcup A = Y$, torej je $c$ prerez $f$.
\end{proof}

Izbor predstavnikov je torej ekvivalenten še nekaterim drugim trditvam. Pa te veljajo? Za
to potrebujemo aksiom.

\begin{aksiom}[Aksiom izbire]
  Vsaka družina nepraznih množic ima funkcijo izbire.
\end{aksiom}

Se pravi, če je $A : I \to \Set$ taka družina množica, da za vsak $i \in I$ velja $A_i \neq \emptyset$,
tedaj obstaja $f : I \to \bigcup A$, za katerega je $f(i) \in A_i$ za vse $i \in I$.
%
O aksiomu izbire bomo še govorili.


\subsection{Univerzalna lastnost kvocientne množice}

Naj bo $E$ ekvivalenčna relacija na $A$ in $B$ množica. Pogosto želimo definirati
preslikavo
%
\begin{equation*}
    f : A/E \to B
\end{equation*}
%
s pomočjo preslikave $A \to B$. Kdaj lahko to naredimo?

\begin{izrek}
  Naj bo $E$ ekvivalenčna relacija na $A$ in $g : A \to B$ preslikava, ki je \emph{skladna} z $E$, kar pomeni da $g$
  slika ekvivalentne elemente v enake: $\all{x, y \in A} x \rel{E} y \lthen g(x) = g(y)$. Tedaj obstaja natanko ena
  preslikava $f : A/E \to B$, da je $f([x]_E) = g(x)$ za vse $x \in A$, ali drugače povedano, $f \circ q_E = g$.
\end{izrek}

\begin{proof}
  Dokažimo najprej, da imamo največ eno tako preslikavo. Denimo da za $f_1 : A/E \to B$ in
  $f_2 : A/E \to B$ velja $f_1 \circ q_E = f_2 \circ q_E$. Ker je $q_E$ surjektivna, je epi in jo smemo
  krajšati na desni, od koder res sledi $f_1 = f_2$.

  Sedaj dokažimo, da $f$ obstaja. V ta namen naj bo $\phi \subseteq A/E \times B$ relacija
  %
  \begin{equation*}
    \phi(\xi, y) \defiff \some{x \in A} x \in \xi \land g(x) = y.
  \end{equation*}
  %
  Trdimo, da je $\phi$ funkcijska relacija:
  %
  \begin{itemize}
  \item
    Enoličnost: če je $\phi(\xi, y_1)$ in $\phi(\xi, y_2)$, potem obstajata $x_1, x_2 \in \xi$, da je $g(x_1) = y_1$
    in $g(x_2) = y_2$. Ker pa velja $x_1 \rel{E} x_2$ in je $g$ skladna z $E$, sledi $y_1 = g(x1) = g(x_2) = y_2$.

  \item  Celovitost: naj bo $\xi \in A/E$. Tedaj obstaja $x \in \xi$. Očitno velja $g(\xi, g(x))$.
  \end{itemize}
  %
  Naj bo $f : A/E \to B$ preslikava, ki je določena s funkcijsko relacijo $\phi$. Za $x \in A$
  velja $\phi([x]_E, f([x]_E))$, od tod pa iz definicije $\phi$ sledi tudi $g(x) = f([x]_E)$.
\end{proof}

\begin{opomba}
  Profesorja prosite, da pojasni ali sem zapiše, zakaj se reče ">univerzalna lastnost"< kvocientne množice.
\end{opomba}


\section{Kanonična razčlenitev preslikave}

Naj bo $f : A \to B$ preslikava. Naj bo $\sim_f$ ekvivalenčna relacija na $A$, ki jo porodi
$f$, in $q_f : A \to A/E$ kanonična kvocientna preslikava (morali bi jo pisati $q_{\sim_f}$,
kar je nečitljivo). Naj bo $i : \img{f}(A) \to B$ kanonična inkluzija slike $f$ v kodomeno.
Preslikava $f : A \to \img{f}(A)$ je skladna s $\sim_f$, zato obstaja (natanko ena) preslikava
$b_f : A/f \to \img{f}(A)$, da velja $b_f([x]_\sim) = f(x)$. Trdimo:
%
\begin{enumerate}
\item $f = i_f \circ b_f \circ q_f$ in
\item $q_f$ je surjektivna, $b_f$ je bijektivna in $i_f$ je injektivna.
\end{enumerate}
%
Računajmo: $f(x) = b_f([x]_\sim) = i_f(b_f([x]_\sim)) = i_f(b_f(q_f(x)))$, za vse $x \in A$, od
koder sledi prva trditev.

Vemo že, da je kanonična kvocientna preslikava surjektivna in kanonična inkluzija
injektivna. Ostane nam še bijektivnost preslikave $b_f$:
%
\begin{itemize}
\item $b_f$ je injektivna: naj bosta $\xi, \zeta \in A/(\sim_f)$ in denimo, da velja $b_f(\xi) = b_f(\zeta)$.
  Obstajata $x, y \in A$, da je $\xi = [x]_\sim$ in $\zeta = [y]_\sim$. Velja
  %
  \begin{equation*}
    f(x) = i_f(b_f(q_f(x))) = i_f(b_f(\xi)) = i_f(b_f(\zeta)) = i_f(b_f(q_f(y))) = f(y),
  \end{equation*}
  %
  torej je $x \sim_f y$ in zato $\xi = [x]_\sim = [y]_\sim = \zeta$.

  \item $b_f$ je surjektivna: naj bo $u \in \img{f}(A)$. Tedaj obstaja $x \in A$, da je $u = f(x)$.
  Vzemimo $\xi = [x]_E$ in preverimo: $b_f(\xi) = b_f([x]_\sim) =f(x) = u$.
\end{itemize}


\chapter{Delne urejenosti}
\textbf{To poglavje še ni predelano v {\LaTeX}.}
%\chapter{Relacije urejenosti}

\section{Relacije urejenosti}
\begin{definicija}
  Relacija $R \subseteq A \times A$ je:
  %
  \begin{enumerate}
  \item \textbf{šibka urejenost}, ko je refleksivna in tranzitivna,
  \item \textbf{delna urejenost}, ko je refleksivna, tranzitivna in antisimetrična,
  \item \textbf{linearna urejenost}, ko je delna urejenost in je strogo sovisna ($\all{x, y \in A} x \rel{R} y \lor y \rel{R} x$).
  \end{enumerate}
\end{definicija}

Za relacije urejenosti ponavadi uporabljamo simbole, ki spominjajo na znak $\leq$, kot so $\preceq$, $\subseteq$, $\sqsubseteq$ ipd.

\begin{zgled}
  Primeri urejenosti:
  \begin{enumerate}
    \item Relacija deljivosti na naravnih številih je delna urejenost.
    \item Relacija deljivosti na celih številih je šibka urejenost, ni pa delna urejenost.
    \item Relacija $\leq$ na realnih številih je linearna urejenost.
    \item Relacija $\subseteq$ na $\pow{A}$ je delna urejenost. Za katere množice $A$ je linearna?
    \item Relacija $=$ je delna urejenost. Imenuje se tudi \textbf{diskretna urejenost}.
  \end{enumerate}
\end{zgled}


\begin{definicija}
  V delni ureditvi $(P, {\leq})$ je \textbf{veriga} taka podmnožica $V \subseteq P$, ki je linearno urejena z relacijo~$\leq$, se pravi $\all{x, y \in V} x \leq y \lor y \leq x$. \textbf{Antiveriga} je taka podmnožica $A \subseteq P$, ki je diskretno urejena z relacijo~$\leq$, se pravi $\all{x, y \in A} x \leq y \lthen x = y$.
\end{definicija}

\begin{zgled}
\end{zgled}

\begin{zgled}
  Primeri verig in antiverig:
  %
  \begin{itemize}
  \item Če je $(P, {\leq})$ linearno urejena, je vsaka njena podmnožica veriga. Na primer, vsaka podmnožica $\NN$ je veriga glede na~$\leq$.
  \item Potence števila $2$ tvorijo verigo v $\NN$ glede na relacijo deljivosti.
  \item Praštevila tvorijo antiverigo v $\NN$ glede na relacijo deljivosti.
  \item V $(\pow{\QQ}, {\subseteq})$ imamo neštevno verigo
    %
    $V = \set{S \in \pow(\QQ) \mid \text{$S$ je doljna množica}}$.
    %
    Množica $S \subseteq \QQ$ je \textbf{doljna}, če velja
    $\all{x y \in \QQ} x \leq y \land y \in \QQ \lthen x \in \QQ$.
    Res, vsak Dedekindov rez je doljna množica, le-teh pa je neštevno mnogo.
  \end{itemize}
\end{zgled}



\subsection{Hassejev diagram}

Končno delno ureditev $(A, \leq)$ lahko predstavimo s \textbf{Hassejevim diagramom}: elemente
množice $A$ narišemo tako, da je $x$ pod $y$, kadar velja $x \leq y$. Nato povežemo vozlišči $x$ in $y$, če je $y$ neposredni naslednik $x$, se pravi, da velja $x \neq y$, $x \leq y$ in iz $x \leq z \leq y$ sledi $x = z \lor z = y$.

\begin{vaja}
  Narišite Hassejev diagram relacije deljivosti na množici $\set{0, 1, \dots, 10}$ ter
  Hassejev diagram relacije $\subseteq$ na množici $\pow(\{a,b,c\})$.
\end{vaja}

\begin{vaja}
  Kako v Hassejevem diagramu prepoznamo verigo? In kako prepoznamo antiverigo?
\end{vaja}


\subsection{Operacije na urejenostih}

\subsubsection{Obratna urejenost}

Če je $\leq$ delna urejenost na $P$ potem je tudi transponirana relacija $\geq$, definirana z
%
\begin{equation*}
    x \geq y \liff x \leq y,
\end{equation*}
%
delna urejenost na $P$. Če je $\leq$ linearna, je $\geq$ linearna.

\subsubsection{Produktna in leksikografska urejenost}

Naj bosta $(P, {\leq_P})$ in $(Q, {\leq_Q})$ delni urejenosti. Na kartezičnem produktu $P \times Q$ lahko definiramo dve urejenosti.

Prva je \textbf{produktna} urejenost
%
\begin{equation*}
  (x_1,y_1) \leq_{\times} (x_2,y_2) \defiff x_1 \leq_P x_2 \land y_1 \leq_Q y_2
\end{equation*}
%
in druga \textbf{leksikografska} urejenost
%
\begin{equation*}
  (x_1,y_1) \preceq_\mathrm{lex} (x_2,y_2)
  \defiff (x_1 \neq x_2 \land x_1 \leq_P x_2) \lor (x_1 = x_2 \land y_1 \leq_Q y_2).
\end{equation*}


\begin{vaja}
  Kako si predstavljamo produktno in leksikografsko ureditev na $[0,1] \times [0,1]$, če $[0,1]$ uredimo z običajno relacijo $\leq$? Na sliki označite območji
  %
  \begin{equation*}
    \set{(x,y) \in [0,1] \times [0, 1] \such (1/2,1/3) \leq_\times (x,y)}
  \end{equation*}
  %
  in
  %
  \begin{equation*}
    \set{(x,y) \in [0,1] \times [0, 1] \such (1/2,1/3) \leq_\mathrm{lex} (x,y)}.
  \end{equation*}
\end{vaja}

\begin{izjava}
  Produktna in leksikografska urejenosti sta delni urejenosti. Leksikografska urejenost linearnih urejenosti je linearna.
\end{izjava}

\begin{proof}
  Dejstvo, da je produktna urejenost refleksivna, tranzitivna in antisimetrična, pustimo za vajo. Preverimo, da je leksikografska urejenost $\leq_\mathrm{lex}$ delna urejenost.

  Dokaz, da je $\leq_\mathrm{lex}$ je refleksivna: za vsak $(x, y) \in P \times Q$ velja $x = x \land y \sqsubseteq y$, torej velja $(x, y) \sqsubseteq (x, y)$.

  Dokaz, da je $\leq_\mathrm{lex}$ je antisimetrična: naj bosta $(x_1,y_1), (x_2,y_2) \in P \times Q$ in denimo, da velja
  %
  \begin{equation*}
    (x_1, y_1) \leq_\mathrm{lex} (x_2, y_2) \land (x_2, y_2) \leq_\mathrm{lex} (x_1, y_1)
  \end{equation*}
  %
  To je ekvivalentno
  %
  \begin{align*}
  & (x_1 \neq x_2 \land x_1 \leq_P x_2 \land x_2 \neq x_1 \land x_2 \leq_P x_1) \lor {}\\
  & (x_1 \neq x_2 \land x_1 \leq_P x_2 \land x_2 = x_1 \land y_2 \leq_Q y_1) \lor {}\\
  & (x_1 = x_2 \land y_1 \leq_Q y_2 \land x_2 \neq x_1 \land x_2 \leq_P x_1) \lor {}\\
  & (x_1 = x_2 \land y_1 \leq_Q y_2 \land x_2 = x_1 \land y_2 \leq_Q y_1).
  \end{align*}
  %
  Če v zgornji formuli upoštevamo, da je $x_1 \neq x_2 \land x_1 = x_2$, vidimo, da sta drugi in tretji disjunkt ekvivalentna $\bot$, zato
  je izjava ekvivalentna:
  \begin{align*}
  &(x_1 \neq x_2 \land x_1 \leq_P x_2 \land x_2 \neq x_1 \land x_2 \leq_P x_1) \lor {}\\
  &(x_1 = x_2 \land y_1 \leq_Q y_2 \land x_2 = x_1 \land y_2 \leq_Q y_1).
  \end{align*}
  %
  A tudi prvi disjunkt je ekvivalenten $\bot$, ker iz $x_1 \leq_P x_2 \land x_2 \leq_P x_1$ sledi $x_1 = x_2$, saj je $\leq_P$ po predpostavki antisimetrična. Torej ostane samo zadnji disjunkt, ki je ekvivalenten
  \begin{equation*}
    x_1 = x_2 \land y_1 \leq_Q y_2 \land y_2 \leq_Q y_1.
  \end{equation*}
  %
  Ker je $\leq_Q$ antisimetrična, sledi $x_1 = x_2$ in $y_1 = y_2$, kar smo želeli dokazati.

  Dokaz, da je $\leq_\mathrm{lex}$ tranzitivna: naj bodo $(x_1,y_1), (x_2,y_2), (x_3, y_3) \in P \times Q$ in denimo, da velja
  %
  \begin{equation*}
    (x_1, y_1) \leq_\mathrm{lex} (x_2, y_2) \land (x_2, y_2) \leq_\mathrm{lex} (x_3, y_3).
  \end{equation*}
  %
  To je ekvivalentno
  %
  \begin{align*}
  & (x_1 \neq x_2 \land x_1 \leq_P x_2 \land x_2 \neq x_3 \land x_2 \leq_P x_3) \lor  {} \\
  & (x_1 \neq x_2 \land x_1 \leq_P x_2 \land x_2 = x_3 \land y_2 \leq_Q y_3) \lor {} \\
  & (x_1 = x_2 \land y_1 \leq_Q y_2 \land x_2 \neq x_3 \land x_2 \leq_P x_3) \lor {} \\
  & (x_1 = x_2 \land y_1 \leq_Q y_2 \land x_2 = x_3 \land y_2 \leq_Q y_3)
  \end{align*}
  %
  Obravnavajmo štiri primere in v vsakem od njih dokažimo $(x_1, y_1) \leq_\mathrm{lex} (x_3, y_3)$, se pravi
  $(x_1 \neq x_3 \land x_1 \leq_P x_3) \lor (x_1 = x_3 \land y_1 \leq_Q y_3)$:
  %
  \begin{enumerate}
  \item Če velja $x_1 \neq x_2 \land x_1 \leq_P x_2 \land x_2 \neq x_3 \land x_2 \leq_P x_3$: ker je $\leq$ tranzitivna sledi $x_1 \leq_P x_3$, poleg tega pa velja $x_1 \neq
    x_3$: če bi veljalo $x_1 = x_3$, bi iz predpostavk dobili $x_3 \leq_P x_2 \land x_2 \leq_P x_3$, od koder bi sledilo $x_2 = x_3$, kar je v
    protislovju s predpostavko $x_2 \neq x_3$.

  \item Če velja $x_1 \neq x_2 \land x_1 \leq_P x_2 \land x_2 = x_3 \land y_2 \leq_Q y_3$: ker je $x_2 = x_3$ iz prvih dveh predpostavk sledi $x_1 \neq x_3 \land x_1 \leq_P x_3$.

  \item Če velja $x_1 = x_2 \land y_1 \leq_Q y_2 \land x_2 \neq x_3 \land x_2 \leq_P x_3$: ker je $x_1 = x_2$ iz zadnjih dveh predpostavk sledi $x_1 \neq x_3 \land x_1 \leq_P x_3$.

  \item Če velja $x_1 = x_2 \land y_1 \leq_Q y_2 \land x_2 = x_3 \land y_2 \leq_Q y_3$: torej je $x_1 = x_3$ ker je $=$ tranzitivna in $y_1 \leq_Q y_3$ ker je $\leq_Q$ tranzitivna.
  \end{enumerate}
  %
  Nazadnje preverimo še, da je $\leq_\mathrm{lex}$ linearna, če sta $\leq$ in $\leq_Q$ linearni. Naj bosta $(x_1,y_1), (x_2,y_2) \in P \times Q$. Dokazati želimo
  %
  \begin{equation*}
    (x_1, y_1) \preceq (x_2, y_2) \lor (x_2, y_2) \preceq (x_1, y_1).
  \end{equation*}
  %
  To je ekvivalentno disjunkciji
  % 
  \begin{align*}
    & (x_1 \neq x_2 \land x_1 \leq_P x_2) \lor {} \\
    & (x_1 = x_2 \land y_1 \leq_Q y_2) \lor {} \\
    & (x_2 \neq x_1 \land x_2 \leq_P x_1) \lor {} \\
    & (x_2 = x_1 \land y_2 \leq_Q y_1),
  \end{align*}
  %
  kar je ekvivalentno
  %
  \begin{align*}
    & (x_1 \neq x_2 \land (x_1 \leq_P x_2 \lor x_2 \leq_P x_1)) \lor {} \\
    &(x_1 = x_2 \land (y_1 \leq_Q y_2 \lor y_2 \leq_Q y_1)).
  \end{align*}
  %
  Ker sta $\leq_P$ in $\leq_Q$ linearni, je to ekvivalentno
  %
  \begin{equation*}
    (x_1 \neq x_2 \land \top) \lor (x_1 = x_2 \land \top),
  \end{equation*}
  %
  kar je ekvivalentno
  \begin{equation*}
    (x_1 \neq x_2) \lor (x_1 = x_2).
  \end{equation*}
  %
  To pa drži po zakonu o izključeni tretji možnosti. S tem je linearnost $\leq_\mathrm{lex}$, dokazana.
\end{proof}

\subsubsection{Vsota urejenosti}

Naj bosta $(P, \leq_P)$ in $(Q, \leq_Q)$ delni urejenosti. Na vsoti $P + Q$ lahko
definiramo urejenost $\leq_{+}$ s predpisom:
%
\begin{equation*}
  u \leq_{+} v \defiff
  \begin{aligned}[t]
    & (\some{x, y \in P} u = \inl(x) \land v = \inl(y) \land x \leq_P y) \lor {} \\
    & (\some{s, t \in Q} u = \inr(s) \land v = \inr(t) \land s \leq_Q t).
  \end{aligned}
\end{equation*}

\subsubsection{Zaporedna vsota urejenosti}

Naj bosta $(P, \leq_P)$ in $(Q, \leq_Q)$ delni urejenosti. Na vsoti $P + Q$ lahko definiramo urejenost $\leq_{\to}$ s predpisom:
%
\begin{equation*}
  u \leq_{\to} v \defiff
  \begin{aligned}[t]
    &(\some{x, y \in P} u = \inl(x) \land v = \inl(y) \land x \leq_P y) \lor {} \\
    &(\some{x \in P} \some{s \in Q} u = \inl(x) \land v = \inr(s)) \lor {} \\
    &(\some{s, t \in Q} u = \inr(s) \land v = \inr(t) \land s \leq_Q t).
  \end{aligned}
\end{equation*}
%
Torej so vsi elementi $P$ pred vsemi elementi $Q$. Zaporedna vsota linearnih urejenosti je linearna.


\subsubsection{Potenca urejenosti}

Naj bo $(P, \leq)$ delna urejenost in $A$ množica. Na eksponentni množici $P^A$ lahko definiramo urejenost $\preceq$ s predpisom:
%
\begin{equation*}
  f \preceq g \defiff \all{x \in A} f(x) \leq g(x).
\end{equation*}

\begin{vaja}
  Ali je $\preceq$ linearna, kadar je $\leq$ linearna?
\end{vaja}


\subsubsection{Delna urejenost, inducirana s šibko ureditvijo}

Naj bo $(P, \leq)$ šibka ureditev. Relacija $\sim$ na $P$, definirana s predpisom
%
\begin{equation*}
  x \sim y \defiff x \leq y \land y \leq x,
\end{equation*}
%
je ekvivalenčna relacija. Na kvocientu $P/{\sim}$ lahko definiramo relacijo $\preceq$ s
predpisom
%
\begin{equation*}
  [x] \preceq [y] \defiff x \leq y.
\end{equation*}
%
Treba je preveriti, da je relacija dobro definirana, saj smo uporabili predstavnike ekvivalenčnih razredov. Se pravi, ali velja
\begin{equation*}
  x \sim x' \land y \sim y' \lthen (x \leq y \liff x' \leq y') ?
\end{equation*}
%
Pa preverimo. Denimo, da velja $x, y, x', y' \in P$ in $x \sim x'$ in $y \sim y'$.
Torej velja
\begin{equation*}
  x \leq x' \land x' \leq x \land y \leq y' \land y' \land x.
\end{equation*}
%
Sedaj dokažimo $x \leq y \liff x' \leq y'$:
%
\begin{enumerate}
\item Če velja $x \leq y$ potem $x' \leq x \leq y \leq y'$.
\item Če velja $x' \leq y'$, potem $x \leq x' \leq y' \leq y$.
\end{enumerate}
%
Torej je $\preceq$ dobro definirana.

\begin{izjava}
  Relacija, ki je inducirana s šibko ureditvijo, je delna ureditev.
\end{izjava}

\begin{proof}
  Refleksivnost in tranzitivnost $\preceq$ sledita iz refleksivnosti in tranzitivnosti~$\leq$. Preverimo antisimetričnost: denimo, da velja $[x] \leq [y]$ in $[y] \leq [x]$. Tedaj velja $x \leq y$ in $y \leq x$, torej velja $x \sim y$ in $[x] = [y]$.
\end{proof}

\begin{zgled}
  Obravnavajmo cela števila $\ZZ$ in deljivost $\mid$, ki je šibka
  ureditev. Za vse $k, m \in \ZZ$ velja
  \begin{equation*}
    k \sim m \liff k \mid m \land m \mid k \liff |k| = |m|.
  \end{equation*}
  %
  Torej je $\ZZ/{\sim} \cong \NN$, kjer izomorfizem preslika $[k] \mapsto |k|$. Delna ureditev na $\ZZ/{\sim}$ inducirana z deljivostjo je spet deljivost (ko jo prenesemo iz $\ZZ/{\sim}$ na $\NN$ s pomočjo izomorfizma).
\end{zgled}


\subsection{Monotone preslikave}

\begin{definicija}
  Preslikava $f : P \to Q$ med delnima urejenostma $(P, {\leq_P})$ in $(Q, {\leq_Q})$ je
  \textbf{monotona} (ali \textbf{naraščajoča}), ko velja $\all{x, y \in P} x \leq_P y \lthen f(x) \leq_Q f(y)$.
\end{definicija}

\begin{definicija}
  Preslikava $f : P \to Q$ med delnima urejenostma $(P, \leq_P)$ in $(Q, \leq_Q)$ je
  \textbf{antitona} (ali \textbf{padajoča}), ko velja $\all{x, y \in P} x \leq_P y \lthen f(y) \leq_Q f(x)$.
\end{definicija}

\begin{opomba}
  V analizi ">monotona"< pomeni ">monotona ali antitona"<. To ni nič
  čudnega, ker ">dan"< tudi pomeni ">dan in noč">.
\end{opomba}

\begin{izrek}
  Kompozicija monotonih preslikav je monotona. Identiteta je monotona.
\end{izrek}

\begin{proof}
  Naj bosta $f : P \to Q$ in $g : Q \to R$ monotoni preslikavi med delnimi
  urejenostmi $(P, {\leq_P})$, $(Q, {\leq_Q})$ in $(R, {\leq_R})$. Če je $x \leq_P y$, potem je zaradi monotonosti $f$ tudi $f(x) \leq_Q f(y)$, nato pa je zaradi monotonosti $g$ spet $g(f(x)) \leq_R g(f(y))$. Identiteta je očitno monotona.
\end{proof}

\begin{zgled}
  Primeri monotonih preslikav:
  \begin{enumerate}
    \item Konstantna preslikava je monotona.
    \item Seštevanje ${+} : \RR \times \RR \to \RR$ je monotona operacija glede na produktno ureditev na $\RR \times \RR$.
    \item Množenje ${\times} : \RR \times \RR \to \RR$ ni monotona operacija.
  \end{enumerate}
\end{zgled}


\subsection{Meje}

\begin{definicija}
  Naj bo $(P, {\leq})$ delna urejenost, $S \subseteq P$ in $x \in P$:
  \begin{itemize}

  \item $x$ je \textbf{spodnja meja} podmnožice $S$, ko velja $\all{y \in S} x \leq y$,

  \item $x$ je \textbf{zgornja meja} podmnožice $S$, ko velja $\all{y \in S} y \leq x$,

  \item $x$ je \textbf{infimum} ali \textbf{največja spodnja meja} ali \textbf{natančna spodnja meja} podmnožice $S$, ko je spodnja meja $S$ in velja: za vse $y \in P$, če je $y$ spodnja meja
    $S$, potem je $y \leq x$,

  \item $x$ je \textbf{supremum} ali \textbf{najmanjša zgornja meja} ali \textbf{natančna zgornja meja} podmnožice $S$, ko je zgornja meja $S$ in velja: za vse $y \in P$, če je $y$ zgornja meja $S$, potem je $x \leq y$,

  \item $x$ je \textbf{minimalni element} podmnožice $S$, ko velja $x \in S$ in $\all{y \in S} y \leq x \lthen x = y$,

  \item $x$ je \textbf{maksimalni element} podmnožice $S$, ko velja $x \in s$ in
      $\all{x \in S} x \leq y \lthen x = y$,

  \item $x$ je \textbf{najmanjši} ali \textbf{prvi} element ali \textbf{minimum} podmnožice $S$, ko velja $x \in S$ in $\all{y \in S} x \leq y$,

  \item $x$ je \textbf{največji} ali \textbf{zadnji} element ali \textbf{maksimum} podmnožice $S$, ko velja $x \in S$ in $\all{y \in S} y \leq x$.
\end{itemize}
\end{definicija}

\begin{opomba}
  Minimalni element ni isto kot minimum (in maksimalni element ni isto kot maksimum).
\end{opomba}

Kadar govorimo o ">prvem elementu"< ali ">maksimalnem elementu"< in ne povemo, na
katero podmnožico se nanaša element, imamo običajno v mislih kar celotno delno
ureditev.

\begin{izrek}
  Naj bo $(P, {\leq})$ delna urejenost in $S \subseteq P$. Tedaj ima $S$ največ en
  infimum in največ en supremum, ki ju zapišemo $\inf S$ ter $\sup S$, kadar obstajata.
\end{izrek}

\begin{proof}
  Denimo, da sta $x$ in $y$ oba infimum $S$. Ker je $y$ spodnja meja za
  $S$ in $x$ njen infimum, velja $y \leq x$. Podobno velja $x \leq y$, torej $x = y$. Za
  supremum je dokaz podoben.
\end{proof}

\begin{zgled}
  Supremum končne neprazne množice $S \subseteq \NN$ za relacijo deljivosti $\mid$
  je najmanjši skupni večkratnik elementov iz $S$. Infimum je največji skupni
  delitelj. Kaj pa, če je $S$ prazna ali neskončna?
\end{zgled}

\subsection{Mreže}

\begin{definicija}
  Naj bo $(P, {\leq})$ delna urejenost:
  %
  \begin{enumerate}
  \item $(P, \leq)$ je \textbf{mreža}, ko imata vsaka dva elementa $x, y \in P$ infimum in supremum.

  \item $(P, \leq)$ je \textbf{omejena mreža}, ko ima vsaka končna podmnožica $P$ infimum in supremum.

  \item $(P, \leq)$ je \textbf{polna mreža}, ko ima vsaka podmnožica $P$ infimum in supremum.
  \end{enumerate}
  %
  Infimum in supremum elementov $x$ in $y$ pišemo $x \land y$ in $x \lor y$.
\end{definicija}

\begin{izrek}
  Delna urejenost $(P, {\leq})$ je omejena mreža natanko tedaj, ko ima
  najmanjši element in največji element, ter imata vsaka sva elementa infimum in supremum.
\end{izrek}

\begin{proof}
  Denimo, da je $(P, \leq)$ omejena mreža. Tedaj $P$ ima najmanjši element, namreč
  $\sup \emptyset$, in največji element, namreč $\inf \emptyset$. Infimum in supremum $x$ in $y$ sta seveda $\inf \set{x, y}$ in $\sup \set{x, y}$.

  Denimo, da ima $P$ najmanjši element $\bot_P$ in največji element $\top_P$, vsaka dva
  elementa pa imata infimum in supremum. Naj bo $S \subseteq P$ končna množica:
  %
  \begin{enumerate}
  \item če je $S = \emptyset$, potem je $\inf S = \top_P$ in $\sup S = \bot_P$,
  \item če je $S = \set{x_1, \ldots, x_n}$ za $n > 0$, potem je $\inf S = \inf \set{x_1, \ldots, x_{n-1}} \lor x_n$ in $\sup S = \sup \set{x_1, \ldots, x_{n-1}} \lor x_n$.
  \end{enumerate}
\end{proof}

\begin{zgled}
  Primeri mrež:
  %
  \begin{enumerate}
  \item Množica $\two = \set{\bot, \top}$ je omejena mreža za relacijo $\lthen$.
  \item Relacija deljivosti na množici pozitivnih naravnih števil je omejena mreža.
  \item Potenčna množica $\pow{A}$, urejena z $\subseteq$, je polna mreža.
  \item Zaprti interval $[a,b]$, urejen z $\leq$, je polna mreža.
  \item Realna števila $R$, urejena z $\leq$,
  \end{enumerate}
\end{zgled}


\chapter{Indukcija in dobra osnovanost}

\section{Dobra osnovanost}

\subsection{Indukcija na naravnih številih}

Poznamo že indukcijo na naravnih številih. Zapišemo jo lahko na več načinov,
kjer naslednika števila `n` označimo `n⁺`:

1. Kot aksiom o predikatih na naravnih številih:

        φ(0) ∧ (∀ n ∈ N . φ(n) ⇒ φ(n⁺)) ⇒ ∀ m ∈ N . φ(m)

2. Kot lastnost podmnožic naravnih števil:

        ∀ S ∈ P(N) . 0 ∈ S ∧ (∀ k ∈ N . k ∈ S ⇒ k⁺ ∈ S) ⇒ S = N

Uporabljali bomo verzijo s podmnožicami. Najprej jo predelajmo v ekvivalentno obliko:

    ∀ S ∈ P(N) . 0 ∈ S ∧ (∀ k ∈ N . k ∈ S ⇒ k⁺ ∈ S) ⇒ S = N
    ∀ S ∈ P(N) . 0 ∈ S ∧ (∀ m ∈ N . (∀ k ∈ N . k⁺ = m ⇒ k ∈ S) ⇒ m ∈ S) ⇒ S = N
    ∀ S ∈ P(N) . (∀ m ∈ N . (∀ k ∈ N . k⁺ = m ⇒ k ∈ S) ⇒ m ∈ S) ⇒ S = N

Kaj smo dosegli? Bazo indukcije in indukcijski korak smo združili v eno samo predpostavko

    ∀ m ∈ N . (∀ k ∈ N . k⁺ = m ⇒ k ∈ S) ⇒ m ∈ S          (1)

Če vstavimo `m := 0`, dobimo:

    (∀ k ∈ N . k⁺ = 0 ⇒ k ∈ S) ⇒ 0 ∈ S
    (∀ k ∈ N . ⊥ ⇒ k ∈ S) ⇒ 0 ∈ S
    (∀ k ∈ N . ⊤) ⇒ 0 ∈ S
    ⊤ ⇒ 0 ∈ S
    0 ∈ S

Če vstavimo `m := n⁺` dobimo:

    ∀ m ∈ N . (∀ k ∈ N . k⁺ = n⁺ ⇒ k ∈ S) ⇒ n⁺ ∈ S
    ∀ m ∈ N . (∀ k ∈ N . k = n ⇒ k ∈ S) ⇒ n⁺ ∈ S
    ∀ m ∈ N . n ∈ S ⇒ n⁺ ∈ S

To pa sta ravno običajna pogoja za indukcijo.

Ali lahko izrazimo indukcijo na naravnih številih tudi brez operacije naslednik?
Da, s pomočjo relacije `<`:

    ∀ S ∈ P(N) . (∀ m ∈ N . (∀ k ∈ N . k < m ⇒ k ∈ S) ⇒ m ∈ S) ⇒ S = N

Temu principu pravimo tudi *krepka indukcija*, z besedami jo povemo takole:
Za podmnožico `S ⊆ N` velja `S = N`, če za vse `m ∈ N` velja:

> Če so vsa števila manjša od `m` v `S`, potem je tudi `m` v `S`.

Denimo, da `S` res ima dano lastnost. Ali je `0 ∈ S`? Da, ker za vse predhodnike `0` velja, da
so `S` (saj jih ni). Ali je `1 ∈ S`? Da, saj za vse predhodnike `1` velja, da so v `S`. Ali je `2 ∈
S`? Da, saj za vse predhodnike `2` velja, da so v `S`. In tako naprej.

\subsection{Dobra osnovanost}

Princip indukcije na naravnih številih posplošimo.

**Definicija:** Relacija `R ⊆ A × A` je **dobro osnovana**, če velja:

    ∀ S ∈ P(A) . (∀ y ∈ A . (∀ x ∈ A . x R y ⇒ x ∈ S) ⇒ y ∈ S) ⇒ S = A.

Množici `S ⊆ A`, ki zadošča pogoju

    ∀ y ∈ A . (∀ x ∈ A . x R y ⇒ x ∈ S) ⇒ y ∈ S

pravimo **`R`-progresivna** množica.

Kaj smo pravzaprav naredili: opazili smo, da ima relacija "`k` je neposredni predhodnik
`m`" na `N` pomembno lastnost (1). Zanima nas, ali imajo tudi druge relacije to lastnost,
saj nam bodo omogočile neke vrste posplošen princip indukcije. Z definicijo smo dali
relacijam, ki nas zanimajo, ime.

\subsubsection{Primer: dvojiška drevesa}

Naravna števila `N` so *induktivno definirana množica*. To pomeni, da elemente `N`
opredelimo s pravili, ki povedo, kako se gradi naravna števila:

1. `0 ∈ N`
2. če je `n ∈ N`, potem je `n⁺ ∈ N`

Množica `N` vsebuje natanko tiste elemente, ki jih lahko zgradimo s pomočjo teh pravil:

    0, 0⁺, 0⁺⁺, 0⁺⁺⁺, 0⁺⁺⁺⁺, …

Tu sta `0` in `⁺` mišljena kot simbolni oznaki, podobno kot `in₁` in `in₂` v definiciji
vsote množic.

Na tak način lahko definiramo tudi druge množice. Na primer, **dvojiška drevesa** so
induktivno definirana množica `Tree`, s predpisoma:

1. `empty ∈ Tree`
2. če je `t₁ ∈ Tree` in `t₂ ∈ Tree`, potem je `tree(t₁, t₂) ∈ Tree`

Z besedami: drevo je bodisi prazno, bodisi je sestavljeno iz dveh *poddreves*. Ali znamo
našteti vsa drevesa, ali še bolje, jih narisati?

    empty,
    tree(empty, empty)
    tree(empty, tree(empty, empty)),
    tree(tree(empty, empty), empty),
    tree(tree(empty, empty), tree(empty, empty)),
    ⋮

Definirajmo relacijo `R ⊆ Tree × Tree` s predpisom:

    t R s ⇔ ∃ u ∈ Tree . s = tree(t, u) ∨ s = tree(u, t)

To je relacija "neposredno poddrevo". Ta relacija je dobro osnovana (česar ne bomo
dokazali) in nje pa dobimo naslednji princip indukcije za dvojiška drevesa.

**Indukcija za dvojiška drevesa:** Naj bo `S ⊆ Tree` podmnožica dreves, za katero velja:

1. Prazno drevo je v `S`.
2. Za vsa drevesa `t₁` in `t₂` velja: če je `t₁ ∈ S` in `t₂ ∈ S`, potem je `tree(t₁, t₂) ∈ S`.

Tedaj je `S = Tree`.

Princip povejmo še kot logični princip:

**Indukcija za dvojiška drevesa:** Naj bo `φ` lastnost dvojiških dreves, za katero velja:

1. Baza indukcije: `φ(empty)`
2. Indukcijski korak: za vsa drevesa `t₁` in `t₂`, če velja `φ(t₁)` in `φ(t₂)`, potem
   `φ(tree(t₁, t₂))`.

Tedaj `∀ t ∈ Tree, φ(t)`.

Kot vidimo, imamo v indukcijskem koraku *dve* indukcijski predpostavki, ker ima vsako
sestavljeno drevo dve poddrevesi.


\subsubsection{Dobra osnovanost in padajoče verige}

Kako pa bi dobili kak protiprimer, se pravi, relacijo, ki ni dobra osnovanost? Poiskati
moramo kako lastnost, ki jo imajo vse dobre osnovanosti.

**Definicija:** Naj bo `R ⊆ A × A` relacija na `A`. **Padajoča veriga** (za relacijo `R`)
je zaporedje `a : N → A`, za katerega velja `∀ i ∈ N . a(i+1) R a(i)`.

Se pravi, da je padajoča veriga zaporedje, za katerega velja

    ⋯ a_4 R a_3 R a_2 R a_1 R a_0

**Cikel** je končna podmnožica `{a_0, …, a_n}` da velja

     a_0 R a_1 R ⋯ R a_n R a_0

Iz takega cikla dobimo padajočo verigo, tako da cikel ponavljamo v nedogled:

     ⋯ R a_0 R ⋯ R a_n R a_0 R a_1 R ⋯ R a_n R a_0

**Lemma:** V dobri osnovanosti ni ciklov in ni padajočih verig.

*Dokaz.* Dovolj je pokazati, da ni padajočih verig, saj iz cikla dobimo padajočo verigo.
Denimo, da je `a : N → A` padajoča veriga za `R ⊆ A × A`. Dokazali bomo, da `R` ni dobro
osnovana. Se pravi, da moramo poiskati `R`-progresivno podmnožico `S ⊆ A`, za katero velja
`S ≠ A`. Vzemimo `S := A \ { a(i) | i ∈ N }`. Očitno velja `S ≠ A`, saj je `a(0) ∈ A` in
`a(0) ∉ S`. Preverimo, da je `S` progresivna, se pravi, da je

    ∀ y ∈ A . (∀ x ∈ A . x R y ⇒ x ∈ S) ⇒ y ∈ S

Naj bo `y ∈ A` in denimo, da velja

    ∀ x ∈ A . x R y ⇒ x ∈ S                               (2)

Dokazati moramo `y ∈ S`. Obravnavamo dve možnosti:

1. Če `y ∈ S`, potem seveda sledi `y ∈ S`.

2. Če `y ∉ S`, potem obstaja `i ∈ N`, da je `y = a(i)`. Ker je `a(i+1) R a(i)`, iz
   predpostavke (2) sledi `y = a(i) ∈ S`.

Torej v vsakem primeru velja `y ∈ S`. □

**Protiprimer:** Sedaj lahko zlahka priskrbimo kak protiprimer. Na primer, cela števila
`Z` z relacijo `R ⊆ Z × Z`

    a R b ⇔ a + 1 = b

niso dobro osnovana, ker imajo padajočo verigo

    ⋯ R (-3) R (-2) R (-1) R 0

Prav tako ni dobro osnovana relacija `<` na intervalu `[0,1]`, ker imamo padajočo verigo
`a(n) = 2^{-n}`.


\section{Dobra urejenost}

Posplošimo sedaj še krepko indukcijo na naravnih številih. Tokrat bomo najprej posplošili
strogo urejenost `<`.

\subsection{Stroge urejenosti}

**Definicija:** Relacija `R ⊆ A × A` je **stroga urejenost**, če je

1. irefleksivna: `∀ x ∈ A . ¬ (x R x)`
2. tranzitivna: `∀ x, y, z ∈ A . x R y ∧ y R z ⇒ x R z`

Stroga urejenost je **linearna**, če je še

3. sovisna: `∀ x, y ∈ A . x R y ∨ x = y ∨ y R x.`

Za stroge urejenosti uporabljamo simbole `<`, `⊂`, `≺`, `⊏` ipd.

Relaciji `<` in `≤` na številih sta med seboj povezani, saj denimo za realna števila velja

    x < y ⇔ x < y ∧ x ≠ y

in

    x ≤ y ⇔ x < y ∨ x = y            (3)

To velja v splošnem. Stroa urejenost `<` na množici `A` porodi delno urejenost `≤` na `A`,
definirano s predpisom:

    x ≤ y  ⇔  x = y ∨ x ≤ y

V obratno smer, delna urejenost `⊑` določa strogo urejenost `⊏`, definirano s predpisom

    a ⊏ b  ⇔  a ≠ b ∧ a ⊑ b         (4)

Seveda je treba preveriti naslednja dejstva:

* če je `<` stroga urejenost, potem je `≤` definirana s (3) delna urejenost
* če je `⊑` delna urejenost, potem je `⊏` definirana s (4) stroga urejenost.

Tako lahko prehajamo med delno in strogo urejenostjo.

\subsection{Dobra ureditev}

**Definicija:** Relacija je **dobra ureditev**, če je dobro osnovana in stroga linearna ureditev.

**Izrek:** Relacija je dobra ureditev natanko tedaj, ko je dobro osnovana in sovisna.

*Dokaz.* V eno smer je ekvivalenca očitna, zato dokažimo samo obratno smer. Denimo, da je
`R ⊆ A × A` dobro osnovana in sovisna relacija. Doazujemo, da je dobra ureditev, se pravi,
da potrebujemo še irefleksivnost in tranzitivnost `R`:

* `R` je irefleksivna: če bi veljalo `x R x` za `x ∈ A`, potem `R` ne bi bila dobro
  osnovana, ker bi vsebovala padajočo verigo `⋯ x R x R x`.

* `R` je tranzitivna: denimo, da velja `x R y` in `x R z`. Dokazujemo `x R z`. Ker je `R`
  sovisna, velja `x R z` ali `x = z` ali `z R x`. Pokažimo, da `x = z` in `z R x` nista
  možna:

      1. če je `x = z`, potem velja `x R y` in `y R x`, torej `x` in `y` tvorita cikel, a
         `R` je dobro osnovana, zato to ni možno.

      2. če velja `z R x`, potem dobimo cikel `x R y R z R x`, kar spet ni možno. □


**Lema:** Denimo, da je `<` stroga urejenost na neprazni množici `B`. Če `B` nima
`≤`-minimalnega elementa, potem ima padajočo verigo.

*Dokaz.* Denimo, da `B` nima minimalnega elementa, torej

    ¬ ∃ x ∈ B . ∀ y ∈ B . y ≤ x ⇒ y = x.

To je ekvivalentno

    ∀ x ∈ B . ∃ y ∈ B . y ≤ x ∧ y ≠ x

kar je ekvivalentno

    ∀ x ∈ B . ∃ y ∈ B . y < x.               (5)

Padajočo verigo `b : N → B` definiramo z zaporedjem izbir: ker je `B` neprazna, lahko izberemo
neki element `b(0) ∈ B`. Denimo, da smo za neki `i ∈ N` že izbrali elemente `b(0), ..., b(i)`
tako, da velja

    b(i) < b(i-1) < ... < b(1) < b(0).

Ker `B` nima minimalnega elementa, `b(i)` ni minimalni, torej po (5) obstaja tak `y ∈ B`, da je
`y < b(i)`. Torej lahko izberemo `b(i+1) ∈ B`, da velja `b(i+1) < b(i)`. □

Pozor: v zgornjem dokazu smo uporabili *aksiom odvisne izbire*, ki je poseben primer
aksioma izbire in o katerem bomo še govorili.

**Izrek:** Naj bo `⊏` relacija na `A`. Tedaj so ekvivalentne naslednje izjave:

1. `⊏` je dobro osnovana
2. vsaka *neprazna* `S ⊆ A` ima `⊑`-minimalni element
3. A nima `⊏`-padajoče verige.

*Dokaz.*

`(1) ⇒ (2)` Denimo, da je `S ⊆ A` neprazna. Če uporabimo (1) na `A \ S` dobimo

     (∀ y ∈ A . (∀ x ∈ A . x ⊏ y ⇒ x ∈ A \ S) ⇒ y ∈ A \ S) ⇒ A \ S = ∅

Ker je `S` neprazna, dobimo zaporedje ekvivalentnih izjav:

     (∀ y ∈ A . (∀ x ∈ A . x ⊏ y ⇒ x ∈ A \ S) ⇒ y ∈ A \ S) ⇒ ⊥

     ¬ (∀ y ∈ A . (∀ x ∈ A . x ⊏ y ⇒ x ∈ A \ S) ⇒ y ∈ A \ S)

     ∃ y ∈ A . (∀ x ∈ A . x ⊏ y ⇒ x ∈ A \ S) ∧ y ∉ A \ S

     ∃ y ∈ A . (∀ x ∈ A . x ⊏ y ⇒ x ∉ S) ∧ y ∈ S

     ∃ y ∈ S . ∀ x ∈ A . x ⊏ y ⇒ x ∉ S

     ∃ y ∈ S . (∀ x ∈ A . x ⊏ y ⇒ x ∉ S)

Torej obstaja element `y ∈ S` z lastnostjo, da pod njim ni nobenega elementa iz
`S`, kar pa pomeni, da je `y` iskani minimalni element.

(2) ⇒ (3) Denimo, da je `a : N → A` padajoča veriga. Tedaj slika `{ a(n) | n ∈ N }` ne bi imela
minimalnega elementa, v nasprotju z (2).

(3) ⇒ (1) Denimo, da je `S ⊆ A` progresivna. Trdimo, da množica `C := A \ S` nima
minimalnega elementa. Če bi bil `c ∈ C` minimalni v `C`, bi to pomenilo

    ∀ x ∈ A . x ⊏ c ⇒ x ∉ C,

kar je ekvivalentno

    ∀ x ∈ A . x ⊏ c ⇒ x ∈ S.

Ker je `S` progresivna, od tod sledi `c ∈ S`, kar ni mogoče.

Dokazati moramo, da je `C` prazna. Če ne bi bila, bi lahko uporabili lemo in dobili padajočo
verigo v `A`, kar je v nasprotju s (3). □

**Izrek:** Naj bo `⊏` stroga urejenost na `A`. Tedaj so ekvivalentne naslednje izjave:

1. `⊏` je dobro urejena
2. vsaka *neprazna* množica `S ⊆ A` ima `⊏`-prvi element: to je tak `x ∈ S`, da velja
   `∀ y ∈ S . x ≠ y ⇒ x ⊏ y`.
3. A nima `⊏`-padajoče verige in `⊏` je sovisna

Dokaz je podoben dokazu prejšnjega izreka. Poskusite ga dokazati sami tako, da predelate
dokaz prejšnjega izreka.

**Primeri:**

1. Naravna števila `N` urejena z relacijo `<`.

2. Končna množica `{0, …, n}` urejena z relacijo `<`.

3. Če sta `(P, ≤_P)` in `(Q, ≤_Q)` dobri urejenosti, potem je dobro urejena tudi
   `P + Q` z relacijo `⊑`, ki `P` postavi pred `Q`:

        u ⊑ v ⇔
         (∃ x ∈ P . ∃ y ∈ Q . u = in₁(x) ∧ v = in₂(y)) ∨
         (∃ x ∈ P . ∃ y ∈ P . u = in₁(x) ∧ v = in₁(y) ∨ x ≤_P y) ∨
         (∃ x ∈ Q . ∃ y ∈ Q . u = in₂(x) ∧ v = in₂(y) ∨ x ≤_Q y).

4. S prejšnjim primerom lahko seštevamo dobre urejenosti, na primer `N + 3` je dobra
   urejenost

        0 < 1 < 2 < ⋯ < ω < ω + 1 < ω + 2

    Ali pa ω + ω

        0 < 1 < 2 < ⋯ < ω < ω + 1 < ω + 2 < ⋯


\chapter{Aksiom izbire}

\section{Odvisna izbira}

V dokazu o karakterizaciji dobro osnovanih relacij smo uporabili

**Aksiom odvisne izbire:**
Naj bo `A` neprazna množica in `R ⊆ A × A` celovita relacija, t.j.,

    ∀ x ∈ A . ∃ y ∈ A . x R y.

Tedaj obstaja zaporedje `a : N → A`, da za vse `n ∈ N` velja `a(n) R a(n+1)`.

\section{Aksiom izbire}

Aksiom odvisne izbire sledi iz aksioma izbire (tega ne bomo dokazali):

**Aksiom izbire (AC):** Vsaka družina nepraznih množic ima funkcijo izbire.

Povedano z drugimi besedami:

* formulacija AC: družina nepraznih množic ima funkcijo izbire
* vsaka surjekcija ima prerez ⇔ AC
* propaganda: [The Banach–Tarski Paradox](https://www.youtube.com/watch?v=s86-Z-CbaHA)


\chapter{Moč množic}

\section{Končne množice}

**Definicija:** *Standardna* končna množica z `n` elementi je

    [n] = {k ∈ N | k < n}

Torej:

    [0] = {}
    [1] = {0}
    [2] = {0, 1}
    [3] = {0, 1, 2}

**Definicija:** Množica je **končna**, če je izomorfna kaki standardni končni množici.

Velja naslednje (ne bomo dokazali): če je `A ≅ [m]` in `A ≅ [n]`, je `m = n`. Torej za končno
množico `A` obstaja natanko en `n ∈ N`, da velja `A ≅ [n]`. Temu n pravimo **moč** množice `A`,
saj nam pove, koliko elementov ima `A`. Moč množice `A` označimo z `|A|`.

Zakoni za moč množic:

    |[n]| = n

    |A × B| = |A| × |B|

    |A + B| = |A| + |B|

    |B^A| = |B|^|A|

Pravilo vključitve/izključitve:

    |A ∪ B| = |A| + |B| - |A ∩ B|

    |A ∪ B ∪ C| = |A| + |B| + |C| - |A ∩ B| - |B ∩ C| - |C ∩ A| + |A ∩ B ∩ C|

In podobno za unijo štirih ali več množic.


\section{Neskončne množice in njihova moč}

**Definicija:** Množica je **neskončna**, če ni končna.

**Izrek:** Množica `A` je neskončna natanko tedaj, ko obstaja injektivna preslikava `N → A`.

Dokaz:

(⇒) Denimo da `A` ni končna. Injektivno preslikavo `e : N → A` definiramo s
pomočjo akisoma odvisne izbire. Ker `A` ni izomorfna `[0]`, ni prazna, torej
obstaja `e(0) ∈ A`. Denimo, da smo že definirali `e` kot injektivno preslikavo
`[n]` → A. Tedaj jo lahko razširimo na injektivno preslikavo `e : [n+1] → A`
takole: ker `e` ni surjektivna (če bi bila, bi veljalo `A ≅ [n]`), obstaja `x ∈
A`, ki ni v sliki `e`. Torej *izberemo* `e(n) ∈ A`, ki ni v sliki. Tako dobimo
`e : N → A`, ki je injektivna.

(⇐) Denimo, da obstaja injektivna preslikava `e : N → A`. Če bi veljalo `A ≅
[n]`, bi imeli izomorfizem `f : A → [n]`. Tedaj bi bil `f ∘ e : N → [n]`
injektivna preslikava, ta pa ne obstaja (dokaz opustimo). □

\subsection{Moč množic}

Tudi neskončnim množicam želimo prirediti *moč*. Potrebujem taka "števila", da
lahko vsaki množici `A` priredimo "število" `|A|`, ki pove, koliko elementov
ima. Za končne množice so to kar naravna števila. Za splošne množice so to
**kardinalna števila**. Zaenkrat še ne bomo povedali natančno, kaj kardinalna
števila so. Lahko pa jih primerjamo med sabo, ne da bi zares vedeli, kaj so!

**Definicija:** Naj bosta `A` in `B` poljubni množici. Pravimo:

1. `A` ima enako moč kot `B`, pišemo `|A| = |B|`, ko obstaja bijektivna preslikava `A → B`.
2. `A` ima moč manjšo ali enako `B`, pišemo `|A| ≤ |B|`, ko obstaja injektivna preslikava `A → B`.
3. `A` ima moč manjšo kot `B`, pišemo `|A| < |B|`, če velja `|A| ≤ |B|` in `|A| ≠ |B|`.

**Izrek:** `|A| ≤ |B|` natanko tedaj, ko je `A = ∅` ali obstaja surjekcija `B → A`.

*Dokaz.*

(⇒) Denimo, da je `f : A → B` injektivna in `A ≠ ∅`. Torej obstaja neki `x₀ ∈ A`.
Tedaj definiramo surjektivno preslikavo `g : B → A` takole:

    g(y) = x  ⇔  f(x) = y ali x = x₀.

(⇐) Denimo, da je `A` prazna ali obstaja surjekcija `f : B → A`. Če je `A`
prazna, je edina preslikava `∅ → B` injektivna. Če je `f : B → A` surjektivna,
ima prerez, ki je injektivna preslikava. □

\subsection{Cantorjev izrek}

**Izrek (Cantor):** `|A| < |P(A)|`.

*Dokaz:*

Najprej dokažimo `|A| ≤ |P(A)|`. Iščemo injektivno preslikava `f : A → P(A)`.
Vzemimo `f(x) = {x}`. Zlahka preverimo, da je `f` res injektivna.

Sedaj dokazujemo, da ne obstaja bijekcija `A → P(A)`. Dokazali bomo, da ne obstaja
surjekcija `A → P(A)`, kar zadostuje. Denimo, da je `g : A → P(A)` poljbuna preslikava.
Trdimo, da `g` ni surjekcija. Res, podmnožica

    S = {x ∈ A | x ∉ g(x) }

ni v sliki `g`. Če bi bila, bi za neki `y ∈ A` veljalo `g(y) = S`, a to bi
vodilo v protislovje:

1. velja `y ∉ S`: če `y ∈ S` potem `y ∉ g(y) = S` po definiciji `S`.
2. velja `¬ (y ∉ S)`: če `y ∉ S` potem `y ∉ g(y) = S`a. □


\subsection{Števne in neštevne množice}

Moč množice `N` označimo z `ℵ₀`. (Zaenkrat še vedno ne vemo, kaj točno so
kardinalna števila, a privzemimo, da imamo kardinalno število `ℵ₀`, ki ustreza
moči množice `N`.)

**Definicija:** Množica `A` je *števna*, če velja velja `|A| ≤ ℵ₀`.

**Definicija:** Množica `A` je *neštevna*, če ni števna.

**Izrek:** Za vsako množico `A` so ekvivalentne naslednje izjave:

1. `A` je števna
2. obstaja injektivna preslikava `A → N`
3. `A` je prazna ali obstaja surjektivna preslikava `N → A`
4. obstaja surjektivna preslikava `N → 1 + A`
5. `A` je končna ali izmoforna `N`

Dokaz.

(1 ⇒ 2) če je `A` števna, velja `|A| ≤ ℵ₀ = |N`, torej obstaja injektivna `A →
N` po definiciji relacije `≤`.

(2 ⇒ 3) To sledi neposredno iz zgornjega izreka

(3 ⇒ 4) Denimo, da je `A` prazna ali obstaja surjektivna preslikava `N → A`:

1. Če je `A = ∅`, potem seveda obstaja surjektivna preslikava `N → 1 + A`, in sicer
`n ↦ in₁()`.

2. Če obstaja surjektivna preslikava `f : N → A`, potem lahko definiramo surjektivno
preslikavo `g : N → 1 + A` s predpisom

        g(0) = in₁()
        g(n) = in₂(f(n-1)) za n > 1

(4 ⇒ 5) Denimo, da obstaja surjektivna preslikava `r : N → 1 + A`. Dokazali
bomo, da je `A` izomorfna `N`, če ni končna. Predpostavimo torej, da `A` ni
končna. Preslikva `r` ima prerez `s : 1 + A → N`, ki je seveda injektivna
preslikava. Preslikva `s ∘ in₂ : A → N` je torej kompozitum injektivnih
preslikav, zato je injektivna. Ker `A` ni končna, obstaja tudi injektivna
preslikava `N → A`. Po izreku Cantor-Schröder-Bernstein je torej `A` izomorfna
`N`.

(5 ⇒ 1) Če je `A` končna, je števna, ker očitno velja `A = |[n]| ≤ |N| = ℵ₀`. Če
je `A` izomorfna `N`, potem seveda velja `|A| = |N| ≤ |N| = ℵ₀`. □


**Izrek:** `N × N ≅ N`.

Pravimo, da je družina `A : I → Set` **števna**, če je števna njena indeksna
množica `I`.

**Izrek:** Unija števne družine števnih množic je števna.

*Dokaz.*

Najprej obravnavajmo unijo družine `A : N → Set`, kjer je `A_n` števna za vse `n ∈ N`.
Tu uporabimo aksiom izbire, da dokažemo števnost unije. Za vsak `n ∈ N` obstaja
surjektivna preslikava `N → A_n + 1`. Po aksiomu izbire obstaja preslikava

    e : ∏_{n ∈ N} { f : N → A_n + 1 | f surjekcija }.

Definiramo `e' : N × N → 1 + ⋃_n A_n`:

    e'(n, k) = e(n)(k).

Trdimo, da je `e'` surjekcija iz `N × N` na `1 + ⋃_n A_n`.

Nato obravnavamo še unijo družine `A : I → Set`, kjer je `I` števna in `A_i`
števna za vsak `i ∈ I`. □

\subsection{Cantor-Schröder-Bernsteinov izrek in zakon trihotomije}

**Izrek** (Cantor-Schröder-Bernstein): Če obstajata injektivni preslikava `A → B`
in `B → A`, potem obstaja bijektivna preslikava `A → B`.

*Dokaz.* Dokaz je v priloženi datoteki [csb.pdf](./csb.pdf)


**Posledica:** Če `|A| ≤ |B|` in `|B| ≤ |A|`, potem `|A| = |B|`.

*Dokaz.* To sledi neposredno iz izreka CSB in definicije `≤`. □

Brez dokaza omenimo še, da velja **zakon trihotomije**: za vsaki množici `A` in `B`
velja `|A| < |B|` ali `|A| = |B|` ali `|B| < |A|`, oziroma ekvivalentno

    |A| ≤ |B| ∨ |B| ≤ |A|.

Relacija `≤` potemtakem uredi moči množic linearno.

TODO: podaj referenco na dokaz (verjetno Prijateljeve Strukture 1).

\subsection{Moč kontinuuma in Cantorjeva hipoteza}

Vemo, da ima množica realnih števil `R` enako moč kot `P(N)`, potenčna množica
naravnih števil (to boste naredili na vajah). Tej moči pravimo **moč
kontinuuma** (ker je "kontinuum" tudi staro ime za `R`).

Že Goerg Cantor, utemelitelj teorije množic, še je vprašal naslednje vprašanje:

**Cantorjeva hipoteza:** Vsaka neštevna podmnožica realnih števil izomorfna `R`.

Povedano, z drugimi besedami, po moči ni nobene množice strogo med `N` in `R`.
Cantorjeva hipoteza je ostala odprta dlje časa. Dokončno je Cohen pred dobrega
pol stoletja dokazal naslednje:

**Izrek (Cohen):** Iz Zermelo-Fraenkelovih aksiomov teorije množic Cantorjeve
hipoteze ne moremo niti dokazati niti ovreči.

Pravimo, da je Cantorjeva hipoteza **neodvisna** od aksiomov teorije množic.

Poznamo še posplošeno Cantorjevo hipotezo, ki se glasi:

**Posplošena Cantorjeva hipoteza:** Če je `|A| ≤ |B| ≤ |P(A)|`, potem je `|B| =
|A|` ali `|B| = |P(A)|`.

Tudi posplošena Cantorjeva hipoteza je nedovisna od aksiomov teorije množic.
Danes vemo zelo veliko o tej hipotezi in poznamo, še mnoge druge izjave o
množicah, ki so neodvisne od Zermelo-Fraenkelovih aksiomov teorije množic. Ti
veljajo za nekakšno uradno različičo teorije množic in jih bomo obravnavali na
naslednjih predavanjih.



\chapter{Zermelo-Fraenkelovi aksiomi}
\textbf{To poglavje še ni predelano v {\LaTeX}.}
%\chapter{Kodiranje matematičnih objektov z množicami}

Z množicami smo izrazili številne matematične objekte, na primer:

* preslikavo `f : A → B` lahko izrazimo kot funkcijsko relacijo med `A` in `B`, torej kot
  podmnožico `A × B`,

* kvocientna množica `A/R` je množica ekvivalenčnih razredov, ekvivalenčni razredi so spet
  množice,

Ali je možno vse matematične objekte predstaviti z množicami? Da!

\subsubsection{Urejeni pari}

Par `(x, y)` lahko predstavimo z množico `{{x}, {x,y}}`. Tako dobimo

    A × B := { {{x}, {x,y}} | x ∈ A, y ∈ B }

\subsubsection{Vsota}

Elemente vsote `A + B` kodiramo takole:

     \inl(x) = (x, 0) = {{x}, {x,∅}}
     \inr(x) = (x, 1) = {{x}, {x,{∅}}}

\subsubsection{Naravna števila}

Na množicah definiramo operacijo naslednik:

    x⁺ := x ∪ {x}

Naravna števila nato kodiramo tako, da za ničlo vzamemo `∅` in uporabljamo
operacijo naslednik:

    0 = ∅
    1 = ∅⁺ = {0} = {∅}
    2 = 1⁺ = {0, 1} = {∅, {∅}}
    3 = 2⁺ = {0, 1, 2} = {∅, {∅}, {∅, {∅}}}
    4 = 3⁺ = {0, 1, 2, 3} = {∅, {∅}, {∅, {∅}}, {∅, {∅}, {∅, {∅}}}}
    ...

Vidimo, da je vsako naravno število kar množica svojih predhodnikov.

\subsubsection{Cela števila}

Cela števila so kvocient `N × N`:

    Z = (N × N)/∼

kjer je

    (a,b) ∼ (c,d) ⇔ a + d = c + b.

Urejeni par `(a, b)` predstavlja razliko števil `a` in `b`.

\subsubsection{Racionalna števila}

Racionalna števila so kvocient:

    Q = (Z × {n ∈ N | n > 0})/≈

kjer je

    (a,m) ≈ (b,n) ⇔ a n = b m.

\subsubsection{Realna števila}

Realno število je Dedekindov rez, torej podmnožica `Q`.

In tako naprej. Ne pravimo, da je kodiranje vseh matematičnih objektov z množicami vedno
dobra ideja, vendar pa je dejstvo, da je to možno, pomembno spoznanje osnov matematike. Iz
njega na primer sledi tole: če je teorija množic neprotislovna, potem je neprotislovna
tudi vsa matematika, ki jo lahko kodiramo z množicami (torej več ali manj vsa običajna
matematika).

\section{Aksiomi teorije množic}

Zermelo-Fraenkelovi aksiomi teorije množic:

1. **Ekstenzionalnost:** množici `A` in `B`, ki imata iste elemente, sta enaki.

2. **Neurejeni par**: za vsak `x` in `y` je `{x, y}` množica, ki vsebuje natanko `x` in `y`:

        ∀ x y z . z ∈ {x, y} ⇔ z = x ∨ z = y

   Okrajšava: `{x} = {x, x}`.

3. **Unija:** za vsako množico `A` je `⋃ A` množica, ki vsebuje natanko vse
   elemente množic iz `A`

        ∀ A x . x ∈ ⋃ A ⇔ ∃ B ∈ A . x ∈ B

4. **Prazna množica:** množica `∅` nima elementa:

        ∀ x . x ∉ ∅

5. **Neskončna množica:** obstaja množica, ki vsebuje `∅` in je zaprta za operacijo naslednik
   (`x⁺ = x ∪ {x}`).

        ∃ A . ∅ ∈ A ∧ ∀ x ∈ A . x⁺ ∈ A

6. **Podmnožica:** za vsako množico `A` in formulo `φ` je `{x ∈ A | φ(x)}`
   množica, ki vsebuje natanko vse element iz `A`, ki zadoščajo `φ`:

        ∀ y . y ∈ {x ∈ A | φ(x)} ⇔ φ(y)

7. **Potenčna množica:** za vsako množico `A` je `P(A)` množica, ki vsebuje
   natanko vse njene podmnožice:

        ∀ S . S ∈ P(A) ⇔ S ⊆ A

8. Zamenjava: če je `A` množica in `f : A → Set` preslikava, je razred

        { y ∈ V | ∃ x ∈ A . y = f(x) }

   množica.

9. **Dobra osnovanost:** relacija `∈` je dobro osnovana.

10. **Aksiom izbire:** vsaka družina nepraznih množic ima funkcijo izbire


\section{Aksiom izbire}

**Definicija:** **Veriga** v delni urejenosti `(P, ≤)` je taka podmnožica `V ⊆
P`, ki je z `≤` linearno urejena, kar pomeni `∀ x y ∈ V . x ≤ y ∨ y ≤ x`.

Primeri:

* Če je `(P, ≤)` linearno urejena, je vsaka podmnožica veriga

* V `(P(Q), ⊆)` imamo neštevno verigo

        V = {S ∈ P(Q) | S je doljna množica}

  Množica `S ⊆ Q` je *doljna*, če velja `∀ x y ∈ Q . x ≤ y ∧ y ∈ Q ⇒ x ∈ Q`.

**Zornova lemma:** Če ima v delni urejenosti `(P, ≤)` vsaka veriga zgornjo mejo,
potem ima `P` maksimalni element.

Dokaz: dokaz se naslanja na aksiom izbire in Bourbaki-Wittov izrek o negibnih točkah (glej
spodaj). Naj bo `C` množica vseh verig v `P`. Uredimo jo z `⊆`. Na njej definiramo preslikavo
`f : C → C`, ki razširi verigo, če ni maksimalna, sicer je ne spremeni (tu uporabimo
izbiro):

* Če je `V ∈ C` maksimalna veriga v `P` (glede na `⊆`), definiramo `f(V) := V`.
* Če `V ∈ C` ni maksimalna veriga v `P`, potem obstaja tak `x ∈ P \ V`, da je `V
  ∪ {x}` spet veriga. V tem primeru *izberemo* tak `x` in definiramo `f(V) := V
  ∪ {x}`.

Po izreku Bourbaki-Witt ima `f` negibno vrednost `V ∈ C`. Ta `V` je maksimalna
veriga `V`, saj bi sicer veljalo, da je `V = f(V) = V ∪ {x}` za neki `x ∉ V`,
kar ni možno. Naj bo `m` zgornja meja za verigo `V`. Trdimo, da je `m`
maksimalni element v `P`: denimo, da velja `m ≤ y` za `m ∈ P`. Ker je `V ∪ {y}`
veriga, ki vsebuje maksimalno verigo `V`, sledi `V = V ∪ {y}`, od tod pa `y ∈ V`
ter `y ≤ m`. Torej je `m = y` in `m` je res maksimalni element. □

**Definicija:** Naj bo `(P, ≤)` delna ureditev. Preslikava `f : P → P` je **progresivna**, ko
velja `x ≤ f(x)` za vsak `x ∈ P`.

*Opomba:* progresivna preslikav ni nujno monotona (poiščite protiprimer!).

**Izrek (Bourbaki-Witt):** Naj bo `(P, ≤)` neprazna delna ureditev, v kateri ima
vsaka veriga zgornjo mejo in `f : P → P` progresivna preslikava. Tedaj ima `f`
negibno točko: to je tak `x ∈ P`, da velja `f(x) = x`.

Dokaz: opuščen.

**Izrek:** V teoriji množic *brez* aksioma izbire so naslednje izjave ekvivalentne:

1. Aksiom izbire
2. Zornova lema
3. Princip dobre urejenosti: vsaka množica ima dobro ureditev

Dokaz:

(1 ⇒ 2) Glej Zornovo lemo.

(2 ⇒ 3) Skica dokaza: naj bo `A` poljubna množica, ki jo želimo dobro urediti.

Definirajmo *delne* dobre ureditev množice `A`: to so pari `(B,R)`, kjer je `B ⊆ A`
in `R ⊆ B × B` dobra ureditev na `B`. Za delni dobri ureditvi `(B,R)` in
`(C,Q)` pravimo, da je `(C,Q)` *razširitev* `(B,R)`, kadar velja `B ⊆ C`, `R ⊆ Q` in
še, da je `B` začetni segment v `C`, kar pomeni:

     ∀ x y ∈ C: x Q y ∧ y ∈ B ⇒ x ∈ B.

Kadar je `(C,Q)` razširitev `(B,R)`, pišemo `(B,R) ≼ (C,Q)`. Naj bo `P` množica vseh delnih
dobrih ureditev množice `A`,

    P = { (B, R) | B ⊆ A in R ⊆ B × B in R je dobra ureditev B },

urejena z relacijo `≼`. Očitno je `≼` delna ureditev. Trdimo, da imajo verige v
`P` zgornje meje glede na `≼`: če je `V ⊆ P` veriga dobro urejenih podmnožic
`A`, je njena zgornja meja `(D,S)` kar unija po komponentah:

     D := ⋃ {B | (B, R) ∈ V}
     S := ⋃ {R | (B, R) ∈ V}

Preverimo, da velja `(D,S) ∈ P`. Očitno je `(D,S)` stroga linearna ureditev
(vaja). Denimo, da bi v `(D,S)` imeli neskončno padajočo verigo

     ... S x₃ S x₂ S x₁ S x₀.

Obstaja `(B,R) ∈ V`, da je `x₀ ∈ B`. Potem bi bila `x₀, x₁, x₂, x₃, ...`
padajoča veriga v `(B,R)`, kar ni možno, saj je `(B,R)` dobro urejena. Res, ker
je `xᵢ ∈ V`, obstaja `(C,Q)`, da je `xᵢ ∈ C`. Če velja `(B,R) ≼ (C,Q)`, potem
`xᵢ ∈ B` po definicijo `≼`. Če velja `(C,Q) ≼ (B,R)`, potem `xᵢ ∈ B`, ker velja
`C ⊆ B`. Torej je `(D,S)` res delna ureditev `P`.

Preverimo še, da velja `(B,R) ≼ (D,S)` za vsak `(B,R) ∈ V`. Denimo, da je `y ∈ D`,
`x ∈ B` in `y S x`. Obstaja `(C,Q) ∈ V`, da je `y ∈ C`. Če velja `(C,Q) ≼ (B,R)`,
potem `y ∈ C ⊆ B`. Če pa velja `(B,R) ≼ (C,Q)`, potem je `y ∈ B` po definiciji `≼`.

Po Zornovi lemi obstaja maksimalni element `(B,R)` v `P`. Trdimo, da je `B = A`. Če bi namreč
obatajal `x ∈ B \ A`, bi lahko razširili `(B,R)` na večjo dobro ureditev tako, da bi dodali `x`
na konec `B`:

    (B ∪ {x}, R')

    y R' z ⇔ z = x ∧ (y,z) ∈ R

To ni možno, ker je `(B,R)` maksimalna delna ureditev. Torej je res `A = B` in
našli so dobro ureditev `A`.

(3 ⇒ 1) Naj bo `A : I → Set` družina nepraznih množic. Naj bo `≺` dobra ureditev
na uniji `⋃ A`. Ker so vse množice `Aᵢ` neprazne, ima vsaka od njih prvi element
glede na `≺`. Torej lahko definiramo funkcijo izbire `f` s predpisom

    f(i) = prvi element Aᵢ. □

**Izrek:** Vsak vektorski prostor ima vektorsko bazo.

Dokaz: Naj bo `L` vektorski prostor. Definiramo množico

    P = { B ⊆ L | B je linearno neodvisna }.

Množico `P` delno uredimo z relacijo `⊆`. Trdimo, da imajo verige v `P` zgornje
meje: zgornja meja verige `V ⊆ P`, je kar njena unija `⋃_(B ∈ V) B`. Seveda je
treba preveriti, da je unija verige linearno neodvisnih množic spet linearno
neodvisna (vaja). Po Zornovi lemi obstaja maksimalni element v `P`, torej
maksimalna linearno neodvisna množica `B` v `L`. To pa je seveda vektorksa baza
za `L`. □



\chapter{Kumulativna hierarhija}
\textbf{To poglavje še ni predelano v {\LaTeX}.}
%\section{Kumulativna hierahija}

Če lahko vse matematične objekte kodiramo z množicami, potem lahko na razred
vseh množic `Set` gledamo kot na celotni matematični svet. Razred `Set` ima
zanimivo strukturo, ki ji pravimo **kumulativna hierarhija**. Namreč, s pomočjo
aksiomov, ki jih bomo spoznali kasneje, lahko tvorimo vse množice iz `∅` z
oparacijama potenčna množica in unija. Postopek je **transfiniten**, kar pomeni,
da se nikoli ne konča in da po svoje številu presega moč vsake množice.

    V₀ = ∅
    V₁ = P(V₀) = {∅}
    V₂ = P(V₁} = {∅, {∅}}
    V₃ = P(V₂) = {∅, {∅}, {{∅}}, {∅, {∅}}}
    ...
    V_ω = ⋃ {Vᵢ | i < ω}
    V_(ω+1) = P(V_ω)
    V_(ω+2) = P(V_(ω+1))
    ...
    V_(ω + ω) = ⋃ {Vᵢ | i < ω + ω)}
    ...

Stopnje konstrukcija indeksiramo s t.i. **ordinalnimi števili**, ki jih bomo spoznali.

\section{Ordinalna števila}

**Definicija:** Množica $x$ je **tranzitivna**, če za vsak `y ∈ x` velja `y ⊆ x`.

Izraz *tranzitivna* je smiselen, ker govori o tranzitivnosti relacije `∈`, saj lahko pogoj `y ∈ x ⇒ y ⊆ x` zapišemo kot `z ∈ y ∧ y ∈ x ⇒ z ∈ x`.

Primeri tranzitivnih množic: `∅`, `{∅}`, `{∅, {∅}}`, `{{{∅}}, {∅}, ∅}`

**Definicija:** Množica je **hereditarno tranzitivna**, če so vsi njeni elementi tranzitivne množice.

(V splošnem se izraz "hereditarno" uporablja, kadar se lastnost nanaša na elemente, pomdnožice, ali podstrukture, se pravi na "potomce".)

**Definicija:** **Ordinalno število** je tranzitivna in hereditarno tranzitivna množica. Razred vseh ordinalnih števil označimo z `On`.

Definirajmo relacijo *naslednik* na množicah: `x^+ = x ∪ {x}`.

Preverimo lahko tole:

1. `∅ ∈ On`
2. če je `α ∈ On`, potem je tudi `α^+ ∈ On`.
3. `On` je zaprt za unije: če je `S ⊆ On` množica, potem je `U S ∈ On`.

Sedaj lahko gradimo `On` iterativno:

* `0 = ∅`
* z opreacijo naslednik dobimo naravna števila `n = {0, ..., n-1}`
* unija vseh naravnih števil je `ω`.
* z naslednik gradimo `ω`, `ω + 1`, `ω + 2`, ...
* unija teh je `ω + ω`
* in tako naprej

\section{Kardinalna števila}

**Definicija:** Ordinalno število `α` je **kardinalno**, če za vsak `β < α`
velja, da ne obstaja injektivna preslikava `α → β`.

\subsection{Zakon trihotomije}

V tem razadelku podamo še oris dokaza, da je aksiom izbire ekvivalenten zakonu trihotomije.

**Definicija:** Naj bo `(P, <)` dobra urejenost. Podmnožica `I ⊆ P` je **začetni
segment**, če je doljna množica: iz `x < y` in `y ∈ I` sledi `x ∈ I`.

**Definicija:** Naj bosta `(P, <_P)` in `(Q, <_Q)` dobri urejenosti. Pravimo, da
je preslikava `e : P → Q` **vložitev**, kadar velja:

1. `e` je strogo monotona in
2. slika `e_(P)` je začetni segment v `Q`.

Vložitev je injektivna preslikava.

**Lemma 1:** Naj bosta `(P, <_P)` in `(Q, <_Q)` dobri urejenosti. Če obstaja
injektivna preslikava `P → Q`, potem obstaja tudi vložitev `P → Q`.

Dokaz: opuščen.

**Lemma 2:** Naj bosta `(P, <_P)` in `(Q, <_Q)` dobri urejenosti. Tedaj bodisi
obstaja vložitev `P → Q` ali vložitev `Q → P`.

Dokaz: opuščen.

**Izrek:** Aksiom izbire je ekvivalenten zakonu trihotomije: za vse množice `X` in `Y` velja
`|X| ≤ |Y|` ali `|Y| ≤ |X|`.

Dokaz:

Najprej predpostavimo, da velja aksiom izbire. Naj bosta `X` in `Y` množici. Ker
velja aksiom izbire, lahko `X` in `Y` dobro uredimo, denimo z relacijama `<_X`
in `<_Y`. Iz zgornje leme sledi, da obstaja vložitev `X → Y` ali `Y → X`.
Ker so vložitve injektivne, torej velja `|X| ≤ |Y|` ali `|Y| ≤ |X|`.

Predpostavimo zdaj, da za vse množice `X` in `Y` velja `|X| ≤ |Y|` ali `|Y| ≤
|X|`. Dokazali bomo, da lahko vsako množico dobro uredimo, iz česar sledi aksiom izbire.



% \chapter{Množice}
\label{chap:mnozice}

V drugem delu predmeta bomo spoznali osnove teorije množic. Najprej pa
se bomo posvetili še naravnim številom in Peanovim aksiomom.

%%%%%%%%%%%%%%%%%%%%%%%%%%%%%%%%%%%%%%%%%%%%%%%%%%%%%%%%%%%%%%%%%%%%%%
\section{Naravna števila}
\label{sec:naravna-stevila}

Naravna števila
%
\begin{equation*}
  0, 1, 2, 3, 4, 5, 6, 7, 8, 9, 10, 11, 12, \ldots
\end{equation*}
%
vsi že dobro poznamo iz osnovne šole.\footnote{V teh zapiskih in v
  logiki nasploh vzamemo za prvo naravno število $0$. V osnovni šoli
  in drugje pa ponavadi za prvo naravno število jemljemo~$1$.} V tem
razdelku pokažimo, kako uvedemo naravna števila kot formalno teorijo v
logiki. V splošnem \emph{formalna teorija} opisuje neko matematično
strukturo ali družino struktur in je podana z osnovnimi simboli
(konstantami in operacijami), aksiomi in pravili sklepanja.

Teorijo naravnih števil, ki jo imenujemo tudi \emph{Peanova
  aritmetika}, sestoji iz konstante $0$, enočlene operacije
naslednik~$\suc{n}$ ter dvočlenih operacij seštevanje~$m + n$ in
množenje~$m \cdot n$. Množenje ia prednost pred seštevanjem, se pravi,
da je $k + m \cdot n = k + (m \cdot n)$ in ne $(k + m) \cdot n$.
Aksiomi in pravila sklepanja se glasijo:
%
\begin{enumerate}
  \item Nič ni naslednik:
  %
  \begin{equation*}
    \inferrule{ }{\suc{n} \neq 0}    
  \end{equation*}
  %
  \item Če sta naslednika enaka, sta števili enaki:
  %
  \begin{equation*}
    \inferrule{\suc{m} = \suc{n}}{m = n}
  \end{equation*}
  %
  \item Pravili za seštevanje:
  %
  \begin{mathpar}
    \inferrule{ }{0 + n = n}
    \and
    \inferrule{ }{\suc{m} + n = (m + n\suc{)}}    
  \end{mathpar}
  %
  \item Pravili za množenje:
  %
  \begin{mathpar}
    \inferrule{ }{0 \cdot n = 0}
    \and
    \inferrule{ }{\suc{m} \cdot n = m \cdot n + n}
  \end{mathpar}
  %
  \item Princip indukcije:
  %
  \begin{equation*}        
    \inferrule{\phi(0) \\ \xall{m}{\NN}{\phi(m) \lthen \phi(\suc{m})}}{\phi(n)}
  \end{equation*}
\end{enumerate}
%
Pri običajnem računanju z naravnimi števili uporabljamo vse znanje, ki
smo ga pridobili v šoli. Ko pa obravnavamo naravna števila kot
formalno teorijo, smemo uporabljati \emph{samo} konstante in simbole,
ki jih vpeljemo v teoriji, in se sklicevati \emph{samo} na Peanove
aksiome. Denimo, ker teorija ne vpelje simbolov $1$ in $2$, ju ne
smemo uporabljati, razen če ju prej definiramo kot okrajšavi za
$\suc{0}$ in $\suc{(\suc{0})}$. Prav tako ne smemo omenjati odštevanja
števil, ker to ni ena od operacij $+$ in $\cdot$, ne smemo govoriti o
parnosti števil, ne da bi prej ta pojem definirali, itn. Tudi osnovne
lastnosti seštevanja in množenja, kot sta komutativnost in
asociativnost, ne smemo uporabiti, če ju prej ne dokažemo. Matematiki
so seveda preverili, da vse običajne lastnosti števil dejansko sledijo
iz Peanovih aksiomov.

Glavno orodje pri dokazovanju lastnosti naravnih števil je princip
indukcije. V besedilu ga uporabimo takole:
%
\begin{quote}
  \em
  %
  Dokazujemo $\phi(n)$ z indukcijo po~$n$:
  %
  \begin{enumerate}
    \item Baza indukcije: (Dokaz, da velja $\phi(0)$.)
    \item Indukcijski korak: denimo, da za naravno število $m$ velja
      $\phi(m)$. (Dokaz, da velja $\phi(\suc{m})$.)
  \end{enumerate}
\end{quote}
%
Za zgled dokažimo, da je seštevanje komutativno. To naredimo v nekaj
korakih.

\begin{izjava}
  \label{izjava:peano-n-plus-0}
  Za vsako naravno število $m$ velja $m + 0 = m$.
\end{izjava}

\begin{dokaz}
  Dokazujemo z indukcijo. Baza indukcije: $0 + 0 = 0$ po enem od
  Peanovih aksiomov.
  %
  Indukcijski korak: denimo, da za naravno število $k$ velja $k + 0 =
  k$. Tedaj je $\suc{k} + 0 = \suc{(k + 0)} = \suc{k}$, kjer smo v
  prvem koraku uporabili enega od Peanovih aksiomov in v drugem
  indukcijsko predpostavko.
\end{dokaz}


\begin{izjava}
  \label{izjava:peano-m-plus-suc-n}
  Za vsaki naravni števili $m$ in $n$ velja $m + \suc{n} = \suc{(m + n)}$.
\end{izjava}

\begin{dokaz}
  Izjavo dokažemo z indukcijo po $m$.
  Baza indukcije: $0 + \suc{n} = \suc{n} = \suc{(0 + n)}$.
  %
  Indkucijski korak: denimo, da za naravno število $k$ velja $k +
  \suc{n} = \suc{(k + n)}$. Tedaj je
  %
  \begin{equation*}
    \suc{k} + \suc{n} = 
    \suc{(k + \suc{n})} =
    \suc{{\suc{(k + n)}}} =
    \suc{(\suc{k} + n)}.
  \end{equation*}
  %
\end{dokaz}

\begin{izjava}
  Za vsaki naravni števili $m$ in $n$ velja $m + n = n + m$.
\end{izjava}

\begin{dokaz}
  Izjavo dokažemo z indukcijo po $m$.
  Baza indukcije: $0 + n = n = n + 0$, kjer smo v prvem koraku uporabili Peanov aksiom in v drugem Izjavo~\ref{izjava:peano-n-plus-0}.
  %
  Indukcijski korak: denimo, da za naravno število $k$ velja $k + n = n + k$. Tedaj je
  %
  \begin{equation*}
    \suc{k} + n =
    \suc{(k + n)} =
    \suc{(n + k)} =
    n + \suc{k}.
  \end{equation*}
  %
  V prvem koraku smo uporabili Peanov aksiom, v drugem indukcijsko predpostavko, v tretjem pa Izjavo~\ref{izjava:peano-m-plus-suc-n}.
\end{dokaz}

%%%%%%%%%%%%%%%%%%%%%%%%%%%%%%%%%%%%%%%%%%%%%%%%%%%%%%%%%%%%%%%%%%%%%%
\section{Množice}
\label{sec:naivne-mnozice}

Množice so osnovni gradniki matematičnih objektov in struktur. V tem razdelku obravnavamo množice \emph{naivno}, se pravi s pomočjo neformalnih razlag. Samo formalno teorijo množic in aksiome bomo obravnavali v razdelku~\ref{sec:zfc}.

Množico si predstavljamo kot skupek ali zbirko poljubnih objektov, jim pravimo \emph{elementi} množice. Dejstvo, da je $x$ element množice $A$ zapišemo $x \in A$. Če $x$ ni element $S$, pišemo $x \not\in S$ kot okrajšavo za $\lnot (x \in S)$. Množica ni odvisna od tega, kako jo opišemo ali skonstruiramo, ampak le od tega, kateri elementi so v njej. To dejstvo izraža \emph{aksiomom o ekstenzionalnosti}, ki pravi, da sta množici $A$ in $B$ enaki natanko tedaj, ko vsebujeta iste elemente, kar zapišemo s formulo kot
%
\begin{equation*}
  A = B \liff \uall{x}{x \in A \liff x \in B}.
\end{equation*}
%
Množice gradimo iz osnovnih množic s pomočjo operacij.

\subsection{Osnovne množice}
\label{sec:osnovne-mnozice}

Najpreprostejša osnovna množica je \emph{prazna množica}, ki jo označimo z $\emptyset$. Dejstvo, da prazna množica ne vsebuje nobenih elementov izrazimo z aksiomom o prazni množici,
%
\begin{equation*}
  \xuall{x}{x \not\in \emptyset}.
\end{equation*}
%
V zvezi s prazno množico omenimo, da za vsako izjavo $\phi$ velja
%
\begin{equation*}
  \xall{x}{\emptyset}{\phi},
\end{equation*}
%
kar dokažemo takole: naj bo $x \in \emptyset$ poljuben. Ker velja $x \not\in \emptyset$, je to protislovje, od koder smemo sklepati $\phi$. Podobno za vsako izjavo $\phi$ velja
%
\begin{equation*}
  \lnot\xsome{x}{\emptyset}{\phi}.
\end{equation*}

\begin{vaja}
  Za katere množice $S$ velja $(\xall{x}{S}{\phi(x)}) \lthen   \xsome{x}{S}{\phi(x)}$?
\end{vaja}

Za osnovno množico vzamemo tudi množico naravnih števil~$\NN$, ki smo jo že spoznali v razdelku~\ref{sec:naravna-stevila}.

\subsection{Konstrukcije množic}
\label{sec:konstrukcije-mnozic}

Iz osnovnih množic lahko konstruiramo nove s pomočjo naslednjih operacij.

\subsubsection{Končne množice}
\label{sec:koncne-mnozice}

Naj bodo $a_1, \ldots, a_n$ poljubni objekti. Tedaj lahko tvorimo množico
%
\begin{equation*}
  \set{a_1, a_2, \ldots, a_n}
\end{equation*}
%
ki sestoji iz naštetih elementov, to je
%
\begin{equation*}
  \uall{x}{x \in \set{a_1, \ldots, a_n} \liff
    x = a_1 \lor \cdots x = a_n}.
\end{equation*}
%
Poseben primer take množice je \emph{enojec} $\set{a}$, za katerega velja
%
\begin{equation*}
  \uall{x}{x \in \set{a} \liff x = a}.
\end{equation*}


\begin{vaja}
  Ali je $\set{a, b} = \set{b, a}$? Ali je $\set{a, a, b} = \set{a, b}$? Uporabi aksiom o ekstenzionalnosti.
\end{vaja}

\subsubsection{Unija in presek}
\label{sec:unija-presek}

Družina množic.

Unije, preseki.

\subsubsection{Podmnožica}
\label{sec:podmnozica}

Podmnožica (separacija).

\subsubsection{Potenčna množica}
\label{sec:potencna-mnozica}

Potenčna množica.

\subsubsection{Kartezični produkt}
\label{sec:kartezicni-produkt}

Kartezični produkt. Produkt s prazno.

\subsubsection{Eksponentna množica}
\label{sec:eksponentna-mnozica}

Eksponentna množica. Eksponent s prazno.

\subsubsection{Vsota}
\label{sec:vsota-mnozic}

Disjunktna unija.

\subsubsection{Razlika in komplement}
\label{sec:vsota-mnozic}


\section{Funkcije}
\label{sec:funkcije}


Funkcija, neformalna definicija.

Kompozitum, asociativnost kompozituma. Identiteta.

Inverz funkcije. Inverz je enoličen, če obstaja.

Slika in inverzna slika.

Kdaj obstaja inverz? Surjektivna, injektivna, bijektivna funckija.

Epi in mono.

Sekcija in retrakcija.

Sekcija je mono, retrakcija je epi.

Standardne bijekcije za vsoto, produkt in eksponent.

\section{Relacije}
\label{sec:relacije}

Definicija relacije.

Nasprotna relacija. Komplement, unija.

\subsection{Funkcijske relacije}
\label{sub:funkcijske_relacije}


\subsection{Ekvivalenčne relacije}
\label{sub:ekvivalencne_relacije}


%Definirali smo pojem ekvivalenčne relacije in kvocienta množice po
%ekvivalenčni relaciji. Pokazali smo razne primere. Dokazali smo, da
%smemo definirati $f : A/{\sim} \to B$ na kvocientu tako, da definiramo
%$g : A \to B$, ki je skladen s~$\sim$.




Defincije. Primeri.

Ekvivalenčna relacija, generirana z relacijo.

Faktorska množica. Kako definiramo preslikavo na faktorski množici.

Kanonični razcep funkcije.

\subsection{Delna ureditev}
\label{sub:delna_ureditev}

Definicija delne ureditve. Primeri.

Linearna ureditev. Stroga linearna ureditev. Veriga.

Zgornja meja, spodna meja, infimum, supremum, maksimum, minimum, minimalni element, maksimalni element.



%%% Local Variables: 
%%% mode: latex
%%% TeX-master: "lmn"
%%% End: 



%--------------------------------------------------------------------
% BIBLIOGRAFIJA

\bibliographystyle{alpha}
\addcontentsline{toc}{chapter}{\numberline{}Literatura}
\markboth{}{Literatura}

{
\raggedright
\renewcommand{\markboth}[2]{}
\bibliography{literatura}
}


\end{document}

%%% Local Variables: 
%%% mode: latex
%%% TeX-master: t
%%% End: 
