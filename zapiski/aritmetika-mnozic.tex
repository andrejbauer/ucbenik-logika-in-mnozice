\chapter{Aritmetika množic}
\label{cha:artimetika-mnozic}

Množice lahko \df{tvorimo} ali \df{konstruiramo} iz drugih množic na različne načine. V
tem razdelku bomo spoznali tri: zmnožek, vsoto in eksponent.
%
Ostale konstrukcije pridejo na vrsto kasneje, ko bomo že nekaj vedeli o logiki.

Ko opišemo novo konstrukcijo množic, jo moramo natančno opredeliti. Pri tem se naslonimo na pravilo ekstenzionalnosti, ki pove, da je množica opredeljena s svojimi elementi.
%
Če torej želimo določiti elemente neke množice~$A$, to lahko storimo s pogojem oblike
\nls{$x \in A$ natanko tedaj, ko \dots}, ali s formulo
%
\begin{equation*}
  x \in A \ \liff\  \ldots.
\end{equation*}
%
Take primere smo že videli:
%
\begin{align*}
  x \in \set{a, b} &\ \liff\  x = a \lor x = b,\\
  x \in \one &\ \liff\  x = \unit,\\
  x \in A \cup B &\ \liff\  x \in A \lor x \in B,\\
  x \in \emptyset &\ \liff\  \bot.
\end{align*}
%
Tudi v nadaljevanju bomo sledili temu receptu in nove konstrukcije množic opisali tako, da bomo natančno opredelili njihove elemente.

\section{Zmnožek}
\label{sec:zmnozek}

V srednji šoli ste že spoznali kartezični produkt ali zmnožek, katerega elementi so urejeni pari.
Tu podajmo vse sestavine te konstrukcije, previdno in podrobno.

\begin{pravilo}[Tvorba zmnožka]
  \label{pravilo:zmnozek-tvorba}
  Za vsaki množici $A$ in $B$ je $A \times B$ množica, ki se imenuje \df{zmnožek} ali
  \df{kartezični produkt} $A$ in $B$.
\end{pravilo}

\noindent
%
Pravilo tvorbe pove, da lahko tvorimo novo množico $A \times B$, ne pove pa, kakšne
elemente ima. To je vsebina naslednjih dveh pravil, ki povesta, kako sestavimo in
razstavimo elemente zmnožka.

\begin{pravilo}[Vpeljava urejenih parov]
  \label{pravilo:zmnozek-vpeljava}
  %
  Za vse $a \in A$ in $b \in B$ je $(a, b) \in A \times B$. Element $(a, b)$ imenujemo
  \df{urejeni par}.
\end{pravilo}

\begin{pravilo}[Uporaba urejenih parov]
  \label{pravilo:zmnozek-uporaba}
    %
  Za vsak $p \in A \times B$ je $\fst(p) \in A$ \df{prva projekcija} in $\snd(p) \in B$
  \df{druga projekcija} elementa~$p$.
\end{pravilo}

\noindent
Potrebujemo še enačbe, ki povedo, kako računamo z urejenimi pari in kako jih primerjamo.

\begin{pravilo}[Računsko pravilo za urejene pare]
  \label{pravilo:zmnozek-racunanje}
  Za vse $a \in A$, $b \in B$ velja $\fst(a, b) = a$ in $\snd(a, b) = b$.
\end{pravilo}

\begin{pravilo}[Ekstenzionalnost urejenih parov]
  \label{pravilo:zmnozek-ekstenzionalnost}
  Za vse $p, q \in A \times B$ velja: če $\fst(p) = \fst(q)$ in $\snd(p) = \snd(q)$,
  potem $p = q$.
\end{pravilo}

Kadar imamo opravka z večimi zmnožki, na primer $A \times B$ in $C \times D$, bi lahko
prišlo do zmede glede projekcij. Takrat jih opremimo še z dodatnimi oznakami množic, da
razločimo projekciji $\fst[A][B] : A \times B \to A$ in $\fst[C][D] : C \times D \to C$,
in podobno za~$\snd$.

Malo bolj naivna konstrukcija zmnožka bi se glasila takole: kartezični produkt
$A \times B$ je množica vseh urejenih parov $(a, b)$, kjer je $a \in A$ in $b \in B$. A
taka konstrukcija ni popolna, saj ne pove, kaj lahko z urejenim parom počnemo. Kako naj
vemo, da iz $(a, b)$ lahko izluščimo $a$ in $b$, in kako preverimo, ali sta dva urejena
para enaka? Če takih zadev ne določimo, bi lahko kdo mislil, da je urejeni par kaka druga
operacija, denimo seštevanje, unija, ali kdovekaj.

Dejstvo, da je vsak element zmnožka množic urejen par, in to celo na en sam način, lahko
dokažemo.

\begin{trditev}
  Naj bosta $A$ in $B$ množici. Za vsak element $p \in A \times B$ obstaja natanko en
  $a \in A$ in natanko en $b \in B$, da velja $p = (a, b)$.
\end{trditev}

\begin{proof}
  Naj bosta $A$ in $B$ množici in $p \in A \times B$. Najprej pokažimo, da $p$ res je enak
  nekemu urejenemu paru, namreč
  %
  \begin{equation*}
    p = (\fst(p), \snd(p)).
  \end{equation*}
  %
  Uporabimo pravilo ekstenzionalnosti za pare, ki nam zagotavlja to enačbo, če dokažemo
  %
  \begin{equation*}
    \fst(p) = \fst(\fst(p), \snd(p))
    \qquad\text{in}\qquad
    \snd(p) = \snd(\fst(p), \snd(p)).
  \end{equation*}
  %
  Ti dve enačbi pa veljata, ker sta primerka računskih pravil za pare.

  Preveriti moramo še, da je $(\fst(p), \snd(p))$ edini urejeni par, ki je enak~$p$.
  Povedano z drugimi besedami, dokazati moramo: če je $p = (a, b)$ za neki $a \in A$ in
  $b \in B$, potem velja $a = \fst(p)$ in $b = \snd(p)$. Pa denimo, da bi za neki
  $a \in A$ in $B \in B$ veljalo $p = (a,b)$. Tedaj bi lahko uporabili računska pravila za
  pare in dobili
  %
  \begin{equation*}
    \fst(p) = \fst(a, b) = a
    \qquad\text{in}\qquad
    \snd(p) = \snd(a, b) = b,
  \end{equation*}
  %
  kar smo želeli dokazati.
\end{proof}

Trditev je prikladna, ko želimo podati funkcijsko pravilo za preslikavo, katere domena je
zmnožek množic. Primer take preslikave je
%
\begin{gather*}
  \RR \times \RR \to \RR \\
  p \mapsto \fst(p) + \snd(p)^2 \cdot \fst(p).
\end{gather*}
%
Ta zapis je precej nepregleden, a sledili smo navodilu, da mora stati na levi strani
funkcijskega predpisa simbol. Prejšnja trditev nam zagotavlja, da lahko vsak element
$\RR \times \RR$ na en sam način izrazimo kot urejeni par $(x, y)$, in zato ne bo nič
narobe, če zapišemo ta isti funkcijski predpis bolj pregledno tako, da upoštevamo, da
je $p$ enak $(x, y)$ za enolično določena $x$ in $y$:
%
\begin{gather*}
  \RR \times \RR \to \RR \\
  (x, y) \mapsto x + y^2 \cdot x.
\end{gather*}
%
Če bi funkcijo poimenovali, denimo $f$, bi dobili običajni zapis:
%
\begin{gather*}
  f : \RR \times \RR \to \RR \\
  f(x, y) \dfeq x + y^2 \cdot x.
\end{gather*}
%
Za tako preslikavo pravimo, da je `funkcija dveh spremenljivk', ker si mislimo, da smo
podali argumenta $x$ in $y$ ločeno drug od drugega. A lahko rekli tudi, da je to funkcija
ene spremenljivke, ki jo uporabimo na urejenem paru:
%
\begin{gather*}
  f : \RR \times \RR \to \RR \\
  f(p) \dfeq \fst(p) + \snd(p)^2 \cdot \fst(p).
\end{gather*}


Poleg zmnožka dveh množic bi lahko tvorili tudi zmnožek treh ali več množic. Pravila bodo
podobna kot za zmnožek dveh množic, le da bi namesto urejenih parov tvorili \df{urejene
  večterice} in da bi imeli več projekcij. Za vsako projekcijo bi zapisali eno računsko
pravilo, princip ekstenzionalnosti pa bi bil tudi podoben tistemu za urejene pare.
Podorobnosti prepustimo za vajo.


\section{Vsota}
\label{sec:vsota}

Spoznali smo že unijo $A \cup B$ množic $A$ in $B$, ki vsebuje tiste elemente, ki so v $A$
ali v $B$. Če imata $A$ in $B$ skupne elemente, bodo ti v uniji seveda nastopili samo
enkrat. V skranjem primeru dobimo $A \cup A = A$. Včasih pa želimo združiti množici tako,
da ne pride do prekrivanja. Taka konstrukcija je \df{vsota} $A + B$ množic $A$ in $B$.
Prekrivanje preprečimo tako, da elemente, ki jih je prispevala~$A$ označimo z eno oznako,
tiste, ki jih je prispevala~$B$, pa z drugo.

\begin{pravilo}[Vsota]
  \label{vsota:tvorba}
  Za vsaki množici $A$ in $B$ je $A + B$ množica, ki se imenuje \df{vsota} ali
  \df{koprodukt} množic $A$ in $B$.
\end{pravilo}

\begin{pravilo}[Vpeljava elementov vsote]
  \label{vsota:vpeljava}
  Za vsaki množici $A$ in $B$ velja:
  %
  \begin{enumerate}
  \item za vsak $a \in A$ je $\inl(a) \in A + B$,
  \item za vsak $b \in B$ je $\inr(b) \in A + B$.
  \end{enumerate}
\end{pravilo}

S pravilom vpeljave smo pojasnili, da uporabljamo oznaki $\inl$ in $\inr$, prvo za
elemente iz~$A$ in drugo za elemente iz~$B$. Oznakama pravimo tudi
\df{injekciji}\footnote{Ni pomembno, kako poimenujemo oznaki, da sta le
  različni. V funkcijskem programiranju, kjer poznamo vsote podatkovnih tipov, programer
  sam določi, kakšne oznake bo uporabljal za injekcije.} in sta preslikavi
%
\begin{equation*}
  \inl : A \to A + B
  \qquad\text{and}\qquad
  \inr : B \to A + B.
\end{equation*}
%
Kadar imamo opravka z večimi vsotami, na primer $A + B$ in $C + D$, bi lahko prišlo do
zmede glede oznak. Takrat injekcije opremimo še z dodatnimi oznakami množic, da razločimo
injekciji $\inl[A][B] : A \to A + B$ in $\inl[C][D] : C \to C + D$, in podobno za~$\inr$.

S pravilom~\ref{vsota:vpeljava} elementi $A + B$ še niso povsem opredeljeni. Kako vemo, da poleg elementov, ki jih predpisuje pravilo, $A + B$ ne vsebuje nobenih drugih? In kako primerjamo elemente~$A + B$? Potrebujemo še eno pravilo.

\begin{pravilo}
  \label{vsota:uporaba}
  Za vsaki množici $A$ in $B$ in za vsak $u \in A + B$, bodisi obstaja natanko en
  $a \in A$, da je $u = \inl(a)$, bodisi obstaja natanko en $b \in B$, da je
  $u = \inr(b)$.
\end{pravilo}

V zgornjem pravilu fraza \nls{bodisi \dots bodisi \dots} pomeni, da velja prva ali druga možnost, a ne obe hkrati.
%
S tem smo v $A + B$ res ločili elemente $A$ od elementov $B$, saj velja $\inl(a) \neq \inr(b)$, tudi ko je $A = B$ in $a = b$. 
%
Fraza \nls{natanko en $a \in A$} pove, da iz $u = \inl(a_1)$ in $u = \inl(a_2)$ sledi $a_1 = a_2$.
Povedano drugače, če velja $\inl(a_1) = \inl(a_2)$, potem je $a_1 = a_2$. Podobno iz
$\inr(b_1) = \inr(b_2)$ sledi $b_1 = b_2$.
%
Podajmo prepost primer, ki verjetno marsikaj pojasni:
%
\begin{equation*}
  \set{a, b, c} + \set{a, d, e} =
  \set{\inl(a), \inl(b), \inl(c), \inr(a), \inr(d), \inr(e)}.
\end{equation*}

Kako definiramo preslikavo $A + B \to C$? Ker je vsak element domene $A + B$ bodisi
$\inl(a)$ za neki $a \in A$ bodisi $\inr(b)$ za neki $b \in B$, \emph{obravnavamo oba
  primera}. Tako funkcijski zapis za preslikavo $A + B \to C$ zapišemo kot
%
\begin{equation*}
  u \mapsto
  \begin{cases}
    \cdots a \cdots & \text{če $u = \inl(a)$,}\\
    \cdots b \cdots & \text{če $u = \inr(b)$,}
  \end{cases}
\end{equation*}
%
kjer smemo v $\cdots a \cdots$ zapisati izraz, ki vsebuje simbol~$a$, in v
$\cdots b \cdots$ izraz, ki vsebuje simbol~$b$. Ker je tak zapis nekoliko neroden, se
dogovorimo, da ga lahko zapišemo tudi z \emph{večdelnim} funkcijskim predpisom:
%
\begin{align*}
  \inl(a) &\mapsto \cdots a \cdots, \\
  \inr(b) &\mapsto \cdots b \cdots.
\end{align*}
%
Če želimo preslikavo poimenovati, zapišemo
%
\begin{align*}
  f : A + B &\to C, \\
  f(\inl(a)) &\dfeq \cdots a \cdots \\
  f(\inr(b)) &\dfeq \cdots b \cdots.
\end{align*}

\begin{zgled}
  Predpis
  %
  \begin{align*}
    f : \set{1, 2} + \set{2, 3} &\to \NN, \\
    f(\inl(x)) &\dfeq x^2 \\
    f(\inr(y)) &\dfeq 6 / x
  \end{align*}
  %
  bi lahko predstavili s tabelo
  %
  \begin{center}
    \begin{tabular}{cc}
      \toprule
      $u$ & $f(u)$ \\ \midrule
      $\inl(1)$ & $1$ \\
      $\inl(2)$ & $4$ \\
      $\inr(2)$ & $3$ \\
      $\inr(3)$ & $2$ \\ \bottomrule
    \end{tabular}
  \end{center}
\end{zgled}

Vsi ti zapisi res določajo celovito in enolično prirejanje, saj nam pravila za vsoto
zagotavljajo, da vedno obvelja natanko en primer. Na sploh lahko podamo funkcijski zapis z
večimi primeri, če le pazimo, da obravnavamo vse možnosti, in da se le-te ne prekrivajo.
Na primer, predpis
%
\begin{align*}
  (A + B) \times C &\to B + A \\
  (\inl[A][B](a), c) &\mapsto \inr[B][A](a) \\
  (\inr[A][B](b), c) &\mapsto \inl[B][A](b)
\end{align*}
%
je celovit in enoličen, medtem ko predpis
%
\begin{align*}
  (A \times A) + B &\to A \\
  \inl(a_1, a_2) &\mapsto a_2
\end{align*}
%
ni veljaven, ker ni celovit, saj manjka primer $\inr(b) \mapsto \cdots$.

Poleg vsote dveh množic bi lahko tvorili vsoto treh ali več množic. Pravila bi bila
podobna, le da bi imeli več injekcij in več primerov.




%% STAR MATERIAL OD DAVORINA. Preveriti, kaj od tega je treba dati v besedilo, in kam.

% Množice ne obstajajo ločene ena od druge pač pa so med sabo povezane s
% \df{preslikavami} oziroma s tujko \df{funkcijami}.  Posamična preslikava slika elemente ene
% množice po določenem predpisu v elemente druge množice.

% Če je $f$ preslikava, ki slika iz množice $X$ v množico $Y$, to zapišemo
% %
% \begin{equation*}
%   f : X \to Y.
% \end{equation*}
% %
% Rečemo, da je množica~$X$ \df{začetna množica} ali \df{domena} preslikave~$f$, množica~$Y$
% pa je \df{ciljna množica} ali \df{kodomena} preslikave $f$.


% Običaj je, da predpis preslikave podamo s pomočjo spremenljivke, tipično z oznako $x$. Na primer, če je $f$ preslikava kvadriranja, njen predpis zapišemo kot
% \[f(x) = x^2.\]
% Na tem mestu je potrebno poudariti več reči.
% \begin{itemize}
% \item
% Velikokrat površno rečemo, da zgornji predpis podaja preslikavo. To ni povsem res --- to je zgolj predpis preslikave. Za to, da preslikavo v celoti podamo, je potrebno navesti tri stvari: poleg predpisa še domeno in kodomeno. Vse to je del informacije o preslikavi.

% To se jasno pokaže, če začnemo razmišljati o lastnostih preslikav. Se še spomnite iz srednje šole, kaj pomeni, da je preslikava surjektivna? (Bomo ponovili v razdelku~\ref{razdelek:injektivnost-in-surjektivnost}.) Če vzamemo, da preslikava $f$ zadošča zgornjemu predpisu in jo obravnavamo kot preslikavo $f\colon \RR \to \RR$, ni surjektivna, če jo obravnavamo recimo kot preslikavo $f\colon \RR_{\geq 0} \to \RR_{\geq 0}$, pa je.
% \item
% Za spremenljivko $x$ velja isto, kot smo razpravljali že v prejšnjem razdelku pri lastnostih elementov množic: spremenljivka $x$ nima vnaprej določene vrednosti, pač pa predstavlja mesto, kamor lahko vstavimo poljubno vrednost. Seveda je potem vseeno, če vzamemo kakšno drugo črko ali čisto drug simbol: $f(y) = y^2$ določa isti predpis kot $f(x) = x^2$; prav tako $f(\heartsuit) = \heartsuit^2$. Se pravi, tudi v tem primeru gre za nemo spremenljivko. Če si torej izberemo neko vrednost, jo lahko vstavimo na mesto spremenljivke in izračunamo vrednost dobljenega izraza, npr.~$f(3) = 3^2 = 9$ oziroma $f(2\pi) = (2\pi)^2 = 4\pi^2$. Predstavljajte si, da je spremenljivka pravzaprav škatlica, kamor lahko vstavite vrednost, torej
% \[f(\argbox) = \argbox^2.\]
% \item
% Alternativen način zapisa $f(x) = x^2$ je
% \[f\colon x \mapsto x^2.\]
% Pazimo: navadna puščica $\to$ podaja domeno in kodomeno, kot razloženo zgoraj. Repata puščica $\mapsto$ pa za posamičen element domene pove, v kateri element kodomene se preslika.

% Zapis z repato puščico je še posebej uporaben, kadar želimo podati preslikavo, ne da bi nam bilo potrebno izbrati ime zanjo. Na primer, realno funkcijo kvadriranja lahko v celoti podamo takole:
% \begin{align*}
% \RR &\to \RR \\
% x &\mapsto x^2
% \end{align*}
% (prva vrstica pove domeno in kodomeno, druga pa predpis). Tako podanim preslikavam potem rečemo \df{brezimne preslikave} (s tujko \df{anonimne funkcije}). Kasneje (v razdelku~\ref{razdelek:brezimne-preslikave}) bomo spoznali bolj strnjen zapis takih preslikav, ki je še posebej primeren za izvajanje operacij med preslikavami; takrat bomo takšno funkcijo zapisali kot $\lam{x \in \RR} x^2$.
% \end{itemize}

% \note{Sklop (kompozicija, kompozitum) preslikav. Identiteta kot enota za sklapljanje. Razčlenitev (dekompozicija, faktorizacija) preslikav.}

% \davorin{Definirati moramo tudi oznako $\set{f(x)}{x \in X}$, kar je druge vrste oznaka kot prej definirana $\set{x \in X}{\phi(x)}$. Se gremo primerjavo s Pythonom (razlika med \texttt{\{f(x) for x in X\}} in \texttt{\{x if phi(x)\}})? Smo matematični hipsterji in uvedemo oznako $\{f(x) \,|\, x \in X \,|\, \phi(x)\}$, ki ustreza \texttt{\{f(x) for x in X if phi(x)\}}, kar bi tudi prišlo prav?}

% Zaenkrat smo imeli primere, ko je bil prepis preslikave dan z eno samo spremenljivko, npr.~$f(x) = x^2$. Zelo pogoste so pa tudi \df{preslikave več spremenljivk}, npr.~$f(x, y) = x^2 + y^2$. Že osnovne računske operacije so take --- na primer, pri seštevanju vzamemo \emph{dva} podatka in vrnemo rezultat (vsoto).

% V takem primeru je smiselno reči: domena preslikave sestoji iz \df{dvojic} ali \df{parov} števil. Pri seštevanju je to, katero število je prvo, katero pa drugo, sicer nepomembno, pri kakšni drugi operaciji (npr.~že odštevanju), pa je, zato posebej zahtevajmo: gre za \df{urejene dvojice} (\df{pare}). Urejeno dvojico elementov $a$ in $b$ (v tem vrstem redu) po dogovoru zapišemo kot $(a, b)$. Vrednosti $a$ in $b$ imenujemo \df{komponenti} tega para; natančneje, $a$ je \df{prva komponenta}, $b$ pa \df{druga komponenta}.

% Če imamo dve množici $A$ in $B$, tedaj množico vseh urejenih dvojic, katerih prva komponenta je element iz $A$, druga komponenta pa element iz $B$, označimo $A \times B$ in imenujemo \df{zmnožek} ali \df{produkt} množic $A$ in $B$. Glede na to, da obstaja mnogo operacij, ki se imenujejo ">produkt"< (poznate že vsaj produkt števil, produkt števila z vektorjem, skalarni produkt vektorjev in vektorski produkt vektorjev, obstaja pa jih še precej več), je koristno produkt množic posebej poimenovati, da ga ločimo od drugih: zanj se je uveljavil izraz \df{kartezični produkt} (izhaja iz imena Cartesius, tj.~latinske različice priimka Renéja Descarta\footnote{René Descartes (1596 -- 1650) je bil francoski filozof, matematik in znanstvenik.}).

% Seštevanje potemtakem lahko razumemo kot preslikavo $+\colon \RR \times \RR \to \RR$. V tem smislu še vedno gre za preslikavo, ki dan vhodni podatek preslika v neki rezultat, le da je vhodni podatek dvojica števil, ne pa zgolj eno število. Kadar imamo produkt več enakih faktorjev, ga lahko (kot običajno) zapišemo v obliki potence; pisali bi lahko tudi $+\colon \RR^2 \to \RR$.

% Seveda nismo omejeni na preslikave samo ene ali dveh spremenljivk. Nič nam ne preprečuje definirati recimo $f(x, y, z) = 2x + y - 3z$. Smiselna domena te preslikave setoji iz \df{urejenih trojic} števil. V splošnem, če jemljemo elemente iz množic $A$, $B$, $C$, tedaj se množica vseh takih trojic označi z $A \times B \times C$. Prejšnji predpis določa potem preslikavo $f\colon \RR \times \RR \times \RR \to \RR$ (oziroma krajše $f\colon \RR^3 \to \RR$).

% Spremenljivk je lahko še več; poleg dvojic in trojic tako dobimo še četverice, peterice, šesterice\ldots V splošnem takšna končna zaporedja elementov imenujemo \df{urejene večterice}. Tudi število spremenljivk je lahko označeno s črko; na primer, preslikava, ki računa povprečje $n$ števil (kjer $n \in \NN_{\geq 1}$), je dana kot
% \begin{align*}
% \RR^n &\to \RR \\
% (x_1, x_2, \ldots, x_n) &\mapsto \frac{x_1 + x_2 + \ldots + x_n}{n}
% \end{align*}
% (če hočemo poudariti, da imajo naše večterice natanko $n$ komponent, jih imenujemo $n$-terice). Nadlega pri tem je sicer spet dvoumnost tropičja. Deloma jo je možno odpraviti tako, da celotno večterico označimo z eno spremenljivko. Pogosta izbira zapisa je $f(\mathbf{x})$ ali $f(\vec{x})$ (razlog za to je, da lahko večterico vidimo kot vektor).

% Marsikdaj želimo delati ne samo z eno preslikavo, pač pa s celo množico preslikav naenkrat. Zato uvedemo: množica vseh preslikav, ki slikajo iz $X$ v $Y$, se označi kot $Y^X$; temu se reče \df{eksponent} množic $X$ in $Y$ (\note{na primernem mestu kasneje} bomo razložili, od kod ta oznaka).

% \begin{zgled}
% Množico vseh preslikav, ki realna števila slikajo nazaj v realna števila, označimo z $\RR^\RR$. Če nas zanimajo realne preslikave, ki so definirana samo na intervalu $\intoo{-1}{1}$, opazujemo množico $\RR^{\intoo{-1}{1}}$. Definiramo lahko preslikavo
% \begin{align*}
% \RR^{\intoo{-1}{1}} &\to \RR \\
% f &\mapsto f(0),
% \end{align*}
% ki preslikavam priredi njihovo vrednost v točki $0$. Ta preslikava torej ima za argumente (tj.~vnose) celotne preslikave in ne števila! Sama po sebi je element množice $\RR^{\RR^{\intoo{-1}{1}}}$.
% \end{zgled}

% \begin{zgled}
% Za poljubne množice $A$, $B$, $C$ lahko definiramo sledečo preslikavo, katere argumenti so pari preslikav.
% \begin{align*}
% B^A \times C^B &\to C^A \\
% (f, g) &\mapsto g \circ f
% \end{align*}
% \end{zgled}


% \davorin{Glede na to, da gre za slovenski učbenik, dajem izrazu `preslikava' prednost pred izrazom `funkcija'. Seveda pa sem pojasnil tudi slednji izraz (v prvem poglavju).}

% \note{Uvod. Definicijsko območje in zaloga vrednosti \davorin{morda dodamo kot možno ime za zalogo vrednosti še prevod angleške besede `range', se pravi `razpon'?}. Zožitve (tako domene kot kodomene); oznake za to so $\rstr{f}_A$, $\rstr{f}^B$, $\rstr{f}_A^B$. Izvrednotenje (evalvacija) preslikave (če ne bomo tega pojasnili že pri eksponentih množic).}


\section{Eksponent}
\label{sec:eksponent}

Pravila, ki smo jih podali do sedaj, ne zagotavljajo obstoja množice vseh preslikav z dano domeno in kodomeno. Potrebujemo novo pravilo.

\begin{pravilo}[Eksponent]
  \label{pravilo:eksponent}%
  Za vsaki množici $A$ in $B$ je \df{eksponent} ali \df{eksponentna množica $B^A$},
  katere elementi so natanko vse preslikave iz~$A$ v~$B$.
\end{pravilo}

Potemtakem je zapis $f : A \to B$ enakovreden zapisu $f \in B^A$.

Zmnožek množic smo podali s pravili tvorbe, vpeljave in uporabe ter računskima praviloma in pravilom ekstenzionalnosti.
%
Tudi eksponent množic smo podali po istem vzorcu v razdelku~\ref{sec:preslikave}:
%
\begin{itemize}
\item \emph{pravilo tvorbe} je prvi del pravila~\ref{pravilo:eksponent},
\item \emph{vpeljava:} preslikava je podana z domeno, kodomeno in prirejanjem,
\item \emph{uporaba:} preslikavo lahko uporabimo na argumentu,
\item \emph{računsko pravilo} je pravilo zamenjave vezane spremenljivke z argumentom,
\item \emph{ekstenzionalnost} zagotavlja enakost preslikav z enakimi vrednostmi.
\end{itemize}


\section{Preslikave višjega reda}
\label{sec:presl-visj-red}

Preslikavo, ki sprejme kot argument preslikavo, imenujemo \df{funkcional} ali \df{preslikava
  višjega reda}. Znan primer je določeni integral, ki kot argumente sprejme realni števili $a, b\in \RR$ in integrabilno funkcijo $f \in \RR^\RR$ ter izračuna ploščino pod~$f$ na intervalu~$[a,b]$.
Še en primer je operacija $\lim$, ki sprejme konvergentno zaporedje $a \in \RR^\NN$ in izračuna njegovo limito.

V razdelku~\ref{sec:kompozicija} mi smo že srečali funkcional, namreč kompozicijo preslikav. Zapišimo jo še enkrat tako, da nastopi operacija $\circ$ kot preslikava:
%
\begin{align*}
  {\circ} &: C^B \times B^A \to C^A, \\
  {\circ} &: (g, f) \mapsto (x \mapsto g(f(x))).
\end{align*}
%
Res, lahko si mislimo, da $\circ$ sprejme urejeni par preslikav $(g, f)$ in izračuna njun kompozitum, ki ga pišemo $g \circ f$ namesto ${\circ}(g, f)$.
%
Tu smo kompozicijo smo zapisali z \emph{gnezdenim} funkcijskim predpisom, ki argumentu $(g, f)$ priredi
preslikavo, podano s predpisom $x \mapsto g(f(x)$. V splošnem je gnezdeni funkcijski
predpis oblike
%
\begin{align*}
  A &\mapsto C^B \\
  a &\mapsto (b \mapsto \cdots),
\end{align*}
%
kjer se lahko v $\cdots$ pojavita~$a$ in~$b$. Na tak zapis se je treba navaditi, a je zelo
prikladen, še posebej v funkcijskem programiranju. Čeprav v matematiki ni pogost, se mu ne bomo izogibali.

Primer gnezdenega funkcijskega predpisa je preslikava~$\konst{}$ iz razdelka~\ref{sec:ident-konst-presl}, ki tvori konstantno preslikavo. Spomnimo se, če sta $A$ in $B$ množici ter $b \in B$, definiramo konstantno preslikavo
%
\begin{align*}
  \konst{b} &: A \to B \\
  \konst{b} &: a \mapsto b.
\end{align*}
%
Pravzaprav imamo opravka s preslikavo, ki sprejme $b \in B$ term tvori konstantno preslikavo:
%
\begin{align*}
  \konst{} &: B \to B^A \\
  \konst{} &: b \mapsto (a \mapsto b).
\end{align*}
%
Uporabili smo gnezdeni funkcijski predpis.

Pri računanju s funkcionali včasih obravnavamo več funkcijskih predpisov hkrati. Če za vse uporabimo isto vezano spremenljivko, se lahko hitro zmedemo. Na primer, kompozitum preslikav
%
\begin{equation*}
  \begin{aligned}
    \RR &\to \RR \\
    x &\mapsto x^2 - 4
  \end{aligned}
  %
  \qquad\text{in}\qquad
  %
  \begin{aligned}
    \RR &\to \RR \\
    x &\mapsto 2 - x
  \end{aligned}
\end{equation*}
%
bi lahko izračunali takole:
%
\begin{align*}
  (x \mapsto x^2  - 4) \circ (x \mapsto 2 - x)
  &= (x \mapsto (x \mapsto x^2 - 4) ((x \mapsto 2 - x) x)) \\
  &= (x \mapsto (x \mapsto x^2 - 4) (2 - x)) \\
  &= (x \mapsto (2 - x)^2 - 4) \\
  &= (x \mapsto x^2 - 4 x).
\end{align*}
%
Tu imamo tri različne $x$-e, saj vsak nastopa kot vezana spremenljivka v svojem funkcijskem predpisu.
%
Lahko bi jih ločili z barvami:
%
{%
\newcommand{\redx}{{\color{red}x}}%
\newcommand{\greenx}{{\color{green}x}}%
\newcommand{\bluex}{{\color{blue}x}}%
\begin{align*}
  (\redx \mapsto \redx^2  - 4) \circ (\greenx \mapsto 2 - \greenx)
  &= (\bluex \mapsto (\redx \mapsto \redx^2 - 4) ((\greenx \mapsto 2 - \greenx) \bluex)) \\
  &= (\bluex \mapsto (\redx \mapsto \redx^2 - 4) (2 - \bluex)) \\
  &= (\bluex \mapsto (2 - \bluex)^2 - 4) \\
  &= (\bluex \mapsto \bluex^2 - 4 \bluex).
\end{align*}}% ne brisat tega komentarja
%
Še posebej nejasen je drugi računski korak, ko imamo opravka s tremi barvami hkrati.
%
Spomnimo se, da lahko vezane spremenljivke vedno preimenujemo in da lahko namesto barv preprosto uporabimo tri različne spremenljivke. Kompozicijo
%
\begin{equation*}
  \begin{aligned}
    \RR &\to \RR \\
    y &\mapsto y^2 - 4
  \end{aligned}
  %
  \qquad\text{in}\qquad
  %
  \begin{aligned}
    \RR &\to \RR \\
    z &\mapsto 2 - z
  \end{aligned}
\end{equation*}
%
izračunamo še tretjič, tokrat bolj pregledno:
%
\begin{align*}
  (y \mapsto y^2  - 4) \circ (z \mapsto 2 - z)
  &= (x \mapsto (y \mapsto y^2 - 4) ((z \mapsto 2 - z) x)) \\
  &= (x \mapsto (y \mapsto y^2 - 4) (2 - x)) \\
  &= (x \mapsto (2 - x)^2 - 4) \\
  &= (x \mapsto x^2 - 4 x).
\end{align*}
%
Da ne bo prihajalo do zapletov z vezanimi spremenljivkami, se dogovorimo: \emph{kadar imamo opravka z večimi vezanimi spremenljivkami, jih po potrebi preimenujemo tako, da so med seboj različne.}


\section{Izomorfizem množic}
\label{sec:izomorfizem-mnozic}

Ko otrok prvič spozna pojem števila, je ta zanimiv sam po sebi. Z vnemo šteje do sto in se
rad pogovarja se o tem, koliko je en miljon. Sčasoma se radovednost osredotoči na
aritmetične operacije in, če ima mladenič ali mladenka v sebi matematično žilico, na
\emph{zakonitosti} števil: množenje z~$1$ nima učinka, vrstni red seštevanja ni pomemben
itd. Ali tudi operacijam na množicah, ki smo jih spoznali do sedaj, vladajo kakšne
podobne zakonitosti?

Za števili $a$ in $b$ velja $a \cdot b = b \cdot a$. Nekaj podobnega velja tudi za množici
$A$ in $B$ in njuna zmnožka $A \times B$ in $B \times A$. V splošnem sicer nista enaka, a
sta v nekem smislu enakovredna, ker lahko par $(x, y) \in A \times B$ pretvorimo v par
$(y, x) \in B \times A$ in obratno. Ta razmislek vodi do pojma izomorfizma.

% Pojasnilo: izomorfnost $A \cong B$ je struktura, ki je naravno podana z dvema
% preslikavama $A \to B$ in $B \to A$ ter dvema enačbama med njima. Zato tu zapišemo
% definicijo, ki hkrati uvede vse te pojme.

\begin{definicija}
  Množici $A$ in $B$ sta \df{izomorfni} in pišemo $A \cong B$, kadar obstajata preslikavi
  %
  \begin{equation*}
    f : A \to B
    \qquad\text{in}\qquad
    g : B \to A,
  \end{equation*}
  %
  za kateri velja
  %
  \begin{equation*}
    g \circ f = \id[A]
    \qquad\text{in}\qquad
    f \circ g = \id[B].
  \end{equation*}
  %
  Pravimo, da je~$f$ \df{izomorfizem} med~$A$ in~$B$ in da je~$g$ \df{inverz} ali
  \df{obrat}~$f$.
\end{definicija}

Preverimo, da velja $A \times B \cong B \times A$ za poljubni množici $A$ in $B$. To
storimo tako, da zapišemo preslikavi med zmnožkoma in preverimo, da tvorita
izomorfizem:\footnote{Držimo se pravila, da nikoli ne uporabimo iste vezane spremenljivke
  dvakrat, zato pravilo za $f$ zapišemo z $x$ in $y$ in pravilo za $g$ z $v$ in $u$.
  Marsikdo bi oba funkcisjka predpisa zapisal z $x$ in $y$, torej
  $f : (x, y) \mapsto (y, x)$ in $g : (y, x) \mapsto (x, y)$. To zmede nekatere študente,
  ker mislijo, da \nls{sta je $x$ v definiciji $f$ isti kot v definiciji $g$}, karkoli že
  naj bi to pomenilo. Poudarimo še enkat: vezana spremenljivka v funkcijskem predpisu nima
  nikakršne zveze z nobeno drugo pojavitvijo iste spremenljivke kje druge.}
%
\begin{align*}
  f &: A \times B \to B \times A &
  g &: B \times A \to A \times B \\
  f &: (x, y) \mapsto (y, x) &
  g &: (v, u) \mapsto (u, v).
\end{align*}
%
Treba je preveriti, da velja $g \circ f = \id[A \times B]$ in
$f \circ g = \id[B \times A]$. To naredimo z uporabo ekstenzionalnosti preslikav, ki pravi
da $g \circ f = \id[A]$ velja, če velja $(g \circ f)(a,b) = \id[A](a,b)$ za vse $a \in A$
in $b \in B$, in podobno za $f \circ g$. Obravnavajmo torej poljubna $a \in A$ in
$b \in B$ in izračunajmo:
%
\begin{equation*}
  (g \circ f)(a, b) =
  g (f (a, b)) = g (b, a) = (a, b).
\end{equation*}
%
Na podoben način preverimo $f \circ g = \id[B \times A]$.

\begin{zgled}\label{zgled:logaritmiranje-je-obratno-od-eksponenciranja}
  Primere izmorfizmov poznamo že iz srednje šole. Naj bo $\RR_{>0}$ množica vseh pozitivnih realnih
  števil. Tedaj logaritem in eksponentna funkcija,
  %
  \begin{equation*}
    \log : \RR_{>0} \to \RR
    \qquad\text{in}\qquad
    \exp : \RR \to \RR_{>0}
  \end{equation*}
  %
  tvorita izomorfizem, saj za vsak $x \in \RR$ velja $\log (\exp x) = x$ in za vsak $y \in \RR_{>0}$
  velja $\exp (\log y) = y$.

  Še več, eksponentna funkcija seštevanje pretvori v množenje,
  $\exp 0 = 1$ in $\exp (x + y) = \exp x \cdot \exp y$, zato ni le izomorfizem množic, ampak tudi izomorfizem \emph{grup} $(\RR, {+}, 0)$ in $(\RR_{>0}, {\cdot}, 1)$.

  Opazko o izomorfizmu grupo smo podtaknili namenoma kot priložnost za nasvet.
  %
  Če ne veste, kaj je grupa in izomorfizem grup, nikar ne obupavajte. Vsak matematik se v
  svojem delu nenehno srečuje z neznanimi pojmi. Znameniti profesor
  France Križanič\footnote{France Križanič (1928--2002), slovenski matematik} v enega od
  svojih učbenikov zapisal, da naj tisti, ki mu je branje dokazov odveč, ravna tako kot Du
  Fu:\footnote{Du Fu (712--770 pr.~n.~š), kitajski pesnik}
  %
  \begin{center}
    \begin{tabular}{l}
      Ko berem knjige,\\
      z vinom se krepčam\\
      in znak preskočim,\\
      če ga ne poznam.
    \end{tabular}
  \end{center}
  %
  Morda pa bi veljalo odkorakati v knjižnico in ugotoviti, kaj vse je še napisal profesor Križanič.
\end{zgled}


Dokažimo nekaj osnovih lastnosti izmorfnosti in izomorizmov. Tokrat ne bomo zapisali
podrobnih dokazov. Za vajo jih dopolnite do tolikšnih podrobnosti, da boste sami sebe
prepričali, da trditve držijo.

\begin{trditev}
  \label{trditev:enolicen-inverz}%
  Če je $f : A \to B$ izomorfizem med množicama $A$ in $B$ ter sta preslikavi
  $g : B \to A$ in $h : B \to A$ obe obrata~$f$, potem je $g = h$.
\end{trditev}

\begin{proof}
  Ker je $g$ obrat $f$, velja
  %
  \begin{equation*}
    g \circ f = \id[A]
    \qquad\text{in}\qquad
    f \circ g = \id[B],
  \end{equation*}
  %
  in ker je $h$ obrat $f$, velja
  %
  \begin{equation*}
    h \circ f = \id[A]
    \qquad\text{in}\qquad
    f \circ h = \id[B].
  \end{equation*}
  %
  Dokazati moramo, da iz teh štirih predpostavk sledi $g = h$, kar storimo z naslednjim
  računom:
  %
  \begin{align*}
    g
    &= \id[A] \circ g \tag{kompozicija z $\id[A]$ nima učinka} \\
    &= (h \circ f) \circ g \tag{predpostavka $h \circ f = \id[A]$} \\
    &= h \circ (f \circ g) \tag{kompozicija je asociativna} \\
    &= h \circ \id[B] \tag{predpostavka $f \circ g = \id[B]$} \\
    &= h. \tag{kompozicija z $\id[B]$ nima učinka}
  \end{align*}
\end{proof}

Če je $f : A \to B$ izomorfizem, potem ima natanko en obrat, ki ga označimo $\inv{f}$. Če
$f$ ni izomorfizem, zapis $f^{-1}$ ni veljaven izraz.

Oznaka za obrat je nekoliko nerodna, ker tudi obratno vrednost števila $x \in \RR$ pišemo $\inv{x}$.
Torej moramo paziti: če je $f : \RR \to \RR$ izomorfizem in $x \in \RR$, je $\inv{(f(x))}$ obrat števila $f(x)$,
medtem ko je $\inv{f}(x)$ število, ki ga dobimo, ko obrat preslikave $f$ uporabimo na~$x$.
Sami premislite, kaj je $\inv{(\inv{f}(x))}$.

\begin{trditev}
  Za vse izmorfizme $f : A \to B$ in $g : B \to C$ velja
  %
  \begin{equation*}
    \inv{(\inv{f})} = f
    \qquad\text{in}\qquad
    \inv{(g \circ f)} = \inv{f} \circ \inv{g}.
  \end{equation*}
\end{trditev}

\begin{proof}
  Dokaz prepuščamo za vajo. Pozor, v desni enakosti se je zamenjal vrstni red $f$ in $g$!
  Nadalje opazimo še to: zapisali smo $\inv{(\inv{f})}$ in $\inv{(g \circ f)}$, ne da bi
  predhodno preverili, ali sta $\inv{f}$ in $g \circ f$ izomorfizma. Torej morate v dokazu
  najprej preveriti, da je sta $\inv{f}$ in $g \circ f$ izomorfizma, če sta $f$ in $g$
  izomorfizma.
\end{proof}

\begin{trditev}
  Za vse množice $A$, $B$ in $C$ velja:
  %
  \begin{enumerate}
  \item $A \cong A$,
  \item če $A \cong B$, potem $B \cong A$,
  \item če $A \cong B$ in $B \cong C$, potem $A \cong C$.
  \end{enumerate}
\end{trditev}

\begin{proof}
  \parbox{0pt}{}
  %
  \begin{enumerate}
  \item $\id[A]$ je izomorfizem iz $A$ v $A$, ki je sam svoj obrat,
  \item če je $f : A \to B$ izomorfizem iz $A$ v $B$, potem je $\inv{f}$ izomorfizem iz
    $B$ v $A$ in $f$,
  \item če je $f : A \to B$ izomorfizem iz $A$ v $B$ in $g : B \to C$ izomorfizem iz $B$ v
    $C$, potem je $g \circ f$ izomorfizem iz $A \to C$. \qedhere
  \end{enumerate}
\end{proof}

% TODO Izomorfnost je kongruenca za produkt, vsoto in eksponent.

\section{Aritmetika množic}
\label{sec:aritmetika-mnozic}

Kot že veste, seštevanje, množenje in potenciranje števil zadoščajo naslednjim
aritmetičnim zakonom:
%
\begin{align*}
  a + 0 &= a                   &     a \cdot 1 &= a \\
  a + b &= b + a               &     a \cdot b &= b \cdot a \\
  a + (b + c) &= (a + b) + c   &     a \cdot (b \cdot c) &= (a \cdot b) \cdot c \\[1ex]
  0 \cdot a &= 0                           &   1^a &= 1 \\
  (a + b) \cdot c &= a \cdot c + b \cdot c &   (a \cdot b)^c &= a^c \cdot b^c \\[1ex]
  a^0 &= 1                     &   a^1 &= a \\
  a^{b + c} &= a^b \cdot a^c   &   a^{b \cdot c} &= (a^b)^c \\[1ex]
  0^a &= 0 \quad \text{če $a \neq 0$.}
\end{align*}
%
Že prej smo opazili, da je zakon $a \cdot b = b \cdot a$ podoben izomorfizmu
$A \times B \cong B \times A$. Kaj pa ostali zakoni?

\begin{izrek}
  \label{izrek:aritmetika-mnozic}
  Za vse množice $A$, $B$ in $C$ velja:
  %
  \begin{align*}
    A + \emptyset &\cong A                   &     A \times \one &\cong A \\
    A + B &\cong B + A               &     A \times B &\cong B \times A \\
    A + (B + C) &\cong (A + B) + C   &     A \times (B \times C) &\cong (A \times B) \times C \\[1ex]
    \emptyset \times A &\cong \emptyset                           &   \one^A &\cong \one \\
    (A + B) \times C &\cong A \times C + B \times C &   (A \times B)^C &\cong A^C \times B^C \\[1ex]
    A^\emptyset &\cong \one                     &   A^\one &\cong A \\
    A^{B + C} &\cong A^B \times A^C   &   A^{B \times C} &\cong (A^B)^C \\[1ex]
    \emptyset^A &\cong \emptyset \quad \text{če $A \neq \emptyset$.}
  \end{align*}
\end{izrek}

Izrek ni sam sebi namen, ampak je v njem nauk: \emph{z množicami lahko računamo}, tako kot
s števili. Preostanek razdelka je posvečen dokazu izreka.

\subsection{Asociativnost}
\label{sec:asociativnost}

Za ogrevanje dokažimo asociativnost zmnožkov,
$A \times (B \times C) \cong (A \times B) \times C$. Splošni element
$A \times (B \times C)$ je urejeni par oblike $(x, (y, z))$, kjer je $x \in A$, $y \in B$
in $z \in C$, med tem ko je splošni element $(A \times B) \times C$ oblike $((u, v), w)$,
kjer je $u \in A$, $v \in B$ in $w \in C$. Izomorfizmov ni težko zapisati:
%
\begin{align*}
  f &:  A \times (B \times C) \to (A \times B) \times C &
  g &: (A \times B) \times C \to A \times (B \times C) \\
  f &: (x, (y, z)) \mapsto ((x, y), z) &
  g &: ((u, v), w) \mapsto (u, (v, w)).
\end{align*}
%
Preverimo, da je $g$ obrat $f$. Za vse $x \in A$, $y \in B$ in $z \in C$ velja:
%
\begin{equation*}
  g(f(x, (y, z))) = g((x, y), z) = (x, (y, z))
\end{equation*}
%
in za vse $u \in A$, $v \in B$ in $w \in C$ velja
%
\begin{equation*}
  f(g((u, v), w)) = f(u, (v, w)) = ((u, v), w).
\end{equation*}
%
Tudi asociativnost vsote, $A + (B + C) \cong (A + B) + C$ ni nič bolj zapletena, le da
imamo opravka z injekcijami in obravnavanjem primerov. Najprej zapišimo izomorfizma s
popolnoma natančnim zapisom, kjer vse injekcije opremimo z oznakami množic:
%
% % f left of g
% \begin{align*}
%   f &:  A + (B + C) \to (A + B) + C &
%   g &: (A + B) + C \to A + (B + C) \\
%   f &: \inl[A][B+C](x)             \mapsto \inl[A+B][C](\inl[A][B](x)) &
%   g &: \inl[A+B][C](\inl[A][B](u)) \mapsto \inl[A][B+C](u)\\
%   f &: \inr[A][B+C](\inl[B][C](y)) \mapsto \inl[A+B][C](\inr[A][B](y)) &
%   g &: \inl[A+B][C](\inr[A][B](v)) \mapsto \inr[A][B+C](\inl[B][C](v))  \\
%   f &: \inr[A][B+C](\inr[B][C](z)) \mapsto \inr[A+B][C](z) &
%   g &: \inr[A+B][C](w)              \mapsto \inr[A][B+C](\inr[B][C](w))
% \end{align*}
% g below f
\begin{align*}
  f &:  A + (B + C) \to (A + B) + C \\
  f &: \inl[A][B+C](x)             \mapsto \inl[A+B][C](\inl[A][B](x)) \\
  f &: \inr[A][B+C](\inl[B][C](y)) \mapsto \inl[A+B][C](\inr[A][B](y)) \\
  f &: \inr[A][B+C](\inr[B][C](z)) \mapsto \inr[A+B][C](z) \\[\baselineskip]
  %
  g &: (A + B) + C \to A + (B + C) \\
  g &: \inl[A+B][C](\inl[A][B](u)) \mapsto \inl[A][B+C](u)\\
  g &: \inl[A+B][C](\inr[A][B](v)) \mapsto \inr[A][B+C](\inl[B][C](v))  \\
  g &: \inr[A+B][C](w)              \mapsto \inr[A][B+C](\inr[B][C](w))
\end{align*}
%
Isti zapis brez oznak množic je precej bolj čitljiv:
%
\begin{align*}
  f &:  A + (B + C) \to (A + B) + C &
  g &: (A + B) + C \to A + (B + C) \\
  f &: \inl(x)             \mapsto \inl(\inl(x)) &
  g &: \inl(\inl(u)) \mapsto \inl(u)\\
  f &: \inr(\inl(y)) \mapsto \inl(\inr(y)) &
  g &: \inl(\inr(v)) \mapsto \inr(\inl(v))  \\
  f &: \inr(\inr(z)) \mapsto \inr(z) &
  g &: \inr(w)              \mapsto \inr(\inr(w))
\end{align*}
%
Ali vidite, zakaj matematiki cenimo kratek in pregleden zapis? Preveč podrobnosti lahko
zakrije bistvo ideje. Preverjanje, da je $g$ obrat $f$, prepustimo tistim, ki radi veliko
pišejo.

\subsection{Preslikave in enojec}
\label{sec:preslikave-enojec}

Preslikavi
%
\begin{align*}
  f &: A \times \one \to A &
  g &: A \to A \times \one \\
  f &: (x, u) \mapsto x &
  g &: y \mapsto (y, \unit)
\end{align*}
%
tvorita izomorfizem $A \times \one \cong A$, saj za vsak $a \in A$ in $t \in \one$ velja,
upoštevaje da so vsi elementi~$\one$ enaki~$\unit$,
%
\begin{equation*}
  g(f(a, t)) = g(a) = (a, \unit) = (a, t)
  \qquad\text{in}\qquad
  f(g(a)) = f(a, t) = a.
\end{equation*}
%
Lahko bi rekli, da je $\one$ nevtralni element za zmnožek \emph{do izomorfizma natančno},
s čimer povemo, da ne velja \emph{enakost} $A \times \one = A$, ampak le
\emph{izomorfizem} $A \times \one \cong A$. Na tem mestu lahko tudi pojasnimo nenavadni
zapis edinega elementa~$\one$. Elementi zmnožka dveh množic so urejene dvojice, zmnožka
treh množic urejene trojice itd. Zmnožek nič množic je nevtralni element za množenje,
torej so njegovi elementi urejen ničterice, oziroma urejena ničterica~$\unit$, ker je ena
sama.

Izomorfizma $A^\one \cong A$ ni težko zapisati:
%
\begin{align*}
  f &: A^\one \to A &
  g &: A \to A^\one \\
  f &: h \mapsto h(\unit) &
  g &: x \mapsto (y \mapsto x)
\end{align*}
%
Preverimo, da je $g$ inverz $f$. Za vsak $x \in A$ velja
%
\begin{equation*}
  f(g(x)) = f(y \mapsto x) = x,
\end{equation*}
%
zato je $f \circ g = \id[A]$. Za vsak $h \in A^\one$ velja
%
\begin{equation*}
  g(f(h)) = g(h(\unit)) = (y \mapsto h(\unit)).
\end{equation*}
%
Ali sta $h$ in $y \mapsto h(\unit)$ enaki preslikavi? Kot vsakič, uporabimo
ekstenzionalnost preslikav, le da je tokrat še posebej preprosta: preslikavi z
domeno~$\one$ sta enaki, če imata enako vrednost pri argumentu $\unit$, saj je to edini
element~$\one$. Torej je $h = (y \mapsto h(\unit))$, saj velja
%
\begin{equation*}
  (y \mapsto h(\unit))(\unit) = h(\unit).
\end{equation*}
%
Izomorfnost $A$ in $A^\one$ pravzaprav pove nekaj zanimivega: preslikave $\one \to A$
lahko obravnavamo kot elemente~$A$ in obratno.

\section{Preslikave in prazna množica}
\label{sec:presl-prazna-mnozica}

Lotimo se izomorfizmov, v katere je vpletena prazna množica. Tu se ne moremo več zanašati
le na prirojen občutek za logiko, saj s prazno množico nimamo vsakdanjih izkušenj, oziroma
jo obravnavamo kot posebnost. Kako bi odgovorili na vprašanje, ali so vsi elementi prazne
množice praštevila? Pravilni odgovor je `da'. In hkrati so vsi elementi prazne množice
sestavljena števila. Zakaj je to res bomo spoznali v
razdelku~\ref{sec:logika-prazna-mnozica}, ko bomo podrobno obravnavali pravila sklepanja.
Zaenkrat si zapomnimo, da je pravilna vsaka izjava \nls{za vse elemente prazne množice velja
\dots}. Pravimo, da je taka izjava \emph{na prazno izpolnjena}

Začnimo z vprašanjem, ali lahko tvorimo kako preslikavo $\emptyset \to A$. Najprej
ugotovimo, da so vse preslikave $\emptyset \to A$ enake. Res, za $f, g : \emptyset \to A$
velja $f = g$ natanko tedaj, ko za vse $x \in \emptyset$ velja $f(x) = g(x)$. A ravnokar
smo povedali, da je vsaka izjava oblike \nls{za vse $x \in \emptyset$ \dots} veljavna. Pa
imamo kako preslikavo $\emptyset \to A$? Odgovor je pritrdilen, če lahko podamo kako
celovito in enolično prirejanje med elementi~$\emptyset$ in~$A$. Ker sta celovitost in
enoličnost spet izavi oblike \nls{za vse $x \in \emptyset$ \dots}, sta na prazno izpolnjena,
zato bo zadoščalo kakršnokoli prirejanje, denimo: nobenemu elementu ne priredimo nobenega
elementa. S tem smo utemeljili naslednjo trditev.

\begin{trditev}
  Za vsako množico $A$ obstaja natanko ena preslikava $\emptyset \to A$.
\end{trditev}

Edini preslikavi $\emptyset \to A$ pravimo \df{prazna preslikava}. S tem smo utemeljili
$A^\emptyset \cong \one$, saj izomorfizem prazni preslikavi priredi
$\unit$, njegov obrat pa priredi $\unit$ prazno preslikavo.

\subsection{Izomorfizmi in eksponenti}
\label{sec:izomorfizmi-in-eksponenti}

Nazadnje se posvetimo še zakonu $A^{B \times C} \cong (A^B)^C$.
Preverimo, da preslikavi\footnote{Saj ste se že naučili grške črke, ali ne?}
%
\begin{align*}
  \Lambda &: A^{B \times C} \to (A^B)^C
  &
  \Theta &: (A^B)^C \to A^{B \times C}
  \\
  \Lambda &: f \mapsto (c \mapsto (b \mapsto f(b, c)))
  &
  \Theta &: g \mapsto ((b, c) \mapsto g(c)(b))
\end{align*}
%
tvorita izomorfizem. Za vse $f \in A^{B \times C}$, $x \in B$ in $y \in C$ velja
%
\begin{align*}
  \Theta(\Lambda(f))(x, y)
  &= ((b, c) \mapsto \Lambda(f)(c)(b)) (x, y)  \\
  &= \Lambda(f)(y)(x) \\
  &= (c \mapsto (b \mapsto f(b, c)))(y)(x) \\
  &= (b \mapsto f(b, y)(x) \\
  &= f(x, y),
\end{align*}
%
zato je $\Theta(\Lambda(f)) = f$. Prav tako za vse $g \in (A^B)^C$ in $x \in B$ in
$y \in C$ velja
%
\begin{align*}
  \Lambda(\Theta(g))(y)(x)
  &= (c \mapsto (b \mapsto \Theta(g)(b, c)))(y)(x) \\
  &= (b \mapsto \Theta(g)(b, y))(x) \\
  &= \Theta(g)(x, y) \\
  &= ((b, c) \mapsto g(c)(b)) (x, y) \\
  &= g(y)(x)
\end{align*}
%
in zato $\Lambda(\Theta(g)) = g$.
%
Preslikavi $\Lambda(f)$ pravimo \df{transpozicija} preslikave~$f$, in prav tako preslikavi
$\Theta(g)$ pravimo transpozicija preslikave~$g$.

Izomorfizem $A^{B \times C} \cong (A^B)^C$ je zanimiv, ker pove, da lahko preslikavo dveh
argumentov vedno prevedemo na preslikavo enega argumenta. Natančneje, če je
$f : B \times C \to A$ preslikava dveh argumentov, je njena transpozicija
$\Lambda(f) : C \to A^B$ preslikava enega argumenta, njena vrednost pa je preslikava, ki
pričakuje še en argument. To dejstvo se s pridom izkorišča v funkcijskem programiranju:
namesto, da bi definirali preslikavo $f : B \times C \to A$, ki sprejme urejeni par
$(b, c)$ in vrne vrednost $f(b,c)$, raje definiramo enakovredno preslikavo
$\tilde{f} : B \to C \to A$, ki sprejme $b$ in vrne preslikavo $\tilde{f}(b)$, ta pa
sprejme še $c$ in vrne vrednost $\tilde{f}(b)(c)$.


% TODO Potence A^n in binomski izrek.

% TODO zmnozek, vsota in eksponent zo kongruenca za izomorfizme

% Pri asociativnosti produkta obravnavamo $A_1 \times A_2 \times \cdots \times A_n$ in
% enojec kot produkt nič množic. Podobno za vsote.

% Tu je treba pojasniti, zakaj pišemo $\unit$ za element $\one$.


% \section{Kar je že Davorin napisal}

% Interval realnih števil podamo s krajiščema intervala v oklepajih --- okrogli oklepaji ( ) označujejo odprtost intervala (krajišče ni vključeno v interval), oglati oklepaji [ ] pa zaprtost (krajišče je vključeno). Tako se npr.~interval realnih števil od $0$ do $1$, ki ne vsebuje krajišč, označi z $(0, 1)$, če jih vsebuje, pa z $[0, 1]$.

% Včasih pridejo prav tudi intervali na drugih množicah kot $\RR$. Zato se dogovorimo, da bomo intervale označevali tako, da podamo množico, ob kateri v indeksu zapišemo krajišči v oklepajih, npr.~$\intco[\NN]{1}{5} = \set{1, 2, 3, 4}$. Realna intervala iz prejšnjega odstavka tako zapišemo kot $\intoo{0}{1}$ in $\intcc{0}{1}$.

% Če interval v katero smer gre v nedogled, preprosto zapišemo množico z ustreznim simbolom za urejenost in krajiščem v indeksu. Na primer, $\RR_{> 0}$ označuje množico pozitivnih realnih števil, $\RR_{\geq 0}$ pa množico nenegativnih realnih števil.

% Primerjave med elementi, kot npr.~pravkar podani $>$ in $\geq$, imenujemo \df{relacije} (podrobneje jih bomo spoznali v poglavju~\ref{poglavje:relacije}). Zgornji zapis bomo uporabljali tudi za druge vrste relacij, ne samo za relacije urejenosti. Na primer, množico vseh neničelnih realnih števil zapišemo kot $\RR_{\neq 0}$.

% \davorin{To bi vsaj bil moj predlog. Na ta način se izognemo dvoumnostim (kar je namen). Na primer, kaj pomeni $\forall\, a > 0$? Če zapišemo $\forall\, a \in \NN_{> 0}$ ali $\forall\, a \in \RR_{> 0}$, je jasno. Razlog, da matematiki ">goljufajo"< in pridejo skozi brez tega, je (napol dogovorjena in ponotranjena, ampak arbitrarna) izbira črk; vsak izkušen matematik ve, da $\forall\, \epsilon > 0$ pomeni $\forall\, \epsilon \in \RR_{> 0}$. Dodaten problem je, da kasneje uporabljamo urejene pare, ki jih vsi na naši fakulteti pišejo z okroglimi oklepaji. Poskusimo se izogniti zmedi, ali $(a, b)$ pomeni urejeni par ali odprti interval. Če se ne strinjate, popravite in pustite komentar.}

% Če imamo dan neki element in neko množico, potem pripadnost tega elementa tej množici izrazimo s simbolom $\in$. Na primer, da je štiri naravno število, zapišemo $4 \in \NN$ (beri: ">štiri pripada množici naravnih števil"<).

% Elementi množic lahko zadoščajo raznim lastnostim. Na primer, recimo, da $\phi$ označuje lastnost ">biti manj od pet"<; to potem zapišemo
% \[\phi(x) \ = \ \ x < 5.\]
% V tem primeru $x$ imenujemo \df{spremenljivka}, saj ne gre za točno določeno vrednost, pač pa predstavlja splošno število (recimo, da se dogovorimo, da s $\phi$ označujemo lastnost na realnih številih).

% Tovrstne lastnosti nam omogočajo, da iz neke množice odberemo elemente z dano lastnostjo in na ta način dobimo novo množico, ki je podmnožica prejšnje. Množico vseh realnih števil, ki so manjša od pet, zapišemo na naslednji način.
% \[\set{x \in \RR}{x < 5}\]
% Seveda, ker je primerjava s števili zelo pogosta lastnost, je uporabno, če uvedemo krajše oznake, ki povejo isto; že prej smo se dogovorili, da tako množico označimo z $\RR_{< 5}$. Za povsem splošne lastnosti pa ne bomo imeli vnaprej dogovorjenih oznak, zato je dobro, da poznamo splošni zapis. Torej, če je $X$ poljubna množica in $\phi$ poljubna lastnost njenih elementov, tedaj podmnožico, ki vsebuje točno tiste elemente množice $X$, ki zadoščajo lastnosti $\phi$, označimo takole.
% \[\set[1]{x \in X}{\phi(x)}\]

% Pri tem se zavedajmo: ni pomembno, da spremenljivko označimo ravno z $x$. Zapis
% \[\set[1]{y \in X}{\phi(y)}\]
% še vedno označuje isto množico. V vsakem primeru gre za množico vseh elementov iz $X$ z lastnostjo $\phi$. Pravzaprav sploh ni nujno, da uporabimo črko; poslužimo se lahko kateregakoli simbola (ki mu nismo predtem že predpisali določenega pomena). Taisto množico lahko zapišemo tudi $\set{\heartsuit \in X}{\phi(\heartsuit)}$.

% Kadar imamo spremenljivko, ki jo lahko preimenujemo, ne da bi spremenili pomen izraza, jo imenujemo \df{nema spremenljivka}. Takšne primere že dobro poznate; na primer, integral $\int_0^1 x^2 \,dx$ se ne spremeni, če preimenujete spremenljivko in zapišete $\int_0^1 y^2 \,dy$.

% \begin{zgled}
% Kako bi zapisali množico vseh sodih naravnih števil? Spomnimo se, da je število sodo, kadar je deljivo z $2$. Za $n \in \NN$ to zapišemo takole: $2 \divides n$ (beri: ">dve deli $n$"<). Množica sodih naravnih števil se potem zapiše kot
% \[\set[1]{n \in \NN}{2 \divides n}.\]
% \end{zgled}


\section{Vaje}


\begin{vaja}
  Zapišite pravila za zmnožek treh množic. Nato premislite še, kako bi podali pravila za
  zmnožek $n$ množic, kjer je~$n$ naravno število. Seveda ne velja uporaba tropičja `\dots`!
\end{vaja}

\begin{vaja}
  Naštejte vse elemente množice $\one + \one + \one$.
\end{vaja}

\begin{vaja}
  Podajte primer izomorfizma $f : \RR \to \RR$ in števila $x \in \RR$, da velja
  $\inv{f}(x) = \inv{(f(x))}$. Nato podajte še primer, ko velja
  $\inv{f}(x) \neq \inv{(f(x))}$.
\end{vaja}

\begin{vaja}
  Ozrimo se še enkrat na dokaz trditve~\ref{trditev:enolicen-inverz}. Ali smo uporabili vse štiri predpostavke?
  Zapišite bolj splošno trditev, se pravi tako, ki navede samo tiste predpostavke,
  ki jih res potrebujemo v dokazu.
\end{vaja}

\begin{vaja}
  Pogosto rečemo, da sta seštevanje in odštevanje obratni operaciji. Strogo vzeto, ti dve
  operaciji nista obratni kot preslikavi, saj obe slikata (recimo, da ju gledamo na
  realnih številih) $\RR \times \RR \to \RR$, tj.~ne slikata v nasprotnih smereh. Ugotovi,
  v kakšnem smislu točno sta seštevanje in odštevanje obratni, tj.~kateri dve preslikavi
  sta pravzaprav druga drugi obratni.
\end{vaja}

\begin{vaja}
  Preveri tiste izomorfnosti iz izreka~\ref{izrek:aritmetika-mnozic}, ki jih v
  razdelku~\ref{sec:aritmetika-mnozic} nismo utemeljili.
\end{vaja}
