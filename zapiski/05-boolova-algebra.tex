\chapter{Boolova algebra}

\section{Resničnostne tabele}

Vsaka izjava ima **resničnostno vrednost**. Resničnostni vrednosti sta `⊥`
(resnica) in `⊤` (neresnica).

Primer: `⊥ ∨ (⊤ ⇒ ⊤)` je resnična, njena resničnostna vrednost je `⊤`.

Primer: `2 + 2 = 5` je neresnična, njena resničnostna vrednost je `⊥`.

Kadar izjava vsebuje spremenljivke (pravimo jim tudi *parametri*), je njena
resničnostna vrednost *odvisna* od parametrov.

*Primer:* Naj bosta `x, y ∈ N`. Resničnostna vrednost izjave `x + y < 3` je odvisna
od `x` in `y`, kar lahko prikažemo z **resničnostno tabelo**:

    x | y | x + 2 * y < 3
    ---------------------
    0 | 0 |    ⊤
    0 | 1 |    ⊤
    1 | 0 |    ⊤
    2 | 0 |    ⊤
    1 | 1 |    ⊥
    0 | 2 |    ⊥
    ...

Kot vidimo, je lahko takšna tabela neskončna, kar ni praktično.

V izjavi lahko nastopajo tudi **izjavne spremenljivke** ali **izjavni simboli**,
to se spremenljivke, ki zavzamejo vrednosti `⊥` in `⊤`.

**Primer:** Naj bosta `p, q ∈ 2`. Tedaj je `¬ p ∨ q` izjava, katere resničnostna tabela je

    p | q | ¬ p ∨ q
    ---------------
    ⊥ | ⊥ |    ⊤
    ⊥ | ⊤ |    ⊤
    ⊤ | ⊥ |    ⊥
    ⊤ | ⊤ |    ⊤

Izjava `φ(p_1, …, p_n)`, v kateri nastopajo izjavne spremenljivke `p_1`, ...,
`p_n` (in nobeni drugi parametri) določa preslikavo

    2 × ⋯ × 2 → 2

s predpisom

    (p_1, …, p_n) ↦ φ(p_1, …, p_n)

Preslikavi, ki slika iz produkta `2 × ⋯ × 2` v `2` pravimo **Boolova
preslikava**. Prikažemo jo lahko z resničnostno tabelo. Če ima preslikava `n`
argumentov, ima tabela `2^n` vrstic.

\subsection{Tavtologije}

Izjava je **tavtologija**, če je njena resničnostna vrednost `⊤` ne glede na
vrednosti parametrov. Premisli: kako iz resničnostne tabele razberemo, ali je
izjava tavtologija?

> **Izrek:** Naj bo `φ` izjava, v kateri nastopajo le izjavni simboli
> `p_1,…,p_n`. Tedaj velja:
>
> 1. Če je `φ` tavtologija, potem ima dokaz.
> 2. Če ima `φ` dokaz, je tavtologija.

*Dokaz.* Dokaz najdete v N. Prijatelj: *Osnove matematične logike* (1. del).

Izrek je pomemben, ker nam pove, da lahko dokazovanje izjav v nekaterih primerih
nadomestimo s preverjanjem resničnostnih tabel.

*Opomba:** Izrek velja samo za izjave, ki jih sestavimo iz izjavnih simbolov,
`⊥`, `⊤` in logičnih veznikov `¬`, `∧`, `∨`, `⇒`, `⇔`. Za splošne izjave, ki
vsebujejo tudi `∀` in `∃` izrek *ne* velja.

\subsection{Polni nabori}

Vsaka formula v izjavnem računu ima resničnostno tabelo. Ali lahko vsako tabelo
dobimo kot resničnostno tabelo neke formule? Na primer, ali obstaja formula,
katere resničnostna tabela se glasi

    p | q | ?
    ----------
    ⊥ | ⊥ | ⊥
    ⊥ | ⊤ | ⊤
    ⊤ | ⊥ | ⊤
    ⊥ | ⊥ | ⊥

Odgovor je pritrdilen. Na kratko povejmo, kako dobimo tako izjavo. Imamo dve
množnosti.

**Disjunktivna oblika:** za vsako vrstico v tabeli, ki ima vrednost `⊤` zapišemo
konjunkcijo simbolov in njihovih negacij, pri čemer negiramo tiste simbole, ki
imajo v dani vrstici vrednost `⊥`. Na primer, v zgornji tabeli imata druga in
tretja vrstica vrednost `⊤`, zanju zapišemo konjukciji:

* 2. vrstica: `¬ p ∧ q`
* 3. vrstica: `p ∧ ¬ q`

Nato tvorimo disjunkcijo tako dobljenih konjukcij:

    (¬ p ∧ q) ∧ (p ∧ ¬ q)

Dobljena formula ima želeno resničnostno tabelo.

**Konjunktivna oblika:** za vsako vrstico v tabeli, ki ima vrednost `⊥` zapišemo
disjunkcijo simbolov in njihovih negacij, pri čemer negiramo tiste simbole, ki
imajo v dani vrstici vrednost `⊤`. Na primer, v zgornji tabeli imata prva in
čertrta vrstica vednost `⊥`, zanju zapišemo disjukciji:

* 1. vrstica: `p ∨ q`
* 4. vrstica: `¬ p ∨ ¬ q`

Nato tvorimo konjunkcijo tako dobljenih disjunkcij:

    (p ∨ q) ∧ (¬p ∨ ¬q)

Zgornjo tabelo bi lahko dobili tudi kot resničnostno tabelo formule

    p ⇔ q

Vidimo, da lahko vsako resničnostno tabelo dobimo z uporabo veznikov `¬`, `∨` in
`∧`. **Polni nabor** je tak izbor veznikov, k katerim lahko dobimo vsako
resničnostno tabelo.

Torej je `¬`, `∨`, `∧` poln nabor. Lahko bi ga še zmanjšali na `¬`, `∧`, saj lahko

    p ∨ q

izrazimo kot

    ¬p ∧ ¬q.

\section{Boolova algebra}

Ekvivalentni izjavi imata enake resničnostne vrednosti, torej lahko ekvivalenco
`⇔` obravnavamo kar kot enakost, saj to tudi je, kar se tiče resničnostnih
vrednosti. Zato lahko namesto `p ⇔ q` pišemo tudi `p = q`, če imamo v mislih le
resničnostne vrednosti.

(*Opomba:* izjavi sta lahko ekvivalentni, a nimata enakega *pomena*. Na primer,
izjavi `∀ x, y ∈ R . x + y = y + x` in `∀ α ∈ R . sin(2 α) = 2 · cos α · sin α`
sta ekvivalentni, saj sta obe resnični, a ne moremo reči, da je njun pomen enak.)

Za logične veznike veljajo *algebrajska pravila*. Ta pravila lahko uporabljamo
kot računska pravila, s katerimi lahko izjavo poenostavmi v ekvivalentno obliko.
Pogosto je tako računanje bolj prikladno kot dokazovanje. Spodaj našteta pravila
lahko preverimo tako, da zapišemo resničnostne tabele izjav in jih primerjamo.

Pravilom, ki veljajo za logične veznike, pravimo **Boolova algebra**.

\subsection{Pravila za `⊤` in `⊥`}

* `⊤ ∨ p = ⊤` (`⊤` absorbira `∨`)
* `⊤ ∧ p = p` (`⊤` je nevtralni element za `∧`)
* `¬ ⊤  = ⊥`

* `⊥ ∧ p = ⊥` (`⊥` absorbira `∧`)
* `⊥ ∨ p = p` (`⊥` je nevtralni element za `∧`)
* `¬⊥ = ⊤`

\subsection{Pravila za negacijo `¬`}

* `¬¬p = p` (negacija je involucija)
* de Morganovi pravili:
    * `¬(p ∧ q) = ¬p ∨ ¬q`
    * `¬(p ∨ q) = ¬p ∧ ¬q`

\subsection{Pravila za konjukcijo in disjunkcijo}

* `p ∧ q = q ∧ p` (konjunkcija je komutativna)
* `p ∧ p = p` (konjunkcija je idempotentna)
* `(p ∧ q) ∧ r = p ∧ (q ∧ r)` (konjunkcija je asociativna)

* `p ∨ q = q ∨ p` (disjunkcija je komutativna)
* `p ∨ p = p` (disjunkcija je idempotentna)
* `(p ∨ q) ∨ r = p ∨ (q ∨ r)` (disjunkcija je asociativna)

Absorbcijski pravili:

* `p ∧ (p ∨ q) = p`
* `p ∨ (p ∧ q) = p`

Distributivnostni pravili:

* `p ∧ (q ∨ r) = (p ∧ q) ∨ (p ∧ r)`
* `p ∨ (q ∧ r) = (p ∨ q) ∧ (p ∨ r)`

\subsection{Ostala pravila}

* `p ∨ ¬ p = ⊤` (izključena tretja možnost)
* `p ∧ ¬ p = ⊥`
* `(p ⇒ q) = (¬q ⇒ ¬p)`
* `(p ⇒ q) = ¬p ∨ q`

\subsection{Ekvivalence za kvantifikatorje}

Zapišimo še uporabna logična pravila za kvantifikatorje. Tokrat uporabimo `⇔`
namesto `=`, ker je to bolj običajno.

* `(∀ x ∈ ∅ . φ(x)) ⇔ ⊤`
* `(∃ x ∈ ∅ . φ(x)) ⇔ ⊥`
* `(∀ x ∈ {a} . φ(x)) ⇔ φ(a)`
* `(∃ x ∈ {a} . φ(x)) ⇔ φ(a)`
* `(¬ ∀ x ∈ A . φ(x)) ⇔ ∃ x ∈ A . ¬ φ(x)`
* `(¬ ∃ x ∈ A . φ(x)) ⇔ ∀ x ∈ A . ¬ φ(x)`
* `(ψ ⇒ ∀ x ∈ A . φ(x)) ⇔ ∀ x ∈ A . ψ ⇒ φ(x)`
* `(ψ ∨ ∀ x ∈ A . φ(x)) ⇔ ∀ x ∈ A . ψ ∨ φ(x)`
* `(ψ ∧ ∃ x ∈ A . φ(x)) ⇔ ∃ x ∈ A . ψ ∧ φ(x)`
* `(∀ u ∈ A × B . φ(u)) ⇔ ∀ x ∈ A . ∀ y ∈ B . φ(x, y)`
* `(∃ u ∈ A × B . φ(u)) ⇔ ∃ x ∈ A . ∃ y ∈ B . φ(x, y)`
* `(∀ u ∈ A + B . φ(u)) ⇔ (∀ x ∈ A . φ(\inl(x))) ∧ (∀ y ∈ B . φ(\inr(y)))`
* `(∀ u ∈ A ∪ B . φ(u)) ⇔ (∀ x ∈ A . φ(x)) ∧ (∀ y ∈ B . φ(y))`
* `(∃ u ∈ A + B . φ(u)) ⇔ (∃ x ∈ A . φ(\inl(x))) ∨ (∃ y ∈ B . φ(\inr(y)))`
* `(∃ u ∈ A ∪ B . φ(u)) ⇔ (∃ x ∈ A . φ(x)) ∨ (∃ y ∈ B . φ(y))`
* `(∀ u ∈ {x ∈ A | ψ(x)} . φ(u)) ⇔ ∀ x ∈ A . ψ(x) ⇒ φ(x)`
* `(∃ u ∈ {x ∈ A | ψ(x)} . φ(u)) ⇔ ∃ x ∈ A . ψ(x) ∧ φ(x)`

Te ekvivalence je treba preveriti tako, da jih dokažemo.
