\chapter{Simbolni zapis}

V matematiki uporabljamo **simbolni zapis** – matematične objekte, konstrukcije in dokaze opišemo s pomočjo izrazov kot
so `3 + 4`, `x ↦ x² + 3`, `∀ x ∈ ℝ . x² + x + 1 ≥ 1/4` itd. Matematično besedilo je mešanica naravnega jezika
(slovenščine) in simbolnega zapisa. Načeloma bi lahko pisali matematiko *samo* s simbolnim zapisom (kar dejansko
počnemo, kadar matematiko *formaliziramo* z računalnikom, a o tem kdaj drugič), a bi bilo to ljudem preveč nerazumljivo.

Danes bomo spoznali pravila simbolnega zapisa in se učili razumeti, brati in pisati logične formule (matematične izjave,
izražene s simbolnim zapisom).

\section{Izrazi}

**(Simbolni) izraz** je zaporedje znakov, ki predstavlja neki matematični pojem, na primer

    3 + 5
    S ∩ (T ∪ V)
    2 x y ≤ x² + y²

Izraz je *pravilno formiran* ali *sintaktično pravilen*, če ustreza pravilom, ki določajo kako postavljamo oklepaje,
vejice, pike, kako uporabljamo razne posebne simbole (`+`, `∨`, `∫`) itd. Na primer, izraz `3 + ) x · 4` ni sintaktično
pravilen, ker ima narobe postavljen zaklepaj.

Natančna sintaktična pravila za pisanje matematičnih izrazov so precej zapletena, ker je matematični zapis raznovrsten
in se je razvijal skozi zgodovino. Na srečo skoraj vsa pravila že poznate ("vsak oklepaj mora imeti ustrezni zaklepaj",
"piše se `a + b` in ne `a b +`, ...). Tu se ne bomo ukvarjali s podajanjem vseh pravil – to je delo za računalničarje,
ki želijo taka pravila naprogramirati. Kljub temu pa velja omeniti nekatere pojme.


\subsection{Prefiksne, postfiksne in infiksne operacije}

V simbolnem zapisu uporabljamo *operacije*, ki jih pišemo pred, za ali med argumente:

* **prefiksne operacije** so take, ki jih pišemo *pred* argument:
    * `-x` za nasprotno vrednost `x`
    * `¬P` za negacijo izjave `P`
* **infiksne operacije** so take, ki jih pišemo *med* argumenta:
    * aritmetične operacije `x + y`, `x - y`, `x · y` itn.
    * logični vezniki `P ∧ Q`, `P ∨ Q`, `P ⇒ Q` itn.
* **postfiksne operacije** so take, ki jih pišemo *za* argument:
    * `n!` za faktorielo števila `n`

Včasih uporabljamo tudi druge oblike zapisa:

* *potenciranje* `Aᴮ`
* *ulomki* `a/b` (mišljen je ulomek, kjer so `a`, črta in `b` zapisani navpično)
* integrali in vsote
* zapis podmnožice `{ x ∈ ℝ | x² + x > 2}`

Operacija je lahko celo "nevidna", oziroma jo pišemo kot presledek med argumentoma:

* `x y` kot zmnožek `x` in `y`
* `sin x` kot uporabe funkcije `sin` na argumentu `x`


\subsection{Oklepaji, prioriteta in asociiranost}

Z oklepaji ponazorimo, katera operacija ima prednost. Na primer, če ne bi imeli dogovora, da ima množenje prednost pred
seštevanjem, potem bi lahko izraz

    3 + 4 × 5

razumeli kot `3 + (4 × 5)` ali kot `(3 + 4) × 5`. Oklepajev ne smemo opustiti, kadar bi lahko prišlo do take zmede.
Nikoli pa ne škodi, če zapišemo kak oklepaj več, kot je to potrebno (v mejah normale).

Da se izognemo pisanju oklepajev, se dogovorimo, da imajo nekatere operacije prednost pred ostalimi, kar so vas učili že
v osnovni šoli. Pravimo, da imajo operacije **prioriteto**. Operacija z višjo prioriteto ima prednost pred operacijo z
nižjo prioriteto.

Primer:

* `×` ima višjo prioriteto kot `+` (to je *dogovor* in ne matematično dejstvo)
* `∧` ima višjo priroriteto kot `∨`

Poleg prioritete imajo nekatere operacije tudi **asociiranost**. Kako naj razumemo izraz `8 - 3 - 2`, kot `(8 - 3) - 2`
ali kot `8 - (3 - 2)`? V šoli so vas učili, da je

    A - B - C = (A - B) - C

Pravimo, da `-` veže na levo oziroma da ima **levo asociiranost**. Ker beremo z leve na desno, ima večina operacij levo
asociiranost. Velja na primer

    A + B + C = (A + B) + C
    A × B × C = (A × B) × C

Morda bo kdo pripomnil, da velja `(A + B) + C = A + (B + C)` in da zato ni pomembno, kako razumemo `A + B + C`. To je
res v preprostih primerih, ko vemo, da smo s `+` označili seštevanje števil. Kaj pa, če s `+` označimo kako drugo
preslikavo? Ali `(A + B) + C = A + (B + C)` velja tudi v programiskih jezikih, pri katerih lahko pride do prekoračitve
največjega možnega števila?

Primer operacije z desno asociiranostjo je implikacija: `P ⇒ Q ⇒ R` je enako `P ⇒ (Q ⇒ R)`.


\subsection{Izrazi predstavljajo drevesa}

Izrazi so zaporedja znakov, ki jih pišemo z leve na desno. A kje drugje bi jih morda pisali z desne na levo ali
navpično. Izrazi so le *predstavitve* tako imenovanih **sintaktičnih dreves**. Na primer `((3 + x) × y)²` predstavlja
sintaktično drevo (potenciranje predstavimo z znakom `^`)

           ^
          / \
         ×   2
        / \
       +   y
      / \
     3   x

O sintaktičnih drevesih ne bomo govorili, a jih omenimo, ker so pomembna iz dveh razlogov: sintaktična drevesa so
*podatkovni tip*, s katerim v programu dejansko obdelujemo izraze; s pomočjo sintaktičnih dreves lahko simbolni zapis
predstavimo kot posebno vrsto algebre, ki razkriva pomembno matematično strukturo izrazov.


\subsection{Ostala sintaktična pravila}

Sintaktičnih pravil je še več, od katerih omenimo le tri.

Argumente operacije ali funkcije včasih zapišemo v **podnapis** ali **nadnapis**. Na primer, če je `a : ℕ → ℝ`
preslikava, pogosto pišemo `aᵢ` namesto `a(i)`.

Argumente operacije lahko opustimo in od bralca pričakujemo, da bo pravilno uganil, kaj smo mislili. Pravimo, da so to
**implicitni argumenti**. Primer implicitnih argumetov smo že videli, ko smo zapisali prvo in drugo projekcijo `pr₁` in
`pr₂`:

    pr₁ : A × B → A
    pr₂ : A × B → A

Če bi bili zelo natančni, bi morali pri projekcijah zapisati tudi množici `A` in `B`, ki tvorita kartezični produkt, na
primer nekaj takega kot

    pr₁ᴬ,ᴮ : A × B → A

Ko torej vpeljemo novo zapis, lahko nekatere argumente razglasimo za *implicitne*, kar pomeni, da jih bomo opuščali,
kadar to ne pripelje do zmede.

**Naloga:** ali ima kompozicija preslikav `∘` implicitne argumente? Katere?

Simbol lahko tudi **preobtežimo**, da ima več pomenov, nato pa od bralca pričakujemo, da bo uganil, katerega smo
mislili. Na primer, `+` uporabljamo za:

* seštevanje naravnih števil
* seštevanje celih števil
* seštevanje racionalnih števil
* seštevanje realnih števil
* seštevanje kompleksnih števil
* seštevanje vektorjev
* seštevanje matrik
* itd.

S preobteževanjem ne gre pretiravati, ker lahko pripelje do zmede. Običajno z istim simbolom označimo različne
operacije, ki imajo kaj skupnega. Na primer, `+` uporabljamo za komutativno, asociativno operacijo z nevtralnim
elementom.


\section{Logične formule}

Izrazi, ki označujejo števila, se imenujejo **aritmetični izrazi**.

Irazi, ki označujejo matematične izjave, se imenujejo **logični izrazi** ali **logične formule**. Razumevanje, branje in pisanje le-teh zahteva kar nekaj treninga, zato se mu bomo posvetili tu in na vajah. Pravzaprav ne bomo vadili le razumevanja zapisa, ampak tudi, kako matematiki razmišljajo in razumejo drug drugega.

Danes o dokazih in pravilih dokazovanja še ne bomo govorili, bomo pa pojasnili intuitivni pomen logičnih operacij.

Logične formule in logika nasploh sestoji iz:

* **izjavni račun** zaobjema logične veznike `¬`, `∧`, `∨`, `⇒`, `⇔`
* **predikatni račun** zaobjema izjavni račun, ter kvantifikatorja `∀` in `∃`


\subsection{Izjavni račun}

**Izjavn vezniki** so naslednje operacije:

* **resničnostni konstanti** `⊥` in `⊤`: beremo ju "neresnica" in "resnica"

* **negacija** `¬`: izjavo `¬A` beremo "`A` ne velja" ali "ni res, da `A`"

* **konjunkcija** `∧`: izjavo `A ∧ B` beremo "`A` in `B`"

* **disjunkcija** `∨`: izjavo `A ∨ B` beremo "`A` ali `B`"

* **implikacija** `⇒`: izjavo `A ⇒ B` lahko beremo na več načinov:

   * "Iz `A` sledi `B`."
   * "Če `A`, potem `B`."
   * "`A` samo če `B`."
   * "`B` sledi iz `A`."
   * "`A` je zadosten pogoj za `B`."
   * "`B` je potreben pogoj za `A`."

* **ekvivalenca** `⇔`: izjavo `A ⇔ B` beremo

   * "`A` je ekvivalentno `B`."
   * "`A`, če in samo če `B`."
   * "`A` natanko tedaj, ko `B`."
   * "`A` je zadosten in potreben pogoj za `B`."

Malo bolj neobičajna je:

* **ekskluzivna disjunkcija** `⊕`: izjava `A ⊕ B` beremo "bodisi `A` bodisi `B`

Prioriteta veznikov, od najvišje do najnižje:

* `¬`
* `∧`
* `∨`, `⊕`
* `⇒`, `⇔`

Primer: `¬ A ∧ B ⇒ C ∨ D` beremo kot `((¬ A) ∧ B) ⇒ (C ∨ D)`.

Asociranost veznikov:

* leva asociiranost: `∧`, `∨`, `⊕`
* desna asociiranost: `⇒`

Ekvivalenca `⇔` nima asociiranost, zato je zapis `A ⇔ B ⇔ C` načeloma dvoumen, a v praksi pomeni `(A ⇔ B) ∧ (B ⇔ C)`

**Opomba:** Tudi zapis `x = y = z` pravzaprav ni smiselen, saj sta `(x = y) = z` in `x = (y = z)` oba nesmiselna. V
praksi `x = y = z` pomeni `(x = y) ∧ (y = z)`. Pa še to: koliko enačb je izraženih z `a = b = c = d`? Tri! Toliko kot je
enačajev.

**Opomba k opombi:** Zapis `x ≠ y ≠ z` je nejasen in se mu je bolje izogibati, saj zlahka pripelje do pomote, ker iz `x
≠ y` in `y ≠ z` ne sledi nujno `x ≠ z`.

Glede razumevanja veznikov, omenimo:

* disjunkcija je *inkluzivna*, kar pomeni, da je `A ∨ B` resnična izjava, če sta `A` in `B` resnični,
* v implikaciji `A ⇒ B` se `A` imenuje **antecedent** in `B` **konsekvent**. Implikacija je veljavna, če je antecedent neveljaven,
* ekvivalenco `A ⇔ B` lahko razumemo kot okrajšavo za `(A ⇒ B) ∧ (B ⇒ A)`.


\subsection{Kvantifikatorja}

Matematične izjave vsebujejo fraze, kot so "za vse", "za neki", "obstaja vsaj en", "za natanko enega" ipd. Le-te izrazimo s **kvantifiaktorji**. Osnovna kvantifikatorja sta **univerzalni** in **eksistenčni**.


\subsubsection{Univerzalni kvantifikator `∀`}

Formulo

    ∀ x ∈ A . ϕ

beremo:

* Za vsak `x` iz `A` velja `ϕ`
* Vsi `x` iz `A` zadoščajo `ϕ`
* `ϕ` za vse `x` iz `A`

Pika pri tem nima nobenega posebnega pomena, lahko bi pisali tudi

* `∀ x ∈ A , ϕ`
* `∀ x : A , ϕ`
* `(∀ x : A) ϕ`

Nekateri matematiki pišejo tudi `ϕ, ∀ x ∈ A`. Ta zapis odsvetujemo, ker ne deluje, ko kombiniramo več
kvantifikatorjev hkrati. (Dobro pa je razumeti, da tako pišejo, ker se lepo sliši "`ϕ` za vse `x` iz `A`".)

Omenili smo že, da `∀ x ∈ ∅ . ϕ` vedno velja. To bomo utemeljili naslednjič.

\subsubsection{Eksistenčni kvantifikator `∃`}

Formulo

    ∃ x ∈ A . ϕ

beremo:

* Obstaja `x` iz `A` velja `ϕ`
* Obstaja vsaj en `x` iz `A` velja `ϕ`
* Za neki `x` iz `A` velja `ϕ`
* `ϕ` za neki `x` iz `A`

S tem povemo, da obstaja *eden ali več* takih `x`. Na primer, izjava `∃ x ∈ ℕ . x
< 3` je veljavna, saj obstajo kar tri naravna števila, ki so manjša od `3`.


\subsubsection{Prioriteta `∀` in `∃`}

Prioriteta kvantifikatorjev `∀` in `∃` je nižja od prioritete veznkov. Na primer:

* `∀ x ∈ A . ϕ ∧ ψ` je enako `∀ x ∈ A . (ϕ ∧ ψ)`
* `∀ x ∈ ℝ . x > 0 ⇒ ϕ` je enako ``∀ x ∈ ℝ . (x > 0 ⇒ ϕ)`

Kvantifikator vedno zaobjame vse, kar zmore:

* `∀ x ∈ A . ϕ ∧ ∃ y ∈ B . ψ` je enako `∀ x ∈ A . (ϕ ∧ (∃ y ∈ B . ψ))` in *ni* enako `(∀ x ∈ A . ϕ) ∧ (∃ y ∈ B . ψ)`

* `(P ∧ ∀ x ∈ A . Q ⇒ R) ⇒ ∃ y ∈ B . S` je enako `(P ∧ ∀ x ∈ A . (Q ⇒ R)) ⇒ (∃ y ∈ B . S)` in *ni* enako 
  `(P ∧ (∀ x ∈ A . Q) ⇒ R) ⇒ (∃ y ∈ B . S)`


\subsubsection{Kombinacija `∀` in `∃`}

Pozor, vrstnega reda kvantifikatorjev ne smemo mešati!

Primer:

* `∀ x ∈ ℝ . ∃ y ∈ ℝ . x < y` pomeni "vsako realno število je manjše od nekega realnega števila" (kar je res),
* `∃ x ∈ ℝ . ∀ y ∈ ℝ . x < y` pomeni "obstaja najmanjše realno število" (kar ni res).

To dejstvo bomo utrjevali na vajah. Zapomnite se, da morate biti tudi pri ostalih predmetih posebej pozorni na vrstni red
"za vsak" in "obstaja". Je profesorica pri analizi rekla "za vsak `ε > 0` obstaja tak `δ > 0` da ..." ali je rekla
"ostaja tak `δ > 0` da za vsak `ε > 0` ...". Če boste zamešali ti dve izjavi na ustnem izpitu iz analize,
boste imeli pokvarjen dan, ali pa cele počitnice!


\subsubsection{Kvantifikator z dodatnim pogojem}

Pogosto kvantifikacijo kombiniramo z dodatnim pogojem, na primer

* "Obstaja liho naravno število, ki ni deljivo s 7."
* "Vsako sodo naravno število je deljivo s 3."

V prvem primeru je dodatni pogoj izražen z besedico "liho" in v drugem s "sodo". Kako zapiemo take izjave s formulo, kam
vtaknemo dodatni pogoj? Ijzavi pretvorimo po korakih:

* "Obstaja liho naravno število, ki ni deljivo s 7."
* "Obstaja naravno število, ki je liho in ki ni deljivo s 7."
* "Obstaja naravno število, ki je liho in deljivo s 7."
* "Obstaja `x` iz `ℕ`, da je `x` lih in `x` je deljiva s 7."
* `∃ x ∈ ℕ . (x je lih) ∧ (x je deljiv s 7)`

In še druga izjava:

* "Vsako sodo naravno število je deljivo s 3."
* "Vsako naravno število, ki je sodo, je deljivo s 3."
* "Za vsako naravno število velja, da če je sodo, potem je deljivo s 3."
* "Za vsak `x` iz `ℕ` velja, če je `x` sod, potem je `x` deljiv s `3`."
* `∀ x ∈ ℕ . x sod ⇒ x deljiv s 3`

Zapomnimo si: **dodatni pogoj pri `∃` izrazimo `∧`** in **dodatni pogoj pri `∀` izrazimo `⇒`**.

\subsubsection{Vezane in proste spremenljivke}

V nekaterih izrazih nastopajo spremenljivke, ki so **vezane**. To pomeni, da je njihovo območje veljavnosti imejeno,
oziroma da so neke vrste lokalne spremenljivke. Spremenljivka, ki ni vezana, je **prosta**. Primeri:

* V funkcijskem predpisu `x ↦ x² + y` je `x` vezan in `y` prost.
* V funkcijskem predpisu `(x,y) ↦ x² + y` sta `x` in `y` vezana.
* V integralu `∫ (x + a)² d x` je `x` vezan in `a` prost.
* V vsoti `∑_(i = 0)^n (i² + 1)` je `i` vezan in `n` prost.
* V formuli `∀ x ∈ ℝ . x³ + 3 x < 7` je `x` vezana spremenljivka.

Če vezano spremenljivko preimenujemo, se izraz ne spremeni:

**Pozor:** za novo ime vezane spremenljivke moramo izbrati spremenljivko, ki ni prosta. Na primer, v integralu

    ∫ (a + x)² d t

smemo `x` preimenovati v `t`:

    ∫ (a + t)² d t

Ta dva izraza štejemo za *enaka*. Ne bi pa smeli `x` preimenovati v `a`:

    ∫ (a + a)² d a

Pravimo, da se je prosta spremenljivka `a` *ujela* v integral.

