\chapter{Simbolni zapis}

Tako kot vsaka stroka ima tudi matematika svoj strokovni jezik, ki obsega matematične
simbole in izraze ter svojevrsten način izražanja. Matematiki stremimo k popolni
natančnosti in nedvoumnosti matematične misli. To je seveda le ideal, ki se mu bolj ali
manj približamo, dejanska matematična besedila pa pišemo ljudje za ljudi, zato ni nič
nenavadnega, da so prežeta s tradicijo in nepisanimi družbenimi dogovori, ki matematiko
oddaljijo od formalnega ideala, a jo tudi naredijo humano.
%
Pred študentom matematike je torej težka naloga, saj se mora hkrati z novo matematiko
učiti še nekoliko nenavadnega jezika. V pomoč se zato najprej posvetimo samo formi
matematičnega izražanja. In ne zamerite nam, če vam dobrohotno ponudimo še kak nasvet o
študiju matematike.

Matematično komuniciranje je raznoliko, saj je namenjeno različnim publikam in zato
posredovano na različne načine. Tako v raziskovalnem matematičnem članku ne bomo našli
pojasnil in izračunov, ki jih profesor matematike zahteva od svojih študentov. In verjetno
ni dveh matematikov, ki bi uporabljala povsem usklajen matematični zapis in izrazoslovje.
Kljub temu je matematični jezik skupen vsem matematikom in v večji meri poenoten.
Nesporazume, ki nastopijo zaradi različnih navad, pa lahko rešimo s pogovorom. Vsi
izkušeni matematiki vedo, da vedo zelo malo in zato vprašajo, ko česa ne vedo. To naj bo
torej prvi nasvet: vprašajte in če ne dobite odgovora, vprašajte še enkrat.

Ker je namen tega učbenika postaviti dobre osnove matematičnega izražanja in mišljenja,
bomo bolj natančni kot večina matematikov v praksi. Začetnik namreč potrebuje oporo v
natančnosti, kasneje, ko razume stvari bolje, pa lahko ubere bližnjice, ki jih bolj
izkušeni kolegi uporabljajo, ne da bi to sploh opazili. Sproti bomo opozarjali nanje,
kakor tudi na manjše nedoslednosti v matematični praksi, ki izhajajo iz zgodovinskega
razvoja matematike.

% * Sestavni deli besedila, brez podrobnih razlag, morda primeri, tu samo opozorimo na
%   raznovrstnost konceptov.
%    * spremno besedilo
%    * konstrukcije
%    * računi
%    * izjave (sinonimi, kako jih številčimo)
%    * dokazi (kako so označeni)
%    * definicije
%    * zgledi
%    * naloge in rešitve (namigi)
%    * formule in izrazi (kako jih številčimo in kako se nanje sklicujemo)
%    * citati in reference


% PRIMERI DRUŽBENIH DOGOVOROV

% $\vec{a}$ uporabljamo za oznako vektorja

% Pri algebri v prvem letniku na FMF je običaj, da se linearno preslikavo označi z
% $\mathcal{A}$, matriko, ki tej linearni preslikavi pripada, pa z $A$.

\section{Pisave in simboli}
\label{sec:pisave-in-simboli}

Matematična abeceda vsebuje precej več simbolov, kot zgolj običajne črke in števke. Nekatere že poznamo, na primer~$=$, $<$, $+$, $\emptyset$, $\cup$, $\cap$, $\int$ in tako naprej, precej jih še bomo spoznali. Poleg tega matematiki uporabljamo različne pisave, kot je prikazano v tabeli~\ref{tabela:oblike-crk}. Na tabli in v zvezku sicer težko ločimo med pokončno, odebeljeno in ležečo pisavo, ali med kaligrafsko in rokopisno, zato nabor pisav omejimo. V tiskanem besedilu se vedno držimo nekaterih pravil glede izbire pisav. Tako posamezne črke $a$, $b$, $c$, \ldots, $x$, $y$, $z$ pišemo v ležeči pisavi, imena elementarnih funkcij pa pokončno: $\sin$, $\cos$, $\log$, \ldots Šumnikov običajno ne uporabljamo. Včasih z uporabo znakov nakažemo povezavo med dvema objektoma: $f$ je funkcija in $F$ njen integral, $\mathcal{A}$ je linearna preslikava in $A$ njej pripadajoča matrika itd.

\begin{table}[ht]
\centering
\begin{tabular}{c|c}
\textbf{Pisava} & \textbf{Črke} \\
\hline
pokončna & $\mathrm{ABCDEFGHIJKLMNOPQRSTUVWXYZ}$ \\
odebeljena & $\mathbf{ABCDEFGHIJKLMNOPQRSTUVWXYZ}$ \\
ležeča & $ABCDEFGHIJKLMNOPQRSTUVWXYZ$ \\
kaligrafska & $\mathcal{ABCDEFGHIJKLMNOPQRSTUVWXYZ}$ \\
rokopisna & $\mathscr{ABCDEFGHIJKLMNOPQRSTUVWXYZ}$ \\
frakturna & $\mathfrak{ABCDEFGHIJKLMNOPQRSTUVWXYZ}$ \\
dvopoudarjena & $\mathbb{ABCDEFGHIJKLMNOPQRSTUVWXYZ}$
\end{tabular}
\caption{Pisave}\label{tabela:oblike-crk}
\end{table}

Črke lahko dodatno opremimo s črticami, vijugami, vektorskimi znaki, strešicami in podobno:
%
\begin{equation*}
 a \quad
 a' \quad
 \dot{a} \quad
 \bar{a} \quad
 \vec{a} \quad
 \tilde{a} \quad
 \hat{a} \quad
 \check{a}.
\end{equation*}
%
Uporabimo lahko tudi \emph{podpis} ali
\emph{nadpis}, ki je lahko črka, številka, ali kak drug simbol, na primer
%
\begin{equation*}
  a_i \quad
  a^i \quad
  a_1 \quad
  a_{\star} \quad
  a^{\dagger}.
\end{equation*}
%
Podpisu in nadpisu pogovorno pravimo tudi \emph{indeks} in \emph{eksponent}, a to ni
najbolj posrečena raba, ker se indeks lahko pojavi tudi v nadpisu ali kje drugje,
eksponent pa lahko pomeni tudi število, s katerim potenciramo.

Kljub temu obilju črk in oznak posežemo še po drugih abecedah, še posebej grški, zato se
jo čimprej naučite! Grške črke skupaj z njihovo izgovorjavo najdete v
tabeli~\ref{tabela:grska-abeceda}. Prostoročni zapis grških črk se boste naučili v
razredu.
%
Pa tudi to matematikom še ni dovolj! V teoriji množic uporabljamo še hebrejske črke
alef~$\aleph$, bet~$\beth$ in gimel~$\gimel$.

\begin{table}[ht]
\begin{center}
\begin{tabular}{cc|cc}
\multicolumn{2}{c|}{\textbf{Grška črka}} & \multicolumn{2}{c}{\textbf{Izgovorjava}} \\
\textit{velika} & \textit{mala} & \textit{v slovenščini} & \textit{v grščini} \\
\hline
A & $\alpha$ & alfa & alfa \\
B & $\beta$ & beta & vita \\
$\Gamma$ & $\gamma$ & gama & {\textgamma}ama \\
$\Delta$ & $\delta$ & delta & delta \\
E & $\epsilon$, $\varepsilon$ & epsilon & epsilon \\
Z & $\zeta$ & zeta & zita \\
H & $\eta$ & eta & ita \\
$\Theta$ & $\theta$, $\vartheta$ & {\scriptsize\textTheta}eta & {\scriptsize\textTheta}ita \\
I & $\iota$ & jota & jota \\
K & $\kappa$ & kapa & kapa \\
$\Lambda$ & $\lambda$ & lambda & lamda \\
M & $\mu$ & mi & mi \\
N & $\nu$ & ni & ni \\
$\Xi$ & $\xi$ & ksi & ksi \\
O & $\omicron$ & omikron & omikron \\
$\Pi$ & $\pi$, $\varpi$ & pi & pi \\
P & $\rho$, $\varrho$ & ro & ro \\
$\Sigma$ & $\sigma$, $\varsigma$ & sigma & si{\textgamma}ma \\
T & $\tau$ & ta\hill{u} & taf \\
$\Upsilon$ & $\upsilon$ & ipsilon & ipsilon \\
$\Phi$ & $\phi$, $\varphi$ & fi & fi \\
X & $\chi$ & hi & {\textchi}i \\
$\Psi$ & $\psi$ & psi & psi \\
$\Omega$ & $\omega$ & omega & ome{\textgamma}a \\
\end{tabular}
\end{center}
\par\medskip
\footnotesize{
Izgovorjava: \hill{u} je ustnični u (kot v besedi `pav');
{\textgamma} je cerkljanski `g' (nekaj med `g' in `h' --- vprašajte sošolce s tega območja);
{\scriptsize\textTheta} je angleški nezveneči `th' (kot v besedi `thing');
{\textchi} je nemški `ch' (kot v besedi `ich').}
\caption{Grška abeceda.}
\label{tabela:grska-abeceda}
\end{table}

In zakaj pravzaprav potrebujemo tako veliko število črk? Verjetno zato, ker je v
matematiki krajši zapis bolj učinkovit, saj zasede manj prostora na papirju, pa še hitreje
ga zapišemo in preberemo. Računalničarji imajo drugačne navade, saj pri njih velja, da naj
se uporablja opisna imena, ki razkrijejo pomen: kjer bi matematik in fizik uporabila~$m$
in~$a$, bi računalničar zapisal $\mathtt{masa\_delca}$ in $\mathtt{pospesek}$.

\section{Izrazi}
\label{sec:irazi}

Matematično besedilo je mešanica naravnega jezika in simbolnega zapisa. Delom besedila, ki
so napisani s simboli, pravimo \emph{simbolni izrazi} ali krajše kar \emph{izrazi}. Vsi
ste jih že videli, denimo
%
\begin{equation*}
  (3 + 4) \cdot 6 \qquad\quad
  \int_0^1 \frac{x}{1 + x^2} \, dx \qquad\quad
  a x^2 + b x + c = 0 \qquad\quad
  x > 0 \lor x \leq 0
\end{equation*}
%
Ste se kdaj vprašali, zakaj pravzaprav pišemo ulomke z vodoravno črto, integral z znakom
$\int$, zakaj ima množenje prednost pred seštevanjem in zakaj seštevamo od leve proti
desni, čeprav bi lahko tudi v drugi smeri? Odgovor je vedno isti: to so splošno sprejete
navade, ki so se izoblikovale v razvoju matematike. To niso matematične resnice, ampak
\emph{dogovori} med ljudmi, ki se jih držimo zato, ker so se izkazali za smiselne. Na
primer, integralski znak $\int$ je Leibniz\footnote{Gottfried Wilhelm von Leibniz
  (1646--1716) je bil nemški filozof, matematik, fizik, pravnik, zgodovinar, jezikoslovec,
  knjižničar in diplomat lužiško sorbskega porekla.} izpeljal iz črke S, ker je na
integral gledal kot na določene vrste vsoto (latinsko `summa').

Z vidika vsebine raznolikost matematičnega zapisa ni potrebna, saj bi lahko vse izraze
pisali na isti način. Namesto simbolov, kot so $+$, $-$ in $\sqrt{\ }$, bi lahko
uporabljali besede $\mathtt{plus}$, $\mathtt{minus}$, $\mathtt{sqrt}$ in jih zapisovali
kot preslikave. Tak zapis je preprost in enoten, saj se nam ni treba ukvarjati s
predponami, medponami in priponami ter z levim in desnim združevanjem. Uporablja se v
računalništvu, a kdo bi želel na tablo namesto $3 + \sqrt{5 - 4}$ zapisati
$\mathtt{plus(3, \mathtt{sqrt}(\mathtt{minus}(5, 4)))}$?

Ni vsako zaporedje znakov pravilen izraz. Denimo, $3 + ) x \cdot 4$ ni pravilen izraz, ker ima narobe postavljen zaklepaj.
%
Izraz je \df{pravilno formiran} ali \df{sintaktično pravilen}, če ustreza pravilom, ki določajo kako postavljamo oklepaje, vejice, pike, kako uporabljamo razne posebne simbole ($+$, $\lor$, $\int$) itd.
%
Natančna \df{sintaktična pravila} za pisanje matematičnih izrazov so precej zapletena, ker je matematični zapis raznovrsten
in se je razvijal skozi zgodovino. Mnoga že poznate (">vsak oklepaj mora imeti ustrezni zaklepaj"<, ">piše se $a + b$ in ne $a b +$"<), zato jih ne bomo podrobno obravnavali -- to je delo za računalničarje, ki želijo taka pravila implementirati. Posvetimo se raje pravilom in dogovorom za zapis izrazov, ki jih pogosto srečamo v matematiki.

\subsection{Predpone, medpone, pripone, nadnapisi in podnapisi}

Aritmetične operacije $+$, $-$, $\cdot$ in $/$ pišemo kot \df{medpone} ali \df{infiksne operacije}, tako da operacija
stoji med obema operandoma, na primer $x + y$. Kadar zapišemo operator za operand,
pravimo, da je \df{pripona} ali \df{postfiksna operacija}, na primer faktoriela~$x!$. Zapis operatorja je
\df{predpona} ali \df{prefiksna operacija}, če stoji pred operandom, na primer nasprotna vrednost~$-x$. Poleg teh
poznamo tudi druge zapise: potenciranje pišemo z eksponentom $x^y$, deljenje z ulomkom
$\frac{x}{y}$, kvadratni koren s posebnim simbolom $\sqrt{x}$ itn. Skrajni primer je zapis
množenja brez simbola, ko namesto $x \cdot y$ zapišemo kar $x y$.

Argumente operacije ali funkcije včasih zapišemo v \df{podnapis} ali \df{nadnapis}. Na primer, če je $a : \NN \to \RR$ preslikava, pogosto pišemo $a_i$ namesto $a(i)$.

\subsection{Prednost in združevanje}

Nekatere operacije imajo \df{prednost} ali \df{prioriteto} pred drugimi in nekatere \df{združujejo} ali \df{asociirajo} levo ali desno. Prednost pove, katera operacija pride prej na vrsto, kadar ni oklepajev:
potenciranje ima prednost pred množenjem in množenje pred seštevanjem. Operacija lahko
tudi združuje levo ali desno. Na primer, seštevanje $+$ združuje levo, zato je $5 + 2 + 1$
enako $(5 + 2) + 1$. Pri seštevanju to sicer ni pomembno, pri odštevanju pa moramo
upoštevati združevanje na levo: $5 - 2 - 1$ je enako $(5 - 2) - 1$ in ne $5 - (2 - 1)$.
Potenciranje združuje na desno, saj $2^{3^4}$ pomeni $2^{(3^4)}$. Nekatere operacije ne
združujejo in v takih primerih moramo uporabiti oklepaje.

Povejmo še to: nič ni narobe, če zapišemo več oklepajev, kot je to nujno potrebno. Izraza $3 \cdot 4 + 5$ in $(((3) \cdot 5) + 5)$ sta enakovredna.

\subsection{Implicitni argumenti, privzete vrednosti in preobteževanje}
\label{sec:implicitni-argumenti}

Argumente operacije lahko opustimo in od bralca pričakujemo, da bo pravilno uganil, kaj smo mislili. Pravimo, da so to
\df{implicitni argumenti}. Primer implicitnih argumentov smo že videli, ko smo zapisali prvo in drugo projekcijo $\fst$ in
$\snd$:
\begin{align*}
  \fst &: A \times B \to A, \\
  \snd &: A \times B \to A.
\end{align*}
%
Če bi bili zelo natančni, bi morali pri projekcijah zapisati tudi množici $A$ in $B$, ki tvorita kartezični produkt, na
primer nekaj takega kot $\fst[A][B] : A \times B \to A$.
%
Ko torej vpeljemo novo zapis, lahko nekatere argumente razglasimo za \df{implicitne}, kar pomeni, da jih bomo opuščali,
kadar to ne pripelje do zmede.

\begin{vaja}
  Ali ima kompozicija preslikav $\circ$ implicitne argumente? Katere?
\end{vaja}

Argument operacije ima lahko \df{privzeto vrednost}. Na primer logaritem $x$ z osnovo $b$ zapišemo $\log_b x$. Če opustimo~$b$, se razume, da je mišljen desetiški logaritem, $\log x = \log_10 x$. Pravimo, da je privzeta vrednost osnove $b = 10$.

Simbol lahko tudi \df{preobtežimo}, da ima več pomenov, nato pa od bralca pričakujemo, da bo uganil, katerega smo
mislili. Na primer, $+$ uporabljamo za
%
seštevanje naravnih števil,
seštevanje celih števil,
seštevanje racionalnih števil,
seštevanje realnih števil,
seštevanje kompleksnih števil,
seštevanje vektorjev,
seštevanje matrik,
itd.
%
S preobteževanjem ne gre pretiravati, ker lahko pripelje do zmede. Običajno z istim simbolom označimo različne operacije, ki imajo kaj skupnega. Na primer, $+$ vedno uporabljamo le za operacijo, ki je komutativna, asociativna in ima nevtralni element.

\subsection{Izrazi predstavljajo drevesa}

Izrazi so zaporedja znakov, ki jih pišemo z leve na desno. A kje drugje na tem svetu bi jih pisali z desne na levo ali
navpično. Izrazi so le \emph{predstavitve} tako imenovanih \df{sintaktičnih dreves}. Na primer $((3 + x) \times y)^2$ predstavlja sintaktično drevo, pri čemer potenciranje predstavimo z znakom ${}^{\wedge}$:
%
\begin{center}
  \begin{tikzpicture}[level/.style={sibling distance=5em/#1},,level distance=2em,
    every node/.style = {align=center}, baseline=(current bounding box.center)
    ]
    \node {${}^{\wedge}$}
    child { node {$\times$}
      child { node {$+$}
        child { node {$3$} }
        child { node {$x$} }
      }
      child { node {$y$} }
    }
    child { node {$2$} } ;
  \end{tikzpicture}
\end{center}
%
O sintaktičnih drevesih ne bomo govorili, a jih omenimo, ker so pomembna iz dveh razlogov: sintaktična drevesa so
\emph{podatkovni tip}, s katerim v programu dejansko obdelujemo izraze; s pomočjo sintaktičnih dreves lahko simbolni zapis predstavimo kot posebno vrsto algebre, ki omogoča matematično obravnavo izrazov.


\section{Slike in diagrami}
\label{sec:slike-in-diagrami}

Matematiki uporabljamo tudi diagrame in slike, slednje predvsem v geometriji in analizi. Z
njimi lahko razjasnimo pojme in si pomagamo pri predstavi zapletenih pojmov in
konstrukcij, zato so nepogrešljivo orodje. To še posebej velja za poučevanje matematike.

Vendar pa moramo biti pri uporabi slik pazljivi, ker nas lahko zavedejo. V poduk podajmo ">dokaz"<, da so vsi trikotniki enakokraki!

\begin{izrek}
  Vsi trikotniki so enakokraki.
\end{izrek}

\begin{dokaz}
  Naj bo $\triangle ABC$ poljuben trikotnik, glej sliko~\ref{fig:trikotnik}.
  %
  \begin{figure}[ht]
    \centering
\tikzset{
    right angle quadrant/.code={
        \pgfmathsetmacro\quadranta{{1,1,-1,-1}[#1-1]}     % Arrays for selecting quadrant
        \pgfmathsetmacro\quadrantb{{1,-1,-1,1}[#1-1]}},
    right angle quadrant=1, % Make sure it is set, even if not called explicitly
    right angle length/.code={\def\rightanglelength{#1}},   % Length of symbol
    right angle length=4ex, % Make sure it is set...
    right angle symbol/.style n args={3}{
        insert path={
            let \p0 = ($(#1)!(#3)!(#2)$) in     % Intersection
                let \p1 = ($(\p0)!\quadranta*\rightanglelength!(#3)$), % Point on base line
                \p2 = ($(\p0)!\quadrantb*\rightanglelength!(#2)$) in % Point on perpendicular line
                let \p3 = ($(\p1)+(\p2)-(\p0)$) in  % Corner point of symbol
            (\p1) -- (\p3) -- (\p2)
        }
    }
}

\begin{center}
\begin{tikzpicture}[scale=0.5]

% Triangle vertices
\coordinate (A) at (3,10);
\coordinate (B) at (0,0);
\coordinate (C) at (8,2);
\coordinate (P) at (4,1);
\coordinate (Q) at (7/2,3);
\coordinate (R) at (243/218, 405/109);
\coordinate (S) at (1119/178, 422/89);

% Draw the perpendiculars from midpoints to opposite sides
\draw[thick] (S) -- (Q);
\draw[thick] (R) -- (Q);
\draw[thick] (P) -- (Q);
\draw[thick] (B) -- (Q);
\draw[thick] (C) -- (Q);
\draw[thick] (A) -- (Q);

% Draw triangle
\draw[very thick] (A) -- (B) -- (C) -- cycle;

% Perpendicular marks
\draw [right angle symbol={B}{A}{Q}];
\draw [right angle symbol={C}{A}{Q}];
\draw [right angle symbol={B}{C}{Q}];

\tkzMarkAngle[size=2,mark=|](B,A,Q)
\tkzMarkAngle[size=2,mark=|](Q,A,S)

% Points
\fill (A) circle (5pt) node[above] {A};
\fill (B) circle (5pt) node[below left] {B};
\fill (C) circle (5pt) node[below right] {C};
\fill (P) circle (5pt) node[below] {P};
\fill (S) circle (5pt) node[above right] {S};
\fill (R) circle (5pt) node[above left] {R};
\fill (Q) circle (5pt) node[above right] {Q};
\end{tikzpicture}
\end{center}
    \caption{trikotnik $\triangle ABC$}
    \label{fig:trikotnik}
  \end{figure}
  %
  Naj bo $P$ središče stranice $BC$ ter $Q$ presečišče simetrale kota $\angle BAC$ in simetrale stranice $BC$.
  Naj bo~$R$ pravokotna projekcija točke~$Q$ na stranico~$AB$ in~$S$ pravokotna projekcija točke~$Q$ na stranico~$AC$.
  %
  Trikotnik~$\triangle BCQ$ je enakokrak z vrhom~$Q$, zato velja $BQ \cong CQ$.
  %
  Trikotnika $\triangle AQR$ in~$\triangle AQS$ sta podobna, saj skladna, ker imata skupno stranico in kota ob njej, torej velja $AR \cong AS$ in $QR \cong QS$.
  %
  Sklepamo, da sta tudi trikotnika $\triangle BQR$ in $\triangle CQS$ skladna, saj sta pravokotna trikotnika s skladno kateto in skladno hipotenuzo. Potemtakem sta skladni še preostali kateti, $RB \cong SC$, od koder izračunamo
  %
  \begin{equation*}
    AB \cong AR + RB \cong AS + SC \cong AC.
  \end{equation*}
  %
  Trikotnik $\triangle ABC$ je res enakokrak.
\end{dokaz}

\newpage
\hbox{}
\vfill
\begin{center}
  \LARGE
  \textbf{Od tu naprej \\ zapiski niso več tako lepo urejeni \\ in vsebujejo več napak.}
\end{center}
\vfill
\hbox{}
\newpage

\section{Logične formule}

Izrazi, ki označujejo števila, se imenujejo \textbf{aritmetični izrazi}.

Izrazi, ki označujejo matematične izjave, se imenujejo \textbf{logični izrazi} ali \textbf{logične formule}. Razumevanje, branje in pisanje le-teh zahteva kar nekaj treninga, zato se mu bomo posvetili tu in na vajah. Pravzaprav ne bomo vadili le razumevanja zapisa, ampak tudi, kako matematiki razmišljajo in razumejo drug drugega.

Tu o dokazih in pravilih dokazovanja še ne bomo govorili, bomo pa pojasnili intuitivni pomen logičnih operacij.

Računanje z logičnimi formulami delimo na:
%
\begin{itemize}
\item \textbf{izjavni račun} zaobjema logične veznike $\neg $, $\land$, $\lor$, $\lthen$, $\liff$,
\item \textbf{predikatni račun} zaobjema izjavni račun, ter kvantifikatorja $\forall$ in $\exists$.
\end{itemize}


\subsection{Izjavni račun}

\textbf{Izjavni vezniki} so naslednje operacije:
%
\begin{itemize}

\item \textbf{resničnostni konstanti} $\bot$ in $\top$: beremo ju ">neresnica"> in ">resnica"<,

\item \textbf{negacija} $\neg$: izjavo $\neg A$ beremo ">$A$ ne velja"< ali ">ni res, da $A$"<,

\item \textbf{konjunkcija} $\land$: izjavo $A \land B$ beremo ">$A$ in $B$"<,

\item \textbf{disjunkcija} $\lor$: izjavo $A \lor B$ beremo ">$A$ ali $B$"<,

\item \textbf{implikacija} $\lthen$: izjavo $A \lthen B$ lahko beremo na več načinov:
  %
  \begin{itemize}
  \item ">Iz $A$ sledi $B$."<
  \item ">Če $A$, potem $B$."<
  \item ">$A$ samo če $B$."<
  \item ">$B$ sledi iz $A$."<
  \item ">$A$ je zadosten pogoj za $B$."<
  \item ">$B$ je potreben pogoj za $A$."<
  \end{itemize}
  %
\item \textbf{ekvivalenca} $\liff$: izjavo $A \liff B$ beremo
  %
  \begin{itemize}
  \item ">$A$ je ekvivalentno $B$."<
  \item ">$A$, če in samo če $B$."<
  \item ">$A$ natanko tedaj, ko $B$."<
  \item ">$A$ je zadosten in potreben pogoj za $B$."<
  \end{itemize}
\end{itemize}
%
Malo bolj neobičajna je:
%
\begin{itemize}
\item \textbf{ekskluzivna disjunkcija} $\oplus$: izjava $A \oplus B$ beremo ">bodisi $A$ bodisi $B$"< ali ">$A$ ali $B$, vendar ne oba hkrati"<.
\end{itemize}
%
Prioriteta veznikov, od najvišje do najnižje:
%
\begin{itemize}
\item $\neg$,
\item $\land$,
\item $\lor$, $\oplus$,
\item $\lthen$, $\liff$.
\end{itemize}

\begin{zgled}
  Izraz $\neg A \land B \lthen C \lor D$ beremo kot $((\neg  A) \land B) \lthen (C \lor D)$.
\end{zgled}

Asociiranost veznikov:
%
\begin{itemize}
\item leva asociiranost: $\land$, $\lor$, $\oplus$,
\item desna asociiranost: $\lthen$.
\end{itemize}
%
Ekvivalenca $\liff$ nima asociiranosti, zato je zapis $A \liff B \liff C$ načeloma dvoumen, a v praksi pomeni $(A \liff B) \land (B \liff C)$.

\begin{opomba}
  Tudi zapis $x = y = z$ pravzaprav ni smiseln, saj sta $(x = y) = z$ in $x = (y = z)$ oba nesmiselna. V praksi $x = y = z$ pomeni $(x = y) \land (y = z)$. Pa še to: koliko enačb je izraženih z $a = b = c = d$? Tri! Toliko kot je enačajev.
\end{opomba}

\begin{opomba}
  Zapis $x \neq y \neq z$ je nejasen in se mu je bolje izogibati, saj zlahka pripelje do pomote, ker iz $x
  \neq y$ in $y \neq z$ ne sledi nujno $x \neq z$.
\end{opomba}

Glede razumevanja veznikov, omenimo:
%
\begin{itemize}
\item disjunkcija je \emph{inkluzivna}, kar pomeni, da je $A \lor B$ resnična izjava, če sta $A$ in $B$ resnični,
\item v implikaciji $A \lthen B$ se $A$ imenuje \textbf{antecedent} in $B$ \textbf{konsekvent}. Implikacija je veljavna, če je antecedent neveljaven,
\item ekvivalenco $A \liff B$ lahko razumemo kot okrajšavo za $(A \lthen B) \land (B \lthen A)$.
\end{itemize}

\subsection{Kvantifikatorja}

Matematične izjave vsebujejo fraze, kot so ">za vse"<, ">za neki"<, ">obstaja vsaj en"<, ">za natanko enega"< ipd. Le-te izrazimo s \textbf{kvantifikatorji}. Osnovna kvantifikatorja sta \textbf{univerzalni} in \textbf{eksistenčni}.


\subsubsection{Univerzalni kvantifikator $\forall$}

Formulo $\all {x \in A} \phi$ beremo:
%
\begin{itemize}
\item ">Za vsak $x$ iz $A$ velja $\phi$."<,
\item ">Vsi $x$ iz $A$ zadoščajo $\phi$."<,
\item ">$\phi$ za vse $x$ iz $A$."<
\end{itemize}
%
Pika pri tem nima nobenega posebnega pomena, pogosti so tudi zapisi
%
\begin{equation*}
  \forall x \in A \,,\, \phi
  \qquad\text{ali}\qquad
  \forall x : A \,,\, \phi
  \qquad\text{ali}\qquad
  (\forall x : A) \phi.
\end{equation*}
%
Nekateri matematiki pišejo po principu ">piši kao što govoriš"<
%
\begin{equation*}
  \phi, \forall x \in A
  \qquad\qquad \text{(">$\phi$ za vse $x$ iz $A$"<)}
\end{equation*}
%
Ta zapis odsvetujemo, ker ne deluje, ko kombiniramo več kvantifikatorjev hkrati.

Omenili smo že, da $\all{x \in \emptyset} \phi$ vedno velja. To bomo utemeljili v poglavju o pravilih sklepanja.

\subsubsection{Eksistenčni kvantifikator $\exists$}

Formulo $\some{x \in A} \phi$ beremo:
%
\begin{itemize}
\item ">Obstaja $x$ iz $A$ velja $\phi$."<
\item ">Obstaja vsaj en $x$ iz $A$ velja $\phi$."<
\item ">Za neki $x$ iz $A$ velja $\phi$."<
\item ">$\phi$ za neki $x$ iz $A$."<
\end{itemize}
%
S tem povemo, da obstaja \emph{eden ali več} takih~$x$. Na primer, izjava $\some{x \in \NN} x
< 3$ je veljavna, saj je $2$ naravno število, ki je manjša od $3$.


\subsubsection{Prioriteta $\forall$ in $\exists$}

Prioriteta kvantifikatorjev $\forall$ in $\exists$ je nižja od prioritete veznikov. Na primer:
%
\begin{itemize}
\item $\all{x \in A} \phi \land \psi$ je enako $\all{x \in A} (\phi \land \psi)$,
\item $\all{x \in \RR} x > 0 \lthen \phi$ je enako $\all{x \in \RR} (x > 0 \lthen \phi)$.
\end{itemize}
% 
Kvantifikator vedno zaobjame vse, kar zmore:
%
\begin{itemize}
\item $\all{x \in A} \phi \land \exists{y \in B} \psi$ je enako $\all {x \in A} (\phi \land (\exists y \in B . \psi))$ in \emph{ni} enako $(\all{x \in A} \phi) \land (\exists {y \in B} \psi)$,

\item $(P \land \all{x \in A} Q \lthen R) \lthen \some{y \in B} S$ je enako $(P \land \all{x \in A}  (Q \lthen R)) \lthen (\some{y \in B} S)$ in \emph{ni} enako 
  $(P \land (\all{x \in A} Q) \lthen R) \lthen (\some{y \in B} S)$
\end{itemize}

\subsubsection{Kombinacija $\forall$ in $\exists$}

Pozor, vrstnega reda kvantifikatorjev ne smemo mešati:
%
\begin{itemize}
\item $\all{x \in \RR} \some{y \in \RR} x < y$ pomeni ">vsako realno število je manjše od nekega realnega števila"< (kar je res),
\item $\some{x \in \RR} \all{y \in \RR} x < y$ pomeni ">obstaja najmanjše realno število"< (kar ni res).
\end{itemize}
%
To dejstvo bomo utrjevali na vajah. Zapomnite se, da morate biti tudi pri ostalih predmetih posebej pozorni na vrstni red
">za vsak"< in ">obstaja"<. Je profesorica pri analizi rekla ">za vsak $\epsilon > 0$ obstaja tak $\delta > 0$ da \dots"< ali je rekla ">obstaja tak $\delta > 0$ da za vsak $\epsilon > 0$ \dots"<? Če boste zamešali ti dve izjavi na ustnem izpitu iz analize, boste imeli pokvarjen dan, ali pa cele počitnice!


\subsubsection{Kvantifikator z dodatnim pogojem}

Pogosto kvantifikacijo kombiniramo z dodatnim pogojem, na primer:
%
\begin{itemize}
\item ">Obstaja \emph{liho} naravno število, ki ni deljivo s 7."<
\item ">Vsako \emph{sodo} naravno število je deljivo s 3."<
\end{itemize}
%
V prvem primeru je dodatni pogoj izražen z besedico ">liho"< in v drugem s ">sodo"<. Kako zapišemo take izjave s formulo, kam
vtaknemo dodatni pogoj? Izjavi pretvorimo po korakih:
%
\begin{itemize}
\item ">Obstaja liho naravno število, ki ni deljivo s 7."<
\item ">Obstaja naravno število, ki je liho in ki ni deljivo s 7."<
\item ">Obstaja naravno število, ki je liho in deljivo s 7."<
\item ">Obstaja $x$ iz $\NN$, da je $x$ lih in $x$ je deljiva s 7."<
\item $\some{x \in \NN} (\text{$x$ je lih}) \land (\text{$x$ je deljiv s 7})$
\item $\some{x \in \NN} (\some{y \in \NN} x = 2 y + 1) \land (\some{z \in \NN} y = 7 z)$
\end{itemize}
%
In še druga izjava:
\begin{itemize}
\item ">Vsako sodo naravno število je deljivo s 3."<
\item ">Vsako naravno število, ki je sodo, je deljivo s 3."<
\item ">Za vsako naravno število velja, da če je sodo, potem je deljivo s 3."<
\item ">Za vsak $x$ iz $\NN$ velja, če je $x$ sod, potem je $x$ deljiv s $3$."<
\item $\forall{x \in \NN} \text{$x$ sod} \lthen \text{$x$ deljiv s 3}$
\item $\forall{x \in \NN} (\some{y \in \NN} x = 2 y) \lthen (\some{z \in \NN} x = 3 z)$
\end{itemize}
%
Zapomnimo si: \textbf{dodatni pogoj pri $\exists$ izrazimo $\land$} in \textbf{dodatni pogoj pri $\forall$ izrazimo $\lthen$}.

Poglejmo še en primer, ko imamo več možnosti za zapis s formulo:
%
\begin{quote}
  ">Za vsako pozitivno realno število $x$ obstaja tako naravno število $n$, da je $x < n$."<
\end{quote}
%
Začetni del ">za vsako pozitivno realno število"< lahko zapišemo na več načinov:
%
\begin{itemize}
\item $\all{x \in \RR_{> 0}} \some{n \in \NN} x < n$,
\item $\all{x \in \set{y \in \RR \such y > 0}} \some{n \in \NN} x < n$,
\item $\all{x \in \RR} x > 0 \lthen \some{n \in \NN} x < n$,
\item $\all{x > 0} \some{n \in \NN} x < n$.
\end{itemize}
%
Pri prvem načinu moramo biti v naprej dogovorjeni, da $\RR_{> 0}$ označuje množico pozitivnih realnih števil.
Pri drugem načinu smo vstavili definicijo $\RR_{> 0}$, zato dogovor ni več potreben, a je zapis bolj nečitljiv.
Pri tretjem načinu smo predstavili pozitivnost kot dodatni pogoj.
Četrti način je najbolj čitljiv in se pogosto uporablja, a nam ne pove, ali je $x$ realno, celo, ali racionalno število.


\subsubsection{Vezane in proste spremenljivke}

V nekaterih izrazih nastopajo spremenljivke, ki so \textbf{vezane}. To pomeni, da je njihovo območje veljavnosti omejeno,
oziroma da so neke vrste ">lokalne spremenljivke"<. Spremenljivka, ki ni vezana, je \textbf{prosta}. Primeri:
%
\begin{itemize}
\item V funkcijskem predpisu $x \mapsto x^2 + y$ je $x$ vezan in $y$ prost.
\item V funkcijskem predpisu $(x,y) \mapsto x^2 + y$ sta $x$ in $y$ vezana.
\item V integralu $\int (x + a)^2 d x$ je $x$ vezan in $a$ prost.
\item V vsoti $\sum_{i=0}^n (i^2 + 1)$ je $i$ vezan in $n$ prost.
\item V formuli $\all{x \in \RR} x^3 + 3 x < 7$ je $x$ vezana spremenljivka.
\end{itemize}

Če vezano spremenljivko preimenujemo, se izraz ne spremeni. Funkcijska predpisa $x \mapsto a \cdot x^2 + 1$ in $y \mapsto y \cdot y^2 + 1$ sta \emph{enaka}. Vendar pozor, če vezano spremenljivko preimenujemo, za novo ime \emph{ne} smemo izbrati spremenljivke, ki se že pojavlja. Na primer, v integralu
%
\begin{equation*}
  \int_0^1 (a + x)^2 \, d x
\end{equation*}
%
smemo $x$ preimenovati v $t$, zato sta integrala enaka izraza (in imata tudi enako vrednost):
%
\begin{equation*}
  \int_0^1 (a + x)^2 \, d x  = \int_0^1 (a + t)^2 \, d t
\end{equation*}
%
Ne bi pa smeli $x$ preimenovati v $a$, saj bi dobili
%
\begin{equation*}
  \int (a + a)^2 \, d a
\end{equation*}
%
Pravimo, da se je prosta spremenljivka $a$ \emph{ujela} v integral.

