\chapter{Relacije}

\section{Predikati}

\textbf{Predikat} na množici $A$ opredeljuje kako lastnost elementov množice $A$. Če
je $P$ predikat na $A$ in $x \in A$, potem se je smiselno vprašati, ali $x$
zadošča predikatu $P$. Odgovor je resničnostna vrednost, ki jo označimo s $P(x)$.

\begin{primer}
  Na množici naravnih števil $\NN$ lahko obravnavamo predikat ">je sodo
  število"<. Tako na primer $4$ zadošča predikatu ">je sodo število"<, $7$ pa mu ne zadošča.
\end{primer}

Predikat $P$ na množici $A$ lahko predstavimo na dva načina:
%
\begin{itemize}
\item kot preslikavo $P : A \to \two$, ki slika $x \in A$ v resničnostno vrednost $P(x)$,
\item kot podmnožico $P \subseteq A$ tistih $x \in A$, za katere velja $P(x)$.
\end{itemize}
%
Oba načina predstavitve sta uporabna, spoznali pa smo že izomorfizem med njima,
saj velja $\pow{A} \iso \two^A$.

\section{Relacije}

Relacije z večmestnimi predikati. Se pravi, relacija $R$ opredeljujejo kako
lastnost urejenih večteric kartezičnega produkta $A_1 \times A_2 \times \cdots \times A_n$. Pravimo, da je $R$ \textbf{$n$-člena} ali \textbf{$n$-mestna relacija} na množicah $A_1, \ldots, A_n$.

\begin{primer}
  Na množici točk v ravnini lahko obravnavamo relacijo kolinearnosti.
  To je trimestna relacija: točke $A$, $B$ in $C$ so kolinearne, kadar obstaja
  premica, ki vsebuje vse tri točke.
\end{primer}

Relacijo $R$ na množicah $A_1, \ldots, A_n$ lahko predstavimo na dva načina, podobno
kot predikate:
\begin{itemize}
\item kot preslikavo $R : A_1 \times A_2 \times \cdots \times A_n \to \two$,
\item kot podmnožico $R \subseteq A_1 \times A_2 \times \cdots \times A_n$.
\end{itemize}
%
Bolj običajna je predstavitev s podmnožicami, zato bomo dejstvo, da je $R$
relacija na množicah $A_1, \ldots, A_n$ zapisali kar kot $R \subseteq A_1 \times A_2 \times \cdots \times A_n$.
Za elemente $x_1 \in A_1, \ldots, x_n \in A_n$ dejstvo, da so v relaciji $R$ zapišemo
$R(x_1, \ldots, x_n)$, včasih pa tudi $(x_1, \ldots, x_n) \in R$.

Na množicah $A_1, \ldots, A_n$ lahko vedno definiramo:
%
\begin{itemize}
\item \textbf{prazno relacijo $\emptyset$}: nobeni elementi niso v prazni relaciji,
\item \textbf{univerzalno relacijo $A_1 \times A_2 \times \cdots \times A_n$}: vsi elementi so v univerzalni relaciji.
\end{itemize}
%
Univerzalna relacija se imenuje tudi \textbf{polna relacija}.

V praksi so najbolj pogoste \textbf{dvomestne relacije}, se pravi relacije na dveh
množicah, $R \subseteq A \times B$.
V tem primeru pravimo množici $A$ \textbf{domena} in $B$ \textbf{kodomena} relacije $R$, relaciji $R$ pa relacija med $A$ in $B$.

Pomembna relacija na množici $A$ je \textbf{enakost} ali \textbf{diagonala} na $A$:
%
\begin{equation*}
    \diag[A] \defeq \set{ (x, y) \in A \times A \such x = y }
\end{equation*}
%
Zakaj ji pravimo diagonala?

Izmed dvočlenih relacij so najbolj pogoste relacije, pri katerih se domena in
kodomena ujemata, torej $R \subseteq A \times A$. V tem primeru pravimo, da je $R$ \textbf{relacija na množici $A$}.

Denimo, da je $R \subseteq A \times B$ relacija, $x \in A$ in $y \in B$. Dejstvo, da sta $x$ in $y$ v relaciji $R$ zapišemo na enega od načinov
%
\begin{equation*}
  (x, y) \in R
  \qquad
  R(x, y)
  \qquad
  x \rel{R} y
\end{equation*}
%
Prvi zapis se uporablja, kadar je $R$ podana kot podmnožica $A \times B$, drugi kadar
podamo~$R$ z logično formulo. Tretji način je tudi pogost, še posebej kadar je
relacija označena s simbolom kot je $=$, $\neq$, $<$, $>$, $\sqsubseteq$, $\sim$ ipd.

Relacijo lahko predstavimo na več načinov, na primer z logično formulo, z resničnostno tabelo, ali z usmerjenim grafom.
%
Z grafom predstavimo $R \subseteq A \times A$ tako, da za vozlišča grafa vzamemo
elemente množice $A$, nato pa narišemo puščico od $x$ do $y$, kadar velja $x \rel{R} y$.


\section{Osnovne lastnosti relacij}

Relacije, ki so pomembne v matematični praksi imajo pogosto lastnosti, ki jih poimenujemo. Za relacijo $R \subseteq A \times A$ pravimo da je:
%
\begin{itemize}
  \item \textbf{refleksivna:} $\all{x \in A} x \rel{R} x$,
  \item \textbf{simetrična:} $\all{x, y \in A} x \rel{R} y \lthen y \rel{R} x$,
  \item \textbf{antisimetrična:} $\all{x, y \in A} x \rel{R} y \land y \rel{R} x \lthen x = y$,
  \item \textbf{tranzitivna:} $\all{x, y, z \in A} x \rel{R} y \land y \rel{R} z \lthen x \rel{R} z$,
  \item \textbf{irefleksivna:} $\all{x \in A} \lnot (x \rel{R} x)$,
  \item \textbf{asimetrična:} $\all{x, y \in A} x \rel{R} y \lthen \lnot (y \rel{R} x)$,
  \item \textbf{sovisna:} $\all{x, y \in A} x \neq y \lthen x \rel{R} y \lor y \rel{R} x$,
  \item \textbf{strogo sovisna:} $\all{x, y \in A} x \rel{R} y \lor y \rel{R} x$.
\end{itemize}
%

\begin{naloga}
  Kako iz usmerjenega grafa relacije razberemo refleksivnost in simetričnost? Kaj pa ostale lastnosti?
\end{naloga}

\section{Operacije na relacijah}

\subsection{Unija, presek in komplement relacij}

Ker so relacije pravzaprav podmnožice, lahko na njih uporabljamo operacije unija $\cup$,
presek $\cap$ in komplement $\compl{\Box}$. Denimo, da sta $R, S \subseteq A \times B$ relaciji. Tedaj velja:
%
\begin{align*}
  x \rel{(R \cup S)} y &\iff x \rel{R} y \lor x \rel{S} y, \\
  x \rel{(R \cap S)} y &\iff x \rel{R} y \land x \rel{S} y, \\
  x \compl{R} y &\iff \lnot (x \rel{R} y).
\end{align*}

\begin{primer}
  Za relacije enakosti in urejenost na realnih številih velja:
  %
  \begin{itemize}
  \item Komplement relacije enakosti $=$ je relacija neenakosti $\neq$.
  \item Unija relacij $<$ in $>$ na realnih številih je relacija $\neq$.
  \item Presek relacij $\leq$ in $\geq$ na realnih številih je relacija $=$.
  \end{itemize}
\end{primer}


\subsection{Transponirana relacija}

Dvojiške relacije lahko tudi \textbf{transponiramo}. Transponiranka relacije $R \subseteq A \times B$ je relacija $\transpose{R} \subseteq B \times A$, definirana s predpisom
%
\begin{equation*}
    y \transpose{R} x \defiff x \rel{R} y
\end{equation*}
%
ali ekvivalentno
%
\begin{equation*}
  \transpose{R} \defeq \set{ (y, x) \in B \times A \such x \rel{R} y }.
\end{equation*}
%
Očitno velja $\transpose{(\transpose{R})} = R$, torej je transponiranje \emph{involucija}.

\begin{primer}
  Transpozicija relacije $<$ na realnih številih $\RR$ je relacija $>$ na $\RR$.
  Komplement relacije $<$ na $\RR$ je relacija $\geq$ na $\RR$.
\end{primer}

\subsection{Kompozitum relacij}

Nadalje definiramo \textbf{kompozitum} relacij $R \subseteq A \times B$ in $S \subseteq B \times C$ kot relacijo $S \circ R \subseteq A \times C$, s predpisom
%
\begin{equation*}
    x \rel{(S \circ R)} z \defiff \some{y \in B} x \rel{R} y \land y \rel{S} z
\end{equation*}
%a
ali ekvivalentno
%
\begin{equation*}
  S \circ R \defeq
  \set{ (x, z) \in A \times C \such \some{y \in B} (x,y) \in R \land (y,z) \in S }.
\end{equation*}
%
Se pravi, da sta $x \in A$ in $z \in C$ v relaciji $S \circ R$, če sta preko $S$ in $R$
povezana s kakim elementom $y \in B$.

\begin{primer}
  Kompozitum relacij ">$x$ je otrok od $y$"< in ">$z$ je mati od $y$"< je relacija
  ">$z$ je babica od $x$"<.
\end{primer}

\begin{izrek}
  Komponiranje relacij je asociativno in diagonala je enota.
\end{izrek}

\begin{naloga}
  Zgornji izrek zapiši bolj natančno, da bo razvidno, kaj so domene in kodomene relacij.
\end{naloga}

\begin{dokaz}
  Najprej dokažimo asociativnost kompozicije.
  %
  Naj bo $R \subseteq A \times B$, $S \subseteq B \times C$ in $T \subseteq C \times D$ ter $a \in A$ in $d \in D$. Tedaj velja
  %
  \begin{align}
    a \rel{(T \circ (S \circ R))} d &\iff  \notag \\
    \some{c \in C} a \rel{(S \circ R)} c \land c \rel{T} d &\iff \notag \\
    \some{c \in C} (\some{b \in B} a \rel{R} b \land b \rel{S} c) \land c \rel{T} d \label{eq:comp-1}
  \end{align}
  %
  in
  %
  \begin{align}
    a \rel{((T \circ S) \circ R)} d &\iff \notag \\
    \some{b \in B} a \rel{R} b \land b \rel{(T \circ S)} d &\iff \notag \\
    \some{b \in B} a \rel{R} b \land (\some{c \in C} b \rel{S} c \land c \rel{T} d) \label{eq:comp-2}
  \end{align}
  %
  Torej je treba dokazati ekvivalenco izjav~\eqref{eq:comp-1} in~\eqref{eq:comp-2}, kar prepuščamo za vajo. Naj namignemo, da je treba pri dokazovanju ekvivalence uporabiti \emph{Frobeniusevo pravilo}
  %
  \begin{equation*}
    (\some{x \in X} p \land q(x)) \liff p \land \some{x \in X} q(x).
  \end{equation*}
  %
  V pravilu je $p$ formula, v kateri $x$ ne nastopa kot prosta spremenljivka.

  Dokažimo še, da je diagonala enota za kompozicijo: naj bo $R \subseteq A \times B$ ter $x \in A$ in $y \in B$. Tedaj velja
  %
  \begin{align*}
    x (\diag[B] \circ R) y &\iff \\
    \some{z \in B} x \rel{R} z \land z \diag[B] y  &\iff \\
    \some{z \in B} x \rel{R} z \land z = y &\iff \\
    x \rel{R} y
  \end{align*}
  %
  V zadnjem koraku smo uporabili ekvivalenco $(\some{u \in U} u = v \land P(v)) \liff P(v)$. Podobno dokažemo, da je diagonala desna enota.
\end{dokaz}

Kompozitum relacij ima torej podobne lastnosti kot kompozitum funkcij.

\subsection{Potenca relacije}

Za $n \in \NN$ definiramo \textbf{$n$-to potenco} relacije $R \subseteq A \times A$ kot relacijo $R^n \subseteq A \times A$ takole:
%
\begin{equation*}
    x R^n y \defiff
    \some{z_0, \ldots, z_n \in A}
    z_0 = x \land z_n = y \land \all{i \in \{0, \ldots, n-1\}} z_i \rel{R} z_{i+1}.
\end{equation*}
%
To je precej nečitljiva formula. Bolj razumljiva definicija je potenca kot $n$-kratni kompozitum relacije $R$ same s sabo:
%
\begin{equation*}
    R^n \defeq \underbrace{R \circ \cdots \circ R}_n
\end{equation*}
%
kjer se desni $R$ ponovi $n$-krat. Kaj dobimo, ko za $n$ vstavimo $0$? Enoto za kompozitum:
%
\begin{equation*}
    R^0 = \diag[A].
\end{equation*}

\section{Funkcijske relacije}

Funkcijo $f : A \to B$ smo definirali kot \emph{prirejanje} med elementi $A$ in $B$. A
kaj pravzaprav je ">prirejanje"<? Je to funkcijski predpis, program, kaj drugega?
Sedaj lahko povemo natančno: prirejanje, s katerim je podana funkcija, je
\emph{relacija} med elementi domene in kodomene.

\begin{definicija}
  Naj bo $f : A \to B$ funkcija. \textbf{Graf} funkcije $f$ je relacija
  $\Gamma_f \subseteq A \times B$, definirana s predpisom
  %
  \begin{equation*}
    x \,\Gamma_{\!f}\, y \liff f(x) = y
  \end{equation*}
  %
  ali ekvivalentno
  %
  \begin{equation*}
    \Gamma_{\!f} \defeq \set{ (x, y) \in A \times B \such f(x) = y }.
  \end{equation*}
\end{definicija}

Skratka, graf funkcije ni nič drugega kot njeno prirejanje.
%
Sedaj pa se vprašajmo: kakšnim pogojem mora zadoščati relacija $R \subseteq A \times B$, da je prirejanje za neko funkcijo? Odgovor poznamo: biti mora enolična in celovita.

\begin{definicija}
  Relacija $R \subseteq A \times B$ je \textbf{funkcijska relacija}, če je
  %
  \begin{itemize}
  \item \textbf{celovita:} $\all{x \in A} \some{y \in B} x \rel{R} y$ in
  \item \textbf{enolična:} $\all{x \in A} \all{y_1, y_2 \in B} x \rel{R} y_1 \land x \rel{R} y_2 \lthen y_1 = y_2$.
  \end{itemize}
  %
  Ekvivalentno oba pogoja skupaj zapišemo: $\all{x \in A} \exactlyone{y \in B} x \rel{R} y$.
\end{definicija}

Graf $\Gamma_{\!f} \subseteq A \times B$ funkcije $f : A \to B$ je vedno funkcijska relacija.
%
Funkcijska relacija $R \subseteq A \times B$ določa preslikavo $\phi_R : A \to B$ definirano s predpisom
%
\begin{equation*}
  \phi_R : x \mapsto \descr{y \in B} x \rel{R} y.
\end{equation*}
%
Če iz funkcije $f : A \to B$ tvorimo njen graf $\Gamma_{\!f}$, nato pa iz njega funkcijo
$\phi_{\Gamma_{\!f}} : A \to B$ dobimo nazaj prvotno funkcijo $f$. Obratno, če je $R$ funkcijska relacija, tedaj je $\Gamma_{\phi_R}$ enaka $R$. Torej imamo izomorfizem
%
\begin{equation*}
  B^A \iso \set{ R \in \pow{A \times B} \such \all{x \in A} \exactlyone{y \in B} x \rel{R} y }.
\end{equation*}
%

\begin{izjava}
  Kompozitum funkcij se ujema s kompozitumom relacij:
  $\Gamma_{g \circ f} = \Gamma_g \circ \Gamma_{\!f}$.
\end{izjava}

\begin{dokaz}
  Dokaz prepustimo za vajo, še prej pa morate izjavo zapisati bolj natančno: od
  kod in kam slikata preslikavi $f$ in $g$, kaj pomeni kompozitum na levi in kaj
  na desni?
\end{dokaz}


\section{Ovojnice relacij}

Pogosto imamo opravka z relacijo $R$, ki nima želene lastnosti (na primer ni
tranzitivna) mi pa želimo relacijo, ki to lastnost ima. Ali lahko $R$ kako
spremenimo, da bo imela želeno lastnost? Če to lahko naredimo na več načinov,
ali se eden od njih odlikuje?

\begin{definicija}
  Naj bo $R \subseteq A \times A$ relacija. Tedaj pravimo, da je relacija $T \subseteq A \times A$ \textbf{tranzitivna ovojnica} relacije $R$, če velja:
  %
  \begin{enumerate}
  \item $T$ je tranzitivna,
  \item $R \subseteq T$ in
  \item če je $S \subseteq A \times A$ tranzitivna in velja $R \subseteq S$, tedaj je $T \subseteq S$.
  \end{enumerate}
\end{definicija}

Povedano drugače: tranzitivna ovojnica relacije $R$ je \textsf{najmanjša} tranzitivna
relacija, ki vsebuje $R$. Zaenkrat ne vemo, ali ima vsaka relacija tranzitivno
ovojnico.

Izraz ">ovojnica"< uporabljamo, ker si lahko mislimo, da smo relacijo ovili
s tranzitivno relacijo tako, da se ji slednja čim bolj prilega. Namesto ">ovojnica"<
rečemo tudi \textbf{ogrinjača} ali \textbf{zaprtje}.

Poleg tranzitivne ovojnice lahko definiramo tudi druge ovojnice:
%
\begin{itemize}
  \item \textbf{Refleksivna ovojnica} relacije $R \subseteq A \times A$ je najmanjša refleksivna relacija, ki vsebuje $R$.
  \item \textbf{Simetrična ovojnica} relacije $R \subseteq A \times A$ je najmanjša simetrična relacija, ki vsebuje $R$.
  \item \textbf{Refleksivna tranzitivna ovojnica} relacije $R \subseteq A \times A$ je najmanjša refleksivna in tranzitivna relacija, ki vsebuje $R$.
\end{itemize}
%
Ali take ovojnice sploh obstajajo? Obravnavajmo le tranzitivne ovojnice, saj so
ostali dokazi zelo podobni. Ključno pri dokazu obstoja tranzitivne ovojnice je
naslednje dejstvo.

\begin{lema}
  Naj bo $A$ množica in $R : I \to \pow{A \times A}$ družina relacij na $A$. Če za
  vsak $i \in I$ velja, da je $R_i$ tranzitivna relacija, potem je tudi presek $\bigcap R$ tranzitivna relacija.
\end{lema}

\begin{dokaz}
  Iz definicije preseka družine množic (relacije so le posebne množice) sledi
  %
  \begin{equation*}
  x (\textstyle\bigcap R) y \liff \all{i \in I} x R_i y.
  \end{equation*}
  %
  Dokažimo, da je $\textstyle\bigcap R$ tranzitivna.
  Naj bodo $x, y, z \in A$ in denimo, da velja
  $x (\textstyle\bigcap R) y$ in $y (\textstyle\bigcap R) z$, kar je ekvivalentno
  %
  \begin{equation*}
    \all{i \in I} x R_i y
    \iinn
    \all{j \in I} y R_j z.
  \end{equation*}
  %
  Dokazati moramo $x (\textstyle\bigcap R) z$, kar je ekvivalentno
  %
  $\all{k \in I} x R_k z$.
  %
  Naj bo torej $k \in I$, dokazujemo $x R_k z$. Uporabimo $\all{i \in I} x R_i y$ pri $i = k$ in dobimo $x R_k y$.
  %
  Uporabimo $\all{j \in I} y R_j z$ pri $j = k$ in dobimo $y R_k z$.
  %
  Po predpostavki je $R_k$ tranzitivna relacija, torej velja $x R_k z$.
\end{dokaz}

\begin{izrek}
  Vsaka relacija ima enolično tranzitivno ovojnico.
\end{izrek}

\begin{dokaz}
  Najprej premislimo, da ima $R$ največ eno tranzitivno ovojnico: če sta
  $S$ in $T$ obe tranzitivni ovojnici $R$, potem iz definicije tranzitivne ovojnice
  sledi $S \subseteq T$ in $T \subseteq S$, torej velja $S = T$.

  Sedaj pokažimo, da $R$ ima tranzitivno ovojnico. Naj bo $R \subseteq A \times A$. Definirajmo množico relacij
  %
  \begin{equation*}
    D \defeq \set{ S \subseteq A \times A \such \text{$R \subseteq S$ in $S$ je tranzitivna} }.
  \end{equation*}
  %
  Trdimo, da je $\textstyle\bigcap D$ tranzitivna ovojnica relacije $R$.
  %
  Iz prejšnje leme sledi, da je $\textstyle\bigcap D$ tranzitivna.
  %
  Ker velja $R \subseteq S$ za vsak $S \in D$, seveda sledi $R \subseteq \textstyle\bigcap D$.
  %
  Če je $R \subseteq T$ in $T \subseteq A \times A$ tranzitivna relacija, tedaj velja $T \in D$, torej je $\bigcap D \subseteq T$.
\end{dokaz}


Po istem kopitu pokažemo, da ima vsaka relacija $R \subseteq A \times A$ tudi ostale
ovojnice. Je pa zgornji izrek neroden, ker nam dokaz ne poda uporabnega opisa
tranzitivne ovojnice. Povejmo, kako lahko razne ovojnice opišemo bolj
eksplicitno:
%
\begin{enumerate}
\item Refleksivna ovojnica relacije $R$ je relacija $R \cup \diag[A]$, se pravi, da
  relaciji $R$ dodamo še diagonalo.
\item 
  Simetrična ovojnica relacije $R$ je relacija $R \cup \transpose{R}$.
\item
  Tranzitivna ovojnica relacije $R$ je relacija $R^{+} \defeq \bigcup_{n \geq 1} R^n$, se pravi
  %
  \begin{equation*}
    R^{+} \defeq R \cup (R \circ R) \cup (R \circ R \circ R) \cup \cdots
  \end{equation*}
\item
  Refleksivna tranzitivna ovojnica relacije $R$ je relacija $R^{*} \defeq \bigcup_{n \geq 0} R^n$, se pravi
  %
  \begin{equation*}
  R^{*} \defeq \diag[A] \cup R \cup (R \circ R) \cup (R \circ R \circ R) \cup \cdots
  \end{equation*}
\end{enumerate}
