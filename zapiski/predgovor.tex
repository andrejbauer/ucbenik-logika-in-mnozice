\chapter*{Predgovor}
\addcontentsline{toc}{chapter}{Predgovor}

Pri predmetu Logika in množice v prvem letniku študija matematike na Fakulteti za matematiko in fiziko Univerze v Ljubljani se študenti učijo osnov matematičnega izražanja---kako beremo in pišemo matematično besedilo in simbolni zapis---hkrati pa spoznavajo temelje matematične logike in teorije množic.
%
Pred študenti matematike je torej težka naloga učiti se novo snov v novem žargonu.

Da bo učbenik v pomoč, bomo pri matematičnem izražanju bolj natančni, kot je to običajno za matematično besedilo.
Pojasnjevali bomo, kako matematiki pišejo, govorijo in razmišljajo v praksi ter marsikaj raje zapisali na dolgo, da bo začetniku bolj prijazno.
%
Bližnjice, ki jih ubirajo izkušeni matematiki, bomo vpeljali zlagoma, hkrati pa opozarjali na nedoslednosti, ki so
večinoma ostanki zgodovinskega razvoja matematike in ki se jim matematična tradicija stežka odreče.
%
Ne zamerite nam, če dobrohotno ponudimo še kak nasvet o študiju matematike.


\paragraph{Zahvala.}
%
Za pomoč pri urejanju zapiskov in opozarjanje na napake se zahvaljujeva študentkam in študentom:
%
Luka Debevc,
Milan Djaković,
Ema Grmšek,
Matija Fajfar,
Miha Gyergyek,
Peter Jereb,
Jan Kastelic,
Jan Malej,
Matej Marinko,
Jan Pantner,
Lev Rus,
Jakob Schrader,
Matija Sirk,
Matej Šafarič,
Gal Zmazek,
Marjetka Zupan in Patrik Žnidaršič.
%
Vse preostale napake so najina last.
\bigskip

\begin{flushright}
Andrej Bauer in Davorin Lešnik
\end{flushright}

\bigskip


%%% Local Variables: 
%%% mode: latex
%%% TeX-master: "lmn"
%%% End: 
