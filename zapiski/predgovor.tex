\chapter*{Predgovor}
\label{chap:predgovor}

Pri predmetu Logika in množice v prvem letniku študija matematike se študenti
učijo osnov matematičnega izražanja: kako beremo in pišemo matematično besedilo, kako
uporabljamo simbolni zapis, kako zapišemo in preberemo dokaz.
%
Hkrati spoznavajo temelje matematične logike in teorije množic.
%
Za semesterski predmet z dvema urama predavanj in dvema urama vaj ima predmet ambiciozen program.
Pričujoči učbenik naj bodo v pomoč slušateljem ter vsem, ki jih v je v svet matematike zvabila želja razumeti, na čem stoji ta prelepa veda.

\paragraph{Zahvala.}
%
Za pomoč pri urejanju zapiskov in opozarjanje na napake se zahvaljujeva študentkam in študentom:
%
Luka Debevc,
Ema Grmšek,
Matija Fajfar,
Miha Gyergyek,
Peter Jereb,
Jan Kastelic,
Jan Malej,
Matej Marinko,
Jan Pantner,
Lev Rus,
Jakob Schrader,
Matija Sirk,
Matej Šafarič,
Gal Zmazek,
Marjetka Zupan in Patrik Žnidaršič.
%
Vse preostale napake so najina last.
\bigskip

\begin{flushright}
Andrej Bauer in Davorin Lešnik
\end{flushright}

\bigskip


%%% Local Variables: 
%%% mode: latex
%%% TeX-master: "lmn"
%%% End: 
