\chapter*{Predgovor}
\label{chap:predgovor}

Pri predmetu Logika in množice v prvem letniku študija matematike na Fakulteti za matematiko in fiziko Univerze v Ljubljani se študenti učijo osnov matematičnega izražanja---kako beremo in pišemo matematično besedilo in simbolni zapis---hkrati pa spoznavajo temelje matematične logike in teorije množic.
%
Pred študenti matematike je težka naloga, saj pri vseh predmetih, ne le pri Logika in množice, predavatelji in asistenti zahtevne matematične novosti razlagajo v zapletenem matematičnem žargonu.
%
Da bi čim bolj olajšala začetno privajanje na ta krasni novi svet, sva se v pričujočem učbeniku razpisala tudi o matematični folklori in navadah, o katerih matematiki le malo kdaj spregovorijo.
%
In ne zamerite nama, če dobrohotno ponudiva še kak nasvet o študiju matematike.


\paragraph{Zahvala.}
%
Za pomoč pri urejanju zapiskov in opozarjanje na napake se zahvaljujeva študentkam in študentom:
%
Luka Debevc,
Ema Grmšek,
Matija Fajfar,
Miha Gyergyek,
Peter Jereb,
Jan Kastelic,
Jan Malej,
Matej Marinko,
Jan Pantner,
Lev Rus,
Jakob Schrader,
Matija Sirk,
Matej Šafarič,
Gal Zmazek,
Marjetka Zupan in Patrik Žnidaršič.
%
Vse preostale napake so najina last.
\bigskip

\begin{flushright}
Andrej Bauer in Davorin Lešnik
\end{flushright}

\bigskip


%%% Local Variables: 
%%% mode: latex
%%% TeX-master: "lmn"
%%% End: 
