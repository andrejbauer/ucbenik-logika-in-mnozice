\chapter{Osnovno o množicah in preslikavah}
\label{cha:mnozice-in-preslikave}


Temeljni gradniki sodobne matematike so \df{množice}, ki so skupki ali zbirke matematičnih
objektov, lahko spet množic. Vsaka množica sestoji iz \df{elementov} in je z njimi
natančno določena.
%
Kadar je $a$ element množice $M$, to zapišemo $$a \in M$$ in beremo \nls{$a$ je element~$M$} ali \nls{$a$ pripada~$M$}.
%
Slišali boste tudi \nls{$a$ je vsebovan v~$M$}, a to rabo odsvetujemo, ker tudi $A \subseteq M$ beremo \nls{$A$ je vsebovan v $M$}.

V splošni razpravi o množicah, ki bi presegala meje ožje matematične vede, bi se opirali na
zgodovinski in družbeni kontekst, jezikovni izvor in rabo besed `množica', `skupek' in
`zbirka', kognitivno analizo, eksperimente, filozofijo itn. Vsi ti vidiki so za matematike
izjemo koristni, saj iz takih `pred-matematičnih' obravnav črpamo sveže zamisli in
matematiko naredimo zares uporabno. Ko pa delujemo znotraj matematike, zunanje vplive
odmislimo in se zanašamo le še na pravila logičnega sklepanja in dogovorjene matematične zakone, da ne
prihaja do nejasnosti in dvomljivih sklepov.

Kot matematiki lahko ustvarimo takšen ali drugačen pojem množice in pri tem imamo popolno
svobodo. Se množica lahko spreminja ali vedno vsebuje iste elemente? Je pomemben vrsti red
elementov v množici? Sme množica biti element same sebe? Ali morajo biti elementi množice
izračunljivi? To so vprašanja, ki nimajo enoznačnega odgovora. In res je znanih več med
seboj nezdružljivih zvrsti teorije množic, ki matematično opredeljujejo različne vidike
običajnega razumevanja besede `množica'. Mi bomo spoznali tisto, ki jo uporablja velika večina matematikov.

Še enkrat poudarimo, da ima vsakdo, še posebej pa mladi um, popolno svobodo matematičnega
ustvarjanja. Želite razmišljati o drugačnih množicah, ki ne zadoščajo pravilom iz tega učbenika?
Ali pa o številih, ki zadoščajo zakonu $x + x = 0$? O geometriji, v
kateri skozi točko lahko potegnemo dve vzporednici k dani premici? Kar dajte! Pri tem vas
le prosimo, celo zahtevamo, da razmišljate temeljito, vztrajno in globoko, da ste iskreni
do sebe in ostalih ter da svoje zamisli in spoznanja predstavite na matematikom razumljiv
način.


\section{Pravilo ekstenzionalnosti}
\label{sec:nacelo-ekstenzionalnosti}

Zamisel, da je množica natančno določena s svojimi elementi, izrazimo z matematičnim
zakonom, ki mu pravimo \df{pravilo ekstenzionalnosti}:

\begin{pravilo}[Ekstenzionalnost množic]
  Množici sta enaki, če vsebujeta iste elemente.
\end{pravilo}

Kaj pravzaprav pomeni, da je to \nls{pravilo}, \nls{matematični zakon} ali \nls{načelo}? So ga
razglasili v parlementu, je to zakon narave, ali morda dogma, ki jo je razglasil profesor
na predavanjih? Bodo tisti, ki pravila ekstenzionalnosti ne spoštujejo, deležni Lešnikove
masti? Ne. Matematični zakoni so \emph{dogovori}, nekakšna pravila matematične igre. Skozi
čas so se uveljavila tista, ki so bila uporabna v naravoslovju in tehniki, ali pa so v njih
matematiki prepoznali notranjo lepoto in lastno uporabno vrednost.

V matematiki je izbor besed, s katerimi poimenujemo pojme, hkrati pomemben in nebistven.
%
S povsem človeškega stališča je pomembno, da izbiramo besede, katerih predhodni vsakdanji pomen nakazuje matematični pomen.
%
Hkrati je izbor besed nebistven, saj strukturo, lastnosti in povezave med
matematičnimi objekti določajo izključno dogovorjeni matematični zakoni in ne raba besed izven polja matematike.
%
Če bi se dogovorili, da namesto besed  `množica' in `element' govorimo `zbor' in `član', ali celo `morje' in `riba', se matematična vsebina teorije morja ne bi čisto nič spremenila.
%
Čeprav se to sliši zabavno, ne gre izzivati svojih stanovskih kolegic in kolegov.

\section{Končne množice}
\label{sec:koncne-mnozice}

Vrnimo se k našim množicam. Pravilo ekstenzionalnosti nam pove, da lahko množico podamo
tako, da natančno opredelimo njene elemente. A to ne pomeni, da množica obstaja, brž ko jo
lahko natančno opredelimo! Ta  pot vodi v protislovje, ki ga je odkril Bertrand Ruseell\footnote{Bertrand Arthur William Russell (1872--1970), angleški filozof logik in matematik.} na začetku 20.~stoletja in ga bomo še obravnavali.
Iz zagate so se matematiki rešili s previdno izbranimi pravili, ki določajo dopustne konstrukcije množic.

Posebej preprosta konstrukcija združi končen nabor matematičnih objektov v množico.
Na primer, če so $a$, $b$ in $c$ matematični objekti, potem lahko tvorimo množico
%
\begin{equation*}
  \set{a, b, c}
\end{equation*}
%
katere elementi so natanko $a$, $b$ in $c$. To pomeni, da za vsak matematični objekt~$x$
velja
%
\begin{equation*}
  \text{$x \in \set{a, b, c}$, če in samo če $x = a$ ali $x = b$ ali $x = c$.}
\end{equation*}
%
Fraza \nls{če in samo če} pomeni:
%
\begin{enumerate}
\item če $x = a$ ali $x = b$ ali $x = c$, potem $x \in \set{a, b, c}$ in
\item če $x \in \set{a, b, c}$, potem $x = a$ ali $x = b$ ali $x = c$.
\end{enumerate}
%
Prva trditev zagotavlja $1+1 \in \set{1, 2, 3}$, ker velja
vsaj ena od možnosti: $1 + 1 = 1$ ali $1 + 1 = 2$ ali $1 + 1 = 3$. Iz druge trditve sledi, da
$5 \in \set{1, 2, 3}$ ne velja, ker ne velja nobena od možnosti: $5 = 1$ ali $5 = 2$ ali
$5 = 3$.

Poskusimo zapisati splošno pravilo, ki zajema zgornje primere.

\begin{pravilo}
  \label{pravilo:koncna-mnozica}
  Za poljuben končen nabor objektov $a, b, \ldots, z$ je $\set{a, b, \ldots, z}$ množica, katere elementi
  so natanko objekti $a, b, \ldots, z$.
\end{pravilo}

Za trenutek ustavimo tok misli in opozorimo, da zapis s tropičjem `$\ldots$' ni dovolj
natančen, saj dopušča dvoumnosti. Denimo, so elementi množice
%
\begin{equation*}
  \set{3, 5, 7, \ldots, 31},
\end{equation*}
%
liha števila med $3$ in $31$, ali samo praštevila? Kljub
temu tak zapis uporabljamo, ker v praksi bralec večinoma pravilno ugane, kaj je
bilo mišljeno, saj imamo ljudje zelo podobne sposobnosti prepoznavanja vzorcev. Z
matematičnega vidika pa to ni dopustno, ker lahko tropičje \emph{vedno} razumemo na več
načinov.

Pa tu še ni konec težav z zapisom pravila~\ref{pravilo:koncna-mnozica}. Ali smemo
tvoriti množico, ki ima več elementov, kot je črk abecede? Ali bi bilo pravilo še vedno
isto, če bi namesto $a, b, \ldots, z$ zapisali $a, b, \ldots, j$? Ali smemo
tvoriti množico z nič elementi? Če namreč vstavimo nič elementov, se pravilo glasi \nls{Za
vse objekte je $\set{\,}$ množica, katere elementi so natanko objekti}, kar bi učiteljica
slovenščine prečrtala z rdečo. Iz nesrečnega tropičja se res ne vidi, kaj je in kaj ni dovoljeno. 
%
Iz zagate se bomo izvili malo kasneje, ko bomo pravilo končnih množic nadomestili s tremi
bolj preprostimi, ki imajo skupaj enakovreden učinek.

Preverimo, ali ima pravilo ekstenzionalnosti vsaj pričakovani učinek.
%
Če res zagotavlja, da vrstni red in število pojavitev elementov v množici ni pomembno, bi moralo slediti
$\set{1, 2} = \set{2, 1, 1}$. Pa je to res?

\begin{trditev}
  Množici $\set{1, 2}$ in $\set{2, 1, 1}$ sta enaki.
\end{trditev}

\begin{proof}
  Dokaz, ki ga bomo zapisali je izjemno podroben in ga v praksi ne bi zapisali,
  saj je z njegovim branjem več dela, kot če bi ga poustvarili sami. Ker pa želimo pokazati, da
  tudi najbolj trivialna dejstva lahko dokažemo, ga zapišimo.

  Izhajati smemo izključno iz naslednji dejstev:
  %
  \begin{itemize}
  \item pravilo ekstenzionalnosti,
  \item $x \in \set{1, 2}$, če in samo če $x = 1$ ali $x = 2$,
  \item $x \in \set{2, 1, 1}$, če in samo če $x = 2$ ali $x = 1$ ali $x = 1$.
  \end{itemize}
  %
  Uporabimo pravilo ekstenzionalnosti na $\set{1, 2}$ in
  $\set{2, 1, 1}$, kar pomeni, da moramo preveriti, ali imata iste elemente. To
  naredimo v dveh korakih:
  %
  \begin{enumerate}
  \item Za vsak element $\set{1, 2}$ dokažemo, da je element $\set{2, 1, 1}$.

    Pa naj bo $x \in \set{1, 2}$. Iz definicije množice $\set{1, 2}$
    sledi, da je $x = 1$ ali $x = 2$. Obravnavamo dva primera:
    %
    \begin{enumerate}
    \item Primer $x = 1$: iz $x = 1$ sledi, da je $x = 2$ ali $x = 1$ ali $x = 1$, zato je $x \in \set{2, 1, 1}$.
    \item Primer $x = 2$: iz $x = 2$ sledi, da je $x = 2$ ali $x = 1$ ali $x = 1$, zato je $x \in \set{2, 1, 1}$.
    \end{enumerate}
    %
  \item Za vsak element $\set{2, 1, 1}$ dokažemo, da je element $\set{1, 2}$.

    Ta korak je zelo podoben prvemu, zato bi ga skoraj vsak matematik \nls{prepustil bralcu za vajo},
    a tokrat se bomo pomujali in ga zapisali.
    %
    Naj bo $x \in \set{2, 1, 1}$. Iz definicije množice $\set{2, 1, 1}$ sledi, da je $x = 2$ ali
    $x = 1$ ali $x = 1$. Obravnavamo tri primere:
    %
    \begin{enumerate}
    \item Primer $x = 2$: iz $x = 2$ sledi, da je $x = 1$ ali $x = 2$, zato je $x \in \set{1, 2}$.
    \item Primer $x = 1$: iz $x = 1$ sledi, da je $x = 1$ ali $x = 2$, zato je $x \in \set{1, 2}$.
    \item Primer $x = 1$: iz $x = 1$ sledi, da je $x = 1$ ali $x = 2$, zato je $x \in \set{1, 2}$.
    \end{enumerate}
    %
  \end{enumerate}
  %
  Dokaz je končan.
\end{proof}

Mimogrede, kvadratek označuje konec dokaza. Imenuje se tudi \emph{halmoš} po
Paulu Halmosu,\footnote{Paul Halmos (1916--2006), ameriški matematik madžarskega rodu.} ki ga je populariziral. Nekateri avtorji pišejo tudi QED, kar je okrajšava za \nls{quod erat demonstrandum}, vsaj ena študentska krilatica pa trdi, da je pravilna okrajšava \nls{quite easily done}.

Še enkrat se pojavi dvom o uporabi pravila~\ref{pravilo:koncna-mnozica}. Nikjer ne piše, da smemo pri naštevanju isti element večkrat ponoviti. Je sploh dovoljeno ponavljati elemente končne množice in pisati $\set{2, 1, 1}$?
%
V vsakdanjem življenju je vsaj nenavadno, da stvari po nepotrebnem ponavljamo.
%
V matematiki razumemo besedilo dobesedno. Ker v pravilu~\ref{pravilo:koncna-mnozica} piše \nls{za \textbf{vse}
objekte}, imamo povsem proste rok. Povedano z drugimi besedami, množico
$\set{2, 1, 1}$ smemo tvoriti, ker nikjer ne piše, da moramo našteti različne elemente.

Nesrečno tropičje v pravilu~\ref{pravilo:koncna-mnozica} povzroča dovolj dvomov, da bi ga bilo dobro odpraviti.
To dosežemo s tremi nadomestnimi pravili, ki imajo skupaj enak učinek kot prvotno pravilo.

\begin{pravilo}
  \label{pravilo:prazna-mnozica}
  \df{Prazna množica} $\emptyset$ je množica, ki nima elementov.
\end{pravilo}

\begin{pravilo}
  \label{pravilo:neurejeni-dvojec}
  Za vsak $x$ in $y$ je \df{(neurejeni) par} ali \df{dvojec} $\set{x, y}$ množica, katere
  elementa sta natanko $x$ in $y$.
\end{pravilo}

\begin{pravilo}
  \label{pravilo:unija}
  \df{Unija $A \cup B$} množic~$A$ in~$B$ je množica, ki ima za elemente
  natanko vse objekte, ki so element $A$ ali element $B$ (lahko tudi obeh).
\end{pravilo}

V pravilu~\ref{pravilo:neurejeni-dvojec} smo besedo \nls{neurejeni} zapisali v oklepaju, kar
pomeni, da jo običajno opustimo in rečemo samo \nls{dvojec}, kar smo že storili v naslovu tega razdelka. Se pravi, da \nls{neurejeni dvojec} in \nls{dvojec} pomenita isto. V primeru nejasnosti raje uporabimo daljšo obliko.

Prvo pravilo pojasni, da lahko tvorimo množico brez elementov.
Poleg oznake~$\emptyset$ se za prazno množico uporablja tudi zapis $\set{\,}$.

Drugo pravilo pove, kako lahko tvorimo množico z dvema elementoma, pa tudi z enim.
Spomnimo se, pravila je treba brati dobesedno: za $x$ in $y$ bi lahko vzeli dvakrat isti
objekt~$z$ in tvorili množico $\set{z, z}$, ki ima natanko elementa $z$ in~$z$. To je
pravzaprav množica z enim samim elementom $z$, zato ji pravimo tudi \df{enojec} in jo
zapišemo~$\set{z}$.

Tretje pravilo nam omogoča, da tvorimo večje množice. Denimo, množico z elementi $a$, $b$,
$c$ lahko tvorimo kot unijo
%
\begin{equation*}
  \set{a, b} \cup \set{c}.
\end{equation*}
%
To ni edini način, enako množico lahko dobimo na več načinov:
%
\begin{equation*}
  (\set{a} \cup \set{b}) \cup \set{c}
  \quad\text{ali}\quad
  \set{b} \cup \set{c, a}
  \quad\text{ali}\quad
  \set{a,c,a} \cup \set{b,c}
  \quad\text{itn.}
\end{equation*}
%
Kogar to zabava, lahko dokazuje, da so vse te množice enake.

Priročno je imeti v roki neko množico z enim elementom, pri čemer je vseeno, kateri element vsebuje.
%
Naslednje pravilo zagotavlja tako množico.

\begin{pravilo}
  \label{pravilo:enojec}
  \df{Standardni enojec}~$\one$ je množica, katere edini element je~$\unit$.
\end{pravilo}

Morda se zdi nenavadno, da množico označimo s številom, a ta občutek bo hitro izginil, ko
bomo računali z množicami. Pravaprav bi lahko prazno množico označili z nič $\mathbf{0}$
in nekateri matematiki to dejansko počnejo.

Edini element množice $\one$ smo zapisali $\unit$. To ni tiskarska napaka,
ampak načrtna izbira, ki bo pojasnjena v razdeleku~\ref{sec:aritmetika-mnozic}.
In seveda velja $\one = \set{\unit}$.

Pravilo~\ref{pravilo:enojec} ni nujno potrebno, saj lahko tvorimo veliko različnih enojcev,
začenši z $\set{\emptyset}$, katerega obstoj zagotavljata pravili za prazno množico in dvojec.
Da se ne bi vsakič znova ukvarjali z nepotrebnim izbiranjem nekega enojca, smo enega od njih
proglasili za prvega med enakimi. S prazno množico nimamo podobnih težav, saj je ena sama.

\section{Preslikave}
\label{sec:preslikave}

Temelj matematike ne tvorijo le množice, ampak tudi drugi matematični pojmi. Prvi izmed
njih je \df{preslikava}, oziroma s tujko \df{funkcija}.\footnote{Nekateri uporabljajo
  izraz `funkcija' samo za tiste preslikave, ki slikajo v realna ali kompleksna števila,
  vendar to navado izpodriva računalništvo, saj funkcije v programskih jezikih nimajo
  omejitev. Dandanes večina matematikov besedo `funkcija' obravnava kot sopomenko besede
  `preslikava' in tako jo bomo uporabljali tudi mi.} V srednji šoli ste že spoznali
nekatere preslikave, kot so na primer linearne preslikave, trigonometrijske funkcije,
logaritem itd. Nas pa ne bodo zanimale posamezne preslikave, ali posebne lastnosti
preslikav, ampak preslikave na splošno.

Vsaka preslikava ima tri sestavne dele: \df{domeno} ali \df{začetno množico},
\df{kodomeno} ali \df{ciljno množico} in \df{prirejanje}. Domeni se pogosto reče tudi
\df{definicijsko območje}. Če govorimo o preslikavi, ki ima domeno~$X$ in kodomeno~$Y$, to
ponazorimo s puščico med $X$ in $Y$, takole
%
\begin{equation*}
  \xymatrix{
    {X} \ar[r] &
    {Y}
  }
\end{equation*}
%
Če želimo preslikavo poimenovati, na primer $f$, zapišemo
%
\begin{equation*}
  \xymatrix{
   {f : X} \ar[r] &
    {Y}
  }
  \qquad\text{ali}\qquad
  \xymatrix{
   {X} \ar[r]^{f} &
   {Y}
  }
\end{equation*}
%
Pravimo, da je \df{$f$ preslikava iz $X$ v $Y$}. Zapis nad puščico je prikladen, kadar
imamo opravka z večimi preslikavami, ki jih predstavimo z diagramom. Na primer,
%
\begin{equation*}
  \xymatrix{
    {X} \ar[r] &
    {Y} \ar[r]^{f} &
    {Z}  &
    {W} \ar[l]_{g}
  }
\end{equation*}
%
nam pove, da imamo opravka z (neimenovano) preslikavo iz $X$ v $Y$, s preslikavo $f$ iz
$Y$ v $Z$ in s preslikavo $g$ is $W$ v $Z$. Diagrami so lahko še precej bolj zapleteni.

Tretji del preslikave je \df{prirejanje}, ki določa, kako elemente domene preslikamo v elemente
kodomene. Kaj pravzaprav to pomeni? Možnih je več odgovorov. V srednji šoli preirejanje
enačimo z matematično formulo, ki tako imenovano `neodvisno spremenljivko' preslika v vrednost, na primer $x$ slika v
$2 \sin(x + \pi/4)$. S simboli tak predpis zapišemo
%
\begin{equation*}
  x \mapsto 2 \sin(x + \pi/4).
\end{equation*}
%
in preberemo \nls{$x$ se slika v dvakrat sinus od $x$ plus pi četrtin.}
%
Matematiki smo natančni, zato ne mešamo uporabe puščic $\to$ in $\mapsto$. Navadna puščica
se uporablja pri oznaki domene in kodomene, repata pa v predpisu. V računalništvu besedo
`predpis' razumemo kot `programska koda' in o preslikavah razmišljajo kar kot o
algoritmih.

\subsection{Identiteta in konstantna preslikava}
\label{sec:ident-konst-presl}

Dve posebej preprosti zvrsti preslikav sta identiteta in konstantna preslikava.
Za vsako množico~$A$ je \df{identiteta} na~$A$ preslikava
%
\begin{equation*}
  \id[A] : A \to A
\end{equation*}
%
ki elementu $x \in A$ priredi~$x$.
%
Za vsaki množici $A$ in $B$ ter $b \in B$ \df{konstantna preslikava}
%
\begin{equation*}
  \konst{b} : A \to B
\end{equation*}
%
priredi vsakemu elementu iz~$A$ element~$b$.
%
S funkcijskim predpisom zapišemo identiteto in konstantno preslikavo takole:
%
\begin{align*}
  \id[A] &: A \to A
  &
  \konst{b} &: A \to B
  \\
  \id[A] &: x \mapsto x
  &
  \konst{b} &: x \mapsto b.
\end{align*}

\subsection{Kako še lahko podamo prirejanje}
\label{sec:drugi-nacin-prirejanje}

V teoriji množic lahko prirejanje med elementi domene~$X$ in kodomene podamo~$Y$ na kak drug način, ki ga ne moremo neposredno izraziti z matematično formulo. Paziti moramo le, da zadošča pogojema:
%
\begin{itemize}
\item \df{celovitost}: vsakemu elementu iz $X$ je prirejen vsaj en element iz $Y$,
\item \df{enoličnost}: če sta elementu $x \in X$ prirejena $y \in Y$ in $z \in Y$, potem $y = z$.
\end{itemize}
%
Obravnavajmo nekaj zgledov.

\begin{zgled}
  \label{zgled:prirejanje-opis}
  Definirajmo preslikavo $f : \RR \to \RR$ s prirejanjem, ki elementu $x \in \RR$ iz domene priredi najmanjše realno število $y \in \RR$ iz kodomene, za katerega velja $y^5 + y = x$.
  %
  Ali to prirejanje sploh poda preslikavo? Ali je res, da za vsak~$x \in \RR$ obstaja natanko en~$y \in \RR$, ki je najmanjše realno število~$y$, ki zadošča $y^5 + y - x = 0$? Odgovor je pritrdilen: $y$ obstaja, ker je $y^5 + y - x$ polinom lihe stopnje v spremenljivki~$y$, torej ima vsaj eno ničlo. Ker je ničel končno mnogo, je natanko ena od njih najmanjša.
  %
  Če bi potenco~$5$ nadomestili s~$6$, bi zgled pokvarili. Znate to utemeljiti?
\end{zgled}

Preslikavo s \emph{končno} domeno lahko podamo s tabelo, denimo:
%
\begin{center}
  $f : \set{1, 2, 3, 5} \to \NN$

  \medskip

  \begin{tabular}{cc}
    \toprule
    $x$ & $f(x)$ \\ \midrule
    $1$ & $10$ \\ 
    $2$ & $10$ \\ 
    $3$ & $20$ \\ 
    $5$ & $10$ \\ \bottomrule
  \end{tabular}
\end{center}
%
Tabela elementu v levem stolpcu priredi istoležni element v desnem stolpcu: $1$ priredi $10$, $2$ priredi $10$, $3$ priredi $20$ in $5$ priredi $10$. Funkcijo lahko predstavimo tudi tako, da naštejemo vsa prirejanja:
%
\begin{align*}
  f(1) &\dfeq 10    &&\text{ali}&    f : 1 &\mapsto 10 \\
  f(2) &\dfeq 10    &&&    f : 2 &\mapsto 10 \\
  f(3) &\dfeq 20    &&&    f : 3 &\mapsto 20 \\
  f(5) &\dfeq 10    &&&    f : 5 &\mapsto 10.
\end{align*}
%
To se še vedno le tabele, predstavljene na drugačen način.
%
Tabelarični prikaz lahko uporabimo, kader je domena~$f$ končna množica, katere elemente lahko naštejemo.

Preslikava je lahko določena tudi z opisom računskega postopka, pravimo mu \df{algoritem},
s pomočjo katerega izračunamo vrednost preslikave pri danem argumentu. Paziti moramo, da je
opis postopka res natančen in nedvoumen, lahko ga kar zapišemo kot program.


\subsection{Funkcijski predpisi}
\label{sec:funkcijski-predpisi}

Predpisu \nls{$x$ se slika v \dots}, ki ga zapišemo
%
\begin{equation*}
  x \mapsto \cdots
\end{equation*}
%
pravimo tudi \df{funkcijski predpis}. Posvetimo se mu in se ob njem naučimo nekaj natančnosti.
Na desni, lahko namesto $\cdots$ zapišemo izraz, v katerem se sme pojaviti simbol~$x$, denimo
%
\begin{equation*}
  x \mapsto 1 + x^2.
\end{equation*}
%
Spremenljivka~$x$ nima v naprej določene vrednosti, pač pa kaže, kam lahko vstavimo
elemente domene. Pravimo, da je $x$ \df{vezana spremenljivka}, kar pomeni, da je veljavna
le v funkcijskem predpisu, nanj je vezana, in da ni pomembno, s katerim simbolom jo
označimo. Tako sta funkcijska predpisa
%
\begin{equation*}
  x \mapsto 1 + x^2
  \qquad\text{in}\qquad
  a \mapsto 1 + a^2
\end{equation*}
%
enaka in lahko bi celo pisali $\Box \mapsto 1 + \Box^2$ ali
$\heartsuit \mapsto 1 + \heartsuit^2$.

V funkcijskem predpisu se smejo pojaviti tudi druge spremenljivke, ki jim
pravimo \df{parametri}. V predpisu
%
\begin{equation*}
  x \mapsto a \cdot x + b,
\end{equation*}
%
tu sta $a$ in $b$ parametra in $x$ vezana spremenljivka. Ni vseeno, katere spremenljivko so vezane in katere parametri:
%
\begin{center}
  \begin{tabular}{ll}
    \toprule
    Predpis & Pomen \\ \midrule
    $x \mapsto a \cdot x + b$ \qquad\hbox{} & \nls{pomnoži z~$a$ in prištej~$b$} \\
    $a \mapsto a \cdot x + b$ & \nls{pomnoži z~$x$ in prištej~$b$} \\
    $b \mapsto a \cdot x + b$ & \nls{prištej~$a \cdot x$} \\
    \bottomrule
  \end{tabular}
\end{center}

V funkcijskem predpisu mora na levi stati en sam simbol, ki na desni kaže, kam je treba
vstaviti element iz domene. Tako
%
\begin{equation*}
  \sin(x) \mapsto \cos(2 x),
  \qquad
  3 + 2 \mapsto 5
  \qquad\text{in}\qquad
  \sin(x) \mapsto 2 \cdot \sin(x)
\end{equation*}
%
\emph{niso} veljavni funkcijski predpisi. Kasneje bomo spoznali izjeme, ko je dovoljeno pisati tudi kaj drugega kot golo spremenljivko.

Seveda dopuščamo možnost, da se vezana spremenljivka pojavi enkrat, večkrat ali sploh ne.
Funkcijska predpisa
%
%
\begin{equation*}
  x \mapsto 42
  \qquad\text{in}\qquad
  x \mapsto x \cdot \sin(x)
\end{equation*}
%
sta torej veljavna.

Če želimo preslikavo z danim funkcijskim predpisom poimenovati, na primer~$f$, zapišemo
%
\begin{equation*}
  f : x \mapsto 1 + x^2.
\end{equation*}
%
To preberemo \nls{$f$ slika $x$ v ena plus $x$ na kvadrat}. Običajna sta tudi zapisa
%
\begin{equation*}
  f(x) = 1 + x^2
  \qquad\text{in}\qquad
  f(x) \dfeq 1 + x^2.
\end{equation*}
%
Mi se bomo držali zapisa z $\dfeq$, da bomo lahko ločili med definicijo~$f$ in trditvijo, da je $f(x)$ enak neki vrednosti.

\begin{zgled}
  Z zgledom pojasnimo, zakaj je pametno zapisati definicijo z enim simbolom in enakost z drugim:
  %
  \begin{quote}
    ``Naj bo preslikava $f : \RR \to \RR$ definirana s predpisom $f(x) \dfeq 1 + x^2$. Tedaj za vse $x \in \RR$ velja $f(x) = (x + 1)^2 - 2 x$.'' 
  \end{quote}
  %
  V prvi povedi smo~$f$ \emph{definirali}, v drugi pa \emph{trdili}, da velja zapisana enakost.
\end{zgled}

Funkcijske predpise je podrobno prvi preučeval Alonzo Church,\footnote{Alonzo Church
  (1903--1995) je bil ameriški matematik in logik, ki je pomembno prispeval k razvoju
  logike in teoretičnega računalništva. Njegov študent, Dana Stewart Scott, je imel
  študenta Marka Petkovška in Andreja Bauerja, slednji pa je imel študenta Davorina
  Lešnika.} ki je uporabljal zapis
%
\begin{equation*}
  \lambda x \,.\, 1 + x^2
\end{equation*}
%
in teorijo funkcijskih predpisov poimenoval \df{$\lambda$-račun}. V logiki se je njegov
zapis obdržal in se uveljavil tudi v programski jezikih:
%
\begin{itemize}
\item v Pythonu pišemo \verb|lambda x: 1 + x ** 2|,
\item v Haskellu pišemo \verb|\x -> 1 + x ** 2| in
\item v OCamlu pišemo \verb|fun x => 1 + x * x|.
\end{itemize}
%
Predvsem v programiranju funkcijskim predpisom pravijo tudi \df{anonimne} ali \df{brezimne
  preslikave}.

Nekateri starejši zapisi funkcijskih predpisov so slabi, a jih ljudje vztrajno
uporabljajo. Opozorimo le na en slab zapis, ki povzroča precej preglavic, ne da bi se
matematiki tega zares zavedali. Funkcijski predpis mora določati vezano spremenljivko,
sicer ne vemo, kako vstaviti vrednosti. Na žalost jo matematiki pogosto izpustijo skupaj
z $\mapsto$ in pišejo $1 + x^2$ namesto $x \mapsto 1 + x^2$.
%
Težava je v tem, da se lahko v funkcijskem predpisu pojavi več kot en simbol. Če vam na primer nekdo pove, da ima v mislih funkcijski predpis
%
\begin{equation*}
  a \cdot x + b
\end{equation*}
%
boste zaradi ustaljenih navad v šolskem sistemu vsi mislili, da je mišljeno $x \mapsto a \cdot x + b$.
%
A pravzaprav bi lahko bilo tudi
%
\begin{equation*}
  a \mapsto a \cdot x + b
  \quad\text{ali}\quad
  b \mapsto a \cdot x + b
  \quad\text{ali celo}\quad
  t \mapsto a \cdot x + b!
\end{equation*}
% %
% Morda pa lahko vezano spremenljivko in $\mapsto$ brez škode izpustimo, če v izrazu nastopa
% samo en simbol, denimo $1 + x^2?$
% %
% A spet bi zabredli v težave. Je $42$ število ali funkcijski predpis $x \mapsto 42$? Je
% $1 + x^2$ funkcijski predpis $x \mapsto 1 + x^2$ ali $a \mapsto 1 + x^2$?

Poudarimo, da je funkcijski predpis le eden od treh sestavnih delov preslikave in zato same \emph{ne} podaja preslikave.
%
Če ne poznamo domene, ne moremo preveriti, ali je funkcijski predpis celovit. Denimo,
%
\begin{equation*}
  x \mapsto \frac{x}{x^2 - 2}
\end{equation*}
%
ni celovit kot preslikava $\mathbb{R} \to \mathbb{R}$ in je celovit kot preslikava $\mathbb{Q} \to \mathbb{Q}$.


\subsection{Uporaba in zamenjava}
\label{sec:uporaba-in-zamenjava}

Do sedaj smo se ukvarjali s tem, kako preslikavo podamo, zdaj pa se vprašajmo, kako lahko
preslikavo uporabimo. Če je $f : X \to Y$ preslikava iz $X$ v $Y$ in je $x \in X$, potem
lahko \df{$f$ uporabimo na $x$} in dobimo \df{vrednost} preslikave~$f$ pri
\df{argumentu}~$x$, to je tisti edini element $Y$, ki ga~$f$ priredi~$x$. Vrednost $f$
pri~$x$ zapišemo
%
\begin{equation*}
  f(x)
  \qquad\text{ali}\qquad
  f\,x
\end{equation*}
%
in preberemo \nls{$f$ od $x$} ali \nls{$f$ pri $x$}. Izraza $f(x)$, oziroma $f\,x$, se imenuje
\df{aplikacija}. Večinoma se uporablja zapis z oklepaji, a ne vedno: navajeni smo pisati
$\ln 2$ in $\sin \alpha$ namesto $\ln(2)$ in $\sin(\alpha)$. Oklepaje izpuščamo tudi v
nekaterih programskih jezikih in občasno v algebri.

V analizi je uveljavljen še en zapis za aplikacijo, ki se uporablja za zaporedja. Namreč,
zaporedje ni nič drugega kot preslikava $a : \NN \to \RR$ iz naravnih v realna števila.
Aplikacijo $a(n)$, ki označuje $n$-ti člen zaporedja, ponavadi pišemo~$a_n$, torej
argument podpišemo.

Preslikavo lahko uporabimo na argumentu tudi, če je nismo poimenovali. Na primer,
preslikavo
%
\begin{align*}
  \RR &\to \RR \\
  x &\mapsto 1 + x^2
\end{align*}
%
uporabimo na argumentu~$3$:
%
\begin{equation*}
  (x \mapsto 1 + x^2)(3).
\end{equation*}
%
Se vam zdi tak zapis nenavaden? Verjetno, a pomislite, zakaj: ker so vas vzgojili, da
je treba vse preslikave poimenovati in se nanje skliceval z danim imenom.\footnote{Če bi veljalo enako tudi za števila, vam v srednji šoli ne bi pustili pisati kar $3 + 5$, nujno bi bilo poimenovanje $a \dfeq 3 + 5$. Tudi trikotnika ne bi smeli narisati, ne da bi mu dali simbolno ime.}
%
Taka vzgoja vzbuja občutek, da so preslikave bolj imenitne in zastrašujoče kot števila, daljice in drugi matematični objekti. Morda je občutek do neke mere utemeljen za laika, a prav hitro bomo spoznali, da preslikave niso nič
posebnega in da lahko z njimi delamo enako kot s števili, vektorji in ostalimi matematičnimi objekti. Računalničarji radi rečejo, da je treba tudi preslikave obravnavati kot enakopravne državljane.

Kako izračunamo vrednost funkcije pri danem argumentu? To je odvisno od tega,
kako je podano prirejanje. Če imamo tabelarični prikaz, poiščemo argument v levem stolpcu
in pogledamo v pripadajoči desni stolpec. Če je preslikava podana s funkcijskim predpisom, argument
vstavimo v predpis. Na primer, če je $f : \RR \to \RR$ podana s funkcijskim predpisom
%
\begin{equation*}
  f(x) \dfeq 1 + x^2,
\end{equation*}
%
potem je vrednost $f(3)$ enaka $1 + 3^2$ -- vezano spremenljivko~$x$ smo zamenjali s~$3$. (Seveda je $1 + 3^2$ enako~$10$, a to je že naslednji korak, ki zahteva dodatno računanje.)
Pravimo, da smo simbol~$x$ \df{zamenjali} ali
\df{substituirali} s~$3$, oziroma da smo~$3$ \df{vstavili} v predpis za~$f$ namesto~$x$. 
%
Preslikavo lahko uporabimo tudi na kakem bolj zapletenem argumentu, na primer:
%
\begin{align*}
  f(3) &= 1 + 3^2, \\
  f(2 + \sqrt{5}) &= 1 + (2 + \sqrt{5})^2, \\
  f(y) &= 1 + y^2, \\
  f(y + 2 z^2) &= 1 + (y + 2 z^2)^2, \\
  f(x) &= 1 + x^2.
\end{align*}
%
V vseh primerih smo le zamenjali vezano spremenljivko~$x$ z argumentom.
%
Tudi zadnja vrstica je zamenjava, v kateri smo (vezano spremenljivko)~$x$ zamenjali z (prosto) spremenljivko~$x$.

Uporabimo lahko tudi funkcijski predpis, pri čemer še vedno velja, da vezano spremenljivko zamenjamo z argumentom.
Tako je
%
\begin{equation*}
  (x \mapsto 1 + x^2)(3)
\end{equation*}
%
spet enako $1 + 3^2$.
%
V razdelku~\ref{sec:eksponent} bomo spoznali še dodatna pravila za vstavljanje izrazov, ki se vrtijo okoli vezanih spremenljivk.


\subsection{Pravilo ekstenzionalnosti preslikav}

Podobno kot za množice tudi za preslikave velja pravilo ekstenzionalnosti, ki pravi, da sta preslikavi enaki, če imata enako domeno in kodomeno ter prirejata argumentom enake vrednosti.


\begin{pravilo}[Ekstenzionalnost preslikav]
  Preslikavi sta enaki, če imata enaki domeni in kodomeni ter imata za vse argumente
  enaki vrednosti.
\end{pravilo}

Natančneje, če sta $f : A \to B$ in $g : C \to D$ preslikavi in velja $A = C$, $B = D$ ter
za vsak $x \in A$ velja $f(x) = g(x)$, tedaj velja $f = g$.

Opozorimo na razliko med
%
\begin{equation*}
  f(x) = g(x)
  \qquad\text{in}\qquad
  f = g.
\end{equation*}
%
Levi izraz pravi, da sta $f(x)$ in $g(x)$ enaka elementa kodomene, desni pa da sta $f$ in~$g$ enaki preslikavi
%
Na sploh je treba razlikovati med $f$ in $f(x)$, saj to nikakor nista enaka objekta: prvi je
preslikava, drugi pa vrednost te preslikave pri~$x$. Verjetno nihče ne bi trdil, da je
preslikava $\cos$ isto kot $\cos \frac{\pi}{4}$, ali ne? Isti razmislek veleva, da
$\cos x$ ni isto kot $\cos$, če tudi si mislimo, da je $x \in \RR$ neko neznano realno število.
Zmeda izhaja iz površnega izražanja, ko na primer rečemo \nls{$x^2 + \cos x$ je soda funkcija}, čeprav bi bilo pravilno \nls{$x \mapsto x^2 + \cos x$ je soda funkcija}.

Če dosledno uporabljamo funkcijske predpise, lažje razumemo, da sta $\cos$ in $x \mapsto \cos x$ enaki preslikavi, zahvaljujoč pravilu ekstenzionalnosti, obe pa sta različni od $\cos x$, ki sploh ni funkcija, ampak realno število, odvisno od parametra~$x$.

V bran tradicionalnemu zapisu moramo vseeno povedati, da se lahko \emph{dogovorimo} za
nekoliko napačen zapis, če to ne povzroča zmede. S tem se izkušeni matematiki izognejo preveč birokratskemu
pisanju nebistvenih podrobnosti in lahko bolj učinkovito komunicirajo.
%
A začetnikom priporočamo, da v dobrobit boljšega razumevanja snovi vsaj na začetku študija raje vztrajajo pri doslednem zapisu.

Vrnimo se še k pravilu ekstenzionalnosti preslikav. Ali ni pravzaprav očitno, da sta
preslikavi enaki, če imata enaki domeni, kodomeni in vrednosti? Morda res, a to ni razlog,
da tega ne bi eksplicitno zapisali. Vsak matematik vam ve povedati kako zgodbo o tem,
kako se je v dokazu skrivala napako ravno tam, kjer je bilo nekaj `očitno'. Poleg tega
pa si lahko predstavljamo razmere, v katerih je smiselno razlikovati med dvema
preslikavama, ki imata vedno enake vrednosti, denimo v programiranju, kjer je računska učinkovitost
zelo pomembna.

\subsection{Kompozicija}
\label{sec:kompozicija}

Kompozicija preslikav je temeljna operacija, ki združi preslikavi $f : A \to B$ in $g : B \to C$ v preslikavo
$g \circ f : A \to C$, podano s prirejanjem
%
\begin{equation*}
  (g \circ f)(x) \dfeq g (f (x)).
\end{equation*}
%
Zakaj pišemo $g \circ f$ in ne obratno $f \circ g$?  Ker si je mnogo lažje zapomniti zgornje računsko pravilo kot pravilo $(f \circ g)(x) = g(f(x))$, ki bi ga dobili, če bi pisali kompozicijo v obratnem vrstnem redu.

\begin{trditev}
  \parbox{0pt}{}
  %
  \begin{enumerate}
  \item Identiteta je nevtralna za kompozicijo: $\id[B] \circ f = f = f \circ \id[A]$.
  \item Kompozicija je asociativna: $(h \circ g) \circ f = h \circ (g \circ f)$.
  \end{enumerate}
\end{trditev}

\begin{proof}
  Trditev je zapisana pomanjkljivo, saj ne piše, kaj so $A$, $B$,$ f$ in~$g$. Avtorja
  trditve bi lahko vprašali, kaj je hotel povedati, a je bolje, da poskusimo to razvozlati
  sami, ker je to odlična vaja iz razumevanja matematičnih besedil.

  Takoj vidimo, da je $A$ množica, sicer zapis $\id[A]$ ne bi bil smiselen, in podobno je
  tudi $B$ množica. Simboli $f$, $g$ in $h$ zagotovo označujejo preslikave, saj nastopajo
  v kompoziciji. Kaj pa njihove domene in kodomene? Preslikava $f$ mora imeti domeno $A$,
  sicer ne bi bilo dovoljeno komponirati $f \circ \id[A]$, in mora imeti kodomeno $B$,
  sicer ne bi bilo dovoljeno komponirati $\id[B] \circ f$. Ostaneta še domeni in kodomeni
  preslikav $g$ in~$h$. Kompozicija $g \circ f$ kaže, da mora biti domena $g$ enaka
  kodomeni~$f$, torej $B$. Kompozicija $h \circ g$ pa pove, da je kodomena $C$ enaka
  domeni $h$. Če vse to zložimo v diagram, dobimo
  %
  \begin{equation*}
    \xymatrix{
      {A} \ar[r]^{f}
      &
      {B} \ar[r]^{g}
      &
      {\text{?}} \ar[r]^{h}
      &
      {\text{?}}
    }
  \end{equation*}
  %
  Trditev moramo razumeti tako, da bo čim bolj splošna in smiselna. Torej bomo za neznani
  množici vzeli kar poljubni množici $C$ in $D$:
  %
  \begin{equation*}
    \xymatrix{
      {A} \ar[r]^{f}
      &
      {B} \ar[r]^{g}
      &
      {C} \ar[r]^{h}
      &
      {D}
    }
  \end{equation*}
  %
  Preverimo, ali smo trditev pravilno razumeli. Ko vstavimo podrobnosti, se prvi del
  glasi: \nls{Za vse množice $A$ in $B$ ter preslikavo $f : A \to B$ velja
  $\id[B] \circ f = f = f \circ \id[A]$.}
  %
  Ker je to smiselna izjava, jo dokažimo. Enakost preslikav se dokaže z ekstenzionalnostjo
  preslikav, torej preverimo, ali imajo $\id[B] \circ f$, $f$ in $f \circ \id[A]$ enako
  vrednost za poljuben $x \in A$:
  %
  \begin{align*}
    (\id[B] \circ f)(x) &= \id[B] (f (x)) = f (x), \\
    f (x) &= f(x), \\
    (f \circ \id[A])(x) &= f (\id[A](x)) = f(x).
  \end{align*}
  %
  Zapišimo podrobno še drugi del: ">Za vse množice $A$, $B$, $C$ in $D$ ter preslikave
  $f : A \to B$, $g : B \to C$ in $h : C \to D$ velja
  $(h \circ g) \circ f = h \circ (g \circ f)$. To spet dokažemo tako, da uporabimo levo in
  desno stran enačbe na poljubnem $x \in A$:
  %
  \begin{align*}
    ((h \circ g) \circ f)(x) &= (h \circ g)(f(x)) = h(g(f(x))) \\
    (h \circ (g \circ f))(x) &= h((g \circ f)(x)) = h(g(f(x))). \qedhere
  \end{align*}
\end{proof}




\subsection{Kosoma podano prirejanje}

Včasih podamo prirejanje `po kosih', kar pomeni, da domeno razdelimo na podmnožice in za vsako posebej povemo, kako na njej deluje prirejanje.
%
Zgled, ki ga že poznate iz srednje šole, je preslikava 'absolutno', ki je definirana po kosih za negativna in nenegativna števila:
% 
\begin{align*}
  \RR &\to \RR \\
  x &\mapsto
      \begin{cases}
        -x & \text{če $x \leq 0$,}\\
        x & \text{če $x \geq 0$.}
      \end{cases}
\end{align*}
% 
Pravilo preberemo: \nls{če je $x \leq 0$, mu priredimo vrednost $-x$, in če je $x \geq 0$, mu priredimo vrednost~$x$}.
%
Domeno $\RR$ smo razdelili na množico nepozitivnih števil $\set{x \in \RR \such x \leq 0}$ in množico nenegativnih\footnote{Preberite pozorno, piše \nls{ne-negativnih}!} števil $\set{x \in \RR \such x \geq 0}$.
%
Celovitost tako podanega prirejanja je zagotovljena, če je unija kosov celotno domeno,
enoličnost pa v primeru, ko se prirejanja skladajo na preseku.
%
V zgornjem zgledu je to res, saj je vsako realno število nepozitivno ali nenegativno, edino hkrati nenegativno in nepozitivno število pa je $0$ in to zadošča $-0 = 0$.

Če bi absolutno vrednost definirali s predpisom
%
\begin{align*}
  \RR &\to \RR \\
  x &\mapsto
      \begin{cases}
        -x & \text{če $x < 0$,}\\
        x & \text{če $x \geq 0$,}
      \end{cases}
\end{align*}
%
se kosa ne bi prekrivala in nam ne bi bilo treba preverjati $-0 = 0$.



\section{Vaje}

\begin{vaja}
Kaj veste povedati o množici~$A$, če zanjo velja, da so vsi njeni elementi enaki?
\begin{resitev}
Množica~$A$ ima kvečjemu en element, tj.~množica~$A$ je bodisi prazna bodisi enojec. Tudi: množica~$A$ je podmnožica kakega enojca oz.~edina preslikava $A \to \one$ je injektivna.
\end{resitev}
\end{vaja}

\begin{vaja}
  Pravilo ekstenzionalnosti preslikav bi lahko zapisali tudi takole:
  %
  \begin{quote}
    Preslikavi $f : A \to B$ in $g : C \to D$ sta enaki, če velja $A = C$, $B = D$ in za
    vse $x_1, x_2 \in A$ velja, da iz $x_1 = x_2$ sledi $f(x_1) = g(x_2)$.
  \end{quote}
  %
  Dokažite, da je ta različica enakovredna običajnem pravilu ekstenzionalnosti.
\end{vaja}

\begin{vaja}
 Definirajmo preslikavo $g : \NN \to \ZZ$ z zahtevo, da naravnemu številu $n \in \NN$ priredi tisto celo število $k \in \ZZ$, za katerega velja $k^2 \leq n < (k+1)^2$.
 %
 Preverite, da je prirejanje celovito in enolično in podajte čim bolj razumljiv besedni opis preslikave~$g$.
\end{vaja}
