%% TO NI GLAVNA DATOTEKA
\documentclass[11pt,a4paper,twoside]{book}

%%%%%%%%%%%%%%%%%%%%%%%%%%%%%%%%%%%%%%%%%%%%%%%%%%%%%%%%%
%%%  Imported Packages
%%%%%%%%%%%%%%%%%%%%%%%%%%%%%%%%%%%%%%%%%%%%%%%%%%%%%%%%%
\usepackage[slovene]{babel}
\usepackage[utf8]{inputenc}
\usepackage[T1]{fontenc}

\usepackage{ifthen}
\usepackage{amssymb}
\usepackage{amsmath}
\usepackage{amsthm} % Must come after amsmath!
\usepackage{textgreek}
\usepackage{phonetic}
\usepackage{tablefootnote}
\usepackage{datetime2}

\usepackage{xcolor}
\definecolor{andrejcolor}{rgb}{0.7,0,0.7}
\definecolor{davorincolor}{rgb}{0,0.45,1}
\definecolor{anjacolor}{rgb}{1.0,0.5,0}

\newcommand{\andrej}[1]{{\small\textcolor{andrejcolor}{(#1 \ \mbox{--Andrej})}}}
\newcommand{\davorin}[1]{{\small\textcolor{davorincolor}{(#1 \ \mbox{--Davorin})}}}
\newcommand{\anja}[1]{{\small\textcolor{anjacolor}{(#1 \ \mbox{--Anja})}}}

\definecolor{notecolor}{rgb}{0.6,0.5,0.7}
\newcommand{\note}[1]{{\small\textcolor{notecolor}{(#1)}}}
\newcommand{\alert}[1]{{\small\textcolor{red}{\textbf{#1}}}}

\usepackage{tikz}
\usepackage{tkz-graph}
\usepackage{xparse}
\usepackage{mathrsfs}
\usepackage{answers}

\usepackage{xypic}

%%% Trenutna verzija, če jo imamo
\IfFileExists{./verzija.tex}{\input{./verzija.tex}}{\newcommand{\OPTversion}{unknown}}

%%% Fonti kot v HoTT book.
\usepackage{mathpazo}
\usepackage[scaled=0.95]{helvet}
\usepackage{courier}
\linespread{1.05} % Palatino looks better with this

%%%%%%%%%%%%%%%%%%%%%%%%%%%%%%%%%%%%%%%%%%%%%%%%%%
%%% Globalne nastavitve
\def\OPTtitle{Logika in množice}

%%%%%%%%%%%%%%%%%%%%%%%%%%%%%%%%%%%%%%%%%%%%%%%%%%%%%%%%%%%%%
%%  Page Style & Margins (A4 page = 210mm x 297mm)

% PAGE GEOMETRY
\usepackage[papersize={210mm,297mm}, % A4
            twoside,
            includehead,
            top=1in, % margina na vrhu strani
            bottom=1in, % margina na dnu strani
            inner=1in, % margina na notranji strani strani
            outer=1in, % margina na zunanji strani strani
            bindingoffset=0pt % dodatna margina na notranji strani
           ]{geometry}

%%%%%%%%%%%%%%%%%%%%%%%%%%%%%%%%%%%%%%%%%%%%%%%%%%%%%%%%%%%%%
% HYPERLINKING AND PDF METADATA
\usepackage[backref=page,
            colorlinks,
            citecolor=linkcolor,
            linkcolor=linkcolor,
            urlcolor=linkcolor,
            unicode,
            pdfauthor={Andrej Bauer, Davorin Lešnik},
            pdftitle={\OPTtitle},
            pdfsubject={matematika},
            pdfkeywords={logika,množice,osnove matematike}]{hyperref}
\renewcommand{\backref}[1]{}
\renewcommand{\backrefalt}[4]{%
   \ifcase #1 %
   (Ni citirano.)
   \or
   (Citirano na strani\ #2.)
   \else
   (Citirano na straneh\ #2.)
   \fi}

\definecolor{linkcolor}{rgb}{0,0,0} % Barva hiperpovezav

%%%%%%%%%%%%%%%%%%%%%%%%%%%%%%%%%%%%%%%%%%%%%%%%%%%%%%%%%%%%%
%%%% Header and footers
%%%%%%%%%%%%%%%%%%%%%%%%%%%%%%%%%%%%%%%%%%%%%%%%%%%%%%%%%%%%%

\usepackage{fancyhdr} % To set headers and footers
\pagestyle{fancyplain}
\setlength{\headheight}{15pt}
\renewcommand{\chaptermark}[1]{\markboth{\textsc{Poglavje \thechapter. #1}}{}}
\renewcommand{\sectionmark}[1]{\markright{\textsc{\thesection\ #1}}}

\lhead[\fancyplain{}{{\thepage}}]%
      {\fancyplain{}{\nouppercase{\rightmark}}}
\rhead[\fancyplain{}{\nouppercase{\leftmark}}]%
      {\fancyplain{}{\thepage}}
\cfoot[\texttt{\footnotesize [delovna verzija {\OPTversion}- \DTMnow]}]{\texttt{\footnotesize [delovna verzija {\OPTversion}- \DTMnow]}}
\lfoot[]{}
\rfoot[]{}

%%%%%%%%%%%%%%%%%%%%%%%%%%%%%%%%%%%%%%%%%%%%%%%%%%%%%%%%%%%%%
%%%%%% Macros
%%%%%%%%%%%%%%%%%%%%%%%%%%%%%%%%%%%%%%%%%%%%%%%%%%%%%%%%%%%%%%%%%%%%%%%%%%%%%%%%%%%%%%%%%%%%%%%%%%%%%%%%%%%%%%%%%%%%%%
%%%  Commands
%%%%%%%%%%%%%%%%%%%%%%%%%%%%%%%%%%%%%%%%%%%%%%%%%%%%%%%%%%%%%%%%%%%%%%%%%%%%%%%%%%%%%%%%%%%%%%%%%%%%%%%%%%%%%%%%%%%%%%


%%%%%%  Auxiliary
%%%%%%%%%%%%%%%%%%%%%%%%%%%%%%%%%%%%%%%%%%%%%%%%%%%%%%%%%%%%%
\newcommand{\sizedescriptor}[2]
{
\ifthenelse{\equal{#1}{0}}{}{
\ifthenelse{\equal{#1}{1}}{\big}{
\ifthenelse{\equal{#1}{2}}{\Big}{
\ifthenelse{\equal{#1}{3}}{\bigg}{
\ifthenelse{\equal{#1}{4}}{\Bigg}{
#2}}}}}
}

\newcommand{\someref}{{\small\textcolor{blue}{[\textbf{ref.}]}}}
\newcommand{\intermission}{\bigskip\medskip}
\newcommand{\qt}[1]{{\quotedblbase}{#1}{‘‘}}  % text in quotation marks
\newcommand{\nls}[1]{\qt{\textit{#1}}}  % sentence in a natural language

\definecolor{andrejcolor}{rgb}{0.7,0,0.7}
\definecolor{davorincolor}{rgb}{0,0.45,1}
\definecolor{markocolor}{rgb}{0.7,0.4,0}
\definecolor{matijacolor}{rgb}{0,0.6,0.4}

\newcommand{\andrej}[1]{{\small\textcolor{andrejcolor}{(#1 \ \mbox{--Andrej})}}}
\newcommand{\davorin}[1]{{\small\textcolor{davorincolor}{(#1 \ \mbox{--Davorin})}}}
\newcommand{\marko}[1]{{\small\textcolor{markocolor}{(#1 \ \mbox{--Marko})}}}
\newcommand{\matija}[1]{{\small\textcolor{matijacolor}{(#1 \ \mbox{--Matija})}}}

\definecolor{notecolor}{rgb}{0.6,0.5,0.7}
\newcommand{\note}[1]{{\small\textcolor{notecolor}{(#1)}}}
\newcommand{\alert}[1]{{\small\textcolor{red}{\textbf{#1}}}}


%%%%%%  Logical Quantifiers, λ- and ι-Expressions
%%%%%%%%%%%%%%%%%%%%%%%%%%%%%%%%%%%%%%%%%%%%%%%%%%%%%%%%%%%%%

%  no parenthesis (add x in front of the name of the command)
\NewDocumentCommand{\xall}
{m O{\empty} m}
{\forall\, {#1} \ifthenelse{\equal{#2}{}}{}{\in{#2}} \,.\, {#3}}
\NewDocumentCommand{\xsome}
{m O{\empty} m}
{\exists\, {#1} \ifthenelse{\equal{#2}{}}{}{\in{#2}} \,.\, {#3}}
\NewDocumentCommand{\xexactlyone}
{m O{\empty} m}
{\exists\;\!!\, {#1} \ifthenelse{\equal{#2}{}}{}{\in{#2}} \,.\, {#3}}
\NewDocumentCommand{\xlam}
{m O{\empty} m O{\empty}}
{\lambda{#1} \ifthenelse{\equal{#2}{}}{}{\in{#2}} \,.\, {#3} \ifthenelse{\equal{#4}{}}{}{\in{#4}}}
\NewDocumentCommand{\xthat}
{m O{\empty} m}
{\iota{#1} \ifthenelse{\equal{#2}{}}{}{\in{#2}} \,.\, {#3}}

%  with parenthesis -- the first optional argument is the size (values 0-4)
\NewDocumentCommand{\all}
{O{auto} m O{\empty} m}
{\xall{#2}[#3]{\sizedescriptor{#1}{\left}( {#4} \sizedescriptor{#1}{\right})}}
\NewDocumentCommand{\some}
{O{auto} m O{\empty} m}
{\xsome{#2}[#3]{\sizedescriptor{#1}{\left}( {#4} \sizedescriptor{#1}{\right})}}
\NewDocumentCommand{\exactlyone}
{O{auto} m O{\empty} m}
{\xexactlyone{#2}[#3]{\sizedescriptor{#1}{\left}( {#4} \sizedescriptor{#1}{\right})}}
\NewDocumentCommand{\lam}
{O{auto} m O{\empty} m O{\empty}}
{\xlam{#2}[#3]{\sizedescriptor{#1}{\left}( {#4} \sizedescriptor{#1}{\right})}[#5]}
\NewDocumentCommand{\that}
{O{auto} m O{\empty} m}
{\xthat{#2}[#3]{\sizedescriptor{#1}{\left}( {#4} \sizedescriptor{#1}{\right})}}


%%%%%%  Logic
%%%%%%%%%%%%%%%%%%%%%%%%%%%%%%%%%%%%%%%%%%%%%%%%%%%%%%%%%%%%%
\newcommand{\tvs}{\Omega}  % set of truth values
\newcommand{\true}{\top}  % truth
\newcommand{\false}{\bot}  % falsehood
\newcommand{\etrue}{\bm{\top}}  % emphasized truth
\newcommand{\efalse}{\bm{\bot}}  % emphasized falsehood
\newcommand{\impl}{\Rightarrow}  % implication sign
\newcommand{\revimpl}{\Leftarrow}  % reverse implication sign
\newcommand{\lequ}{\Leftrightarrow}  % equivalence sign
\newcommand{\xor}{\mathbin{\veebar}}  % exclusive disjunction sign
\newcommand{\shf}{\mathbin{\uparrow}}  % Sheffer connective
\newcommand{\luk}{\mathbin{\downarrow}}  % Łukasiewicz connective


%%%%%%  Sets
%%%%%%%%%%%%%%%%%%%%%%%%%%%%%%%%%%%%%%%%%%%%%%%%%%%%%%%%%%%%%
%  \set{1, 2, 3}         ->  {1, 2, 3}
%  \set{a \in X}{a < 1}  ->  {a ∈ X | a < 1}
\NewDocumentCommand{\set}
{O{auto} m G{\empty}}
{\sizedescriptor{#1}{\left}\{ {#2} \ifthenelse{\equal{#3}{}}{}{ \; \sizedescriptor{#1}{\middle}| \; {#3}} \sizedescriptor{#1}{\right}\}}
%\newcommand{\vsubset}{\Mapstochar\cap}
%\newcommand{\finseq}[1]{{#1}^*}
\newcommand{\pst}{\mathcal{P}}
\renewcommand{\complement}[1]{{#1}^C}


%%%%%%  Number Sets, Intervals
%%%%%%%%%%%%%%%%%%%%%%%%%%%%%%%%%%%%%%%%%%%%%%%%%%%%%%%%%%%%%
\newcommand{\NN}{\mathbb{N}}
\newcommand{\ZZ}{\mathbb{Z}}
\newcommand{\QQ}{\mathbb{Q}}
\newcommand{\RR}{\mathbb{R}}
\newcommand{\CC}{\mathbb{C}}
\newcommand{\intoo}[3][\RR]{{#1}_{(#2, #3)}}
\newcommand{\intcc}[3][\RR]{{#1}_{[#2, #3]}}
\newcommand{\intoc}[3][\RR]{{#1}_{(#2, #3]}}
\newcommand{\intco}[3][\RR]{{#1}_{[#2, #3)}}


%%%%%%  Maps and Relations
%%%%%%%%%%%%%%%%%%%%%%%%%%%%%%%%%%%%%%%%%%%%%%%%%%%%%%%%%%%%%
\newcommand{\id}[1][]{\textrm{Id}_{#1}}  % identity map
\newcommand{\argbox}{{\;\!\fbox{\phantom{M}}\;\!}}  % box for a function argument
\newcommand{\rstr}[1]{\left.{#1}\right|}  % map restriction
\newcommand{\im}{\mathrm{im}}  % map image
\newcommand{\parto}{\mathrel{\rightharpoonup}}  % partial mapping sign
\NewDocumentCommand{\rel}
{O{\empty} O{\empty}}
{\ifthenelse{\equal{#1}{}}{\mathscr{R}}{{#1} \mathrel{\mathscr{R}} {#2}}}  % a relation
\NewDocumentCommand{\srel}
{O{\empty} O{\empty}}
{\ifthenelse{\equal{#1}{}}{\mathscr{S}}{{#1} \mathrel{\mathscr{S}} {#2}}}  % a second relation
\newcommand{\dom}{\mathrm{dom}}  % domain
\newcommand{\cod}{\mathrm{cod}}  % codomain
\newcommand{\dd}[1]{D_{#1}}  % domain of definition
\newcommand{\rn}[1]{Z_{#1}}  % range
\newcommand{\graph}[1]{\Gamma_{#1}}  % graph of a (partial) function
\NewDocumentCommand{\img}  % image
{O{\empty} m G{\empty}}
{{#2}_*\ifthenelse{\equal{#3}{}}{}{\!\sizedescriptor{#1}{\left}( {#3} \sizedescriptor{#1}{\right})}}
\NewDocumentCommand{\pim}  % preimage
{O{\empty} m G{\empty}}
{{#2}^*\ifthenelse{\equal{#3}{}}{}{\!\sizedescriptor{#1}{\left}( {#3} \sizedescriptor{#1}{\right})}}
\newcommand{\ec}[2][]{[\:\!{#2}\:\!]_{#1}}  % equivalence class
\newcommand{\transposed}[1]{\widehat{#1}}


%%%%%%  Misc.
%%%%%%%%%%%%%%%%%%%%%%%%%%%%%%%%%%%%%%%%%%%%%%%%%%%%%%%%%%%%%
\newcommand{\df}[1]{\emph{\textbf{#1}}}  % defined notion
\newcommand{\oper}{\mathop{\circledast}}  % symbol for a general operation
\newcommand{\ism}{\cong}  % isomorphic
\newcommand{\equ}{\sim}  % equivalent
\newcommand{\dfeq}{\mathrel{\mathop:}=}  % definitional equality
\newcommand{\revdfeq}{=\mathrel{\mathop:}}  % reverse definitional equality
\newcommand{\isdefined}[1]{{#1}\!\downarrow}  % given value is defined
\newcommand{\kleq}{\simeq}  % Kleene equality
\newcommand{\claim}[3]{{#1} \;\colon\; \frac{#2}{#3}}  % claim, divided on context, assumptions, conclusions
\newcommand{\unit}{\mathord{\bm{*}}}  % element in a generic singleton
\NewDocumentEnvironment{implproof}  % proof of an implication
{O{\empty} G{\empty} O{=>} G{\empty}}
{
\begin{description}
\item[\quad$\sizedescriptor{#1}{\left}({#2}
\ifthenelse{\equal{#3}{=>}}{\impl}{
\ifthenelse{\equal{#3}{<=}}{\revimpl}{
\ifthenelse{\equal{#3}{->}}{\rightarrow}{
\ifthenelse{\equal{#3}{<-}}{\leftarrow}{
#3
}}}} {#4}\sizedescriptor{#1}{\right})$]\ \vspace{0.3em}\\
}
{
\end{description}
}


%%%%%%%%%%%%%%%%%%%%%%%%%%%%%%%%%%%%%%%%%%%%%%%%%%%%%%%%%%%%%%%%%%%%%%%%%%%%%%%%%%%%%%%%%%%%%%%%%%%%%%%%%%%%%%%%%%%%%%


\begin{document}

%--------------------------------------------------------------------
%--------------------------------------------------------------------
% TITLE PAGE

% če se spremeni naslov, je treba spremeniti tudi zgoraj v paketu hyperref
\title{\OPTtitle\\\texttt{\OPTversion}}
\author{Andrej Bauer \and Davorin Lešnik}
\maketitle

%--------------------------------------------------------------------
%--------------------------------------------------------------------
% Foreword

\chapter*{Predgovor}%\addcontentsline{toc}{chapter}{\numberline{}Predgovor}

%--------------------------------------------------------------------
%--------------------------------------------------------------------
% TOC

\tableofcontents
% \listoftables % To se meni zdi nepotrebno, zakaj se to daje v knjige? (Andrej)

%--------------------------------------------------------------------
%--------------------------------------------------------------------
% BODY

\Opensolutionfile{resitve}

\chapter{Matematično izražanje}

	\section{Simbolni zapis}
	\section{Pravila dokazovanja}
	\section{Definicije}
\chapter{Preproste množice}
\label{cha:preproste-mnozice}


Temeljni gradniki sodobne matematike so \df{množice}, ki so skupki ali zbirke matematičnih
objektov, lahko spet množice. Vsaka množica sestoji iz \df{elementov} in je z njimi
natančno določena. Kadar je $a$ element množice $M$, to zapišemo kot $a \in M$.

Ideja množice kot poljubne zbirke elementov je zavajajoče preprosta, kar so na lastni koži
izkusili matematiki na prelomu iz 19.~v 20.~stoletje. Takrat so že vedeli, da so množice zelo
uporabne in da lahko iz njih tvorimo razne vrste matematičnih objektov. A znameniti
matematik in filozof Bertrand Russell je odkril paradoks, ki se imenuje po njem, in gre
takole. Naj bo~$R$ množica vseh množic, ki niso element same sebe. Ali $R$ je element~$R$?
Če je $R$ element $R$, potem iz definicije $R$ sledi, da $R$ ni element $R$. In če $R$ ni
element $R$, spet iz definicije $R$ sledi, da $R$ je element $R$. Torej $R$ hkrati je in
ni svoj element, kar je protislovje! Russellov paradoks ste morda že spoznali v
priljubljeni različici, ki govori o vaškem brivcu, ki brije vse vaščane, ki ne brijejo
samih sebe.

Russellov paradoks je povzročil pravo krizo v temeljih matematike. Ker so bile množice
nepogrešljivo orodje, jih niso hoteli kar zavreči, po drugi strani pa je bilo treba
preprečiti Russellov in druge paradokse, ki so jih še odkrili. Bertrand Russell je
predlagal rešitev, ki jo je poimenoval \df{teorija tipov}. Russellova teorija tipov je
pomembno vplivala na nadaljni razvoj temeljev matematike, sodobna teorija tipov pa je
pomembno orodje v računalništvu. Tako kot množice so bili tipi skupki elementov, a so
tvorili neskončno hierarhijo, v kateri so bili elementi tipa vedno iz nižjega nivoja
hierarhije kot tip, ki so mu pripadali. Za potrebe večine matematike zadostuje že
preprostejša dvoslojna hierarhija množic in \df{razredov}. Množice smejo biti elementi
množic in razredov, razredi pa ne. Russellov paradoks izgine, ker je $R$ razred vseh
tistih množic, ki niso same svoj element. Vprašanje, ali je $R$ element samega sebe, tako
postane nesmiselno, saj $R$ ni množica. A zaenkrat odložimo podrobnejšo obravnavo razredov
in se raje posvetimo osnovnima pojmoma, množica in preslikava.

V splošni razpravi o množicah, ki bi presegala meje matematične vede, bi se opirali na
zgodovinski in družbeni kontekst, jezikovni izvor in rabo besed `množica', `skupek' in
`zbirka', kognitivno analizo, eksperimente, filozofijo itn. Vsi ti vidiki so za matematike
izjemo koristni, saj iz takih ``pred-matematičnih'' obravnav črpamo sveže zamisli in
matematiko naredimo zares uporabno. Ko pa delujemo znotraj matematike, zunanje vplive
odmislimo in se zanašamo le še na pravila logičnega sklepanja in matematične zakone, da ne
prihaja do nejasnosti in dvomljivih sklepov.

Kot matematiki lahko ustvarimo takšen ali drugačen pojem množice in pri tem imamo popolno
svobodo. Se množica lahko spreminja ali vedno vsebuje iste elemente? Je pomemben vrsti red
elementov v množici? Sme množica biti element same sebe? Ali morajo biti elementi množice
izračunljivi? To so vprašanja, ki nimajo enoznačnega odgovora. In res je znanih več med
seboj nezdružljivih zvrsti teorije množic, ki matematično opredeljujejo različne vidike
običajnega razumevanja besede `množica'. Mi bomo spoznali ``standardno'' teorijo množic,
ki jo uporablja velika večina matematikov.


\section{Načelo ekstenzionalnosti}
\label{sec:nacelo-ekstenzionalnosti}

Zamisel, da je množica natančno določena s svojimi elementi, izrazimo z matematičnim
zakonom, ki mu pravimo \df{načelo ekstenzionalnosti}:

\begin{pravilo}[Ekstenzionalnost množic]
  Množici sta enaki, če vsebujeta iste elemente.
\end{pravilo}

Kaj pravzaprav pomeni, da je to ``pravilo'', ``matematični zakon'' ali ``načelo''? So ga
razglasili v parlementu, je to zakon narave, ali morda dogma, ki jo je razglasil profesor
na predavanjih? Bodo tisti, ki načela ekstenzionalnosti ne spoštujejo, deležni Lešnikove
masti? Ne. Matematični zakoni so \emph{dogovori}, nekakšna pravila matematične igre. V
zgodovinskem razvoju matematike so se uveljavili tisti dogovori, ki so bili uporabni v
naravoslovju in tehniki, ali pa so v njih matematiki videli notranjo lepoto in lastno
uporabno vrednost.

Pravkar smo se dogovorili, da bomo obravnavali matematične objekte množice, ki vsebujejo
elemente in da zanje velja načelo ekstenzionalnosti. Namesto besed `množica' in `element'
bi lahko izbrali tudi kaki drugi besedi, denimo `zbor' in `član', ali celo `morje' in
`riba', s čimer se matematična vsebina pojmov ne bi čisto nič spremenila, čeprav ne gre
preveč izzivati svojih stanovskih kolegic in kolegov. Strukturo, lastnosti in povezave med
matematičnimi objekti namreč določajo dogovorjeni matematični zakoni in ne besede, s
katerimi jih poimenujemo.

Še enkrat poudarimo, da ima vsakdo, še posebej pa mladi um, popolno svobodo matematičnega
ustvarjanja. Želite razmišljati o drugačnih množicah, ki ne zadoščajo načelou
ekstenzionalnsti? Ali pa o številih, ki zadoščajo zakonu $x + x = 0$? O geometriji, v
kateri skozi točko lahko potegnemo dve vzporednici k dani premici? Kar dajte! Pri tem vas
le prosimo, celo zahtevamo, da razmišljate temeljito, vztrajno in globoko, da ste iskreni
do sebe in ostalih ter da svoje zamisli in spoznanja predstavite na matematikom razumljiv
način.

Vrnimo se k našim množicam. Načelo ekstenzionalnosti nam pove, da lahko množico podamo
tako, da natančno opredelimo njene elemente. A to ne pomeni, da množica obstaja, brž ko jo
lahko natančno opredelimo! To je pot, ki vodi naravnost do Russelovega paradoksa, saj so
elementi paradoksalne množice~$R$ natančno opredeljeni. Potrebujemo dodatna pravila, ki
določajo dopustne \df{konstrukcije množic}. Izbrati jih moramo previdno, da se izognemo
težavam.

\section{Končne množice}
\label{sec:koncne-mnozice}

Posebej preprosta konstrukcija množic združi končen nabor matematičnih objektov v množico.
Na primer, če so $a$, $b$ in $c$ matematični objekti, potem lahko tvorimo množico
%
\begin{equation*}
  \set{a, b, c}
\end{equation*}
%
katere objekti so natanko $a$, $b$ in $c$. To pomeni, da za vsak matematični objekt~$x$
velja
%
\begin{equation*}
  \text{$x \in \set{a, b, c}$, če in samo če $x = a$ ali $x = b$ ali $x = c$.}
\end{equation*}
%
Fraza ``če in samo če'' tu pomeni, da velja dvoje:
%
\begin{enumerate}
\item Če $x = a$ ali $x = b$ ali $x = c$, potem $x \in \set{a, b, c}$.
\item Če $x \in \set{a, b, c}$, potem $x = a$ ali $x = b$ ali $x = c$.
\end{enumerate}
%
Tako nam na primer prva trditev zagotavlja $1+1 \in \set{1, 2, 4}$, ker velja
vsaj ena od možnosti: $1 + 1 = 1$ ali $1 + 1 = 2$ ali $1 + 1 = 3$. Iz druge trditve sledi, da
$5 \in \set{1, 2, 3}$ ne velja, ker ne velja nobena od možnosti: $5 = 1$ ali $5 = 2$ ali
$5 = 3$.

Splošna konstrukcija končnih množic poteka takole.

\begin{pravilo}
  \label{pravilo:koncna-mnozica}
  Za vse objekte $a$, $b$, \dots, $z$ je $\set{a, b, \ldots, z}$ množica, katere elementi
  so natanko objekti $a$, $b$, \dots, $z$.
\end{pravilo}

Za trenutek ustavimo tok misli in opozorimo, da zapis s tropičjem `$\ldots$' ni dovolj
natančen, saj dopušča dvoumnosti. Denimo, so elementi množice
%
\begin{equation*}
  \set{3, 5, 7, \ldots, 31},
\end{equation*}
%
liha števila med $3$ in $31$, ali samo praštevila? Zapis res ni dovolj natančen. Kljub
temu tak zapis v praksi uporabljamo, ker v praksi bralec večinoma pravilno ugane, kaj je
bilo mišljeno, saj imamo ljudje zelo podobne sposobnosti prepoznavanja vzorcev. Z
matematičnega vidika pa to ni dopustno, saj lahko tropičje \emph{vedno} razumemo na več
načinov. (Ne verjamete? Naslednji člen v zaporedju $1, 2, 3, \ldots$ je seveda~$5$, ker je
naslednji člen vsota prejšnjih dveh, kot v Fibonaccijevem zaporedju.)

Kot smo že omenili, želimo pojem množice, pri kateri vrstni red elementov ni pomemben.
Torej bi morali biti množici $\set{1, 2}$ in $\set{2, 1}$ enaki. Pa je to res? Velja ena
od treh možnosti:
%
\begin{enumerate}
\item Iz načela ekstenzionalnosti in konstrukcije množic $\set{1, 2}$ in $\set{2, 1}$ sledi, da sta enaki.
\item Iz načela ekstenzionalnosti in konstrukcije množic $\set{1, 2}$ in $\set{2, 1}$ sledi, da nista enaki.
\item Načelo ekstenzionalnosti in konstrukcije množic $\set{1, 2}$ in $\set{2, 1}$ ne določajo, ali sta enaki.
\end{enumerate}
%
V prvem primeru bi želeli dokazati enakost. V drugem primeru smo v zagati, saj smo se
dogovorili za matematična pravila, ki imajo neželene posledice. V tretjem primeru moramo
dodati še kakšne nove zakone o množicah. Na srečo obvelja prva možnost.

\begin{trditev}
  Množici $\set{1, 2}$ in $\set{2, 1}$ sta enaki.
\end{trditev}

\begin{proof}
  Dokaz, ki ga bomo zapisali je izjemno podroben in ga v praksi matematik ne bi zapisal,
  saj je z njegovim branjem več dela, kot če bi naredili sami. Ker pa želimo pokazati, da
  tudi najbolj trivialna dejstva lahko dokažemo, ga zapišimo.

  Izhajati smemo izključno iz naslednji dejstev:
  %
  \begin{itemize}
  \item načelo ekstenzionalnosti,
  \item $x \in \set{1, 2}$, če in samo če $x = 1$ ali $x = 2$,
  \item $x \in \set{2, 1}$, če in samo če $x = 2$ ali $x = 1$.
  \end{itemize}
  %
  Najprej uporabimo načelo ekstenzionalnosti, ki zagotavlja, da sta $\set{1, 2}$ in
  $\set{2, 1}$ enaki, če imata iste elemente. Dokažimo torej, da imata iste elemente. To
  naredimo v dveh korakih:
  %
  \begin{enumerate}
  \item Dokažimo, da za vsak element $\set{1, 2}$ dokažemo, da je element $\set{2, 1}$.
    Naj bo $x \in \set{1, 2}$. Iz definicije množice $\set{1, 2}$
    sledi, da je $x = 1$ ali $x = 2$. Obravnavamo dva podprimera:
    %
    \begin{enumerate}
    \item Primer $x = 1$: iz $x = 1$ sledi, da je $x = 2$ ali $x = 1$, zato je $x \in \set{2, 1}$.
    \item Primer $x = 2$: iz $x = 2$ sledi, da je $x = 2$ ali $x = 1$, zato je $x \in \set{2, 1}$.
    \end{enumerate}
    %
  \item Dokažimo, da za vsak element $\set{2, 1}$ dokažemo, da je element $\set{1, 2}$.

    Ta korak je povsem podoben prvemu, le da je treba povsod zamenjati~$1$ in~$2$.
    Matematik bi zato na tem mestu zapisal, da je drugi korak podoben prevemu in dokaz
    zaključil. A tega tokrat ne bomo storili in bomo zapisali popoln dokaz.

    Naj bo $x \in \set{2, 1}$. Iz definicije množice $\set{2, 1}$ sledi, da je $x = 2$ ali
    $x = 1$. Obravnavamo dva primera:
    %
    \begin{enumerate}
    \item Primer $x = 2$: iz $x = 2$ sledi, da je $x = 1$ ali $x = 2$, zato je $x \in \set{1, 2}$.
    \item Primer $x = 1$: iz $x = 1$ sledi, da je $x = 1$ ali $x = 2$, zato je $x \in \set{1, 2}$. \qedhere
    \end{enumerate}
    %
  \end{enumerate}
\end{proof}

Mimogrede, črn kvadratek označuje konec dokaza. Imenuje se tudi ``Halmos'' po matematiku
Paulu Halmosu, ki ga je prvi uporabljal. S podobnim razmislekom, ki ga prepuščamo za vajo,
lahko dokažemo, da ni pomembno, ali se element pojavi enkrat ali večkrat.

\begin{naloga}
  Podrobno dokažite, da sta množici $\set{1, 1, 2}$ in $\set{1, 2}$ enaki.
\end{naloga}

V prejšnji nalogi smo zapisali $\set{1, 1, 2}$. Pa je to sploh dovoljeno?
Pravilo~\ref{pravilo:koncna-mnozica} pravi, da lahko iz objektov $a, b, c, \ldots, z$
tvorimo končno množico $\set{a, b, \ldots, z}$. Nikjer ne piše, da smeta biti $a$ in $b$
enaka, zato je upravičeno vprašanje, ali je dovoljeno za $a$ in $b$ vzeti~$1$. V
matematiki vse razumemo dobesedno. V pravilu~\ref{pravilo:koncna-mnozica} piše ``Za vse
objekte'', torej imamo povsem proste roke. Povedano z drugimi besedami, množico
$\set{1, 1, 2}$ smemo tvoriti, ker nikjer ne piše, da morajo biti elementi različni.

V zvezi s pravilom~\ref{pravilo:koncna-mnozica} se pojavljajo še drugi dvomi. Ali smemo
tvoriti množico, ki ima več elementov, kot je črk abecede? Ali bi bilo pravilo še vedno
isto, če bi namesto ``$a, b, \ldots, z$'' zapisali ``$a, b, \ldots, j$''? Ali smemo
tvoriti množico z nič elementi? Če namreč vstavimo nič elementov, se pravilo glasi ``Za
vse objekte je $\set{\,}$ množica, katere elementi so natanko objekti,'' kar je vsaj
nenavadno. Iz nesrečnega tropičja se res ne vidi, kaj je in kaj ni dovoljeno. Če poškilite
v razdelek~\ref{sec:aksiomi-teorije-mnozic}, kjer so našteti ``uradni'' aksiomih teorije
množic, tam pravila o končnih množicah ne boste našli, saj sledi iz treh bolj osnovnih
pravil.

\begin{pravilo}
  \label{pravilo:prazna-mnozica}
  \df{Prazna množica} $\emptyset$ je množica, ki nima elementov.
\end{pravilo}

\begin{pravilo}
  \label{pravilo:neurejeni-dvojec}
  Za vsak $x$ in $y$ je \df{(neurejeni) par} ali \df{dvojec} $\set{x, y}$ množica, katere
  elementa sta natanko $x$ in $y$.
\end{pravilo}

\begin{pravilo}
  \label{pravilo:unija}
  Za vsaki množici $A$ in $B$ je \df{unija $A \cup B$} množica, ki ima za elemente
  natanko vse objekte, ki so element $A$ ali element $B$.
\end{pravilo}

V pravilu~\ref{pravilo:neurejeni-dvojec} smo besedo ``neurejeni'' zapisali v oklepaju, kar
pomeni, da beseda pravzaprav ni pombembna in bi jo lahko tudi izpustili. Se pravi, da
``neurejeni dvojec'' in ``dvojec'' pomenita isto. V primeru nejasnosti raje uporabimo
daljšo obliko.

Tri nova pravila skupaj nadomestijo pravilo~\ref{pravilo:koncna-mnozica} in odstranijo
marsikateri dvom o uporabi. Prvo pravilo pojasni, da lahko tvorimo množico brez elementov.
Poleg oznake $\emptyset$ je za prazno množico smiselno uporabiti tudi zapis $\set{\,}$.

Drugo pravilo pove, kako lahko tvorimo množico z dvema elementoma, pa tudi z enim.
Spomnimo se, pravila je treba brati dobesedno: za $x$ in $y$ bi lahko vzeli dvakrat isti
objekt~$z$ in tvorili množico $\set{z, z}$, ki ima natanko elementa $z$ in $z$. To je
pravzaprav množica z enim samim elementom $z$, zato ji pravimo tudi \df{enojec} in jo
zapišemo~$\set{z}$.

Tretje pravilo nam omogoča, da tvorimo večje množice. Denimo, množico z elementi $a$, $b$,
$c$ lahko tvorimo kot unijo
%
\begin{equation*}
  \set{a, b} \cup \set{c}.
\end{equation*}
%
To ni edini način, enako množico lahko dobimo na več načinov:
%
\begin{equation*}
  (\set{a} \cup \set{b}) \cup \set{c}
  \quad\text{ali}\quad
  \set{b} \cup \set{c, a}
  \quad\text{ali}\quad
  \set{a,c,a} \cup \set{b,c}
  \quad\text{itn.}
\end{equation*}
%
Seveda bi morali dokazati, da so vse te množice enake, a tega ne bomo storili.

Pogosto nam bo prišlo prav, da bomo imeli pri roki množico z enim elementom, pri čemer nam
bo vseeno, kaj ta element je. V ta namen postavimo pravilo, ki zagotavlja obstoj množice z
enim elementom.

\begin{pravilo}
  \label{pravilo:enojec}
  \df{Standardni enojec} je množica~$\one$, katere edini element je~$\unit$.
\end{pravilo}

Morda se zdi nenavadno, da množico označimo s številom, a ta občutek bo hitro izginil, ko
bomo računali z množicami. Pravaprav bi lahko prazno množico označili z nič $\mathbf{0}$,
in nekateri matematiki to dejansko počnejo.

Edini element množice $\one$ smo označili z nenavadnim zapisom $\unit$. Na tem mestu ne
bomo pojasnili, zakaj pišemo tako, radovedneži pa lahko pogledajo v
razdelek~\ref{sec:aritmetika-tarskega}. Mimogrede, seveda velja $\one = \set{\unit}$.

Pravilo~\ref{pravilo:enojec} ni nujno potrebno, saj lahko tvorimo veliko različnih enojcev
kar sami $\set{\emptyset}$, $\set{42}$, $\set{\set{\emptyset}}$ itn. Ali je kateri od njih
``prvi med enakimi'' in bi ga lahko uporabljali kot ``standardni'' enojec? Ker je odgovor
v veliki meri stvar osebnega mnenja, je bolje, da razglasimo pravilo, ki ustoliči
standardni enojec. S prazno množico nimamo podobnih težav, saj je ena sama.

% \subsection{Druge množice}

% \andrej{To ne paše sem, ker bi bilo tu dosti bolj naravno nadaljevati s preslikavami.
%  To bomo prestavili na mesto, kjer bo dejansko prišlo prav.}

% Množice, s katerimi v matematiki delamo, tipično vsebujejo števila, ali pa so vsaj na tak ali drugačen način izpeljane iz številskih množic. Spomnimo se standardnih oznak najpogosteje uporabljanih številskih množic.
% \begin{center}
% \begin{tabular}{|cc|}
% \hline
% \textbf{Množica} & \textbf{Oznaka} \\
% \hline
% množica naravnih števil & $\NN$ \\
% množica celih števil & $\ZZ$ \\
% množica racionalnih števil & $\QQ$ \\
% množica realnih števil & $\RR$ \\
% množica kompleksnih števil & $\CC$ \\
% \hline
% \end{tabular}
% \end{center}

% Nekateri $0$ vzamejo za naravno število, nekateri ne. To je v celoti stvar dogovora, kaj pomeni pojem ``naravno število''. Za nas bo prišlo bolj prav, če ničlo štejemo kot element množice naravnih števil, torej $\NN = \set{0, 1, 2, 3, \ldots}$.

\section{Preslikave}

Temelj matematike ne tvorijo le množice, ampak tudi drugi matematični pojmi. Prvi izmed
njih je \df{preslikava}, oziroma s tujko \df{funkcija}.\footnote{Nekateri uporabljajo
  izraz ``funkcija'' samo za tiste preslikave, ki slikajo v realna ali kompleksna števila,
  vendar to navado izpodriva računalništvo, saj funkcije v programskih jezikih nimajo
  omejitev. Dandanes večina matematikov besedo ``funkcija'' obravnava kot sopomenko besede
  ``preslikava'' in tako jo bomo uporabljali tudi mi.} V srednji šoli ste že spoznali
nekatere preslikave, kot so na primer linearne preslikave, trigonometrijske funkcije,
logaritem itd. Nas pa ne bodo zanimale posamezne preslikave, ali posebne lastnosti
preslikav, ampak preslikave na splošno.

Vsaka preslikava ima tri sestavne dele: \df{domeno} ali \df{začetno množico},
\df{kodomeno} ali \df{ciljno množico} in \df{predpis}. Domeni se pogosto reče tudi
\df{definicijsko območje}. Če govorimo o preslikavi, ki ima domeno~$X$ in kodomeno~$Y$, to
ponazorimo s puščico med $X$ in $Y$, takole
%
\begin{equation*}
  \xymatrix{
    {X} \ar[r] &
    {Y}
  }
\end{equation*}
%
Če želimo preslikavo poimenovati, na primer $f$, zapišemo
%
\begin{equation*}
  \xymatrix{
   {f : X} \ar[r] &
    {Y}
  }
  \qquad\text{ali}\qquad
  \xymatrix{
   {X} \ar[r]^{f} &
   {Y}
  }
\end{equation*}
%
Pravimo, da je \df{$f$ preslikava iz $X$ v $Y$}. Zapis nad puščico je prikladen, kadar
imamo opravka z večimi preslikavami, ki jih predstavimo z diagramom. Na primer,
%
\begin{equation*}
  \xymatrix{
    {X} \ar[r] &
    {Y} \ar[r]^{f} &
    {Z}  &
    {W} \ar[l]_{g}
  }
\end{equation*}
%
nam pove, da imamo opravka z (neimenovano) preslikavo iz $X$ v $Y$, s preslikavo $f$ iz
$Y$ v $Z$ in s preslikavo $g$ is $W$ v $Z$. Diagrami so lahko še precej bolj zapleteni.

Tretji del preslikave je predpis, ki določa, kako elemente domene preslikamo v elemente
kodomene. Kaj pravzaprav to pomeni? Možnih je več odgovorov. V srednji šoli predpis
enačimo z matematično formulo, ki spremenljivko preslika v vrednost, na primer $x$ slika v
$2 \sin(x + \pi/4)$. S simboli to zapišemo
%
\begin{equation*}
  x \mapsto 2 \sin(x + \pi/4).
\end{equation*}
%
in preberemo ``$x$ se slika v dvakrat sinus od $x$ plus pi četrtin.''
%
Matematiki smo natančni, zato ne mešamo uporabe puščic $\to$ in $\mapsto$. Navadna puščica
se uporablja pri oznaki domene in kodomene, repata pa v predpisu. V računalništvu besedo
`predpis' razumemo kot `programska koda' in o preslikavah razmišljajo kar kot o
algoritmih --- tudi to je eden od možnih pogledov na preslikave.

V teoriji množic razumemo besedo `predpis' kot kakršnokoli prirejanje med elementi množic
domene~$X$ in kodomene~$Y$, mora pa veljati:
%
\begin{itemize}
\item \df{celovitost}: vsakemu elementu iz $X$ je prirejen vsaj en element iz $Y$,
\item \df{enoličnost}: če sta elementu $x$ prirejena $y \in Y$ in $z \in Y$, potem $y = z$.
\end{itemize}


\subsection{Funkcijski predpisi}
\label{sec:funkcijski-predpisi}

Predpise lahko podamo na različne načine, najbolj pogost pa je \df{funkcijski predpis}, ki
se mu še posebej posvetimo in se ob njem naučimo nekaj natančnosti. Funkcijski predpis ima
obliko
%
\begin{equation*}
  x \mapsto \cdots,
\end{equation*}
%
ki smo jo že videli maloprej. Na desni, lahko namesto $\cdots$ zapišemo izraz, v katerem
se sme pojaviti simbol~$x$, denimo
%
\begin{equation*}
  x \mapsto 1 + x^2.
\end{equation*}
%
Ni nujno, da se~$x$ pojavi, denimo $x \mapsto 42$ vsakemu elementu iz domene priredi
število $42$. V funkcijskem predpisu se smejo pojaviti tudi drugi simboli, ki jim
pravimo \df{parametri}. Tako je
%
\begin{equation*}
  x \mapsto a \cdot x + b
\end{equation*}
%
funkcijski predpis s parametroma $a$ in $b$, ki elementu $x$ priredi element $a \cdot x + b$.

Spremenljivka $x$ nima v naprej določene vrednosti, pač pa kaže, kam lahko vstavimo
elemente domene. Pravimo, da je $x$ \df{vezana spremenljivka}, kar pomeni, da je veljavna
le v funkcijskem predpisu, nanj je vezana, in da ni pomembno, s katerim simbolom jo
označimo. Tako sta funkcijska predpisa
%
\begin{equation*}
  x \mapsto 1 + x^2
  \qquad\text{in}\qquad
  a \mapsto 1 + a^2
\end{equation*}
%
enaka in lahko bi celo pisali $\Box \mapsto 1 + \Box^2$ ali
$\heartsuit \mapsto 1 + \heartsuit^2$.

V funkcijskem predpisu mora na levi stati en sam simbol, ki na desni kaže, kam je treba
vstaviti element iz domene. Tako
%
\begin{equation*}
  \sin(x) \mapsto \cos(2 x),
  \qquad
  3 + 2 \mapsto 5
  \qquad\text{in}\qquad
  \sin(x) \mapsto 2 \cdot \sin(x)
\end{equation*}
%
\emph{niso} veljavni funkcijski predpisi.

Seveda dopuščamo možnost, da se vezana spremenljivka pojavi enkrat, večkrat ali sploh ne.
Funkcijska predpisa
%
%
\begin{equation*}
  x \mapsto 42
  \qquad\text{in}\qquad
  x \mapsto x \cdot \sin(x)
\end{equation*}
%
sta torej veljavna.

Če želimo preslikavo z danim funkcijskim predpisom poimenovati, na primer $f$, zapišemo
%
\begin{equation*}
  f : x \mapsto 1 + x^2.
\end{equation*}
%
To preberemo ``$f$ slika $x$ v ena plus $x$ na kvadrat.'' Običajna sta tudi zapisa
%
\begin{equation*}
  f(x) = 1 + x^2
  \qquad\text{in}\qquad
  f(x) \dfeq 1 + x^2.
\end{equation*}
%
Funkcijske predpise je podrobno prvi preučeval Alonzo Church,\footnote{Alonzo Church
  (1903--1995) je bil ameriški matematik in logik, ki je pomembno prispeval k razvoju
  logike in teoretičnega računalništva. Njegov študent, Dana Stewarta Scott, je imel
  študenta Marka Petkovška in Andreja Bauerja, slednji pa je imel študenta Davorina
  Lešnika.} ki je uporabljal zapis
%
\begin{equation*}
  \lambda x \,.\, 1 + x^2
\end{equation*}
%
in teorijo funkcijskih predpisov poimenoval \df{$\lambda$-račun}. V logiki se je njegov
zapis obdržal in se uveljavil tudi v programski jezikih:
%
\begin{itemize}
\item v Pythonu pišemo \verb|lambda x : 1+x**2|,
\item v Haskellu pišemo \verb|\x -> 1+x**2| in
\item v OCamlu pišemo \verb|fun x => 1+x*x|.
\end{itemize}
%
Predvsem v programiranju funkcijskim predpisom pravijo tudi \df{anonimne} ali \df{brezimne
  preslikave}.

Nekateri starejši zapisi funkcijskih predpisov so slabi, a jih ljudje vztrajno
uporabljajo. Opozorimo le na en slab zapis, ki povzroča precej preglavic, ne da bi se
matematiki tega zares zavedali. Funkcijski predpis mora določati vezano spremenljivko,
sicer ne vemo, kako vstaviti vrednosti, a na žalost jo matematiki pogosto izpustijo skupaj
z $\mapsto$, da ostane samo izraz na desni.
%
Težava je v tem, da se lahko v funkcijskem predpisu pojavi več kot en simbol. Če vam na primer povem, da imam v mislih funkcijski predpis
%
\begin{equation*}
  a \cdot x + b
\end{equation*}
%
boste vsi mislili, da je mišljeno $x \mapsto a \cdot x + b$. A pravzprav bi lahko bilo
tudi $a \mapsto a \cdot x + b$ ali $b \mapsto a \cdot x + b$ ali celo
$t \mapsto a \cdot x + b$! Namreč, nič ni narobe s funkcijskim predpisom, v katerem se
pojavijo dodatni simboli.

Morda pa lahko vezano spremenljivko in $\mapsto$ brez škode izpustimo, če v izrazu nastopa
samo en simbol, denimo $1 + x^2?$
%
A spet bi zabredli v težave. Je $42$ število ali funkcijski predpis $x \mapsto 42$? Je
$1 + x^2$ funkcijski predpis $x \mapsto 1 + x^2$ ali $a \mapsto 1 + x^2$?

Velikokrat površno rečemo, da funkcijski predpis podaja preslikavo. To ni res, saj smo že
prej povedali, da ima vsaka preslikava tri sestavne dele: domeno, kodomeno in prirejanje.
Res, če ne poznamo domene, ne moremo preveriti, ali je funkcijski predpis celovit. Denimo,
funkcijski predpis
%
\begin{equation*}
  x \mapsto \frac{x}{x^2 - 2}
\end{equation*}
%
ni celovit, če je domena množica realnih števil, in je celovit, če je domena množica
racionalnih števil. Tudi kodomeno moramo poznati, sicer ne moremo določiti nekaterih
lastnosti preslikave, kot je na primer surjektivnost, glej
razdelek~\ref{razdelek:injektivnost-in-surjektivnost}.



\subsection{Ostali načini podajanja preslikav}
\label{sec:ostali-predpisi}

Funkcijski predpisi niso edini način za podajanje prirejanja, zato omenimo še nekatere
druge.

Preslikavo s končno domeno lahko podamo s tabelo, na primer:
%
\begin{center}
  $f : \set{1, 2, 3, 5} \to \set{10, 20, 30}$

  \medskip

  \begin{tabular}{|c|c|} \hline
    1 & 10 \\ \hline
    2 & 10 \\ \hline
    3 & 20 \\ \hline
    5 & 10 \\ \hline
  \end{tabular}
\end{center}
%
To seveda pomeni, da $f$ elementu $1$ priredi $10$, $2$ priredi $10$, $3$ priredi $20$ in $5$
priredi $10$. Tabelo lahko predstavimo na različne načine, lahko kar naštejemo vsa prirejanja:
%
\begin{align*}
  f(1) &= 10 \\
  f(2) &= 10 \\
  f(3) &= 20 \\
  f(5) &= 10.
\end{align*}
%
Tudi
%
\begin{align*}
  1 &\mapsto 10 \\
  2 &\mapsto 10 \\
  3 &\mapsto 20 \\
  5 &\mapsto 10.
\end{align*}
%
je še vedno le tabela, ki prikazuje prirejanje. Ne sme nas motiti dejstvo, da smo
$\mapsto$ uporabili za naštevanje prirejanj, namesto za funkcijski prdpis.

Preslikava je lahko določena tudi z opisom računskega postopka, pravimo mu \df{algoritem},
s pomočjo katerega izračunamo vrednost preslikave pri danem argumentu. Paziti moramo, da je
opis postopka res natančen in nedvoumen, lahko ga kar zapišemo kot program. Teoretični
računalničar bi pripomnil, da je treba pri tem izbrati programski jezik, ki ima ustrezno
matematično definicijo.

Preslikave lahko podamo tudi tako, da opišemo pogoje, pri katerih je element kodomene
prirejen elementu domene. Na primer, preslikavo $f : \NN \to \ZZ$ bi lahko definirali z
zahtevo, da naravnemu številu $n \in \NN$ priredimo celo število $k \in \ZZ$, kadar velja
%
\begin{equation*}
  k^2 \leq n < (k+1)^2.
\end{equation*}
%
To prirejanje je veljavno, če je celovito in enolično, česar ne bomo preverjali, lahko pa
poskusite sami. Nekaj prirejanj $f$ prikazuje naslednja razpredelnica:
%
\begin{align*}
0 &\mapsto 0   &   4 &\mapsto 2   &    8  &\mapsto 2   &   12 &\mapsto 3 \\
1 &\mapsto 1   &   5 &\mapsto 2   &    9  &\mapsto 3   &   13 &\mapsto 3 \\
2 &\mapsto 1   &   6 &\mapsto 2   &    10 &\mapsto 3   &   14 &\mapsto 3 \\
3 &\mapsto 1   &   7 &\mapsto 2   &    11 &\mapsto 3   &   15 &\mapsto 3
\end{align*}
%
Ali znate z besedami opisati preslikavo~$f$?

V splošnem je lahko preslikava podana s precej zapleteno konstrukcijo, ki zahteva veliko
preverjanja in dokazovanja. Osnovne načine podajanja preslikav bomo spoznali skupaj s
konstrukcijami množic.


\subsection{Aplikacija in substitucija}
\label{sec:aplikacija-in-subsitucija}

Do sedaj smo se ukvarjali s tem, kako preslikavo podamo, zdaj pa se vprašajmo, kako lahko
preslikavo uporabimo. Če je $f : X \to Y$ preslikava iz $X$ v $Y$ in je $x \in X$, potem
lahko \df{$f$ uporabimo na $x$} in dobimo \df{vrednost} preslikave~$f$ pri
\df{argumentu}~$x$, to je tisti edini element $Y$, ki ga~$f$ priredi~$x$. Vrednost $f$
pri~$x$ zapišemo
%
\begin{equation*}
  f(x)
  \qquad\text{ali}\qquad
  f\,x
\end{equation*}
%
in preberemo ``$f$ od $x$'' ali ``$f$ pri $x$''. Izraz $f(x)$, oziroma $f\,x$, se imenuje
\df{aplikacija}. Večinoma se uporablja zapis z oklepaji, a ne vedno: navajeni smo pisati
$\ln 2$ in $\sin \alpha$ namesto $\ln(2)$ in $\sin(\alpha)$. Oklepaje izpuščamo tudi v
nekaterih programskih jezikih in občasno v algebri.

V analizi je uveljavljen še en zapis za aplikacijo, ki se uporablja za zaporedja. Namreč,
zaporedje ni nič drugega kot preslikava $a : \NN \to \RR$ iz naravnih v realna števila.
Aplikacijo $a(n)$, ki označuje $n$-ti člen zaporedja, ponavadi pišemo~$a_n$, torej
argument podpišemo.

Preslikavo lahko uporabimo na argumentu tudi, če je nismo poimenovali. Na primer,
preslikavo $\RR \to \RR$, podano s funkcijskim predpisom
%
\begin{equation*}
  x \mapsto 1 + x^2
\end{equation*}
%
uporabimo na argumentu~$3$:
%
\begin{equation*}
  (x \mapsto 1 + x^2)(3).
\end{equation*}
%
Se vam zdi tak zapis nenavaden? Verjetno, a pomislite, zakaj je tako: ker običajno
preslikave poimenujemo in se nanje vedno sklicujemo z njihovim imenom. Prav nobenega
razloga ni, da ne bi s funkcijskimi predpisi delali tako, kot s števili, vektorji in
ostalimi matematičnimi objekti, na katere smo že navajeni. Računalničarji radi rečejo, da
je treba tudi preslikave obravnavati kot ``enakopravne državljane''. Prav imajo, zato
bomo vadili uporabo funkcijskih predpisov ter z njimi delali, kot da niso nič posebnega,
saj niso!

Kako pravzaprav določimo vrednost funkcije pri danem argumentu? To je odvisno od tega,
kako je podano prirejanje. Če imamo tabelarični prikaz, poiščemo argument v levem stolpcu
in pogledamo v desni stolpec. Če je preslikava podana s funkcijskim predpisom, argument
vstavimo v predpis. Na primer, če je $f : \RR \to \RR$ podana s funkcijskim predpisom
%
\begin{equation*}
  f(x) = 1 + x^2,
\end{equation*}
%
potem je vrednost $f(3)$ enaka $1 + 3^2$, kar je seveda enako~$10$, a to zahteva dodaten
račun, ki nas v tem trenutku ne zanima. Pravimo, da smo simbol~$x$ \df{zamenjali} ali
\df{substituirali} s~$3$, oziroma da smo~$3$ \df{vstavili} v~$f$ namesto~$x$. Seveda lahko
vstavimo argument neposredno v funkcijski predpis, zato je aplikacija
%
\begin{equation*}
  (x \mapsto 1 + x^2)(3)
\end{equation*}
%
seveda spet enaka $1 + 3^2$.

Preslikavo smemo uporabiti na poljubnem elementu domene, ki je lahko zapisan na bolj ali
manj zapleten način, pri čemer gre še vedno samo za zamenjavo. Na primer, v zgornjo
preslikavo~$f$ lahko vstavimo $3 + 4$ in dobimo $1 + (3 + 4)^2$ ali pa za neki $u \in \RR$
vstavimo $u + 2$ in dobimo $1 + (u + 2)^2$. V razdelku~\ref{sec:eksponent} bomo spoznali
še dodatna pravila za vstavljanje izrazov, ki se vrtijo okoli vezanih spremenljivk.


\subsection{Načelo ekstenzionalnosti preslikav}

Kot smo že omenili, je možih več pogledov na preslikave. Ali je pomembno, kako učinkovito
računamo vrednosti preslikave? Vsekakor, ampak ali naj to pomeni, da sta preslikavi
različni, če imata enake vrednosti, a je ena podana z učinkovitim pravilom in druga z
neučinkovitim? V matematiki je odgovor nikalen.

\begin{pravilo}[Ekstenzionalnost preslikav]
  Preslikavi sta enaki, če imata enaki domeni in kodomeni ter imata za vse argumente
  enaki vrednosti.
\end{pravilo}

Natančneje, če sta $f : A \to B$ in $g : C \to D$ preslikavi in velja $A = C$, $B = D$ ter
za vsak $x \in A$ velja $f(x) = g(x)$, tedaj velja $f = g$.

Takoj opozorimo na razliko med
%
\begin{equation*}
  f(x) = g(x)
  \qquad\text{in}\qquad
  f = g
\end{equation*}
%
saj bi marsikdo trdil, da med njima ni razlike. Levi izraz pravi, da sta $f(x)$ in $g(x)$
enaka elementa množice $C$, desni pa da sta $f$ in~$g$ enaki preslikavi iz $A$ v $B$. Na
sploh je treba razlikovati med $f$ in $f(x)$, saj to nikakor nista enaka objekta: prvi je
preslikava, drugi pa vrednost te preslikave pri~$x$. Verjetno nihče ne bi trdil, da je
preslikava $\cos$ isto kot $\cos \frac{\pi}{4}$, ali ne? Isti razmislek veleva, da
$\cos x$ ni isto kot $\cos$, če tudi si mislimo, da je $x$ poljuben. Zmeda izhaja iz
neprimernega zapisa preslikav. Če bi že od malih nog pravilno uporabljali funkcijske
predpise, bi seveda vedeli, da načelo ekstenzionalnosti za preslikave zagotavlja enakost
~$\cos$ in $x \mapsto \cos x$, oba pa sta različna od $\cos x$, ki sploh ni preslikava,
ampak neko realno število. Čeprav je število $\cos x$ odvisno od parametra~$x$, je še
vedno le število.

V bran tradicionalnemu zapisu pa moramo vseeno povedati, da se lahko \emph{dogovorimo} za
nekoliko napačen zapis, če to ne povzroča zmede. S tem se izognemu preveč birokratskemu
pisanju nebistvenih podrobnosti in lahko bistveno izboljšamo komunikacijo in razumevanje
med izkušenimi matematiki. A začetnikom priporočamo, da v dobrobit boljšega razumevanja
snovi vsaj na začetku študija raje vztrajajo pri doslednem zapisu.

Vrnimo se še k načelu ekstenzionalnosti preslikav. Ali ni pravzaprav očitno, da sta
preslikavi enaki, če imata enaki domeni, kodomeni in vrednosti? Morda res, a to ni razlog,
da tega ne bi eksplicitno zapisali. Vsak matematik vam ve povedati kako zgodbo o tem,
kako se je v dokazu skrivala napako ravno tam, kjer je bilo nekaj ``očitno''. Poleg tega
pa si lahko predstavljamo razmere, v katerih je smiselno razlikovati med dvema
preslikavama, ki imata vedno enake vrednosti, denimo v programiranju, kjer je učinkovitost
zelo pomembna.

%% STAR MATERIAL OD DAVORINA. Preveriti, kaj od tega je treba dati v besedilo, in kam.

% Množice ne obstajajo ločene ena od druge pač pa so med sabo povezane s
% \df{preslikavami} oziroma s tujko \df{funkcijami}.  Posamična preslikava slika elemente ene
% množice po določenem predpisu v elemente druge množice.

% Če je $f$ preslikava, ki slika iz množice $X$ v množico $Y$, to zapišemo
% %
% \begin{equation*}
%   f : X \to Y.
% \end{equation*}
% %
% Rečemo, da je množica~$X$ \df{začetna množica} ali \df{domena} preslikave~$f$, množica~$Y$
% pa je \df{ciljna množica} ali \df{kodomena} preslikave $f$.


% Običaj je, da predpis preslikave podamo s pomočjo spremenljivke, tipično z oznako $x$. Na primer, če je $f$ preslikava kvadriranja, njen predpis zapišemo kot
% \[f(x) = x^2.\]
% Na tem mestu je potrebno poudariti več reči.
% \begin{itemize}
% \item
% Velikokrat površno rečemo, da zgornji predpis podaja preslikavo. To ni povsem res --- to je zgolj predpis preslikave. Za to, da preslikavo v celoti podamo, je potrebno navesti tri stvari: poleg predpisa še domeno in kodomeno. Vse to je del informacije o preslikavi.

% To se jasno pokaže, če začnemo razmišljati o lastnostih preslikav. Se še spomnite iz srednje šole, kaj pomeni, da je preslikava surjektivna? (Bomo ponovili v razdelku~\ref{razdelek:injektivnost-in-surjektivnost}.) Če vzamemo, da preslikava $f$ zadošča zgornjemu predpisu in jo obravnavamo kot preslikavo $f\colon \RR \to \RR$, ni surjektivna, če jo obravnavamo recimo kot preslikavo $f\colon \RR_{\geq 0} \to \RR_{\geq 0}$, pa je.
% \item
% Za spremenljivko $x$ velja isto, kot smo razpravljali že v prejšnjem razdelku pri lastnostih elementov množic: spremenljivka $x$ nima vnaprej določene vrednosti, pač pa predstavlja mesto, kamor lahko vstavimo poljubno vrednost. Seveda je potem vseeno, če vzamemo kakšno drugo črko ali čisto drug simbol: $f(y) = y^2$ določa isti predpis kot $f(x) = x^2$; prav tako $f(\heartsuit) = \heartsuit^2$. Se pravi, tudi v tem primeru gre za nemo spremenljivko. Če si torej izberemo neko vrednost, jo lahko vstavimo na mesto spremenljivke in izračunamo vrednost dobljenega izraza, npr.~$f(3) = 3^2 = 9$ oziroma $f(2\pi) = (2\pi)^2 = 4\pi^2$. Predstavljajte si, da je spremenljivka pravzaprav škatlica, kamor lahko vstavite vrednost, torej
% \[f(\argbox) = \argbox^2.\]
% \item
% Alternativen način zapisa $f(x) = x^2$ je
% \[f\colon x \mapsto x^2.\]
% Pazimo: navadna puščica $\to$ podaja domeno in kodomeno, kot razloženo zgoraj. Repata puščica $\mapsto$ pa za posamičen element domene pove, v kateri element kodomene se preslika.

% Zapis z repato puščico je še posebej uporaben, kadar želimo podati preslikavo, ne da bi nam bilo potrebno izbrati ime zanjo. Na primer, realno funkcijo kvadriranja lahko v celoti podamo takole:
% \begin{align*}
% \RR &\to \RR \\
% x &\mapsto x^2
% \end{align*}
% (prva vrstica pove domeno in kodomeno, druga pa predpis). Tako podanim preslikavam potem rečemo \df{brezimne preslikave} (s tujko \df{anonimne funkcije}). Kasneje (v razdelku~\ref{razdelek:brezimne-preslikave}) bomo spoznali bolj strnjen zapis takih preslikav, ki je še posebej primeren za izvajanje operacij med preslikavami; takrat bomo takšno funkcijo zapisali kot $\lam{x \in \RR} x^2$.
% \end{itemize}

% \note{Sklop (kompozicija, kompozitum) preslikav. Identiteta kot enota za sklapljanje. Razčlenitev (dekompozicija, faktorizacija) preslikav.}

% \davorin{Definirati moramo tudi oznako $\set{f(x)}{x \in X}$, kar je druge vrste oznaka kot prej definirana $\set{x \in X}{\phi(x)}$. Se gremo primerjavo s Pythonom (razlika med \texttt{\{f(x) for x in X\}} in \texttt{\{x if phi(x)\}})? Smo matematični hipsterji in uvedemo oznako $\{f(x) \,|\, x \in X \,|\, \phi(x)\}$, ki ustreza \texttt{\{f(x) for x in X if phi(x)\}}, kar bi tudi prišlo prav?}

% Zaenkrat smo imeli primere, ko je bil prepis preslikave dan z eno samo spremenljivko, npr.~$f(x) = x^2$. Zelo pogoste so pa tudi \df{preslikave več spremenljivk}, npr.~$f(x, y) = x^2 + y^2$. Že osnovne računske operacije so take --- na primer, pri seštevanju vzamemo \emph{dva} podatka in vrnemo rezultat (vsoto).

% V takem primeru je smiselno reči: domena preslikave sestoji iz \df{dvojic} ali \df{parov} števil. Pri seštevanju je to, katero število je prvo, katero pa drugo, sicer nepomembno, pri kakšni drugi operaciji (npr.~že odštevanju), pa je, zato posebej zahtevajmo: gre za \df{urejene dvojice} (\df{pare}). Urejeno dvojico elementov $a$ in $b$ (v tem vrstem redu) po dogovoru zapišemo kot $(a, b)$. Vrednosti $a$ in $b$ imenujemo \df{komponenti} tega para; natančneje, $a$ je \df{prva komponenta}, $b$ pa \df{druga komponenta}.

% Če imamo dve množici $A$ in $B$, tedaj množico vseh urejenih dvojic, katerih prva komponenta je element iz $A$, druga komponenta pa element iz $B$, označimo $A \times B$ in imenujemo \df{zmnožek} ali \df{produkt} množic $A$ in $B$. Glede na to, da obstaja mnogo operacij, ki se imenujejo ``produkt'' (poznate že vsaj produkt števil, produkt števila z vektorjem, skalarni produkt vektorjev in vektorski produkt vektorjev, obstaja pa jih še precej več), je koristno produkt množic posebej poimenovati, da ga ločimo od drugih: zanj se je uveljavil izraz \df{kartezični produkt} (izhaja iz imena Cartesius, tj.~latinske različice priimka Renéja Descarta\footnote{René Descartes (1596 -- 1650) je bil francoski filozof, matematik in znanstvenik.}).

% Seštevanje potemtakem lahko razumemo kot preslikavo $+\colon \RR \times \RR \to \RR$. V tem smislu še vedno gre za preslikavo, ki dan vhodni podatek preslika v neki rezultat, le da je vhodni podatek dvojica števil, ne pa zgolj eno število. Kadar imamo produkt več enakih faktorjev, ga lahko (kot običajno) zapišemo v obliki potence; pisali bi lahko tudi $+\colon \RR^2 \to \RR$.

% Seveda nismo omejeni na preslikave samo ene ali dveh spremenljivk. Nič nam ne preprečuje definirati recimo $f(x, y, z) = 2x + y - 3z$. Smiselna domena te preslikave setoji iz \df{urejenih trojic} števil. V splošnem, če jemljemo elemente iz množic $A$, $B$, $C$, tedaj se množica vseh takih trojic označi z $A \times B \times C$. Prejšnji predpis določa potem preslikavo $f\colon \RR \times \RR \times \RR \to \RR$ (oziroma krajše $f\colon \RR^3 \to \RR$).

% Spremenljivk je lahko še več; poleg dvojic in trojic tako dobimo še četverice, peterice, šesterice\ldots V splošnem takšna končna zaporedja elementov imenujemo \df{urejene večterice}. Tudi število spremenljivk je lahko označeno s črko; na primer, preslikava, ki računa povprečje $n$ števil (kjer $n \in \NN_{\geq 1}$), je dana kot
% \begin{align*}
% \RR^n &\to \RR \\
% (x_1, x_2, \ldots, x_n) &\mapsto \frac{x_1 + x_2 + \ldots + x_n}{n}
% \end{align*}
% (če hočemo poudariti, da imajo naše večterice natanko $n$ komponent, jih imenujemo $n$-terice). Nadlega pri tem je sicer spet dvoumnost tropičja. Deloma jo je možno odpraviti tako, da celotno večterico označimo z eno spremenljivko. Pogosta izbira zapisa je $f(\mathbf{x})$ ali $f(\vec{x})$ (razlog za to je, da lahko večterico vidimo kot vektor).

% Marsikdaj želimo delati ne samo z eno preslikavo, pač pa s celo množico preslikav naenkrat. Zato uvedemo: množica vseh preslikav, ki slikajo iz $X$ v $Y$, se označi kot $Y^X$; temu se reče \df{eksponent} množic $X$ in $Y$ (\note{na primernem mestu kasneje} bomo razložili, od kod ta oznaka).

% \begin{zgled}
% Množico vseh preslikav, ki realna števila slikajo nazaj v realna števila, označimo z $\RR^\RR$. Če nas zanimajo realne preslikave, ki so definirana samo na intervalu $\intoo{-1}{1}$, opazujemo množico $\RR^{\intoo{-1}{1}}$. Definiramo lahko preslikavo
% \begin{align*}
% \RR^{\intoo{-1}{1}} &\to \RR \\
% f &\mapsto f(0),
% \end{align*}
% ki preslikavam priredi njihovo vrednost v točki $0$. Ta preslikava torej ima za argumente (tj.~vnose) celotne preslikave in ne števila! Sama po sebi je element množice $\RR^{\RR^{\intoo{-1}{1}}}$.
% \end{zgled}

% \begin{zgled}
% Za poljubne množice $A$, $B$, $C$ lahko definiramo sledečo preslikavo, katere argumenti so pari preslikav.
% \begin{align*}
% B^A \times C^B &\to C^A \\
% (f, g) &\mapsto g \circ f
% \end{align*}
% \end{zgled}


% \davorin{Glede na to, da gre za slovenski učbenik, dajem izrazu `preslikava' prednost pred izrazom `funkcija'. Seveda pa sem pojasnil tudi slednji izraz (v prvem poglavju).}

% \note{Uvod. Definicijsko območje in zaloga vrednosti \davorin{morda dodamo kot možno ime za zalogo vrednosti še prevod angleške besede `range', se pravi `razpon'?}. Zožitve (tako domene kot kodomene); oznake za to so $\rstr{f}_A$, $\rstr{f}^B$, $\rstr{f}_A^B$. Izvrednotenje (evalvacija) preslikave (če ne bomo tega pojasnili že pri eksponentih množic).}


\section{Zmnožek}
\label{sec:zmnozek}

Množice lahko \df{tvorimo} ali \df{konstruiramo} iz drugih množic na različne načine. V
tem poglavju bomo spoznali tri osnovne konstrukcije, ostale pa kasneje, ko bomo že nekaj
vedeli o logiki. Najprej obravnavajmo zmnožek ali kartezični produkt.

Takoj se zastavi vprašanje, kako sploh opisati novo konstrukcijo množic. Načelo
ekstenzionalnosti pove, da je množica opredeljena s svojimi elementi. Torej moramo
pojasniti, kaj so elementi nove množice, se pravi, kako jih vpeljemo, kaj lahko z njimi
počnemo in kakšne so njihove zakonitosti. Natančneje, novo konstrukcijo množic
določajo naslednja pravila:
%
\begin{enumerate}
\item pravilo \df{tvorbe}, ki vpelje novo množico,
\item pravila \df{vpeljave} elementov, ki podajo operacije, s katerimi gradimo elemente,
\item pravila \df{uporabe}, ki podajo opreacije, s katerimi razgradimo ali uporabimo elemente,
\item \df{enačbe}, ki opredeljujejo zakonitosti, ki veljajo za operacije vpeljave in uporabe.
\end{enumerate}
%
Najbolje je, da si postopek ogledamo na primeru.

\begin{pravilo}[Tvorba zmnožka]
  \label{pravilo:zmnozek-tvorba}
  Za vsaki množici $A$ in $B$ je $A \times B$ množica, ki se imenuje \df{zmnožek} ali
  \df{kartezični produkt} $A$ in $B$.
\end{pravilo}

\noindent
%
Pravilo tvorbe pove, da lahko tvorimo novo množico $A \times B$, ne pove pa, kakšne
elemente ima. To je vsebina naslednjih dveh pravil, ki povesta, kako sestavimo in
razstavimo elemente zmnožka.

\begin{pravilo}[Vpeljava urejenih parov]
  \label{pravilo:zmnozek-vpeljava}
  %
  Za vse $a \in A$ in $b \in B$ je $(a, b) \in A \times B$. Element $(a, b)$ imenujemo
  \df{urejeni par}.
\end{pravilo}

\begin{pravilo}[Uporaba urejenih parov]
  \label{pravilo:zmnozek-uporaba}
    %
  Za vsak $p \in A \times B$ je $\fst(p) \in A$ \df{prva projekcija} in $\snd(p) \in B$
  \df{druga projekcija} elementa~$p$.
\end{pravilo}

Nazadnje podamo še enačbe.

\begin{pravilo}[Računsko pravilo za urejene pare]
  \label{pravilo:zmnozek-racunanje}
  Za vse $a \in A$, $b \in B$ velja $\fst(a, b) = a$ in $\snd(a, b) = b$.
\end{pravilo}

\begin{pravilo}[Ekstenzionalnost urejenih parov]
  \label{pravilo:zmnozek-ekstenzionalnost}
  Za vse $p, q \in A \times B$ velja: če $\fst(p) = \fst(q)$ in $\snd(p) = \snd(q)$,
  potem $p = q$.
\end{pravilo}

\noindent
%
Računsko pravilo se tako imenuje, ker lahko z njim poenostavljamo izraze, drugo pa je
načelo ekstenzionalnosti, ker pravi, da je urejeni par določen s prvo in drugo projekcijo.

Kadar imamo opravka z večimi zmnožki, na primer $A \times B$ in $C \times D$, bi lahko
prišlo do zmede glede projekcij. Takrat jih opremimo še z dodatnimi oznakami množic, da
razločimo projekciji $\fst[A][B] : A \times B \to A$ in $\fst[C][D] : C \times D \to C$,
in podobno za~$\snd$.

Malo bolj naivna konstrukcija zmnožka bi se glasila takole: kartezični produkt
$A \times B$ je množica vseh urejenih parov $(a, b)$, kjer je $a \in A$ in $b \in B$. A
taka konstrukcija ni popolna, saj ne pove, kaj lahko z urejenim parom počnemo. Kako naj
vemo, da iz $(a, b)$ lahko izluščimo $a$ in $b$, in kako preverimo, ali sta dva urejena
para enaka? Če takih zadev ne določimo, bi lahko kdo mislil, da je urejeni par kaka druga
operacija, denimo seštevanje, unija, ali kdovekaj.

Dejstvo, da je vsak element zmnožka množic urejen par, in to celo na en sam način, lahko
dokažemo.

\begin{trditev}
  Naj bosta $A$ in $B$ množici. Za vsak element $p \in A \times B$ obstaja natanko en
  $a \in A$ in natanko en $b \in B$, da velja $p = (a, b)$.
\end{trditev}

\begin{proof}
  Naj bosta $A$ in $B$ množici in $p \in A \times B$. Najprej pokažimo, da $p$ res je enak
  nekemu urejenemu paru, namreč
  %
  \begin{equation*}
    p = (\fst(p), \snd(p)).
  \end{equation*}
  %
  Uporabimo načelo ekstenzionalnosti za pare, ki nam zagotavlja to enačbo, če dokažemo
  %
  \begin{equation*}
    \fst(p) = \fst(\fst(p), \snd(p))
    \qquad\text{in}\qquad
    \snd(p) = \snd(\fst(p), \snd(p)).
  \end{equation*}
  %
  Ti dve enačbi pa veljata, ker sta primerka računskih pravil za pare.

  Preveriti moramo še, da je $(\fst(p), \snd(p))$ edini urejeni par, ki je enak~$p$.
  Povedano z drugimi besedami, dokazati moramo: če je $p = (a, b)$ za neki $a \in A$ in
  $b \in B$, potem velja $a = \fst(p)$ in $b = \snd(p)$. Pa denimo, da bi za neki
  $a \in A$ in $B \in B$ veljalo $p = (a,b)$. Tedaj bi lahko uporabili računska pravila za
  pare in dobili
  %
  \begin{equation*}
    \fst(p) = \fst(a, b) = a
    \qquad\text{in}\qquad
    \snd(p) = \snd(a, b) = b,
  \end{equation*}
  %
  kar smo želeli dokazati.
\end{proof}

Trditev je prikladna, ko želimo podati funkcijsko pravilo za preslikavo, katere domena je
zmnožek množic. Primer take preslikave je
%
\begin{gather*}
  \RR \times \RR \to \RR \\
  p \mapsto \fst(p) + \snd(p)^2 \cdot \fst(p).
\end{gather*}
%
Ta zapis je precej nepregleden, a sledili smo navodilu, da mora stati na levi strani
funkcijskega predpisa simbol. Prejšnja trditev nam zagotavlja, da lahko vsak element
$\RR \times \RR$ na en sam način izrazimo kot urejeni par $(x, y)$, in zato ne bo nič
narobe, če zapišemo ta isti funkcijski predpis bolj pregledno tako, da upoštevamo, da
je $p$ enak $(x, y)$ za enolično določena $x$ in $y$:
%
\begin{gather*}
  \RR \times \RR \to \RR \\
  (x, y) \mapsto x + y^2 \cdot x.
\end{gather*}
%
Če bi funkcijo poimenovali, denimo $f$, bi dobili običajni zapis:
%
\begin{gather*}
  f : \RR \times \RR \to \RR \\
  f(x, y) = x + y^2 \cdot x.
\end{gather*}
%
Za tako preslikavo pravimo, da je ``funkcija dveh spremenljivk'', ker si mislimo, da smo
podali argumenta $x$ in $y$ ločeno drug od drugega. Tu pravzaprav vidimo, da bi lahko
rekli tudi, da je funkcija dveh spremenljivk pravzaprav običajna funkcija, katere
arugmenti so urejeni pari.

Poleg zmnožka dveh množic bi lahko tvorili tudi zmnožek treh ali več množic. Pravila bodo
podobna kot za zmnožek dveh množic, le da bi namesto urejenih parov tvorili \df{urejene
  večterice} in da bi imeli več projekcij. Za vsako projekcijo bi zapisali eno računsko
pravilo, princip ekstenzionalnosti pa bi bil tudi podoben tistemu za urejene pare.
Podorobnosti prepustimo za vajo.


\section{Vsota}
\label{sec:vsota}

Spoznali smo že unijo $A \cup B$ množic $A$ in $B$, ki vsebuje tiste elemente, ki so v $A$
ali v $B$. Če imata $A$ in $B$ skupne elemente, bodo ti v uniji seveda nastopili samo
enkrat. V skranjem primeru dobimo $A \cup A = A$. Včasih pa želimo združiti množici tako,
da ne pride do prekrivanja. Taka konstrukcija je \df{vsota} $A + B$ množic $A$ in $B$.
Prekrivanje preprečimo tako, da elemente, ki jih je prispevala~$A$ označimo z eno oznako,
tiste, ki jih je prispevala~$B$, pa z drugo.

\begin{pravilo}[Vsota]
  \label{vsota:tvorba}
  Za vsaki množici $A$ in $B$ je $A + B$ množica, ki se imenuje \df{vsota} ali
  \df{koprodukt} množic $A$ in $B$.
\end{pravilo}

\begin{pravilo}[Vpeljava elementov vsote]
  \label{vsota:vpeljava}
  Za vsaki množici $A$ in $B$ velja:
  %
  \begin{enumerate}
  \item za vsak $a \in A$ je $\inl(a) \in A + B$,
  \item za vsak $b \in B$ je $\inr(b) \in A + B$.
  \end{enumerate}
\end{pravilo}

S pravilom vpeljave smo pojasnili, da uporabljamo oznaki $\inl$ in $\inr$, prvo za
elemente iz~$A$ in drugo za elemente iz~$B$. Oznakama pravimo tudi
\df{injekciji}\footnote{Pravzaprav niti ni pomembno, kako poimenujemo oznaki, da sta le
  različni. V funkcijskem programiranju, kjer poznamo vsote podatkovnih tipov, programer
  sam določi, kakšne oznake bo uporabljal za injekcije.} in sta preslikavi
%
\begin{equation*}
  \iota_1 : A \to A + B
  \qquad\text{and}\qquad
  \iota_2 : B \to A + B.
\end{equation*}
%
Kadar imamo opravka z večimi vsotami, na primer $A + B$ in $C + D$, bi lahko prišlo do
zmede glede oznak. Takrat injekcije opremimo še z dodatnimi oznakami množic, da razločimo
injekciji $\inl[A][B] : A \to A + B$ in $\inl[C][D] : C \to C + D$, in podobno za~$\inr$.

Potrebujemo še pravili za uporabo in enakost elementov vsote, ki ju združimo v eno samo
pravilo.

\begin{pravilo}
  \label{vsota:uporaba}
  Za vsaki množici $A$ in $B$ in za vsak $u \in A + B$, bodisi obstaja natanko en
  $a \in A$, da je $u = \inl(a)$, bodisi obstaja natanko en $b \in B$, da je
  $u = \inr(b)$.
\end{pravilo}

Fraza ``bodisi \dots bodisi'' pomeni, da je vsak element $u \in A + B$ enak $\inl(a)$
za natanko en $a \in A$ ali $\inr(b)$ za natanko en $b \in B$, ne more pa se zgoditi oboje
hkrati ali nič od tega. Torej $\inl(a) = \inr(b)$ ne drži in celo v primeru, ko je
$A = B$ in $a = b$, je $\inl(a) \neq \inr(b)$. S tem smo v $A + B$ res ločili elemente $A$
od elementov $B$.
%
Fraza ``natanko en'' pove, da iz $u = \inl(a_1)$ in $u = \inl(a_2)$ sledi $a_1 = a_2$.
Povedano drugače, če velja $\inl(a_1) = \inl(a_2)$, potem je $a_1 = a_2$. Podobno iz
$\inr(b_1) = \inr(b_2)$ sledi $b_1 = b_2$.
%
Podajmo prepost primer, ki verjetno marsikaj pojasni:
%
\begin{equation*}
  \set{a, b, c} + \set{a, d, e} =
  \set{\inl(a), \inl(b), \inl(c), \inr(a), \inr(d), \inr(e)}.
\end{equation*}

Kako definiramo preslikavo $A + B \to C$? Ker je vsak element domene $A + B$ bodisi
$\inl(a)$ za neki $a \in A$ bodisi $\inr(b)$ za neki $b \in B$, \emph{obravnavamo oba
  primera}. Tako funkcijski zapis za preslikavo $A + B \to C$ zapišemo kot
%
\begin{equation*}
  u \mapsto
  \begin{cases}
    \cdots a \cdots & \text{če $u = \inl(a)$,}\\
    \cdots b \cdots & \text{če $u = \inr(b)$,}
  \end{cases}
\end{equation*}
%
kjer smemo v $\cdots a \cdots$ zapisati izraz, ki vsebuje simbol~$a$, in v
$\cdots b \cdots$ izraz, ki vsebuje simbol~$b$. Ker je tak zapis nekoliko neroden, se
dogovorimo, da ga lahko zapišemo tudi s \emph{večdelnim} funkcijskim predpisom:
%
\begin{align*}
  \inl(a) &\mapsto \cdots a \cdots, \\
  \inr(b) &\mapsto \cdots b \cdots.
\end{align*}
%
Če želimo preslikavo poimenovati, zapišemo
%
\begin{align*}
  f &: A + B \to C, \\
  f(\inl(a)) &= \cdots a \cdots \\
  f(\inr(b)) &= \cdots b \cdots.
\end{align*}
%
Vsi ti zapisi res določajo celovito in enolično prirejanje, saj nam pravila za vsoto
zagotavljajo, da vedno obvelja natanko en primer. Na sploh lahko podamo funkcijski zapis z
večimi primeri, če le pazimo, da obravnavamo vse možnosti, in da se le-te ne prekrivajo.
Na primer, predpis
%
\begin{align*}
  (A + B) \times C &\to B + A \\
  (\inl[A][B](a), c) &\mapsto \inr[B][A](a) \\
  (\inr[A][B](b), c) &\mapsto \inl[B][A](b)
\end{align*}
%
je celovit in enoličen, medtem ko predpis
%
\begin{align*}
  (A \times A) + B &\to A \\
  \inl(a_1, a_2) &\mapsto a_2
\end{align*}
%
ni veljaven, ker ni celovit, saj manjka primer $\inr(b) \mapsto \cdots$.

Poleg vsote dveh množic bi lahko tvorili zmnožek treh ali več množic. Pravila bi bila
podobna, le da bi imeli več injekcij in več primerov.

\section{Eksponent}
\label{sec:eksponent}

Denimo, da sta $A$ in $B$ množici. Tedaj lahko obravnavamo preslikave
%
\begin{equation*}
  A \to B
\end{equation*}
%
z domeno $A$ in kodomeno $B$. Ali vse take preslikave tvorijo množico? Russellov paradoks
nas je izučil, da moramo pazljivo postaviti pravila za konstrukcije množic, nato pa jih
strogo držati. Pravila, ki smo jih podali do sedaj, ne zagotavljajo knostrukcije množic
vseh preslikav iz $A$ v $B$. Potrebujemo novo pravilo.

\begin{pravilo}[Eksponent]
  Za vsaki množici $A$ in $B$ ima \df{eksponent} ali \df{eksponentna množica $B^A$} za
  elemente natanko vse preslikave iz~$A$ v~$B$.
\end{pravilo}

Potemtakem je zapis $f : A \to B$ enakovreden zapisu $f \in B^A$.

Pravila, ki opredeljujejo elemente množice $B^A$ smo že spoznali. Pravilo vpeljave pravi,
da je preslikava podana z domeno, kodomeno ter celovitim in enoličnim prirejanjem med
njima. Pravilo uporabe je kar aplikacija: če je $f \in B^A$ in $a \in A$, lahko tvorimo
$f(a) \in B$. Tudi računsko pravilo za preslikave smo že spoznali, saj je to kar pravilo
zamenjave: funkcijski predpis uporabimo na argumentu tako, da vezano spremenljivko v
predpisu zamenjamo z argumentom. In ekstenzionalnost preslikav pove, kdaj sta dve
preslikavi enaki.

Preslikavi, ki sprejme kot argument preslikavo, pravimo \df{funkcional} ali \df{preslikava
  višjega reda}. Primer take preslikave je \df{kompozicija}:
%
\begin{align*}
  {\circ} &: C^B \times B^A \to C^A \\
  {\circ} &: (g, f) \mapsto (x \mapsto g(f(x))).
\end{align*}
%
Pišemo jo kot operacijo, torej $g \circ f$ namesto ${\circ}(g, f)$. V zgornjem zapisu smo
uporabili eksponente, a v tem primeru je bolj pregleden diagram:
%
\begin{equation*}
  \xymatrix{
    {A}
    \ar[r]^{f}
    \ar@/_2ex/[rr]_{g \circ f}
    &
    {B}
    \ar[r]^{g}
    &
    {C}
  }
\end{equation*}
%
Zakaj smo $\circ$ definirali tako, da kompozicijo $f$ in $g$ pišemo $g \circ f$ namesto $f \circ g$? Ker si je mnogo lažje zapomniti računsko pravilo
%
\begin{equation*}
  (g \circ f)(x) = g(f(x)),
\end{equation*}
%
ki velja z našo definicijo, kot pa $(f \circ g)(x) = g(f(x))$, kar bi veljalo, če bi
zamenjali vlogi~$f$ in~$g$.

Kompozicijo smo zapisali z \emph{vgnezdenim} funkcijskim predpisom, ki argumentu priredi
preslikavo, ki je spet podana s funkcijskim predpisom. V splošnem je vgnezdeni funkcijski
predpis oblike
%
\begin{align*}
  A &\mapsto C^B \\
  a &\mapsto (b \mapsto \cdots),
\end{align*}
%
kjer se lahko v $\cdots$ pojavita~$a$ in~$b$. Na tak zapis se je treba navaditi, a je zelo
prikladen, še posebej v funkcijskem programiranju. V matematiki ni zelo pogost, a mi se ga
ne bomo bali.

Pri računanju s preslikavami višjega reda včasih hkrati obravnavamo več funkicjskih
predpisov in lahko pride do zmede, če za vse uporabimo isto vezano spremenljivko. Na
primer, kompozitum preslikav
%
\begin{equation*}
  \begin{aligned}
    \RR &\to \RR \\
    x &\mapsto x^2 - 4
  \end{aligned}
  %
  \qquad\text{in}\qquad
  %
  \begin{aligned}
    \RR &\to \RR \\
    x &\mapsto 2 - x
  \end{aligned}
\end{equation*}
%
bi lahko izračunali takole:
%
\begin{align*}
  (x \mapsto x^2  - 4) \circ (x \mapsto 2 - x)
  &= (x \mapsto (x \mapsto x^2 - 4) ((x \mapsto 2 - x) x)) \\
  &= (x \mapsto (x \mapsto x^2 - 4) (2 - x)) \\
  &= (x \mapsto (2 - x)^2 - 4) \\
  &= (x \mapsto x^2 - 4 x).
\end{align*}
%
Tu imamo tri pojavitve $x$, ki bi jih morali ločiti, ker vsaka nastopa kot vezana
spremenljivka v svojem funkcijskem predpisu. Še posebej nejasen je računski korak
$(x \mapsto (x \mapsto x^2 - 4) (2 - x)) = (x \mapsto (2 - x)^2 - 4)$, ko vezano
spremenljivko~$x$ v funkcijskem predpisu zamenjamo z izrazom $2 - x$, ki tudi vsebuje~$x$.
To sta dva različna $x$-a! Spomnimo se, da lahko vezane spremenljivke vedno preminujemo.
Ponovimo račun, a tokrat tako, da imajo različni funkcijski predpisi različne vezane
spremenljivke. Kompozitum
%
\begin{equation*}
  \begin{aligned}
    \RR &\to \RR \\
    y &\mapsto y^2 - 4
  \end{aligned}
  %
  \qquad\text{in}\qquad
  %
  \begin{aligned}
    \RR &\to \RR \\
    z &\mapsto 2 - z
  \end{aligned}
\end{equation*}
%
izračunamo takole:
%
\begin{align*}
  (y \mapsto y^2  - 4) \circ (z \mapsto 2 - z)
  &= (x \mapsto (y \mapsto y^2 - 4) ((z \mapsto 2 - z) x)) \\
  &= (x \mapsto (y \mapsto y^2 - 4) (2 - x)) \\
  &= (x \mapsto (2 - x)^2 - 4) \\
  &= (x \mapsto x^2 - 4 x).
\end{align*}
%
To je dosti bolj pregledno. Da ne bo prihajalo do zapletov z vezanimi spremenljivkami, se
dogovorimo: \emph{kadar imamo opravka z večimi vezanimi spremenljivkami, jih vedno
  preimenujemo tako, da so med seboj različne.}

Funkcionale srečamo v analizi in funkcijskem programiranju. Limita zaporedja je
funkcional, ker sprejme kot argument zaporedje realnih števil, se pravi element $\RR^\NN$,
in mu priredi realno število. Odvod je funkcional, ki sprejme element $\RR^\RR$ in mu
priredi element $\RR^\RR$. Če smo povsem natančni, limita kot preslikava $\RR^\NN \to \RR$
ni celovit funkcional, ker nekatera zaporedja ne konvergirajo. Prav tako odvod kot
preslikave $\RR^\RR \to \RR^\RR$ ni celovit, ker nekatere preslikave niso odvedljive.
Preslikavam, ki niso celovite, pravimo \emph{delne} in o njih bomo več povedali v
razdelku~\ref{delne-preslikve}.

\section{Aritmetika množic}
\label{sec:aritmetika-mnozic}

Ko otrok prvič spozna pojem števila, je ta zanimiv sam po sebi. Z vnemo šteje do sto in se
rad pogovarja se o tem, koliko je en miljon. Sčasoma se radovednost osredotoči na
aritmetične operacije in, če ima mladenič ali mladenka v sebi matematično žilico, na
\emph{zakonitosti} števil: množenje z~$1$ nima učinka, vrstni red seštevanja ni pomemben
itd. Ali tudi operacijam na množicah, ki smo jih spoznali do sedaj, vladajo kakšne
zakonitosti?

\subsection{Izomorfizem množic}
\label{sec:izomorfizem-mnozic}

Za števili $a$ in $b$ velja $a \cdot b = b \cdot a$. Nekaj podobnega velja tudi za množici
$A$ in $B$ in njuna zmnožka $A \times B$ in $B \times A$. V splošnem sicer nista enaka, a
sta v nekem smislu enakovredna, ker lahko par $(x, y) \in A \times B$ pretvorimo v par
$(y, x) \in B \times A$ in obratno. Ta razmislek vodi do pojma izomorfizma.

% Pojasnilo: izomorfnost $A \ism B$ je struktura, ki je naravno podana z dvema
% preslikavama $A \to B$ in $B \to A$ ter dvema enačbama med njima. Zato tu zapišemo
% definicijo, ki hkrati uvede vse te pojme.

\begin{definicija}
  Množici $A$ in $B$ sta \df{izomorfni} in pišemo $A \ism B$, kadar obstajata preslikavi
  %
  \begin{equation*}
    f : A \to B
    \qquad\text{in}\qquad
    g : B \to A,
  \end{equation*}
  %
  za kateri velja
  %
  \begin{equation*}
    g \circ f = \id[A]
    \qquad\text{in}\qquad
    f \circ g = \id[B].
  \end{equation*}
  %
  Pravimo, da je~$f$ \df{izomorfizem} med~$A$ in~$B$ in da je~$g$ \df{inverz} ali
  \df{obrat}~$f$.
\end{definicija}

Preverimo, da velja $A \times B \cong B \times A$ za poljubni množici $A$ in $B$. To
storimo tako, da zapišemo preslikavi med zmnožkoma in preverimo, da tvorita
izomorfizem:\footnote{Držimo se pravila, da nikoli ne uporabimo iste vezane spremenljivke
  dvakrat, zato pravilo za $f$ zapišemo z $x$ in $y$ in pravilo za $g$ z $v$ in $u$.
  Marsikdo bi oba funkcisjka predpisa zapisal z $x$ in $y$, torej
  $f : (x, y) \mapsto (y, x)$ in $g : (y, x) \mapsto (x, y)$. To zmede nekatere študente,
  ker mislijo, da ``sta je $x$ v definiciji $f$ isti kot v definiciji $g$'', karkoli že
  naj bi to pomenilo. Poudarimo še enkat: vezana spremenljivka v funkcijskem predpisu nima
  nikakršne zveze z nobeno drugo pojavitvijo iste spremenljivke kje druge.}
%
\begin{align*}
  f &: A \times B \to B \times A &
  g &: B \times A \to A \times B \\
  f &: (x, y) \mapsto (y, x) &
  g &: (v, u) \mapsto (u, v).
\end{align*}
%
Treba je preveriti, da velja $g \circ f = \id[A \times B]$ in
$f \circ g = \id[B \times A]$. To naredimo z uporabo ekstenzionalnosti preslikav, ki pravi
da $g \circ f = \id[A]$ velja, če velja $(g \circ f)(a,b) = \id[A](a,b)$ za vse $a \in A$
in $b \in B$, in podobno za $f \circ g$. Obravnavajmo torej poljubna $a \in A$ in
$b \in B$ in izračunajmo:
%
\begin{equation*}
  (g \circ f)(a, b) =
  g (f (a, b)) = g (b, a) = (a, b).
\end{equation*}
%
Na podoben način preverimo $f \circ g = \id[B \times A]$.

\begin{zgled}\label{zgled:logaritmiranje-je-obratno-od-eksponenciranja}
  Nekatere druge primere izmorfizma že poznamo iz srednje šole. Denimo, naj bo $\RR$
  množica vseh realnih števil (glej razdelek~\ref{sec:realna-stevila}) in $\RR_{>0}$
  množica vseh pozitivnih realnih števil. Tedaj logaritem in eksponentna funkcija,
  %
  \begin{equation*}
    \log : \RR_{>0} \to \RR
    \qquad\text{in}\qquad
    \exp : \RR \to \RR_{>0}
  \end{equation*}
  %
  tvorita izomorfizem, saj za vse $x \in \RR$ velja $\log (\exp x) = x$ in za vse
  $y \in \RR_{>0}$ velja $\exp (\log y) = y$.
\end{zgled}

Dokažimo nekaj osnovih lastnosti izmorfnosti in izomorizmov.Tokrat ne bomo zapisali
podrobnih dokazov. Za vajo jih dopolnite do tolikšnih podrobnosti, da boste sami sebe
prepričali, da trditve držijo.

\begin{trditev}
  Če je $f : A \to B$ izomorfizem med množicama $A$ in $B$ ter sta preslikavi
  $g : B \to A$ in $h : B \to A$ obe obrata~$f$, potem je $g = h$.
\end{trditev}

\begin{proof}
  Ker je $g$ obrat $f$, velja
  %
  \begin{equation*}
    g \circ f = \id[A]
    \qquad\text{in}\qquad
    f \circ g = \id[B],
  \end{equation*}
  %
  in ker je $h$ obrat $f$, velja
  %
  \begin{equation*}
    h \circ f = \id[A]
    \qquad\text{in}\qquad
    f \circ h = \id[B].
  \end{equation*}
  %
  Dokazati moramo, da iz teh štirih predpostavk sledi $g = h$, kar storimo z naslednjim
  računom:
  %
  \begin{align*}
    g
    &= \id[A] \circ g \tag{kompozicija z $\id[A]$ nima učinka} \\
    &= (h \circ f) \circ g \tag{predpostavka $h \circ f = \id[A]$} \\
    &= h \circ (f \circ g) \tag{kompozicija je asociativna} \\
    &= h \circ \id[B] \tag{predpostavka $f \circ g = \id[B]$} \\
    &= h. \tag{kompozicija z $\id[B]$ nima učinka}
  \end{align*}
\end{proof}

Če je $f : A \to B$ izomorfizem, potem ima natanko en obrat, ki ga označimo $\inv{f}$. Če
$f$ ni izomorfizem, zapis $f^{-1}$ ni veljaven izraz.

Oznaka za obrat je nekoliko nerodna, ker se prekriva z zapisom za obratno vrednost
števila: če je $x \in \RR$ neničelno realno število, potem je $\inv{x}$ tisto realno
število, za katerega velja $x \cdot \inv{x} = 1$. Torej moramo paziti: če je
$f : \RR \to \RR$ izomorfizem in $x \in \RR$, je $\inv{(f(x))}$ obrat števila $f(x)$,
medtem ko je $\inv{f}(x)$ število, ki ga dobimo, ko obrat preslikave $f$ uporabimo na~$x$.
Sami premislite, kaj je $\inv{(\inv{f}(x))}$.

\begin{vaja}
  Podajte primer izomorfizma $f : \RR \to \RR$ in števila $x \in \RR$, da velja
  $\inv{f}(x) = \inv{(f(x))}$. Nato podajte še primer, ko velja
  $\inv{f}(x) \neq \inv{(f(x))}$.
\end{vaja}

\begin{vaja}
  Ozrimo se še enkrat na dokaz prejšnje trditve. Ali smo uporabili vse štiri predpostavke?
  Zapišite \emph{bolj splošno trditev}, se pravi tako, ki navede samo tiste predpostavke,
  ki jih res potrebujemo v dokazu.
\end{vaja}

\begin{trditev}
  Za vse izmorfizme $f : A \to B$ in $g : B \to C$ velja
  %
  \begin{equation*}
    \inv{(\inv{f})} = f
    \qquad\text{in}\qquad
    \inv{(g \circ f)} = \inv{f} \circ \inv{g}.
  \end{equation*}
\end{trditev}

\begin{proof}
  Dokaz prepuščamo za vajo. Pozor, v desni enakosti se je zamenjal vrstni red $f$ in $g$!
  Nadalje opazimo še to: zapisali smo $\inv{(\inf{f})}$ in $\inv{(g \circ f)}$, ne da bi
  predhodno preverili, ali sta $\inv{f}$ in $g \circ f$ izomorfizma. Torej morate v dokazu
  preveriti še, da je ovrat inverza izomorfizem in da je kompozitum izomorfizmov spet
  izomorfizem.
\end{proof}

\begin{trditev}
  Za vse množice $A$, $B$ in $C$ velja:
  %
  \begin{enumerate}
  \item $A \ism A$,
  \item če $A \ism B$, potem $B \ism A$,
  \item če $A \ism B$ in $B \ism C$, potem $A \ism C$.
  \end{enumerate}
\end{trditev}

\begin{proof}
  \parbox{0pt}{}
  %
  \begin{enumerate}
  \item $\id[A]$ je izomorfizem iz $A$ v $A$, ki je sam svoj obrat,
  \item če je $f : A \to B$ izomorfizem iz $A$ v $B$, potem je $\inv{f}$ izomorfizem iz
    $B$ v $A$ in $f$,
  \item če je $f : A \to B$ izomorfizem iz $A$ v $B$ in $g : B \to C$ izomorfizem iz $B$ v
    $C$, potem je $g \circ f$ izomorfizem iz $A \to C$. \qedhere
  \end{enumerate}
\end{proof}

\begin{trditev}
  Preslikava ima največ en inverz.
\end{trditev}

\begin{naloga}
  Pogosto rečemo, da sta seštevanje in odštevanje obratni operaciji. Strogo vzeto, ti dve
  operaciji nista obratni kot preslikavi, saj obe slikata (recimo, da ju gledamo na
  realnih številih) $\RR \times \RR \to \RR$, tj.~ne slikata v nasprotnih smereh. Ugotovi,
  v kakšnem smislu točno sta seštevanje in odštevanje obratni, tj.~kateri dve preslikavi
  sta pravzaprav druga drugi obratni.
\end{naloga}

% TODO Izomorfnost je kongruenca za produkt, vsoto in eksponent.

\section{Aritmetika Tarskega}
\label{sec:aritmetika-tarskega}

Kot že veste, seštevanje, množenje in potenciranje števil zadoščajo naslednjim arimetičnim
zakonom:
%
\begin{align*}
  a + 0 &= a                   &     a \cdot 1 &= a \\
  a + b &= b + a               &     a \cdot b &= b \cdot a \\
  a + (b + c) &= (a + b) + c   &     a \cdot (b \cdot c) &= (a \cdot b) \cdot c \\[1ex]
  0 \cdot a &= 0                           &   1^a &= 1 \\
  (a + b) \cdot c &= a \cdot c + b \cdot c &   (a \cdot b)^c &= a^c \cdot b^c \\[1ex]
  a^0 &= 1                     &   a^1 &= a \\
  a^{b + c} &= a^b \cdot a^c   &   a^{b \cdot c} &= (a^b)^c \\[1ex]
  0^a &= 0 \quad \text{če $a \neq 0$.}
\end{align*}
%
Že prej smo opazili, da je zakon $a \cdot b = b \cdot a$ podoben izomorfizmu
$A \times B \ism B \times A$. Kaj pa ostali zakoni?

\begin{izrek}
  \label{izrek:aritmetika-tarskega}
  Za vse množice $A$, $B$ in $C$ velja:
  %
  \begin{align*}
    A + \emptyset &= A                   &     A \times \one &= A \\
    A + B &= B + A               &     A \times B &= B \times A \\
    A + (B + C) &= (A + B) + C   &     A \times (B \times C) &= (A \times B) \times C \\[1ex]
    \emptyset \times A &= \emptyset                           &   \one^A &= \one \\
    (A + B) \times C &= A \times C + B \times C &   (A \times B)^C &= A^C \times B^C \\[1ex]
    A^\emptyset &= \one                     &   A^\one &= A \\
    A^{B + C} &= A^B \times A^C   &   A^{B \times C} &= (A^B)^C \\[1ex]
    \emptyset^A &= \emptyset \quad \text{če $A \neq \emptyset$.}
  \end{align*}
\end{izrek}

Izmed vseh zakonov, ki so zapisani v izreku, dokažimo nekaj bolj poučnih, ostale pa
pustimo za vajo.

TODO

\begin{naloga}
  Zapiši še preostale izomorfizme iz izreka~\ref{izrek:aritmetika-tarskega}.
\end{naloga}


% Aritmetični zakoni Tarskega za množice. Podobnost z običajno aritmetiko.

% Pri asociativnosti produkta obravnavamo $A_1 \times A_2 \times \cdots \times A_n$ in
% enojec kot produkt nič množic. Podobno za vsote.

% Tu je treba pojasniti, zakaj pišemo $\unit$ za element $\one$.


% \section{Kar je že Davorin napisal}

% Interval realnih števil podamo s krajiščema intervala v oklepajih --- okrogli oklepaji ( ) označujejo odprtost intervala (krajišče ni vključeno v interval), oglati oklepaji [ ] pa zaprtost (krajišče je vključeno). Tako se npr.~interval realnih števil od $0$ do $1$, ki ne vsebuje krajišč, označi z $(0, 1)$, če jih vsebuje, pa z $[0, 1]$.

% Včasih pridejo prav tudi intervali na drugih množicah kot $\RR$. Zato se dogovorimo, da bomo intervale označevali tako, da podamo množico, ob kateri v indeksu zapišemo krajišči v oklepajih, npr.~$\intco[\NN]{1}{5} = \set{1, 2, 3, 4}$. Realna intervala iz prejšnjega odstavka tako zapišemo kot $\intoo{0}{1}$ in $\intcc{0}{1}$.

% Če interval v katero smer gre v nedogled, preprosto zapišemo množico z ustreznim simbolom za urejenost in krajiščem v indeksu. Na primer, $\RR_{> 0}$ označuje množico pozitivnih realnih števil, $\RR_{\geq 0}$ pa množico nenegativnih realnih števil.

% Primerjave med elementi, kot npr.~pravkar podani $>$ in $\geq$, imenujemo \df{relacije} (podrobneje jih bomo spoznali v poglavju~\ref{poglavje:relacije}). Zgornji zapis bomo uporabljali tudi za druge vrste relacij, ne samo za relacije urejenosti. Na primer, množico vseh neničelnih realnih števil zapišemo kot $\RR_{\neq 0}$.

% \davorin{To bi vsaj bil moj predlog. Na ta način se izognemo dvoumnostim (kar je namen). Na primer, kaj pomeni $\forall\, a > 0$? Če zapišemo $\forall\, a \in \NN_{> 0}$ ali $\forall\, a \in \RR_{> 0}$, je jasno. Razlog, da matematiki ``goljufajo'' in pridejo skozi brez tega, je (napol dogovorjena in ponotranjena, ampak arbitrarna) izbira črk; vsak izkušen matematik ve, da $\forall\, \epsilon > 0$ pomeni $\forall\, \epsilon \in \RR_{> 0}$. Dodaten problem je, da kasneje uporabljamo urejene pare, ki jih vsi na naši fakulteti pišejo z okroglimi oklepaji. Poskusimo se izogniti zmedi, ali $(a, b)$ pomeni urejeni par ali odprti interval. Če se ne strinjate, popravite in pustite komentar.}

% Če imamo dan neki element in neko množico, potem pripadnost tega elementa tej množici izrazimo s simbolom $\in$. Na primer, da je štiri naravno število, zapišemo $4 \in \NN$ (beri: ``štiri pripada množici naravnih števil'').

% Elementi množic lahko zadoščajo raznim lastnostim. Na primer, recimo, da $\phi$ označuje lastnost ``biti manj od pet''; to potem zapišemo
% \[\phi(x) \ = \ \ x < 5.\]
% V tem primeru $x$ imenujemo \df{spremenljivka}, saj ne gre za točno določeno vrednost, pač pa predstavlja splošno število (recimo, da se dogovorimo, da s $\phi$ označujemo lastnost na realnih številih).

% Tovrstne lastnosti nam omogočajo, da iz neke množice odberemo elemente z dano lastnostjo in na ta način dobimo novo množico, ki je podmnožica prejšnje. Množico vseh realnih števil, ki so manjša od pet, zapišemo na naslednji način.
% \[\set{x \in \RR}{x < 5}\]
% Seveda, ker je primerjava s števili zelo pogosta lastnost, je uporabno, če uvedemo krajše oznake, ki povejo isto; že prej smo se dogovorili, da tako množico označimo z $\RR_{< 5}$. Za povsem splošne lastnosti pa ne bomo imeli vnaprej dogovorjenih oznak, zato je dobro, da poznamo splošni zapis. Torej, če je $X$ poljubna množica in $\phi$ poljubna lastnost njenih elementov, tedaj podmnožico, ki vsebuje točno tiste elemente množice $X$, ki zadoščajo lastnosti $\phi$, označimo takole.
% \[\set[1]{x \in X}{\phi(x)}\]

% Pri tem se zavedajmo: ni pomembno, da spremenljivko označimo ravno z $x$. Zapis
% \[\set[1]{y \in X}{\phi(y)}\]
% še vedno označuje isto množico. V vsakem primeru gre za množico vseh elementov iz $X$ z lastnostjo $\phi$. Pravzaprav sploh ni nujno, da uporabimo črko; poslužimo se lahko kateregakoli simbola (ki mu nismo predtem že predpisali določenega pomena). Taisto množico lahko zapišemo tudi $\set{\heartsuit \in X}{\phi(\heartsuit)}$.

% Kadar imamo spremenljivko, ki jo lahko preimenujemo, ne da bi spremenili pomen izraza, jo imenujemo \df{nema spremenljivka}. Takšne primere že dobro poznate; na primer, integral $\int_0^1 x^2 \,dx$ se ne spremeni, če preimenujete spremenljivko in zapišete $\int_0^1 y^2 \,dy$.

% \begin{zgled}
% Kako bi zapisali množico vseh sodih naravnih števil? Spomnimo se, da je število sodo, kadar je deljivo z $2$. Za $n \in \NN$ to zapišemo takole: $2 \divides n$ (beri: ``dve deli $n$''). Množica sodih naravnih števil se potem zapiše kot
% \[\set[1]{n \in \NN}{2 \divides n}.\]
% \end{zgled}


\section{Vaje}

\begin{vaja}
  Kaj veste povedati o množici $A$, če zanjo velja, da so vsi njeni elementi enaki?
\end{vaja}

\begin{vaja}
  Načelo ekstenzionalnosti preslikav bi lahko zapisali tudi takole:
  %
  \begin{quote}
    Preslikavi $f : A \to B$ in $g : C \to D$ sta enaki, če velja $A = C$, $B = D$ in za
    vse $x_1, x_2 \in A$ velja, da iz $x_1 = x_2$ sledi $f(x_1) = g(x_2)$.
  \end{quote}
  %
  Dokažite, da je ta različica enakovredna običajnem načelu ekstenzionalnosti.
\end{vaja}

\begin{vaja}
  Zapišite pravila za zmnožek treh množic. Nato premislite še, kako bi podali pravila za
  zmnožek $n$ množic, kjer je~$n$ naravno število.
\end{vaja}


%%% Local Variables:
%%% mode: latex
%%% TeX-master: "ucbenik-lmn"
%%% End:


% Pod to vrstico so poglavja, ki še niso pripravljena
% za javno dostopno verzijo
\ifOPTincludeall
\chapter{Logika in pravila sklepanja (dodatno poglavje)}
\label{chap:logika}


\textbf{Opomba:} To poglavje je del učbenika v nastajanju in ni povsem v skladu s predavanji. Kljub temu ga vključujem v te zapiske, ker vsebuje precej koristnih nasvetov in misli.

%%%%%%%%%%%%%%%%%%%%%%%%%%%%%%%%%%%%%%%%%%%%%%%%%%%%%%%%%%%%%%%%%%%%%%
\section{Kaj je matematični dokaz?}
\label{sec:kaj-je-dokaz}

V srednji šoli se dijaki pri matematiki učijo, \emph{kako} se kaj
izračuna. Na univerzi imajo študentje matematike pred seboj
zahtevnejšo nalogo: poleg \emph{kako} morajo vedeti tudi \emph{zakaj}.
Od njih se pričakuje, da bodo računske postopke znali tudi utemeljiti,
ne pa samo slediti pravilom, ki jih je predpisal učitelj. Razumeti
morajo dokaze znamenitih izrekov in sami poiskati dokaze preprostih
izjav. Da bi se lažje spopadli s temi novimi nalogami, bomo prvi del
predmeta Logika in množice posvetili matematični infrastrukturi:
izjavam, pra\-vi\-lom sklepanja in dokazom. Učili se bomo, kako pišemo
dokaze, kako jih analiziramo in kako jih sami poiščemo.

Osrednji pojem matematične aktivnosti je \emph{dokaz}. Namen dokaza je
s pomočjo točno določenih in vnaprej dogovorjenih \emph{pravil
  sklepanja} utemeljiti neko matematično \emph{izjavo}. Načeloma mora
dokaz vsebovati vse podrobnosti in natanko opisati posamezne korake
sklepanja, ki privedejo do želene matematične izjave. Ker bi bili taki
dokazi zelo dolgi in bi vsebovali nezanimive podrobnosti, matematiki
običajno predstavijo samo oris ali glavno zamisel dokaza. Izkušenemu
matematiku to zadošča, saj zna oris sam dopolniti do pravega dokaza.
Matematik začetnik seveda potrebuje več podrobnosti. Poglejmo si
primer.

\begin{izrek}
  \label{izr:n3-n-deljivo-3}
  Za vsako naravno število $n$ je $n^3 - n$ deljivo s~$3$.
\end{izrek}

\noindent
Po kratkem premisleku bi izkušeni matematik zapisal:

\begin{quote}
  \begin{proof}
    Očitno.
  \end{proof}
\end{quote}

\noindent
To ni dokaz, izkušeni matematik nam le dopoveduje, da je (zanj) izrek
zelo lahek in da nima smisla izgubljati časa s pisanjem dokaza.
Začetnik, ki težko razume že sam izrek, bo ob takem ">dokazu"< seveda
zgrožen. Verjetno bo najprej preizkusil izrek na nekaj primerih:
%
\begin{align*}
  1^3 - 1 &= 0,\\
  2^3 - 2 &= 8 - 2 = 6,\\
  3^3 - 3 &= 27 - 4 = 24,\\
  4^3 - 4 &= 64 - 4 = 60.
\end{align*}
%
Res dobivamo večkratnike števila~$3$. Ali smo izrek s tem dokazali? Seveda ne!
Preizkusili smo le štiri primere, ostane pa jih še neskončno mnogo. Kdor
misli, da lahko iz nekaj primerov sklepa na splošno veljavnost, naj v poduk
vzame naslednjo nalogo.

\begin{vaja}
  Ali je $n^2 - n + 41$ praštevilo za vsako naravno število~$n$?
\end{vaja}

\noindent
Ko izkušenega matematika prosimo, da naj nam vsaj pojasni idejo dokaza,
zapiše:

\begin{quote}
  \begin{proof}
    Število $n^3 - n$ je zmnožek treh zaporednih naravnih števil.
  \end{proof}
\end{quote}

\noindent
To še vedno ni dokaz, ampak samo namig. Starejši študenti matematike pa
bi iz namiga morali znati sestaviti naslednji dokaz:

\begin{quote}
  \begin{proof}
    Ker je $n^3 - n = (n-1) \cdot n \cdot (n+1)$, je $n^3 - n$ zmnožek
    treh zaporednih naravnih števil, od katerih je eno deljivo s~$3$,
    torej je tudi $n^3 - n$ deljivo s~$3$.
  \end{proof}
\end{quote}

\noindent
(Mimogrede, s škatlico $\Box$ označimo konec dokaza.) Čeprav bi bila
večina matematikov s tem dokazom zadovoljna, bi morali za popoln dokaz
preveriti še nekaj podrobnosti:
%
\begin{enumerate}
\item Ali res velja $n^3 - n = (n-1) \cdot n \cdot (n+1)$?
\item Ali je res, da je izmed treh zaporednih naravnih števil eno
  vedno deljivo s~$3$?
\item Ali je res, da je zmnožek treh števil deljiv s~$3$, če je eno od
  števil deljivo s~$3$?
\end{enumerate}
%
S srednješolskim znanjem algebre ugotovimo, da je odgovor na prvo
vprašanje pritrdilen. Tudi odgovora na drugo in tretje vprašanje sta
očitno pritrdilna, mar ne? To pa ne pomeni, da ju ni treba dokazati.
Nasprotno, zgodovina matematike nas uči, da moramo prav ">očitne"<
izjave še posebej skrbno preveriti.

\begin{vaja}
  Kakšno je tvoje mnenje o resničnosti naslednjih izjav? Vprašaj
  starejše kolege, asistente in učitelje, kaj menijo oni. Ali znajo
  svoje mnenje utemeljiti z dokazi?
  %
  \begin{enumerate}
  \item Sodih števil je manj kot naravnih števil.
  \item Kroglo je mogoče razdeliti na pet delov tako, da lahko iz njih
    sestavimo dve krogli, ki sta enako veliki kot prvotna krogla.
  \item Sklenjena krivulja v ravnini, ki ne seka same sebe, razdeli
    ravnino na dve območji, eno omejeno in eno neomejeno.
  \item S krivuljo ne moremo prekriti notranjosti kvadrata.
  \item Če ravnino razdelimo na tri območja, potem zagotovo obstaja
    točka, ki je dvomeja in ni tromeja med območji.
  \end{enumerate}
\end{vaja}

\noindent
%
Vrnimo s k izreku~\ref{izr:n3-n-deljivo-3}. Če dokaz zapišemo preveč
podrobno, postane dolgočasen in ne\-ra\-zumljiv:

\begin{quote}
  \begin{proof}
    Naj bo $n$ poljubno naravno število. Tedaj velja
    %
    \begin{align*}
      n^3 - n
      &= n \cdot n^2 - n \cdot 1 \\
      &= n \cdot (n^2 - 1) \\
      &= n \cdot ((n + 1) \cdot (n - 1)) \\
      &= n \cdot ((n - 1) \cdot (n + 1)) \\
      &= (n \cdot (n - 1)) \cdot (n + 1) \\
      &= (n - 1) \cdot n \cdot (n + 1).
    \end{align*}
    %
    Vidimo, da je $n^3 - n$ zmnožek treh zaporednih naravnih števil.
    Dokažimo, da je eno od njih deljivo s~$3$. Število $n$ lahko
    enolično zapišemo kot $n = 3 k + r$, kjer je $k$ naravno število
    in $r = 0$, $r = 1$ ali $r = 2$. Obravavajmo tri primere:
    %
    \begin{itemize}
    \item če je $r = 0$, je $n = 3 k$, zato je $n$ deljiv s~$3$,
    \item če je $r = 1$, je $n - 1 = (3 k + 1) - 1 = 3 k + (1 - 1) = 3
      k + 0 = 3 k$, zato je $n-1$ deljiv s~$3$,
    \item če je $r = 2$, je $n + 1 = (3 k + 2) + 1 = 3 k + (2 + 1) = 3
      k + 3 = 3 k + 3 \cdot 1 = 3 (k +1)$, zato je $n+1$ deljiv s~$3$.
    \end{itemize}
    %
    Vemo torej, da je $n - 1$, $n$ ali $n + 1$ deljiv s~$3$.
    Obravnavamo tri primere:
    %
    \begin{itemize}
    \item Če je $n - 1$ deljiv s~$3$, tedaj  obstaja naravno število
      $k$, da je $n - 1 = 3 k$. V tem primeru je $(n - 1) n (n + 1) =
      (3 k) n (n + 1) = 3 (k n (n + 1))$, zato je $(n - 1) n (n + 1)$
      deljivo s~$3$.
    \item Če je $n$ deljiv s~$3$, tedaj obstaja naravno število $k$,
      da je $n = 3 k$. V tem primeru je $(n - 1) n (n + 1) = (n - 1)
      (3 k) n (n + 1) = (3 k) (n - 1) (n + 1) = 3 (k (n - 1) (n +
      1))$, zato je $(n - 1) n (n + 1)$ deljivo s~$3$.
    \item Če je $n + 1$ deljiv s~$3$, tedaj obstaja naravno število
      $k$, da je $n + 1 = 3 k$. V tem primeru je $(n - 1) n (n + 1) =
      (n - 1) n (3 k) = (n - 1) (3 k) n = (3 k) (n - 1) n = 3 (k (n -
      1) n)$, zato je $(n - 1) n (n + 1)$ deljivo s~$3$.
    \end{itemize}
    %
    V vsakem primeru je $(n - 1) n (n + 1)$ deljivo s~$3$. Ker smo
    dokazali, da je $n^3 = n = (n - 1) n (n + 1)$, je tudi $n^3 - n$
    deljivo s~$3$.
  \end{proof}
\end{quote}

\begin{vaja}
  S kolegi se igraj naslednjo igro.\footnote{%
    Igranje odsvetujemo zunaj prostorov Fakultete za matematiko in fiziko.}
  Prvi igralec v zgornjem dokazu poišče korak, ki ga je treba še dodatno
  utemeljiti. Drugi igralec ga utemelji. Nato prvi igralec poišče nov korak,
  ki ga je treba še dodatno utemeljiti in igra se ponovi. Zgubi tisti, ki se prvi naveliča igrati. Ali lahko igra traja neskončno dolgo?
\end{vaja}

Matematični dokaz ima dvojno vlogo. Po eni strani je utemeljitev matematične
izjave, zato mora biti čim bolj podroben. V idealnem primeru bi bil dokaz
zapisan tako, da bi lahko njegovo pravilnost preverili mehansko, z
računalnikom. Po drugi strani je dokaz sredstvo za komunikacijo idej med
matematiki, zato mora vsebovati ravno pravo mero podrobnosti. Mera pa je
odvisna od tega, komu je dokaz namenjen. Te socialne komponente se študenti
učijo skozi prakso v toku študija. Dokazu kot povsem matematičnemu pojmu pa se
bomo posvetili prav pri predmetu Logika in množice. Pojasnili bomo, kaj je
dokaz kot matematični konstrukt in kako ga zapišemo tako podrobno, da je res
mehansko preverljiv. Naučili se bomo tudi nekaj preprostih tehnik iskanja
dokazov, ki pa še zdaleč ne bodo zadostovale za reševanje zares zanimivih
matematičnih problemov, ki zahtevajo poglobljeno znanje, vztrajnost, kanček
talenta in nekaj sreče.


%%%%%%%%%%%%%%%%%%%%%%%%%%%%%%%%%%%%%%%%%%%%%%%%%%%%%%%%%%%%%%%%%%%%%%

\section{Simbolni zapis matematičnih izjav}
\label{sec:simbolni-zapis-izjav}

Matematična \textbf{izjava} je smiselno besedilo, ki izraža kako lastnost ali
razmerje med matematičnimi objekti (števili, liki, funkcijami, množicami
itn.). Primeri matematičnih izjav:
%
\begin{itemize}
\item $2 + 2 = 5$.
\item Točke $P$, $Q$ in $R$ so kolinearne.
\item Enačba $x^2 + 1 = 0$ nima realnih rešitev.
\item $a > 5$.
\item $\phi \lor \psi \lthen (\lnot \phi \lthen \psi)$.
\end{itemize}
%
Vidimo, da je lahko izjava resnična, neresnična, ali pa je resničnost
izjave \emph{odvisna} od vrednosti spremenljivk, ki nastopajo v njej.
Primeri besedila, ki niso matematične izjave:
%
\begin{itemize}
\item Ali je $2 + 2 = 5$?
\item Za vsak $x > 5$.
\item Študenti bi morali znati reševati diferencialne enačbe.
\item Od nekdaj lepe so Ljubljanke slovele, al lepše od Urške bilo ni nobene.
\item $\phi \lor ) \psi \lthen \psi$.
\end{itemize}
%
Matematične izjave običajno pišemo kombinirano v naravnem jeziku in z
matematični simboli, saj so tako najlažje razumljive ljudem. Za
potrebe matematične logike pa izjave pišemo \emph{samo} z
matematičnimi simboli. Tako zapisani izjavi pravimo \textbf{logična
  formula}. V ta namen moramo nadomestiti osnovne gradnike izjav, kot
so ">in"<, ">ali"< in ">za vsak"<, z \textbf{logičnimi operacijami}.
Le-te delimo na tri sklope. V prvi sklop sodita \textbf{logični
  konstanti}:
%
\begin{itemize}
\item resnica $\top$,
\item neresnica $\bot$.
\end{itemize}
%
V računalništvu resnico $\top$ pogosto označimo z $1$ ali \texttt{True} in
neresnico $\bot$ z $0$ ali \texttt{False}. Naslednji sklop so \textbf{logični
vezniki}, s katerimi sestavljamo nove izjave iz že danih:
%
\begin{itemize}
\item konjunkcija $\phi \land \psi$, beremo ">$\phi$ in $\psi$"<,
\item disjunkcija $\phi \lor \psi$, beremo ">$\phi$ ali $\psi$"<,
\item implikacija $\phi \lthen \psi$, beremo ">če $\phi$ potem $\psi$"<,
\item ekvivalenca $\phi \liff \psi$, beremo ">$\phi$ če, in samo če, $\psi$"< ali pa ">$\phi$ natanko tedaj, kadar~$\psi$"<,
\item negacija $\lnot \phi$, beremo ">ne $\phi$"<,
\end{itemize}
%
V tretji sklop sodita \textbf{logična kvantifikatorja}:
%
\begin{itemize}
\item univerzalni kvantifikator $\all{x \in S} \phi$, beremo ">za vse $x$
  iz $S$ velja $\phi$"<,
\item eksistenčni kvantifikator $\some{x \in S} \phi$, beremo ">obstaja
  $x$ v $S$, da velja $\phi$"<,
\end{itemize}
%
Pri tem je $S$ množica, razred\footnote{V poglavju~\ref{chap:mnozice}
  bomo spoznali razliko med množicami in razredi, zaenkrat si $S$
  predstavljamo kot množico.} ali tip spremenljivke~$x$. V praksi se
uporablja več inačic zapisa za kvantifikatorje, kot so ">$\forall x : S
.\, \phi$"<, ">$\forall x \in S : \phi$"< in ">$(\forall x \in S)
\phi$"<. Srečamo tudi zapis ">$\phi, \forall x \in S$"<, ki pa ga
odsvetujemo.

\textbf{Neomejena kvantifikatorja} $\all{x} \phi$ in
$\some{x} \phi$ se uporabljata, kadar je vnaprej znana množica $S$,
po kateri teče spremenljivka~$x$. V matematičnem besedilu je običajno
razvidna iz spremnega besedila, včasih pa je treba upoštevati
ustaljene navade: $n$ je naravno ali celo število, $x$ realno, $f$ je
funkcija ipd.

V uporabi so nekatere ustaljene okrajšave:
%
\begin{xalignat*}{3}
  &\some{x,y \in S} \phi,&
  &\text{pomeni}&
  &\some{x}{S} (\some{y}{S} \phi),\\
  %
  &\all{x \in S,y \in T} \phi,&
  &\text{pomeni}&
  &\all{x \in S}(\all{y \in T} \phi),\\
  %
  &\phi \liff \psi \liff \rho \liff \sigma&
  &\text{pomeni}&
  &(\phi \liff \psi) \land (\psi \liff \rho) \land (\rho \liff \sigma),\\
  %
  &f(x) = g(x) = h(x) = i(x)&
  &\text{pomeni}&
  &f(x) = g(x) \land g(x) = h(x) \land h(x) = i(x),\\
  &a \leq b < c \leq d&
  &\text{pomeni}&
  &a \leq b \land b < c \land c \leq d.
\end{xalignat*}
%
Nekatere okrajšave odsvetujemo. V nizu neenakosti naj gredo vse
primerjave v isto smer. Torej ne pišemo $a \leq b < c \geq d$, ker se
zlahka zmotimo in mislimo, da velja $a \geq d$. To bi morali zapisati
ločeno kot $a \leq b < c$ in $c \geq d$. Prav tako ne nizamo neenakosti,
saj premnogi iz $f(x) \neq g(x) \neq h(x)$ ">sklepajo"< $f(x) \neq
h(x)$, čeprav neenakost \emph{ni} tranzitivna relacija. Zapis $f(x) =
g(x) \neq h(x) = i(x)$ je v redu, saj ena sama neenakost ne povzroči
težav.

\begin{vaja}
  Zapiši $f(x) = g(x) \neq h(x) = i(x)$ brez okrajšav.
\end{vaja}

Povejmo še nekaj o pisanju oklepajev. Oklepaji povedo, katera
operacija ima prednost. Kadar manjkajo, moramo poznati dogovorjeno
\textbf{prioriteto} operacij. Na primer, ker ima množenje višjo
prioriteto kot seštevanje, je $5 \cdot 3 + 8$ enako $(5 \cdot 3) + 8$
in ne $5 \cdot (3 + 8)$. Tudi logične operacije imajo svoje
prioritete, ki pa niso tako splošno znane kot prioritete aritmetičnih
operacij. Zato bodite pazljivi, ko berete tuje besedilo.

Mi bomo privzeli naslednje prioritete logičnih operacij:
%
\begin{itemize}
\item negacija $\lnot$ ima prednost pred
\item konjunkcijo $\land$, ki ima prednost pred
\item disjunkcijo $\lor$, ki ima prednost pred
\item implikacijo $\lthen$, ki ima prednost pred
\item kvantifikatorjema $\forall$ in $\exists$.
\end{itemize}
%
Na primer:
%
\begin{itemize}
\item $\lnot \phi \lor \psi$ je isto kot $(\lnot \phi) \lor \psi$,
\item $\lnot \lnot \phi \lthen \phi$ je isto kot $(\lnot(\lnot\phi))
  \lthen \phi$,
\item $\phi \lor \psi \land \rho$ je isto kot $\phi \lor (\psi \land \rho)$,
\item $\phi \land \psi \lthen \phi \lor \psi$ je isto kot $(\phi
  \land \psi) \lthen (\phi \lor \psi)$,
\item $\all{x \in S} \phi \lthen \psi$ je isto kot $\all{x \in S} (\phi
    \lthen \psi)$,
\item $\some{x \in S} \phi \land \psi$ je isto kot $\some{x \in S} (\phi
    \land \psi)$.
\end{itemize}

V aritmetiki poznamo poleg prioritete operacij tudi \textbf{levo} in
\textbf{desno asociranost}. Denimo, seštevanje je levo asocirano,
ker beremo $5 + 3 + 7$ kot $(5 + 3) + 7$, saj najprej izračunamo $5 +
3$ in nato $8 + 7$. Pri seštevanju to sicer ni pomembno in bi lahko
seštevali tudi $3 + 7$ in nato $5 + 10$. Drugače je z odštevanjem,
kjer $5 - 3 - 7$ pomeni $(5 - 3) - 7$ in ne $5 - (3 - 7)$. Tudi za
logične operacije velja dogovor o njihovi asociranosti. Konjunkcija in
disjunkcija sta levo asocirani:
% 
\begin{align*}
  \phi \land \psi \land \rho
  &\qquad\text{pomeni}\qquad
  (\phi \land \psi) \land \rho,\\
  \phi \lor \psi \lor \rho
  &\qquad\text{pomeni}\qquad
  (\phi \lor \psi) \lor \rho.
\end{align*}
%
Za disjunkcijo in konjunkcijo sicer ni pomembno, kako postavimo
oklepaje, ker sta obe možnosti med seboj ekvivalentni, vendar je prav,
da natančno določimo, katera od njiju je mišljena. V logiki je
implikacija desno asocirana:
%
\begin{equation*}
  \phi \lthen \psi \lthen \rho
  \qquad\text{pomeni}\qquad
  \phi \lthen (\psi \lthen \rho).
\end{equation*}
%
Tu \emph{ni} vseeno, kako postavimo oklepaje, saj $\phi \lthen (\psi
\lthen \rho)$ in $(\phi \lthen \psi) \lthen \rho$ v splošnem nista
ekvivalentna. Vendar pozor! Ko matematiki, ki niso logiki, v
matematičnem besedilu zapišejo
%
\begin{equation*}
  \phi \lthen \psi \lthen \rho,
\end{equation*}
%
s tem skoraj vedno mislijo
%
\begin{equation*}
  (\phi \lthen \psi) \land (\psi \lthen \rho).
\end{equation*}
%
Zakaj? Zato ker je to priročen zapis, ki nakazuje zaporedje sklepov
">iz $\phi$ sledi $\psi$ in nato iz $\psi$ sledi $\rho$"<, še posebej,
če je zapisan v več vrsticah. Recimo, za nenegativni števili $x$ in
$y$ bi takole zapisali utemeljitev neenakosti med aritmetično in
geometrijsko sredino:
%
\begin{align*}
  & (x - y)^2 \geq 0 \lthen \\
  & x^2 - 2 x y + y^2 \geq 0 \lthen
  \tag{razstavimo}\\
  & x^2 + 2 x y + y^2 \geq 4 x y \lthen
  \tag{prištejemo $4 x y$}\\
  & (x + y)^2 \geq 4 x y \lthen
  \tag{faktoriziramo}\\
  & \frac{(x + y)^2}{4} \geq x y \lthen
  \tag{delimo s $4$}\\
  & \frac{x+y}{2} \geq \sqrt{x y}.
  \tag{korenimo}
\end{align*}
%ANDREJ: meni je Gordon rekel, da utemeljitve sledijo sklepu, torej so
% eno vrstico niže.
%
Matematiki radi celo spustijo znak $\lthen$ in preprosto vsak
naslednji sklep napišejo v svojo vrstico. Ker torej velja tak ustaljen
način pisanja zaporedja sklepov, je varneje pisati $\phi \lthen (\psi
\lthen \rho)$ brez oklepajev, da ne povzročamo zmede.
%ANDREJ: zadnjega stavka ne razumem.

%%%%%%%%%%%%%%%%%%%%%%%%%%%%%%%%%%%%%%%%%%%%%%%%%%%%%%%%%%%%%%%%%%%%%%

\section{Kako beremo in pišemo simbolni zapis}
\label{sec:simbolni-zapis}

Izjave, zapisane v simbolni obliki, ni težko prebrati. Na primer,
%
\begin{equation*}
  \all{x, y \in \RR}
    x^2 = 4 \land y^2 = 4 \lthen x = y,
\end{equation*}
%
preberemo:
%
\begin{quote}
  ">Za vse realne $x$ in $y$, če je $x^2$ enako $4$ in $y^2$ enako
  $4$, potem je $x$ enako $y$."<
\end{quote}
%
Več izkušenj pa je potrebnih, da \emph{razumemo} matematični pomen
take izjave, v tem primeru:
%
\begin{quote}
  ">Enačba $x^2 = 4$ ima največ eno realno rešitev."<
\end{quote}
%
Začetnik potrebuje nekaj vaje, da se navadi brati simbolni zapis. Tudi
prevod v obratno smer, iz besedila v simbolno obliko, ni enostaven,
zato povejmo, kako se prevede nekatere standardne fraze.

\subsubsection{">$\phi$ je zadosten pogoj za $\psi$."<}

To pomeni, da zadošča dokazati $\phi$ zato, da dokažemo $\psi$, ali v
simbolni obliki
%
\begin{equation*}
  \phi \lthen \psi.
\end{equation*}

\subsubsection{">$\phi$ je potreben pogoj za $\psi$."<}

To pomeni, da $\psi$ ne more veljati, ne da bi veljal~$\phi$. Z drugimi
besedami, če velja $\psi$, potem velja tudi $\phi$, kar se v simbolni obliki
zapiše
%
\begin{equation*}
  \psi \lthen \phi.
\end{equation*}

\subsubsection{">$\phi$ je zadosten in potreben pogoj za $\psi$."<}

To je kombinacija prejšnjih dveh primerov, ki trdi, da iz $\phi$ sledi
$\psi$ in iz $\psi$ sledi $\phi$, kar pa je ekvivalenca:
%
\begin{equation*}
  \phi \liff \psi.
\end{equation*}

\begin{vaja}
  Je ">$n$ je sod in $n > 2$"< \textbf{potreben} ali \textbf{zadosten}
  pogoj za ">$n$ ni praštevilo"<?
\end{vaja}


\subsubsection{">Naslednje izjave so ekvivalentne: $\phi$, $\psi$, $\rho$ in $\sigma$."<}

To pomeni, da sta vsaki dve izmed danih izjav ekvivalentni, se pravi
%
\begin{equation*}
  (\phi \liff \psi) \land (\phi \liff \rho) \land (\phi \liff \sigma) \land (\psi \liff \rho)
  \land (\psi \liff \sigma) \land (\rho \liff \sigma).
\end{equation*}
%
Ker je ekvivalenca tranzitivna relacija, ni treba obravnavati vseh
kombinacij, zadostujejo že tri, ki dane izjave ">povežejo"< med seboj:
%
\begin{equation*}
  (\phi \liff \psi) \land (\psi \liff \rho) \land (\rho \liff \sigma).
\end{equation*}
%
To pišemo krajše kar kot
%
\begin{equation*}
  \phi \liff \psi \liff \rho \liff \sigma,
\end{equation*}
%
čeprav je formalno gledano tako zapis nepravilen. V
razdelku~\ref{sec:ekvivalenca} bomo spoznali, kako se tako zaporedje
ekvivalenc dokaže s ciklom implikacij $\phi \lthen \psi \lthen \rho
\lthen \sigma \lthen \phi$.

\begin{vaja}
  Podaj konkretne primere izjav $\phi$, $\psi$ in $\rho$, iz katerih
  je razvidno, da izjava $(\phi \liff \psi) \land (\psi \liff \rho)$
  \emph{ni} ekvivalentna niti $(\phi \liff \psi) \liff \rho$ niti
  $\phi \liff (\psi \liff \rho)$.
\end{vaja}



\subsubsection{">Za vsak $x$ iz $S$, za katerega velja $\phi$, velja tudi
  $\psi$."<}

To lahko preberemo tudi kot ">Za vsak $x$ iz $S$, če zanj velja $\phi$,
potem velja $\psi$,"< kar je v simbolni obliki
%
\begin{equation*}
  \all{x \in S} \phi \lthen \psi.
\end{equation*}
%
Tudi izjave oblike ">vsi $\phi$-ji so $\psi$-ji"< so te oblike, denimo ">vsa
od dva večja praštevila so liha"< zapišemo
%
\begin{equation*}
  \all{n \in \NN} n > 2 \land \text{$n$ je praštevilo} \lthen \text{$n$ je lih}.
\end{equation*}

\begin{vaja}
  V simbolni obliki zapiši ">$n$ je lih"< in ">$n$ je praštevilo"<.
  Namig: $n$ je lih, kadar obstaja naravno število $k$, za katerega
  velja $n = 2 k + 1$, in $n$ je praštevilo, kadar \emph{ni} zmnožek
  dveh naravnih števil, ki sta obe večji od~$1$.
\end{vaja}


\subsubsection{">Enačba $f(x) = g(x)$ nima realne rešitve."<}

To lahko povemo takole: ni res, da obstaja $x \in \RR$, za katerega bi
veljalo $f(x) = g(x)$. S simboli zapišemo
%
\begin{equation*}
  \lnot \some{x \in \RR} f(x) = g(x).
\end{equation*}
%
Opozoriti velja, da iz same enačbe ne moremo vedno sklepati, kaj je
neznanka. V enačbi $a x^2 + b x + c = 0$ bi za neznanko lahko načeloma
imeli katerokoli od štirih spremenljivk $a$, $b$, $c$ in $x$, ali pa
kar vse. Večina matematikov bi sicer uganila, da je najverjetneje
neznanka $x$, vendar se v splošnem ne moremo zanašati na običaje in
uganjevanje, ampak moramo točno povedati, kateri simboli so
\textbf{neznanke} in kateri \textbf{parametri}.

\begin{vaja}
  Zapiši v simbolni obliki: ">Sistem enačb
  %
  \begin{align*}
    a_1 x + b_1 y &= c_1,\\
    a_2 x + b_2 y &= c_2
  \end{align*}
  %
  nima pozitivnih realnih rešitev $x, y$."<
\end{vaja}

\begin{vaja}
  Zapiši v simbolni obliki:
  \begin{enumerate}
  \item ">Enačba $f(x) = g(x)$ ima največ eno realno rešitev."<
  \item ">Enačba $f(x) = g(x)$ ima več kot eno realno rešitev."<
  \item ">Enačba $f(x) = g(x)$ ima natanko dve realni rešitvi."<
  \end{enumerate}
\end{vaja}


\subsubsection{">Brez izgube za splošnost."<}

V matematičnih besedilih najdemo frazo ">brez izgube za splošnost"<
kot v naslednjem primeru.

\begin{izrek}
  \label{izrek:abc-vsota-razlik-soda}
  Za vsa cela števila $a$, $b$ in $c$ je $|a-b|+|b-c|+|c-a|$ sodo
  število.
\end{izrek}

\begin{proof}
  Brez izgube za splošnost smemo predpostaviti $a \geq b \geq c$.
  Tedaj velja
  %
  \begin{equation*}
    |a-b| + |b-c| + |c-a| = (a - b) + (b - c) - (c - a) = 2 (a - c),
  \end{equation*}
  %
  kar je sodo število.
\end{proof}

Fraza ">brez izgube za splošnost"< nakazuje, da dokaz obravnava le eno
od večih možnosti. Načeloma bi morali obravnavati tudi ostale
možnosti, ki pa jih je pisec dokaza opustil, ker so bodisi zelo lahke
bodisi zelo podobne tisti, ki jo dokaz obravnava. Za začetnika je
najtežje dognati, katere so preostale možnosti in zakaj se je pisec
dokaza pravzaprav odločil zanje. Avtor zgornjega dokaza je verjetno
opazil, da števila $a$, $b$ in $c$ v izrazu $|a-b|+|b-c|+|c-a|$
nastopajo \emph{simetrično}: če jih premešamo, se izraz ne spremeni.
Denimo, ko zamenjamo $a$ in $b$, dobimo $|b-a|+|a-c|+|c-b|$, kar je
enako prvotnemu izrazu $|a-b|+|b-c|+|c-a|$. Torej lahko izmed šestih
možnosti
%
\begin{xalignat*}{3}
  & a \geq b \geq c,&
  & a \geq c \geq b,&
  & b \geq a \geq c,\\
  & b \geq c \geq a,&
  & c \geq a \geq b,&
  & c \geq b \geq a
\end{xalignat*}
%
obravnavamo le eno. Seveda pisanje dokazov, pri katerih večji del
dokaza opustimo, zahteva pazljivost in nekaj izkušenj.

\begin{vaja}
  Dokaži izrek~\ref{izrek:abc-vsota-razlik-soda} tako, da obravnavaš
  samo možnost $b \geq c \geq a$ in zraven dopišeš ">brez izgube za
  splošnost"<.
\end{vaja}


%%%%%%%%%%%%%%%%%%%%%%%%%%%%%%%%%%%%%%%%%%%%%%%%%%%%%%%%%%%%%%%%%%%%%%
\section{Definicije}
\label{sec:definicije}


Poznamo tri vrste definicij. Prva in najpreprostejša je definicija, ki
služi kot \textbf{okrajšava} za daljši izraz. To smemo vedno nadomestiti
s prvotnim izrazom. Na primer, funkcija ">hiperbolični tangens"<
$\tanh(x)$ je definirana kot
%
\begin{equation*}
  \tanh(x) = \frac{e^{2 x} - 1}{e^{2 x} + 1}.
\end{equation*}
%
Lahko bi shajali tudi brez zapisa $\tanh(x)$, vendar bi morali potem
povsod pisati daljši izraz $\frac{e^{2 x} - 1}{e^{2 x} + 1}$, kar bi
bilo nepregledno.

Druga vrsta definicije je vpeljava novega matematičnega pojma.
Študenti prvega letnika matematike spoznajo celo vrsto novih pojmov
(grupa, vektorski prostor, limita, stekališče, metrika itn.), s
katerimi si razširijo sposobnost matematičnega razmišljanja.
Matematiki cenijo dobre definicije in vpeljavo novih matematičnih
pojmov vsaj toliko, kot dokaze težkih izrekov.

Tretja vrsta definicije je \textbf{konstrukcija} matematičnega objekta s
pomočjo dokaza o enoličnem obstoju. O tem bomo povedali več v
razdelku~\ref{sec:enolicni-obstoj}.

\section{Pravila sklepanja in dokazi}
\label{sec:pravila-sklepanja-in-dokazi}


Povedali smo že, da je dokaz utemeljitev neke matematične izjave. V
razdelku~\ref{sec:kaj-je-dokaz} smo govorili o tem, da so dokazi
mešanica besedila in simbolov, ki jih matematiki uporabljajo tako za
utemeljitev matematičnih izjav, kakor tudi za razlago in podajanje
matematičnih idej. V tem razdelku se posvetimo \textbf{formalnim
  dokazom}, ki so logične konstrukcije namenjene mehanskemu
preverjanju pravilnosti izjav.

Za vsako logično operacijo bomo podali \textbf{formalna pravila
  sklepanja}, ki jih smemo uporabljati v formalnem dokazu. Pravilo
sklepanja shematsko zapišemo
%
\begin{equation*}
  \inferrule{\phi \\ \psi \\ \rho}{\sigma}
\end{equation*}
%
in ga preberemo: ">Če smo dokazali $\phi$, $\psi$ in $\rho$, smemo
sklepati $\sigma$."< Izjavam nad črto pravimo \textbf{hipoteze}, izjavi
pod črto pa \textbf{sklep}. Hipotez je lahko nič ali več, sklep mora
biti natanko en. Pravilo sklepanja brez hipotez se imenuje
\textbf{aksiom}.

Da bomo lahko pojasnili, kaj je dokaz, podajmo pravila sklepanja za
$\top$ in $\land$, ki jih bomo v naslednjem razdelku še enkrat bolj
pozorno obravnavali:
%
\begin{mathpar}
  \inferrule{\quad}{\top}
  %
  \and
  %
  \inferrule
  {\phi \\ \psi}
  {\phi \land \psi}
  %
  \and
  %
  \inferrule
  {\phi \land \psi}
  {\phi}
  %
  \and
  %
  \inferrule
  {\phi \land \psi}
  {\psi}  
\end{mathpar}
%
Po vrsti beremo:
%
\begin{itemize}
\item Velja $\top$.
\item Če velja $\phi$ in $\psi$, smemo sklepati $\phi \land \psi$.
\item Če velja $\phi \land \psi$, smemo sklepati $\phi$.
\item Če velja $\phi \land \psi$, smemo sklepati $\psi$.
\end{itemize}
%
Formalni dokaz ima drevesno obliko in prikazuje, kako iz danih
\textbf{hipotez} dokažemo neko \textbf{sodbo}. Pri dnu je zapisana izjava,
ki jo dokazujemo, nad njo pa dokaz. Vsako vejišče je eno od pravil
sklepanja. Vsaka veja se mora zaključiti z aksiomom ali s hipotezo.
Oglejmo si dokaz izjave $(\alpha \land \alpha) \land (\top
\land \beta)$ iz hipoteze $\beta \land \alpha$:
%
\begin{equation*}
  \inferrule{
    \inferrule{
      \inferrule{\beta \land \alpha}{\alpha}
      \\
      \inferrule{\beta \land \alpha}{\alpha}}
      {\alpha \land \alpha}
    \\
    \inferrule{
      \inferrule{ }{\top}
      \\
      \inferrule{\beta \land \alpha}{\beta}
    }{\top \land \beta}
  }{(\alpha \land \alpha) \land (\top \land \beta)}
\end{equation*}
%
Dokaz se razveji na dve veji, vsaka od njiju pa še na dve veji. Tako
pri vrhu dobimo štiri liste, od katerih se trije izjava $\beta \land
\alpha$ in en aksiom za $\top$.

\begin{vaja}
  Preveri, da je vsako vejišče v zgornjem dokazu res uporaba enega od
  zgoraj podanih pravil sklepanja.
\end{vaja}

V praksi matematično besedilo bolj ali manj odraža strukturo
formalnega dokaza, le da se besedilo ne veji, ampak so sestavni kosi
dokaza zloženi v zaporedje. Formalni dokazi so uporabni, kadar želimo
preveriti veljavnost najbolj osnovnih logičnih dejstev. Ni mišljeno,
da bi matematiki pisali ali preverjali velike formalne dokaze
pomembnih matematičnih izrekov, to je delo za račualnike. Formalna
pravila sklepanja in formalni dokazi so za matematike pomembni, ker
nam omogočajo, da natančno in v celoti povemo, kakšna so ">pravila
igre"< v matematiki.


\section{Izjavni račun}
\label{sec:izjavni-racun}

Izjavni račun je tisti del logike, ki govori o logičnih konstantah
$\bot$, $\top$ in o logičnih operacijah $\land$, $\lor$, $\lthen$,
$\liff$, $\lnot$. Za vsako od njih podamo formalna pravila sklepanja,
ki so dveh vrst. Pravila \textbf{vpeljave} povedo, kako se izjave
dokaže, pravila \textbf{uporabe} pa povedo, kako se že dokazane izjave uporabi.

\subsection{Konjunkcija}
\label{sec:konjunkcija}

Konjunkcija ima eno pravilo vpeljave in dve pravili uporabe:
%
\begin{mathpar}
  \inferrule
  {\phi \\ \psi}
  {\phi \land \psi}
  \and
  \inferrule
  {\phi \land \psi}
  {\phi}  
  %
  \and
  %
  \inferrule
  {\phi \land \psi}
  {\psi}
\end{mathpar}
%
Pravilo vpeljave pove, da konjunkcijo $\phi \land \psi$ dokažemo
tako, da dokažemo posebej $\phi$ in posebej $\psi$. Pravili uporabe pa
povesta, da lahko $\phi \land \psi$ ">razstavimo"< na izjavi~$\phi$
in~$\psi$.

V matematičnem besedilu je dokaz konjunkcije $\phi \land \psi$ zapisan
kot zaporedje dveh pod-dokazov:
%
\begin{quote}
  \it 
  %
  Dokazujemo $\phi \land \psi$:
  \begin{enumerate}
  \item (Dokaz $\phi$)
  \item (Dokaz $\psi$)
  \end{enumerate}
  Dokazali smo $\phi \land \psi$.
\end{quote}
%
Manj podroben dokaz ne vsebuje začetnega in končnega stavka, ampak
samo dokaza za $\phi$ in $\psi$. Bralec mora sam ugotoviti, da je s
tem dokazana izjava $\phi \land \psi$.

\subsection{Implikacija}
\label{sec:implikacija}

Preden zapišemo pravila sklepanja za implikacijo, si oglejmo primer
neformalnega dokaza.

\begin{izrek}
  Če je $x > 2$, potem je $x^3 + x + 7 > 16$.
\end{izrek}

\begin{proof}
  Predpostavimo, da velja $x > 2$. Tedaj je $x^3 > 2^3 = 8$, zato
  velja
  %
  \begin{equation*}
    x^3 + x + 7 > 8 + 2 + 7 = 17 > 16.
  \end{equation*}
  %
  Dokazali smo $x > 2 \lthen x^3 + x + 7 > 16$.
\end{proof}

\noindent
%
Prvi stavek dokaza z besedico ">predpostavimo"< uvede \textbf{začasno
  hipotezo} $x > 2$, iz katere nato izpeljemo posledico $x^3 + x + 7 >
16$. Implikacijo $\phi \lthen \psi$ torej dokažemo tako, da začasno
predpostavimo $\phi$ in dokažemo $\psi$. Tako pravilo vpeljave
zapišemo
%
\begin{equation*}
  \inferrule{\infer*{\psi}{[\phi]}}{\phi \lthen \psi}  
\end{equation*}
%
Zapis $[\phi]$ z oglatimi oklepaji pomeni, da $\phi$ ni prava, ampak
samo začasna hipoteza. Zapis
%
\begin{equation*}
  \infer*{\psi}{[\phi]}
\end{equation*}
%
pomeni ">dokaz izjave $\phi$ s pomočjo začasne hipoteze $\phi$."<

Pravilo uporabe za implikacijo se imenuje \textbf{modus ponens} in se
glasi
%
\begin{mathpar}
  \inferrule{\phi \lthen \psi \\ \phi}{\psi}
\end{mathpar}
%
V matematičnem besedilu se modus ponens pojavi kot uporaba že prej
dokazanega izreka izreka oblike $\phi \lthen \psi$.

\subsection{Disjunkcija}
\label{sec:disjunkcija}

Disjunkcija ima dve pravili vpeljave in eno pravilo uporabe:
%
\begin{mathpar}
  \inferrule
  {\phi}
  {\phi \lor \psi}
  \and
  \inferrule
  {\psi}
  {\phi \lor \psi}
  \and
  \inferrule
  {\phi \lor \psi \\ \infer*{\rho}{[\phi]} \\ \infer*{\rho}{[\psi]}}
  {\rho}
\end{mathpar}
%
Pravili sklepanja povesta, da lahko dokažemo disjunkcijo $\phi \lor
\psi$ tako, da dokažemo enega od disjunktov.

Pojasnimo še pravilo uporabe. Denimo, da bi radi dokazali $\rho$, pri
čemer že vemo, da velja $\phi \lor \psi$. Pravilo uporabe pravi, da je
treba obravnavati dva primera: iz začasne hipoteze $\phi$ je treba
dokazati $\rho$ in iz začasne hipoteze $\psi$ je treba dokazati
$\rho$.

Ponazorimo pravilo uporabe v dokazu izjave $(\alpha \lor \gamma) \land
(\beta \lor \gamma)$ iz hipoteze $(\alpha \land \beta) \lor \gamma$.
Dokazno drevo je precej veliko, v njem pa se dvakrat pojavi uporaba
disjunkcije:
%
\begin{equation*}
  \inferrule
  {\inferrule*
    {\inferrule*{}{(\alpha \land \beta) \lor \gamma}
      \\
      \inferrule*{
        \inferrule
        {[\alpha \land \beta]}
        {\alpha}
      }{\alpha \lor \gamma}
      \\
      \inferrule*{[\gamma]}{\alpha \lor \gamma}
    }
    {\alpha \lor \gamma}
    \\
    \inferrule*
    {\inferrule*{}{(\alpha \land \beta) \lor \gamma}
      \\
      \inferrule*{
        \inferrule
        {[\alpha \land \beta]}
        {\beta}
      }{\beta \lor \gamma}
      \\
      \inferrule*{[\gamma]}{\beta \lor \gamma}
    }
    {\beta \lor \gamma}
  }
  {(\alpha \lor \gamma) \land (\beta \lor \gamma)}
\end{equation*}
%
Poglejmo na primer levo vejo tega dokaza, desna je podobna:
%
\begin{equation*}
  \inferrule*
    {\inferrule*{}{(\alpha \land \beta) \lor \gamma}
      \\
      \inferrule*{
        \inferrule
        {[\alpha \land \beta]}
        {\alpha}
      }{\alpha \lor \gamma}
      \\
      \inferrule*{[\gamma]}{\alpha \lor \gamma}
    }
    {\alpha \lor \gamma}
\end{equation*}
%
Res je to uporaba disjunkcije $\phi \lor \psi$, kjer smo vzeli $\phi =
\alpha \land \beta$ in $\psi = \gamma$, dokazali pa smo izjavo $\rho =
\alpha \lor \gamma$.

\begin{vaja}
  Iz hipoteze $(\alpha \lor \gamma) \land (\beta \lor \gamma)$ dokaži
  $(\alpha \land \beta) \lor \gamma$.
\end{vaja}

V besedilu dokažemo disjunkcijo s pravilom za vpeljavo takole:
%
\begin{quote}
  \it
  %
  Dokazujemo $\phi \lor \psi$. Zadostuje dokazati $\phi$:
  \begin{enumerate}
  \item[] (Dokaz $\phi$.)
  \end{enumerate}
  %
  Dokazali smo $\phi \lor \psi$.
\end{quote}
%
Pravilo uporabe disjunkcije se v besedilu zapiše kot obravnava
primerov:
%
\begin{quote}
  \it
  %
  Dokazujemo $\rho$. To bomo dokazali z obravnavo primerov $\phi$ in
  $\psi$:
  \begin{enumerate}
  \item (Dokaz $\phi \lor \rho$)
  \item Predpostavimo, da velja $\phi$. (Dokaz $\rho$)
  \item Predpostavimo, da velja $\psi$. (Dokaz $\rho$)
  \end{enumerate}
  %
  Dokazali smo $\rho$.
\end{quote}
%
Še primer konkretnega dokaza, ki je tako napisan.

\begin{izrek}
  \label{izrek:x-3-5}
  Naj bo $x$ realno število. Če je $|x - 3| > 5$, potem je $x^4 > 15$.
\end{izrek}

\begin{proof}
  Dokazujemo $|x - 3| > 5 \lthen x^4 > 15$. Predostavimo $|x - 3| > 5$
  in dokažimo $x^4 > 15$. To bomo dokazali z obravavo primerov $x \leq
  3$ in $x \geq 3$:
  %
  \begin{enumerate}
  \item $x \leq 3 \lor x \geq 3$ velja, ker so realna števila linearno
    urejena z relacijo $\leq$.
  \item Predpostavimo $x \leq 3$. Tedaj je $x - 3 \leq 0$ in zato $|x
    - 3| = 3 - x$, od koder sledi $3 - x = |x - 3| > 5$, oziroma $x <
    -2$. Tako dobimo
    %
    \begin{equation*}
      x^4 > (-2)^4 = 16 > 15.
    \end{equation*}
  \item Predpostavimo $x \geq 3$. Tedaj je $x - 3 \geq 0$ in zato$|x -
    3| = x - 3$, od koder sledi $x - 3 = |x - 3| > 5$, oziroma $x >
    8$. Tako dobimo
    %
    \begin{equation*}
      x^4 > 8^4 = 4096 > 15.
    \end{equation*}
  \end{enumerate}
  %
  Iz predpostavke $|x - 3| > 5$ smo izpeljali $x^4 > 15$. S tem smo
  dokazali $|x - 3| > 5 \lthen x^4 > 15$.
\end{proof}

Težji del tega dokaza se skriva v izbiri disjunkcije. Kako je pisec
uganil, da je treba obravnavati primera $x \leq 3$ in $x \geq 3$?
Zakaj ni raje obravnaval $x < 3$ in $x \geq 3$, ali morda $x \leq 17$
in $x \geq 17$? Odgovor se skriva v definiciji absolutne vrednosti:
%
\begin{equation*}
  |a| =
  \begin{cases}
    a & \text{če je $a \geq 0$,}\\
    -a & \text{če je $a \leq 0$.}
  \end{cases}
\end{equation*}
%
Ker v izreku nastopa izraz $|x - 3|$, bo obravnava primerov $x - 3
\geq 0$ in $x - 3 \leq 0$ omogočila, da $|x - 3|$ poenostavimo enkrat
v $x - 3$ in drugič v $3 - x$. Seveda pa je $x - 3 \geq 0$
ekvivalentno $x \geq 3$ in $x - 3 \leq 0$ ekvivalentno $x \leq 3$.

\begin{vaja}
  Ali bi lahko izrek~\ref{izrek:x-3-5} dokazali tudi z obravnavo
  primerov $x < 3$ in $x \geq 3$?
\end{vaja}

\subsection{Resnica in neresnica}
\label{sec:resnica-neresnica}

Logična konstanta $\top$ označuje resnico. Kar je res, je res, in tega
ni treba posebej dokazovati. To dejstvo izraža aksiom
%
\begin{equation*}
  \inferrule{\qquad}{\top}
\end{equation*}
%
Logična konstanta $\top$ nima pravila uporabe, ker iz $\top$ ne moremo
sklepati nič koristnega.

Logična konstanta $\bot$ označuje neresnico. Ker se tega, kar ni res,
ne more dokazati, $\bot$ nima pravila vpeljave. Pravilo uporabe je
%
\begin{equation*}
  \inferrule{\quad\bot\quad}{\phi}
\end{equation*}
%
se imenuje \textbf{ex falso (sequitur) quodlibet}, kar pomeni ">iz
neresnice sledi karkoli"<.

V matematičnem besedilu se $\top$ in $\bot$ ne pojavljata pogosto, ker
matematiki izraze, v katerih se $\top$ in $\bot$ pojavita, vedno
poenostavijo s pomočjo ekvivalenc:
%
\begin{mathpar}
  \top \land \phi \liff \phi
  \and
  \top \lor \phi \liff \phi
  \and
  \bot \land \phi \liff \bot
  \and
  \bot \lor \phi \liff \phi
  \\
  (\top \lthen \phi) \liff \phi
  \and
  (\bot \lthen \phi) \liff \top
  \and
  (\phi \lthen \top) \liff \top
\end{mathpar}
%

\subsection{Ekvivalenca}
\label{sec:ekvivalenca}

Logična ekvivalenca $\phi \liff \psi$ je okrajšava za
%
\begin{equation*}
  (\phi \lthen \psi) \land (\psi \lthen \phi).
\end{equation*}
%
Ker je to konjunkcija (dveh implikacij), so pravila za vpeljavo in
uporabo ekvivalence samo poseben primer pravil sklepanja za
konjunkcijo:
%
\begin{mathpar}
  \inferrule
  {\phi \lthen \psi \\ \psi \lthen \phi}
  {\phi \liff \psi}
  \and
  \inferrule{\phi \liff \psi}{\phi \lthen \psi}
  \and
  \inferrule{\phi \liff \psi}{\psi \lthen \phi}
\end{mathpar}
%
V matematičnem besedilu ekvivalenco dokažemo takole:
%
\begin{quote}
  \it
  %
  Dokazujemo $\phi \liff \psi$:
  %
  \begin{enumerate}
  \item (Dokaz $\phi \lthen \psi$)
  \item (Dokaz $\psi \lthen \phi$)
  \end{enumerate}
  Dokazali smo $\phi \liff \psi$.
\end{quote}

Če sta izjavi $\phi$ in $\psi$ logično ekvivalentni, lahko eno
zamenjamo z drugo. To matematiki s pridom uporabljajo pri dokazovanju
izjav, čeprav pogosto sploh ne omenijo, katero ekvivalenco so
uporabili.

Kadar dokazujemo medsebojno ekvivalenco večih izjav $\phi_1$,
$\phi_2$, \ldots, $\phi_n$, zadostuje dokazati cikel implikacij
%
\begin{equation*}
  \phi_1 \lthen \phi_2 \lthen \cdots \lthen \phi_{n-1} \lthen \phi_n \lthen \phi_1.
\end{equation*}
%
(Ne spreglejte zadnje implikacije $\phi_n \lthen \phi_1$, ki zaključi
cikel). V besedilu to dokažemo:

\begin{quote}
  \it
  %
  Dokazujemo, da so izjave $\phi_1, \phi_2, \ldots, \phi_n$
  ekvivalentne:
  %
  \begin{enumerate}
  \item (Dokaz $\phi_1 \lthen \phi_2$)
  \item (Dokaz $\phi_2 \lthen \phi_3$)
  \item \dots
  \item (Dokaz $\phi_{n-1} \lthen \phi_n$)
  \item (Dokaz $\phi_n \lthen \phi_1$)
  \end{enumerate}
\end{quote}

\noindent
%
Seveda smemo pred samim dokazovanjem izjave $\phi_1, \ldots, \phi_n$
preurediti tako, da je zahtevane implikacije kar najlažje dokazati.
Dokaz lahko tudi razdelimo na dva ločena cikla implikacij
%
\begin{equation*}
  \phi_1 \lthen \cdots \lthen \phi_k \lthen \phi_1
\end{equation*}
%
in
%
\begin{equation*}
  \phi_{k+1} \lthen \cdots \lthen \phi_n \lthen \phi_{k+1}
\end{equation*}
%
in nato dokažemo še eno ekvivalenco $\phi_i \liff \phi_j$, pri čemer
je $\phi_i$ iz prvega in $\phi_j$ iz drugega cikla.

\subsection{Negacija}
\label{sec:negacija}


Negacija $\lnot\phi$ je definirana kot okrajšava za $\phi
\lthen \bot$. Iz pravil sklepanja za $\lthen$ in $\bot$ tako izpeljemo
pravili sklepanja za negacijo:
%
\begin{mathpar}
  \inferrule
  {\infer*{\bot}{[\phi]}}
  {\lnot \phi}
  %
  \and
  %
  \inferrule
  {\lnot\phi \\ \phi}
  {\psi}
\end{mathpar}
%
V besedilu dokazujemo $\lnot\phi$ takole:
%
\begin{quote}
  \it
  %
  Dokazujemo $\lnot\phi$.
  \begin{itemize}
  \item[] Predpostavimo $\phi$.
  \item[] (Dokaz $\bot$.)
  \end{itemize}
  Dokazali smo $\lnot\phi$.
\end{quote}
%
Tu ">Dokaz $\bot$"< pomeni, da iz danih predpostavk izpeljemo
protislovje. Mnogi matematiki menijo, da se takemu dokazu reče ">dokaz
s protislovjem"<, vendar to ni res. To je samo navaden dokaz negacije.
Dokazovanje s protislovjem bomo obravnavali v razdelku~\ref{sec:lem}.

Pravilo uporabe za $\lnot\phi$ v besedilu ni eksplicitno vidno, ampak
ga matematiki uporabijo, ko sredi dokaza, da velja $\psi$, izpeljejo
protislovje:
%
\begin{quote}
  \it
  %
  Dokazujemo $\psi$.
  %
  \begin{itemize}
  \item[] (Dokaz $\phi$.)
  \item[] (Dokaz $\lnot\phi$.)
  \end{itemize}
  %
  To je nesmisel, in ker iz nesmisla sledi karkoli, sledi $\psi$.
\end{quote}

\subsection{Aksiom o izključenem tretjem}
\label{sec:lem}

Aksiom o izključenem tretjem se glasi
%
\begin{equation*}
  \inferrule{ }{\phi \lor \lnot \phi}
\end{equation*}
%
Povedano z besedami, vsaka izjava je bodisi resnična bodisi
neresnična. Torej ni ">tretje možnosti"< za resničnostno vrednost
izjave $\phi$, od koder izhaja tudi ime aksioma.

Aksiom o izključenem tretjem omogoča \emph{posredne} dokaze izjav, od
katerih je najbolj znano \textbf{dokazovanje s protislovjem}: pri tem ne
utemeljimo izjave $\phi$, ampak utemeljimo, zakaj $\lnot\phi$
\emph{ne} velja. Natančneje povedano, izjavo $\phi$ zamenjamo z njej
ekvivalentno izjavo $\lnot\lnot\phi$ in dokažemo $\lnot\lnot\phi$.
Dokaz ekvivalence $\phi \liff \lnot\lnot\phi$ sestoji iz dokazov dveh
implikacij:
%
\begin{mathpar}
  \inferrule
  {\inferrule{
      \inferrule{[\lnot\phi] \\ [\phi]}{\bot}
    }
    {\lnot\lnot\phi}
  }
  {\phi \lthen \lnot\lnot\phi}
  %
  \and
  %
  \inferrule*
  {\inferrule*
    {\inferrule*{ }{\phi \lor \lnot\phi} \\
     [\phi] \\
     \inferrule*{
       \inferrule*{
         [\lnot\lnot\phi] \\ [\lnot\phi]
       }
       {\bot}
     }
     {\phi}
    }
    {\phi}
  }
  {\lnot\lnot\phi \lthen \phi}
\end{mathpar}
%
V dokazu $\lnot\lnot\phi \lthen \phi$ smo uporabili aksiom o
izključenem tretjem. V matematičnem besedilu se dokaz s protislovjem
glasi:
%
\begin{quote}
  \it
  %
  Dokažimo $\phi$ s protislovjem.
  %
  \begin{itemize}
  \item[] Predpostavimo, da bi veljalo $\lnot\phi$.
  \item[] (Dokaz neresnice $\bot$.)
  \end{itemize}
  %
  Ker torej $\lnot\phi$ pripelje do protislovja, velja $\phi$.
\end{quote}
%
Praviloma izvemo o vsebini matematične izjave~$\phi$ več, če jo
dokažemo neposredno. Dokazovanja s protislovjem zato ni smiselno
uporabljati vsepovprek, ampak le takrat, ko je zares potreben ali ko
nam zelo olajša dokazovanje.

Ostali načini za sestavljanje posrednih dokazov slonijo na
ekvivalencah
%
\begin{mathpar}
  (\phi \lor \psi) \liff \lnot (\lnot\phi \land \lnot\psi),\and
  (\phi \lor \psi) \liff (\lnot\phi \lthen \psi),\and
  (\phi \lthen \psi) \liff (\lnot\psi \lthen \lnot\phi),\and
  (\all{x \in S} \phi) \liff \lnot \some{x \in S} \lnot \phi,\and
  (\some{x \in S} \phi) \liff \lnot \all{x \in S} \lnot \phi.
\end{mathpar}
%
V vseh petih primerih implikacija $\lthen$ iz leve na desno velja brez
uporabe aksioma o izključenem tretjem. Za dokaz implikacij
$\Leftarrow$ iz desne na levo pa potrebujemo aksiom o izključenem
tretjem.

\begin{vaja}
  Sestavi formalne dokaze za zgornjih pet ekvivalenc. Pri dokazovanju
  ekvivalenc za $\forall$ in $\exists$ si pomagaj s pravili sklepanja
  iz razdelkov~\ref{sec:univerzalni-kvantifikator}
  in~\ref{sec:eksistencni-kvantifikator}.
\end{vaja}

Povejmo, kako zgornje ekvivalence uporabimo v besedilu za posredni
dokaz izjave:
%
\begin{itemize}
\item $(\phi \lor \psi) \liff \lnot (\lnot\phi \land \lnot\psi)$
  uporabimo takole:
  %
  \begin{quote}
    \it
    %
    Dokazujemo $\phi \lor \psi$.
    %
    \begin{itemize}
    \item[] Predpostavimo, da velja $\lnot\phi$ in $\lnot\psi$.
    \item[] (Dokaz neresnice $\bot$.)
    \end{itemize}
    %
    Ker torej nista $\phi$ in $\psi$ oba neresnična, je eden od njiju
    resničen. Dokazali smo $\phi \lor \psi$.
  \end{quote}
\item $(\phi \lor \psi) \liff (\lnot\phi \lthen \psi)$ uporabimo
  takole:
  %
  \begin{quote}
    \it
    %
    Dokazujemo $\phi \lor \psi$.
    %
    \begin{itemize}
    \item[] Predpostavimo $\lnot\phi$.
    \item[] (Dokaz $\psi$.)
    \end{itemize}
    %
    Če torej ne velja $\lnot\phi$, velja $\psi$. Torej velja vsaj
    eden, zato smo dokazali $\phi \lor \psi$.
  \end{quote}
\item $(\phi \lthen \psi) \liff (\lnot\psi \lthen \lnot\phi)$
  uporabimo takole:
  %
  \begin{quote}
    \it
    %
    Dokazujemo $\phi \lthen \psi$.
    %
    \begin{enumerate}
    \item Predpostavimo $\lnot\psi$.
    \item (Dokaz $\lnot\psi$.)
    \end{enumerate}
    %
    Dokazali smo, da iz $\phi$ sledi $\psi$.
  \end{quote}
\item $(\all{x \in S} \phi) \liff \lnot \some{x \in S} \lnot \phi$
  uporabimo takole:
  %
  \begin{quote}
    \it
    %
    Dokazujemo, da za vsak $x \in S$ velja $\phi$.
    %
    \begin{enumerate}
    \item Predpostavimo, da obstaja $x \in S$, za katerega $\phi$
      \emph{ne} velja.
    \item (Dokaz neresnice $\bot$.)
    \end{enumerate}
    %
    Predpostavka, da obstaja $x \in S$, za katerega $\phi$ ne velja,
    pripelje do protislovja. Torej za vsak $x \in S$ velja $\phi$.
  \end{quote}
\item $(\some{x \in S} \phi) \liff \lnot \all{x \in S} \lnot \phi$
  uporabimo takole:
  %
  \begin{quote}
    \it
    %
    Dokazujemo, da obstaja tak $x \in S$, za katerega velja $\phi$.
    %
    \begin{enumerate}
    \item Predpostavimo, da bi veljalo $\lnot\phi$ za vse $x \in S$.
    \item (Dokaz neresnice $\bot$.)
    \end{enumerate}
    %
    Predpostavka, da velja $\lnot\phi$ za vse $x \in S$, pripelje do
    protislovja. Torej obstaja $x \in S$, za katerega velja $\phi$.
  \end{quote}
\end{itemize}

Negacijo poljubne izjave $\phi$ tvorimo preprosto tako, da pred njo
postavimo $\lnot$. Vendar nam to ne pove dosti o matematični vsebini
negirane izjave. V večini primerov je negacijo lažje razumeti, če
simbol~$\lnot$ ">porinemo"< navznoter do osnovnih izjav z uporabo
naslednjih ekvivalenc:
%
\begin{align*}
  \lnot (\phi \land \psi) &\iff \lnot\phi \lor \lnot\psi \\
  \lnot (\phi \lor \psi) &\iff \lnot\phi \land \lnot\psi \\
  \lnot (\phi \lthen \psi) &\iff \phi \land \lnot\psi \\
  \lnot (\lnot \phi) &\iff \phi \\
  \lnot (\all{x \in S} \phi) &\iff \some{x \in S} \lnot\phi \\
  \lnot (\some{x \in S} \phi) &\iff \all{x \in S} \lnot\phi
\end{align*}

\begin{zgled}
  Denimo, da bi radi ovrgli izjavo
  % 
  \begin{quote}
    ">Vsako zaporedje pozitivnih realnih števil ima limito~$0$."<
  \end{quote}
  % 
  Da izjavo ovržemo, moramo dokazati njeno negacijo. Načeloma lahko
  negacijo tvorimo tako, da pred izjavo napišemo ">ni res, da velja
  \dots"<, a nam to ne pove, kako bi negacijo dokazali. Zapišimo
  prvotno izjavo v delni simbolni obliki:
  % 
  \begin{equation}
    \label{eq:pozitivno-limita-0}
    \all{a \in \RR^\NN}{\text{$(a_n)_n$ pozitivno zaporedje}
      \lthen \text{$0$ je limita zaporedja $(a_n)_n$}}.
  \end{equation}
  % 
  Zgornja pravila za računanje negacije nam povedo, da se
  $\lnot\forall$ spremeni v $\exists\lnot$ in da se nato implikacija
  oblike $\phi \lthen \psi$ spremeni v $\phi \land \lnot\psi$. Tako
  izrazimo negacijo izjave~\eqref{eq:pozitivno-limita-0}:
  % 
  \begin{equation*}
    \some{a \in \RR^\NN}{\text{$(a_n)_n$ pozitivno zaporedje}
      \land \lnot (\text{$0$ je limita zaporedja $(a_n)_n$})}.
  \end{equation*}
  % 
  To preberemo z besedami:
  % 
  \begin{quote}
    ">Obstaja tako zaporedje $(a_n)_n$, da je $(a_n)_n$ zaporedje
    pozitivnih števil in da $0$ ni limita zaporedja $(a_n)_n$."<
  \end{quote}
  %
  Če se še malo potrudimo, preberemo bolj razumljivo:
  % 
  \begin{quote}
    ">Obstaja tako zaporedje pozitivnih realnih števil, da $0$ ni
    njegova limita."<
  \end{quote}
  %
  S tem še nismo končali, saj je tudi ">Število $0$ ni limita
  zaporedja $(a_n)_n$"< negacija. Izjavo ">$0$ je limita zaporedja
  $(a_n)_n$"< najprej zapišemo simbolno:
  % 
  \begin{equation}
    \label{eq:limita-0}
    \all{\epsilon > 0}
      \some{m}{\NN}
        \all{n \geq m}
          |a_n - 0| < \epsilon.
  \end{equation}
  % 
  Z zgornjimi pravili za negiranje izračunamo negacijo
  izjave~\eqref{eq:limita-0}. Operacijo $\lnot$ postopoma ">porivamo"<
  navznoter:
  % 
  % 
  \begin{align*}
    \lnot \all{\epsilon > 0} \some{m \in \NN} \all{n \geq m} |a_n
          - 0| < \epsilon & \iff
    \\
    \some{\epsilon > 0} \lnot \some{m}{\NN} \all{n \geq m}
          |a_n - 0| < \epsilon &\iff
    \\
    \some{\epsilon > 0} \all{m}{\NN} \lnot \all{n \geq m}
          |a_n - 0| < \epsilon &\iff
    \\
    \some{\epsilon > 0} \all{m}{\NN} \some{n \geq m}
          \lnot (|a_n - 0| < \epsilon) &\iff
    \\
    \some{\epsilon > 0} \all{m}{\NN} \some{n \geq m}
          |a_n - 0| \geq \epsilon &\iff
    \\
    \some{\epsilon > 0} \all{m}{\NN} \some{n \geq m}
          a_n \geq \epsilon.
  \end{align*}
  % 
  V zadnjem koraku smo upoštevali, da za pozitivno število $a_n$ velja
  $|a_n - 0| = |a_n| = a_n$. Tako smo dobili podrobno zapisano
  negacijo prvotne izjave
  % 
  \begin{quote}
    ">Obstaja tako zaporedje pozitivnih števil $(a_n)_n$ in obstaja
    tak $\epsilon > 0$, da za vsak $m \in \NN$ obstaja $n \geq m$, za
    katerega velja $a_n > \epsilon$."<
  \end{quote}
  % 
  To izjavo pa znamo dokazati tako, da podamo konkreten primer
  zaporedja $(a_n)_n$ in konkretno vrednost $\epsilon$, ki zadoščata
  pogoju, denimo $a_n = 2 + n$ in $\epsilon = 1$. Res, če je $m \in
  \NN$ poljuben, lahko vzamemo kar $n = m$, saj potem velja $a_n = a_m
  = 2 + m > 1 = \epsilon$.

  Pričujoči primer smo zapisali zelo podrobno. Izkušeni matematik tega
  seveda ne bo pisal, saj bo izračunal negacijo prvotne izjave kar v
  glavi in takoj podal primer zaporedja, ki dokazuje, da prvotna
  izjava ne velja.
\end{zgled}

%%%%%%%%%%%%%%%%%%%%%%%%%%%%%%%%%%%%%%%%%%%%%%%%%%%%%%%%%%%%%%%%%%%%%%
\section{Predikatni račun}
\label{sec:predikatni-racun}

Predikatni račun je tisti del logike, ki obravnava predikate ter
kvantifikatorja~$\forall$ in~$\exists$.

Predikate tvorimo z logičnimi operacijami in kvantifikatorji iz
\textbf{osnovnih predikatov}. Katere osnovne predikate imamo na voljo,
je odvisno od snovi, ki jo obravnavamo.\footnote{Na primer, če
  obravnavamo ravninsko geometrijo, potem so osnovni predikati ">točka
  $x$ leži na premici $y$"<, ">premici $p$ in $q$ se sekata"< itn.}
Vedno imamo na voljo tudi \textbf{enakost} $x = y$, ki jo bomo
obravnavali v razdelku~\ref{sec:enakost}.

V osnovnih predikatih nastopajo \textbf{izrazi} ali \textbf{termi}. Katere
izraze lahko tvorimo je spet odvisno od tega, katere konstante in
operacije imamo na voljo. Na primer, če obravnavamo aritmetiko celih
števil, so na voljo operacije $+$, $-$, $\times$, če pa obravnavamo
realna števila, so na voljo operacije $+$, $-$, $\times$, $/$. V
izrazih vedno lahko nastopajo \textbf{spremenljivke}. Kadar uporabimo
spremenljivko, moramo povedati njen \textbf{tip} oziroma \textbf{množico}
vrednosti, ki jih lahko zavzame spremenljivka. Pogosto je tip
spremenljivke razviden iz spremnega besedila ali iz ustaljene uporabe:
$n$ se uporablja za naravno število, $x$ za realno, $f$ za funkcijo
ipd.

Ponazorimo pravkar definirane pojme s primerom. Predikat
%
\begin{equation*}
  0 < f(x) \land f(x) < \pi/4 \lthen \sin(2 f(x)) = 1/3
\end{equation*}
%
je sestavljen s pomočjo logičnih operacij $\land$ in $\lthen$ iz treh
osnovnih predikatov, zgrajenih iz osnovnih relacij $<$ in $=$,
%
\begin{mathpar}
  0 < f(x)
  \and
  f(x) < \pi/4
  \and
  \sin(2 f(x)) = 1/3,
\end{mathpar}
%
v katerih nastopa pet izrazov:
%
\begin{mathpar}
  0
  \and
  f(x)
  \and
  \pi/4
  \and
  \sin(2 f(x))
  \and
  1/3
\end{mathpar}
%
V teh izrazih nastopa spremenljivka $x$, katere tip je množica
realnih števil (to moramo uganiti) in spremenljivka $f$, ki označuje
funkcijo iz realnih v realna števila (tudi to moramo uganiti).
Nadalje, v izrazih nastopajo konstante $0$, $\pi$, $4$, $2$,
$1$ in $3$, operacija $\sin$ in operacija množenja.


%%%%%%%%%%%%%%%%%%%%%%%%%%%%%%%%%%%%%%%%%%%%%%%%%%
\subsection{Proste in vezane spremenljivke}
\label{sec:spremenljivke}

V predikatih in izrazih se pojavljajo spremenljivke. Pri tem moramo
ločiti med \textbf{prostimi} in \textbf{vezanimi} spremenljivkami. Oglejmo
si naslednja izraza in predikat:
%
\begin{equation*}
  \sum_{i=0}^{n} a_i,
  \qquad
  \int_0^1 f(t) \, d t,
  \qquad
  \forall x \in A .\, \phi(x) \;.
\end{equation*}
%
V vsoti je vezana spremenljivka $i$, spremenljivki $n$ in $a$ sta
prosti. To pomeni, da je $i$ neke vrste ">lokalna
spremenljivka"<,\footnote{Podobnost z lokalnimi spremenljivkami v
  programskih jezikih ni zgolj naključje. Lokalna spremenljivka in
  števec v zanki sta tudi primera vezanih spremenljivk v teoriji
  programskih jezikov.} katere veljavnost je samo znotraj vsote,
medtem ko sta spremenljiki $n$ in $a$ veljavni tudi zunaj samega
izraza. Podobno je v integralu $t$ vezana spremenljivka in $f$ prosta,
v izjavi na desni pa je vezana spremenljivka $x$, spremenljivki $A$ in
$\phi$ sta prosti.

Vezane spremenljivke so ">nevidne"< zunaj izraza in jih lahko vedno
preimenujemo, ne da bi spremenili pomen izraza (seveda se novo ime ne
sme mešati z ostalimi spremenljivkami, ki nastopajo v izrazu): izraza
$\int_0^1 f(t)\, d t$ in $\int_0^1 f(x)\, d x$ štejemo za
\emph{enaka}, ker se razlikujeta le v imenu vezane spremenljivke.
Spremenljivki, ki ni vezana, pravimo \textbf{prosta}. Izrazu, v katerem
ni prostih spremenljivk, pravimo \textbf{zaprt izraz}. Zaprta
logična izjava se imenuje \textbf{stavek}.

Pomembno se je zavedati, da vezana spremenljivka ">zunaj"< svojega
območja ne obstaja. Matematiki so glede tega precej površni in na
primer pišejo
%
\begin{equation*}
  \int x^2 \, d x = x^3/3 + C,
\end{equation*}
%
kar je strogo gledano nesmisel. Na levi strani v integralu stoji
vezana spremenljivka~$x$, ki je na desni ">pobegnila"< iz integrala.
Še več, če je $x \in \RR$ in $C \in \RR$, potem je izraz $x^3/3 + C$
\emph{število} (odvisno od vrednosti $x$ in $C$), saj je vsota dveh
realnih števil. Na desni strani bi morala stati oznaka za
\emph{funkcijo}, recimo
%
\begin{equation*}
  \int x^2 \, d x = (x \mapsto x^3/3 + C),
\end{equation*}
%
vendar tega v praksi nihče ne piše. Seveda pri vsem tem ostane še
vprašanje, kakšno vlogo ima v zgornjem izrazu~$C$. Pri analizi se
učimo, da je~$C$ ">poljubna konstanta"<. Poskusimo to razumeti
natančno s stališča logike. Besedico ">poljubno"< ponavadi razumemo
kot ">za vsak"<, vendar to ne gre, saj je
%
\begin{equation*}
  \all{C \in \RR} \int x^2 \, d x = (x \mapsto x^3/3 + C)
\end{equation*}
%
nesemisel. Če bi to bilo res, bi veljalo za $C = 1$ in za $C = 2$, od
koder bi dobili
%
\begin{equation*}
  (x \mapsto x^3/3 + 1) =
  \int x^2 \, d x =
  (x \mapsto x^3/3 + 2).
\end{equation*}
%
Potemtakem bi morali biti funkciji $(x \mapsto x^3/3 + 1)$ in $(x
\mapsto x^3/3 + 1)$ enaki, od koder sledi nesmisel $1 = 2$. Težave
nastopajo iz dejstva, da poskušamo nedoločeni integral razumeti kot
operacijo, ki slika funkcije v funkcije, kar ni. Nedoločeni integral
preslika funkcijo~$f$ v \emph{množico} vseh funkcij $F$, za katere
velja $F' = f$. Če bi to želeli zapisati zares pravilno, bi dobili
%
\begin{equation*}
  \int x^2 \, d x =
  \set{(x \mapsto x^3/3 + C) \such C \in \RR}.
\end{equation*}
%
Ali naj torej sklepamo, da so matematiki pravzaprav zelo površni pri
pisanju integralov? Da, s stališča formalne logike prav gotovo. Vendar
to ni nujno slabo: matematični zapis v praksi služi ljudem za
sporazumevanje in prav je, da si izberejo tak zapis, s katerim najbolj
učinkovito komunicirajo drug z drugim. Kljub temu pa se je treba
zavedati, kdaj gredo matematiki ">po bližnjici"< in ne zapišejo ali
povedo vsega dovolj natančno, da bi to bilo sprejemljivo za standarde,
ki jih postavlja formalna logika.


%%%%%%%%%%%%%%%%%%%%%%%%%%%%%%%%%%%%%%%%%%%%%%%%%%
\subsection{Substitucija}
\label{sec:substitucija}

\textbf{Substitucija} je osnovna sintaktična operacija, v kateri
\emph{proste} spremenljivke zamenjamo z izrazi. Zapis
%
\begin{equation*}
  \xsubst{e}{x_1 \subto e_1, \ldots, x_n \subto e_n}
\end{equation*}
%
pomeni: ">v izrazu $e$ \emph{hkrati} zamenjaj proste spremenljivke
$x_1$ z $e_1$, $x_2$ z $e_2$, \dots in $x_n$ z $e_n$."<  Na primer,
%
\begin{equation*}
  \subst{x^2 + y}{x \subto 3, y \subto 5, z \subto 12}
\end{equation*}
%
je enako $3^2 + 5$. Nič hudega ni, če se v substituciji omenja
spremenljivko $z$, ki se v izrazu $x^2 + y$ ne pojavi.

Ko naredimo substitucijo, moramo paziti, da se proste spremenljivke ne
">ujamejo"<. Denimo, da želimo v integralu
%
\begin{equation*}
  \int_0^1 \frac{x}{a - x^2} \; dx
\end{equation*}
%
parameter $a$ zamenjati z $y^2$. To naredimo s substitucijo
%
\begin{equation*}
  \xsubst{\left(\int_0^1 \frac{x}{a - x^2} \; dx\right)}{a \subto y^2} =
  \int_0^1 \frac{x}{y^2 - x^2} \; dx.
\end{equation*}
%
Vse lepo in prav. Kaj pa, če želimo $a$ zamenjati z $1 + x$? Ker je
spremenljivka $x$ vezana v integralu, \emph{ne smemo} delati takole:
%
\begin{equation*}
  \xsubst{\left(\int_0^1 \frac{x}{a - x^2} \; dx\right)}{a \subto x^2} =
  \int_0^1 \frac{x}{x^2 - x^2} \; dx ?!
\end{equation*}
%
Ker vstavljamo v integral spremenljivko $x$, moramo vezano
spremenljivko $x$ najprej preimenovati v kaj drugega, na primer $t$,
šele nato vstavimo:
%
\begin{equation*}
  \xsubst{\left(\int_0^1 \frac{x}{a - x^2}\; dt \right)}{a \subto x^2} =
  \xsubst{\left(\int_0^1 \frac{t}{a - t^2} \; dt\right)}{a \subto x^2} =
  \int_0^1 \frac{t}{x^2 - t^2} \; dt.
\end{equation*}
%
Podajmo še nekaj primerov substitucij:
%
\begin{align*}
  \subst{x + y + 1}{x \subto 2} &= 2 + y + 1 \;,
  \\
  \subst{x + y^2 + 1}{x \subto y, y \subto x} &= y + x^2 + 1 \;
  \\
  \subst{\subst{x + y^2 + 1}{x \subto y}}{y \subto x} &=
  x + x^2 + 1 \;,
  \\
  \textstyle
  \subst{x + \int_0^1 x \cdot y \;, d x}{x \subto 2}
  &= \textstyle  2 + \int_0^1 x \cdot y \;, d x \;,
  \\
  \textstyle
  \subst{\int_0^1 x \cdot y \; d x}{y \subto x^2}
  &= \textstyle \int_0^1 t \cdot x^2 \; d t \;.
\end{align*}
%
Ločiti je treba med hkratno in zaporedno substitucijo:
%
\begin{align*}
  \subst{x + y^2}{x \subto y, y \subto x} &= y + x^2
  \\
  \subst{\subst{x + y^2}{x \subto y}}{y \subto x} &=
  \subst{y + y^2}{y \subto x} = x + x^2
  \\
  \subst{\subst{x + y^2}{y \subto x}}{x \subto y} &=
  \subst{x + x^2}{x \subto y} = y + y^2.
\end{align*}
%

V nadaljevanju bomo obravnavali pravila sklepanja za univerzalne in
eksistenčne kvantifkatorje, v katerih se pojavi substitucija. Ker je
sam zapis za substitucijo nekoliko nepregleden, bomo uporabili
nekoliko manj pravilen, a bolj praktičen zapis. Denimo, da imamo
logično formulo $\phi$, v kateri se morda pojavi spremenljivka $x$, ni
pa to nujno. Tedaj pišemo $\phi(x)$. Če želimo zamenjati $x$ z izrazom
$e$, zapišemo $\phi(e)$. To je pravzaprav običajni zapis, kot ga
uporabljajo matematiki za zapis funkcij, mi pa smo ga uporabili za
zapis logičnih formul. Če bi uporabili zapis s substitucijo, bi
formulo označili samo s $\phi$ namesto s $\phi(x)$ in zamenjavo s
$\xsubst{\phi}{x \subto e}$ namesto s $\phi(e)$. Zakaj je ta bolj
priročen zapis hkrati manj pravilen? V formalni logiki strogo ločimo
med \emph{simbolnim zapisom} matematičnega pojma, ki je zaporedje
znakov na papirju, in njegovim \emph{pomenom}, ki je matematična
abstrakcija. Substitucija $\xsubst{\phi}{x \subto e}$ nam pove, kako
niz znakov $\phi$ predelamo v novi niz znakov, torej deluje na novoju
simbolnega zapisa. Ko pišemo $\phi(x)$ pa si že predstavljamo, da je
$\phi$ matematična funkcija, ki deluje na argumentu $x$. S tem nastopi
zmešnjava med simbolnim zapisom in pomenom. Dokler se zmešnjave
zavedamo, je vse v redu.

\subsection{Univerzalni kvantifikator}
\label{sec:univerzalni-kvantifikator}

Univerzalna kvantifikacija $\all{x \in S} \phi$ se prebere ">Za vse $x$
iz $S$ velja $\phi$."< Pravili sklepanja sta
%
\begin{mathpar}
  \inferrule
  {\infer*{\phi(x)}{[x \in S]}}{\all{x \in S} \phi(x)} \ \text{($x$ svež)}
  \and
  \inferrule{\all{x \in S} \phi(x) \\ e \in S}{\phi(e)}
\end{mathpar}
%
pri čemer je $x$ spremenljivka, $\phi(x)$ logična formula in $e$ poljuben izraz.

V besedilu dokažemo se pravilo vpeljave zapiše:
%
\begin{quote}
  \it
  %
  Dokazujemo $\all{x \in S} \phi(x)$:
  %
  \begin{itemize}
  \item[] Naj bo $x \in S$ poljuben.
  \item[] (Dokaz, da velja $\phi(x)$).
  \end{itemize}
  %
  Dokazali smo $\all{x \in S} \phi(x)$.
\end{quote}
%
Pravilo uporabe v besedilu ponavadi ni eksplicitno navedeno, če pa bi
ga že zapisali, bi šlo takole:
%
\begin{quote}
  \it
  %
  Dokazujemo, da velja $\phi(e)$:
  \begin{itemize}
  \item[] (Dokaz, da velja $\all{x \in S} \phi(x)$.)
  \item[] (Dokaz, da velja $e \in S$.)
  \end{itemize}
  %
  Torej velja $\phi(e)$.
\end{quote}


Ob pravilu vpeljave stoji stranski pogoj, da mora biti spremenljivka
$x$ ">sveža"<. To pomeni, da se $x$ ne sme pojavljati drugje v dokazu,
saj bi sicer lahko prišlo do zmešnjave med vezanimi in prostimi
spremenljivkami. V besedilu se dejstvo, da je $x$ svež izraža z
besedico ">poljuben"< ali ">katerikoli"<. Primer, kako gredo stvari
narobe, če ne pazimo in pomešamo spremenljivke:

\begin{izrek}[z napako v dokazu]
  Če je $x$ večji od~$42$, so vsa realna števila večja od~$23$.
\end{izrek}

\begin{proof}
  Denimo, da bi nekoliko nerodno zapisali izrek simbolno takole:
  %
  \begin{equation*}
    x > 42 \lthen \all{x \in \RR} x > 23.
  \end{equation*}
  %
  To je sicer dovoljeno, saj se prosti $x$, ki stoji zunaj $\forall$
  ni ujel, ni pa preveč smotrno, ker smo na dobri poti, da bomo
  zunanji prosti $x$ in vezanega znotraj $\forall$ pomešali. Res, če
  ne upoštevamo pravila, da mora biti $x$ svež, dobimo tale nepravi
  ">dokaz"<:
  %
  \begin{equation*}
    \inferrule*
    {
      \inferrule*
      {\inferrule*
        {[x > 42] \\ 42 > 23}
        {x > 23}
      }
      {\all{x \in \RR} x > 23}
    }
    {x > 42 \lthen \all{x \in \RR} x > 23}
  \end{equation*}
  %
  Pri pravilu za vpeljavo $\forall$ smo uporabili spremenljivko $x$,
  ki pa je že nastopala v začasni hipotezi $x > 42$. Z besedilom bi se
  isti dokaz glasil takole:
  %
  \begin{quote}
    ">Dokazujemo $x > 42 \lthen \all{x \in \RR} x > 23$. Predpostavimo,
    da velja $x > 42$ in dokažimo $\all{x \in \RR} x > 23$. Naj bo $x
    \in \RR$. Po predpostavki je $x > 42$ in ker je $42 > 23$, od tod
    sledi $x > 3$."<
  \end{quote}
  %
  Če bi izrek zapisali bolje kot $x > 42 \lthen \all{y \in \RR} y >
    23$, težav ne bi bilo, saj bi se prejšnji dokaz ">zataknil"<:
  %
  \begin{quote}
    ">Dokazujemo $x > 42 \lthen \all{y \in \RR} y > 23$. Predpostavimo,
    da velja $x > 42$ in dokažimo $\all{y \in \RR} y > 23$. Naj bo $y
    \in \RR$. (Kaj zdaj? Lahko sicer dokažemo $x > 23$, a zares bi
    morali dokazati $y > 23$, kar ne gre.)"<
  \end{quote}
\end{proof}

Pogoj, da mora biti spremenljivka $x$ v pravilu za vpeljavo ">sveža"<,
se v praksi kaže v tem, da pri uvajanju nove spremenljivke izberemo
zanjo novo ime, ki se še ni pojavilo v dokazu.


\subsection{Eksistenčni kvantifikator}
\label{sec:eksistencni-kvantifikator}

Eksistenčna kvantifikacija $\some{x \in S} \phi$ se prebere ">obstaja
$x$ iz $S$, za katerega velja $\phi$"< ali ">za neki $x$ iz $S$ velja
$\phi$."< Pravili sklepanja za eksistenčni kvantifikator se glasita
%
\begin{mathpar}
  \inferrule
  {\phi(e) \\ e \in S}
  {\some{x \in S} \phi(x)}
  \and
  \inferrule
  {\some{x \in S} \phi(x)
    \\
    \infer*{\psi}{[x \in S \land \phi(x)]}}
  {\psi}\ \text{($x$ svež)}
\end{mathpar}
%
kjer je $e$ poljuben izraz in $x$ spremenljivka. Pri tem mora biti $x$
v pravilu uporabe svež. V besedilu pravilo vpeljave uporabimo takole:
%
\begin{quote}
  \it
  %
  Dokazujemo $\some{x \in S} \phi(x)$:
  %
  \begin{enumerate}
  \item (Skonstruiramo element $e \in S$.)
  \item (Dokažemo, da velja $\phi(e)$.)
  \end{enumerate}
  %
  Dokazali smo $\some{x \in S} \phi(x)$.
\end{quote}
%
Pravilo uporabe pa se v besedilu izraža takole:
%
\begin{quote}
  \it
  %
  Dokazujemo $\psi$:
  %
  \begin{enumerate}
  \item (Dokaz izjave $\some{x \in S} \phi(x)$.)
  \item Predpostavimo, da za $x \in S$ velja $\phi(x)$:
    %
    \begin{itemize}
    \item[] (Dokaz izjave $\psi$.)
    \end{itemize}
  \end{enumerate}
  %
  Dokazali smo $\psi$.
\end{quote}

\subsubsection{Enolični obstoj}
\label{sec:enolicni-obstoj}

Poleg običajnega eksistenčnega kvantifikatorja $\exists$ poznamo tudi
\emph{enolični} eksistenčni kvantifikator $\exists!$. Izjavo
$\exactlyone{x}{S}{\phi}$ preberemo ">obstaja natanko en $x \in S$, za
katerega velja $\phi(x)$"<.

Enolični eksistenčni kvantifikator ni osnovni logični operator, ampak
je $\exactlyone{x}{S}{\phi}$ le okrajšava za
%
\begin{equation}
  \label{eq:uniqe-exists}
  \some{x \in S} \phi(x) \land (\all{y \in S} \phi(y) \lthen x = y).
\end{equation}
%
Z besedami preberemo to izjavo takole: ">obstaja $x$ iz $S$, za
katerega velja $\phi(x)$ in za vsak $y \in S$ za katerega velja
$\phi(y)$ sledi $x = y$"<. To je samo zapleten način, kako povedati,
da obstaja natanko en element množice~$S$, ki zadošča pogoju $\phi$.

Pravilo sklepanja za vpeljavo enoličnega obstoja izpeljemo
iz~\eqref{eq:uniqe-exists}:
%
\begin{equation*}
  \inferrule{
    e \in S
    \\
    \phi(e)
    \\
    \infer*{y = e}{y \in S \land \phi(y)}
  }
  {\exactlyone{x}{S}{\phi}}
\end{equation*}
%
V besedilu dokažemo enolični obstoj takole:
%
\begin{quote}
  \it
  %
  Dokazujemo, da obstaja natanko en $x \in S$, za katerega velja
  $\phi(x)$:
  %
  \begin{enumerate}
  \item Obstoj: (Konstrukcija elementa $e \in S$ in dokaz, da velja $\phi(x)$.)
  \item Enoličnost: denimo da za $y \in S$ velja $\phi(y)$:
    %
    \begin{itemize}
    \item[] (Dokaz, da je $e = y$).
    \end{itemize}
  \end{enumerate}
  %
  Dokazali smo $\exactlyone{x \in S} \phi(x)$.
\end{quote}

Če dokažemo enolični obstoj $\exactlyone{x \in S} \phi(x)$, lahko
vpeljemo novo konstanto $c$, ki označuje tisti element iz $S$, ki
zadošča pogoju~$\phi$, pri čemer moramo seveda paziti, da znaka $c$
nismo že prej uporabili za kak drug pomen. Nova konstanta~$c$ je
opredeljena s praviloma
%
\begin{mathpar}
  \inferrule{ }{\phi(c)}
  \and
  \inferrule{
    y \in S
    \\
    \phi(y)
  }
  {y = c}
\end{mathpar}
%
Če v formuli $\phi$ poleg spremenljivke $x$ nastopajo še druge proste
spremenljivke, denimo $y_1, \ldots, y_n$, potem je nova konstanta~$c$
v resnici \emph{funkcija} parametrov $y_1, \ldots, y_n$.

\subsection{Enakost in reševanje enačb}
\label{sec:enakost}

Enakost $=$ je osnovna relacija, ki zadošča naslednjim aksiomom in
pravilom sklepanja:
%
\begin{mathpar}
  \inferrule{ }{a = a}
  \and
  \inferrule{a = b}{b = a}
  \and
  \inferrule{a = b \\ b = c}{a = c}
  \and
  \inferrule{\phi(a) \\ a = b}{\phi(b)}
\end{mathpar}
%
Po vrsti so so pravilo \emph{refleksivnosti}, \emph{simetrije},
\emph{tranzitivnosti} in \emph{zamenjave}. Zaenkrat enakosti ne bomo
posvečali posebne pozornosti, saj jo v praksi študenti dobro
obvladajo.

V osnovni iz srednji šoli se učimo pravil za reševanje enačb: enačbi
smemo na obeh straneh prišteti ali odšteti poljuben izraz, pomnožiti
ali deliti smemo s poljubnim \emph{neničelnim} izrazom, ipd. Od kod
izhajajo ta pravila? Kaj sploh pomeni, da smo enačbo ">rešili"<? Ko
rešimo kvadratno enačbo
%
\begin{equation*}
  x^2 - 5 x + 6 = 0
\end{equation*}
%
običajno zapišemo rešitev takole:
%
\begin{equation*}
  x_1 = 2, \quad x_2 = 3.
\end{equation*}
%
Kako naj to razumemo iz stališča matematične logike? Treba je
pojasniti dvoje: kaj pomenita $x_1$ in $x_2$, saj v prvotni enačbi
nastopa spremenljivka $x$, ter kako naj razumemo vejico med izjavama
$x_1 = 2$ in $x_2 = 3$. Z indeksoma $1$ in $2$ štejemo rešitve enačbe
in sta v resnici nepotrebna,\footnote{Kako pa bi zapisali rešitve
  enačbe $x_1^2 - 5 x_1 + 6 x = 0$?} na kar kaže tudi dejstvo, da
pišemo $x = \ldots$ in ne $x_1 = \ldots$, kadar je rešitev ena sama.
Torej bi lahko rešitev zapisali kot
%
\begin{equation*}
  x = 2, \quad x = 3.
\end{equation*}
%
Sedaj pa je tudi jasno, da bi namesto vejice morala stati disjunkcija,
se pravi
%
\begin{equation*}
  x = 2 \lor x = 3.
\end{equation*}
%
Začetna enačba in tako zapisana rešitev sta logično ekvivalentni:
%
\begin{equation*}
  x^2 - 5 x + 6 = 0 \iff
  x = 2 \lor x = 3.
\end{equation*}
%
Povzemimo: reševanje enačbe je postopek, s katerim dano enačbo $f(x) =
g(x)$ prevedemo v njen \emph{logično ekvivalentno} obliko $x = a_1
\lor x = a_2 \lor \cdots \lor x = a_n$, iz katere so neposredno razvidne
rešitve enačbe.

Pravila za reševanje enačb torej niso nič drugega kot recepti, s
pomočjo katerih enačbo predelamo v njen \emph{ekvivalentno} obliko, ki
je korak bližje končni obliki, v kateri bi radi zapisali rešitev. To
pojasnjuje srednješolska pravila za reševanje enačb. Na primer, za
realna števila $a, b, c \in \RR$ vedno velja
%
\begin{equation*}
  a = b \lthen c \cdot a = c \cdot b,
\end{equation*}
%
medtem ko obratna implikacija
%
\begin{equation*}
  c \cdot a = c \cdot b \lthen a = b
\end{equation*}
%
za splošne $a$ in $b$ velja le v primeru, ko je $c \neq 0$. Ker pri
reševanju enačb potrebujemo implikacijo v obe smeri, srednješolce
učimo, da smejo enačbo množiti samo z od nič različnimi števili.

\begin{vaja}
  Kako bi srednješolcem pojasnil, od kod izvira pravilo za množenje
  enačbe z neničelnim številom?
\end{vaja}

\begin{vaja}
  Enačbo $f(x) = g(x)$ smo ">rešili"< z zaporedjem korakov
  %
  \begin{align*}
    f(x) = g(x) &\liff \\
    f_1(x) = g_1(x) &\liff \\
    \vdots & \\
    f_k(x) = g_k(x) &\lthen \\
    f_{k+1}(x) = g_{k+1}(x) &\liff \\
    \vdots & \\
    x = a_1 \lor \cdots \lor x = a_n
  \end{align*}
  %
  kjer smo v $k$-tem koraku namesto ekvivalence pomotoma naredili
  implikacijo. Smo s tem dobili preveč ali premalo rešitev prvotne
  enačbe?
\end{vaja}



%%% Local Variables: 
%%% mode: latex
%%% TeX-master: "lmn"
%%% End: 

\chapter{Dokazovanje}\label{poglavje:dokazovanje}

        Matematične izsledke običajno podajamo preko jasno izraženih izjav. Med študijem matematike hitro opazite, da se takšne izjave podajajo pod imeni `izrek', `trditev', `lema', {posledica} in podobno. Kdaj uporabiti katerega teh imen ni natanko določeno, pač pa je prepuščeno presoji matematika. Približno vodilo je naslednje:
        \begin{itemize}
                \item
                        \df{izrek}: osrednji, bistven rezultat,
                \item
                        \df{trditev}: stranski rezultat,
                \item
                        \df{lema}: rezultat, ki sam po sebi nima toliko vsebine, se pa uporabi pri dokazovanju pomembnejšega rezultata,\footnote{Sicer ni nujno, da se resnična pomembnost izjav takoj pokaže. Mnogo je primerov, ko se kak matematični članek po določenem času začne ceniti ne toliko zaradi glavnega izreka, pač pa zaradi neke leme, ki se je za dokaz glavnega izreka uporabila.}
                \item
                        \df{posledica}: rezultat, ki je zanimiv sam po sebi, ki pa hitro sledi iz predhodne izjave.
        \end{itemize}

        Če skrbno analizirate izreke, trditve itd.~s predavanj (ali iz matematičnih člankov), opazite, da sestojijo iz treh delov: kontekst, predpostavke, sklepi.
        \begin{itemize}
                \item
                        \df{Kontekst} pove, katere objekte obravnavamo in kakšne vrste so.
                \item
                        \df{Predpostavke} so izjave, ki jih privzamemo.
                \item
                        \df{Sklepi} so izjave, ki jih (pri danih predpostavkah) dokazujemo.
        \end{itemize}

        Oglejmo si konkreten primer. Rolleov izrek je znan in uporaben izrek v analizi (če ga še niste spoznali, ga boste v kratkem).

        \begin{izrek}[Rolle]
                Naj bo $f$ realna funkcija, definirana na intervalu $\intcc{a}{b}$, kjer sta $a$ in $b$ realni števili in $a < b$. Če je $f$ zvezna na celem $\intcc{a}{b}$ in odvedljiva na odprtem intervalu $\intoo{a}{b}$ ter zavzame enaki vrednosti v krajiščih, tj.~$f(a) = f(b)$, tedaj ima $f$ stacionarno točko na $\intoo{a}{b}$.
        \end{izrek}

        Analizirajmo, kaj so kontekst, predpostavke in sklepi pri tem izreku.

        \begin{itemize}
                \item
                        Kontekst je sledeč:
                        \[a \in \RR, \qquad b \in \RR_{> a}, \qquad f \in \RR^{\intcc{a}{b}}.\]
                        To so objekti (in njihove vrste), o katerih govori izrek. Smiselno je, da jih zapišemo v tem vrstnem redu; na primer, $f$ zapišemo nazadnje, saj je njena domena odvisna od $a$ in $b$. Kadar imamo objekte, ki so neodvisni med sabo, jih lahko zapišemo v poljubnem vrstnem redu.
                \item
                        Predpostavke so tri. Vsako navedimo v običajnem jeziku in nato še s simbolnim matematičnim zapisom.
                        \begin{itemize}
                                \item
                                        $f$ je zvezna na $\intcc{a}{b}$.
                                        \[
                                                \hspace{-2em}
                                                \all{x \in \intcc{a}{b}}
                                                        \all{\epsilon \in \RR_{> 0}}
                                                                \some{\delta \in \RR_{> 0}}
                                                                        \all{y \in \intcc{a}{b}}
                                                                                (|x - y| < \delta \impl \big|f(x) - f(y)\big| < \epsilon
                                                                        )
                                        \]
                                \item
                                        $f$ je odvedljiva na $\intoo{a}{b}$.
                                        \begin{multline*}
                                                \all{x \in \intoo{a}{b}}
                                                        \some{v \in \RR}
                                                                \all{\epsilon \in \RR_{> 0}}
                                                                        \some{\delta \in \RR_{> 0}}
                                                \all{h \in \RR_{\neq 0}} \\
                                                        (|h| < \delta \implies \Big|\frac{f(x + h) - f(x)}{h} - v\Big| < \epsilon)
                                        \end{multline*}
                                \item
                                        $f$ na krajiščih intervala zavzame enaki vrednosti.
                                        \[f(a) = f(b)\]
                        \end{itemize}
                        Če se vam morda zdita formuli za zveznost in odvedljivost begajoči, imate dve tolažbi. Prva je ta, da se boste čez čas takšnih formul navadili. ;) Druga je, da so tudi drugi matematiki leni po naravi in zato uvedejo oznake za daljše izraze, ki se pogosto uporabljajo. Zgornja zveznost se na krajše zapiše $f \in \mathcal{C}(\intcc{a}{b})$ ($\mathcal{C}$ kot ``continuous'', tj.~zvezen), odvedljivost pa $f \in \mathcal{D}^1(\intoo{a}{b})$ ($\mathcal{D}$ kot ``differentiable'', tj.~odvedljiv, enka pa pomeni ``(vsaj) enkrat odvedljiv'').
                \item
                        Sklep je eden: $f$ ima stacionarno točko na $\intoo{a}{b}$, kar simbolno zapišemo takole.
                        \[\some{x \in \intoo{a}{b}} f'(x) = 0\]
        \end{itemize}

        V splošnem imamo določeno mero svobode, kako natančno razčleniti izrek. Na primer, za Rolleov izrek bi lahko kontekst zapisali tudi kot $a \in \RR, b \in \RR, f \in \RR^{\intcc{a}{b}}$ in pogoj $a < b$ dodali med predpostavke.

        Da ne bomo pisali dolgih seznamov, se dogovorimo za sledeče oznake. Izrek podamo tako, da najprej zapišemo kontekst, nato dvopičje, nato narišemo vodoravno črto, nad črto zapišemo predpostavke (ločene z vejicami), pod črto pa sklepe (ločene z vejicami). Rolleov izrek bi potemtakem povzeli takole.
        \[\claim{a \in \RR, b \in \RR_{> a}, f \in \RR^{\intcc{a}{b}}}{f \in \mathcal{C}(\intcc{a}{b}), f \in \mathcal{D}^1(\intoo{a}{b}), f(a) = f(b)}{\some{x \in \intoo{a}{b}} f'(x) = 0}\]

        V splošnem velja: vse proste spremenljivke, ki se pojavijo v predpostavkah ali sklepih, morajo biti navedene v kontekstu. Po domače povedano: če trdite, da za neko stvar nekaj velja, morate najprej povedati, o kateri stvari sploh govorite.

        Medtem ko je za težje matematične izreke potrebno obilo ustvarjalnosti, da se jih dokaže, pa lažje trditve pogosto lahko avtomatično dokažemo (dobesedno --- obstajajo avtomatični dokazovalniki \davorin{koliko povemo na to temo?}), pa tudi za težje je pomembno, da vemo, kako pristopiti k dokazu. Gre za to, da za vse logične veznike in kvantifikatorje obstajajo splošna pravila, kako ravnamo, če nastopajo kot predpostavke oziroma kot sklepi. To si bomo zdaj ogledali.

        \begin{itemize}
                \item\textbf{Konjunkcija}
                        \begin{itemize}
                                \item
                                        Če $p \land q$ nastopa kot \emph{predpostavka}:
                                        \begin{quote}
                                                predpostavko $p \land q$ nadomestimo s predpostavkama $p$, $q$ (to se pravi, pri dokazovanju lahko uporabimo tako predpostavko $p$ kot predpostavko $q$). S simboli, od trditve
                                                \[\claim{\Gamma}{\Pi', p \land q, \Pi''}{\Sigma}\]
                                                preidemo do trditve
                                                \[\claim{\Gamma}{\Pi', p, q, \Pi''}{\Sigma}\]
                                                (pri zapisih splošnih izrekov bomo kontekst označevali z $\Gamma$, predpostavke s $\Pi$ in sklepe s $\Sigma$).
                                        \end{quote}
                                \item
                                        Če $p \land q$ nastopa kot \emph{sklep}:
                                        \begin{quote}
                                                sklep $p \land q$ dokažemo tako, da dokažemo posebej $p$ in posebej $q$. S simboli:
                                                \[\claim{\Gamma}{\Pi}{\Sigma', p \land q, \Sigma''}\]
                                                preoblikujemo v
                                                \[\claim{\Gamma}{\Pi}{\Sigma', p, q, \Sigma''}\]
                                                (in se zavedamo, da je za dokaz izreka potrebno dokazati \emph{vse} sklepe).
                                        \end{quote}
                        \end{itemize}
                \item\textbf{Disjunkcija}
                        \begin{itemize}
                                \item
                                        Če $p \lor q$ nastopa kot \emph{predpostavka}:
                                        \begin{quote}
                                                ločimo primere: sklepe dokažemo posebej pri predpostavki $p$ (skupaj z ostalimi predpostavkami) in posebej pri predpostavki $q$ (skupaj z ostalimi). Torej, dokazati
                                                \[\claim{\Gamma}{\Pi', p \lor q, \Pi''}{\Sigma}\]
                                                pomeni isto, kot dokazati tako
                                                \[\claim{\Gamma}{\Pi', p, \Pi''}{\Sigma} \qquad \text{kot} \qquad \claim{\Gamma}{\Pi', q, \Pi''}{\Sigma}.\]
                                        \end{quote}
                                \item
                                        Če $p \lor q$ nastopa kot \emph{sklep}:
                                        \begin{quote}
                                                izberemo si enega od $p$, $q$ in ga dokažemo. Se pravi, če imamo
                                                \[\claim{\Gamma}{\Pi}{\Sigma', p \lor q, \Sigma''},\]
                                                si izberemo eno od trditev
                                                \[\claim{\Gamma}{\Pi}{\Sigma', p, \Sigma''} \qquad \text{oziroma} \qquad \claim{\Gamma}{\Pi}{\Sigma', q, \Sigma''}\]
                                                in jo izpeljemo.
                                        \end{quote}
                        \end{itemize}
                \item\textbf{Implikacija}
                        \begin{itemize}
                                \item
                                        Če $p \impl q$ nastopa kot \emph{predpostavka}:
                                        \begin{quote}
                                                če nam kadarkoli uspe izpeljati $p$, lahko dodamo $q$ med predpostavke. Torej, če znamo dokazati
                                                \[\claim{\Gamma}{\Pi', p \impl q, \Pi''}{q},\]
                                                potem za dokaz
                                                \[\claim{\Gamma}{\Pi', p \impl q, \Pi''}{\Sigma}\]
                                                zadostuje dokazati
                                                \[\claim{\Gamma}{\Pi', p \impl q, q, \Pi''}{\Sigma}\]
                                                (kar je lažje, ker imamo eno predpostavko več). To je smiselno: če vemo, da velja $p \impl q$ in dodatno ugotovimo, da velja $p$, potem vemo, da velja tudi $q$.
                                        \end{quote}
                                \item
                                        Če $p \impl q$ nastopa kot \emph{sklep}:
                                        \begin{quote}
                                                sklep $p \impl q$ nadomestimo s $q$, medtem ko $p$ dodamo med predpostavke. Pojasnimo. Trditev $p \impl q$ trdi nekaj samo v primeru, kadar $p$ velja --- v nasprotnem primeru je avtomatično resnična in ni ničesar za dokazati. Torej se lahko omejimo na primer, ko $p$ velja, se pravi, lahko predpostavimo $p$. Kadar $p$ velja, pa trditev $p \impl q$ pravi, da mora veljati tudi $q$. To pomeni, da pri predpostavki $p$ dokazujemo $q$. Simbolno, da dokažemo
                                                \[\claim{\Gamma}{\Pi}{\Sigma', p \impl q, \Sigma''},\]
                                                zadostuje dokazati
                                                \[\claim{\Gamma}{\Pi}{\Sigma', \Sigma''} \qquad \text{in} \qquad \claim{\Gamma}{\Pi, p}{q}.\]
                                        \end{quote}
                        \end{itemize}
                \item\textbf{Univerzalni kvantifikator}
                        \begin{itemize}
                                \item
                                        Če $\all{x \in X} \phi(x, y)$ nastopa kot \emph{predpostavka}:
                                        \begin{quote}
                                                če vemo za (ali med dokazom najdemo) katerikoli konkreten element $a \in X$, tedaj lahko med predpostavke dodamo $\phi(a, y)$. Namreč, če vemo, da lastnost $\phi$ (z morebitnimi nadaljnjimi parametri) velja za vse elemente množice $X$, potem ta lastnost velja za poljuben konkreten element. Simbolno, od
                                                \[\claim{\Gamma', a \in X, \Gamma''}{\Pi', \all{x \in X} \phi(x, y), \Pi''}{\Sigma}\]
                                                preidemo do
                                                \[\claim{\Gamma', a \in X, \Gamma''}{\Pi', \all{x \in X} \phi(x, y), \phi(a, y), \Pi''}{\Sigma}.\]
                                        \end{quote}
                                \item
                                        Če $\all{x \in X} \phi(x, y)$ nastopa kot \emph{sklep}:
                                        \begin{quote}
                                                v kontekst dodamo $x \in X$, sklep $\all{x \in X} \phi(x, y)$ pa nadomestimo s sklepom $\phi(x, y)$. S simboli, od
                                                \[\claim{\Gamma}{\Pi}{\Sigma', \all{x \in X} \phi(x, y), \Sigma''}\]
                                                preidemo do
                                                \[\claim{\Gamma, x \in X}{\Pi}{\Sigma', \phi(x, y), \Sigma''}\]
                                                Zakaj tako postopamo in kaj smo s tem pravzaprav naredili? Premislimo: želimo dokazati, da neka lastnost velja za vse elemente dane množice $X$. Če ima $X$ slučajno samo končno mnogo elementov, bi lahko lastnost preverili za vsakega posebej, ampak povečini delamo z neskončnimi množicami, kjer to ne deluje. Morda ima množica $X$ kakšno posebno lastnost, zaradi katere lahko univerzalni kvantifikator dokažemo na svojevrsten način (na primer, univerzalno kvantificirane izjave nad $\NN$ lahko dokazujemo z matematično indukcijo --- glej \note{razdelek o naravnih številih}), ampak to se zgodi v izjemnih primerih.

                                                V splošnem nimamo druge možnosti, kot da si izberemo simbol (tipično kar spremenljivko v kvantifikatorju), ki nam predstavlja poljuben, katerikoli element množice in zanj dokažemo želeno lastnost. Ideja je, da spremenljivka spet nastopa v vlogi ``škatlice'', kamor lahko vstavimo poljuben element množice $X$. Če nam je dokaz lastnosti uspel, ne da bi za spremenljivko predpostavili karkoli več, kot da predstavlja element množice $X$, tedaj dobimo dokaz lastnosti za katerikoli dejanski element množice $X$ tako, da v dobljeni dokaz namesto spremenljivke vstavimo ta element. Na ta način smo potem dejansko dobili dokaz lastnosti za vse elemente množice $X$.

                                                Besedni dokazi univerzalno kvantificirane izjave se zato tipično začnejo takole: ``Vzemimo poljuben $x \in X$. Dokažimo, da zanj velja dana lastnost.''
                                        \end{quote}
                        \end{itemize}
                \item\textbf{Eksistenčni kvantifikator}
                        \begin{itemize}
                                \item
                                        Če $\some{x \in X} \phi(x, y)$ nastopa kot \emph{predpostavka}:
                                        \begin{quote}
                                                v kontekst dodamo $x \in X$, eksistenčno predpostavko pa nadomestimo s $\phi(x, y)$. S simboli,
                                                \[\claim{\Gamma}{\Pi', \some{x \in X} \phi(x, y), \Pi''}{\Sigma}\]
                                                popravimo v
                                                \[\claim{\Gamma, x \in X}{\Pi', \phi(x, y), \Pi''}{\Sigma}.\]
                                                Zakaj to deluje? Naša predpostavka je, da v množici $X$ obstaja element z lastnostjo $\phi$ (z morebitnimi nadaljnjimi parametri). Torej si lahko vzamemo neki konkreten element množice $X$ s to lastnostjo, ki ga lahko uporabljamo kasneje v dokazu (za to ga moramo nekako označiti; v praksi ga tipično označimo kar z isto spremenljivko, kot v kvantifikatorju).
                                        \end{quote}
                                \item
                                        Če $\some{x \in X} \phi(x, y)$ nastopa kot \emph{sklep}:
                                        \begin{quote}
                                                da dokažemo eksistenčno izjavo, moramo podati neki konkreten element $x \in X$ in zanj dokazati dano lastnost $\phi(x, y)$. \davorin{Hm, kako točno to zapišemo simbolno v zgornji obliki?}
                                        \end{quote}
                        \end{itemize}
        \end{itemize}

        V zgornjem seznamu nismo omenili vseh veznikov in kvantifikatorjev. To je zato, ker jih pri dokazovanju nadomestimo z zgornjimi. Konkretno:
        \begin{itemize}
                \item
                        Za negacijo velja $\lnot{p} \equiv p \impl \false$. Med drugim to pomeni, da $\lnot{p}$ dokažemo na sledeči način: predpostavimo $p$ in iz tega izpeljemo neresnico.
                \item
                        Za ekvivalenco velja $p \lequ q \equiv (p \impl q) \land (p \revimpl q)$. To pomeni, da ekvivalenco dokažemo tako, da dokažemo implikacijo med $p$ in $q$ v obe smeri --- se pravi, enkrat predpostavimo $p$ in izpeljemo $q$, drugič pa predpostavimo $q$ in izpeljemo $p$.
                \item
                        Za veznike $\xor$, $\shf$, $\luk$ si preprosto izberemo eno od izražav z negacijo, konjunkcijo in disjunkcijo in nato delamo z njo.
                \item
                        Kvantifikator $\exactlyone{x \in X} \phi(x, y)$ ločimo na dva dela: na obstoj in enoličnost, in vsakega posebej dokažemo. Se pravi, skličemo se na izražavo
                        \[\exactlyone{x \in X} \phi(x, y) \equiv \some{x \in X} \phi(x, y) \land \all{a, b \in X} (\phi(a, y) \land \phi(b, y) \implies a = b).\]
                        Včasih je lažje, če najprej dokažemo obstoj elementa in ta element pri dokazu enoličnosti že uporabimo, torej dokazujemo izražavo
                        \[\exactlyone{x \in X} \phi(x, y) \equiv \some{x \in X} (\phi(x, y) \land \all{a \in X} (\phi(a, y) \implies a = x)).\]
        \end{itemize}

        Seveda ne bo možno dokazati vsakega izreka s slepim sledenjem zgornjim pravilom; včasih moramo uporabiti še kakšno dodatno strategijo. Spodnji dve sta zelo pogosti.
        \begin{itemize}
                \item
                        Med predpostavke dodamo trditev, za katero že vemo, da je resnična. Morda gre za trditev, ki smo jo že dokazali, morda pa gre kar za istorečje. Pogost primer tega je, da uporabimo zakon o izključenem tretjem in za dodatno predpostavko vzamemo $p \lor \lnot{p}$ (kjer je $p$ katerakoli konkretna izjava). Po zgornjih pravilih to potem pomeni, da ločimo primere in trditev dokažemo posebej pri predpostavki $p$ ter posebej pri predpostavki $\lnot{p}$.
                \item
                        Nekatere predpostavke ali sklepe nadomestimo z enakovrednimi izjavami. Na primer, velja
                        \[p \lor q \equiv \lnot(\lnot{p} \land \lnot{q}) \equiv \lnot{p} \impl q \equiv \lnot{q} \impl p.\]
                        To pomeni, da lahko disjunkcijo (poleg zgoraj omenjenega načina) dokažemo tudi tako, da predpostavimo, da nobena od možnosti ne velja, in od tod izpeljemo neresnico, ali pa predpostavimo, da ena od možnosti ne velja, in od tod izpeljemo drugo.

                        Zelo pogosta uporaba te ideje je \df{dokaz s protislovjem}, ki temelji na zakonu o dvojni negaciji $p \equiv \lnot\lnot{p}$. Izjavo torej lahko dokažemo tako, da predpostavimo njeno negacijo, in od tod izpeljemo neresnico. Tipičen besedni dokaz s protislovjem izgleda takole: ``Dokazujemo $p$. Pa recimo, da $p$ ne velja. Potem /neki sklepi/. To je v nasprotju s tem, kar smo dokazali prej, torej smo izpeljali protislovje. Se pravi, ni možno, da $p$ ne bi veljal, torej mora veljati.''
        \end{itemize}

        \note{mnogo zgovornih primerov dokazov, ki ponazorijo zgornje postopke}


\section{Vaje}


%%% Local Variables:
%%% mode: latex
%%% TeX-master: "ucbenik-lmn"
%%% End:

\chapter{Konstrukcije množic}

\section{Preprosti primeri}
\note{prazna množica, enojci}

\section{Podmnožice}
\davorin{Če \qt{embedding} prevajamo kot \qt{vložitev}, kako potem prevedemo \qt{inclusion}? Imamo sicer tujko \qt{inkluzija}, ampak fino bi bilo imeti še slovenski izraz. Vključitev?}

\section{Potenčna množica}
\davorin{Verjetno je smiselno, da ta razdelek sledi razdelku o podmnožicah. Morda kar združimo ta dva razdelka?}

\section{Družine množic}

\section{Produkt množic}

\section{Vsota množic}

\section{Unija in presek}

\section{Eksponentna množica}

\davorin{Vrstni red teh razdelkov bomo najbrž še premešali.}


%%% Local Variables:
%%% mode: latex
%%% TeX-master: "ucbenik-lmn"
%%% End:

\chapter{Preslikave}


\section{Prestavljeno iz matematičnega izražanje sem, začasno}
\label{sec:prest-iz-matem}



\section{Preslikave}

Množice ne obstajajo povsem ločene ena od druge, pač pa so med sabo povezane s \df{preslikavami} oziroma s tujko \df{funkcijami}\footnote{Nekateri uporabljajo izraz \qt{funkcija} samo za tiste preslikave, ki slikajo v realna ali kompleksna števila, ampak ta uporaba je že nekoliko zastarela. Dandanes večina matematiko besedo \qt{funkcija} obravnava kot sopomenko besede \qt{preslikava}. Tako jo bomo uporabljali tudi v tej knjigi.}. Posamična preslikava slika elemente ene množice po določenem predpisu v elemente druge množice.

Če je $f$ preslikava, ki slika iz množice $X$ v množico $Y$, to zapišemo $f\colon X \to Y$. Rečemo, da je množica $X$ \df{začetna množica} ali \df{domena} preslikave $f$, množica $Y$ pa je \df{ciljna množica} ali \df{kodomena} preslikave $f$.

Začetni množici ste v srednji šoli rekli tudi \qt{definicijsko območje}, ampak v tej knjigi bomo morali biti bolj previdni. Predpis za preslikavo po definiciji velja za vse elemente domene, ampak kasneje (v razdelku~\ref{razdelek:izpeljava-preslikav-iz-relacij}) obravnavamo delne preslikave, ki niso definirane na celi domeni; zanje je torej definicijsko območje manjše kot domena. Držimo se raje zgoraj danega poimenovanja.

Običaj je, da predpis preslikave podamo s pomočjo spremenljivke, tipično z oznako $x$. Na primer, če je $f$ preslikava kvadriranja, njen predpis zapišemo kot
\[f(x) = x^2.\]
Na tem mestu je potrebno poudariti več reči.
\begin{itemize}
\item
Velikokrat površno rečemo, da zgornji predpis podaja preslikavo. To ni povsem res --- to je zgolj predpis preslikave. Za to, da preslikavo v celoti podamo, je potrebno navesti tri stvari: poleg predpisa še domeno in kodomeno. Vse to je del informacije o preslikavi.

To se jasno pokaže, če začnemo razmišljati o lastnostih preslikav. Se še spomnite iz srednje šole, kaj pomeni, da je preslikava surjektivna? (Bomo ponovili v razdelku~\ref{razdelek:injektivnost-in-surjektivnost}.) Če vzamemo, da preslikava $f$ zadošča zgornjemu predpisu in jo obravnavamo kot preslikavo $f\colon \RR \to \RR$, ni surjektivna, če jo obravnavamo recimo kot preslikavo $f\colon \RR_{\geq 0} \to \RR_{\geq 0}$, pa je.
\item
Za spremenljivko $x$ velja isto, kot smo razpravljali že v prejšnjem razdelku pri lastnostih elementov množic: spremenljivka $x$ nima vnaprej določene vrednosti, pač pa predstavlja mesto, kamor lahko vstavimo poljubno vrednost. Seveda je potem vseeno, če vzamemo kakšno drugo črko ali čisto drug simbol: $f(y) = y^2$ določa isti predpis kot $f(x) = x^2$; prav tako $f(\heartsuit) = \heartsuit^2$. Se pravi, tudi v tem primeru gre za \note{dummy variable}. Če si torej izberemo neko vrednost, jo lahko vstavimo na mesto spremenljivke in poračunamo, npr.~$f(3) = 3^2 = 9$ oziroma $f(2\pi) = (2\pi)^2 = 4\pi^2$. Predstavljajte si, da je spremenljivka pravzaprav škatlica, kamor lahko vstavite vrednost, torej
\[f(\argbox) = \argbox^2.\]
\item
Alternativen način zapisa $f(x) = x^2$ je
\[f\colon x \mapsto x^2.\]
Pazimo: navadna puščica $\to$ podaja domeno in kodomeno, kot razloženo zgoraj. Repata puščica $\mapsto$ pa za posamičen element domene pove, v kateri element kodomene se preslika.

Zapis z repato puščico je še posebej uporaben, kadar želimo podati preslikavo, ne da bi nam bilo potrebno izbrati ime zanjo. Na primer, realno funkcijo kvadriranja lahko v celoti podamo takole:
\begin{align*}
\RR &\to \RR \\
x &\mapsto x^2
\end{align*}
(prva vrstica pove domeno in kodomeno, druga pa predpis). Tako podanim preslikavam potem rečemo \df{brezimne preslikave} (s tujko \df{anonimne funkcije}). Kasneje (v razdelku~\ref{razdelek:brezimne-preslikave}) bomo spoznali bolj strnjen zapis takih preslikav, ki je še posebej primeren za izvajanje operacij med preslikavami; takrat bomo takšno funkcijo zapisali kot $\xlam{x}[\RR]{x^2}[\RR]$.
\end{itemize}

\note{Sklop (kompozicija, kompozitum) preslikav. Identiteta kot enota za sklop. Razčlenitev (dekompozicija, faktorizacija) preslikav.}

\davorin{Definirati moramo tudi oznako $\set{f(x)}{x \in X}$, kar je druge vrste oznaka kot prej definirana $\set{x \in X}{\phi(x)}$. Se gremo primerjavo s Pythonom (razlika med \texttt{\{f(x) for x in X\}} in \texttt{\{x if phi(x)\}})? Smo matematični hipsterji in uvedemo oznako $\{f(x) \,|\, x \in X \,|\, \phi(x)\}$, ki ustreza \texttt{\{f(x) for x in X if phi(x)\}}, kar bi tudi prišlo prav?}

Zaenkrat smo imeli primere, ko je bil prepis preslikave dan z eno samo spremenljivko, npr.~$f(x) = x^2$. Zelo pogoste so pa tudi \df{preslikave več spremenljivk}, npr.~$f(x, y) = x^2 + y^2$. Že osnovne računske operacije so take --- na primer, pri seštevanju vzamemo \emph{dva} podatka in vrnemo rezultat (vsoto).

V takem primeru je smiselno reči: domena preslikave sestoji iz \df{dvojic} ali \df{parov} števil. Pri seštevanju je to, katero število je prvo, katero pa drugo, sicer nepomembno, pri kakšni drugi operaciji (npr.~že odštevanju), pa je, zato posebej zahtevajmo: gre za \df{urejene dvojice} (\df{pare}). Urejeno dvojico elementov $a$ in $b$ (v tem vrstem redu) po dogovoru zapišemo kot $(a, b)$. Vrednosti $a$ in $b$ imenujemo \df{komponenti} tega para; natančneje, $a$ je \df{prva komponenta}, $b$ pa \df{druga komponenta}.

Če imamo dve množici $A$ in $B$, tedaj množico vseh urejenih dvojic, katerih prva komponenta je element iz $A$, druga komponenta pa element iz $B$, označimo $A \times B$ in imenujemo \df{zmnožek} ali \df{produkt} množic $A$ in $B$. Glede na to, da obstaja mnogo operacij, ki se imenujejo \qt{produkt} (poznate že vsaj produkt števil, produkt števila z vektorjem, skalarni produkt vektorjev in vektorski produkt vektorjev, obstaja pa jih še precej več), je koristno produkt množic posebej poimenovati, da ga ločimo od drugih: zanj se je uveljavil izraz \df{kartezični produkt} (izhaja iz imena Cartesius, tj.~latinske različice priimka Renéja Descarta\footnote{René Descartes (1596 -- 1650) je bil francoski filozof, matematik in znanstvenik.}).

Seštevanje potemtakem lahko razumemo kot preslikavo $+\colon \RR \times \RR \to \RR$. V tem smislu še vedno gre za preslikavo, ki dan vhodni podatek preslika v neki rezultat, le da je vhodni podatek dvojica števil, ne pa zgolj eno število. Kadar imamo produkt več enakih faktorjev, ga lahko (kot običajno) zapišemo v obliki potence; pisali bi lahko tudi $+\colon \RR^2 \to \RR$.

Seveda nismo omejeni na preslikave samo ene ali dveh spremenljivk. Nič nam ne preprečuje definirati recimo $f(x, y, z) = 2x + y - 3z$. Smiselna domena te preslikave setoji iz \df{urejenih trojic} števil. V splošnem, če jemljemo elemente iz množic $A$, $B$, $C$, tedaj se množica vseh takih trojic označi z $A \times B \times C$. Prejšnji predpis določa potem preslikavo $f\colon \RR \times \RR \times \RR \to \RR$ (oziroma krajše $f\colon \RR^3 \to \RR$).

Spremenljivk je lahko še več; poleg dvojic in trojic tako dobimo še četverice, peterice, šesterice\ldots V splošnem takšna končna zaporedja elementov imenujemo \df{urejene večterice}. Tudi število spremenljivk je lahko označeno s črko; na primer, preslikava, ki računa povprečje $n$ števil (kjer $n \in \NN_{\geq 1}$), je dana kot
\begin{align*}
\RR^n &\to \RR \\
(x_1, x_2, \ldots, x_n) &\mapsto \frac{x_1 + x_2 + \ldots + x_n}{n}
\end{align*}
(če hočemo poudariti, da imajo naše večterice natanko $n$ komponent, jih imenujemo $n$-terice). Nadlega pri tem je sicer spet dvoumnost tropičja. Deloma jo je možno odpraviti tako, da celotno večterico označimo z eno spremenljivko. Pogosta izbira zapisa je $f(\bm{x})$ ali $f(\vec{x})$ (razlog za to je, da lahko večterico vidimo kot vektor).

Marsikdaj želimo delati ne samo z eno preslikavo, pač pa s celo množico preslikav naenkrat. Zato uvedemo: množica vseh preslikav, ki slikajo iz $X$ v $Y$, se označi kot $Y^X$; temu se reče \df{eksponent} množic $X$ in $Y$ (\note{na primernem mestu kasneje} bomo razložili, od kod ta oznaka).

\begin{zgled}
Množico vseh preslikav, ki realna števila slikajo nazaj v realna števila, označimo z $\RR^\RR$. Če nas zanimajo realne preslikave, ki so definirana samo na intervalu $\intoo{-1}{1}$, opazujemo množico $\RR^{\intoo{-1}{1}}$. Definiramo lahko preslikavo
\begin{align*}
\RR^{\intoo{-1}{1}} &\to \RR \\
f &\mapsto f(0),
\end{align*}
ki preslikavam priredi njihovo vrednost v točki $0$. Ta preslikava torej ima za argumente (tj.~vnose) celotne preslikave in ne števila! Sama po sebi je element množice $\RR^{\RR^{\intoo{-1}{1}}}$.
\end{zgled}

\begin{zgled}
Za poljubne množice $A$, $B$, $C$ lahko definiramo sledečo preslikavo, katere argumenti so pari preslikav.
\begin{align*}
B^A \times C^B &\to C^A \\
(f, g) &\mapsto g \circ f
\end{align*}
\end{zgled}


\davorin{Glede na to, da gre za slovenski učbenik, dajem izrazu `preslikava' prednost pred izrazom `funkcija'. Seveda pa sem pojasnil tudi slednji izraz (v prvem poglavju).}

\note{Uvod. Definicijsko območje in zaloga vrednosti \davorin{morda dodamo kot možno ime za zalogo vrednosti še prevod angleške besede `range', se pravi `razpon'?}. Zožitve (tako domene kot kodomene); oznake za to so $\rstr{f}_A$, $\rstr{f}^B$, $\rstr{f}_A^B$. Izvrednotenje (evalvacija) preslikave (če ne bomo tega pojasnili že pri eksponentih množic).}


\section{Brezimne preslikave}\label{razdelek:brezimne-preslikave}

\note{Tj.~anonimne oz.~čiste funkcije. Na tem mestu pride tudi $\lambda$-notacija in določena mera $\lambda$-računa.}

\davorin{Pripravil sem ukaza \texttt{$\backslash${lam}} in \texttt{$\backslash${xlam}}.}


\section{Slike in praslike}

Preslikava kot taka nam pove za posamične elemente, kam se slikajo. Marsikdaj pa nas zanima več: kam se slikajo celotne množice elementov. Na primer, zanima nas lahko, v kaj se projicira neko prostorsko telo na ravnino.

\note{luštna slika projekcije nekega prostorskega objekta na neko ravnino}

Da dobimo sliko celotne množice, moramo zbrati skupaj slike vseh posamičnih elementov množice. Smiselna je torej naslednja definicija.

\begin{definicija}\label{definicija:slika}
Naj bo $f\colon X \to Y$ preslikava. \df{Slika} množice $A \subseteq X$ je označena in definirana kot
\[\img{f}{A} \dfeq \set[1]{f(x)}{x \in A} = \set[1]{y \in Y}{\xsome{x}[A]{y = f(x)}}.\]
Ta predpis definira preslikavo $\img{f}\colon \pst(X) \to \pst(Y)$.
\end{definicija}

\begin{opomba}
Kot običajno, obstajajo različne oznake v uporabi. Sliko $\img{f}{A}$ se označuje tudi kot $f[A]$ ali celo kar kot $f(A)$. V slednjem primeru se predpostavlja zadostna matematična zrelost bralca, da zna razbrati, kdaj $f$ označuje preslikavo $f\colon X \to Y$, kdaj pa preslikavo $f\colon \pst(X) \to \pst(Y)$.

V tej knjigi se bomo načrtno izogibali takšnim dvoumnostim in za sliko dosledno uporabljali oznako iz definicije~\ref{definicija:slika}.
\end{opomba}

\begin{vaja}
Prepričaj se, da za poljubno preslikavo $f\colon X \to Y$ velja sledeče:
\begin{itemize}
\item
$\img{f}{X} = \rn{f}$,
\item
$\img{f}{\emptyset} = \emptyset$,
\item
$\img[1]{f}{\set{x}} = \set[1]{f(x)}$ za vsak $x \in X$.
\end{itemize}
\end{vaja}

\note{primeri in lastnosti slik že tu ali kasneje skupaj s primeri/lastnostmi praslik?}

Včasih pa imamo obratno nalogo: iz dane slike ugotoviti, kaj vse se je z neko preslikavo vanjo preslikalo. Zato vpeljemo še sledečo definicijo.

\begin{definicija}\label{definicija:praslika}
Naj bo $f\colon X \to Y$ preslikava. \df{Praslika} množice $B \subseteq Y$ je označena in definirana kot
\[\pim{f}{B} \dfeq \set[1]{x \in X}{f(x) \in B}.\]
Ta predpis definira preslikavo $\pim{f}\colon \pst(Y) \to \pst(X)$.
\end{definicija}

\begin{opomba}
Tudi za prasliko obstajajo različne oznake. Praslika $\pim{f}{B}$ se označi tudi kot $f^{-1}[B]$ ali kar kot $f^{-1}(B)$. V slednjem primeru se spet zanašamo na izkušenost bralca, da praslike $f^{-1}\colon \pst(Y) \to \pst(X)$ ne zamenja z obratom $f^{-1}\colon Y \to X$. Slednji morda sploh ne obstaja! (Praslika seveda obstaja za vse funkcije.)

Če inverz funkcije obstaja, tedaj velja $\pim[1]{f}{\set{y}} = \set[1]{f^{-1}(y)}$ za vsak $y \in Y$ (premisli!), kar nekoliko pojasni oznako $f^{-1}$ tudi za prasliko. Kljub vsemu, z namenom izogibanja dvoumnostim bomo se v tej knjigi skrbno držali oznake iz definicije~\ref{definicija:praslika} za prasliko.

Ko smo že pri alternativnih, potencialno zavajajočih oznakah: pri prasliki enojca se tipično spuščajo zaviti oklepaji, torej se namesto $\pim[1]{f}{\set{y}}$ piše $\pim{f}{y}$ (ali celo $f^{-1}(y)$).
\end{opomba}

\note{primeri, vaje}

\note{lastnosti: ohranjanje unij, presekov, komplementov}


\section{Injektivnost in surjektivnost}\label{razdelek:injektivnost-in-surjektivnost}

\note{Vključno z ekvivalenco z mono- in epimorfizmi.}


\section{Bijektivnost in obratne preslikave}\label{razdelek:bijektivnost-in-obratne-preslikave}

Kot dobro veste že iz srednje šole, injektivnost in surjektivnost nam omogočata definicijo bijektivnosti.

\begin{definicija}
Preslikava je \df{bijektivna}, kadar je injektivna in surjektivna.
\end{definicija}

To pomeni: če imamo bijektivno preslikavo (na kratko kar: \df{bijekcijo}) $f\colon X \to Y$, smo povezali elemente množice $X$ z elementi množice $Y$, in sicer tako, da vsakemu elementu v katerikoli od množic $X$ oz.~$Y$ pripišemo natanko en element druge množice.

\note{slika dveh množic s poparjenimi pikami}

Rečemo, da so elementi množice $X$ v \df{bijektivni korespondenci} (ali po slovensko \df{povratno enolični zvezi}) z elementi množice $Y$. Bijektivnost se na grafih kaže takole: preslikava je bijektivna, kadar vsaka vodoravnica seka njen graf natanko enkrat.

Bijektivne preslikave igrajo pomembno vlogo v matematiki. Oglejmo si tri primere.
\begin{itemize}
\item
Če imamo povratno enolično zvezo med elementi dveh množic, je jasno, da imata isto število elementov. To nam omogoča definicijo \df{kardinalnosti} množic --- glej poglavje~\note{o kardinalnosti}.
\item
Predstavljajmo si, da so elementi neke množice $X$ imena za določene objekte. Na bijektivno preslikavo $f\colon X \to Y$ lahko potem gledamo kot na preimenovanje teh objektov. Seveda preimenovanje ne spremeni narave (ali če hočete natančnejši izraz, matematične strukture) objektov --- z drugimi besedami, $X$ in $Y$ se razlikujeta zgolj po imenih svojih elementov. To nas privede do pojma \df{izomorfizma}. Za več podrobnosti glej poglavje~\note{o strukturiranih množicah}.
\item
Če imamo povratno enolično zvezo med elementi množic $X$ in $Y$, potem ta zveza ne podaja zgolj preslikave v smeri $X \to Y$, pač pa tudi v smeri $Y \to X$, ker za vsak element iz $Y$ obstaja enolično določen element iz $X$, ki se vanj preslika. Z drugimi besedami, bijektivne preslikave imajo \df{obrate}.
\end{itemize}

Povejmo več o obratih preslikav. Začnimo s formalno definicijo.

\begin{definicija}
Naj bo $f\colon X \to Y$ poljubna preslikava. Za preslikavo $g\colon Y \to X$ rečemo, da je \df{obrat} ali \df{inverz} preslikave $f$, kadar velja
\[g \circ f = \id[X] \qquad\qquad \text{in} \qquad\qquad f \circ g = \id[Y].\]
Z drugimi besedami, $g$ je obrat $f$, kadar slika v nasprotni smeri in za vsak $x \in X$ velja $g\big(f(x)\big) = x$ ter za vsak $y \in Y$ velja $f\big(g(y)\big) = y$. Kadar obrat preslikave $f$ obstaja, rečemo, da je $f$ \df{obrnljiva} (ali \df{invertibilna}) preslikava.
\end{definicija}

\begin{zgled}\label{zgled:logaritmiranje-je-obratno-od-eksponenciranja}
Kot veš že iz srednje šole, logaritmiranje je obratno od eksponenciranja. Če smo natančnejši: preslikavi $\xlam{x}[\RR]{b^x}[\RR_{> 0}]$ in $\xlam{x}[\RR_{>0}]{\log_b x}[\RR]$ sta si obratni pri vsaki osnovi $b \in \RR_{> 0} \setminus \set{1}$.
\end{zgled}

\begin{vaja}\label{vaja:enolicnost-obrata-preslikave}
Dokaži: če sta $g$ in $h$ obrata iste preslikave $f$, tedaj $g = h$.
\end{vaja}

Vaja~\ref{vaja:enolicnost-obrata-preslikave} pove, da je obrat funkcije enolično določen, tj.~vsaka funkcija ima kvečjemu en obrat. Zato lahko uvedemo izrecno oznako: obrat preslikave $f$ (kadar obstaja) označimo z $f^{-1}$. Velja torej: kadar je preslikava $f\colon X \to Y$ obrnljiva, določa preslikavo $f^{-1}\colon Y \to X$.

Ta oznaka je nekoliko nerodna --- pomembno se je zavedati, da $f^{-1}(x)$ pomeni obrat preslikave $f$, izvrednoten na $x$, medtem kot $\big(f(x)\big)^{-1}$ pomeni obratna vrednost (v smislu deljenja) izvrednotenja preslikave $f$ na $x$. Za primerjavo, kot omenjeno v zgledu~\ref{zgled:logaritmiranje-je-obratno-od-eksponenciranja}, je obrat eksponenciranja logaritmiranje, medtem ko je obratna vrednost od $b^x$ enaka $(b^x)^{-1} = \frac{1}{b^x} = b^{-x}$.

\begin{vaja}
Premisli: če ima preslikava $f$ obrat $f^{-1}$, tedaj je tudi $f^{-1}$ obrnljiva preslikava in velja $(f^{-1})^{-1} = f$ (torej, obrat obrata je izvorna preslikava).
\end{vaja}

\begin{vaja}
Pogosto rečemo, da sta seštevanje in odštevanje obratni operaciji. Strogo vzeto, ti dve operaciji nista obratni kot preslikavi, saj obe slikata (recimo, da ju gledamo na realnih številih) $\RR \times \RR \to \RR$, tj.~ne slikata v nasprotnih smereh. Ugotovi, v kakšnem smislu točno sta seštevanje in odštevanje obratni, tj.~kateri dve preslikavi sta pravzaprav druga drugi obratni.
\end{vaja}

Zakaj se sploh ukvarjamo z obrati? Pogosto obravnavamo preslikavo, ki izhaja iz nekega konkretnega (na primer fizikalnega) problema, v smislu, da preslikava vzame začetne podatke in nam vrne, kaj se bo na koncu zgodilo. Marsikdaj pa hočemo rešiti obraten problem: želimo določene končne rezultate in se sprašujemo, kaj morajo biti začetni pogoji, da jih bomo dosegli. V takem primeru pride prav obratna preslikava.

Kot omenjeno, obrat preslikave je enoličen. Ne velja pa, da za poljubne preslikave sploh obstaja. Na primer, naj bo $f$ edina možna preslikava $\set{0, 1} \to \set{\unit}$, torej tista, ki tako $0$ kot $1$ preslika v $\unit$. Nobena preslikava $g\colon \set{\unit} \to \set{0, 1}$ ne more biti obrat preslikave $f$, saj je $g \circ f$ gotovo konstantna in potemtakem ne more biti identiteta na $\set{0, 1}$.

Kdaj torej obstaja obrat preslikave?

\begin{trditev}
Za poljubno preslikavo $f\colon X \to Y$ sta ekvivalentni sledeči trditvi.
\begin{enumerate}
\item
Preslikava $f$ je obrnljiva.
\item
Preslikava $f$ je bijektivna.
\end{enumerate}
\end{trditev}

\begin{dokaz}
\begin{implproof}{1}{2}
Predpostavljamo, da obstaja obrat $f^{-1}$.

Dokažimo, da je $f$ injektivna. Vzemimo poljubna $x, y \in X$, za katera velja $f(x) = f(y)$. Tedaj $x = f^{-1}\big(f(x)\big) = f^{-1}\big(f(y)\big) = y$.

Dokažimo, da je $f$ surjektivna. Vzemimo poljuben $y \in Y$. Tedaj $y = f\big(f^{-1}(y)\big)$.
\end{implproof}
\begin{implproof}{2}{1}
Če je $f$ bijekcija, za vsak $y \in Y$ velja, da je $\pim[1]{f}{\set{y}}$ enojec (glej \note{ustrezne predhodne trditve v razdelku o injektivnosti in surjektivnosti}). Definirajmo $g\colon Y \to X$ na naslednji način: za vsak $y \in Y$ naj bo $g(y)$ tisti element $x \in X$, za katerega velja $\pim[1]{f}{\set{y}} = \set{x}$. \note{Iz lastnosti praslike sledi, da je $g$ obrat $f$.}
\end{implproof}
\end{dokaz}

Iz dokaza te trditve vidimo, da bi bilo koristno imeti oznako za \qt{tisti element}, če želimo podajati tovrstne preslikave s simboli. Naj bo $\phi$ lastnost elementov množice $X$ (torej predikat $\phi\colon X \to \tvs$), ki je resnična za natanko en element. Dogovorimo se, da
\[\xthat{x}[X]{\phi(x)}\]
pomeni \qt{tisti (edini) element množice $X$, ki ima lastnost $\phi$} (simbolček na začetku je mala grška črka jota). Zdaj lahko izrecno zapišemo: če je $f\colon X \to Y$ bijekcija, tedaj je njen obrat $f^{-1}\colon Y \to X$ dan s predpisom
\[f^{-1}(y) = \that[1]{x}[X]{f(x) = y}.\]

\davorin{Andrej, omenjal si, da želiš imeti to oznako. Če sem kaj zgrešil, prosim popravi.}

Zaenkrat smo to joto uporabljali zgolj kot okrajšavo za stavek v običajnem jeziku, ampak če želimo $\iota$-izraze uporabljati v matematičnih dokazih, jim moramo dati natančen matematični pomen. Definirajmo torej joto formalno matematično.

Naj bo $X$ poljubna množica. Na njej imamo enakost; obravnavajmo jo na tem mestu kot lastnost dvojic elementov iz $X$, torej kot predikat $=_X\colon X \times X \to \tvs$ (za vsak par elementov vrnemo resničnostno vrednost, ali sta komponenti para enaki). Transponirajmo to preslikavo; dobimo $\transposed{=_X}\colon X \to \tvs^X$. Ta transponiranka je injektivna: če se za $a, b \in X$ preslikavi $\xlam{x}[X]{a = x}$ in $\xlam{x}[X]{b = x}$ ujemata, se ujemata tudi njuni vrednosti pri $b$. Ker drži $b = b$, potem drži tudi $a = b$.

Če zožimo kodomeno preslikave $\transposed{=_X}$ na njeno sliko, potemtakem dobimo bijekcijo. Naj bo jota njen obrat, torej $\iota \dfeq \big(\rstr{\transposed{=_X}}^{\rn{\transposed{=_X}}}\big)^{-1}$. V tem smislu je zgornja oznaka $\xthat{x}[X]{\phi(x)}$ okrajšava za $\iota\big(\xlam{x}[X]{\phi(x)}\big)$ (kar bi seveda lahko še skrajšali do $\iota(\phi)$, ampak v praksi je to običajno manj zgovorno).


%%% Local Variables:
%%% mode: latex
%%% TeX-master: "ucbenik-lmn"
%%% End:

\chapter{Relacije}\label{POGLAVJE: Relacije}

	V matematiki pogosto želimo izraziti, da so določeni objekti v nekem odnosu, npr.~eno število je večje od drugega; temu s tujko rečemo \df{relacija}. Kako to formalno izraziti? Ideja je, da relacijo podamo z množico vseh skupin elementov, ki so v relaciji. Na primer, relacijo $\leq$ na naravnih številih podamo kot podmnožico
	\[\set[1]{(a, b) \in \NN \times \NN}{\xsome{n}[\NN]{a + n = b}}.\]
	Torej, število $a$ je v relaciji $\leq$ s številom $b$ takrat, ko par $(a, b)$ pripada tej množici.
	
	Splošne relacije so lahko med poljubno mnogo elementi iz poljubnih (ne nujno istih) množic. Na primer, relacija komplanarnosti štirih točk v prostoru je podmnožica produkta $\RR^3 \times \RR^3 \times \RR^3 \times \RR^3$, relacija pripadnosti $\in$ med elementi neke množice $X$ in podmnožicami množice $X$ pa je podmnožica produkta $X \times \pst(X)$.
	
	Splošna definicija relacije je potemtakem naslednja.
	\begin{definicija}
		\df{Relacija} na družini množic $\mathscr{D}$ je podmnožica produkta $\prod_{X \in \mathscr{D}} X$.
	\end{definicija}
	
	V praksi se povečini uporabljajo relacije med dvema elementoma.
	\begin{definicija}
		\df{Dvojiška relacija}\footnote{Oziroma s tujko \df{binarna relacija}.} med elementi množic $X$ in $Y$ je podmnožica produkta $X \times Y$. \df{Dvojiška relacija} na množici $X$ je podmnožica produkta $X \times X$.
	\end{definicija}
	
	Skoraj vse relacije, ki nas zanimajo v tej knjigi, so dvojiške. Zato se dogovorimo, da z izrazom \qt{relacija} vselej mislimo dvojiško relacijo, razen če je izrecno rečeno drugače.
	
	Če je $R \subseteq X \times Y$ relacija, potemtakem lahko zapišemo, da sta $x \in X$ in $y \in Y$ v relaciji $R$ takole: $(x, y) \in R$. Ampak to vodi do čudnih zapisov v primeru običajnih relacij, npr.~$(2, 3) \in <$. To seveda raje zapišemo kot $2 < 3$ in posledično se dogovorimo, da v primeru dvojiške relacije raje uporabljamo zapis $x \mathrel{R} y$.
	
	
	\section{Grafi relacij}
	
		\GraphInit[vstyle = Normal]
		\tikzset
		{
			EdgeStyle/.append style = {->, bend left}
		}
		
		Relacije na majhnih množicah lahko lepo ponazorimo z usmerjenimi grafi. Graf relacije $R \subseteq X \times X$ je definiran takole: vozlišča grafa so elementi množice $X$ in za vsaka dva elementa $a, b \in X$, za katera velja $a \mathrel{R} b$, narišemo puščico od $a$ do $b$.
		
		\begin{zgled}
			Naj bo $X = \set{A, B, C, D, E, F}$ in naj bo
			\[R \dfeq \set{...}\]
			relacija na $X$. Njen graf izgleda takole.
			\begin{center}
				\begin{tikzpicture}
					\SetGraphUnit{3}
					\Vertex[Math=true, x=0, y=0]{A}
					\Vertex[Math=true, x=3, y=2]{B}
					\Vertex[Math=true, x=2, y=-3]{C}
					\Vertex[Math=true, x=6, y=1]{D}
					\Vertex[Math=true, x=8, y=-1]{E}
					\Vertex[Math=true, x=10, y=2]{F}
					
					\Edge(A)(B)
					\Loop[dist = 5em, dir = EA](B)
				\end{tikzpicture}
			\end{center}
		\end{zgled}
		\note{izgled grafa je še treba popraviti}
	
	
	\section{Operacije z relacijami}\label{RAZDELEK: Operacije z relacijami}
	
		Običajno je, da iz že danih matematičnih objektov lahko skonstruiramo nove preko določenih operacij. Z relacijami ni nič drugače; v tem razdelku si bomo ogledali običajne operacije na relacijah.
		
		Ker so relacije podmnožice, imamo za začetek vse operacije na podmnožicah. Torej, za poljubno družino $(R_i)_{i \in I}$ podmnožic produkta $X \times Y$ sta tudi unija $\bigcup_{i \in I} R_i$ in presek $\bigcap_{i \in I} R_i$ relaciji. Če je $R \subseteq X \times Y$ relacija, je njena komplementarna relacija $\complement{R} = X \times Y \setminus R \ \subseteq \ X \times Y$.
		
		Posebej imamo \df{prazno relacijo} $\emptyset \subseteq X \times Y$ (nobena dva elementa nista v relaciji) in \df{polno relacijo} $X \times Y\subseteq X \times Y$ (vsaka dva elementa sta v relaciji), ki sta si medsebojno komplementarni.
		
		Poleg operacij, ki jih relacije podedujejo od podmnožic, imamo še operacije, ki upoštevajo produktno strukturo.
		
		Če so $X$, $Y$, $Z$ množice in $R \subseteq X \times Y$, $S \subseteq Y \times Z$ relaciji, tedaj je \df{sklop} (\df{kompozitum}) \df{relacij} definiran kot
		\[S \circ R \dfeq \set[1]{(x, z) \in X \times Z}{\some{y}[Y]{x \mathrel{R} y \land y \mathrel{S} z}}\]
		(po vzoru funkcij tudi kompozicijo relacij pišemo v obratnem vrstnem redu; glej razdelek~\ref{RAZDELEK: Funkcije kot funkcijske relacije}). Sklapljanje je asociativna operacija, torej pri sklopu večih relacij oklepaji niso pomembni.
		
		Večkraten sklop relacije $R \subseteq X \times X$ same s sabo označimo
		\[R^n \dfeq \underbrace{R \circ R \circ \ldots \circ R}_{\text{$n$ $R$-jev}}\]
		za $n \in \NN_{\geq 2}$. Seveda je smiselno definirati, da je $R^1$ enak $R$ in da je $R^0$ relacija enakosti na množici $X$, saj je to enota za sklapljanje relacij na $X$, tj.~$=_X \circ R = R = R \circ =_X$ (premisli, da je to res!).
		
		Za poljubno relacijo $R \subseteq X \times Y$ definiramo \df{obratno} (\df{inverzno}) \df{relacijo} kot
		\[R^{-1} \dfeq \set{(y, x) \in Y \times X}{x \mathrel{R} y}.\]
		Posledično lahko za poljubno relacijo $R \subseteq X \times X$ definiramo njeno potenco s poljubno celo stopnjo: $R^{-n} \dfeq (R^{-1})^n = (R^n)^{-1}$.
		
		\begin{zgled}
			Naj bo $L$ množica ljudi. Vpeljimo oznake za naslednje relacije na $L$:
			\begin{itemize}
				\item
					$\texttt{St}$ je relacija \qt{je starš od},
				\item
					$\texttt{Oč}$ je relacija \qt{je oče od},
				\item
					$\texttt{Ma}$ je relacija \qt{je mati od},
				\item
					$\texttt{Si}$ je relacija \qt{je sin od},
				\item
					$\texttt{Hč}$ je relacija \qt{je hči od},
				\item
					$\texttt{Br}$ je relacija \qt{je brat od},
				\item
					$\texttt{Se}$ je relacija \qt{je sestra od}
			\end{itemize}
			
			Na primer: Marko $\texttt{Br}$ Metka pomeni \qt{Marko je brat od Metke.} (oz.~v lepši slovenščini \qt{Marko je Metkin brat.}).
			
			Velja med drugim:
			
			\begin{tabular}{l}
				$\texttt{Oč} \cup \texttt{Ma} = \texttt{St}$, \\
				$\texttt{St} \circ \texttt{St} = \texttt{St}^2 = \text{\qt{je stari starš od}}$, \\
				$\texttt{St} \circ \texttt{Br} = \text{\qt{je stric od}}$, \\
				$\texttt{Br} \cup \texttt{Se} = \text{\qt{je sorojenec od}}$, \\
				$\texttt{St}^{-1} = \text{\qt{je otrok od}}$, \\
				$\bigcup_{n \in \NN_{\geq 1}} \texttt{St}^n = \text{\qt{je prednik od}}$, \\
				$\bigcup_{n \in \NN_{\geq 1}} \texttt{St}^{-n} = \text{\qt{je potomec od}}$, \\
				$\texttt{St} \circ (\texttt{Br} \cup \texttt{Se}) \circ \texttt{Hč} = \text{\qt{je sestrična od}}$.
			\end{tabular}
			
			Sklapljanje relacij ni komutativno; na primer $\texttt{Ma} \circ \texttt{Oč}$ je stari oče po materini strani, $\texttt{Oč} \circ \texttt{Ma}$ pa stara mama po očetovi strani.
			
			\note{V tem zgledu sicer predpostavljamo, da je vsaka oseba bodisi moškega bodisi ženskega spola, kar ni čisto res. Ima kdo kakšno idejo, kako se temu izogniti (in še vedno imeti lahko razumljiv zgled)?}
		\end{zgled}
	
	
	\section{Lastnosti relacij}
		\note{Med drugim lastnosti relacij, izražene z operacijami. Mogoče združimo s prejšnjim razdelkom?}
	\section{Funkcije kot funkcijske relacije}\label{RAZDELEK: Funkcije kot funkcijske relacije}
	\section{Relacije urejenosti}
		\note{Vključno z urejenostnimi strukturami. Vključno z morfizmi?}
	\section{Ekvivalenčne relacije}
	\section{Kvocientne množice}
		\note{Sem dodajmo kanonični razcep funkcije (na surjekcijo/kvocient, bijekcijo, injekcijo/vložitev).}
\chapter{Strukture}


Informacija, ki jo posamična množica podaja, je zgolj, katere elemente vsebuje. Izkušnje hitro pokažejo, da ta informacija ni najbolje naravnana za matematično delo. Po eni strani je del te informacije pogosto odveč: tipično si lahko z neko množico pomagamo enako, če njene elemente preimenujemo, tj.~če obravnavamo izomorfno množico. Po drugi strani pa je te informacije premalo: ni dovolj, da vemo, katere elemente imamo na voljo, želimo vedeti tudi, kaj lahko s temi elementi počnemo. Podatek o tem imenujemo \df{struktura} te množice.

Vzemimo za primer množico realnih števil~$\RR$. Njene elemente lahko poljubno seštevamo, odštevamo in množimo, tj.~izvajamo določene operacije na njih (seveda imamo še cel kup drugih operacij, vključno z delnimi, kot so deljenje, potenciranje, logaritmiranje\ldots). Strukturo, ki je dana z operacijami, imenujemo \df{algebrska} (ali \df{algebrajska} ali \df{algebraična}).

Množico lahko opremimo tudi z raznimi relacijami, tipično z relacijami urejenosti. Na primer, na $\RR$ imamo relaciji $\leq$ in $<$. To imenujemo \df{struktura urejenosti} (ali \df{urejenostna struktura}).

Realna števila si lahko predstavljamo kot točke na številski premici. Vidimo, da lahko potem računamo razdaljo med njimi. Pravimo, da realna števila tvorijo \df{metrični prostor} oziroma da imajo realna števila \df{metrično strukturo}.

Za realne intervale tudi znamo povedati, kdaj so odprti oz.~zaprti. Kadar imamo pojem odprtosti oz.~zaprtosti, to imenujemo \df{topološka struktura}. Prav tako znamo povedati dolžino intervalov. Kadar imamo pojem velikosti podmnožic, to imenujemo \df{merska struktura}.

Te in še nadaljnje strukture boste podrobneje spoznavali pri raznih matematičnih predmetih, v tej knjigi pa se bomo osredotočili zgolj na nekatere osnovne algebrske in urejenostne strukture.

Tipično velja: več kot imamo strukture na neki množici, bolj uporabna je (še zlasti, kadar se strukture med sabo prepletajo --- na primer, dejstvo, da je seštevanje na $\RR$ monotono, povezuje algebrsko in urejenostno strukturo na $\RR$). Ker imajo realna števila tako bogato strukturo, ni presenetljivo, da jih kar naprej uporabljamo. Za primerjavo: množico vseh permutacij $n$ elementov, ki se imenuje simetrična grupa in označi z $S_n$, uporabljate redkeje (je pa še vedno uporabna, saj premore nekaj operacij --- permutacije lahko sklapljamo in obračamo).

Množico, opremljeno z neko strukturo, imenujemo \df{strukturirana množica}. V tem kontekstu golo množico (brez njene dodatne strukture) imenujemo \df{nosilna množica} (te strukture).

Proučevanje struktur je ena temeljnih matematičnih dejavnosti. Na primer, pri predmetu Algebra spoznavate algebrske strukture, pri Topologiji topološke strukture, pri Analizi metrične in gladke strukture itd.

Za proučevanje strukture pa ne zadostuje opazovati zgolj množic, opremljenih s to strukturo, pač pa tudi preslikave med njimi, ki to strukturo na smiseln način ohranjajo. Tovrstnim preslikavam rečemo \df{homomorfizmi}. Kaj točno to pomeni, bomo spoznali pri konkretnih strukturah v nadaljevanju tega poglavja.

\note{nekje (ne nujno tu) debata, kako strukturirano množico podamo preko njene karakterizacije --- potrebna obstoj in enoličnost do izomorfizma}


\section{Algebrske strukture}

Kot rečeno, algebrska struktura je struktura, dana z operacijami. Operacije, na katere ste navajeni, imajo \df{mestnost}, tj.~koliko podatkov (ki jih imenujemo \df{argumenti} ali \df{operandi}) sprejmejo, da vrnejo rezultat. Na primer, seštevanje vzame dva podatka (seštevanca ali sumanda), ki ju zapišemo na levo in desno stran plusa, da dobimo rezultat (vsoto). Seštevanje je torej dvomestna operacija.

Odštevanje je prav tako dvomestna operacija --- od zmanjševanca odštejemo odštevanec in dobimo razliko. To je dvomestni minus, imamo pa tudi enomestni minus, ki vzame število in vrne njegovo nasprotno število. To sta dve različni operaciji in posledično imate zanju tudi dve različni tipki na kalkulatorju. Dvomestni minus je običajno označen kot $-$, enomestni pa kot ${}^+/_-$\;.

Še en primer enomestne operacije je faktoriela: za vsak $n \in \NN$ lahko naračunamo $n!$, kar je spet naravno število. Primer tromestne operacije je mešani produkt vektorjev v trorazsežnem prostoru: za poljubne tri vektorje je njihov mešani produkt število, katerega absolutna vrednost pove prostornino paralelepipeda, ki ga ti vektorji razpenjajo, predznak pa pove orientacijo tega paralelepipeda.

V splošnem je $n$-mestna operacija na množici $A$ dana kot preslikava $A^n \to A$, vsaj ko gre za operacijo, ki tako vzame kot vrne podatke iz množice $A$ --- taki operaciji rečemo \df{notranja}. Če to ne velja, je operacija \df{zunanja}. Vektorji lepo ponazorijo razliko. Seštevanje vektorjev v prostoru je preslikava $\RR^3 \times \RR^3 \to \RR^3$, torej dvomestna notranja operacija. Množenje vektorjev s skalarji $\RR \times \RR^3 \to \RR^3$ je dvomestna zunanja operacija, kjer enega od argumentov vzamemo iz neke druge množice (v tem primeru iz $\RR$). Skalarno množenje $\RR^3 \times \RR^3 \to \RR$ je prav tako dvomestna zunanja operacija, le da je tokrat rezultat iz druge množice. Prej omenjeni mešani produkt je tromestna zunanja operacija $\RR^3 \times \RR^3 \times \RR^3 \to \RR$.

V definiciji $n$-mestne operacije lahko vzamemo tudi $n = 0$. Ničmestna (notranja) operacija je torej preslikava $\one \to A$, se pravi izbira elementa iz $A$.

Obstajajo še splošnejše vrste operacij (npr.~takšne, ki so odvisne od neskončno argumentov), ampak v tej knjigi se ne bomo ukvarjali z njimi.


\subsection{Osnovne algebrske strukture}

Operacije, s katerimi imamo najpogosteje opravka, so tipično dvomestne. Če želimo obravnavati takšne operacije na splošno, si definiramo strukturo, ki zajema zgolj eno tako operacijo.

\begin{definicija}
	\df{Magma} je množica, opremljena z dvomestno notranjo operacijo.
\end{definicija}

Strukturirano množico običajno zapišemo tako, da znotraj okroglih oklepajev najprej zapišemo simbol za nosilno množico, nato pa naštejemo vse sestavne dele strukture (ločene z vejicami). Če imamo strukturo magme na množici $A$ in dano operacijo označimo z $\oper$, tedaj to magmo zapišemo kot $(A, \oper)$. Če hočemo poudariti, da je $\oper$ dvomestna notranja operacija, lahko še natančneje zapišemo $(A,\ \oper\colon A \times A \to A)$.

\note{polgrupe, monoidi, grupe}

\note{homomorfizmi}

\subsection{Polkolobarji}
\subsection{Kolobarji}
\subsection{Obsegi}
\section{Strukture urejenosti}
\subsection{Mreže}
\subsection{Boolove mreže}
\section{Kategorije}


%%% Local Variables:
%%% mode: latex
%%% TeX-master: "ucbenik-lmn"
%%% End:

\chapter{Številske množice}

Številske množice (naravna števila, cela števila, \ldots) poznate že od nekdaj. O njih imate zadosti občutka oz.~intuitivne predstave, da jih lahko uporabljate in pridete do pravilnih rezultatov. Tudi v tej knjigi smo jih že kar naprej izkoriščali za razne primere.

Ampak intuitivna predstava je tudi vse, kar zaenkrat imamo o številskih množicah. Nismo še podali natančne matematične definicije zanje, na osnovi katere bi lahko neizpodbitno dokazovali izreke o njih.

Za vajo lahko sami premislite, ali bi znali na tem mestu podati natančno definicijo, kaj pomeni biti naravno, celo, racionalno oz.~realno število. Definicija seveda mora biti natančna --- npr.~reči, da so realna števila tista, ki ležijo na številski premici, ni zadovoljiva definicija (vsaj ne, če ne pojasnite nedvoumno, kaj pomeni ``številska premica'' in kaj pomeni ``ležati'' na njej).

V tem poglavju se bomo sistematično lotili obravnave najpogosteje uporabljanih številskih množic. Podali bomo njihove konstrukcije, karakterizacije in temeljne lastnosti.


\section{Naravna števila}

\subsection{Peanovi aksiomi}

Če vas kdo vpraša, kako dobiti vsa naravna števila, verjetno odgovorite nekaj v naslednjem smislu: naravna števila so $0$ in vsa tista števila, ki jih dobite s prištevanjem enice, tj.~jemanjem naslednika. Torej, začnemo z $0$, vzamemo naslednika in dobimo $1$, nato še enkrat vzamemo naslednika in dobimo $2$ itd.

Prvi, ki je znal to intuitivno predstavo preliti v natančno matematično definicijo, je bil Peano\footnote{Giuseppe Peano (1858 -- 1932) je bil italijanski matematik.} komaj dobro stoletje nazaj. Pogoje, ki jih zahtevamo za neko množico, da jo lahko imenujemo ``množica naravnih števil'', po njem imenujemo \df{Peanovi aksiomi}. \davorin{Nekje bomo predebatirali, kaj je aksiom in zakaj jih uporabljamo. Peanove aksiome povežimo s tem.}

Če boste brskali po literaturi, boste naleteli na mnogo različnih inačic Peanovih aksiomov. Mi bomo izbrali sledečo jedrnato različico.

\begin{definicija}[Peano]\label{definicija:naravna-stevila}
\df{Množica naravnih števil} je množica (običajno označena z $\NN$), skupaj z izbranim njenim elementom (običajno označenim z $0$, kar beremo ``ničla'' ali ``nič'') in preslikavo na tej množici (običajno označeno z $\suc\colon \NN \to \NN$, ki jo imenujemo ``naslednik''), kadar veljajo naslednje lastnosti:
\begin{itemize}
\item
$\suc$ je injektivna preslikava,
\item
$0 \notin \rn{\suc}$,
\item
velja načelo \df{matematične indukcije}: če je $\phi$ predikat na $\NN$, za katerega velja
\[\phi(0) \qquad\qquad \text{in} \qquad\qquad \all{n \in \NN} (\phi(n) \implies \phi\big(\suc(n)\big)),\]
tedaj $\phi$ velja za vse elemente $\NN$.
\end{itemize}
\end{definicija}

Poskusimo si zdaj natančno pojasniti pomen teh pogojev.

S pomočjo elementa $0$ in preslikave $\suc$ lahko v nedogled generiramo elemente množice $\NN$. Začnemo z $0$, nato vzamemo naslednika in dobimo $\suc(0)$, nato vzamemo naslednika tega elementa in dobimo $\suc(\suc(0))$, nato naslednika $\suc(\suc(\suc(0)))$ itd. Takšen zapis je sicer precej nepraktičen --- si predstavljate, da rečete ``dobimo se čez naslednika od naslednika od naslednika od naslednika od naslednika ničle ur'' (namesto ``dobimo se čez pet ur'')? Zato sprejmemo dogovor: $\suc(0)$ označimo krajše z $1$ in preberemo ``ena'', $\suc(\suc(0))$ označimo z $2$ in preberemo ``dve'' in tako naprej.\footnote{Trenutno dogovorjena sistematična imena za števila gredo do \df{centiljona}, ki ga zapišemo z enico, ki ji sledi 600 ničel (vsaj pri nas; ponekod po svetu centiljon pomeni enica s 303 ničlami). To pomeni, da lahko sistematično izrazimo števila do $10^{606}-1$ (= devetsto devetindevetdeset centiljard devetsto devetindevetdeset centiljonov devetsto devetindevetdeset novemnonagintiljard\ldots). Nekateri razširijo to lestvico še z nadaljnjimi latinskimi izpeljankami, obstajajo pa tudi posebna imena za nekatera posamična velika števila, na primer \df{gugol} za $10^{100}$ (od tod izhaja ime spletnega brskalnika Google).}

Smo na ta način dobili neskončno različnih elementov $\NN$? Če ne bi zahtevali zgornjih pogojev, to ne bi bilo nujno. Lahko bi se namreč zaciklali (v smislu, da je naslednik nekega elementa element, ki smo ga že prej navedli).

Včasih je takšno zaciklanje nekaj, kar dejansko hočemo. Na primer, pri algebri boste spoznali tako imenovane \df{ciklične grupe}. Ciklično grupo z $n$ elementi označimo $\ZZ_n$, njene elemente pa kar z $0, 1, \ldots, n-1$. Spodaj je slika ciklične grupe $\ZZ_5$.

\note{slika usmerjenega grafa, ki predstavlja $\ZZ_5$}

Puščice označujejo, kako slika naslednik v tej grupi: naslednik $0$ je $1$, naslednik $1$ je $2$, naslednik $2$ je $3$, naslednik $3$ je $4$, nato pa se zacikla in naslednik $4$ je $0$.

Pogoj $0 \notin \rn{\suc}$ reče, da nič ni naslednik nobenega naravnega števila. Na ta način se izognemo, da bi naravna števila tvorila ciklično grupo.

Obstaja pa še en način, kako se lahko jemanje naslednika zacikla. Vzemimo spodnji primer.

\note{slike polgrupe $\set{0, \ldots, 4}$, ki se zacikla $4 \to 2$}

Nasledniki se lahko zaciklajo tudi pri elementu, ki ni $0$. V danem primeru je naslednik $0$ element $1$, naslednik $1$ je $2$, naslednik $2$ je $3$, naslednik $3$ je $4$, naslednik $4$ pa je $2$.

Zakaj naravna števila niso taka? Ker v danem primeru $\suc$ ni injektivna preslikava. Pogoj o injektivnosti nam v bistvu pove sledeče: naravna števila se ne morejo zaciklati pri nobenem nasledniku.

Vidimo, da se naravna števila ne morejo zaciklati niti na začetku (pri $0$) niti nekje vmes v verigi naslednikov --- torej gredo v nedogled, kot želimo. Z drugimi besedami, $0$, $\suc(0)$, $\suc(\suc(0))$, $\suc(\suc(\suc(0)))$,\ldots so medsebojno različni elementi množice $\NN$ in naravnih števil je posledično neskončno.

Čemu pa služi zadnji pogoj iz definicije~\ref{definicija:naravna-stevila}, tj.~načelo o indukciji? Že brez tega pogoja vemo, da so $0$, $\suc(0)$, $\suc(\suc(0))$, $\suc(\suc(\suc(0)))$,\ldots naravna števila, česar pa ne vemo, je, da so to \emph{vsa} naravna števila --- da torej ni nobenih drugih.

\begin{vaja}
Premisli, da množica $\RR_{> -1}$ z naslednikom $\suc(x) \dfeq x+1$ zadošča vsem pogojem iz definicije~\ref{definicija:naravna-stevila}, razen načelu indukcije.
\end{vaja}

Vidimo, da bi brez načela indukcije lahko imeli v množici $\NN$ odvečna števila (takšna, ki jih ne štejemo kot naravna). S predpostavko o indukciji se to ne more zgoditi. Ta namreč pravi: če neka lastnost velja za začetni element verige $0$, $\suc(0)$, $\suc(\suc(0))$, $\suc(\suc(\suc(0)))$,\ldots in če lahko sklepamo, da kakor hitro ta lastnost velja za določen element verige, velja tudi za naslednjega, potem ta lastnost velja za vsa naravna števila. Če za lastnost vzamemo ``biti element te verige'', iz načela o indukciji sklenemo, da se vsako naravno število nahaja nekje v tej verigi. Peanovi aksiomi torej podajajo strukturo, ki ustreza naši intuitivni predstavi množice naravnih števil.

Glede na to, da je načelo o matematični indukciji eden od osnovnih aksiomov, s katerimi so naravna števila podana, ne preseneča, da je indukcija eden najpogostejših načinov, kako dokazujemo izjave na naravnih številih. Natančneje rečeno, z matematično indukcijo dokazujemo univerzalno kvantificirane izjave na naravnih številih, torej izjave oblike
\[xall{n \in \NN} \phi(n).\]
Po načelu indukcije za dokaz take izjave zadostuje narediti naslednje. Najprej dokažemo
\[\phi(0)\]
(da torej lastnost $\phi$ velja za začetno naravno število). To imenujemo \df{temelj} ali \df{osnova} ali \df{baza} indukcije. Nato dokažemo izjavo
\[\all{n \in \NN} \phi(n) \implies \phi(\suc(n));\]
to imenujemo \df{indukcijski korak}. Z besedami, dokažemo, da kakor hitro velja lastnost $\phi$ za neko naravno število, mora veljati ta lastnost tudi za naslednje.

Intuitivno je jasno, da to mora delovati. Temelj indukcije nam pove, da dana lastnost velja za $0$. Ker zdaj vemo, da velja za $0$, mora po indukcijskem koraku veljati za naslednika ničle, torej za $1$. Zdaj vemo, da velja za $1$, torej mora po indukcijskem koraku veljati tudi za $2$. Tako nadaljujemo: sklepamo, da lastnost velja za $3$, nato za $4$ in tako naprej. Ker se vsa naravna števila pojavijo v verigi naslednikov ničle, mora z indukcijo dokazana lastnost dejansko veljati za vsa naravna števila.

V poglavju~\ref{poglavje:indukcija} se bomo vrnili k indukciji, jo natančneje preučili in si ogledali primere dokazovanja z njo. Na tem mestu pa jo bomo uporabili za izpeljavo \emph{rekurzije}, ki nam bo služila za definicijo nadaljnje strukture na naravnih številih.

\subsection{Rekurzija}

Poenostavljeno povedano, rekurzija pomeni, da določimo vrednost preslikave pri nekem argumentu iz (že prej naračunanih) vrednosti pri manjših argumentih. Tipičen primer rekurzivno podane preslikave je faktoriela: če zapišemo $0! \dfeq 1$ in $n! \dfeq (n+1) \cdot n!$ za vse $n \in \NN$, smo s tem enolično podali preslikavo $!\colon \NN \to \NN$.

Naračunajmo nekaj vrednosti te preslikave. Neposredno iz definicije dobimo $0! = 1$ --- to je \df{temelj} oz.~\df{osnova} oz.~\df{baza} rekurzije. Od tod s pomočjo \df{rekurzijskega koraka} izpeljemo
\[1! = 1 \cdot 0! = 1 \cdot 1 = 1.\]
S pomočjo te vrednosti in z rekurzijskim korakom lahko naračunamo vrednost faktoriele pri naslednjem naravnem številu.
\[2! = 2 \cdot 1! = 2 \cdot 1 = 2\]
In tako naprej.
\[3! = 3 \cdot 2! = 3 \cdot 2 = 6\]
\[4! = 4 \cdot 3! = 4 \cdot 6 = 24\]
\[5! = 5 \cdot 4! = 5 \cdot 24 = 120\]
\[\vdots\]
Vidimo, da lahko po tem postopku prej ali slej naračunamo $n!$ za poljuben $n \in \NN$.

V primeru faktoriele smo neko vrednost naračunali iz predhodne, uporabljajo se pa tudi splošnejše rekurzivne definicije, kjer vrednost naračunamo iz večih prejšnjih. Slovit primer je \df{Fibonaccijevo zaporedje} $F\colon \NN \to \NN$, podano kot $F_0 \dfeq 0$, $F_1 \dfeq 1$ in $F_{n+2} \dfeq F_{n+1} + F_n$ za vse $n \in \NN$. Od tod lahko naračunamo:
\begin{align*}
F_0 &= 0, \\
F_1 &= 1, \\
F_2 &= F_1 + F_0 = 1 + 0 = 1, \\
F_3 &= F_2 + F_1 = 1 + 1 = 2, \\
F_4 &= F_3 + F_2 = 2 + 1 = 3, \\
F_5 &= F_4 + F_3 = 3 + 2 = 5, \\
F_6 &= F_5 + F_4 = 5 + 3 = 8, \\
&\vdots
\end{align*}
Bo pa za naše potrebe zaenkrat zadostovala oblika rekurzije, kjer se skličemo samo na en predhodni člen, in na tako se bomo v tej knjigi tudi omejili. \davorin{Lahko pa vseeno v kakšni vaji zahtevamo od študentov, da zapišejo in dokažejo splošnejše načelo rekurzije.}

Zakaj bi pa sploh podajali preslikave rekurzivno namesto z izrecnim (eksplicitnim) predpisom? Včasih to sledi iz narave problema. Na primer, imamo stanje, ki se razvija korak za korakom, kjer je trenutno stanje odvisno od prejšnjega. Zanima nas, kako se naš sistem razvija, in v tem primeru je naravno podati trenutno stanje sistema kot rekurzivno preslikavo. \note{ponazorimo s primerom}

Včasih preslikavo podamo rekurzivno, ker je rekurzivni predpis mnogo enostavnejši kot izrecni. Na primer, izrecna predpisa za faktorielo in Fibonaccijevo zaporedje sta
\[n! = \int_0^\infty x^n e^{-x} \; dx\]
in
\[F_n = \frac{\Big(\frac{1+\sqrt{5}}{2}\Big)^n - \Big(\frac{1-\sqrt{5}}{2}\Big)^n}{\sqrt{5}}.\]
Odvisno od tega, katera vrednost vas zanima, utegneta biti ta dva predpisa mnogo bolj okorna za računanje, kot pa rekurzivna. Pravzaprav nekaj časa traja, da sploh dokažete, da so rezultati teh predpisov naravna števila!

Včasih pa preslikavo podamo rekurzivno preprosto zato, ker nimamo druge možnosti. Zgornja predpisa sicer podajata preslikavi izrecno, ampak cena za to je uporaba zapletenih operacij na realnih številih, kot so integral, eksponentna funkcija z naravno osnovo in korenjenje. Strogo vzeto smo zaenkrat od številskih množic definirali samo naravna števila, pa še zanje znamo povedati zgolj, kaj je $0$ in kaj je naslednik. V bistvu še ne ``znamo'' niti seštevati!

S pomočjo rekurzije bomo lahko definirali ostalo strukturo, ki jo poznamo na naravnih številih: seštevanje, množenje in tako naprej. Za začetek pa natančno izoblikujmo in dokažimo načelo o rekurziji na naravnih številih. Iz zgornje razprave je jasno, da je rekurzija tesno povezana z indukcijo, od koder jo bomo tudi izpeljali.

\begin{izrek}[Načelo rekurzije]\label{izrek:rekurzija}
Imejmo poljubni množici $X$ in $Y$ ter preslikavi $b\colon X \to Y$ in $r\colon X \times Y \times \NN \to Y$. Tedaj obstaja natanko ena preslikava $f\colon X \times \NN \to Y$, za katero velja
\[f(x, 0) = b(x)\]
in
\[f\big(x, \suc(n)\big) = r\big(x, f(n, x), n\big)\]
za vse $x \in X$ in $n \in \NN$.

Temu natančneje rečemo \df{načelo parametrizirane rekurzije}, ker pri preslikavi $f$ na naravnih številih dopuščamo še poljuben parameter iz množice $X$. Če za $X$ vzamemo enojec, se zgornja izjava reducira na sledeče \df{načelo neparametrizirane rekurzije}.

Če imamo množico $Y$, element $b \in Y$ in preslikavo $r\colon Y \times \NN \to Y$, tedaj obstaja natanko ena preslikava $f\colon \NN \to Y$, za katero velja
\[f(0) = b\]
in
\[f\big(\suc(n)\big) = r\big(f(n), n\big)\]
za vse $n \in \NN$.
\end{izrek}

\begin{dokaz}
\end{dokaz}

Rekurzijo smo na ta način izpeljali iz indukcije, poudarimo pa, da je možen tudi obraten pristop: načelo o rekurziji vzamemo kot osnoven aksiom naravnih števil \emph{namesto} indukcije, nato pa od tod izpeljemo načelo o indukciji. Poglejmo, kako to storimo.

Vzemimo poljuben predikat $\phi\colon \NN \to \tvs$, za katerega velja $\phi(0)$ in $\all{n \in \NN} (\phi(n) \implies \phi(\suc(n)))$. Po načelu rekurzije obstaja natanko ena preslikava $f\colon \NN \to \tvs$, za katero velja $f(0) = \true$ in $f\big(\suc(n)\big) = f(n) \lor \phi\big(\suc(n)\big)$ za vse $n \in \NN$. Ampak predikat $\phi$ sam zadošča tema pogojema, saj lahko izjavo $\phi(n) \implies \phi\big(\suc(n)\big)$ enakovredno zapišemo kot $\phi\big(\suc(n)\big) = \phi(n) \lor \phi\big(\suc(n)\big)$. Očitno pa tudi povsod resničen predikat zadošča danima pogojema, od koder zaključimo $\phi = \lam{n \in \NN}{\true}$.

V tem smislu sta načeli rekurzije in indukcije enakovredni. Kot vidimo, lahko pravzaprav na indukcijo gledamo kot na poseben primer rekurzije, konkretno za preslikave oblike $\NN \to \tvs$. To nam pove, da je ta primer tako generičen, da je iz njega možno dobiti načelo za poljubne preslikave oblike $X \times \NN \to Y$.

\note{Rekurzor kot preslikava. Morda pripomba, ki zgornjo diskusijo poveže s primitivno rekurzijo iz teorije izračunljivosti.}

\subsection{Računske operacije}

Uporabimo zdaj izpeljano rekurzijo za natančno matematično definicijo strukture na naravnih številih, ki jo neformalno poznate že od malih nog. Začnimo z osnovnimi računskimi operacijami.

Seštevanje želimo definirati kot preslikavo $\NN \times \NN \to \NN$. Da ga definiramo rekurzivno, moramo povedati, kaj pomeni prišteti ničlo in kaj pomeni prišteti naslednika nekega števila (izraženo z vsoto, ki jo dobimo iz prištetja tega števila samega). Smiselno je podati naslednje.
\begin{align*}
m + 0 &\dfeq m \\
m + \suc(n) &\dfeq \suc(m + n)
\end{align*}
V tej definiciji $m$ nastopa kot parameter --- se pravi, uporabili bomo načelo parametrizirane rekurzije. Glede na oznake iz izreka~\ref{izrek:rekurzija} smo vzeli $X = \NN$, $Y = \NN$, $b(m) = m$ (torej je $b$ identiteta na $\NN$) in $r(m, v, n) = \suc(v)$ (se pravi, $r$ je kompozicija projekcije na drugo komponento in preslikave naslednika). Po tem izreku dobimo enolično določeno preslikavo $+\colon \NN \times \NN \to \NN$ (ki igra vlogo preslikave $f$ iz izreka).

Dokažimo, da pravkar definirano seštevanje zadošča zakonom, na katere smo navajeni. Začnimo s tem, da preverimo, da je $0$ enota za seštevanje.

Seveda velja $a + 0 = a$ za vse $a \in \NN$ --- to je del definicije seštevanja. Od tod pa ne smemo takoj sklepati na $0 + a = a$, saj še nismo dokazali izmenljivosti seštevanja. Lahko bi na tem mestu začeli z dokazom izmenljivosti, ampak kot bomo videli, bomo za to že potrebovali dejstvo, da je $0$ enota. Dokažimo torej $0 + a = a$ za vse $a \in \NN$ neposredno.

Trditev dokazujemo z indukcijo. Najprej dokažemo trditev za $a = 0$, torej $0 + 0 = 0$. To je res po definiciji.

Privzemimo, da velja $0 + a = a$ za neki $a \in \NN$. Dokazujemo $0 + \suc(a) = \suc(a)$. Preverimo:
\[0 + \suc(a) = \suc(0 + a) = \suc(a).\]

Kaj pa, če namesto $0$ prištejemo $1$? Takrat seveda pričakujemo, da dobimo naslednika. Preverimo.

Za poljuben $a \in \NN$ dobimo $a + 1 = a + \suc(0) = \suc(a + 0) = \suc(a)$. Tukaj sploh nismo potrebovali indukcije. Jo pa potrebujemo za dokaz, da za vsak $a \in \NN$ velja $1 + a = \suc(a)$. Za $a = 0$ je to definicija oznake $1$. Recimo, da za neki $a \in \NN$ velja $1 + a = \suc(a)$. Tedaj $1 + \suc(a) = \suc(1 + a) = \suc(\suc(a))$.

Prepričajmo se zdaj o družilnosti (asociativnosti) seštevanja. Dokazati želimo izjavo
\[\all{a \in \NN}\all{b \in \NN}\all{c \in \NN}{(a + b) + c = a + (b + c)}.\]
Vzemimo poljubna $a, b \in \NN$, notranjo univerzalno kvantificirano izjavo pa dokažimo z indukcijo (po spremenljivki $c$). Če vzamemo $c = 0$, izjava velja: $(a + b) + 0 = a + b = a + (b + 0)$. Privzemimo zdaj, da pri nekem $c \in \NN$ velja $(a + b) + c = a + (b + c)$. Poračunamo
\[(a + b) + \suc(c) = \suc\big((a + b) + c\big) = \suc\big(a + (b + c)\big) = a + \suc(b + c) = a + \big(b + \suc(c)\big).\]

Zdaj lahko dokažemo izmenljivost (komutativnost) seštevanja. Dokazati želimo izjavo
\[\all{a \in \NN}\all{b \in \NN}{a + b = b + a}.\]
Vzemimo poljuben $a \in \NN$, nato pa nadaljujmo z indukcijo (po $b$). Za $b = 0$ trdimo $a + 0 = 0 + a$. To smo že dokazali --- obe strani enakosti sta enaki $a$, saj vemo, da je $0$ enota za seštevanje.

Predpostavimo zdaj, da velja $a + b = b + a$ za neki $b \in \NN$. Izpeljati želimo $a + \suc(b) = \suc(b) + a$. Preverimo:
\[a + \suc(b) = \suc(a + b) = \suc(b + a) = 1 + (b + a) = (1 + b) + a = \suc(b) + a.\]

Na podoben način lahko definiramo množenje in dokažemo njegove lastnosti. Smiselna rekurzivna definicija množenja je sledeča.
\begin{align*}
m \cdot 0 &\dfeq 0 \\
m \cdot \suc(n) &\dfeq m \cdot n + m
\end{align*}
Če primerjamo z izrekom~\ref{izrek:rekurzija}, smo vzeli $X = Y = \NN$, $b(m) = 0$ (torej je $b$ konstantna ničelna preslikava) in $r(m, v, n) = v + m$ (to preslikavo lahko definiramo s pomočjo pravkar definiranega seštevanja). Izrek nam porodi enolično določeno preslikavo $\cdot\colon \NN \times \NN \to \NN$.

Podobno kot prej pri seštevanju za začetek ugotovimo, kaj se zgodi, ko množimo z $0$ oziroma $1$. Po definiciji vemo $a \cdot 0 = 0$ za vse $a \in \NN$. Dokažimo še $0 \cdot a = 0$ za vse $a \in \NN$. Za $a = 0$ velja $0 \cdot 0 = 0$ po definiciji. Vzemimo, da velja $0 \cdot a = 0$ za neki $a \in \NN$. Tedaj $0 \cdot \suc(a) = 0 \cdot a + 0 = 0 + 0 = 0$.

Število $1$ bi morala biti enota za množenje. Preverimo. Najprej $a \cdot 1 = a \cdot s(0) = a \cdot 0 + a = 0 + a = a$. Po drugi strani trditev, da za vse $a \in \NN$ velja $1 \cdot a = a$, dokažemo z indukcijo. Enakost $1 \cdot 0 = 0$ je jasna. Recimo, da trditev velja za neki $a \in \NN$. Tedaj $1 \cdot \suc(a) = 1 \cdot a + 1 = a + 1 = \suc(a)$.

Preden se lotimo družilnosti in izmenljivosti množenja, dokažimo, da je množenje razčlenitveno (distributivno) čez seštevanje. Se pravi, dokazati želimo izjavi
\[\all{a \in \NN}\all{b \in \NN}\all{c \in \NN}{(a + b) \cdot c = a \cdot c  + b \cdot c}\]
in
\[\all{a \in \NN}\all{b \in \NN}\all{c \in \NN}{a \cdot (b + c) = a \cdot b + a \cdot c}.\]
Pri prvi od izjav (desni razčlenitvi) vzemimo poljubna $a, b \in \NN$, nato pa se lotimo indukcije po $c$. Dobimo $(a + b) \cdot 0 = 0 = 0 + 0 = a \cdot 0 + b \cdot 0$. Če velja $(a + b) \cdot c = a \cdot c  + b \cdot c$ za neki $c$, tedaj
\[(a + b) \cdot \suc(c) = (a + b) \cdot c + (a + b) = a \cdot c + b \cdot c + a + b = a \cdot c + a + b \cdot c + b = a \cdot \suc(c) + b \cdot \suc(c).\]
Pri drugi izjavi (levi razčlenitvi) sklepamo podobno: $a \cdot (b + 0) = a \cdot b = a \cdot b + 0 = a \cdot b + a \cdot 0$. Nato privzamemo izjavo za neki $c$ in poračunamo
\[a \cdot \big(b + \suc(c)\big) = a \cdot \suc(b + c) = a \cdot (b + c) + a = a \cdot b + a \cdot c + a = a \cdot b + a \cdot \suc(c).\]

Preverimo zdaj družilnost množenja, torej izjavo
\[\all{a \in \NN}\all{b \in \NN}\all{c \in \NN}{(a \cdot b) \cdot c = a \cdot (b \cdot c)}.\]
Vzemimo poljubna $a, b \in \NN$ in se lotimo indukcije po $c$. Za $c = 0$ dobimo $(a \cdot b) \cdot 0 = 0 = a \cdot 0 = a \cdot (b \cdot 0)$. Predpostavimo izjavo za neki $c$ in poračunamo
\[(a \cdot b) \cdot \suc(c) = (a \cdot b) \cdot c + a \cdot b = a \cdot (b \cdot c) + a \cdot b = a \cdot (b \cdot c + b) = a \cdot \big(b \cdot \suc(c)\big).\]

Naposled preverimo še izmenljivost množenja na naravnih številih, torej izjavo
\[\all{a \in \NN}\all{b \in \NN}{a \cdot b = b \cdot a}.\]
Vzemimo poljuben $a \in \NN$. Za $b = 0$ dobimo $a \cdot 0 = 0 = 0 \cdot a$. Vzemimo, da izjava velja za neki $b$. Tedaj
\[a \cdot \suc(b) = a \cdot b + a = b \cdot a + a = b \cdot a + 1 \cdot a = (b + 1) \cdot a = \suc(b) \cdot a.\]

Na kratko lahko to celotno razpravo povzamemo: množica naravnih števil $\NN$ tvori izmenljiv polkolobar z enico. \davorin{ta pojem bo pojasnjen že v prejšnjem poglavju o strukturah} Seveda pa ne tvori kolobarja; vemo, da naravnih števil ne moremo poljubno odštevati. Še vedno pa lahko odštevanje na naravnih številih podamo kot \emph{delno} operacijo, torej kot delno preslikavo $-\colon \NN \times \NN \parto \NN$. Spomnimo se namreč \note{od polkolobarjev v prejšnjem poglavju}, da je odštevanje delna preslikava natanko tedaj, ko je polkolobar krajšalen.

Dokažimo krajšalnost polkolobarja naravnih števil, torej izjavo
\[\all{a \in \NN}\all{b \in \NN}\all{x \in \NN}\big(a + x = b + x \implies a = b\big).\]
Vzemimo poljubna $a, b \in \NN$, nato pa se kot običajno poslužimo indukcije. Pri $x = 0$ smo takoj na koncu. Privzemimo izjavo $a + x = b + x \implies a = b$ za neki $x$ in naj velja $a + \suc(x) = b + \suc(x)$. Tedaj $\suc(a + x) = \suc(b + x)$ in ker je $\suc$ injektivna preslikava (eden od Peanovih aksiomov!), sklepamo $a + x = b + x$, od tod pa $a = b$.\footnote{Injektivnost preslikave $\suc$ je točno to, kar potrebujemo za krajšalnost. Velja namreč tudi obrat: če imamo $\suc(a) = \suc(b)$, tj.~$a + 1 = b + 1$, in lahko krajšamo, potem $a = b$.}

S pomočjo (delnega) odštevanja lahko definiramo \df{predhodnika} na naravnih številih, in sicer kot $\prd(n) \dfeq n - 1$. Tudi to je zgolj delna preslikava $\prd\colon \NN \parto \NN$; ničla je edino naravno število, ki ni v njenem definicijskem območju.

\begin{vaja}
Dokaži $\all{n \in \NN}{\prd\big(\suc(n)\big) \kleq n}$.
\end{vaja}

Včasih je pa uporabno imeti obliko predhodnika in odštevanja, ki sta celoviti preslikavi. Pri predhodniku se dogovorimo, da se pomaknemo za eno nazaj, če se le da (pri ničli torej ostanemo, kjer smo). To različico predhodnika lahko definiramo z rekurzijo na naslednji način.
\begin{align*}
\tilde{\prd}(0) &\dfeq 0 \\
\tilde{\prd}\big(\suc(n)\big) &\dfeq n
\end{align*}
Po načelu o neparametrizirani rekurziji dobimo enolično določeno preslikavo $\tilde{\prd}\colon \NN \to \NN$ (konkretno, v izreku~\ref{izrek:rekurzija} vzamemo $Y = \NN$, $b = 0$ in $r(v, n) = n$).\footnote{Morda se vam zdi vprašljivo, če bi to definicijo sploh imenovali ``rekurzivna'', saj $\tilde{\prd}\big(\suc(n)\big)$ nismo izrazili s $\tilde{\prd}(n)$ (ali z drugimi besedami, preslikava $r$ ni odvisna od svojega prvega argumenta). Ampak izrek~\ref{izrek:rekurzija} za ta primer še vedno velja in zgornja definicija torej podaja dobro definirano preslikavo $\tilde{\prd}\colon \NN \to \NN$.}

Od tod lahko definiramo tako imenovano \df{prisekano odštevanje} na naravnih številih. Simbol za to operacijo je $\monus$, kar se prebere ``monus'' (torej: $1 + 2$ se bere ``ena plus dve'', $1 - 2$ se bere ``ena minus dve'' in $1 \monus 2$ se bere ``ena monus dve'').

Ideja prisekanega odštevanja je, da zmanjševanec zmanjšamo za tolikšen del odštevanca, kolikor le lahko (tako da še ostanemo v okviru naravnih števil). Z drugimi besedami: če se običajno odštevanje izide v naravnih številih, velja $a \monus b = a - b$, sicer pa velja $a \monus b = 0$. Natančna rekurzivna definicija je sledeča.
\begin{align*}
m \monus 0 &\dfeq m \\
m \monus \suc(n) &\dfeq \tilde{\prd}(m \monus n)
\end{align*}
Se pravi, če v izreku~\ref{izrek:rekurzija} vzamemo $X = Y = \NN$, $b(m) = m$ in $r(m, v, n) = \tilde{\prd}(v)$, dobimo preslikavo $\monus\colon \NN \times \NN \to \NN$.

\davorin{Ko se dokončno dogovorimo, kako bomo prevajali precendenco in asociiranje, povejmo, da ima prisekano odštevanje isto precendenco kot navadno odštevanje in da se asociira z leve.}

Oglejmo si nekaj lastnosti, ki veljajo za prisekano odštevanje. Po definiciji je $0$ desna enota, ni pa leva enota, kot takoj sledi iz naslednje vaje (od tod je jasno tudi, da $\monus$ ni izmenljiv).

\begin{vaja}
Dokaži $\all{n \in \NN}{0 \monus n = 0}$.
\end{vaja}

Bolj zvito je preveriti, da za vse $n \in \NN$ velja $n \monus n = 0$. Če poskusimo to neposredno dokazati z indukcijo, bomo hitro naleteli na oviro. Namesto tega se raje lotimo splošnejše trditve: dokažimo
\[\all{n \in \NN}\all{a \in \NN}{(n + a) \monus n = a}.\]
Dokažimo trditev za $n = 0$. Vzemimo poljuben $a \in \NN$ in poračunajmo $(0 + a) \monus 0 = a \monus 0 = 0$. Predpostavimo zdaj, da velja trditev $\all{a \in \NN}{(n + a) \monus n = a}$ za neki $n$. Dokazati želimo $\all{a \in \NN}{\big(\suc(n) + a\big) \monus \suc(n) = a}$. Vzemimo poljuben $a \in \NN$. Tedaj
\[\big(\suc(n) + a\big) \monus \suc(n) = \big(n + 1 + a\big) \monus \suc(n) = \big(n + \suc(a)\big) \monus \suc(n) =\]
\[= \tilde{\prd}\Big(\big(n + \suc(a)\big) \monus n\Big) = \tilde{\prd}\big(\suc(a)\big) = a.\]
Razmisli natančno, zakaj velja predzadnji enačaj! Če zamenjamo univerzalna kvantifikatorja v začetni izjavi, da dobimo $\all{a \in \NN}\all{n \in \NN}{(n + a) \monus n = a}$, in uporabimo trditev za $a = 0$, sklenemo naposled $n \monus n = 0$ za vse $n \in \NN$.

Prisekano odštevanje seveda ni družilno (nič bolj kot navadno odštevanje). Kot nadomestek pa nam služi naslednja trditev:
\[\all{a \in \NN}\all{b \in \NN}\all{c \in \NN}{(a \monus b) \monus c = a \monus (b + c)}.\]
Dokažimo jo. Vzemimo poljubna $a, b \in \NN$ in se lotimo indukcije po $c$. Pri $c = 0$ dobimo $(a \monus b) \monus 0 = a \monus b = a \monus (b + 0)$. Privzemimo zdaj enakost pri nekem $c$ in jo preverimo pri nasledniku:
\[(a \monus b) \monus \suc(c) = \tilde{\prd}\big((a \monus b) \monus c\big) = \tilde{\prd}\big(a \monus (b + c)\big) = a \monus \suc(b + c) = a \monus \big(b + \suc(c)\big).\]

Preden dokažemo še členjenje množenja čez prisekanego odštevanje, si pripravimo pomožno trditev.

\begin{lema}
Velja sledeče.
\begin{enumerate}
\item
Vsako naravno število je bodisi nič bodisi naslednik; se pravi, velja trditev
\[\all{n \in \NN}{n = 0 \xor \some{m \in \NN}{n = \suc(m)}}.\]
Lahko smo še natančnejši: če je število naslednik, je naslednik svojega predhodnika. Trdimo torej
\[\all{n \in \NN}{n = 0 \xor n = \suc\big(\tilde{\prd}(n)\big)}.\]
\item
Velja
\[\all{a \in \NN}\all{b \in \NN}{a \cdot \tilde{\prd}(b) = a \cdot b \monus a}.\]
\end{enumerate}
\end{lema}

\begin{dokaz}
\begin{enumerate}
\item
Po Peanovih aksiomih velja $0 \notin \rn{\suc}$, torej naravno število ne more biti hkrati nič in naslednik. Zadostuje potemtakem, da dokažemo samo še $\all{n \in \NN}{n = 0 \lor n = \suc\big(\tilde{\prd}(n)\big)}$.

Preverimo z indukcijo. Trditev očitno velja za $n = 0$. Recimo, da velja za neki $n$, in se lotimo dokazovanja za $\suc(n)$. Pri dokazovanju disjunkcije si izberimo, da dokazujemo drugi disjunkt. Z računom smo takoj konec: $\suc\Big(\tilde{\prd}\big(\suc(n)\big)\Big) = \suc(n)$ (uporabili smo družilnost sklapljanja preslikav in rekurzivno definicijo celovitega predhodnika).
\item
Vzemimo poljubna $a, b \in \NN$. Po prejšnji točki velja bodisi $b = 0$ bodisi $b = \suc\big(\tilde{\prd}(b)\big)$. V prvem primeru dobimo
\[a \cdot \tilde{\prd}(0) = a \cdot 0 = 0 = 0 \monus a = a \cdot 0 \monus a.\]
Predpostavimo, da smo v drugem primeru, da torej velja $b = \suc\big(\tilde{\prd}(b)\big)$. Račun bo bolj očiten, če začnemo z druge strani:
\[a \cdot b \monus a = a \cdot \suc\big(\tilde{\prd}(b)\big) \monus b = \big(a \cdot \tilde{\prd}(b) + a\big) \monus a = a \cdot \tilde{\prd}(b).\]
\end{enumerate}
\end{dokaz}

Zdaj smo naposled pripravljeni, da dokažemo členjenje množenja čez prisekano odštevanje v naravnih številih. Ker že vemo, da je množenje izmenljivo, zadostuje preveriti členjenje samo na eni strani. Dokažimo torej $\all{a \in \NN}\all{b \in \NN}\all{c \in \NN}{a \cdot (b \monus c) = a \cdot b \monus a \cdot c}$.

Vzemimo poljubna $a, b \in \NN$ in nadaljujmo z indukcijo po $c$. Pri $c = 0$ takoj dobimo $a \cdot (b \monus 0) = a \cdot b = a \cdot b \monus 0 = a \cdot b \monus a \cdot 0$. Recimo zdaj, da trditev velja za neki $c$. Dokažimo jo za $\suc(c)$:
\[a \cdot \big(b \monus \suc(c)\big) = a \cdot \tilde{\prd}(b \monus c) = a \cdot (b \monus c) \monus a = (a \cdot b \monus a \cdot c) \monus a = a \cdot b \monus (a \cdot c + a) = a \cdot b \monus a \cdot \suc(c).\]

\note{Ena od vaj: rekurzivna definicija in dokaz lastnosti potenciranja}

\subsection{Urejenost}
\subsection{Nadaljnje karakterizacije}


\section{Cela števila}
\section{Racionalna števila}
\section{Realna števila}
\section{Kompleksna števila}

\davorin{Se ustavimo že pri realnih številih? Gremo še dlje do kvaternionov?}


%%% Local Variables:
%%% mode: latex
%%% TeX-master: "ucbenik-lmn"
%%% End:

\chapter{Indukcija}

\davorin{Zaenkrat bom \qt{well-founded} prevajal kot \qt{dobro osnovan}, ker je ta izraz uporabljal Marko pri LMN. Ampak dobro bi bilo, da bi ta prevod še predebatirali.}

V razdelku~\note{o naravnih številih} smo videli, kako lahko z matematično indukcijo dokazujemo univerzalno kvantificirane izjave na naravnih številih. Pravzaprav smo si ogledali indukcijo v dveh oblikah. V prvem primeru smo morali pokazati temelj indukcije in indukcijski korak s števila na njegovega naslednika. V drugem primeru smo imeli samo indukcijski korak, pri katerem pa smo sklepali na veljavnost izjave za neki element z veljavnosti izjave za vse prejšnje elemente.

Ideja indukcije pa je mnogo globlja in splošnejša, kot omenjeni primer na naravnih številih.

\section{Dobro osnovano urejene množice}

Na kratko rečeno, dobro osnovano urejene množice so množice s toliko strukture, da lahko na njih izvajamo indukcijo. Natančna definicija je sledeča.

\begin{definicija}
	Naj bo $X$ množica in $\ll$ relacija na njej.
	\begin{itemize}
		\item
			Za predikat $\phi$ na $X$ rečemo, da je \df{$\ll$-induktiven}, kadar za vsak $a \in X$ velja: če $\phi$ velja za vse $x \in X_{\ll a}$, tedaj velja tudi za $a$. Simbolno zapisano:
			\[\all[2]{a}[X]{\xall{x}[X_{\ll a}]{\phi(x)} \implies \phi(a)}.\]
		\item
			Množica $X$, skupaj z relacijo $\ll$, je \df{dobro osnovano urejena}, kadar je posod resničen predikat edini $\ll$-induktiven predikat.
	\end{itemize}
\end{definicija}

Torej: če želimo dokazati, da neka lastnost $\phi$ velja za vse elemente dobro osnovano urejene množice, dokažemo indukcijski korak za $\phi$ (v smislu: če lastnost velja za vse elemente, \qt{manjše} od $a$, potem velja tudi za $a$).

Zgornja definicija dobro osnovane urejenosti je neposredno naravnana na indukcijo. To je njena prednost, je pa tudi njena slabost: kako vemo, da neka relacija $\ll$ dejansko dobro osnovano ureja dano množico? Neposredno preveriti definicijo je lahko težje, kot pa neposredno preveriti željeno univerzalno kvantificirano lastnost; v tem primeru nismo nič pridobili. Zato je dobro imeti alternativne karakterizacije dobro osnovane urejenosti.

\begin{izrek}
	Naslednje izjave so ekvivalentne za poljubno množico $X$ in relacijo $\ll$ na njej.
	\begin{itemize}
		\item
			$\ll$ dobro osnovano ureja $X$.
		\item
			Ne obstaja neskončna padajoča veriga v $X$. Natančneje: ne obstaja zaporedje $a\colon \NN \to X$, za katerega velja $a_{n+1} \ll a_n$ za vse $n \in \NN$.
		\item
			Vsaka nahajajoča podmnožica $S \subseteq X$ ima minimalni element v naslednjem smislu: obstaja $a \in S$, tako da za noben $x \in S$ ne velja $x \ll a$.
	\end{itemize}
\end{izrek}

\begin{dokaz}
\end{dokaz}

Splošne dobro osnovane urejenosti so lahko precej razvejane (kot bomo kasneje videli iz primerov). Včasih se zato želimo omejiti na tako imenovane dobre urejenosti

\begin{definicija}
	Množica $X$, opremljena z relacijo $\ll$, je \df{dobra urejena}, kadar je dobro osnovano urejena in relacija $\ll$ je stroga linearna urejenost.
\end{definicija}

Tudi za dobre urejenosti imamo karakterizacijo.

\begin{izrek}
	Naslednji izjavi sta ekvivalentni za poljubno množico $X$ in relacijo $\ll$ na njej.
	\begin{itemize}
		\item
			$\ll$ dobro ureja $X$.
		\item
			Vsaka nahajajoča podmnožica $S \subseteq X$ ima najmanjši element v naslednjem smislu: obstaja $a \in S$, tako da za vsak $x \in S \setminus \set{a}$ velja $a \ll x$.
	\end{itemize}
\end{izrek}

\begin{dokaz}
\end{dokaz}

\note{primeri dobro (osnovano) urejenih množic, primeri dokazov z indukcijo na njih}

\section{Strukturna indukcija}


%%% Local Variables:
%%% mode: latex
%%% TeX-master: "ucbenik-lmn"
%%% End:

\chapter{Kumulativna hierarhija}

\section{Aksiomi teorije množic}

\andrej{Zagotovo pa ne ZFC, ampak neka aksiomatizacija, ki ima razrede, recimo BGN ali MK.}

% Aksiom izbire ne paše sem, obravnavan bo že dosti prej.

%%% Local Variables:
%%% mode: latex
%%% TeX-master: "ucbenik-lmn"
%%% End:

\chapter{Kardinalna števila}

\section{Končnost in neskončnost}
\section{Števnost}
\section{Kardinalnost množice}

%%% Local Variables:
%%% mode: latex
%%% TeX-master: "ucbenik-lmn"
%%% End:

\chapter{Ordinalna števila}

\davorin{Mogoče združimo kardinalna in ordinalna števila v eno poglavje?}

%%% Local Variables:
%%% mode: latex
%%% TeX-master: "ucbenik-lmn"
%%% End:

\Closesolutionfile{resitve}
\chapter{Rešitve vaj}
\begin{Resitev}{2.1}
Množica~$A$ ima kvečjemu en element, tj.~množica~$A$ je bodisi prazna bodisi enojec. Tudi: množica~$A$ je podmnožica kakega enojca oz.~edina preslikava $A \to \one$ je injektivna.
\end{Resitev}
\begin{Resitev}{3.8}
Imamo dve stikali, imenujmo ju $p$ in $q$. Opazujemo, kdaj luč sveti. Na začetku sta obe stikali ugasnjeni in luč ne sveti. Če prižgemo eno stikalo, mora luč zasvetiti. Če prižgemo nato še drugo stikalo mora luč ugasniti. Ugotovimo, da je luč prižgana, ko je prižgano natanko eno stikalo. To ponazorimo v naslednji tabeli:

\begin{center}
                        \begin{tabular}{cc|c}
                                $p$ & $q$ & \text{ luč sveti } \\
                                \hline
                                $\true$ & $\true$& $\false$ \\
                                $\true$ & $\false$  & $\true$ \\
                                $\false$ & $\true$ & $\true$ \\
                                $\false$ & $\false$  & $\false$
                        \end{tabular}
\end{center}
Opazimo, da ima to enako tabelo, kot izjava $p  \xor q$. Torej moramo to izjavo izraziti z izjavnimi vezniki $\land, \lor$ in $\neg$. En način, kako to naredimo je, da zapišemo $p \xor q \equiv (p \lor q) \land \neg (p \land q)$, in tako konstruiramo vezja z vrati ``in'', ``ali'' in negacijo takole:
\begin{center}
\begin{circuitikz} \draw
(2,0) node[anchor=north] (q) {}
(2,8) node[anchor=south] (p) {}
(4 ,2) node[or port] (myor1) {}
(4,6) node[and port] (myand1) {}
(6,6) node[not port](mynot1){}
(8,4) node[and port](myand2){}
(0,0) to[switch, l^=$q$, -*] (2,0)
(0,8) to[switch,  l^=$p$, -*] (2,8)
(p) -- (myor1.in 1)
(q) -- (myor1.in 2)
(p) -- (myand1.in 1)
(q) -- (myand1.in 2)
(myand1.out) -- (mynot1.in)
(mynot1.out) -- (myand2.in 1)
(myor1.out) -- (myand2.in 2)
(myand2.out) to [lamp] (10,4);
%(myand1.out) -- (myxnor.in 1)
%(myand2.out) -- (myxnor.in 2);
\end{circuitikz}
\end{center}
Le z veznikoma $\land$ in $\lor$ tega ne moremo storiti, saj veznika ne predstavljata polnega nabora. Z uporabo zgolj Łukasiewiczevega veznika pa je to mogoče, saj predstavlja poln nabor.
\end{Resitev}

\appendix
\chapter*{Pomembnejši makroji (razlaga uporabe)}

\davorin{To poglavje je namenjeno zgolj za nas pisce, ne pa za bralce. Tu razložim uporabo nekaterih latexovskih makrojev, ki sem jih definiral. Če dodate svoje, katerih uporaba ni očitna, njihovo razlago prosim dodajte sem.}

\newcommand{\ponazoritev}[2]{

\medskip
\begin{tabular}{lll}
\textbf{Koda:} && \texttt{#1} \\[1ex]
\textbf{Prikaz:} && {#2}
\end{tabular}
\bigskip

}

\section*{Nekateri zapisi}

Za podajanje latexovskih ukazov uporabimo \ltc{ltc}. \davorin{Okolje \texttt{verbatim} ne deluje dobro znotraj makrojev, ampak če kdo ve, kako to razrešiti, naj popravi.}
\ponazoritev{\ltc{ltc}\{sqrt$\backslash$\{2$\backslash$\}\}}{\ltc{sqrt\{2\}}}

Narekovaje pišemo tako, kot je to običajno v {\LaTeX}u, saj lahko kasneje določimo, kako
se jih dejansko prikazuje.

Včasih bomo želeli podati stavek v naravnem jeziku (namesto v simbolnem matematičnem).
\ponazoritev{\ltc{nls}\{Stavek v naravnem jeziku.\}}{\nls{Stavek v naravnem jeziku.}}

Za definirani izraz uporabimo \ltc{df}.
\ponazoritev{Funkcija je \ltc{df}\{zvezna\}, kadar\ltc{ldots}}{Funkcija je \df{zvezna}, kadar\ldots}

Za definicijsko enakost uporabimo \ltc{dfeq} oz.~za enakost v nasprotni smeri \ltc{revdfeq}.
\ponazoritev{\textdollar{f(x,y) \ltc{dfeq} x + y}\textdollar}{$f(x,y) \dfeq x + y$}
\ponazoritev{\textdollar{e\^{}2 + \ltc{pi} \ltc{revdfeq} a}\textdollar}{$e^2 + \pi \revdfeq a$}

\section*{Množice}

Za množice uporabimo ukaz \ltc{set}. Podamo lahko enega ali dva argumenta.
\ponazoritev{\textdollar{\ltc{set}\{1,2,3\}}\textdollar}{$\set{1,2,3}$}
\ponazoritev{\textdollar{\ltc{set}\{x \ltc{in} \ltc{RR}\}\{x > 1\}}\textdollar}{$\set{x \in \RR}{x > 1}$}

Zaviti oklepaji se samodejno prilagajajo velikosti besedila.
\ponazoritev{\textdollar{\ltc{set}\{1, \ltc{displaystyle}\{\ltc{frac}\{3\}\{4\}\}\} \ltc{cup} \ltc{set}\{x \ltc{in} \ltc{NN}\}\{x > 2\^{}\{2\^{}\{100\}\}\}}\textdollar}{$\set{1, \displaystyle{\frac{3}{4}}} \cup \set{x \in \NN}{x > 2^{2^{100}}}$}

Če nam privzeta velikost oklepajev ni všeč, jo lahko spremenimo z izbirnim parametrom, ki je število od 0 do 4.
\ponazoritev{\textdollar{\ltc{set}[0]\{0\}, \ltc{set}[1]\{1\}, \ltc{set}[2]\{2\}, \ltc{set}[3]\{3\}, \ltc{set}[4]\{4\}}\textdollar}{$\set[0]{0}, \set[1]{1}, \set[2]{2}, \set[3]{3}, \set[4]{4}$}

Za generični enojec uporabimo ukaz \ltc{one}, za njegov element pa \ltc{unit}.
\ponazoritev{\textdollar{\ltc{one} = \ltc{set}\{\ltc{unit}\}}\textdollar}{$\one = \set{\unit}$}

\section*{Intervali}

\davorin{Glej diskusijo, ki se trenutno nahaja v razdelku~2.1 (ampak to se bo spremenilo).}

Za intervale uporabljamo ukaze \ltc{intoo}, \ltc{intoc}, \ltc{intco}, \ltc{intcc}, kjer \texttt{o} označuje odprtost, \texttt{c} pa zaprtost intervala. Krajišči intervala podamo kot argumenta.
\ponazoritev{\textdollar{\ltc{intoo}\{0\}\{1\}, \ltc{intoc}\{2\}\{3\}, \ltc{intco}\{4\}\{5\}, \ltc{intcc}\{6\}\{7\}}\textdollar}{$\intoo{0}{1}, \intoc{2}{3}, \intco{4}{5}, \intcc{6}{7}$}

Če želimo interval na neki drugi množici kot $\RR$, podamo to množico kot izbirni argument.
\ponazoritev{\textdollar{\ltc{intco}[\ltc{NN}]\{1\}\{5\} = \ltc{set}\{1,2,3,4\}}\textdollar}{$\intco[\NN]{1}{5} = \set{1,2,3,4}$}

\section*{Kvantifikatorji, $\lambda$- in $\iota$-izrazi}

Vsi kvantifikatorji imajo enako obliko, ponazorimo jo z univerzalnim kvantifikatorjem:
%
% Meni se ful ful ne da uporabljati teh makrojev, da bo 2% lepše, to bomo itak izbrisali.
% (Andrej)
%
\begin{itemize}
\item Koda: \verb|\all{x \in A} \Phi|
\item Prikaz: $\all{x \in A} \Phi$
\end{itemize}
%
Če želimo oklepaje okoli $\Phi$, jih enostavno napišemo. Če želimo imeti neomejen kvantifikator, lahko napišemo \verb|\all{x} \Phi| itd.

Ostali kvantifikatorji si:
%
\begin{itemize}
\item eksistenčni: \verb|\some{x \in A} \Phi|, dobimo $\some{x \in A} \Phi$
\item enolični obtoj: \verb|\exactlyone{x \in A} \Phi|, dobimo $\exactlyone{x \in A} \Phi$
\item funkcija: \verb|\lam{x \in A} e|, dobimo $\lam{x \in A} e$
\item opis: \verb|\that{x \in A} \Phi|, dobimo $\that{x \in A} \Phi$
\end{itemize}


\section*{Kanonične projekcije in injekcije}

Nismo še sprejeli odločitve, kako bomo označevali projekcije oz.~injekcije pri dvojiških produktih oz.~vsotah. Tudi ko jo bomo, bomo verjetno šli skozi več iteracij. Imejmo torej makroje zanje, ki jih bomo lahko na koncu poljubno spreminjali.

\davorin{Projekcije in injekcije so naravne preslikave. Tega verjetno ne bomo omenjali študentom, dobro pa bi bilo, da se sami tega zavedamo in izrecno pišemo indekse komponent. Na ta način se izognemo zmedi v situacijah, kjer obravnavamo več kot en (ko)produkt.}

Ukazi za leve oz.~desne projekcije oz.~injekcije so sledeči.
\begin{center}
\begin{tabular}{c|cc}
& leva & desna \\
\hline
projekcija & \ltc{lpr} & \ltc{rpr} \\
injekcija & \ltc{lin} & \ltc{rin}
\end{tabular}
\end{center}

Tem ukazom kot izbirna parametra podamo faktorja oz.~sumanda.
\ponazoritev{\textdollar{X \ltc{stackrel}\{\ltc{lpr}[X][Y]\}\{\ltc{longleftarrow}\} X \ltc{times} Y \ltc{stackrel}\{\ltc{rpr}[X][Y]\}\{\ltc{longrightarrow}\} Y}\textdollar}{$X \stackrel{\lpr[X][Y]}{\longleftarrow} X \times Y \stackrel{\rpr[X][Y]}{\longrightarrow} Y$}
\ponazoritev{\textdollar{X \ltc{stackrel}\{\ltc{lin}[X][Y]\}\{\ltc{longrightarrow}\} X + Y \ltc{stackrel}\{\ltc{rin}[X][Y]\}\{\ltc{longleftarrow}\} Y}\textdollar}{$X \stackrel{\lin[X][Y]}{\longrightarrow} X + Y \stackrel{\rin[X][Y]}{\longleftarrow} Y$}

%%% Local Variables:
%%% mode: latex
%%% TeX-master: "ucbenik-lmn"
%%% End:

\else
\fi

\end{document}

%%% Local Variables:
%%% mode: latex
%%% TeX-master: "ucbenik-lmn"
%%% End:
