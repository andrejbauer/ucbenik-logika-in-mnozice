\documentclass{article}

\usepackage[utf8]{inputenc}
\usepackage{amsmath}
\usepackage{amssymb}
\usepackage{amsthm}
\usepackage{ifthen}
\usepackage{xparse}
%\usepackage{tikz}
%\usetikzlibrary{decorations.fractals}

\newcommand{\sizedescriptor}[2]
{
\ifthenelse{\equal{#1}{0}}{}{
\ifthenelse{\equal{#1}{1}}{\big}{
\ifthenelse{\equal{#1}{2}}{\Big}{
\ifthenelse{\equal{#1}{3}}{\bigg}{
\ifthenelse{\equal{#1}{4}}{\Bigg}{
#2}}}}}
}

\NewDocumentCommand{\set}
{O{auto} m G{\empty}}
{\sizedescriptor{#1}{\left}\{ {#2} \ifthenelse{\equal{#3}{}}{}{ \; \sizedescriptor{#1}{\middle}| \; {#3}} \sizedescriptor{#1}{\right}\}}

\newcommand{\all}[1]{\forall #1 .\,}
\newcommand{\some}[1]{\exists #1 .\,}
\newcommand{\exactlyone}[1]{\exists{!} #1 .\,}
\newcommand{\lam}[1]{\lambda #1 .\,}
\newcommand{\that}[1]{\iota #1 .\,}

\usepackage[slovene]{babel}
\newcommand{\lthen}{\Rightarrow}
\newcommand{\two}{\mathbf{2}}
\newcommand{\true}{\top}
\newcommand{\false}{\bot}

\newcommand{\NN}{\mathbb{N}}
\newcommand{\ZZ}{\mathbb{Z}}
\newcommand{\QQ}{\mathbb{Q}}
\newcommand{\RR}{\mathbb{R}}

%%%%%%%%%%%%%%%%%%%%%%%%%%%%%%%%%%%%%%%%%%%%%%%%%%%%%%%%%%%%%%%%%%%%%%%%%%%%%%%%%%%%%%%%%%%%%%%%%%%%%%%%%%%%%%%%%%%%%%%
%%%  Commands
%%%%%%%%%%%%%%%%%%%%%%%%%%%%%%%%%%%%%%%%%%%%%%%%%%%%%%%%%%%%%%%%%%%%%%%%%%%%%%%%%%%%%%%%%%%%%%%%%%%%%%%%%%%%%%%%%%%%%%


%%%%%%  Auxiliary
%%%%%%%%%%%%%%%%%%%%%%%%%%%%%%%%%%%%%%%%%%%%%%%%%%%%%%%%%%%%%
\newcommand{\sizedescriptor}[2]
{
\ifthenelse{\equal{#1}{0}}{}{
\ifthenelse{\equal{#1}{1}}{\big}{
\ifthenelse{\equal{#1}{2}}{\Big}{
\ifthenelse{\equal{#1}{3}}{\bigg}{
\ifthenelse{\equal{#1}{4}}{\Bigg}{
#2}}}}}
}

\newcommand{\someref}{{\small\textcolor{blue}{[\textbf{ref.}]}}}
\newcommand{\intermission}{\bigskip\medskip}
\newcommand{\qt}[1]{{\quotedblbase}{#1}{‘‘}}  % text in quotation marks
\newcommand{\nls}[1]{\qt{\textit{#1}}}  % sentence in a natural language

\definecolor{andrejcolor}{rgb}{0.7,0,0.7}
\definecolor{davorincolor}{rgb}{0,0.45,1}
\definecolor{markocolor}{rgb}{0.7,0.4,0}
\definecolor{matijacolor}{rgb}{0,0.6,0.4}

\newcommand{\andrej}[1]{{\small\textcolor{andrejcolor}{(#1 \ \mbox{--Andrej})}}}
\newcommand{\davorin}[1]{{\small\textcolor{davorincolor}{(#1 \ \mbox{--Davorin})}}}
\newcommand{\marko}[1]{{\small\textcolor{markocolor}{(#1 \ \mbox{--Marko})}}}
\newcommand{\matija}[1]{{\small\textcolor{matijacolor}{(#1 \ \mbox{--Matija})}}}

\definecolor{notecolor}{rgb}{0.6,0.5,0.7}
\newcommand{\note}[1]{{\small\textcolor{notecolor}{(#1)}}}
\newcommand{\alert}[1]{{\small\textcolor{red}{\textbf{#1}}}}


%%%%%%  Logical Quantifiers, λ- and ι-Expressions
%%%%%%%%%%%%%%%%%%%%%%%%%%%%%%%%%%%%%%%%%%%%%%%%%%%%%%%%%%%%%

%  no parenthesis (add x in front of the name of the command)
\NewDocumentCommand{\xall}
{m O{\empty} m}
{\forall\, {#1} \ifthenelse{\equal{#2}{}}{}{\in{#2}} \,.\, {#3}}
\NewDocumentCommand{\xsome}
{m O{\empty} m}
{\exists\, {#1} \ifthenelse{\equal{#2}{}}{}{\in{#2}} \,.\, {#3}}
\NewDocumentCommand{\xexactlyone}
{m O{\empty} m}
{\exists\;\!!\, {#1} \ifthenelse{\equal{#2}{}}{}{\in{#2}} \,.\, {#3}}
\NewDocumentCommand{\xlam}
{m O{\empty} m O{\empty}}
{\lambda{#1} \ifthenelse{\equal{#2}{}}{}{\in{#2}} \,.\, {#3} \ifthenelse{\equal{#4}{}}{}{\in{#4}}}
\NewDocumentCommand{\xthat}
{m O{\empty} m}
{\iota{#1} \ifthenelse{\equal{#2}{}}{}{\in{#2}} \,.\, {#3}}

%  with parenthesis -- the first optional argument is the size (values 0-4)
\NewDocumentCommand{\all}
{O{auto} m O{\empty} m}
{\xall{#2}[#3]{\sizedescriptor{#1}{\left}( {#4} \sizedescriptor{#1}{\right})}}
\NewDocumentCommand{\some}
{O{auto} m O{\empty} m}
{\xsome{#2}[#3]{\sizedescriptor{#1}{\left}( {#4} \sizedescriptor{#1}{\right})}}
\NewDocumentCommand{\exactlyone}
{O{auto} m O{\empty} m}
{\xexactlyone{#2}[#3]{\sizedescriptor{#1}{\left}( {#4} \sizedescriptor{#1}{\right})}}
\NewDocumentCommand{\lam}
{O{auto} m O{\empty} m O{\empty}}
{\xlam{#2}[#3]{\sizedescriptor{#1}{\left}( {#4} \sizedescriptor{#1}{\right})}[#5]}
\NewDocumentCommand{\that}
{O{auto} m O{\empty} m}
{\xthat{#2}[#3]{\sizedescriptor{#1}{\left}( {#4} \sizedescriptor{#1}{\right})}}


%%%%%%  Logic
%%%%%%%%%%%%%%%%%%%%%%%%%%%%%%%%%%%%%%%%%%%%%%%%%%%%%%%%%%%%%
\newcommand{\tvs}{\Omega}  % set of truth values
\newcommand{\true}{\top}  % truth
\newcommand{\false}{\bot}  % falsehood
\newcommand{\etrue}{\bm{\top}}  % emphasized truth
\newcommand{\efalse}{\bm{\bot}}  % emphasized falsehood
\newcommand{\impl}{\Rightarrow}  % implication sign
\newcommand{\revimpl}{\Leftarrow}  % reverse implication sign
\newcommand{\lequ}{\Leftrightarrow}  % equivalence sign
\newcommand{\xor}{\mathbin{\veebar}}  % exclusive disjunction sign
\newcommand{\shf}{\mathbin{\uparrow}}  % Sheffer connective
\newcommand{\luk}{\mathbin{\downarrow}}  % Łukasiewicz connective


%%%%%%  Sets
%%%%%%%%%%%%%%%%%%%%%%%%%%%%%%%%%%%%%%%%%%%%%%%%%%%%%%%%%%%%%
%  \set{1, 2, 3}         ->  {1, 2, 3}
%  \set{a \in X}{a < 1}  ->  {a ∈ X | a < 1}
\NewDocumentCommand{\set}
{O{auto} m G{\empty}}
{\sizedescriptor{#1}{\left}\{ {#2} \ifthenelse{\equal{#3}{}}{}{ \; \sizedescriptor{#1}{\middle}| \; {#3}} \sizedescriptor{#1}{\right}\}}
%\newcommand{\vsubset}{\Mapstochar\cap}
%\newcommand{\finseq}[1]{{#1}^*}
\newcommand{\pst}{\mathcal{P}}
\renewcommand{\complement}[1]{{#1}^C}


%%%%%%  Number Sets, Intervals
%%%%%%%%%%%%%%%%%%%%%%%%%%%%%%%%%%%%%%%%%%%%%%%%%%%%%%%%%%%%%
\newcommand{\NN}{\mathbb{N}}
\newcommand{\ZZ}{\mathbb{Z}}
\newcommand{\QQ}{\mathbb{Q}}
\newcommand{\RR}{\mathbb{R}}
\newcommand{\CC}{\mathbb{C}}
\newcommand{\intoo}[3][\RR]{{#1}_{(#2, #3)}}
\newcommand{\intcc}[3][\RR]{{#1}_{[#2, #3]}}
\newcommand{\intoc}[3][\RR]{{#1}_{(#2, #3]}}
\newcommand{\intco}[3][\RR]{{#1}_{[#2, #3)}}


%%%%%%  Maps and Relations
%%%%%%%%%%%%%%%%%%%%%%%%%%%%%%%%%%%%%%%%%%%%%%%%%%%%%%%%%%%%%
\newcommand{\id}[1][]{\textrm{Id}_{#1}}  % identity map
\newcommand{\argbox}{{\;\!\fbox{\phantom{M}}\;\!}}  % box for a function argument
\newcommand{\rstr}[1]{\left.{#1}\right|}  % map restriction
\newcommand{\im}{\mathrm{im}}  % map image
\newcommand{\parto}{\mathrel{\rightharpoonup}}  % partial mapping sign
\NewDocumentCommand{\rel}
{O{\empty} O{\empty}}
{\ifthenelse{\equal{#1}{}}{\mathscr{R}}{{#1} \mathrel{\mathscr{R}} {#2}}}  % a relation
\NewDocumentCommand{\srel}
{O{\empty} O{\empty}}
{\ifthenelse{\equal{#1}{}}{\mathscr{S}}{{#1} \mathrel{\mathscr{S}} {#2}}}  % a second relation
\newcommand{\dom}{\mathrm{dom}}  % domain
\newcommand{\cod}{\mathrm{cod}}  % codomain
\newcommand{\dd}[1]{D_{#1}}  % domain of definition
\newcommand{\rn}[1]{Z_{#1}}  % range
\newcommand{\graph}[1]{\Gamma_{#1}}  % graph of a (partial) function
\NewDocumentCommand{\img}  % image
{O{\empty} m G{\empty}}
{{#2}_*\ifthenelse{\equal{#3}{}}{}{\!\sizedescriptor{#1}{\left}( {#3} \sizedescriptor{#1}{\right})}}
\NewDocumentCommand{\pim}  % preimage
{O{\empty} m G{\empty}}
{{#2}^*\ifthenelse{\equal{#3}{}}{}{\!\sizedescriptor{#1}{\left}( {#3} \sizedescriptor{#1}{\right})}}
\newcommand{\ec}[2][]{[\:\!{#2}\:\!]_{#1}}  % equivalence class
\newcommand{\transposed}[1]{\widehat{#1}}


%%%%%%  Misc.
%%%%%%%%%%%%%%%%%%%%%%%%%%%%%%%%%%%%%%%%%%%%%%%%%%%%%%%%%%%%%
\newcommand{\df}[1]{\emph{\textbf{#1}}}  % defined notion
\newcommand{\oper}{\mathop{\circledast}}  % symbol for a general operation
\newcommand{\ism}{\cong}  % isomorphic
\newcommand{\equ}{\sim}  % equivalent
\newcommand{\dfeq}{\mathrel{\mathop:}=}  % definitional equality
\newcommand{\revdfeq}{=\mathrel{\mathop:}}  % reverse definitional equality
\newcommand{\isdefined}[1]{{#1}\!\downarrow}  % given value is defined
\newcommand{\kleq}{\simeq}  % Kleene equality
\newcommand{\claim}[3]{{#1} \;\colon\; \frac{#2}{#3}}  % claim, divided on context, assumptions, conclusions
\newcommand{\unit}{\mathord{\bm{*}}}  % element in a generic singleton
\NewDocumentEnvironment{implproof}  % proof of an implication
{O{\empty} G{\empty} O{=>} G{\empty}}
{
\begin{description}
\item[\quad$\sizedescriptor{#1}{\left}({#2}
\ifthenelse{\equal{#3}{=>}}{\impl}{
\ifthenelse{\equal{#3}{<=}}{\revimpl}{
\ifthenelse{\equal{#3}{->}}{\rightarrow}{
\ifthenelse{\equal{#3}{<-}}{\leftarrow}{
#3
}}}} {#4}\sizedescriptor{#1}{\right})$]\ \vspace{0.3em}\\
}
{
\end{description}
}


%%%%%%%%%%%%%%%%%%%%%%%%%%%%%%%%%%%%%%%%%%%%%%%%%%%%%%%%%%%%%%%%%%%%%%%%%%%%%%%%%%%%%%%%%%%%%%%%%%%%%%%%%%%%%%%%%%%%%%

{
\theoremstyle{definition}
\newtheorem{vaja}{Vaja}
}


\begin{document}

\title{Logika in množice -- vaje}
\date{8.~12.~2017}
\maketitle


\begin{vaja}
	Na množici naravnih števil definiramo naslednjo relacijo:
	\[m R n \iff \text{$m n$ je kvadrat naravnega števila}.\]
	\begin{itemize}
		\item
		Pokažite, da je $R$ ekvivalenčna relacija.
		\item
		Kaj je $[30]_R$? Kaj je $[12]_R$?
		\item
		Poiščite tako množico $A \subseteq \NN$, ki bo vsebovala natanko en element iz vsakega ekvivalenčnega razreda.
	\end{itemize}
\end{vaja}

\begin{vaja}
	Čevljar gospod Šuštar že leta izdeluje čevlje ob Šuštarskem mostu v Ljubljani. Zaradi izjemno kvalitetne obrti ima že dolg spisek rednih strank. Trenutno oblikuje nov model čevlja, ki bi ga želel ponuditi prav vsem rednim strankam.
	
  Naj bo $\mathcal{S}$ množica vseh strank in $\mathcal{C}$ množica številk čevljev. Dana naj bo tudi funkcija $f\colon \mathcal{S} \to \mathcal{C}$, ki osebi priredi številko stopala. Na množici $\mathcal{S}$ definiramo relacijo $R$ takole:
  \begin{equation*}
    x R y \iff f(x) = f(y).
  \end{equation*}
  \begin{itemize}
  	\item  Razložite pomen relacije in ugotovite, kaj so njeni ekvivalenčni razredi.
  	\item Naj bo $\mathcal{I}$ množica vseh imen in naj bo $g \colon \mathcal{S} / _R \to \mathcal{I} $ preslikava definirana s predpisom $g([x]) = \text{“ime stranke $x$”}$. Ali je preslikava dobro definirana?
  	\item Naj bo $c \colon \mathcal{S} / _R \to \{\text{majho, srednje, veliko}\}$ preslikava, ki opisno opredeli velikost stopala, torej je dana s predpisom
  	\begin{equation*}
  	c([x]) = 
  	\begin{cases}
  	\text{majhno} & \text{ če $f(x) \leq 37$,} \\
  	\text{srednje} & \text{ če $37 < f(x) \leq 45$,} \\
  	\text{veliko} & \text{ če $45 < f(x)$.} \\
  	\end{cases}
  	\end{equation*}
  	Ali je preslikava dobro definirana?
  \end{itemize}
\end{vaja}

%\begin{vaja}
%	Naj bo $\mathbb{P}$ množica vseh psov in $\mathbb{B}$ množica barv. Dana naj bo tudi funkcija $f\colon \mathbb{P} \to \mathbb{B}$, ki psu priredi barvo njegove dlake. Na množici $\mathbb{P}$ definiramo relacijo $R$ takole:
%	\begin{equation*}
%		x R y \iff f(x) = f(y).
%	\end{equation*}
%	Dokažite, da je ta relacija ekvivalenčna, razložite njen pomen in ugotovite, kaj so njeni ekvivalenčni razredi.
%\end{vaja}

\begin{vaja}
	Naj bo $f\colon A \to B$ poljubna funkcija. Na $A$ definiramo relacijo $\sim$ s predpisom $x \sim y \iff f(x) = f(y)$.
	\begin{enumerate}
		\item
		Preverite, da je $\sim$ ekvivalenčna relacija. Pravimo, da je $\sim$ \emph{inducirana} s preslikavo~$f$.
		\item
		Naj bo $f\colon A \to B$ surjektivna funkcija in $\sim$ ekvivalenčna relacija, inducirana z~$f$. Dokažite, da je $B \cong A/_\sim$.
		\item
		Definirajmo relacijo $\sim$ na $\RR$ s predpisom
		%
		\begin{equation*}
			x \sim y \iff \some{k \in \ZZ}{x - y = 2 \pi k}.
		\end{equation*}
		%
		Poiščite preslikavo $f\colon \RR \to \RR \times \RR$, ki inducira $\sim$. Ali obstaja več takih preslikav?
	\end{enumerate}
\end{vaja}


%\begin{vaja}
%  Na množico naravnih števil, ki so strogo večja od $1$, vpeljemo naslednjo relacijo:
%  \[x \mathcal{P} y \iff \text{$x$ in $y$ imata enak največji praštevilski delitelj}.\]
%  \begin{itemize}
%    \item
%      Pokažite, da je $\mathcal{P}$ ekvivalenčna relacija.
%    \item
%      Določite ekvivalenčni razred $\mathcal{P}[3]$.
%    \item
%      Določite ekvivalenčni razred poljubnega praštevila.
%    \item
%      Poiščite množico $\mathcal{A}$, ki vsebuje natanko en element iz vsakega ekvivalenčnega razreda relacije $\mathcal{P}$.
%  \end{itemize}
%\end{vaja}

%\begin{vaja}
%  Vsako naravno število $n \geq 2$ ima (enoličen) praštevilski razcep $n = p_1^{k_1} \cdot \ldots \cdot p_r^{k_r}$. Definiramo
%  \[\psi(n) = k_1 + \ldots + k_r\]
%  in na $\NN \setminus \set{0, 1}$ vpeljemo relacijo
%  \[m R n \iff \psi(m) = \psi(n).\]
%  \begin{itemize}
%    \item Dokažite, da je $R$ ekvivalenčna relacija.
%    \item Ali sta števili $360$ in $900$ v relaciji $R$?
%    \item Opišite $[143]$.
%    \item Kaj je ekvivalenčni razred praštevila?
%    \item Opišite ekvivalenčne razrede, ki razbijejo $\NN \setminus \set{0, 1}$.
%  \end{itemize}
%\end{vaja}

%\begin{vaja}
%  Na $\NN$ definiramo naslednjo relacijo
%  \[a R b \iff 7 | (5a+2b).\]
%  \begin{itemize}
%    \item Pokažite, da je $R$ ekvivalenčna relacija.
%    \item Določite ekvivalenčne razrede relacije $R$.
%    \item Poiščite še faktorsko množico $\NN/R$.
%  \end{itemize}
%\end{vaja}


%\begin{vaja}
%  Na množici naravnih števil je definirana relacija
%  \begin{equation*}
%    a R b \iff \text{$5$ deli $(3a+2b)$}.
%  \end{equation*}
%  Dokažite, da je $R$ ekvivalenčna relacija.
%\end{vaja}

\begin{vaja}
	Naj bo $2\ZZ$ množica sodih celih števil, tj.\  
	$$2\ZZ = \{x \in \ZZ | x \text{ je sodo število}\} = \{\ldots, -4, -2, 0, 2, 4, \ldots\}.$$
	Na množici $\ZZ$ definiramo relacijo $\sim$ s predpisom
	%
	\begin{equation*}
	x \sim y \iff x - y \in 2\ZZ.
	\end{equation*}
	Označimo z $\ZZ / _{2\ZZ}$ kvocientno množico $\ZZ / _\sim$\footnote{Pri algebri boste spoznali, da je ta oznaka standardna, kadar je relacija definirana z dejstvom, da je razlika elementov v dani množici.} in z $\ZZ_2 $ množico $ \{0,1\}$.
	Dokažite, da velja 
	$$ \ZZ / _{2\ZZ} \cong \ZZ_2,$$
	in utemeljite dobro definiranost izomorfizmov. 
\end{vaja}

\begin{vaja}
  Na množici $\NN \times \NN$ definiramo relacijo $\sim$ s predpisom
  %
  \begin{equation*}
    (a,b) \sim (c,d) \iff a + d = b + c.
  \end{equation*}
  \begin{enumerate}
    \item
      Preverite, da je $\sim$ ekvivalenčna relacija.
    \item
      Naj bo $Z = (\NN \times \NN)/_\sim$ kvocientna množica. Andrej je želel na $Z$ definirati operaciji seštevanja in množenja s predpisoma
      %
      \begin{align*}
        [a,b] \oplus [c,d] &= [a+c, b+d],\\
        [a,b] \otimes [c,d] &= [a c, b d],
      \end{align*}
      %
      kjer smo zapisali ekvivalenčni razred $[(a,b)]_\sim$ krajše kot $[a,b]$. Ali sta $\oplus$ in $\otimes$ dobro definirani preslikavi $Z \times Z \to Z$? 
      %Če ne, kako bi ju popravili, da bi dobili na~$Z$ strukturo kolobarja? Kateri kolobar smo dobili?
  \end{enumerate}
\end{vaja}

\end{document}