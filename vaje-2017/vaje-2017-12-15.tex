\documentclass{article}

\usepackage[utf8]{inputenc}
\usepackage{amsmath}
\usepackage{amssymb}
\usepackage{amsthm}
\usepackage{ifthen}
\usepackage{xparse}
%\usepackage{tikz}
%\usetikzlibrary{decorations.fractals}

\newcommand{\sizedescriptor}[2]
{
\ifthenelse{\equal{#1}{0}}{}{
\ifthenelse{\equal{#1}{1}}{\big}{
\ifthenelse{\equal{#1}{2}}{\Big}{
\ifthenelse{\equal{#1}{3}}{\bigg}{
\ifthenelse{\equal{#1}{4}}{\Bigg}{
#2}}}}}
}

\NewDocumentCommand{\set}
{O{auto} m G{\empty}}
{\sizedescriptor{#1}{\left}\{ {#2} \ifthenelse{\equal{#3}{}}{}{ \; \sizedescriptor{#1}{\middle}| \; {#3}} \sizedescriptor{#1}{\right}\}}

\newcommand{\pow}{\mathcal{P}}

\newcommand{\all}[1]{\forall #1 .\,}
\newcommand{\some}[1]{\exists #1 .\,}
\newcommand{\exactlyone}[1]{\exists{!} #1 .\,}
\newcommand{\lam}[1]{\lambda #1 .\,}
\newcommand{\that}[1]{\iota #1 .\,}

\usepackage[slovene]{babel}
\newcommand{\lthen}{\Rightarrow}
\newcommand{\two}{\mathbf{2}}
\newcommand{\true}{\top}
\newcommand{\false}{\bot}

\newcommand{\NN}{\mathbb{N}}
\newcommand{\ZZ}{\mathbb{Z}}
\newcommand{\QQ}{\mathbb{Q}}
\newcommand{\RR}{\mathbb{R}}

%%%%%%%%%%%%%%%%%%%%%%%%%%%%%%%%%%%%%%%%%%%%%%%%%%%%%%%%%%%%%%%%%%%%%%%%%%%%%%%%%%%%%%%%%%%%%%%%%%%%%%%%%%%%%%%%%%%%%%%
%%%  Commands
%%%%%%%%%%%%%%%%%%%%%%%%%%%%%%%%%%%%%%%%%%%%%%%%%%%%%%%%%%%%%%%%%%%%%%%%%%%%%%%%%%%%%%%%%%%%%%%%%%%%%%%%%%%%%%%%%%%%%%


%%%%%%  Auxiliary
%%%%%%%%%%%%%%%%%%%%%%%%%%%%%%%%%%%%%%%%%%%%%%%%%%%%%%%%%%%%%
\newcommand{\sizedescriptor}[2]
{
\ifthenelse{\equal{#1}{0}}{}{
\ifthenelse{\equal{#1}{1}}{\big}{
\ifthenelse{\equal{#1}{2}}{\Big}{
\ifthenelse{\equal{#1}{3}}{\bigg}{
\ifthenelse{\equal{#1}{4}}{\Bigg}{
#2}}}}}
}

\newcommand{\someref}{{\small\textcolor{blue}{[\textbf{ref.}]}}}
\newcommand{\intermission}{\bigskip\medskip}
\newcommand{\qt}[1]{{\quotedblbase}{#1}{‘‘}}  % text in quotation marks
\newcommand{\nls}[1]{\qt{\textit{#1}}}  % sentence in a natural language

\definecolor{andrejcolor}{rgb}{0.7,0,0.7}
\definecolor{davorincolor}{rgb}{0,0.45,1}
\definecolor{markocolor}{rgb}{0.7,0.4,0}
\definecolor{matijacolor}{rgb}{0,0.6,0.4}

\newcommand{\andrej}[1]{{\small\textcolor{andrejcolor}{(#1 \ \mbox{--Andrej})}}}
\newcommand{\davorin}[1]{{\small\textcolor{davorincolor}{(#1 \ \mbox{--Davorin})}}}
\newcommand{\marko}[1]{{\small\textcolor{markocolor}{(#1 \ \mbox{--Marko})}}}
\newcommand{\matija}[1]{{\small\textcolor{matijacolor}{(#1 \ \mbox{--Matija})}}}

\definecolor{notecolor}{rgb}{0.6,0.5,0.7}
\newcommand{\note}[1]{{\small\textcolor{notecolor}{(#1)}}}
\newcommand{\alert}[1]{{\small\textcolor{red}{\textbf{#1}}}}


%%%%%%  Logical Quantifiers, λ- and ι-Expressions
%%%%%%%%%%%%%%%%%%%%%%%%%%%%%%%%%%%%%%%%%%%%%%%%%%%%%%%%%%%%%

%  no parenthesis (add x in front of the name of the command)
\NewDocumentCommand{\xall}
{m O{\empty} m}
{\forall\, {#1} \ifthenelse{\equal{#2}{}}{}{\in{#2}} \,.\, {#3}}
\NewDocumentCommand{\xsome}
{m O{\empty} m}
{\exists\, {#1} \ifthenelse{\equal{#2}{}}{}{\in{#2}} \,.\, {#3}}
\NewDocumentCommand{\xexactlyone}
{m O{\empty} m}
{\exists\;\!!\, {#1} \ifthenelse{\equal{#2}{}}{}{\in{#2}} \,.\, {#3}}
\NewDocumentCommand{\xlam}
{m O{\empty} m O{\empty}}
{\lambda{#1} \ifthenelse{\equal{#2}{}}{}{\in{#2}} \,.\, {#3} \ifthenelse{\equal{#4}{}}{}{\in{#4}}}
\NewDocumentCommand{\xthat}
{m O{\empty} m}
{\iota{#1} \ifthenelse{\equal{#2}{}}{}{\in{#2}} \,.\, {#3}}

%  with parenthesis -- the first optional argument is the size (values 0-4)
\NewDocumentCommand{\all}
{O{auto} m O{\empty} m}
{\xall{#2}[#3]{\sizedescriptor{#1}{\left}( {#4} \sizedescriptor{#1}{\right})}}
\NewDocumentCommand{\some}
{O{auto} m O{\empty} m}
{\xsome{#2}[#3]{\sizedescriptor{#1}{\left}( {#4} \sizedescriptor{#1}{\right})}}
\NewDocumentCommand{\exactlyone}
{O{auto} m O{\empty} m}
{\xexactlyone{#2}[#3]{\sizedescriptor{#1}{\left}( {#4} \sizedescriptor{#1}{\right})}}
\NewDocumentCommand{\lam}
{O{auto} m O{\empty} m O{\empty}}
{\xlam{#2}[#3]{\sizedescriptor{#1}{\left}( {#4} \sizedescriptor{#1}{\right})}[#5]}
\NewDocumentCommand{\that}
{O{auto} m O{\empty} m}
{\xthat{#2}[#3]{\sizedescriptor{#1}{\left}( {#4} \sizedescriptor{#1}{\right})}}


%%%%%%  Logic
%%%%%%%%%%%%%%%%%%%%%%%%%%%%%%%%%%%%%%%%%%%%%%%%%%%%%%%%%%%%%
\newcommand{\tvs}{\Omega}  % set of truth values
\newcommand{\true}{\top}  % truth
\newcommand{\false}{\bot}  % falsehood
\newcommand{\etrue}{\bm{\top}}  % emphasized truth
\newcommand{\efalse}{\bm{\bot}}  % emphasized falsehood
\newcommand{\impl}{\Rightarrow}  % implication sign
\newcommand{\revimpl}{\Leftarrow}  % reverse implication sign
\newcommand{\lequ}{\Leftrightarrow}  % equivalence sign
\newcommand{\xor}{\mathbin{\veebar}}  % exclusive disjunction sign
\newcommand{\shf}{\mathbin{\uparrow}}  % Sheffer connective
\newcommand{\luk}{\mathbin{\downarrow}}  % Łukasiewicz connective


%%%%%%  Sets
%%%%%%%%%%%%%%%%%%%%%%%%%%%%%%%%%%%%%%%%%%%%%%%%%%%%%%%%%%%%%
%  \set{1, 2, 3}         ->  {1, 2, 3}
%  \set{a \in X}{a < 1}  ->  {a ∈ X | a < 1}
\NewDocumentCommand{\set}
{O{auto} m G{\empty}}
{\sizedescriptor{#1}{\left}\{ {#2} \ifthenelse{\equal{#3}{}}{}{ \; \sizedescriptor{#1}{\middle}| \; {#3}} \sizedescriptor{#1}{\right}\}}
%\newcommand{\vsubset}{\Mapstochar\cap}
%\newcommand{\finseq}[1]{{#1}^*}
\newcommand{\pst}{\mathcal{P}}
\renewcommand{\complement}[1]{{#1}^C}


%%%%%%  Number Sets, Intervals
%%%%%%%%%%%%%%%%%%%%%%%%%%%%%%%%%%%%%%%%%%%%%%%%%%%%%%%%%%%%%
\newcommand{\NN}{\mathbb{N}}
\newcommand{\ZZ}{\mathbb{Z}}
\newcommand{\QQ}{\mathbb{Q}}
\newcommand{\RR}{\mathbb{R}}
\newcommand{\CC}{\mathbb{C}}
\newcommand{\intoo}[3][\RR]{{#1}_{(#2, #3)}}
\newcommand{\intcc}[3][\RR]{{#1}_{[#2, #3]}}
\newcommand{\intoc}[3][\RR]{{#1}_{(#2, #3]}}
\newcommand{\intco}[3][\RR]{{#1}_{[#2, #3)}}


%%%%%%  Maps and Relations
%%%%%%%%%%%%%%%%%%%%%%%%%%%%%%%%%%%%%%%%%%%%%%%%%%%%%%%%%%%%%
\newcommand{\id}[1][]{\textrm{Id}_{#1}}  % identity map
\newcommand{\argbox}{{\;\!\fbox{\phantom{M}}\;\!}}  % box for a function argument
\newcommand{\rstr}[1]{\left.{#1}\right|}  % map restriction
\newcommand{\im}{\mathrm{im}}  % map image
\newcommand{\parto}{\mathrel{\rightharpoonup}}  % partial mapping sign
\NewDocumentCommand{\rel}
{O{\empty} O{\empty}}
{\ifthenelse{\equal{#1}{}}{\mathscr{R}}{{#1} \mathrel{\mathscr{R}} {#2}}}  % a relation
\NewDocumentCommand{\srel}
{O{\empty} O{\empty}}
{\ifthenelse{\equal{#1}{}}{\mathscr{S}}{{#1} \mathrel{\mathscr{S}} {#2}}}  % a second relation
\newcommand{\dom}{\mathrm{dom}}  % domain
\newcommand{\cod}{\mathrm{cod}}  % codomain
\newcommand{\dd}[1]{D_{#1}}  % domain of definition
\newcommand{\rn}[1]{Z_{#1}}  % range
\newcommand{\graph}[1]{\Gamma_{#1}}  % graph of a (partial) function
\NewDocumentCommand{\img}  % image
{O{\empty} m G{\empty}}
{{#2}_*\ifthenelse{\equal{#3}{}}{}{\!\sizedescriptor{#1}{\left}( {#3} \sizedescriptor{#1}{\right})}}
\NewDocumentCommand{\pim}  % preimage
{O{\empty} m G{\empty}}
{{#2}^*\ifthenelse{\equal{#3}{}}{}{\!\sizedescriptor{#1}{\left}( {#3} \sizedescriptor{#1}{\right})}}
\newcommand{\ec}[2][]{[\:\!{#2}\:\!]_{#1}}  % equivalence class
\newcommand{\transposed}[1]{\widehat{#1}}


%%%%%%  Misc.
%%%%%%%%%%%%%%%%%%%%%%%%%%%%%%%%%%%%%%%%%%%%%%%%%%%%%%%%%%%%%
\newcommand{\df}[1]{\emph{\textbf{#1}}}  % defined notion
\newcommand{\oper}{\mathop{\circledast}}  % symbol for a general operation
\newcommand{\ism}{\cong}  % isomorphic
\newcommand{\equ}{\sim}  % equivalent
\newcommand{\dfeq}{\mathrel{\mathop:}=}  % definitional equality
\newcommand{\revdfeq}{=\mathrel{\mathop:}}  % reverse definitional equality
\newcommand{\isdefined}[1]{{#1}\!\downarrow}  % given value is defined
\newcommand{\kleq}{\simeq}  % Kleene equality
\newcommand{\claim}[3]{{#1} \;\colon\; \frac{#2}{#3}}  % claim, divided on context, assumptions, conclusions
\newcommand{\unit}{\mathord{\bm{*}}}  % element in a generic singleton
\NewDocumentEnvironment{implproof}  % proof of an implication
{O{\empty} G{\empty} O{=>} G{\empty}}
{
\begin{description}
\item[\quad$\sizedescriptor{#1}{\left}({#2}
\ifthenelse{\equal{#3}{=>}}{\impl}{
\ifthenelse{\equal{#3}{<=}}{\revimpl}{
\ifthenelse{\equal{#3}{->}}{\rightarrow}{
\ifthenelse{\equal{#3}{<-}}{\leftarrow}{
#3
}}}} {#4}\sizedescriptor{#1}{\right})$]\ \vspace{0.3em}\\
}
{
\end{description}
}


%%%%%%%%%%%%%%%%%%%%%%%%%%%%%%%%%%%%%%%%%%%%%%%%%%%%%%%%%%%%%%%%%%%%%%%%%%%%%%%%%%%%%%%%%%%%%%%%%%%%%%%%%%%%%%%%%%%%%%

{
\theoremstyle{definition}
\newtheorem{vaja}{Vaja}
}


\begin{document}

\title{Logika in množice -- vaje}
\date{15.~12.~2017}
\maketitle

\bigskip

\noindent
Naj bo $(P, \leq)$ delna ureditev.
\begin{itemize}
  \item
    $S \subseteq P$ je \emph{veriga}, kadar velja $\all{x, y \in S}{x \leq y \lor y \leq x}$.
  \item
    $S \subseteq P$ je \emph{antiveriga}, kadar velja $\all{x, y \in S}{x \leq y \implies x = y}$.
\end{itemize}

\bigskip

\noindent
Naj bosta $(P, \leq_P)$, $(Q, \leq_Q)$ delni ureditvi. Preslikava $f\colon P \to Q$ je \emph{monotona}, kadar velja
\begin{equation*}
  \all{x, y \in P}{\ x \leq_P y \implies f(x) \leq_Q f(y)}.
\end{equation*}

\bigskip

\begin{vaja}
  Obravnavajmo množico $\set{1, 2, 3, \ldots, 12}$, delno urejeno z relacijo deljivosti $|$.
  \begin{enumerate}
    \item
      Narišite Hassejev diagram te delne ureditve.
    \item
      Poiščite čim večjo verigo, čim večjo antiverigo in kakšno podmnožico, ki ni niti veriga niti antiveriga.
    \item
      Ali obstajajo minimalni, maksimalni, prvi ali zadnji elementi? Poiščite jih. Kako se odgovor spremeni, če ureditvi odvzamemo $1$ ali dodamo $0$?
  \end{enumerate}
\end{vaja}

\begin{vaja}
  Množico vseh nepraznih zaprtih podintervalov intervala $[0, 1]$, torej $\set[1]{[a, b]}{0 \leq a \leq b \leq 1}$, delno uredimo z relacijo vsebovanosti $\subseteq$. Ali obstajajo minimalni, maksimalni, prvi ali zadnji elementi? Poiščite jih. Kaj pa, če vzamemo množico vseh (tudi praznih) zaprtih podintervalov?
\end{vaja}

\begin{vaja}
  Na množici kompleksnih števil definiramo relacijo
  \begin{equation*}
    z \preccurlyeq w \iff \text{$z$ in $w$ ležita na istem poltraku iz izhodišča in $|z| \leq |w|$}.
  \end{equation*}
  Dokažite, da je $\preccurlyeq$ relacija delne urejenosti. Ali obstajajo minimalni, maksimalni, prvi ali zadnji elementi? Poiščite jih.
\end{vaja}

\begin{vaja}
  Poiščite delno urejeno množico, ki ima natanko en minimalni element, vendar nima prvega elementa.
\end{vaja}

\begin{vaja}
  Za vsako od naslednjih množic $S$ ugotovite, ali je navzgor in navzdol omejena, ali ima največji in najmanjši element ter supremum in infimum.
  \begin{enumerate}
    \item
      Odprti interval $S = (a,b)$ kot podmnožica $(\RR, \leq)$.
    \item
      Zaprti interval $S = [a,b]$ kot podmnožica $(\RR, \leq)$.
    \item
      Družina $S = \set[1]{\set{n}}{n \in \NN}$ kot podmnožica $(\pow(\NN), \subseteq)$.
    \item
      $S = \set{A \subseteq \NN}{0 \in A}$ kot podmnožica $(\pow(\NN), \subseteq)$.
  \end{enumerate}
\end{vaja}

\begin{vaja}
  Poiščite podmnožico $S \subseteq \pow(\NN)$, ki nima najmanjšega in nima največjega elementa glede na relacijo $\subseteq$. Ali taka podmnožica $S$ ima infimum in supremum?
\end{vaja}

\begin{vaja}
  Na $\RR^2$ definiramo:
  \[(x,y) R (z,w) \iff y \leq w \text{ in } x - y \leq z - w.\]
  \begin{enumerate}
    \item
      Pokažite, da je $R$ delna urejenost.
    \item
      Ali je $R$ linearna urejenost?
    \item
      Poiščite kakšno neskončno množico $A \subseteq \RR^2$, ki bo imela prvi element glede na $R$.
    \item
      Za poljubna elementa $(x,y), (z,w) \in \RR^2$ poiščite supremum $(x,y) \vee (z,w)$ in infimum $(x,y) \wedge (z,w)$ glede na relacijo $R$. Sklepajte, da je $(\RR^2, R)$ mreža.
    \item
      Relacijo $R$ zožimo na $[0, \infty) \times \RR$. Ali je $[0, \infty) \times \RR$ z zožitvijo relacije $R$ mreža? Določite minimalne, maksimalne, prve in zadnje elemente.
  \end{enumerate}
\end{vaja}

\begin{vaja}
  Za spodnje preslikave ugotovite, ali so monotone (številske množice so urejene z $\leq$, potenčne množice pa z $\subseteq$). Pri tem je $A$ množica, $S \subseteq A$ pa izbrana podmnožica.
  \begin{enumerate}
    \item
      $\NN \to \NN$, $n \mapsto n^2$
    \item
      $\ZZ \to \ZZ$, $n \mapsto n^2$
    \item
      $\NN \setminus \set{0} \to \NN \setminus \set{0}$, $n \mapsto \text{število deliteljev $n$ v $\NN \setminus \set{0}$}$
    \item
      $\RR \to \pow(\RR)$, $x \mapsto \set{x}$
    \item
      $\RR \to \pow(\RR)$, $x \mapsto (-\infty, x)$
    \item
      $\pow(A) \to \pow(A)$, $X \mapsto X^C$
    \item
      $\pow(A) \to \pow(A)$, $X \mapsto X \cup S$
    \item
      $\pow(A) \to \pow(A)$, $X \mapsto X \cap S$
    \item
      $\pow(A) \to \pow(A)$, $X \mapsto X \setminus S$
    \item
      $\pow(A) \to \pow(A)$, $X \mapsto S \setminus X$
    \item
      $\pow(A) \to \pow\big(\pow(A)\big)$, $X \mapsto \pow(X)$
  \end{enumerate}
\end{vaja}

\begin{vaja}
  Potenčno množico $\pow(A)$ množice $A$ uredimo z relacijo $\subseteq$. Vzemimo preslikavo $f\colon A \to A$ ter podmnožici $S, T \subseteq A$. Katere od spodnjih preslikav $\pow(A) \to \pow(A)$ so monotone ne glede na izbiro $A$, $f$, $S$ in $T$?
  \begin{enumerate}
    \item $X \mapsto f_*\big(X \setminus f^*(S \setminus T)\big)$
    \item $X \mapsto f^*\big(X \setminus f_*(S \setminus T)\big)$
    \item $X \mapsto f_*\big(S \setminus f_*(X \setminus T)\big)$
    \item $X \mapsto f^*\big(S \setminus f^*(X \setminus T)\big)$
    \item $X \mapsto f^*\big(S \setminus f^*(T \setminus X)\big)$
    \item $X \mapsto f^*\big(S \setminus f_*(T \setminus X)\big)$
  \end{enumerate}
\end{vaja}

\begin{vaja}
  Utemeljite sledeče trditve.
  \begin{enumerate}
    \item
      Kompozitum dveh monotonih preslikav je monotona preslikava.
    \item
      Kompozitum dveh nemonotonih preslikav ni nujno nemonotona preslikava.
    \item
      Ali je kompozitum monotone in nemonotone preslikave (v takem ali drugačnem vrstnem redu) nujno (ne)monotona preslikava?
  \end{enumerate}
\end{vaja}

\begin{vaja}
  Za delno urejeni množici $(P, \leq_P)$ in $(Q \leq_Q)$ lahko na kartezičnem produktu $P \times Q$ definiramo delno urejenost $\leq$ s predpisom
  \[(p_1, q_1) \leq (p_2, q_2) \iff p_1 \leq_P p_2 \land q_1 \leq_Q q_2.\]
  \begin{enumerate}
    \item
      Preverite, da je $\leq$ res delna urejenost.
    \item
      Če sta $\leq_P$ in $\leq_Q$ linearni urejenosti, ali je to tudi $\leq$?
    \item
      Kako lahko na kartezičnem produktu dveh linearno urejenih množic definiramo linearno urejenost?
  \end{enumerate}
\end{vaja}

\begin{vaja}
  Naj bo $(P, \leq)$ delna ureditev in $A$ poljubna množica. Na eksponentni množici $P^A$ definiramo ureditev
  %
  \begin{equation*}
    f \preceq g \iff \all{x \in A}{f(x) \leq g(x)}.
  \end{equation*}
  %
  Denimo, da je $\leq$ linearna ureditev. V kakšnem primeru je tudi $\preceq$ linearna ureditev? (Obravnavajte primere, ko imata $P$ in $A$ nič, en in več kot en element.)  
\end{vaja}

\begin{vaja}
  Naj bosta $(P, \leq_P)$ in $(Q, \leq_Q)$ delni ureditvi ter $f\colon P \to P$ monotona preslikava.
  \begin{enumerate}
    \item
      Dokažite, da $f$ slika omejene množice v omejene množice.
    \item
      Denimo, da ima množica $S \subseteq P$ supremum v $P$ in množica $f_*(S)$ supremum v $Q$. Dokažite, da velja $\bigvee_{x \in S} f(x) \leq f(\bigvee_{x \in S} x)$. Ali velja tudi enakost, tj.~ali monotone preslikave v splošnem ohranjajo supremume?
  \end{enumerate}
\end{vaja}

\begin{vaja}
  Označimo z $\leq$ običajno urejenost na $\RR^2$ (tj.~po komponentah) in z $\leq_{\:lex}$ leksikografsko urejenost. Naj bo dan $(u,v) \in \RR^2$ in naj bo $(x,y)$ tak element, da velja $(0,0) \leq (nx,ny) \leq (u,v)$ za vsak $n \in \NN$. Pokažite, da je $(x,y) = (0,0)$. Ali to velja tudi v primeru, ko $\leq$ zamenjamo z $\leq_{\:lex}$? Če velja, dokažite, sicer pa poiščite protiprimer.
\end{vaja}

\begin{vaja}
  Naj bo $(A_n, \leq_n)_{n \in \NN}$ družina delno urejenih množic. Za $\displaystyle{f, g \in \prod_{n \in \NN} A_n}$ definiramo relacijo $\leq$:
  \[f \leq g \iff (f = g) \lor \big(f \neq g \land f(x_0) \leq g(x_0)\big),\]
  pri čemer je z $x_0$ označen $\min\set{x \in \NN}{f(x) \neq g(x)}$. Pokažite:
  \begin{enumerate}
    \item
      $\leq$ je delna urejenost,
    \item
      če so $\leq_n$ linearne urejenosti, je tudi $\leq$ linearna urejenost.
  \end{enumerate}
\end{vaja}

\begin{vaja}
  Naj bo $(A, \preceq)$ šibka ureditev. Relacijo $\sim$ na $A$ definiramo takole:
  \[x \sim y \iff x \preceq y \land y \preceq x.\]
  \begin{enumerate}
    \item
      Pokažite, da je $\sim$ ekvivalenčna relacija.
    \item
      Na množici ekvivalenčnih razredov $A/_\sim$ definiramo relacijo $\leq$ takole:
      \[[x] \leq [y] \iff x \preceq y.\]
      Pokažite, da je $\leq$ dobro definirana in je delna urejenost.
  \end{enumerate}
\end{vaja}

\begin{vaja}
  Naj bo $f\colon A \to B$ preslikava in $\leq$ relacija na $B$. Na $A$ definiramo relacijo $\preceq$:
  \[x \preceq y \iff f(x) \leq f(y).\]
  Dokažite sledeče.
  \begin{enumerate}
    \item
      Če je $\leq$ šibka urejenost na $B$, je $\preceq$ šibka urejenost na $A$.
    \item
      Če je $\leq$ delna urejenost na $B$, ni nujno, da je $\preceq$ delna urejenost na $A$.
    \item
      Poiščite potreben in zadosten pogoj na preslikavo $f$, da iz delne urejenosti $\leq$ sledi delna urejenost $\preceq$.
    \item
      Pokažite: za vsako šibko ureditev $(A, \preceq)$ obstaja delna ureditev $(B, \leq)$ in preslikava $f\colon A \to B$, da je $\preceq$ porojena iz $\leq$ na zgoraj podani način.
  \end{enumerate}
\end{vaja}

\begin{vaja}
  Naj bo dana družina funkcij $(f_i\colon A \to \RR)_{i \in I}$. Na $A$ definiramo relacijo $R$ s predpisom
  \[x R y \iff \all{i \in I}{f_i(x) \leq f_i(y)}.\]
  Ali je $R$ nujno delna urejenost? Kaj je potreben in zadosten pogoj za to, da bo $R$ antisimetrična?
\end{vaja}

\end{document}