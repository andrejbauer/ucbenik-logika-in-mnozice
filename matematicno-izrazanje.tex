\chapter{Matematični jezik}
\label{cha:matematicni-jezik}

Tako kot vsaka stroka ima tudi matematika svoj jezik, ki obsega matematične simbole in
izraze ter svojevrsten način izražanja. Matematiki stremimo k popolni natančnosti in
nedvoumnosti matematične misli. To je seveda le ideal, ki se mu bolj ali manj približamo,
dejanska matematična besedila pa pišemo ljudje za ljudi, zato ni nič nenavadnega, da so
prežeta s tradicijo in nepisanimi družbenimi dogovori, ki matematiko oddaljijo od
formalnega ideala, a jo tudi naredijo ljudem razumljivo in zanimivo.

Pred študentom matematike je torej težka naloga, saj se mora hkrati z novo matematiko
učiti še novega jezika, ki je zapisan z nenavadnimi simboli. V pomoč se zato najprej
posvetimo samo matematičnemu izražanju.

\section{O matematičnem izražanju}
\label{sec:o-matematicnem-izrazanju}

% * Matematično besedilo stremi k _idealu_, to je popolno natančnosti in nedvoumnosti, z
%   vsemi podrobnostmi popolnoma razloženimi in preverjenimi. Ta ideal v manjši ali večji
%   meri dosežemo v praksi.
% 
% * Idealno matematično besedilo je _formalno_ in ga v praksi dosežemo le s pomočjo
%   računalnikov.
% 
% * Tradicija, navade in nepisani družbeni dogovori so pristoni v matematiki, tako kot
%   povsod drugje v človeški družbi. Tega se moramo zavedati in upoštevati, da so nekatere
%   tradicije preživete, zaradi drugih še vedno uporabljamo neprimerno simbole itd. Kljub
%   temu, da je matematični jezik zelo poenoten, je še vedno raznolik in se lahko menja celo
%   znotraj ene fakultete (seveda se trudimo, da bi bilo tega vse manj). Ta raznolikost
%   odraža raznolikost pogledov na to kaj matematika je in kako deluje, zato jo jemljimo kot
%   _bogastvo_ in ne samo nadelžno nedoslednost.
% 
% * Ker je namen tega učbenika postaviti dobre osnove matematičnega izražanja, bomo bolj
%   natančni kot večina matematikov v praksi. Začetnik namreč potrebuje tako natančnost,
%   kasneje, ko razume stvari bolje, pa lahko ubere bližnjice, ki jih bolj izkušeni kolegi
%   uporabljajo, ne da bi to sploh opazili. Sproti bomo opozarjali na take bližnjice in kje
%   matematična praksa zaostaja v doslednosti in pravilnosti, h kateri bomo strmeli mi.
% 
% * Vedno mora biti jasno, ali govorimo o _matematični vsebini_ (na primer, kaj je delna
%   urejenost) ali o _družbenem dogovoru_ (na primer, da za delno urejenost običajno
%   uporabljamo simbole, kot so ≤, ≼, ⊑ itd). V ta namen bi verjetno lahko uporabili
%   vizualne pripomočke.

% * Vrste besedila, opozorimo, da so namenjene različnim bralcev in napisani za različne
%   namene, pod različnimi pogoji:
%    * članek
%    * učbenik
%    * zapiski v zvezku
%    * predavanje
%    * blogi in ostale spletne strani

% * Sestavni deli besedila, brez podrobnih razlag, morda primeri, tu samo opozorimo na
%   raznovrstnost konceptov.
%    * spremno besedilo
%    * konstrukcije
%    * računi
%    * izjave (sinonimi, kako jih številčimo)
%    * dokazi (kako so označeni)
%    * definicije
%    * zgledi
%    * naloge in rešitve (namigi)
%    * formule in izrazi (kako jih številčimo in kako se nanje sklicujemo)
%    * citati in reference

% PRIMERI DRUŽBENIH DOGOVOROV

% $\vec{a}$ uporabljamo za oznako vektorja

% Pri algebri v prvem letniku na FMF je običaj, da se linearno preslikavo označi z
% $\mathcal{A}$, matriko, ki tej linearni preslikavi pripada, pa z $A$.




\section{Matematični zapis}
\label{sec:matematicni-zapis}

\subsection{Pisave in črke}
\label{sec:pisave-in-crke}

% * Pisave, uporaba črk iz različnih abeced

Matematična abeceda vsebuje precej več simbolov, kot zgolj običajne črke in števke. Nekatere že poznamo, na primer~$=$, $<$, $+$, $\emptyset$, $\cup$, $\cap$, $\int$, $\infty$ in tako naprej, precej jih še bomo spoznali. Poleg tega matematiki uporabljamo različne pisave, kot je prikazano v tabeli~\ref{tabela:oblike-crk}. V tabli in zvezku sicer težko ločimo med pokončno in ležečo pisavo, ali med kaligrafsko in rokopisno, v tiskanem besedilu pa se vedno držimo nekaterih pravil. Tako posamezne črke $a$, $b$, $c$, \dots, $x$, $y$, $z$ pišemo v ležeči pisavi, imena elementarnih funkcij pa pokončno: $\sin$, $\cos$, $\log$, \ldots Šumnikov običajno ne uporabljamo.

\begin{table}[ht]
\centering
\begin{tabular}{c|c}
\textbf{Pisava} & \textbf{Črke} \\
\hline
pokončna & $\mathrm{ABCDEFGHIJKLMNOPQRSTUVWXYZ}$ \\
matematična ležeča & $ABCDEFGHIJKLMNOPQRSTUVWXYZ$ \\
kaligrafska & $\mathcal{ABCDEFGHIJKLMNOPQRSTUVWXYZ}$ \\
rokopisna & $\mathscr{ABCDEFGHIJKLMNOPQRSTUVWXYZ}$ \\
frakturna & $\mathfrak{ABCDEFGHIJKLMNOPQRSTUVWXYZ}$ \\
dvopoudarjena & $\mathbb{ABCDEFGHIJKLMNOPQRSTUVWXYZ}$
\end{tabular}
\caption{Pisave}\label{tabela:oblike-crk}
\end{table}

Simbole lahko dodatno opremimo s črticami, vijugami, vektorskimi znaki, strešicami in podobno:
%
\begin{equation*}
 a \quad
 a' \quad
 \dot{a} \quad
 \bar{a} \quad
 \vec{a} \quad
 \tilde{a} \quad
 \hat{a} \quad
 \check{a}.
\end{equation*}
%
\andrej{Kako prevedemo subscript in supscript?} Znak lahko dodatno opremimo s
\emph{podpisom} ali \emph{nadpisom}, ki je lahko črka, številka, ali kak drug simbol, na
primer
%
\begin{equation*}
  a_i \quad
  a^i \quad
  a_1 \quad
  a_{\star} \quad
  a^{\dagger}.
\end{equation*}
%
Kljub temu obilju črk in oznak posežemo še po drugih abecedah, še posebej grški, zato
se jo čimprej naučite! Grške črke skupaj z njihovo izgovoravjo najdete v
tabeli~\ref{tabela:grska-abeceda}. Prostoročni zapis grških črk se boste naučili v
razredu.

\begin{table}[ht]
\begin{center}
\begin{tabular}{cc|cc}
\multicolumn{2}{c|}{\textbf{Grška črka}} & \multicolumn{2}{c}{\textbf{Izgovorjava}} \\
\textit{velika} & \textit{mala} & \textit{v matematiki} & \textit{v grščini} \\
\hline
A & $\alpha$ & alfa & alfa \\
B & $\beta$ & beta & vita \\
$\Gamma$ & $\gamma$ & gama & {\textgamma}ama \\
$\Delta$ & $\delta$ & delta & delta \\
E & $\epsilon$, $\varepsilon$ & epsilon & epsilon \\
Z & $\zeta$ & zeta & zita \\
H & $\eta$ & eta & ita \\
$\Theta$ & $\theta$, $\vartheta$ & {\scriptsize\textTheta}eta & {\scriptsize\textTheta}ita \\
I & $\iota$ & jota & jota \\
K & $\kappa$ & kapa & kapa \\
$\Lambda$ & $\lambda$ & lambda & lamda \\
M & $\mu$ & mi & mi \\
N & $\nu$ & ni & ni \\
$\Xi$ & $\xi$ & ksi & ksi \\
O & $\omicron$ & omikron & omikron \\
$\Pi$ & $\pi$, $\varpi$ & pi & pi \\
P & $\rho$, $\varrho$ & ro & ro \\
$\Sigma$ & $\sigma$, $\varsigma$ & sigma & si{\textgamma}ma \\
T & $\tau$ & ta\hill{u} & taf \\
$\Upsilon$ & $\upsilon$ & upsilon & ipsilon \\
$\Phi$ & $\phi$, $\varphi$ & fi & fi \\
X & $\chi$ & hi & {\textchi}i \\
$\Psi$ & $\psi$ & psi & psi \\
$\Omega$ & $\omega$ & omega & ome{\textgamma}a \\
\end{tabular}
\end{center}
\par\medskip
\footnotesize{
Izgovorajava: \hill{u} je ustnični u (kot v besedi `pav');
{\textgamma} je cerkljanski `g' (nekaj med `g' in `h' --- vprašajte sošolce s tega območja);
{\scriptsize\textTheta} je angleški nezveneči `th' (kot v besedi `thing');
{\textchi} je nemški `ch' (kot v besedi `ich').}
\caption{Grška abeceda.}
\label{tabela:grska-abeceda}
\end{table}

Pa tudi to matematikom še ni dovolj. Kot bomo videli kasneje, se za neskončne vrednosti uporablja hebrejska črka $\aleph$ (alef).


\subsection{Simboli in izrazi}
\label{sec:simboli-in-irazi}

% * Apostrofi, nadčrtaji, subskripti, supskripti
% * Izrazi, oklepaji
% * Izrazi kot sintaktična drevesa
% * Razni simboli, kateri se uporabljajo za kaj (omenimo LaTeX)
% * Infiksni operatorji: oklepaji, precedenca, asociativnost
% * Ostali zapisi (ulomki, množice)

\subsection{Slike in diagrami}
\label{sec:slike-in-diagrami}

% * Diagrami in slike
% * Ali se lahko zanašamo na slike?

% * Razlika med natančnim in nenatančnim zapisom
%   * ne moremo se zanašati na `...`
%   * napačna uporaba vezanih spremenljivk, recimo `f(x) ≡ 0` namesto `f ≡ 0` ali `∀ x . f(x) ≡ 0`


\davorin{Za začetek bom vnašal oporne točke besedila. Slog bo treba še popraviti in besedilo dopolniti.}

Za matematično delo je bistveno, da se lahko zanašamo na pravilnost naših trditev. To pomeni:
\begin{itemize}
\item
matematične izjave morajo imeti \emph{nedvoumen pomen},
\item
matematične izjave lahko \emph{dokažemo}.
\end{itemize}

Stavki v običajnih jezikih nimajo nedvoumnega pomena, zato matematične izjave raje podamo v \emph{matematičnem jeziku}. Za to potrebujemo \qt{matematično abecedo}, tj.~simbolni zapis, v katerem podamo izjave. V tem poglavju si bomo izoblikovali intuitivno predstavo temeljnih matematičnih pojmov ter spoznali simbolni zapis za delo z njimi. Kasneje bomo tem pojmom in simbolom dali natančnejši, formalni pomen. Kar se pisanja dokazov tiče, bomo pa mu namenili lastno poglavje (poglavje~\ref{poglavje:dokazovanje}).


%%% Local Variables:
%%% mode: latex
%%% TeX-master: "ucbenik-lmn"
%%% End:
